\documentstyle{amsppt}
\input epsf
\magnification=1200
\pagewidth{6.5truein} \pageheight{8.9truein}

\redefine\le{\leqslant} \redefine\ge{\geqslant}

\def\yawn #1{\nobreak\par\nobreak
\vskip #1 plus .1 in \penalty 200}
%%% This puts at least #1 units of space in your output.  #1 should be in
%%% braces {}.  Follow it with \noindent in your input file.

\define\R{\Bbb R} \define\Z{\Bbb Z} \define\Q{\Bbb Q} \redefine\P{\Bbb P}
\define\cal{\Cal} \define\F{{\Bbb F}} \define\C{\Bbb C} \define\N{\Bbb N}
\define\SL{\operatorname{SL}} \define\SP{\operatorname{Sp}}
\define\GL{\operatorname{GL}}
\define\op{\operatorname}
\define\bs{\backslash}
\define\im{\operatorname{im}}
\define\coker{\operatorname{coker}}

\define\shh{{\smc Sheafhom}}

\NoBlackBoxes

%%%%%%%%%%%%%%%%%%%%%%%%%%%%%%%%%%%%%%%%

\document

\heading Description of Sheafhom---Mark McConnell \endheading

\subheading{(1). Description}  \shh\ is a package of programs whose basic elements include:
\roster
\item"$\diamond$" finite-dimensional vector spaces over~$\Q$.
\item"$\diamond$" morphisms of finite-dimensional vector spaces
over~$\Q$ (i.e.~linear maps).
\item"$\diamond$" chain complexes $A_0\overset d_0 \to\leftarrow
A_1\overset d_1 \to\leftarrow\cdots\leftarrow A_n$, where the $A_i$
and $d_i$ are in the category of $\Q$-vector spaces---or, recursively,
in any category~$\cal A$ defined by the rules on this list.
\item"$\diamond$" graded objects, which are vectors $(A_0, \dots,
A_n)$ of $\cal A$-objects.
\item"$\diamond$" morphisms of chain complexes (i.e.~chain maps),
and of graded objects.
\item"$\diamond$" partially ordered sets, which we use to implement
finite regular cell complexes (including simplicial complexes).
\item"$\diamond$" sheaves of $\cal A$-objects on regular cell
complexes.
\item"$\diamond$" ``infinite-dimensional chain complexes'' $(C_*, \partial_*)$,
modeling the singular chain complex $C_*(X;\Q)$ on a space~$X$.
These contain chains which represent sections of a sheaf.  A complex
$C_*$ will grow as the program runs.
\item"$\diamond$" spectral sequence pages.  If a page $E^1_{pq}$ is
given, the system computes $E^2_{pq}$, $E^3_{pq}$, \dots $E^\infty_{pq}$,
constructing the differentials $d_{pq}^2$, $d_{pq}^3$, \dots as it
goes along.
\item"$\diamond$" operations like kernel, image, direct sum, tensor
product, and exterior power, defined (where appropriate) on  the above
categories, and behaving functorially on their morphisms.
\endroster
All these items are implemented and working, except for some
chain complex functions.

As an example, take objects $A$, $B$ and $C$, and morphisms $f$ and
$g$, in some category $\cal A$, with $gf=0$.

\smallskip
\epsfxsize= 1.5in
\qquad\qquad\qquad\qquad\qquad\qquad\quad\epsfbox[0 0 153
90]{shh11296.eps}
\smallskip

\noindent The homology object $H = (\ker g)/(\im f)$ at $B$ can be
computed as follows:

{\tt k := ker(g):}

{\tt j := pullback-to-ker(f,g):}

{\tt H := target(coker(j)):}

\noindent The first line defines an injective morphism $k$ from $K$, a
kernel object, to $B$.  The $K$ is constructed automatically and
stored with~$k$, though it is usually hidden from the user.  The
second line constructs the unique pullback morphism $j : A \to K$ such
that $kj=f$.  Then $\coker j$ is a surjection $K \twoheadrightarrow H$
whose target space~$H$ is, by definition, the homology object we want.
This is of course only an illustration. \shh's operation {\tt
homology(...)} does this process directly.  One point of the programs
is to free the user from low-level details.

The same three lines of code give different results according to the
category one is in.  If $A$, $B$ and $C$ are vector spaces, then $H$
is ordinary homology.  If they are graded objects, $H$ might be one row of
the $E^1$ term of a spectral sequence.

Everything in the programs works functorially.  For example, if {\tt
homology(...)} is given a morphism $f$ of chain complexes, it returns
$f_*$, the induced morphism on the homology graded objects.

Many identities can be used.  For instance, if $A$ and $B$ are vector
spaces, and\footnote{$\bigwedge\nolimits_i A$ denotes $\bigwedge\nolimits^i A$.  We
lower the index because \shh\ uses chain complexes, not cochain
complexes.} $C = \bigwedge\nolimits_* A \otimes \bigwedge\nolimits_* B$ and $D =
\bigwedge\nolimits_*(A\oplus B)$, then

\smallskip\noindent
(1.1) {\tt
$\qquad\qquad$ f := distribute-tensor-over-exterior-algebra(C,D):}

\smallskip\noindent
sets $f$ equal to the canonical isomorphism $C \overset
\sim\to\longrightarrow D$.  To ``use'' the identity, compose $f$ or
$f^{-1}$ with other maps.

Every morphism of vector spaces carries within it a matrix
representing the map with respect to standard coordinates.  The
matrices are created automatically and are normally hidden from the
user.  A major goal of the programs is to free users from matrix
manipulation and to allow them to think categorically.

\subheading{(2). Deligne's construction of intersection homology for
toric varieties} Let $X$ be a toric variety (over~$\C$, complete,
normal, not necessarily smooth or projective).  Since the local normal
structure of~$X$ is relatively simple, \shh\ can compute the
intersection homology of~$X$ directly from first principles, i.e.~from
Deligne's construction.

You can only say what your language allows you to say.  \shh\ provides
part of the standard mathematical vocabulary, and allows one to
express deep ideas in something like standard mathematical language.

A toric variety $X$ of dimension~$n$ is characterized by having a
$(\C^*)^n$ action with a single dense orbit~$S^{n1}$.  The orbits
under this action partition $X$ into strata $S^{ij} \cong (\C^*)^i$
for $i=0,\dots,n$.  (The~$j$ is a dummy variable indexing the strata
of dimension~$i$.)  The open set $U^{ij} = \bigcup_{\overline{S^{i_1 j_1}}
\supseteq S^{ij}} S^{i_1 j_1}$ is a neighborhood of $S^{ij}$ in $X$,
and is homeomorphic to a product $S^{ij} \times c^{ij}$, where
$c^{ij}$ is the {\sl normal cone\/} for~$S^{ij}$.  (The topology of
toric varieties is special in two ways.  (a)~For most spaces, only a
small subset of $U^{ij}$ near $S^{ij}$ would have a local product
structure.  (b)~Usually there is monodromy, meaning $U^{ij}$ would
only be a bundle with fiber $c^{ij}$.)

Let $IH_k(\dots)$ denote the intersection homology $IH^{\bar
p}_k(\dots; \Q)$ for arbitrary perversity~$\bar p$.  This includes
ordinary cohomology and homology as special cases.

Fix $V = \Q^n$.  To each $S^{ij}$ is associated a rational subspace
$T^{ij} \subseteq V$ of dimension~$i$, representing the tangential
directions along $S^{ij}$ (the orthocomplements of the corresponding
cones in the fan for~$X$).  If $S^{ij} \subseteq \overline{S^{i_1
j_1}}$, then $T^{ij} \subseteq T^{i_1 j_1}$.  A basic fact is that
$IH_k(S^{ij}) \cong \bigwedge\nolimits_k T^{ij}$.  

\def\oc{\overset\circ\to c}  \def\oU{\overset\circ\to U}

Let $\oc^{ij}$ be $c^{ij}$ with its vertex deleted.  Let $\oU^{ij} =
S^{ij} \times \oc^{ij}$, a deleted neighborhood of $S^{ij}$.  
The main idea of Deligne's construction is to compute
$IH_*(\oU^{ij})$, $IH_*(U^{ij})$, and $IH_*(U^{ij}, \oU^{ij})$
simultaneously by descending induction on~$i$.  By a K\"unneth
formula,
$$
\align
IH_k(\oU^{ij})
  &= \bigoplus_{k_1 + k_2 = k} IH_{k_1}(S^{ij}) \otimes
                                 IH_{k_2}(\oc^{ij}) \\
  &\cong \bigoplus_{k_1 + k_2 = k} \left( \bigwedge\nolimits_{k_1} T^{ij} \right)
                                      \otimes  IH_{k_2}(\oc^{ij}),
\endalign
$$
There are similar product formulas for $IH_*(U^{ij})$ (replace
$\oc^{ij}$ with $c^{ij}$), and for $IH_*(U^{ij},
\oU^{ij})$ (replace $\oc^{ij}$ with $(c^{ij}, \oc^{ij})$).  The
central task is to compute $IH_*$ of $\oc^{ij}$, of $c^{ij}$, and of
$c^{ij}$ relative to $\oc^{ij}$.

We first set up the initial step ($i=n$) of the induction.  Since the top
stratum $S^{n1}$ is smooth, we have $T^{n1}=V$, $c^{n1} = \{\text{point}\}$,
and $\oc^{n1} = \varnothing$.  We set up graded objects $N_*^{n1}$
and $N_{*,\text{rel}}^{n1}$, both having $\Q$ in degree~0 and 0 in degrees~1
through~$n$; these represent $IH_*(c^{n1})$ and $IH_*(c^{n1},
\oc^{n1})$ respectively.  We create an ``infinite-dimensional chain
complex'' $(C_*, \partial_*)$, initially with $C_k =
\bigwedge\nolimits_k V$ for $0\le k \le 
n$, $C_k = 0$ for $n < k \le 2n$, and all differentials $\partial_k =
0$; this stands for a part of the intersection
chain complex $IC_*^{\bar p}(X)$ which represents the classes of
$IH_*((\C^*)^n) \cong \bigwedge\nolimits_*V$.  (As the computation
progresses, $C_*$ will grow larger.)  To $S^{n1}$ we attach 
$$
\gathered
\text{the {\sl section }}
  \zeta_*^{n1} : IH_*(U^{n1}) \cong \left( \bigwedge\nolimits_* T^{n1} \right) \otimes N^{n1}_*
    \hookrightarrow C_* \\
\text{the {\sl local (relative) section }}
  \zeta_{*,\text{rel}}^{n1} : IH_*(U^{n1}, \oU^{n1}) \cong \left(
\bigwedge\nolimits_* T^{n1} \right) \otimes
                    N^{n1}_{*,\text{rel}}
    \hookrightarrow C_*
\endgathered
\tag 2.1
$$
The injections $\hookrightarrow$ at this initial stage are essentially just
isomorphisms $\bigwedge\nolimits_k V \otimes \Q \cong \bigwedge\nolimits_k V$.

At the $i$-th step, assume by induction that for each $i_1 j_1$ with
$i_1 > i$ we
have computed graded objects $N^{i_1 j_1}_{*}$ and $N^{i_1
j_1}_{*,\text{rel}}$ and injections to $C_*$ as follows:
$$
\gather
\text{{\sl sections }}
  \zeta_*^{i_1 j_1} : IH_*(U^{i_1 j_1}) \cong \left(
\bigwedge\nolimits_* T^{i_1 j_1} \right)
\otimes N^{i_1 j_1}_* 
    \hookrightarrow C_* \tag {2.2} \\
\text{{\sl local (relative) sections }}
  \zeta_{*,\text{rel}}^{i_1 j_1} : IH_*(U^{i_1 j_1}, \oU^{i_1 j_1})
\cong \left( \bigwedge\nolimits_* T^{i_1 j_1} \right) \otimes
                    N^{i_1 j_1}_{*,\text{rel}}
    \hookrightarrow C_* \tag{2.3}
\endgather
$$
Consider a given stratum $S^{ij}$.  The sets $U^{i_1 j_1}$ for which
$i_1 > i$ and $\overline{S^{i_1 j_1}} \supset S^{ij}$ form an open cover of $\oU^{ij}$.  We now
explain how \shh's spectral sequence code finds $IH_*(\oc^{ij})$ and
$IH_*(\oU^{ij})$.

In~(2.3), the left-hand tensor factors are all related to
$\bigwedge\nolimits_*(T^{ij})$, because $T^{ij} \subset T^{i_1 j_1}$.  We rewrite
the left side of~(2.3) in each case as
$$
\left( \bigwedge\nolimits_* T^{i_1 j_1} \right) \otimes N^{i_1 j_1}_{*,\text{rel}}
\cong
\left( \bigwedge\nolimits_* T^{ij} \right) \otimes
  \undersetbrace := N^{\prime i_1 j_1}_{*, \text{rel}}
      \to {[\left( \bigwedge\nolimits_* T^{\prime i_1 j_1} \right)
              \otimes N^{i_1 j_1}_{*, \text{rel}}]}
$$
where $T^{\prime i_1 j_1}$ is the orthocomplement of $T^{ij}$ in
$T^{i_1 j_1}$.  This uses~(1.1).  Keeping track of these tensors
and wedges is the most frustrating part of doing the computation by
hand, and it
accounts for much of the computer's run time.

We assemble the $N^{\prime i_1 j_1}_{*,\text{rel}}$ into the $E^1$
page of a spectral sequence, with each row containing the terms for a
given~$i_1$, and with a map $\zeta^{E^1} : E^1 \to C_*$ obtained
from the $\zeta^{ij}_{*, \text{rel}}$'s.  For each element $\alpha$ in
the $i_1$-th row of the
$E^1$ page, we compute $\zeta^{E^1}(\alpha)$, which is a chain in
$C_*$, and take the boundary $\partial \zeta^{E^1}(\alpha)$, which is
a cycle in $C_{*-1}$.  This cycle pulls back under $\zeta^{E^1}$ to a
(possibly zero) class $\beta$ in the $(i_1+1)$-st row of $E^1$.  The
map $\alpha \mapsto \beta$ is the $d^1$ differential in $E^1$, and
taking homology with respect to $d^1$ gives the $E^2$ page of the
spectral sequence.  We repeat this process inductively (using the same
code!) with $E^r$ replacing $E^1$ for $r=2, 3, \dots$; here $\beta$
will lie in the $(i_1+r)$-th row.  We construct the $d^2$ differential
and the page $E^3$, construct $d^3$ and $E^4$, etc.  Keeping track of
the correct row for~$\beta$ (the filtration level) happens
automatically, because of how $C_*$ and $\partial_*$ are constructed;
no programming effort is required.  The spectral sequence converges
after finitely many steps to $IH_*(\oc^{ij})$, and provides an
appropriate injection $IH_*(\oc^{ij}) \hookrightarrow C_*$.  We use
the tensor product to compute $IH_*(\oU^{ij}) \cong \left(
\bigwedge\nolimits_* T^{ij}\right) \otimes [IH_*(\oc^{ij})]
\hookrightarrow C_*$.

Once we have $IH_*(\oc^{ij})$, doing Deligne's construction is
straightforward.  The section $IH_*(U^{ij})$ is
$\left(\bigwedge\nolimits_* T^{ij} \right) \otimes [\op{tr}_{\le d}
IH_*(\oc^{ij})]$, where $\op{tr}_{\le d}$ is the truncation operator
which keeps only degrees $\le d$ (this $d$ depends on~$i$ and~$\bar
p$).  The map $\zeta_*^{ij} : IH(U^{ij}) \hookrightarrow C_*$ is
induced by the one for $IH_*(\oU^{ij})$.  The local relative section
$IH_*(U^{ij}, \oU^{ij})$ is $\left( \bigwedge\nolimits_* T^{ij}
\right) \otimes [\op{tr}_{> d} IH_*(\oc^{ij})[1]]$; we keep only
degrees~$>d$ in the second factor, and shift them one degree up (which
is what the~[1] means).  If $\alpha \in IH_k(\oc^{ij})$ is a class (for
$k>d$) represented by a geometric cycle~$z \in C_k$, we adjoin to
$C_{k+1}$ a new formal chain $\frak c \alpha$ representing a cone with
base~$z$ and vertex on $S^{ij}$, and we extend the definition of $
\partial_{k+1}$ by setting $
\partial(\frak c \alpha) = z$.  We define
$\zeta_{*,\text{rel}}^{ij} : IH_*(U^{ij}, \oU^{ij}) \hookrightarrow
C_*$ using $\zeta_{*,\text{rel}}^{ij}(\alpha[1]) = \frak c \alpha$ in
the right-hand tensor factor.

By this induction, we compute $IH_*^{\bar p}$ of the $U^{ij}$ and
$(U^{ij}, \oU^{ij})$ for all~$i$.  Finding the global $IH_*^{\bar
p}(X)$ calls for just one more spectral sequence involving all the
$IH_*^{\bar p}(U^{ij}, \oU^{ij})$.

The sections and local relative sections are stored in a data
structure called a {\tt sheaf}.  The {\tt sheaf} also includes the
perversity~$\bar p$ and a partially ordered set embodying the face
relations among the $S^{ij}$.

\subheading{(3).  ``Non-rational toric varieties''} If the method
in~(2) is carried out over~$\R$, the rationality of the $T^{ij}$ is
never used.  We can use any family of subspaces $T^{ij}\subseteq
\R^n$, rational or not, satisfying inclusions $T^{ij} \subseteq T^{i_1
j_1}$ coming from an appropriate partially ordered set.  In
particular, the method defines ``intersection homology Betti numbers''
(the $h$-vector) for the fictitious toric varieties coming from
non-rational convex polytopes (see~(4)).

\subheading{(4).  Intersection Products and Convex Polytopes} In
this section only, let $IH_*$ denote the middle-perversity intersection
homology $IH^{\bar m}_*$, and assume $H^*$, $H_*$ and $IH_*$ all have
coefficients in~$\C$.

Let $\Delta_0$ be a simplicial convex polytope of dimension~$n$---that
is, a convex polytope whose codimension-one faces are simplices.  Let
$f_i$ be the number of $i$-dimensional faces of $\Delta_0$, and call
$(f_0, \dots, f_{n-1}) \in \N^n$ the {\sl $f$-vector\/} of~$\Delta_0$.
Peter McMullen gave a characterization of the set of $f$-vectors of
simplicial polytopes, which was proved by Lou Billera and Carl Lee
(sufficiency of McMullen's conditions) \cite{Bi-L} and by Richard
Stanley (necessity) \cite{St1}.

Assume $\Delta_0$ is rational and has the origin in its interior.
Then $\Delta_0$ defines a fan~$\Sigma$, from which we get a toric
variety $X$.  This is a projective variety, so it has a Lefschetz
class $\omega \in H^2(X)$ given by a generic hyperplane section.
$H^k(X) = 0$ when $k$ is odd.

The {\sl $h$-vector\/} $(h_0, h_1, \dots)$ of $\Delta_0$ is defined by
$h_i = \dim H^{2i}(X)$.  It depends only on the combinatorial type
of $\Delta_0$.  Characterizing the set of $h$-vectors is equivalent to
characterizing the set of $f$-vectors.

Since $\Delta_0$ is simplicial, $X$ is smooth except for finite quotient
singularities.  The key step in Stanley's proof is to use the
following facts:
\roster
\item $H^*(X)$ is a graded ring under the usual intersection
product, where the $i$-th graded piece is $H^{2i}$;
\item $H^*(X)$ is generated as a
$\C$-algebra by $H^2$;
\item $H^*(X)/(\omega)$ is a graded ring, where $(\omega)$ is the
ideal generated by the Lefschetz class;
\item $H^*(X)/(\omega)$ is generated by the image mod~$(\omega)$
of~$H^2$.
\endroster

Now let $\Delta$ be any rational convex polytope, not necessarily
simplicial.  Currently there is no satisfactory generalization of the
previous story for rational $\Delta$.  Instead of the $f$-vector, one
considers the {\sl flag vector\/} $(f_{i_1 \dots i_k})$ of $\Delta$,
where $f_{i_1 \dots i_k}$ is the number of sets $\{\sigma_1, \dots,
\sigma_k\}$ of faces of $\Delta$ for which $\dim \sigma_{j} = i_j$ and
$\sigma_j \subset \overline{\sigma_{j+1}}$.  It is a difficult problem
to characterize the set of flag vectors.  (When $\Delta$ is
simplicial, the $f$-vector determines the flag vector, but not in
general.)

One can still associate a fan~$\Sigma$ and a toric variety $X$ to
$\Delta$, but $X$ will in general be singular (with more than just
finite quotient singularities).  $X$ is still projective, so there is
a Lefschetz class $\omega\in H^2(X)$.

The ordinary cohomology Betti numbers of $X$ are not combinatorial
invariants of $\Delta$, so the $h$-vector as given above is not well
defined~\cite{M0}.  However, the intersection homology Betti numbers
{\sl are\/} combinatorial invariants, and we redefine the $h$-vector
to be $h_i = IH_{2i}(X)$.  (The definitions agree when $\Delta$ is
simplicial.  For all~$\Delta$, $IH_k(X) = 0$ for odd~$k$.)  The
$h$-vector does not determine the flag vector.  However,
characterizing the set of $h$-vectors would give new and important
information about polytopes.

(As mentioned in~(3), section~(2) defines an $h$-vector even when
$\Delta$ is not rational.  It will have $h_i \ge 0$, but we do not
know whether it satisfies Poincar\'e duality (the generalized
Dehn-Somerville equations) or the Lefschetz inequalities $h_0 \le h_1
\le \dots \le h_{\lfloor n/2 \rfloor}$.  We do not even know if this
``$IH_k$'' is zero for odd~$k$.)

Properties~(1)--(4) have analogues~($1'$)--($4'$) in which $H^*$ is
replaced by $IH_*$.  A priori these analogues may be true or false.
One could characterize the set of $h$-vectors if one could
prove~($3'$)--($4'$).  

The candidate for the $i$-th graded piece in $IH_*$ would be
$IH_{2n-2i}$.  In general, there is a canonical intersection product
$$
IH_i(X) \times IH_j(X) \to H_{i+j-2n}(X). \tag 4.1
$$
There is also a canonical map $IH_k(X) \to H_k(X)$.  However, (4.1)
does not pull back to $IH_*(X)$, even for toric varieties in dimension
$n=3$.  This means $IH_*(X)$ is not a
ring: ($1'$)~is false, and ($2'$)~does not make sense.  (Frances
Kirwan has announced that $IH^{\bar m}_*$ of toric varieties {\sl
is\/} a ring, if one uses a different
product coming from the theory of symplectic quotients.)

To understand~($3'$), recall that there is an intersection product
$H^i(X) \times IH_j(X) \to IH_{j-i}(X)$.  Thus it makes sense to mod
$IH_*$ out by the submodule $(\omega) = \omega\cdot IH_*$.  The
quotient $IH_*(X)/(\omega)$ is trivially a ring for $n\le 3$, but the
cases $n\ge 4$ are open.  Even if ($3'$) and ($4'$) are false, it
would be important to study why they fail.  I propose to investigate
these questions---whether $IH_*/(\omega)$ is a ring, and, if so,
whether it is generated by its first graded piece.  I believe this is
a field where computing examples will pay a high mathematical
dividend.

One reason for computing $IH^{\bar p}_*(X)$ by explicit cycles, as
in~(2), is that it will be possible to work out the intersection
product~(4.1) explicitly.  In the sections over the smooth
part~(2.1), the product is just the wedge product in
$\bigwedge\nolimits_* V$, up to a duality.  Whenever a new $\frak c
\alpha$ is adjoined to $C_*$, we may assume $\frak c \alpha$ is in
general position with respect to pre-existing chains, and we can work
out its intersection with them.  Let $\mu : X \to \Delta^*$ be the
moment map, with $\Delta^*$ the dual polytope for $\Delta$.  For each
$\frak c \alpha$, we will have to choose a PL subset of $\Delta^*$
representing $\mu(\frak c \alpha)$.  Then $\frak c \alpha \cap \frak c
\beta$ will be zero if $\mu(\frak c \alpha)$ and $\mu(\frak c \beta)$
are disjoint; otherwise, we can compute $\frak c \alpha \cap \frak c
\beta$, which lies in $\mu^{-1}(\mu(\frak c \alpha) \cap \mu(\frak c
\beta))$, by induction.  The PL subsets of $\Delta^*$ can be expressed
in combinatorial terms with reference to a barycentric subdivision of
$\Delta^*$.  A major programming goal for \shh\ is to write this code.

\subheading{(5). Relation to other work} I have not heard of another
computer algebra system which deals with homological algebra and
sheaves the way \shh\ does.  Many systems deal with tensors and
differential forms, but in a fixed vector space.  One very interesting
system is EAT (Effective Algebraic Topology), written by a group led
by Francis Sergeraert---see {\tt
http://www-fourier.ujf-grenoble.fr/${\sim}$sergerar/} \quad This deals with
chain complexes and spectral sequences, and computes homotopy and
homology groups of finite simplicial complexes, loop spaces, and other
objects.

For the middle perversity~$\bar m$, the {\sl purity\/} condition
implies that our spectral sequences collapse at $E^2$ (up to some
indexing issues).  The middle perversity Betti numbers $\dim IH^{\bar
m}_k(X)$ are known via a beautiful recursive formula discovered
independently by Bernstein, Khovanskii and MacPherson (whose proof was
first published in~\cite{F}); this depends on the Decomposition
Theorem, which is another expression of purity.  Nevertheless, I have
not incorporated purity results into the \shh\ toric variety code.  It
is important for~(4) to represent $IH^{\bar p}_*(X)$ by explicit
cycles, so I have preferred a direct computation of it.  Of course,
purity considerations will be critical in interpreting our results,
and they could help speed up the code.

One can invert the induction in~(2), forgetting the normal cones and
using the $S^{ij}$ alone for ascending~$i$.  This gives a spectral
sequence for $H_*(X)$, first written down by Burt Totaro~\cite{T}.  A
form of purity implies this spectral sequence collapses at $E^2$.  A
version of \shh\ from May 1996 computes $H_*(X)$ using this algorithm.
It does not work with other perversities.

Jonathan Fine, encouraged by Gil Kalai, has recently created a
combinatorial scheme that takes~$\Delta$ and produces numbers $\tilde
h_i$.  These equal the $h_i$ in many cases, but not always; an example
of Marge Bayer in dimension five even shows some $\tilde h_i$ can be
negative.  Fine's scheme has elements which seem to stand for classes
in $\bigwedge\nolimits_k T^{ij}$, in $N^{ij}_{*,\text{rel}}$, and for
the $\frak c \alpha$, all relative to middle perversity.  I will run
my spectral sequence code on Bayer's example and others, analyzing
each page $E^r$ and comparing it to the elements of Fine's scheme.  I
suspect that Fine's $\tilde h_i$ are sums and differences of the
dimensions of the terms in the $E^1$ and $E^2$ pages, but that they
miss the $E^\infty$ pages.

It is important to understand the intersection product on the ordinary
cohomology $H^*(X)$, as well as on $IH_*(X)$, because of the
importance of intersection theory in algebraic geometry.  For example,
there are close relations between $H^*(X)$, the Chow group, and the
``operational'' Chow ring \cite{F-M-S-S} \cite{T}.

In this thesis, David Yavin defined a chain complex that computes
$IH^{\bar p}_*$ of toric varieties.  It used objects like the $T^{ij}$
and $c^{ij}$, with a mixture of finite- and infinite-dimensional
complexes.  My work is independent of Yavin's.

\subheading{(6). Other applications} The class of spaces which can
be handled by a ``direct Deligne's construction'' as in~(2) is
probably small.  (Perhaps one could handle Schubert varieties and
other spherical varieties this way, after careful study of the
geometry of the links.)  I now briefly mention other applications in
which $X$ is any finite regular cell complex (e.g.~simplicial
complex).

\comment
Sections~1 and~2 provide examples using cell complexes.  Gunnells and
I have used \shh, along with Gunnells' algorithm~(1.5), to try to
compute Hecke operators.  The ease with which \shh\ expresses
pullbacks and homology has been very helpful.
\endcomment

MacPherson has recently found an elementary interpretation of perverse
sheaves for regular cell complexes~$X$.  Over each cell in~$X$ one
puts a stalk, which is a finite-dimensional $\Q$-vector space.  There
is a perversity function $\delta : \N \to \Z$ satisfying certain
conditions.  If $\tau, \sigma$ are a $j$-cell and an $i$-cell with one
lying in the other's boundary, and if $\delta(j) + 1 = \delta(i)$,
then there is a map from the stalk on~$\sigma$ to the stalk on~$\tau$;
the aggregate of these maps must satisfy certain axioms.  If
$\delta(k) := -k$ this gives an ordinary sheaf, but for other $\delta$
it gives perverse sheaves.  A book presenting the theory has been
begun by Mark Goresky, MacPherson, Maxim Vybornov, and myself.  \shh\
could easily include these perverse sheaves.

\subheading{(7). Availability} A working version of \shh, written in
Common Lisp and its object-oriented extension CLOS, is available on
the Web at

\noindent {\tt
http://www.math.okstate.edu/${\sim}$mmcconn/shh.html}\newline There is
enough documentation so that people who don't know Lisp, but have
access to a Lisp compiler with CLOS, can get started using the
programs.

\subheading{(8). Computational Issues} I intend to make \shh\ widely
available to mathematicians.  Sadly, though, Common Lisp is not a
widely-used computer language today.  If I translate \shh\ into
another language, the language will have to support object-oriented
programming, since the current version is heavily object-oriented.  I
could translate \shh\ into C++, building on LiDIA, an efficient system
of integer- and matrix-crunching routines.  \shh\ would then become a
stand-alone system.  I could also incorporate it into Magma, or the
nascent Macaulay~II.  The choice will be made soon.  I would like to
write a Maple version, since that may have the widest audience, but
one wonders if it would run too slowly.

Currently \shh\ implements linear maps over~$\Q$ as matrices~$A$
over~$\Z$.  I could implement them as pairs $(A,d)$ representing
$\frac 1 d \cdot A$ with $d\ge 1$ and $A$ over~$\Z$, but so far there
has been no need; the~$d$ can usually be absorbed into the choice of
basis of $A$'s domain or target space.  The main linear algebra
routines are based on the integer~LLL algorithm.  Currently I use
arbitrary-precision integers; this is relatively slow, but robust and
elegant.  \shh\ could be adapted to include vector spaces over finite
fields.  One could add $\Z$-modules, though that would introduce the
complications of torsion.

It will be essential to add sparse matrices to \shh.  Since~1987 I
have written several sparse integer matrix routines for different
purposes; I will use an improved version of one of these.

Let $\Delta$ be the 4-dimensional hypercube with vertices $(\pm1,
\pm1, \pm1, \pm1)$, and let $X$ be the associated toric variety.  $X$
has 81 different strata.  The version of \shh\ using Totaro's
algorithm~(5) computes the ordinary homology Betti numbers (1, 0, 1,
0, 4, 11, 12, 0, 1) of $X$ in 6--7 seconds on a {\smc Sparc}station~1000.
Up-to-date timing data for the algorithm in~(2) is available on the
Web page mentioned in~(7).

%%%%%%%%%%%%%%%%%%%%%%%%%%%%%%%%%%%%%%%%

\Refs

\widestnumber\key{F-M-S-S}

\ref \key Bi-L
\by L. Billera, C. Lee
\paper A proof of the sufficiency of McMullen's conditions for
$f$-vectors of simplicial convex polytopes
\jour J. Combin. Thy. Ser. A
\vol 31 \yr 1981 \pages 237--255
\endref

\ref \key F
\by K.-H. Fieseler
\paper Rational intersection cohomology of projective toric varieties
\jour J.~Reine Angew. Math.
\vol 413 \yr 1991 \pages 88--98
\endref

\ref \key F-M-S-S
\by W. Fulton, R. MacPherson, F. Sottile, B. Sturmfels
\paper Intersection theory on spherical varieties
\jour J.~Algebraic Geom. \vol 4 \yr 1995 \pages 181-193
\endref

\ref \key M0
\by M. McConnell
\paper The rational homology of toric varieties is not a combinatorial
invariant
\jour Proc.~AMS \vol 105 \yr 1989 \pages 986--991
\endref

\ref \key St1
\by R. Stanley
\paper The number of faces of a simplicial convex polytope
\jour Adv. in Math. \vol 35 \yr 1980 \pages 236--238
\endref

\ref \key St2
\bysame
\paper Generalized $H$-vectors, intersection cohomology of toric
varieties, and related results
\inbook Commutative Algebra and Combinatorics (Kyoto 1985)
\bookinfo Adv. Stud. in Pure Math. {\bf 11}, North Holland,
Amsterdam, New York
\yr 1987 \pages 187--213
\endref

\ref \key T
\by B. Totaro
\paper Chow groups, Chow cohomology, and linear varieties
\paperinfo to appear in {\sl J. Algebraic Geom}
\endref

\endRefs

\enddocument
