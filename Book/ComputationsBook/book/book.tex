% This tex file has been produced automatically, do not edit.
% Title: Computations in algebraic geometry with Macaulay 2
\def\editors{Editors: D. Eisenbud, D. Grayson, M. Stillman, and B. Sturmfels}
\documentclass[runningheads]{lncse}
 \usepackage{makeidx,multicol}\makeindex
 \def\rhpage{\ifodd\value{page}\else\thispagestyle{empty}\null\vfill\eject\fi}
%%%%%%%%%%%%%%%%%%%%%%%%%%%%%%%%%%%%%%%%%%%%%%%%
%%%%%
%%%%% book-macros.tex
%%%%%
%%%%%%%%%%%%%%%%%%%%%%%%%%%%%%%%%%%%%%%%%%%%%%%%

% \usepackage{amsthm} % can't use this!
\usepackage{amsmath}
\usepackage{amscd}
\usepackage{amssymb,latexsym}
\usepackage{xispace}
% \usepackage[light]{draftcopy} \draftcopySetGrey{0.965}
\usepackage{verbatim}

\usepackage[PostScript=dvips]{diagrams}           % used only in the Eisenbud-Decker paper
\let\labelstyle\undefined       % xy redefines this, anyway
\let\objectstyle\undefined      % xy redefines this, anyway
\let\objectwidth\undefined      % xy redefines this, anyway
\let\objectheight\undefined     % xy redefines this, anyway
\let\stop\undefined             % xy redefines this, anyway
\let\diagram\undefined          % xy redefines this, anyway

\usepackage[matrix,arrow,curve]{xy}
% \usepackage{hyperref} \hypersetup{
%         bookmarks=true,
%         bookmarksnumbered=true,
%         colorlinks=true
%         }
\input xypic.tex
\input epsf.tex
\numberwithin{theorem}{section}
\newtheorem{Remark}[theorem]{Remark}{\bfseries}{\rm}
\newtheorem{Example}[theorem]{Example}{\bfseries}{\rm}
\newtheorem{code}[theorem]{Code}{\bfseries}{\rm}
\newcommand{\Mtwo}{{\textsl{Macaulay~2}}\xispace}
\newcommand{\Macaulay}{\textsl{Macaulay}\xispace}
\date{\ifcase\month\or January\or February\or March\or April\or May\or
  June\or July\or August\or September\or October\or November\or December\fi
  \space\number\day, \number\year}
\bibliographystyle{springer}
\newcommand{\ie}[1]{#1\index{#1}}
\newcommand{\indexcmd}[1]{\index{#1@{\tt #1}}}

\setcounter{tocdepth}{1}

% we need this definition here so it can be used in the index!
\def\Ext{\operatorname{Ext}}
\hyphenation{Mac-au-lay}

\input merge

%% this overrides some junk in lncse.cls:
\def\bbbr{{\mathbb R}}
\def\bbbm{{\mathbb M}}
\def\bbbn{{\mathbb N}}
\def\bbbf{{\mathbb F}}
\def\bbbh{{\mathbb H}}
\def\bbbk{{\mathbb K}}
\def\bbbp{{\mathbb P}}
\def\bbbc{{\mathbb C}}
\def\bbbq{{\mathbb Q}}
\def\bbbt{{\mathbb T}}
\def\bbbs{{\mathbb S}}
\def\bbbz{{\mathbb Z}}
 \begin{document}
  \frontmatter

    % \hypersetup{
    %     pdftitle=Computations in algebraic geometry with Macaulay 2
    %     pdfsubject=Macaulay 2,
    %     pdfkeywords=computations -- syzygies -- Macaulay 2,
    %     pdfauthor=\editors}

    \thispagestyle{empty}\null
	\vskip 2 in
	\centerline{\Large\textbf{Computations in algebraic geometry}}
	\bigskip
	\centerline{\Large\textbf{with \textsl{Macaulay 2}}}
	\bigskip
	\bigskip
	\bigskip
	\centerline{\editors}
    \vfill\eject
    \rhpage
  \pagenumbering{roman}
  \makeatletter \c@page=5 \makeatother

%%%%%%%%%%%%%%%%%%%%%%%%%%%%%%%%%%%%%%%%%%%%%%%%
%%%%%
%%%%% ../chapters/preface/chapter-m2.tex and ../chapters/preface/chapter-wrapper.bbl
%%%%%
%%%%%%%%%%%%%%%%%%%%%%%%%%%%%%%%%%%%%%%%%%%%%%%%

\bgroup
% $Source: /home/cvs/M2/Macaulay2/ComputationsBook/chapters/preface/chapter.tex,v $
% $Revision: 1.36.2.3 $
% $Date: 2001/05/22 14:47:48 $

\chapter*{Preface}
\markboth{Preface}{Preface}
\addcontentsline{toc}{section}{Preface}

\bgroup

Systems of polynomial equations arise throughout mathematics, science, and
engineering.
Algebraic geometry provides powerful theoretical techniques for
studying the qualitative and quantitative features of their solution sets.
Recently developed algorithms have made theoretical aspects of the subject
accessible to a broad range of mathematicians and scientists.
The algorithmic approach to the subject has two
principal aims: developing new tools for research within mathematics,
and providing new tools for modeling and solving problems
that arise in the sciences and engineering.  A healthy
synergy emerges, as new theorems yield new algorithms and emerging
applications lead to new theoretical questions.

This book presents algorithmic tools for algebraic geometry and experimental 
applications of them.  It also introduces a
software system in which
the tools have been implemented and with which the experiments can
be carried out.  \Mtwo is a computer
algebra system devoted to supporting research in algebraic geometry,
commutative algebra, and their applications. 
The reader of this book will encounter \Mtwo in the context of concrete
applications and
practical computations in algebraic geometry.  

The expositions of the algorithmic tools presented here are designed to serve
as a useful guide for those wishing to bring such tools to bear on their own
problems.
A wide range of mathematical scientists should find these expositions valuable.
This includes both the users of other programs
similar to \Mtwo (for example, Singular and CoCoA) and those who
are not interested in explicit machine computations at all.  

The chapters are ordered roughly by increasing mathematical
difficulty. The first part of the book is meant to be accessible to
graduate students and computer algebra users from across the
mathematical sciences and is primarily concerned
with introducing \Mtwo.  The second part emphasizes the
mathematics: each chapter exposes some domain of mathematics at an
accessible level, presents the relevant algorithms, sometimes with
proofs, and illustrates the use of the program.  In both parts, each chapter comes with
its own abstract and its own bibliography; the index at the back of
the book covers all of them.

One of the first computer algebra packages aimed at algebraic geometry
was {\sl Macaulay,} the predecessor of \Mtwo, written during the years
1983-1993 by Dave Bayer and Mike Stillman. Worst-case estimates
suggested that trying to compute Gr\"obner bases might be a hopeless
approach to solving problems. But from the first prototype, \Macaulay
was successful surprisingly often, perhaps because of the geometrical
origin of the problems attacked.  \Macaulay improved steadily during
its first decade.  It helped transform the theoretical notion of a
projective resolution into an exciting new practical research tool,
and became widely used for research and teaching in commutative
algebra and algebraic geometry.  It was possible to write routines in
the top-level language, and many important algorithms were added by
David Eisenbud and other users, enhancing the system and broadening its
usefulness.

There were certain practical drawbacks for the researcher who wanted to use
\Macaulay effectively.  A minor annoyance was that only finite prime fields
were available as coefficient rings.  The major problem was that the language
made available to users was primitive and barely supported high-level
development of new algorithms; it had few basic data types and didn't support
the addition of new ones. 

\Mtwo is based on experience gained from writing and using its predecessor
\Macaulay, but is otherwise a fresh start.  It was written by Dan Grayson and
Mike Stillman with the generous financial support of the U.S. National
Science Foundation, with the work starting in 1993\footnote{NSF grants DMS
  92-10805, 92-10807, 96-23232, 96-22608, 99-70085, and 99-70348.}.  
It also incorporates
some code from other authors: the package
SINGULAR-FACTORY\footnote{\texttt{SINGULAR-FACTORY},
        a subroutine library for factorization,
        by G.-M. Greuel, R. Stobbe, G. Pfister, H. Schoenemann, and J. Schmidt;
        available at ftp://helios.mathematik.uni-kl.de/pub/Math/Singular/Factory/.}
provides for factorization of polynomials; 
SINGULAR-LIBFAC\footnote{\texttt{SINGULAR-LIBFAC},
        a subroutine library for characteristic sets and irreducible decomposition,
        by M. Messollen;
        available at ftp://helios.mathematik.uni-kl.de/pub/Math/Singular/Libfac/.}
uses FACTORY to enable the computation of characteristic
sets and thus the decomposition of subvarieties into their irreducible
components; and
GNU MP\footnote{\texttt{GMP}, a library for arbitrary precision arithmetic,
        by Torbj\"orn Granlund,
        John Amanatides,
        Paul Zimmermann,
        Ken Weber,
        Bennet Yee,
        Andreas Schwab,
        Robert Harley,
        Linus Nordberg,
        Kent Boortz,
        Kevin Ryde, and
        Guillaume Hanrot;
        available at {ftp://ftp.gnu.org/gnu/gmp/}.}
by Torbj\"orn Granlund and others provides
for multiple precision arithmetic.

\Mtwo aims to support efficient computation associated with a wide
variety of high level mathematical objects, including Galois fields,
number fields, polynomial rings, exterior algebras, Weyl algebras,
quotient rings, ideals, modules, homomorphisms of rings and modules,
graded modules, maps between graded modules, chain complexes, maps
between chain complexes, free resolutions, algebraic varieties, and
coherent sheaves.  To make the system easily accessible, standard 
mathematical notation is followed closely.

As with \Macaulay, it was hoped that users would join in the further
development of new algorithms for \Mtwo, so the developers
tried to make the language available to
the users as powerful as possible, yet easy to use.
Indeed, much of the high-level part of the system is written in the same language
available to the user.  This ensures that the user will find it just as easy
as the developers did to implement a new type of mathematical object or to
modify the high-level aspects of the current algorithms.

The language available to the user is interpreted.  The interpreter itself
is written in a convenient language designed to be mostly type-safe and to
handle memory allocation and initialization automatically.  For maximum
efficiency, the core mathematical algorithms are written in C++ and compiled,
not interpreted.  This includes the arithmetic operations of rings, modules,
and matrices, the Gr\"obner basis algorithm (in several enhanced versions,
tailored for various situations), several algorithms for computing free
resolutions of modules, the algorithm for computing the Hilbert series of a
graded ring or module, the algorithms for computing determinants and
Pfaffians, the basis reduction algorithm, factoring, etc.

In one way \Mtwo is like a standard computer algebra system, such as {\sl
Mathematica} or {\sl Maple}: the user enters
mathematical expressions at the keyboard, and the program computes the value
of the expression and displays the answer.

Here is the first input prompt offered to the user.
\[\text{\tt i1\ :\ }\]
In response to the prompt, the user may enter, for example, a simple arithmetic expression.
\beginOutput
i1 : 3/5 + 7/11\\
\emptyLine
\     68\\
o1 = --\\
\     55\\
\emptyLine
o1 : QQ\\
\endOutput
The answer itself is displayed to the right of the output
label \[\text{\tt o1\ = \ }\] and its type (or class) is displayed to the
right of the following label.
\[\text{\tt o1\ :\ }\]
The symbol {\tt QQ} appearing in this example 
denotes the class of all rational numbers, and is meant to be reminiscent of
the notation $\mathbb Q$.

\Mtwo often finds itself being run in a window with horizontal scroll bars,
so by default it does not wrap output lines, but instead lets them grow
without bound.  This book was produced by an automated mechanism that submits
code provided by the authors to \Mtwo and incorporates the result into the
text.  Output lines that exceed the width of the pages of this book are
indicated with ellipses, as in the following example.
\beginOutput
i2 : 100!\\
\emptyLine
o2 = 93326215443944152681699238856266700490715968264381621468592963895 $\cdot\cdot\cdot$\\
\endOutput

Next we describe an important difference between 
general computer algebra systems (such as {\sl Maple} and {\sl Mathematica})
and \Mtwo.  Before entering an expression involving variables (such as {\tt x+y})
into \Mtwo the user must first create a ring containing those variables.  
Rings are important
objects of study in algebraic geometry; quotient rings of polynomial rings,
for example, encapsulate the essential information about a system of
polynomial equations, including, for example, the field from which the
coefficients are drawn.  Often one has several rings under consideration at
once, along with ring homomorphisms between them, so it is important to treat
them as first-class objects in the computer, capable of being named and
manipulated the same way numbers and characters can be manipulated in simpler
programming languages.

Let's give a hint of the breadth of types of mathematical objects available
in \Mtwo with some examples.
In \Mtwo one defines a quotient ring of a polynomial ring $R$ over the
rational numbers by entering a command such as the one below.
\beginOutput
i3 : R = QQ[x,y,z]/(x^3-y^3-z^3)\\
\emptyLine
o3 = R\\
\emptyLine
o3 : QuotientRing\\
\endOutput
Having done that, we can compute in the ring.
\beginOutput
i4 : (x+y+z)^3\\
\emptyLine
\       2        2     3     2               2        2       2     3\\
o4 = 3x y + 3x*y  + 2y  + 3x z + 6x*y*z + 3y z + 3x*z  + 3y*z  + 2z\\
\emptyLine
o4 : R\\
\endOutput
We can make matrices over the ring.
\beginOutput
i5 : b = vars R\\
\emptyLine
o5 = | x y z |\\
\emptyLine
\             1       3\\
o5 : Matrix R  <--- R\\
\endOutput
\beginOutput
i6 : c = matrix \{\{x^2,y^2,z^2\}\}\\
\emptyLine
o6 = | x2 y2 z2 |\\
\emptyLine
\             1       3\\
o6 : Matrix R  <--- R\\
\endOutput
We can make modules over the ring.
\beginOutput
i7 : M = coker b\\
\emptyLine
o7 = cokernel | x y z |\\
\emptyLine
\                            1\\
o7 : R-module, quotient of R\\
\endOutput
\beginOutput
i8 : N = ker c\\
\emptyLine
o8 = image \{2\} | x  0   -y2 -z2 |\\
\           \{2\} | -y -z2 x2  0   |\\
\           \{2\} | -z y2  0   x2  |\\
\emptyLine
\                             3\\
o8 : R-module, submodule of R\\
\endOutput
We can make projective resolutions of modules.
\beginOutput
i9 : res M\\
\emptyLine
\      1      3      4      4      4\\
o9 = R  <-- R  <-- R  <-- R  <-- R\\
\                                  \\
\     0      1      2      3      4\\
\emptyLine
o9 : ChainComplex\\
\endOutput
We can make projective varieties.
\beginOutput
i10 : X = Proj R\\
\emptyLine
o10 = X\\
\emptyLine
o10 : ProjectiveVariety\\
\endOutput
We can make coherent sheaves and compute their  cohomology.
\beginOutput
i11 : HH^1 cotangentSheaf X \\
\emptyLine
\        1\\
o11 = QQ\\
\emptyLine
o11 : QQ-module, free\\
\endOutput

At this writing, \Mtwo is available for GNU/Linux and other flavors of Unix, 
and also for Microsoft Windows and the Macintosh operating system. Although it can be used
as a free-standing program, it is most convenient to use it in an 
editor's buffer;
Emacs (on Unix or Windows systems) or MPW on Macintosh systems are
currently the editors of choice.
To obtain \Mtwo, download it from the
website\footnote{{\Mtwo}, a software system for research in algebraic geometry,
        by Daniel R. Grayson and Michael E. Stillman;
        available online in source code form and compiled for various architectures 
        at \texttt{http://www.math.uiuc.edu/Macaulay2/}.}
and unpack the file.
Among the resulting files will be a file called {\tt Macaulay2/README.txt},
which you should read.  It will tell you how to run the {\tt setup} script,
and how to install a few lines of code in your {\tt emacs} init file to
enable you to run {\tt M2} in an emacs buffer and to edit \Mtwo code.  A
system administrator of a Unix system may optionally arrange for those lines
of code to be available to every emacs user.

The editors thank the authors of the chapters for their valuable
contributions and hard work, and the National Science Foundation for funding
the development of \Mtwo and for partial funding of the authors who have
contributed to this volume.

\nobreak
\bigskip
\nobreak
\vbox{
    \noindent
    May, 2001       \hfill\itshape   David Eisenbud\break
    \raggedleft                      Daniel R. Grayson\break
                                     Michael E. Stillman\break
                                     Bernd Sturmfels\par
}

\egroup
\egroup

  \vfill\eject
  \rhpage
  \tableofcontents
  \vfill\eject
  \rhpage

%%%%%%%%%%%%%%%%%%%%%%%%%%%%%%%%%%%%%%%%%%%%%%%%
%%%%%
%%%%% addresses.tex
%%%%%
%%%%%%%%%%%%%%%%%%%%%%%%%%%%%%%%%%%%%%%%%%%%%%%%

\twocolumn
\bgroup

    \tracingpages=1

    \chapter*{List of Contributors}
    \addcontentsline{toc}{section}{List of Contributors}
    \parindent=0pt
    \font\afont = cmr8 \afont \baselineskip=9pt
    \font\bfont = cmbx8
    \def\addr#1{%
        \vskip 12pt
        \bgroup
        \catcode`\~=\the\catcode`a
        \catcode`\@=\the\catcode`a
        \obeylines
        {\bfont #1}%
        \vskip 2pt
        }
    \def\endaddr{\egroup}

    \def\aspace{\vskip 0 pt plus 5 pt minus 3 pt}

    \vbox to 432 pt {

        \addr{Luchezar L. Avramov}
                Department of Mathematics
                Purdue University
                West Lafayette, IN 47907, USA
                avramov@math.purdue.edu
                www.math.purdue.edu/~avramov/
        \endaddr

        \aspace

        \addr{Wolfram Decker}
                FB Mathematik
                Universit\"at des Saarlandes
                66041 Saarbr\"ucken, Germany
                decker@math.uni-sb.de
                loge.math.uni-sb.de/~agdecker/
        \endaddr

        \aspace

        \addr{David Eisenbud}
                Mathematical Sciences Research Institute
                1000 Centennial Drive
                Berkeley, CA 94720, USA,
                and
                Department of Mathematics
                University of California at Berkeley
                Berkeley, CA 94720, USA
                de@msri.org
                www.msri.org/people/staff/de/
        \endaddr

        \aspace

        \addr{Daniel R. Grayson}
                Department of Mathematics
                University of Illinois at Urbana-Champaign
                Urbana, IL 61801, USA
                dan@math.uiuc.edu
                www.math.uiuc.edu/~dan/
        \endaddr

        \aspace

        \addr{Serkan Ho\c{s}ten}
                Department of Mathematics
                San Francisco State University
                San Francisco, CA 94132, USA
                serkan@math.sfsu.edu
        \endaddr

        \aspace

        \addr{Frank-Olaf Schreyer}
                Fakult\"at f\"ur Mathematik und Physik
                Universit\"at Bayreuth
                95440 Bayreuth, Germany
                frank.schreyer@uni-bayreuth.de
        \endaddr

    }

    \vfill\eject

    \vbox to 432 pt {

        \addr{Gregory G. Smith}
		Department of Mathematics
		Barnard College, Columbia University
		New York, NY 10027, USA
                ggsmith@math.berkeley.edu
        \endaddr

        \aspace

        \addr{Frank Sottile}
                Department of Mathematics and Statistics
                University of Massachusetts, Amherst
                Amherst, MA 01003, USA
                sottile@math.umass.edu
                www.math.umass.edu/~sottile/
        \endaddr

        \aspace

        \addr{Michael Stillman}
                Department of Mathematics
                Cornell University
                Ithaca, NY 14853, USA
                mike@math.cornell.edu
        \endaddr

        \aspace

        \addr{Bernd Sturmfels}
                Department of Mathematics
                University of California
                Berkeley, California 94720, USA
                bernd@math.berkeley.edu
                www.math.berkeley.edu/~bernd/
        \endaddr

        \aspace

        \addr{Rekha Thomas}
                Department of Mathematics
                University of Washington
                Seattle, WA 98195, USA
                thomas@math.washington.edu
                www.math.washington.edu/~thomas/
        \endaddr

        \aspace

        \addr{Fabio Tonoli}
                Mathematisches Institut
                Georg--August Universit\"at
                37073 G\"ottingen, Germany
                tonoli@uni-math.gwdg.de
        \endaddr

        \aspace

        \addr{Uli Walther}
                Department of Mathematics
                Purdue University
                West Lafayette, IN 47907, USA
                walther@math.purdue.edu
                www.math.purdue.edu/~walther/
        \endaddr

    }

\egroup
\vfill\eject
\onecolumn

%%%%%%%%%%%%%%%%%%%%%%%%%%%%%%%%%%%%%%%%%%%%%%%%
%%%%%
%%%%% end of addresses.tex
%%%%%
%%%%%%%%%%%%%%%%%%%%%%%%%%%%%%%%%%%%%%%%%%%%%%%%

  \rhpage
  \pagenumbering{arabic}
  \mainmatter
  \part{Introducing \Mtwo}

%%%%%%%%%%%%%%%%%%%%%%%%%%%%%%%%%%%%%%%%%%%%%%%%
%%%%%
%%%%% ../chapters/varieties/chapter-m2.tex and ../chapters/varieties/chapter-wrapper.bbl
%%%%%
%%%%%%%%%%%%%%%%%%%%%%%%%%%%%%%%%%%%%%%%%%%%%%%%

  \bgroup
\title{Ideals, Varieties and \Mtwo}
\titlerunning{Ideals, Varieties and \Mtwo}
\toctitle{Ideals, Varieties and \Mtwo}
\author{Bernd Sturmfels\thanks{Partially supported by
the National Science Foundation (DMS-9970254).}}
\authorrunning{B. Sturmfels}
% \institute{University of California at Berkeley, Department of Mathematics,
%   Berkeley, CA 94720, USA}
\maketitle

\begin{abstract}
This chapter introduces \Mtwo commands for
some elementary computations in algebraic geometry.
Familiarity with Gr\"obner bases is assumed.
\end{abstract}

Many students and researchers alike have their first encounter with
Gr\"ob\-ner bases through the delightful text books \cite{CLO1} and \cite{CLO2}
by David Cox, John Little and Donal O'Shea. This chapter illustrates
the use of \Mtwo for some computations discussed in these books.
It can be used as a supplement for an advanced undergraduate course or 
first-year graduate course in computational algebraic geometry. The
mathematically advanced reader will find this chapter a useful summary
of some basic \Mtwo commands.

\section{A Curve in Affine Three-Space}

Our first example concerns geometric objects in
(complex) affine 3-space. We start by
setting up the ring of polynomial functions with rational coefficients.
\beginOutput
i1 : R = QQ[x,y,z]\\
\emptyLine
o1 = R\\
\emptyLine
o1 : PolynomialRing\\
\endOutput
Various monomial orderings are available in \Mtwo; since we did not specify
one explicitly, the monomials in the ring ${\tt R}$ will be sorted in 
graded reverse lexicographic order  \cite[\S I.2, Definition 6]{CLO1}.
We define an ideal generated by two polynomials
in this ring and assign it to the variable named 
{\tt curve}.

\beginOutput
i2 : curve = ideal( x^4-y^5, x^3-y^7 )\\
\emptyLine
\               5    4     7    3\\
o2 = ideal (- y  + x , - y  + x )\\
\emptyLine
o2 : Ideal of R\\
\endOutput
We compute the reduced Gr\"obner basis of our ideal:
\beginOutput
i3 : gb curve\\
\emptyLine
o3 = | y5-x4 x4y2-x3 x8-x3y3 |\\
\emptyLine
o3 : GroebnerBasis\\
\endOutput
By inspecting leading terms (and using \cite[\S 9.3, Theorem 8]{CLO1}),
we see that our ideal {\tt curve} does indeed 
define a one-dimensional affine variety. This can be tested directly
with the following commands in \Mtwo:
\beginOutput
i4 : dim curve\\
\emptyLine
o4 = 1\\
\endOutput
\beginOutput
i5 : codim curve\\
\emptyLine
o5 = 2\\
\endOutput
The {\it degree} of a curve in complex affine $3$-space is the 
number of intersection points with a general plane. It coincides
with the degree  \cite[\S 6.4]{CLO2} of the projective closure
\cite[\S 8.4]{CLO1} of our curve, which we compute as follows:
\beginOutput
i6 : degree curve\\
\emptyLine
o6 = 28\\
\endOutput
The Gr\"obner basis in {\tt o3} contains two polynomials which are not
irreducible: they contain a factor of $x^3$. This shows that our curve
is not irreducible over ${\bf Q}$. We first extract the components
which are transverse to the plane $x=0$:
\beginOutput
i7 : curve1 = saturate(curve,ideal(x))\\
\emptyLine
\               2       5    4   5    3\\
o7 = ideal (x*y  - 1, y  - x , x  - y )\\
\emptyLine
o7 : Ideal of R\\
\endOutput
And next we extract the component which lies in the plane $x=0$:
\beginOutput
i8 : curve2 = saturate(curve,curve1)\\
\emptyLine
\             3   5\\
o8 = ideal (x , y )\\
\emptyLine
o8 : Ideal of R\\
\endOutput
The second component is a multiple line. Hence our input ideal was not radical.
To test equality of ideals we use the command {\tt ==}\indexcmd{==} .
\beginOutput
i9 : curve == radical curve\\
\emptyLine
o9 = false\\
\endOutput
We now replace our curve by its first component:
\beginOutput
i10 : curve = curve1\\
\emptyLine
\                2       5    4   5    3\\
o10 = ideal (x*y  - 1, y  - x , x  - y )\\
\emptyLine
o10 : Ideal of R\\
\endOutput
\beginOutput
i11 : degree curve\\
\emptyLine
o11 = 13\\
\endOutput
The ideal of this curve is radical:
\beginOutput
i12 : curve == radical curve\\
\emptyLine
o12 = true\\
\endOutput
Notice that the variable ${\bf z}$ does not appear
among the generators of the ideal. Our curve consists of
$13$ straight lines (over {\bf C}) parallel to the {\tt z}-axis.

\section{Intersecting Our Curve With a Surface}

In this section we explore basic operations on ideals,
starting with those described in \cite[\S 4.3]{CLO1}.
Consider the following surface in affine $3$-space:
\beginOutput
i13 : surface = ideal( x^5 + y^5 + z^5 - 1)\\
\emptyLine
\             5    5    5\\
o13 = ideal(x  + y  + z  - 1)\\
\emptyLine
o13 : Ideal of R\\
\endOutput
The union of the curve and the surface is represented by the 
intersection of their ideals:
\beginOutput
i14 : theirunion = intersect(curve,surface)\\
\emptyLine
\              6 2      7      2 5    5    5    5      2       5 5    1 $\cdot\cdot\cdot$\\
o14 = ideal (x y  + x*y  + x*y z  - x  - y  - z  - x*y  + 1, x y  + y  $\cdot\cdot\cdot$\\
\emptyLine
o14 : Ideal of R\\
\endOutput
In this example this coincides with the product of the two ideals:
\beginOutput
i15 : curve*surface == theirunion\\
\emptyLine
o15 = true\\
\endOutput
The intersection of the curve and the surface is represented by the 
sum of their ideals. We get a finite set of points:
\beginOutput
i16 : ourpoints = curve + surface\\
\emptyLine
\                2       5    4   5    3   5    5    5\\
o16 = ideal (x*y  - 1, y  - x , x  - y , x  + y  + z  - 1)\\
\emptyLine
o16 : Ideal of R\\
\endOutput
\beginOutput
i17 : dim ourpoints\\
\emptyLine
o17 = 0\\
\endOutput
The number of points is sixty five:
\beginOutput
i18 : degree ourpoints\\
\emptyLine
o18 = 65\\
\endOutput
Each of the points is multiplicity-free:
\beginOutput
i19 : degree radical ourpoints\\
\emptyLine
o19 = 65\\
\endOutput
The number of points coincides with the number of 
monomials not in the initial ideal \cite[\S 2.2]{CLO2}.
These are called the {\it standard monomials}.
\beginOutput
i20 : staircase = ideal leadTerm ourpoints\\
\emptyLine
\                2   5   5   5\\
o20 = ideal (x*y , z , y , x )\\
\emptyLine
o20 : Ideal of R\\
\endOutput
The {\tt basis} command can be used to list all the standard monomials
\beginOutput
i21 : T = R/staircase;\\
\endOutput
\beginOutput
i22 : basis T\\
\emptyLine
o22 = | 1 x x2 x3 x4 x4y x4yz x4yz2 x4yz3 x4yz4 x4z x4z2 x4z3 x4z4 x3y $\cdot\cdot\cdot$\\
\emptyLine
\              1       65\\
o22 : Matrix T  <--- T\\
\endOutput

The assignment of the quotient ring to the global variable {\tt T} had a side
effect: the variables {\tt x}, {\tt y}, and {\tt z} now have values in that
ring.
To bring the variables of {\tt R} to the fore again, we must say:
\beginOutput
i23 : use R;\\
\endOutput
Every polynomial function on our 65 points can be written uniquely
as a linear combination of these standard monomials. This 
representation can be computed using the normal form command {\tt \%}\indexcmd{\%}.

\beginOutput
i24 : anyOldPolynomial = y^5*x^5-x^9-y^8+y^3*x^5\\
\emptyLine
\       5 5    9    5 3    8\\
o24 = x y  - x  + x y  - y\\
\emptyLine
o24 : R\\
\endOutput
\beginOutput
i25 : anyOldPolynomial {\char`\%} ourpoints\\
\emptyLine
\       4     3\\
o25 = x y - x y\\
\emptyLine
o25 : R\\
\endOutput
Clearly, the normal form is zero if and only the polynomial is in the ideal.
\beginOutput
i26 : anotherPolynomial = y^5*x^5-x^9-y^8+y^3*x^4\\
\emptyLine
\       5 5    9    8    4 3\\
o26 = x y  - x  - y  + x y\\
\emptyLine
o26 : R\\
\endOutput
\beginOutput
i27 : anotherPolynomial {\char`\%} ourpoints\\
\emptyLine
o27 = 0\\
\emptyLine
o27 : R\\
\endOutput


\section{Changing the Ambient Polynomial Ring}

During a \Mtwo session it sometimes becomes necessary to change the
ambient ring in which the computations takes place. Our original
ring, defined in {\tt i1}, is the polynomial ring in three variables
over the field  {\bf Q} of rational numbers
with the graded reverse lexicographic order. In this section 
two modifications are made: first we replace the field of coefficients
by a finite field, and later we replace the  monomial order
by an elimination order.

An important operation in algebraic geometry is 
the decomposition of algebraic varieties
into irreducible components \cite[\S 4.6]{CLO1}.
Algebraic algorithms for this purpose are based on the
{\it primary decomposition} of ideals \cite[\S 4.7]{CLO1}.
A future version of \Mtwo will have an implementation of
primary decomposition over any polynomial ring.
The current version of \Mtwo has a command
{\tt decompose} for finding all the minimal primes of an ideal,
but, as it stands, this works only over a finite field.

Let us change our coefficient field to the field with $101$ elements:
\beginOutput
i28 : R' = ZZ/101[x,y,z];\\
\endOutput

We next move our ideal from the previous section into the new ring
(fortunately, none of the coefficients of its generators have 101 in the
denominator):

\beginOutput
i29 : ourpoints' = substitute(ourpoints,R')\\
\emptyLine
\                2       5    4   5    3   5    5    5\\
o29 = ideal (x*y  - 1, y  - x , x  - y , x  + y  + z  - 1)\\
\emptyLine
o29 : Ideal of R'\\
\endOutput
\beginOutput
i30 : decompose ourpoints'\\
\emptyLine
\                                                                       $\cdot\cdot\cdot$\\
o30 = \{ideal (z + 36, y - 1, x - 1), ideal (z + 1, y - 1, x - 1), idea $\cdot\cdot\cdot$\\
\emptyLine
o30 : List\\
\endOutput
Oops, that didn't fit on the display, so let's print them out one per line.
\beginOutput
i31 : oo / print @@ print;\\
ideal (z + 36, y - 1, x - 1)\\
\emptyLine
ideal (z + 1, y - 1, x - 1)\\
\emptyLine
ideal (z - 6, y - 1, x - 1)\\
\emptyLine
ideal (z - 14, y - 1, x - 1)\\
\emptyLine
ideal (z - 17, y - 1, x - 1)\\
\emptyLine
\        3      2              2                      3    2     2      $\cdot\cdot\cdot$\\
ideal (x  - 46x  + 28x*y - 27y  + 46x + y + 27, - 16x  + x y + x  - 15 $\cdot\cdot\cdot$\\
\emptyLine
\            2                                            2             $\cdot\cdot\cdot$\\
ideal (- 32x  - 16x*y + x*z - 16x - 27y - 30z - 14, - 34x  - 14x*y + y $\cdot\cdot\cdot$\\
\emptyLine
\          2                                         2            2     $\cdot\cdot\cdot$\\
ideal (44x  + 22x*y + x*z + 22x - 26y - 30z - 6, 18x  + 12x*y + y  + 1 $\cdot\cdot\cdot$\\
\emptyLine
\            2                                           2            2 $\cdot\cdot\cdot$\\
ideal (- 41x  + 30x*y + x*z + 30x + 38y - 30z + 1, - 26x  - 10x*y + y  $\cdot\cdot\cdot$\\
\emptyLine
\          2                                            2            2  $\cdot\cdot\cdot$\\
ideal (39x  - 31x*y + x*z - 31x - 46y - 30z + 36, - 32x  - 13x*y + y   $\cdot\cdot\cdot$\\
\emptyLine
\            2                                          2            2  $\cdot\cdot\cdot$\\
ideal (- 10x  - 5x*y + x*z - 5x - 40y - 30z - 17, - 37x  + 35x*y + y   $\cdot\cdot\cdot$\\
\emptyLine
\endOutput
If we just want to see the degrees of the irreducible components, then
we say:
\beginOutput
i32 : ooo / degree\\
\emptyLine
o32 = \{1, 1, 1, 1, 1, 30, 6, 6, 6, 6, 6\}\\
\emptyLine
o32 : List\\
\endOutput
Note that the expressions ${\tt oo}$ 
and ${\tt ooo}$ refer to the previous and
prior-to-previous output lines respectively.

\medskip

Suppose we wish to compute the $x$-coordinates of our sixty five points.
Then we must use an elimination order, for instance, the
one described in \cite[\S 3.2, Exercise 6.a]{CLO1}.
We define a  new polynomial ring with the elimination order
for $\{y,z\} > \{x\}$ as follows:
\beginOutput
i33 : S = QQ[z,y,x, MonomialOrder => Eliminate 2]\\
\emptyLine
o33 = S\\
\emptyLine
o33 : PolynomialRing\\
\endOutput
We move our ideal into the new ring,
\beginOutput
i34 : ourpoints'' = substitute(ourpoints,S)\\
\emptyLine
\              2        5    4     3    5   5    5    5\\
o34 = ideal (y x - 1, y  - x , - y  + x , z  + y  + x  - 1)\\
\emptyLine
o34 : Ideal of S\\
\endOutput
and we compute the reduced Gr\"obner basis in this new order:
\beginOutput
i35 : G = gens gb ourpoints''\\
\emptyLine
o35 = | x13-1 y-x6 z5+x5+x4-1 |\\
\emptyLine
\              1       3\\
o35 : Matrix S  <--- S\\
\endOutput
To compute the elimination ideal we use the following command:
\beginOutput
i36 : ideal selectInSubring(1,G)\\
\emptyLine
\             13\\
o36 = ideal(x   - 1)\\
\emptyLine
o36 : Ideal of S\\
\endOutput

\section{Monomials Under the Staircase}

Invariants of an algebraic variety, such as its dimension
and degree, are computed from an initial monomial ideal.
This computation amounts to the combinatorial task
of analyzing the collection of standard monomials,
that is, the monomials under the staircase \cite[Chapter 9]{CLO1}.
In this section we demonstrate some basic operations on
monomial ideals in \Mtwo.

Let us create a non-trivial staircase in three dimensions
by taking the third power of the initial monomial from line {\tt i20}.
\beginOutput
i37 : M = staircase^3\\
\emptyLine
\              3 6   2 4 5   2 9   7 4     2 10     7 5   6 2 5     12  $\cdot\cdot\cdot$\\
o37 = ideal (x y , x y z , x y , x y , x*y z  , x*y z , x y z , x*y  , $\cdot\cdot\cdot$\\
\emptyLine
o37 : Ideal of R\\
\endOutput
The number of current generators of this ideal equals
\beginOutput
i38 : numgens M\\
\emptyLine
o38 = 20\\
\endOutput
To see all generators we can transpose the matrix of minimal generators:
\beginOutput
i39 : transpose gens M\\
\emptyLine
o39 = \{-9\}  | x3y6   |\\
\      \{-11\} | x2y4z5 |\\
\      \{-11\} | x2y9   |\\
\      \{-11\} | x7y4   |\\
\      \{-13\} | xy2z10 |\\
\      \{-13\} | xy7z5  |\\
\      \{-13\} | x6y2z5 |\\
\      \{-13\} | xy12   |\\
\      \{-13\} | x6y7   |\\
\      \{-13\} | x11y2  |\\
\      \{-15\} | z15    |\\
\      \{-15\} | y5z10  |\\
\      \{-15\} | x5z10  |\\
\      \{-15\} | y10z5  |\\
\      \{-15\} | x5y5z5 |\\
\      \{-15\} | x10z5  |\\
\      \{-15\} | y15    |\\
\      \{-15\} | x5y10  |\\
\      \{-15\} | x10y5  |\\
\      \{-15\} | x15    |\\
\emptyLine
\              20       1\\
o39 : Matrix R   <--- R\\
\endOutput
Note that this generating set is not minimal; see {\tt o48} below.
The number of standard monomials equals
\beginOutput
i40 : degree M\\
\emptyLine
o40 = 690\\
\endOutput
To list all the standard monomials we first create the residue ring
\beginOutput
i41 : S = R/M\\
\emptyLine
o41 = S\\
\emptyLine
o41 : QuotientRing\\
\endOutput
and then we ask for a vector space basis of the residue ring:
\beginOutput
i42 : basis S\\
\emptyLine
o42 = | 1 x x2 x3 x4 x5 x6 x7 x8 x9 x10 x11 x12 x13 x14 x14y x14yz x14 $\cdot\cdot\cdot$\\
\emptyLine
\              1       690\\
o42 : Matrix S  <--- S\\
\endOutput
Let us count how many standard monomials there are of a given degree.
The following table represents the Hilbert function
of the residue ring.
\beginOutput
i43 : tally apply(flatten entries basis(S),degree)\\
\emptyLine
o43 = Tally\{\{0\} => 1  \}\\
\            \{1\} => 3\\
\            \{10\} => 63\\
\            \{11\} => 69\\
\            \{12\} => 73\\
\            \{13\} => 71\\
\            \{14\} => 66\\
\            \{15\} => 53\\
\            \{16\} => 38\\
\            \{17\} => 23\\
\            \{18\} => 12\\
\            \{19\} => 3\\
\            \{2\} => 6\\
\            \{3\} => 10\\
\            \{4\} => 15\\
\            \{5\} => 21\\
\            \{6\} => 28\\
\            \{7\} => 36\\
\            \{8\} => 45\\
\            \{9\} => 54\\
\emptyLine
o43 : Tally\\
\endOutput
Thus the largest degree of a standard monomial is nineteen,
and there are three standard monomials of that degree:
\beginOutput
i44 : basis(19,S)\\
\emptyLine
o44 = | x14yz4 x9yz9 x4yz14 |\\
\emptyLine
\              1       3\\
o44 : Matrix S  <--- S\\
\endOutput
The most recently defined ring involving {\tt x}, {\tt y}, and {\tt z} was
{\tt S}, so all computations involving those variables are done in the
residue ring {\tt S}.
%% The current ring {\tt S} is the residue ring. All calculations
%% are done in {\tt S}. 
For instance, we can also obtain the
standard monomials of  degree nineteen as follows:
\beginOutput
i45 : (x+y+z)^19\\
\emptyLine
\            14   4          9   9         4   14\\
o45 = 58140x  y*z  + 923780x y*z  + 58140x y*z\\
\emptyLine
o45 : S\\
\endOutput
An operation on ideals which will occur frequently throughout this
book is the computation of minimal free resolutions. This is done as follows:
\beginOutput
i46 : C = res M\\
\emptyLine
\       1      16      27      12\\
o46 = R  <-- R   <-- R   <-- R   <-- 0\\
\                                      \\
\      0      1       2       3       4\\
\emptyLine
o46 : ChainComplex\\
\endOutput
This shows that our ideal {\tt M} has sixteen minimal generators.
They are the entries in the leftmost matrix of the chain complex {\tt C}:
\beginOutput
i47 : C.dd_1\\
\emptyLine
o47 = | x3y6 x7y4 x2y9 x2y4z5 x11y2 xy12 x6y2z5 xy7z5 xy2z10 x15 y15 x $\cdot\cdot\cdot$\\
\emptyLine
\              1       16\\
o47 : Matrix R  <--- R\\
\endOutput
This means that four of the twenty generators in {\tt o39} were redundant.
We construct the set consisting of the four redundant generators
as follows:
\beginOutput
i48 : set flatten entries gens M - set flatten entries C.dd_1\\
\emptyLine
\            6 7   10 5   5 10   5 5 5\\
o48 = Set \{x y , x  y , x y  , x y z \}\\
\emptyLine
o48 : Set\\
\endOutput
Here {\tt flatten entries} turns the matrix ${\tt M}$ into a single list.
The command {\tt set} turns that list into a set, to which we
can apply the difference operation for sets.

Let us now take a look at the first syzygies 
(or {\it minimal S-pairs} \cite[\S 2.9]{CLO1})
among  the sixteen minimal generators.
They correspond to the columns of the second matrix in our resolution {\tt C}:
\beginOutput
i49 : C.dd_2\\
\emptyLine
o49 = \{9\}  | -y3 -x4 0   -z5 0   0   0   0   0   0   0   0   0   0   0 $\cdot\cdot\cdot$\\
\      \{11\} | 0   y2  0   0   0   -x4 0   0   -z5 0   0   0   0   0   0 $\cdot\cdot\cdot$\\
\      \{11\} | x   0   -y3 0   0   0   0   0   0   -z5 0   0   0   0   0 $\cdot\cdot\cdot$\\
\      \{11\} | 0   0   0   xy2 -y3 0   -x4 0   x5  y5  0   -z5 0   0   0 $\cdot\cdot\cdot$\\
\      \{13\} | 0   0   0   0   0   y2  0   0   0   0   0   0   0   -x4 0 $\cdot\cdot\cdot$\\
\      \{13\} | 0   0   x   0   0   0   0   -y3 0   0   0   0   0   0   0 $\cdot\cdot\cdot$\\
\      \{13\} | 0   0   0   0   0   0   y2  0   0   0   0   0   0   0   - $\cdot\cdot\cdot$\\
\      \{13\} | 0   0   0   0   x   0   0   0   0   0   -y3 0   0   0   0 $\cdot\cdot\cdot$\\
\      \{13\} | 0   0   0   0   0   0   0   0   0   0   0   xy2 -y3 0   0 $\cdot\cdot\cdot$\\
\      \{15\} | 0   0   0   0   0   0   0   0   0   0   0   0   0   y2  0 $\cdot\cdot\cdot$\\
\      \{15\} | 0   0   0   0   0   0   0   x   0   0   0   0   0   0   0 $\cdot\cdot\cdot$\\
\      \{15\} | 0   0   0   0   0   0   0   0   0   0   0   0   0   0   y $\cdot\cdot\cdot$\\
\      \{15\} | 0   0   0   0   0   0   0   0   0   0   x   0   0   0   0 $\cdot\cdot\cdot$\\
\      \{15\} | 0   0   0   0   0   0   0   0   0   0   0   0   0   0   0 $\cdot\cdot\cdot$\\
\      \{15\} | 0   0   0   0   0   0   0   0   0   0   0   0   x   0   0 $\cdot\cdot\cdot$\\
\      \{15\} | 0   0   0   0   0   0   0   0   0   0   0   0   0   0   0 $\cdot\cdot\cdot$\\
\emptyLine
\              16       27\\
o49 : Matrix R   <--- R\\
\endOutput
The first column represents the S-pair between the
first generator $x^3 y^6 $ and the third generator $x^2 y^9$.
It is natural to form the {\it S-pair graph} with $16$ vertices and
$27$ edges represented by  this matrix. According to the
general theory described in \cite{MS}, this is a planar graph
with $12$ regions. The regions correspond to the $12$ second syzygies,
that is, to the columns of the matrix
\beginOutput
i50 : C.dd_3\\
\emptyLine
o50 = \{12\} | z5  0   0   0   0   0   0   0   0   0   0   0   |\\
\      \{13\} | 0   z5  0   0   0   0   0   0   0   0   0   0   |\\
\      \{14\} | 0   0   z5  0   0   0   0   0   0   0   0   0   |\\
\      \{14\} | -y3 -x4 0   0   0   0   0   0   0   0   0   0   |\\
\      \{14\} | 0   0   -y5 z5  0   0   0   0   0   0   0   0   |\\
\      \{15\} | 0   0   0   0   z5  0   0   0   0   0   0   0   |\\
\      \{15\} | 0   0   0   0   -x5 z5  0   0   0   0   0   0   |\\
\      \{16\} | 0   0   0   0   0   0   z5  0   0   0   0   0   |\\
\      \{16\} | 0   y2  0   0   -x4 0   0   0   0   0   0   0   |\\
\      \{16\} | x   0   -y3 0   0   0   0   0   0   0   0   0   |\\
\      \{16\} | 0   0   0   0   0   0   -y5 z5  0   0   0   0   |\\
\      \{16\} | 0   0   0   -y3 0   -x4 0   0   0   0   0   0   |\\
\      \{16\} | 0   0   0   0   0   0   0   -y5 z5  0   0   0   |\\
\      \{17\} | 0   0   0   0   0   0   0   0   0   z5  0   0   |\\
\      \{17\} | 0   0   0   0   0   0   0   0   0   -x5 z5  0   |\\
\      \{17\} | 0   0   0   0   0   0   0   0   0   0   -x5 z5  |\\
\      \{18\} | 0   0   0   0   y2  0   0   0   0   -x4 0   0   |\\
\      \{18\} | 0   0   x   0   0   0   -y3 0   0   0   0   0   |\\
\      \{18\} | 0   0   0   0   0   y2  0   0   0   0   -x4 0   |\\
\      \{18\} | 0   0   0   x   0   0   0   -y3 0   0   0   0   |\\
\      \{18\} | 0   0   0   0   0   0   0   0   -y3 0   0   -x4 |\\
\      \{20\} | 0   0   0   0   0   0   0   0   0   y2  0   0   |\\
\      \{20\} | 0   0   0   0   0   0   x   0   0   0   0   0   |\\
\      \{20\} | 0   0   0   0   0   0   0   0   0   0   y2  0   |\\
\      \{20\} | 0   0   0   0   0   0   0   x   0   0   0   0   |\\
\      \{20\} | 0   0   0   0   0   0   0   0   0   0   0   y2  |\\
\      \{20\} | 0   0   0   0   0   0   0   0   x   0   0   0   |\\
\emptyLine
\              27       12\\
o50 : Matrix R   <--- R\\
\endOutput
But we are getting ahead of ourselves. Homological algebra and resolutions
will be covered in the next chapter, and monomial ideals
will appear in the chapter of Ho\c{s}ten and Smith.
Let us return to Cox, Little and O'Shea \cite{CLO2}.

\section{Pennies, Nickels, Dimes and Quarters}

We now come to an application of Gr\"obner bases which appears in
\cite[Section 8.1]{CLO2}: {\sl Integer Programming}. This is the problem of
 minimizing a linear objective function over the set of non-negative 
integer solutions of a system of linear equations.  We demonstrate
some techniques for doing this in \Mtwo. Along the way, we learn about
multigraded polynomial rings and how to compute
Gr\"obner bases with respect to monomial orders defined by weights.
Our running example is the linear system defined  by the matrix:
\beginOutput
i51 : A = \{\{1, 1, 1, 1\},\\
\           \{1, 5,10,25\}\}\\
\emptyLine
o51 = \{\{1, 1, 1, 1\}, \{1, 5, 10, 25\}\}\\
\emptyLine
o51 : List\\
\endOutput
For the algebraic study of integer programming problems, a good starting
point is to work in a multigraded polynomial ring, here in four variables:
\beginOutput
i52 : R = QQ[p,n,d,q, Degrees => transpose A]\\
\emptyLine
o52 = R\\
\emptyLine
o52 : PolynomialRing\\
\endOutput
The degree of each variable is the corresponding column vector of the matrix
Each variable represents one of the four coins in the U.S. currency system:
\beginOutput
i53 : degree d\\
\emptyLine
o53 = \{1, 10\}\\
\emptyLine
o53 : List\\
\endOutput
\beginOutput
i54 : degree q\\
\emptyLine
o54 = \{1, 25\}\\
\emptyLine
o54 : List\\
\endOutput
Each monomial represents a collection of coins. For instance, suppose
you own four  pennies, eight nickels, ten dimes, and three quarters:
\beginOutput
i55 : degree(p^4*n^8*d^10*q^3)\\
\emptyLine
o55 = \{25, 219\}\\
\emptyLine
o55 : List\\
\endOutput
Then you have a total of 25 coins worth two dollars and nineteen cents.
There are nine other possible ways of having 25 coins of the same value:
\beginOutput
i56 : h = basis(\{25,219\}, R)\\
\emptyLine
o56 = | p14n2d2q7 p9n8d2q6 p9n5d6q5 p9n2d10q4 p4n14d2q5 p4n11d6q4 p4n8 $\cdot\cdot\cdot$\\
\emptyLine
\              1       9\\
o56 : Matrix R  <--- R\\
\endOutput
For just counting the number of columns of this matrix
we can use the command
\beginOutput
i57 : rank source h\\
\emptyLine
o57 = 9\\
\endOutput
How many ways can you make change for ten dollars using $100$ coins?
\beginOutput
i58 : rank source basis(\{100,1000\}, R)\\
\emptyLine
o58 = 182\\
\endOutput
A typical integer programming problem is this: among all 182 ways of
expressing ten dollars using 100 coins, which one uses the fewest dimes?
We set up the  Conti-Traverso algorithm \cite[\S 8.1]{CLO2} for 
answering this question. We use the following ring with the lexicographic
order and with the variable order:
dimes (d) before pennies (p) before nickels (n) before quarters (q).
\beginOutput
i59 : S = QQ[x, y, d, p, n, q, \\
\          MonomialOrder => Lex, MonomialSize => 16]\\
\emptyLine
o59 = S\\
\emptyLine
o59 : PolynomialRing\\
\endOutput
The option {\tt MonomialSize} advises \Mtwo to use more space to store the
exponents of monomials, thereby avoiding a potential overflow.

We define an ideal with one generator for each column of the matrix A.
\beginOutput
i60 : I = ideal( p - x*y, n - x*y^5, d - x*y^10, q - x*y^25)\\
\emptyLine
\                             5           10           25\\
o60 = ideal (- x*y + p, - x*y  + n, - x*y   + d, - x*y   + q)\\
\emptyLine
o60 : Ideal of S\\
\endOutput
The integer program is solved by normal form reduction with respect
to the following Gr\"obner basis consisting of binomials.
\beginOutput
i61 : transpose gens gb I\\
\emptyLine
o61 = \{-6\}  | p5q-n6     |\\
\      \{-4\}  | d4-n3q     |\\
\      \{-3\}  | yn2-dp     |\\
\      \{-6\}  | yp4q-dn4   |\\
\      \{-4\}  | yd3-pnq    |\\
\      \{-6\}  | y2p3q-d2n2 |\\
\      \{-5\}  | y2d2n-p2q  |\\
\      \{-7\}  | y2d2p3-n5  |\\
\      \{-6\}  | y3p2q-d3   |\\
\      \{-6\}  | y3dp2-n3   |\\
\      \{-5\}  | y4p-n      |\\
\      \{-6\}  | y5n-d      |\\
\      \{-8\}  | y6d2-pq    |\\
\      \{-16\} | y15d-q     |\\
\      \{-7\}  | xq-y5d2    |\\
\      \{-5\}  | xn-y3p2    |\\
\      \{-2\}  | xd-n2      |\\
\      \{-2\}  | xy-p       |\\
\emptyLine
\              18       1\\
o61 : Matrix S   <--- S\\
\endOutput
We fix the quotient ring, so the reduction to normal form
will happen automatically.
\beginOutput
i62 : S' = S/I\\
\emptyLine
o62 = S'\\
\emptyLine
o62 : QuotientRing\\
\endOutput
You need at least two dimes to express one dollar with ten coins.
\beginOutput
i63 : x^10 * y^100\\
\emptyLine
\       2 6 2\\
o63 = d n q\\
\emptyLine
o63 : S'\\
\endOutput
But you can express ten dollars with a hundred coins none of which is a dime.
\beginOutput
i64 : x^100 * y^1000\\
\emptyLine
\       75 25\\
o64 = n  q\\
\emptyLine
o64 : S'\\
\endOutput
The integer program is infeasible if and only if the normal form still
contains the variable $x$ or the variable $y$. For instance, you cannot
express ten dollars with less than forty coins:
\beginOutput
i65 : x^39 * y^1000\\
\emptyLine
\       25 39\\
o65 = y  q\\
\emptyLine
o65 : S'\\
\endOutput
We now introduce a new term order on the polynomial ring, defined
by assigning a weight to each variable. Specifically, we assign
weights for each of the coins. For instance,
let pennies have weight 5, nickels weight 7, 
dimes weight 13 and quarters weight 17.
\beginOutput
i66 : weight = (5,7,13,17)\\
\emptyLine
o66 = (5, 7, 13, 17)\\
\emptyLine
o66 : Sequence\\
\endOutput
We set up a new ring with the resulting weight term order, and work modulo
the same ideal as before in this new ring.
\beginOutput
i67 : T = QQ[x, y, p, n, d, q, \\
\                Weights => \{\{1,1,0,0,0,0\},\{0,0,weight\}\},\\
\                MonomialSize => 16]/\\
\            (p - x*y, n - x*y^5, d - x*y^10, q - x*y^25);\\
\endOutput
One dollar with ten coins:
\beginOutput
i68 : x^10 * y^100\\
\emptyLine
\       5 2 3\\
o68 = p d q\\
\emptyLine
o68 : T\\
\endOutput
Ten dollars with one hundred coins:
\beginOutput
i69 : x^100 * y^1000\\
\emptyLine
\       60 3 37\\
o69 = p  n q\\
\emptyLine
o69 : T\\
\endOutput
Here is an optimal solution which involves all four types of coins:
\beginOutput
i70 : x^234 * y^5677\\
\emptyLine
\       2 4 3 225\\
o70 = p n d q\\
\emptyLine
o70 : T\\
\endOutput
\begin{thebibliography}{1}

\bibitem{CLO1}
David Cox, John Little, and Donal O'Shea:
\newblock {\em Ideals, varieties, and algorithms}.
\newblock Springer-Verlag, New York, second edition, 1997.
\newblock An introduction to computational algebraic geometry and commutative
  algebra.

\bibitem{CLO2}
David Cox, John Little, and Donal O'Shea:
\newblock {\em Using algebraic geometry}.
\newblock Springer-Verlag, New York, 1998.

\bibitem{MS}
Ezra Miller and Bernd Sturmfels:
\newblock Monomial ideals and planar graphs.
\newblock In S.~Lin M.~Fossorier, H.~Imai and A.~Poli, editors, {\em Applied
  Algebra, Algebraic Algorithms and Error-Correcting Codes}, volume 1719 of
  {\em Springer Lecture Notes in Computer Science}, pages 19--28, 1999.

\end{thebibliography}
  \egroup
 \makeatletter
 \renewcommand\thesection{\@arabic\c@section}
 \makeatother


%%%%%%%%%%%%%%%%%%%%%%%%%%%%%%%%%%%%%%%%%%%%%%%%
%%%%%
%%%%% ../chapters/geometry/chapter-m2.tex and ../chapters/geometry/chapter-wrapper.bbl
%%%%%
%%%%%%%%%%%%%%%%%%%%%%%%%%%%%%%%%%%%%%%%%%%%%%%%

  \bgroup
%%%% Commands to cover (and index):
%% Matrix stuff: matrix, map, source, target, jacobian, minors, det, resolution,
%% Ext, Tor, what a module is, presentation, coker, image, kernel, annihilator,
%% sheaf cohomology, transpose, prune, trim, random, chain complexes, dd, C.dd_i,
%% ring maps. saturate
%%% Didn't put in ``annihilator'', random, chain complexes


\title{Projective Geometry and Homological Algebra}
\titlerunning{Projective Geometry and Homological Algebra}
\toctitle{Projective Geometry and Homological Algebra}
\author{David Eisenbud\thanks{Supported by the NSF.}}
\authorrunning{D. Eisenbud}
% \institute{1000 Centennial Drive, Mathematical Sciences Research Institute, Berkeley, CA
% 94720, USA, and
% University of California at Berkeley, Department of Mathematics, Berkeley, CA
% 94720, USA}

\maketitle

\begin{abstract}
We provide an introduction to many of the homological
commands in \Mtwo (modules, free resolutions, Ext and Tor\dots)
by means of examples showing how to use homological tools to
study projective varieties.
\end{abstract}

%%% TeX requirements
% \input diagrams.tex
\def\P{{\mathbb P}} %% Projective space. Change to \Bbb if available -- done
\def\Z{{\mathbb Z}} %% The integers. Change to \Bbb if available -- done
\def\H{{\rm H}}
\def\cO{{\cal O}}
\def\O{{\cal O}}
\def\iso{{\cong}}

In this chapter we will illustrate how one can 
manipulate
projective varieties and sheaves,
using the rich collection of tools
\Mtwo provides.
One of our goals is to show how homological
methods can be effective in solving concrete geometric problems.
\index{homological methods!introduction to}

The first four sections can be read by anyone who knows about
projective varieties at the level of a first graduate course
and knows the definitions of Ext and Tor. The last section assumes
that the reader is familiar with the theory of curves and surfaces
roughly at the level of the books of Hartshorne \cite{Hartshorne} and 
Harris \cite{Harris}.

We will work with projective schemes over a field {\tt
kk}. \Mtwo can work over any finite field of characteristic at most
\index{32749}
32749, and also a variety of fields in characteristic 0 (except for
the primary decomposition commands, which at this writing are still
restricted to positive characteristics). Our main interest is in
geometry over an algebraically closed field of characteristic
0. Nevertheless, it is most convenient to work over a large prime
{}field. 
\index{finite fields!use of}
It is known that the intermediate results in Gr\"obner basis
computations (as in the Euclidean Algorithm computations they
generalize) often involve coefficients far larger than those in the
input data, so that work in characteristic zero essentially requires
infinite precision arithmetic, a significant additional overhead. If
we work over a finite field where the scalars can be represented in
one machine word, we avoid this coefficient explosion. Experience with
the sort of computations we will be doing shows that working over
$\Z/p$, where $p$ is a moderately large prime, gives results
identical to the results we would get in characteristic 0.
%% This isn't strictly true -- look at all the coefficients in your output
%% that have more than 3 digits: their digits will probably change in as p
%% varies.
Of course one still
has to be careful about the fact that our fields are not
algebraically closed, especially when using primary
decomposition. The largest prime $p$ we can work with being 32749, we
choose the field ${\bf Z}/32749$. 
The name of the \Mtwo constant representing the
integers is {\tt ZZ}, and by analogy we will call our field {\tt kk}:
\beginOutput
i1 : kk = ZZ/32749\\
\emptyLine
o1 = kk\\
\emptyLine
o1 : QuotientRing\\
\endOutput
%% I had to leave the characteristic at 32003, because your code breaks when
%% I put in 32749, which is actually the largest characteristic available.
%% The reason it breaks is probably that the coefficients in the polynomials
%% defining idealX are no longer correct, since they were precomputed in
%% characteristic 32003.  Should we compute idealX automatically?  One way
%% would be to define a function in a file called "mystery.m2" that computes
%% it for you.  You could reveal the code of the function later.

In \Mtwo we will represent \ie{projective space} $\P^n$ by its
homogeneous coordinate ring ${\tt ringPn} = {\tt kk}[x_0,\dots,x_n]$.  A
projective scheme $X$ in $\P^n$ may be most conveniently
represented, depending on the situation, by its \ie{homogeneous ideal} {\tt
idealX} or its \ie{homogeneous coordinate ring}, represented either as a
ring {\tt ringPn/idealX} or as a module {\tt OX} over {\tt ringPn}.
\index{coherent sheaf!representation of}
\index{sheaf!representation of}
Coherent sheaves on the projective space, or on its subvarieties, will
be represented by finitely generated
graded
modules over {\tt ringPn}, using the
\ie{Serre correspondence}.
%%% Dan, the previous line didn't print in the version you sent!
%%%  -- It's printing for me.
For example, the structure sheaf $\cO_X$
of the subvariety $X$ would be represented by the module 
{\tt ringPn\char`\^1/idealX}; here {\tt ringPn\char`\^1}
denotes the free module of rank one over the ring
{\tt ringPn}.

\section{The Twisted Cubic}

\index{twisted cubic}\index{cubic space curve}
As a first illustration, we give three constructions
of the twisted cubic
curve in $\P^3$. We represent $\P^3$ by
\beginOutput
i2 : ringP3 = kk[x_0..x_3]\\
\emptyLine
o2 = ringP3\\
\emptyLine
o2 : PolynomialRing\\
\endOutput

The twisted cubic is the image of the map $\P^1\to\P^3$ sending
a point with homogeneous coordinates $(s,t)$ to the point with
homogeneous coordinates $(s^3, s^2t, st^2, t^3)$. We can compute
its relations directly with
\index{matrix}\index{map of rings}
\beginOutput
i3 : ringP1 = kk[s,t]\\
\emptyLine
o3 = ringP1\\
\emptyLine
o3 : PolynomialRing\\
\endOutput
\beginOutput
i4 : cubicMap = map(ringP1,ringP3,\{s^3, s^2*t, s*t^2, t^3\})\\
\emptyLine
\                         3   2      2   3\\
o4 = map(ringP1,ringP3,\{s , s t, s*t , t \})\\
\emptyLine
o4 : RingMap ringP1 <--- ringP3\\
\endOutput
\index{kernel of a ring map}%
\beginOutput
i5 : idealCubic = kernel cubicMap\\
\emptyLine
\             2                       2\\
o5 = ideal (x  - x x , x x  - x x , x  - x x )\\
\             2    1 3   1 2    0 3   1    0 2\\
\emptyLine
o5 : Ideal of ringP3\\
\endOutput
We could also use \Mtwo's built-in facility, and say
\indexcmd{monomialCurveIdeal}
\beginOutput
i6 : idealCubic2 = monomialCurveIdeal(ringP3,\{1,2,3\})\\
\emptyLine
\                          2          2\\
o6 = ideal (x x  - x x , x  - x x , x  - x x )\\
\             1 2    0 3   2    1 3   1    0 2\\
\emptyLine
o6 : Ideal of ringP3\\
\endOutput
which uses precisely the same method.

Of course we might remember that the ideal of the twisted
cubic is generated by the $2\times 2$ minors of the 
matrix
$$\begin{pmatrix}
x_0&x_1&x_2\cr
x_1&x_2&x_3
\end{pmatrix},
$$
which we can realize with the commands
\beginOutput
i7 : M = matrix\{\{x_0,x_1,x_2\},\{x_1,x_2,x_3\}\}\\
\emptyLine
o7 = | x_0 x_1 x_2 |\\
\     | x_1 x_2 x_3 |\\
\emptyLine
\                  2            3\\
o7 : Matrix ringP3  <--- ringP3\\
\endOutput
\indexcmd{minors}\index{ideal}%
\beginOutput
i8 : idealCubic3 = minors(2, M)\\
\emptyLine
\               2                           2\\
o8 = ideal (- x  + x x , - x x  + x x , - x  + x x )\\
\               1    0 2     1 2    0 3     2    1 3\\
\emptyLine
o8 : Ideal of ringP3\\
\endOutput
We can get some useful information about the ideal
{\tt idealCubic} with
\indexcmd{codimension}%
\beginOutput
i9 : codim idealCubic\\
\emptyLine
o9 = 2\\
\endOutput
\index{degree of a projective variety}%
\beginOutput
i10 : degree idealCubic\\
\emptyLine
o10 = 3\\
\endOutput
This shows that we do indeed have a cubic curve. Note that the
command
\index{dimension of a projective variety}%
\beginOutput
i11 : dim idealCubic\\
\emptyLine
o11 = 2\\
\endOutput
gives 2, not 1; it represents the dimension of the ideal 
in {\tt ringP3}, the dimension of the affine cone over the
curve.

We can easily assure ourselves that these ideals are the same.
{}For example, to see whether the ideal {\tt idealCubic}
is contained in the ideal of minors of {\tt M}, we can
reduce the former modulo the latter, and see whether 
we get zero. The reduction operator {\tt \%} takes two
maps with the same target as its arguments, so
we must replace each ideal by a matrix whose entries 
generate it. This is done by the function {\tt gens} as in
\indexcmd{gens}\indexcmd{generators}%
\beginOutput
i12 : gens idealCubic\\
\emptyLine
o12 = | x_2^2-x_1x_3 x_1x_2-x_0x_3 x_1^2-x_0x_2 |\\
\emptyLine
\                   1            3\\
o12 : Matrix ringP3  <--- ringP3\\
\endOutput
Thus for one of the inclusions we check
\indexcmd{\%}%
\index{reduced form}%
\beginOutput
i13 : 0 == (gens idealCubic){\char`\%}(gens idealCubic3)\\
\emptyLine
o13 = true\\
\endOutput
Both inclusions can be checked automatically in this way
with
\beginOutput
i14 : idealCubic == idealCubic3\\
\emptyLine
o14 = true\\
\endOutput



\section{The Cotangent Bundle of $\P^3$}

\index{cotangent bundle}%
Many invariants of varieties are defined in
terms of their tangent and cotangent bundles. We 
identify a bundle with its sheaf of
sections, which is locally free.
Any coherent locally free sheaf arises this way.
(One can also regard a bundle
as a variety in its own right, but this view
is used in algebraic geometry more rarely.)
In this section and the next we construct
the cotangent  bundle $\Omega_{\P^3}$ of $\P^3$ and
its restriction to the twisted cubic above.

Consulting Hartshorne \cite[Theorem II.8.13]{Hartshorne}, we find 
that the cotangent bundle to $\P^n$ can be described
by the {\it cotangent sequence\/}:
\index{cotangent sequence}%
$$
0\rTo 
\Omega_{\P^n}\rTo 
\cO_{\P^n}(-1)^{n+1}\rTo^f 
\cO_{\P^n} \rTo 
0
$$
where $f$ is defined by the matrix 
of variables $(x_0,\dots,x_n)$.
We can translate this description directly into the 
language of \Mtwo, here in the case $n=3$:
\beginOutput
i15 : f = vars ringP3\\
\emptyLine
o15 = | x_0 x_1 x_2 x_3 |\\
\emptyLine
\                   1            4\\
o15 : Matrix ringP3  <--- ringP3\\
\endOutput
\index{kernel of a module map}%
\beginOutput
i16 : OmegaP3 = kernel f\\
\emptyLine
o16 = image \{1\} | 0    0    0    -x_1 -x_2 -x_3 |\\
\            \{1\} | 0    -x_2 -x_3 x_0  0    0    |\\
\            \{1\} | -x_3 x_1  0    0    x_0  0    |\\
\            \{1\} | x_2  0    x_1  0    0    x_0  |\\
\emptyLine
\                                        4\\
o16 : ringP3-module, submodule of ringP3\\
\endOutput
\index{modules!how to represent}%
Note that the module which we specified as a kernel
is now given as the image of a matrix. We can recover this
matrix with
\beginOutput
i17 : g=generators OmegaP3\\
\emptyLine
o17 = \{1\} | 0    0    0    -x_1 -x_2 -x_3 |\\
\      \{1\} | 0    -x_2 -x_3 x_0  0    0    |\\
\      \{1\} | -x_3 x_1  0    0    x_0  0    |\\
\      \{1\} | x_2  0    x_1  0    0    x_0  |\\
\emptyLine
\                   4            6\\
o17 : Matrix ringP3  <--- ringP3\\
\endOutput
and we could correspondingly write
\indexcmd{image}%
\beginOutput
i18 : OmegaP3=image g\\
\emptyLine
o18 = image \{1\} | 0    0    0    -x_1 -x_2 -x_3 |\\
\            \{1\} | 0    -x_2 -x_3 x_0  0    0    |\\
\            \{1\} | -x_3 x_1  0    0    x_0  0    |\\
\            \{1\} | x_2  0    x_1  0    0    x_0  |\\
\emptyLine
\                                        4\\
o18 : ringP3-module, submodule of ringP3\\
\endOutput
An even more elementary way to give a module is by generators
and relations, and we can see this ``free presentation'' too with
\indexcmd{presentation}%
\beginOutput
i19 : presentation OmegaP3\\
\emptyLine
o19 = \{2\} | x_1  0    0    x_0  |\\
\      \{2\} | x_3  x_0  0    0    |\\
\      \{2\} | -x_2 0    x_0  0    |\\
\      \{2\} | 0    x_2  x_3  0    |\\
\      \{2\} | 0    -x_1 0    x_3  |\\
\      \{2\} | 0    0    -x_1 -x_2 |\\
\emptyLine
\                   6            4\\
o19 : Matrix ringP3  <--- ringP3\\
\endOutput
The astute reader will have noticed that we have just been computing
the first few terms in the free resolution of the cokernel of the
map of free modules corresponding to
{\tt f}. We could see the whole resolution at once with
\index{resolution!free}\indexcmd{res}\indexcmd{coker}\indexcmd{cokernel}
\index{chain complex}\index{complex}
\beginOutput
i20 : G = res coker f\\
\emptyLine
\            1           4           6           4           1\\
o20 = ringP3  <-- ringP3  <-- ringP3  <-- ringP3  <-- ringP3  <-- 0\\
\                                                                   \\
\      0           1           2           3           4           5\\
\emptyLine
o20 : ChainComplex\\
\endOutput
and then see all the matrices in the resolution with
\indexcmd{dd}\index{differentials of a complex}
\index{complex!differentials in}
\beginOutput
i21 : G.dd\\
\emptyLine
\                1                                4\\
o21 = 0 : ringP3  <----------------------- ringP3  : 1\\
\                     | x_0 x_1 x_2 x_3 |\\
\emptyLine
\                4                                                  6\\
\      1 : ringP3  <----------------------------------------- ringP3  : 2\\
\                     \{1\} | -x_1 -x_2 0    -x_3 0    0    |\\
\                     \{1\} | x_0  0    -x_2 0    -x_3 0    |\\
\                     \{1\} | 0    x_0  x_1  0    0    -x_3 |\\
\                     \{1\} | 0    0    0    x_0  x_1  x_2  |\\
\emptyLine
\                6                                        4\\
\      2 : ringP3  <------------------------------- ringP3  : 3\\
\                     \{2\} | x_2  x_3  0    0    |\\
\                     \{2\} | -x_1 0    x_3  0    |\\
\                     \{2\} | x_0  0    0    x_3  |\\
\                     \{2\} | 0    -x_1 -x_2 0    |\\
\                     \{2\} | 0    x_0  0    -x_2 |\\
\                     \{2\} | 0    0    x_0  x_1  |\\
\emptyLine
\                4                         1\\
\      3 : ringP3  <---------------- ringP3  : 4\\
\                     \{3\} | -x_3 |\\
\                     \{3\} | x_2  |\\
\                     \{3\} | -x_1 |\\
\                     \{3\} | x_0  |\\
\emptyLine
\                1\\
\      4 : ringP3  <----- 0 : 5\\
\                     0\\
\emptyLine
o21 : ChainComplexMap\\
\endOutput
or just one of them, say the second, with
\index{dd\_i@{\tt dd\_i}, i-th differential of a complex}
\beginOutput
i22 : G.dd_2\\
\emptyLine
o22 = \{1\} | -x_1 -x_2 0    -x_3 0    0    |\\
\      \{1\} | x_0  0    -x_2 0    -x_3 0    |\\
\      \{1\} | 0    x_0  x_1  0    0    -x_3 |\\
\      \{1\} | 0    0    0    x_0  x_1  x_2  |\\
\emptyLine
\                   4            6\\
o22 : Matrix ringP3  <--- ringP3\\
\endOutput
Note that this matrix does not look exactly the same as
the matrix produced by computing the kernel of {\tt f}.
This is because when \Mtwo is asked to compute a whole
resolution, it does not do the ``obvious'' thing and
compute kernels over and over; it defaults to a more efficient
algorithm, first proposed by Frank Schreyer
\cite[Appendix]{s1}.
\index{Schreyer's algorithm for free resolutions}

Any graded map of free modules, such as a map in a graded
{}free resolution of a graded module, comes with some numerical data: 
the degrees of the 
generators of the source and target free modules.
We can extract this information one module at a time with
the command {\tt degrees}, as in
\indexcmd{degrees}
\indexcmd{source}\indexcmd{target}
\beginOutput
i23 : degrees source G.dd_2\\
\emptyLine
o23 = \{\{2\}, \{2\}, \{2\}, \{2\}, \{2\}, \{2\}\}\\
\emptyLine
o23 : List\\
\endOutput
\beginOutput
i24 : degrees target G.dd_2\\
\emptyLine
o24 = \{\{1\}, \{1\}, \{1\}, \{1\}\}\\
\emptyLine
o24 : List\\
\endOutput

\Mtwo has a more convenient
mechanism for examining this numerical data,
which we take time out to explain. First, for the resolution just
computed, we can call
\beginOutput
i25 : betti G\\
\emptyLine
o25 = total: 1 4 6 4 1\\
\          0: 1 4 6 4 1\\
\endOutput
\index{Betti diagram}\index{diagram, Betti}\indexcmd{betti}%
The diagram shows the degrees of the generators of each free module
in the resolution in coded form. To understand the code, it may
be helpful to look at a less symmetric example, say the free
resolution of {\tt ringP3\char`\^1/I} where $I$ is
the ideal generated by the minors of the following $2\times 4$ matrix.
\beginOutput
i26 : m = matrix\{\{x_0^3, x_1^2, x_2,x_3\},\{x_1^3,x_2^2,x_3,0\}\}\\
\emptyLine
o26 = | x_0^3 x_1^2 x_2 x_3 |\\
\      | x_1^3 x_2^2 x_3 0   |\\
\emptyLine
\                   2            4\\
o26 : Matrix ringP3  <--- ringP3\\
\endOutput
We do this with
\beginOutput
i27 : I = minors(2,m)\\
\emptyLine
\                5    3 2     3      3       3    2      3      2      2\\
o27 = ideal (- x  + x x , - x x  + x x , - x  + x x , -x x , -x x , -x )\\
\                1    0 2     1 2    0 3     2    1 3    1 3    2 3    3\\
\emptyLine
o27 : Ideal of ringP3\\
\endOutput
\beginOutput
i28 : F = res(ringP3^1/I)\\
\emptyLine
\            1           6           8           3\\
o28 = ringP3  <-- ringP3  <-- ringP3  <-- ringP3  <-- 0\\
\                                                       \\
\      0           1           2           3           4\\
\emptyLine
o28 : ChainComplex\\
\endOutput
\beginOutput
i29 : betti F\\
\emptyLine
o29 = total: 1 6 8 3\\
\          0: 1 . . .\\
\          1: . 1 . .\\
\          2: . 2 2 .\\
\          3: . 2 2 .\\
\          4: . 1 4 3\\
\endOutput

The resulting Betti diagram should be interpreted as follows.
{}First, the maps go from right to left, so the beginning of the 
resolution is on the left. The given Betti diagram
thus corresponds to an exact sequence of graded free modules
$$
{}F_0\lTo F_1\lTo F_2\lTo F_3\lTo 0.
$$
The top row of the diagram, 1,6,8,3, shows the
ranks of the free modules $F_i$ in the resolution. For example the 1
on the left means that $F_0$ has rank 1 (and,
indeed, the module {\tt ringP3\char`\^1/I} we are resolving is cyclic). 
The 6 shows that the rank of $F_1$ is 6, or equivalently that
the ideal {\tt I} is minimally generated by 6 elements---in this
case the $6 = {\binom 4 2}$ minors of size 2 of the $2\times 4$
matrix $m$. 

The first column of the diagram shows degrees. The successive
columns indicate how many generators of each degree occur in the
successive $F_i$.  The free module $F_0$ has a single generator in
degree 0, and this is the significance of the second column.  Note
that $F_1$ could not have any generators of degree less than or equal
to zero, because the resolution is minimal! Thus for compactness, the
diagram is skewed: in each successive column the places correspond to
larger degrees.  More precisely, a number $a$ occurring opposite the
degree indication ``{\tt i:}'' in the column corresponding to $F_j$ signifies
that $F_j$ has $a$ generators in degree $i+j$.  Thus for example the 1
in the third column opposite the one on the left corresponds to a
generator of degree 2 in the free module $F_1$; and
altogether $F_1$ has one generator of degree 2, two generators of
degree 3, two of degree 4 and one of degree 5.

Returning to the diagram
\beginOutput
i30 : betti G\\
\emptyLine
o30 = total: 1 4 6 4 1\\
\          0: 1 4 6 4 1\\
\endOutput
we see that the successive free modules of {\tt G} are 
each generated in degree 1 higher than the previous one; that
is, the matrices in {\tt G.dd} all have linear entries, as
we have already seen.




\section{The Cotangent Bundle of a Projective Variety}

\index{cotangent bundle}%
It is easy to construct the cotangent bundle $\Omega_X$ of a projective
variety $X$ starting from the cotangent bundle 
of the ambient
projective space. We use the {\it conormal sequence}
\index{conormal sequence}
(Hartshorne \cite[Proposition II.8.12]{Hartshorne} or Eisenbud
\cite[Proposition 16.3]{eCA}).
Writing $I$ for the ideal of a variety $X$ in $\P^n$ there is
an exact sequence of sheaves
$$
I \rTo^\delta \Omega_{\P^n}\otimes \cO_X\rTo \Omega_X\rTo 0
$$
where the map $\delta$ takes a function $f$
to the element $df\otimes 1$. If $I$ is generated by 
{}forms $f_1,\dots,f_m$ then $\delta$ is represented
by the \ie{Jacobian matrix} $(df_i/dx_j)$. 


{}First of all, we must compute a module corresponding to
$\Omega_{\P^n}\otimes \cO_X$, the restriction of the sheaf
$\Omega_{\P^n}$ to $X$. The simplest approach would be
to take the tensor product of graded modules representing
$\Omega_{\P^n}$ and  $\cO_X$. The result would represent the
right sheaf, but would not be the module of twisted global
sections of $\Omega_{\P^n}\otimes \cO_X$ (the unique module
of depth two representing the sheaf). This would make further
computations less efficient. 

Thus we take a different approach:
since the cotangent sequence given 
in the previous section
is a sequence of locally free sheaves, it is locally split, and
thus remains exact when tensored by $\cO_X$. Consequently
$\Omega_{\P^n}\otimes \cO_X$ is also represented by the kernel
of the map $f\otimes  \cO_X$, where $f$ is the map used
in the definition of the cotangent bundle of $\P^n$.
In \Mtwo, working on $\P^3$, with $X$ the twisted cubic, we
can translate this into
\beginOutput
i31 : OmegaP3res = kernel (f ** (ringP3^1/idealCubic))\\
\emptyLine
o31 = subquotient (\{1\} | -x_3 0    0    -x_2 -x_3 0    -x_1 -x_2 -x_3  $\cdot\cdot\cdot$\\
\                   \{1\} | x_2  -x_3 0    x_1  0    -x_3 x_0  0    0     $\cdot\cdot\cdot$\\
\                   \{1\} | 0    x_2  -x_3 0    x_1  0    0    x_0  0     $\cdot\cdot\cdot$\\
\                   \{1\} | 0    0    x_2  0    0    x_1  0    0    x_0   $\cdot\cdot\cdot$\\
\emptyLine
\                                          4\\
o31 : ringP3-module, subquotient of ringP3\\
\endOutput
(The operator {\tt \char`\*\char`\*} is \Mtwo's symbol for tensor product.)
Since the map is a map between free modules
over {\tt ringP3/idealCubic}, the kernel has depth (at least) two.

Next, we form the Jacobian matrix of the generators
of
% the ideal
{\tt idealCubic}, which represents a map
from this ideal to the free module {\tt ringP3\char`\^4}.
\indexcmd{jacobian}
\beginOutput
i32 : delta1 = jacobian idealCubic\\
\emptyLine
o32 = \{1\} | 0    -x_3 -x_2 |\\
\      \{1\} | -x_3 x_2  2x_1 |\\
\      \{1\} | 2x_2 x_1  -x_0 |\\
\      \{1\} | -x_1 -x_0 0    |\\
\emptyLine
\                   4            3\\
o32 : Matrix ringP3  <--- ringP3\\
\endOutput
We need to make this into a map to {\tt OmegaP3res}, which
as defined is a subquotient of {\tt ringP3\char`\^4}. To this end
we must first express the image of {\tt delta1} in terms of
the generators of {\tt OmegaP3res}. The \ie{division command} {\tt //}
does this with
\beginOutput
i33 : delta2 = delta1 // (gens OmegaP3res)\\
\emptyLine
o33 = \{2\} | 0  1  0  |\\
\      \{2\} | 2  0  0  |\\
\      \{2\} | 0  0  0  |\\
\      \{2\} | 0  0  2  |\\
\      \{2\} | 0  1  0  |\\
\      \{2\} | -1 0  0  |\\
\      \{2\} | 0  0  0  |\\
\      \{2\} | 0  0  -1 |\\
\      \{2\} | 0  -1 0  |\\
\emptyLine
\                   9            3\\
o33 : Matrix ringP3  <--- ringP3\\
\endOutput
Once this is done we can use this matrix to form the
necessary map $\delta: I\to \Omega_{\P^3}\otimes\O_X$:
\index{map of modules}
\beginOutput
i34 : delta = map(OmegaP3res, module idealCubic, delta2)\\
\emptyLine
o34 = \{2\} | 0  1  0  |\\
\      \{2\} | 2  0  0  |\\
\      \{2\} | 0  0  0  |\\
\      \{2\} | 0  0  2  |\\
\      \{2\} | 0  1  0  |\\
\      \{2\} | -1 0  0  |\\
\      \{2\} | 0  0  0  |\\
\      \{2\} | 0  0  -1 |\\
\      \{2\} | 0  -1 0  |\\
\emptyLine
o34 : Matrix\\
\endOutput
A minimal free presentation of $\Omega_X$ --- or rather of
one module over
{\tt ringP3} that represents it --- can be obtained with
\indexcmd{prune}\index{presentation!minimal}
\beginOutput
i35 : OmegaCubic = prune coker delta\\
\emptyLine
o35 = cokernel \{2\} | -10917x_3 0    -10917x_3 x_2      0        0      $\cdot\cdot\cdot$\\
\               \{2\} | 0         0    x_2       0        16374x_3 0      $\cdot\cdot\cdot$\\
\               \{2\} | 0         -x_3 0         16373x_3 0        x_2    $\cdot\cdot\cdot$\\
\               \{2\} | x_3       x_2  0         0        0        0      $\cdot\cdot\cdot$\\
\               \{2\} | 0         0    0         0        0        -2x_3  $\cdot\cdot\cdot$\\
\               \{2\} | 0         0    0         0        x_2      0      $\cdot\cdot\cdot$\\
\emptyLine
\                                       6\\
o35 : ringP3-module, quotient of ringP3\\
\endOutput
We have used the function {\tt prune} to compute
minimal presentation matrices; these often make subsequent
computations faster, and also allow us to inspect the final 
answer more easily. 

The module {\tt OmegaCubic} represents the sheaf $\Omega_{X}$, where
$X$ is the cubic, but it is not the simplest possibility.
A better representative is the graded module
$\oplus_{d\in \Z} \H^0(\Omega_X(d))$. We can at least find 
a minimal presentation of the
submodule $\oplus_{d\geq 0} \H^0(\Omega_X(d))$ with
\index{sheaf cohomology}\index{cohomology!sheaf}\indexcmd{HH}\indexcmd{sheaf}
\beginOutput
i36 : prune HH^0((sheaf OmegaCubic)(>=0))\\
\emptyLine
o36 = cokernel \{1\} | 16374x_3 16374x_2 16374x_1 |\\
\               \{1\} | x_2      x_1      x_0      |\\
\emptyLine
\                                       2\\
o36 : ringP3-module, quotient of ringP3\\
\endOutput
The large coefficients appearing in the matrix arise in finite characteristic as the result
of chance division by small integers.
We see from the degrees labeling the rows of the matrix in the
output of this command that
the generators of the submodule are in degree 1,
so in particular $\H^0(\Omega_X) = 0$.
It follows that that
$\H^0(\Omega_X(d)) = 0$ for all $d\leq 0$, so
the submodule we computed was actually the whole module
that we wanted! (If this had not been the case we could have tried
{\tt HH\char`\^0((sheaf OmegaCubic)(>=d))} to compute the cohomology
of all the twists greater than a given negative integer $d$,
or simply used the submodule we had already computed, since
it also represents the sheaf $\Omega_X$.)


The sequence of commands we have used to construct the
cotangent sheaf can be obtained also with
the following built-in commands.
\beginOutput
i37 : Cubic = Proj(ringP3/idealCubic)\\
\emptyLine
o37 = Cubic\\
\emptyLine
o37 : ProjectiveVariety\\
\endOutput
\beginOutput
i38 : cotangentSheaf Cubic\\
\emptyLine
o38 = cokernel \{1\} | x_2  x_1  x_0  |\\
\               \{1\} | -x_3 -x_2 -x_1 |\\
\emptyLine
\                                                  2\\
o38 : coherent sheaf on Cubic, quotient of OO      (-1)\\
\                                             Cubic\\
\endOutput


Since $X$ is a smooth curve, its cotangent bundle is equal
to its {\it canonical bundle}, and also to its {\it dualizing sheaf} 
(see Hartshorne \cite[sections II.8 and III.7]{Hartshorne} 
{}for definitions).
\index{cotangent bundle}\index{canonical bundle}\index{dualizing sheaf}
We will
see another (generally more efficient) method of computing this
dualizing sheaf by using {\tt Ext} and \ie{duality} theory.

\goodbreak

\section{Intersections by Serre's Method}

\index{intersection theory}\index{Serre's intersection formula}%
To introduce homological algebra in a simple geometric context,
consider the problem of computing the \ie{intersection multiplicities}
of two varieties $X$ and $Y$ in $\P^n$, assuming for simplicity
that $\mathop{\rm dim} X +\mathop{\rm dim} Y = n$ and that the two
meet in a zero-dimensional scheme. Beginning in the
19th century, many people struggled to make a definition of
local intersection multiplicity
that would make {\it B\'ezout's Theorem\/} true: the product 
of the degrees of $X$ and $Y$ should be the number of points
of intersection, each counted with its local intersection multiplicity
(multiplied by the degree of the point, if the point is not
rational over the ground field).
\index{Bezout's Theorem@B\'ezout's Theorem}
In the simplest case, where the two varieties are Cohen-Macaulay,
the right answer is that a point $p$ should count with multiplicity
equal to the length of the local ring 
\index{length of a module}
$\cO_{X,p}\otimes_{\cO_{\P^n,p}} \cO_{Y,p}$,
and at first it was naively assumed that this would be the right answer
in general. 

Here is a famous example in ${\bf P}^4$ showing that
the naive value can be wrong: in it, the scheme $X$ is a 2-plane
and the scheme $Y=L_1\cup L_2$ is the union of two 2-planes. 
The planes $L_1$ and $L_2$ meet at just one point $p$,
and we assume that $X$ passes through $p$ as well, and
is general enough so that it meets
$Y$ only in $p$.
Since $\mathop{\rm degree}(X) = 1, \mathop{\rm degree}(Y) = 2$, 
B\'ezout's Theorem requires that
the multiplicity of the intersection at $p$ should be 2. 
However, we have:
\beginOutput
i39 : ringP4 = kk[x_0..x_4]\\
\emptyLine
o39 = ringP4\\
\emptyLine
o39 : PolynomialRing\\
\endOutput
\beginOutput
i40 : idealX = ideal(x_1+x_3, x_2+x_4)\\
\emptyLine
o40 = ideal (x  + x , x  + x )\\
\              1    3   2    4\\
\emptyLine
o40 : Ideal of ringP4\\
\endOutput
\beginOutput
i41 : idealL1 = ideal(x_1,x_2)\\
\emptyLine
o41 = ideal (x , x )\\
\              1   2\\
\emptyLine
o41 : Ideal of ringP4\\
\endOutput
\beginOutput
i42 : idealL2 = ideal(x_3,x_4)\\
\emptyLine
o42 = ideal (x , x )\\
\              3   4\\
\emptyLine
o42 : Ideal of ringP4\\
\endOutput
\beginOutput
i43 : idealY = intersect(idealL1,idealL2)\\
\emptyLine
o43 = ideal (x x , x x , x x , x x )\\
\              2 4   1 4   2 3   1 3\\
\emptyLine
o43 : Ideal of ringP4\\
\endOutput
\beginOutput
i44 : degree(idealX+idealY)\\
\emptyLine
o44 = 3\\
\endOutput
That is, 
the length of 
$\cO_{X,p}\otimes_{\cO_{\P^n,p}} \cO_{Y,p}$
is 3 rather than 2. (We can do this computation
without first passing to local rings because there
is only one point of intersection, and because all the
constructions we are using commute with localization.)

It was the happy discovery
of Jean-Pierre Serre \cite[V.B.3]{Serre} that the naive 
measure of intersection multiplicity can be fixed in a simple
way that works for all intersections in smooth varieties.
One simply replaces the length of the tensor product
$$
\cO_{X,p}\otimes_{\cO_{\P^n,p}} \cO_{Y,p}=
\mathop{\rm Tor}\nolimits_0^{\cO_{{\P^n},p} }
               (\cO_{X,p}, \cO_{Y,p})
$$
with the alternating sum of the Tor functors
$$
\sum_i(-1)^i \mathop{\rm length}
\mathop{\rm Tor}\nolimits_i^{\cO_{{\P^n},p} }
               (\cO_{X,p}, \cO_{Y,p}).
$$
In \Mtwo we can proceed as follows:
\indexcmd{Tor}
\beginOutput
i45 : degree Tor_0(ringP4^1/idealX, ringP4^1/idealY)\\
\emptyLine
o45 = 3\\
\endOutput
\beginOutput
i46 : degree Tor_1(ringP4^1/idealX, ringP4^1/idealY)\\
\emptyLine
o46 = 1\\
\endOutput
\beginOutput
i47 : degree Tor_2(ringP4^1/idealX, ringP4^1/idealY)\\
\emptyLine
o47 = 0\\
\endOutput
The other Tor's are 0 because the projective
dimension of {\tt ringP4\char`\^1/idealX} is only two,
as we see from
\beginOutput
i48 : res (ringP4^1/idealX)\\
\emptyLine
\            1           2           1\\
o48 = ringP4  <-- ringP4  <-- ringP4  <-- 0\\
\                                           \\
\      0           1           2           3\\
\emptyLine
o48 : ChainComplex\\
\endOutput
Thus, indeed, the alternating sum is 2, and B\'ezout's
Theorem is upheld.

\section{A Mystery Variety in $\P^3$}

In the file {\tt mystery.m2} is a function called {\tt mystery} that will
compute the ideal of a subvariety $X$ of $\P^3$.  We'll reveal what it does
at the end of the chapter.  Let's run it.
\beginOutput
i49 : ringP3 = kk[x_0..x_3];\\
\endOutput
\beginOutput
i50 : load "mystery.m2"\\
\endOutput
\beginOutput
i51 : idealX = mystery ringP3\\
\emptyLine
\              4       2      2        2 2   2 2           2            $\cdot\cdot\cdot$\\
o51 = ideal (x  - 2x x x  - x x x  + x x , x x  - 10915x x x  - 10917x $\cdot\cdot\cdot$\\
\              1     0 1 3    1 2 3    0 3   0 1         0 1 2          $\cdot\cdot\cdot$\\
\emptyLine
o51 : Ideal of ringP3\\
\endOutput
We can't see all the generators of the ideal; the same file contains a
function {\tt prettyPrint} which will display the generators visibly.
\beginOutput
i52 : prettyPrint gens idealX\\
x_1^4-2*x_0*x_1^2*x_3-x_1^2*x_2*x_3+x_0^2*x_3^2,\\
x_0^2*x_1^2-10915*x_0*x_1^2*x_2-10917*x_0^3*x_3+10916*x_0^2*x_2*x_3-\\
\   10916*x_0*x_2^2*x_3-10916*x_1*x_3^3,\\
x_0*x_1^2*x_2^2+11909*x_0^4*x_3+5954*x_0^3*x_2*x_3+2977*x_0^2*x_2^2*x_3+\\
\   11910*x_0*x_2^3*x_3-2978*x_1^3*x_3^2+14887*x_0*x_1*x_3^3+\\
\   11910*x_1*x_2*x_3^3,\\
x_0*x_1^3*x_2-13099*x_1^3*x_2^2-6550*x_0^3*x_1*x_3-\\
\   13100*x_0^2*x_1*x_2*x_3-6550*x_0*x_1*x_2^2*x_3+13099*x_1*x_2^3*x_3+\\
\   13100*x_1^2*x_3^3+13099*x_0*x_3^4,\\
x_0^5+5*x_0^2*x_2^3+5*x_0*x_2^4-3*x_0*x_1^3*x_3-4*x_1^3*x_2*x_3+\\
\   4*x_0^2*x_1*x_3^2+10*x_0*x_1*x_2*x_3^2+5*x_1*x_2^2*x_3^2,\\
x_1^2*x_2^4-8932*x_0^4*x_2*x_3+11909*x_0^3*x_2^2*x_3+5954*x_0^2*x_2^3*x_3-\\
\   8934*x_0*x_2^4*x_3-x_2^5*x_3+2*x_0*x_1^3*x_3^2-5952*x_1^3*x_2*x_3^2-\\
\   x_0^2*x_1*x_3^3-2979*x_0*x_1*x_2*x_3^3-8934*x_1*x_2^2*x_3^3+x_3^6\\
\endOutput
Imagine that you found yourself looking at the
scheme $X$ in $\P^3$ defined by the 6 equations above.
\beginOutput
i53 : X = variety idealX\\
\emptyLine
o53 = X\\
\emptyLine
o53 : ProjectiveVariety\\
\endOutput
How would you analyze the scheme $X$? 
We will illustrate one approach.

In outline, we will first look at the topological invariants:
\index{topology of a projective variety}
the number and dimensions of the irreducible components,
and how they meet if there is more than one; the topological
type of each component; and the degree of each component in
$\P^3$. We will then see what we can say about the analytic
invariants of $X$ using \ie{adjunction theory} (we give some references
at the end).

Since we are interested in the projective scheme defined by 
{\tt idealX} we could work with any ideal having the same
saturation. It is usually the case that working with the
saturation itself greatly eases subsequent computation so,
as a matter of good practice, 
we begin by checking whether the ideal is saturated. If
not, we should replace it with its saturation.
\indexcmd{saturate}\index{saturation}
\beginOutput
i54 : idealX == saturate idealX\\
\emptyLine
o54 = true\\
\endOutput
Thus we see that {\tt idealX} is already saturated.
Perhaps the most basic invariant of $X$ is its dimension:
\beginOutput
i55 : dim X\\
\emptyLine
o55 = 1\\
\endOutput
This shows that {\tt X} consists of a curve, and
possibly some zero-dimensional components.
The command
\beginOutput
i56 : idealXtop = top idealX\\
\emptyLine
\              4       2      2        2 2   2 2           2            $\cdot\cdot\cdot$\\
o56 = ideal (x  - 2x x x  - x x x  + x x , x x  - 10915x x x  - 10917x $\cdot\cdot\cdot$\\
\              1     0 1 3    1 2 3    0 3   0 1         0 1 2          $\cdot\cdot\cdot$\\
\emptyLine
o56 : Ideal of ringP3\\
\endOutput
\index{top dimensional part of an ideal}%
%% Dan the \index command above seems to insert an extra space
%% at the beginning of the line!
%%
%% The extra space comes from not putting a % at the end of the line.
returns the ideal of the largest dimensional components of {\tt X}.
If there were 0-dimensional components (or if idealX were not saturated)
then {\tt idealXtop} would be larger than {\tt idealX}.
To test this we reduce {\tt idealXtop} modulo {\tt idealX}
and see whether we get 0:
\beginOutput
i57 : (gens idealXtop){\char`\%}(gens idealX) == 0\\
\emptyLine
o57 = true\\
\endOutput
Thus
{\tt X}
is a purely one-dimensional scheme.


Is {\tt X}
singular?
\beginOutput
i58 : codim singularLocus idealX\\
\emptyLine
o58 = 4\\
\endOutput
%%Dan, Same spacing problem again, below!
\index{singular locus of a scheme}%
%% Same fix, above.
A variety of codimension 4 in $\P^3$ must be empty, so $X$
is a nonsingular curve. 

A nonsingular curve in $\P^3$ could still be reducible,
but since the intersection of two components
would be a singular point, the curve would then be disconnected.
A straightforward
way to decide is to use the command {\tt decompose},
which returns a list of irreducible components defined over {\tt kk}. 
The length
of this list,
\beginOutput
i59 : # decompose idealX\\
\emptyLine
o59 = 1\\
\endOutput
\indexcmd{decompose}\index{decompose a variety}%
\index{primary decomposition}%
\index{irreducible decomposition}%
is thus the number of irreducible components
that are defined over {\tt kk}, and
we see there is only one. (Warning: at this writing (December 2000), 
the command
``decompose'' works only in positive characteristic).  

Often what we really want to know is
whether {\tt X} is 
{\it absolutely irreducible\/} (that is, irreducible over the 
algebraic closure of {\tt kk}).
\index{irreducible!absolutely}
\index{absolutely irreducible}
The property of being smooth transfers to the algebraic
closure, so again the question is the number of connected
components we would get over the algebraic closure.
{}For any reduced scheme {\tt X} over a perfect
{}field (such as our finite field {\tt kk}) this number is
$\mathop{\rm h}\nolimits^0\cO_X := \mathop{\rm dim}\nolimits_{\tt kk}\mathop{\rm H}\nolimits^0\cO_X$.
We compute it with
\beginOutput
i60 : HH^0 OO_X\\
\emptyLine
\        1\\
o60 = kk\\
\emptyLine
o60 : kk-module, free\\
\endOutput
\beginOutput
i61 : rank oo\\
\emptyLine
o61 = 1\\
\endOutput
This command works much faster than the decompose command.
(You can compute the time by adding the command {\tt time}
\indexcmd{time}
to the beginning of the line where the command to be timed
starts.) Since we already know that {\tt idealX} is saturated,
this also shows that {\tt idealX} is prime.

We next ask for the genus of the curve $X$.
\index{genus of a curve}
Here the {\it genus} may be defined as the dimension of
the space $\operatorname{H}\nolimits^1\cO_X$. 
We can get this space with
\beginOutput
i62 : HH^1 OO_X\\
\emptyLine
\        6\\
o62 = kk\\
\emptyLine
o62 : kk-module, free\\
\endOutput
The genus of the curve is the dimension of this space,
which we can see to be 6.
Next, the cohomology class of {\tt X} in $\P^3$ is determined
by the degree of {\tt X}:
\beginOutput
i63 : degree idealX\\
\emptyLine
o63 = 10\\
\endOutput
In sum: {\tt X} is a smooth, absolutely irreducible curve of 
genus 6 and degree 10.

We next ask for
analytic information about the curve and the embedding.
A reasonable place to start is
with the relation between the line bundle defining the
embedding and the canonical sheaf $\omega_X$.
Notice first that the degree of the hyperplane divisor (the
degree of the curve) is 10 = 2g-2, the same as the canonical
bundle. By Riemann-Roch the embedding line bundle either is the canonical
bundle or has first cohomology 0, which we can check with
\beginOutput
i64 : P3 = Proj ringP3\\
\emptyLine
o64 = P3\\
\emptyLine
o64 : ProjectiveVariety\\
\endOutput
\beginOutput
i65 : HH^1((OO_P3(1)/idealX)(>=0))\\
\emptyLine
o65 = cokernel | x_3 x_2 x_1 x_0 |\\
\emptyLine
\                                       1\\
o65 : ringP3-module, quotient of ringP3\\
\endOutput
Let's examine the degree of the generator of that module.
\beginOutput
i66 : degrees oo\\
\emptyLine
o66 = \{\{0\}\}\\
\emptyLine
o66 : List\\
\endOutput
% Note the construction {\tt ringP3\char`\^\{1\}}, which denotes
% the free module of rank 1 corresponding to $\cO_{\P^3}(1)$.
% (In general, {\tt ringP3\char`\^\{a,b,\dots\}} denotes
% the free module over {\tt ringP3} corresponding to the sheaf
% $\cO_{\P^3}(a)\oplus \cO_{\P^3}(b)\oplus \cdots$ --- that is,
% the free module with generators in degrees $-a, -b, \dots$.)
From that and the presentation matrix above
%% $${\tt dualModule\char`\_\{0\}\ \char`\|\ x\char`\_3\ x\char`\_2\ x\char`\_1\ x\char`\_0\ \char`\|}$$
%%Dan, please check that the previous line will typeset correctly!
%% I've changed it a bit -- I don't understand the juxtaposition of the
%% expression and the matrix.  Also, dualModule hasn't been defined yet!
%% Why not just reserve \tt font for computer stuff, and try to typeset math here?
we see that this cohomology module is the residue class field
${\tt ringP3/(x_0,x_1,x_2,x_3)}$,
concentrated in degree 0.
Thus the embedding line bundle $\cO_X(1)$
is isomorphic to $\omega_X$. On the other hand
the dimension of the space of sections of this line bundle has already been
computed; it is $g = 6$. The curve is embedded in $\P^3$, so
only 4 of these sections were used---the embedding is a projection
of the same curve,  embedded in  $\P^6$ by the {\it canonical map\/}.
\index{canonical embedding}

We next ask more about the curve itself. After the genus,
the \ie{gonality} and the \ie{Clifford index} are
among the most interesting invariants.
Recall that the
{\it gonality\/} of $X$ is the smallest degree of a mapping from
$X$ to $\P^1$. To define the Clifford index
of $X$ 
we first define the {\it Clifford index of a line bundle\/} $L$
on $X$ to be
$\mathop{\rm degree}(L)-2(\mathop{\rm h}\nolimits^0(L)-1)$.
{}For example, the Clifford indices of the structure sheaf
$\cO_X$ and the canonical sheaf $\omega_X$ are both equal to 0.
The {\it Clifford index of the curve\/} 
$X$ is defined to be the minimum value
of the Clifford index of a line bundle $L$
on $X$ for which both
$\mathop{\rm h}\nolimits^0(L)\geq 2$ and $\mathop{\rm h}\nolimits^1(L)\geq 2$. 
The Clifford index of a curve of genus $g$
lies between 0 (for a \ie{hyperelliptic curve})
\index{curve!hyperelliptic} and
$\lfloor (g-1)/2\rfloor$ (for a general curve\index{curve!general}).
The Clifford index of any curve is bounded above
by the gonality minus 2.

{}For a curve of genus 6 such as $X$, the gonality is either
2 (the hyperelliptic case), 3 (the trigonal case) or 4
(the value for general curves). The Clifford index, on the
other hand is either 0 (the hyperelliptic case) or 1
(the case of a \ie{trigonal curve}
\index{curve!trigonal} OR a smooth 
\index{curve!plane quintic} plane quintic curve---which
is necessarily of gonality 4)
or 2 (the case of a general curve). Thus for most curves 
(and this is true in any genus)
the Clifford index is equal to the gonality minus 2.

We can make a start on distinguishing these cases already:
since our curve is embedded in $\P^3$ by a subseries of the
canonical series, $X$ cannot be hyperelliptic (for hyperelliptic
curves, the canonical series maps the curve two-to-one onto
a rational curve.)

To make further progress we use an idea of Mark Green
\index{Green's conjecture}
(see Green and Lazarsfeld \cite{gl}). Green conjectured a formula
{}for the Clifford index that depends only on
numerical data about the free resolution of the
curve in its complete canonical embedding (where the hyperplanes
cut out all the canonical divisors). 
The conjecture is known for genus 6 and
in many other cases; see for example Schreyer \cite{s}.

We therefore begin by 
computing the canonical embedding of $X$. We could proceed
to find the \ie{canonical bundle}
as in the computation for $\P^3$ above, or indeed
as $\cO_X(1)$, but instead we describe the general
method that is most efficient: duality, as described (for
example) in the book of Altman and Kleiman \cite{ak}. The 
module $\oplus_{d\in \Z}\mathop{\rm H}\nolimits^0(\omega_X(d))$ can be
computed as
\indexcmd{Ext}
\beginOutput
i67 : omegaX = Ext^(codim idealX)(ringP3^1/idealX, ringP3^\{-4\})\\
\emptyLine
o67 = cokernel \{0\}  | 9359x_3           -4677x_3         -10105x_1     $\cdot\cdot\cdot$\\
\               \{0\}  | 12014x_1          2552x_1          2626x_0       $\cdot\cdot\cdot$\\
\               \{-1\} | x_0x_3-2553x_2x_3 x_1^2-1702x_2x_3 x_0x_1-8086x_ $\cdot\cdot\cdot$\\
\emptyLine
\                                       3\\
o67 : ringP3-module, quotient of ringP3\\
\endOutput

To find the equations of
the \ie{canonical embedding} of $X$, we 
first compute a basis of ${\rm H}^0(\omega_X)$, which
is the degree 0 part of the module {\tt omegaX}.
The desired equations are computed as the algebraic
relations among the images of this basis under any
monomorphism $\omega_X \to \O_X$.

As the ring {\tt ringP3/idealX} is a domain,
and $\omega_X$ is the module corresponding to
a line bundle, any nonzero map
{}from  $\omega_X$ to {\tt ringP3/idealX}  will be
an embedding. We can compute the module of such maps with
\beginOutput
i68 : dualModule = Hom(omegaX, ringP3^1/idealX)\\
\emptyLine
o68 = subquotient (\{0\} | x_0^3x_2^2+10915x_0^2x_2^3+807x_0x_2^4+4043x_ $\cdot\cdot\cdot$\\
\                   \{0\} | 10105x_0x_1x_2^3+6063x_1x_2^4+11820x_0x_1^2x_ $\cdot\cdot\cdot$\\
\                   \{1\} | 10105x_0^2x_2^2-11322x_0x_2^3+11322x_2^4+8396 $\cdot\cdot\cdot$\\
\emptyLine
\                                          3\\
o68 : ringP3-module, subquotient of ringP3\\
\endOutput
and examine it with
\beginOutput
i69 : betti prune dualModule\\
\emptyLine
o69 = relations : total: 10 26\\
\                      3:  3  2\\
\                      4:  6 14\\
\                      5:  1  9\\
\                      6:  .  1\\
\endOutput
For want of a better idea
we take the first generator, {\tt dualModule\char`\_\char`\{0\char`\}}, which
we can turn into an actual homomorphism with
\beginOutput
i70 : f = homomorphism dualModule_\{0\}\\
\emptyLine
o70 = | x_0^3x_2^2+10915x_0^2x_2^3+807x_0x_2^4+4043x_2^5+7655x_0x_1x_2 $\cdot\cdot\cdot$\\
\emptyLine
o70 : Matrix\\
\endOutput
The image of a basis of $\omega_X$ is
given by the columns of the matrix
\beginOutput
i71 : canGens = f*basis(0,omegaX)\\
\emptyLine
o71 = | x_0^3x_2^2+10915x_0^2x_2^3+807x_0x_2^4+4043x_2^5+7655x_0x_1x_2 $\cdot\cdot\cdot$\\
\emptyLine
o71 : Matrix\\
\endOutput
regarded as elements of 
\beginOutput
i72 : ringX = ringP3/idealX\\
\emptyLine
o72 = ringX\\
\emptyLine
o72 : QuotientRing\\
\endOutput
Because of the particular homomorphism we chose,
they have degree 5.

We can now compute the defining ideal for $X$ in its canonical
embedding as the relations on these elements. We first define
a ring with 6 variables corresponding to the columns of {\tt canGens}
\beginOutput
i73 : ringP5 = kk[x_0..x_5]\\
\emptyLine
o73 = ringP5\\
\emptyLine
o73 : PolynomialRing\\
\endOutput
and then compute the canonical ideal as the
kernel of the corresponding map from
this ring to {\tt ringX} with
\indexcmd{trim}
\beginOutput
i74 : idealXcan = trim kernel map(ringX, ringP5, \\
\                                     substitute(matrix canGens,ringX),\\
\                                     DegreeMap => i -> 5*i)\\
\emptyLine
\              2                                                        $\cdot\cdot\cdot$\\
o74 = ideal (x  + 5040x x  - 8565x x  - 11589x x , x x  - 6048x x  - 1 $\cdot\cdot\cdot$\\
\              3        0 5        2 5         4 5   1 3        0 5     $\cdot\cdot\cdot$\\
\emptyLine
o74 : Ideal of ringP5\\
\endOutput
Here the command {\tt trim} is used to extract a minimal set
of generators of the desired ideal, and the command {\tt matrix}
replaces the map of (nonfree) modules {\tt canGens} by the matrix that
gives its action on the generators.  The {\tt DegreeMap} option specifies a
function which transforms degrees (represented as lists of integers) as the
ring homomorphism does; using it here makes the ring map homogeneous.

To get information about the Clifford index, we examine the 
{}free resolution with
\beginOutput
i75 : betti res idealXcan\\
\emptyLine
o75 = total: 1 9 16 9 1\\
\          0: 1 .  . . .\\
\          1: . 6  8 3 .\\
\          2: . 3  8 6 .\\
\          3: . .  . . 1\\
\endOutput
Quite generally, for a non-hyperelliptic curve of genus 
$g\geq 3$ the ideal
of the canonical embedding requires ${g-\binom 2 2}$ 
quadratic generators, in our case 6. It is known that
the curve is trigonal (Clifford index 1) if and only if the
ideal also requires cubic generators, that is, the
{}first term in the free resolution requires generators of degree
$3 = 1+2$; and Green's conjecture says in general that the
curve has Clifford index $c$ if the $c-1$ term in the
resolution does not require generators of degree
$(c-1)+2 = c+1$ but the $c$ term does require generators
of degree $c+2$. Thus from the Betti diagram above, and
the truth of Green's conjecture in low genus, we
see that our curve has Clifford index 1 and is thus either
trigonal or a plane quintic. 

If $X$ is trigonal, that is, $X$ has a map of degree 3 to $\P^1$,
then the fibers of this map form a linear series whose elements
are divisors of degree three. The geometric form
\index{Riemann-Roch theorem!geometric}
of the Riemann-Roch theorem says that if 
$$
p_1,\dots,p_d\in X\subset \P^g
$$
are points on a canonically embedded curve $X$,
then the dimension of the linear system in which the divisor
$p_1+\cdots+p_d$ moves is the amount by which the points fail
to be linearly independent: $d-1$ minus the dimension of the 
projective plane spanned by the points.
In particular, the 3 points in the fiber of a three-to-one
map to $\P^1$ are linearly dependent, that is,
they span a projective line.
This ``explains'' why the ideal of a
trigonal curve requires cubic generators: the quadrics all 
contain three points of these lines and thus contain the whole
lines! It is known (see St-Donat \cite{s-d}) that, in the 
trigonal case, the 6 quadrics in the ideal of the canonical curve
generate the defining ideal of the variety which is the union of these
lines, and that variety is a \ie{rational normal scroll}\index{scroll!rational normal}.
In case $X$ is a plane quintic, the {\it adjunction formula\/}
(Hartshorne \cite[II.8.20.3]{Hartshorne}) 
shows that the canonical embedding of $X$ is
obtained from the plane embedding by composing with the Veronese
embedding of the plane in $\P^5$ as the \ie{Veronese surface}; 
and the 6 quadrics in the ideal
of the canonical curve generate the defining ideal of the
Veronese surface. 

Thus if we let $S$ denote the variety defined by the quadrics
in the ideal of $X$, we can decide whether $X$ is a 
trigonal curve or a plane quintic by deciding whether $S$ is
a rational normal scroll or a Veronese surface.
To compute the ideal of $S$ we first ascertain which
of the generators of the ideal of the canonical curve
have degree 2 with
\indexcmd{positions}
\beginOutput
i76 : deg2places = positions(degrees idealXcan, i->i==\{2\})\\
\emptyLine
o76 = \{0, 1, 2, 3, 4, 5\}\\
\emptyLine
o76 : List\\
\endOutput
and then compute
\beginOutput
i77 : idealS= ideal (gens idealXcan)_deg2places\\
\emptyLine
\              2                                                        $\cdot\cdot\cdot$\\
o77 = ideal (x  + 5040x x  - 8565x x  - 11589x x , x x  - 6048x x  - 1 $\cdot\cdot\cdot$\\
\              3        0 5        2 5         4 5   1 3        0 5     $\cdot\cdot\cdot$\\
\emptyLine
o77 : Ideal of ringP5\\
\endOutput

One of the scrolls that could appear is singular, the cone
over the rational quartic in $\P^4$. We check for singularity
{}first:
\beginOutput
i78 : codim singularLocus idealS\\
\emptyLine
o78 = 6\\
\endOutput
Since the codimension is 6, the surface $S$ is nonsingular,
and thus must be one of the nonsingular scrolls or the Veronese
surface (which is by definition the image of ${\bf P}^2$, 
embedded in ${\bf P}^5$ by the linear series of conics.)

The ideals defining any rational normal
scroll of codimension 3, and the ideal
of a Veronese surface all have free resolutions with
the same Betti diagrams, so we need a subtler method
to determine the identity of $S$. The most powerful tool
{}for such purposes is adjunction theory; we will use a 
simple version. 

The idea is to compare the embedding bundle
(the ``hyperplane bundle'') with the canonical bundle.
On the Veronese surface,  the canonical bundle is 
the bundle associated to $-3$ lines in ${\bf P}^2$,
while the hyperplane bundle is associated to
2 lines in ${\bf P}^2$. Thus the inverse of the 
square of the canonical bundle is the cube of the 
hyperplane bundle, $\O_S(3)$. For a scroll on the other hand,
these two bundles are different.

As before we follow the
homological method for computing the canonical bundle:
\beginOutput
i79 : omegaS = Ext^(codim idealS)(ringP5^1/idealS, ringP5^\{-6\})\\
\emptyLine
o79 = cokernel \{2\} | 4032x_5  0       14811x_5 -4032x_3    6549x_3     $\cdot\cdot\cdot$\\
\               \{2\} | x_3      x_2     x_1      -x_4        x_0-14291x_ $\cdot\cdot\cdot$\\
\               \{2\} | -6852x_5 6549x_3 362x_5   x_1-6248x_3 0           $\cdot\cdot\cdot$\\
\emptyLine
\                                       3\\
o79 : ringP5-module, quotient of ringP5\\
\endOutput
\beginOutput
i80 : OS = ringP5^1/idealS\\
\emptyLine
o80 = cokernel | x_3^2+5040x_0x_5-8565x_2x_5-11589x_4x_5 x_1x_3-6048x_ $\cdot\cdot\cdot$\\
\emptyLine
\                                       1\\
o80 : ringP5-module, quotient of ringP5\\
\endOutput

We want the square of the canonical bundle, which we can compute
as the tensor square
\beginOutput
i81 : omegaS**omegaS\\
\emptyLine
o81 = cokernel \{4\} | 4032x_5  0       14811x_5 -4032x_3    6549x_3     $\cdot\cdot\cdot$\\
\               \{4\} | x_3      x_2     x_1      -x_4        x_0-14291x_ $\cdot\cdot\cdot$\\
\               \{4\} | -6852x_5 6549x_3 362x_5   x_1-6248x_3 0           $\cdot\cdot\cdot$\\
\               \{4\} | 0        0       0        0           0           $\cdot\cdot\cdot$\\
\               \{4\} | 0        0       0        0           0           $\cdot\cdot\cdot$\\
\               \{4\} | 0        0       0        0           0           $\cdot\cdot\cdot$\\
\               \{4\} | 0        0       0        0           0           $\cdot\cdot\cdot$\\
\               \{4\} | 0        0       0        0           0           $\cdot\cdot\cdot$\\
\               \{4\} | 0        0       0        0           0           $\cdot\cdot\cdot$\\
\emptyLine
\                                       9\\
o81 : ringP5-module, quotient of ringP5\\
\endOutput
But while this module represents the correct sheaf, it is
hard to interpret, since it may not be (is not, in this case)
the module of all twisted global sections of the square of the
line bundle. Since the free resolution of {\tt OS}
(visible inside the Betti diagram of the 
resolution of {\tt idealXcan}) has length 3,
the module {\tt OS}  has depth 2. Thus we can
{}find the module of all twisted global sections
of {\tt omega2S} by taking the double dual
\beginOutput
i82 : omega2S = Hom(Hom(omegaS**omegaS, OS),OS)\\
\emptyLine
o82 = cokernel \{3\} | x_3^2+5040x_0x_5-8565x_2x_5-11589x_4x_5 x_1x_3-60 $\cdot\cdot\cdot$\\
\emptyLine
\                                       1\\
o82 : ringP5-module, quotient of ringP5\\
\endOutput

We see from the output that this module is
generated by 1 element of degree 3.
It follows that
$\omega_S^2\cong \cO_S(-3)$. This in turn shows
that $S$ is the Veronese surface.

We now know that the canonical embedding of the curve
$X$ is the Veronese map applied to a planar embedding
of $X$ of degree 5, and we can ask to see the plane embedding.
Since the {\it anticanonical bundle\/} $\omega_S^{-1}$
on $S$ corresponds to 3 lines
in the plane and the hyperplane bundle to 2 lines,
we can recover the line bundle corresponding to 1 line,
giving
the isomorphism of $X$ to the plane, as the quotient
\beginOutput
i83 : L = Hom(omegaS, OS**(ringP5^\{-1\}))\\
\emptyLine
o83 = subquotient (\{-1\} | 14401x_2+16185x_4    x_0-14291x_4 -5359x_1+1 $\cdot\cdot\cdot$\\
\                   \{-1\} | -1488x_1-10598x_3    -6549x_3     -11789x_5  $\cdot\cdot\cdot$\\
\                   \{-1\} | x_0+7742x_2-15779x_4 x_2          x_1+6551x_ $\cdot\cdot\cdot$\\
\emptyLine
\                                          3\\
o83 : ringP5-module, subquotient of ringP5\\
\endOutput
and the line bundle on {\tt Xcan} that gives the
embedding in $\P^2$ will be the restriction of {\tt L}
to {\tt Xcan}. 
To realize the map from $X$ to $\P^2$, we proceed as before:
\beginOutput
i84 : dualModule = Hom(L, OS)\\
\emptyLine
o84 = subquotient (| x_0+7742x_2-15779x_4 14401x_2+16185x_4 x_1-301x_3 $\cdot\cdot\cdot$\\
\                   | x_2                  x_0-14291x_4      4032x_3    $\cdot\cdot\cdot$\\
\                   | x_1+6551x_3          -5359x_1+14409x_3 -9874x_5   $\cdot\cdot\cdot$\\
\emptyLine
\                                          3\\
o84 : ringP5-module, subquotient of ringP5\\
\endOutput
\beginOutput
i85 : betti generators dualModule\\
\emptyLine
o85 = total: 3 3\\
\          0: 3 3\\
\endOutput
Again, we may choose any homomorphism from {\tt L}
to {\tt OS}, for example
\beginOutput
i86 : g = homomorphism dualModule_\{0\}\\
\emptyLine
o86 = | x_0+7742x_2-15779x_4 x_2 x_1+6551x_3 |\\
\emptyLine
o86 : Matrix\\
\endOutput
\beginOutput
i87 : toP2 = g*basis(0,L)\\
\emptyLine
o87 = | x_0+7742x_2-15779x_4 x_2 x_1+6551x_3 |\\
\emptyLine
o87 : Matrix\\
\endOutput
\beginOutput
i88 : ringXcan = ringP5/idealXcan\\
\emptyLine
o88 = ringXcan\\
\emptyLine
o88 : QuotientRing\\
\endOutput
\beginOutput
i89 : ringP2 = kk[x_0..x_2]\\
\emptyLine
o89 = ringP2\\
\emptyLine
o89 : PolynomialRing\\
\endOutput
\beginOutput
i90 : idealXplane = trim kernel map(ringXcan, ringP2, \\
\                                        substitute(matrix toP2,ringXcan))\\
\emptyLine
\             5         4           3 2        2 3           4        5 $\cdot\cdot\cdot$\\
o90 = ideal(x  + 13394x x  - 13014x x  + 9232x x  + 12418x x  - 2746x  $\cdot\cdot\cdot$\\
\             0         0 1         0 1        0 1         0 1        1 $\cdot\cdot\cdot$\\
\emptyLine
o90 : Ideal of ringP2\\
\endOutput

We have effectively computed the square root of the line bundle
embedding $X$ in $\P^3$ with which we started, and exchanged a messy
set of defining equations of an unknown scheme for a single equation
defining a smooth plane curve whose properties are easy to deduce.
The  same curve may also be defined by a much simpler plane equation
(see Appendix \ref{How} below). I do not know any general method for choosing
a coordinate transformation to simplify a given equation! Can the
reader find one that will work at least in this case?

There is not yet a textbook-level exposition of the
sort of methods we have used
(although an introduction
will be contained in a forthcoming elementary book
of Decker and Schreyer). 
The reader who would like to go further into
such ideas can find a high-level survey of how adjunction theory
is used
in the paper of Decker and Schreyer \cite{ds}.
For a group of powerful methods with a different flavor,
see Aure, Decker, Hulek, Popescu, and
Ranestad \cite{adhpr}.


\appendix

\section{How the ``Mystery Variety'' was Made}\label{How}

{}For those who would like to try out the
computations above over a different field (perhaps the
{}field of rational numbers {\tt QQ}),
and for the curious, we include the code
used to produce the equations of the variety $X$ above.

Start with the Fermat quintic in the plane
\beginOutput
i91 : ringP2 = kk[x_0..x_2]\\
\emptyLine
o91 = ringP2\\
\emptyLine
o91 : PolynomialRing\\
\endOutput
\beginOutput
i92 : idealC2 = ideal(x_0^5+x_1^5+x_2^5)\\
\emptyLine
\             5    5    5\\
o92 = ideal(x  + x  + x )\\
\             0    1    2\\
\emptyLine
o92 : Ideal of ringP2\\
\endOutput
Embed it by the Veronese map in $\P^5$:
\beginOutput
i93 : ringC2 = ringP2/idealC2\\
\emptyLine
o93 = ringC2\\
\emptyLine
o93 : QuotientRing\\
\endOutput
\beginOutput
i94 : ringP5 = kk[x_0..x_5]\\
\emptyLine
o94 = ringP5\\
\emptyLine
o94 : PolynomialRing\\
\endOutput
\beginOutput
i95 : idealC5 = trim kernel map(ringC2, ringP5, \\
\              gens (ideal vars ringC2)^2)\\
\emptyLine
\              2                                    2                   $\cdot\cdot\cdot$\\
o95 = ideal (x  - x x , x x  - x x , x x  - x x , x  - x x , x x  - x  $\cdot\cdot\cdot$\\
\              4    3 5   2 4    1 5   2 3    1 4   2    0 5   1 2    0 $\cdot\cdot\cdot$\\
\emptyLine
o95 : Ideal of ringP5\\
\endOutput
{}Finally, choose a projection into $\P^3$, from a line not meeting
{\tt C5}, which is an isomorphism
onto its image. (This requires the image to be a smooth curve
of degree 10).
\beginOutput
i96 : ringC5 = ringP5/idealC5\\
\emptyLine
o96 = ringC5\\
\emptyLine
o96 : QuotientRing\\
\endOutput
\beginOutput
i97 : use ringC5\\
\emptyLine
o97 = ringC5\\
\emptyLine
o97 : QuotientRing\\
\endOutput
\beginOutput
i98 : idealC = trim kernel map(ringC5, ringP3,\\
\              matrix\{\{x_0+x_1,x_2,x_3,x_5\}\})\\
\emptyLine
\              4       2      2        2 2   2 2           2            $\cdot\cdot\cdot$\\
o98 = ideal (x  - 2x x x  - x x x  + x x , x x  - 10915x x x  - 10917x $\cdot\cdot\cdot$\\
\              1     0 1 3    1 2 3    0 3   0 1         0 1 2          $\cdot\cdot\cdot$\\
\emptyLine
o98 : Ideal of ringP3\\
\endOutput
Let's check that this is the same ideal as that of the mystery variety.
\beginOutput
i99 : idealC == idealX\\
\emptyLine
o99 = true\\
\endOutput
Here is the code of the function {\tt mystery}, which does the steps above.
\beginOutput
i100 : code mystery\\
\emptyLine
o100 = -- mystery.m2:1-13\\
\       mystery = ringP3 -> (\\
\          kk := coefficientRing ringP3;\\
\          x := local x;\\
\          ringP2 := kk[x_0..x_2];\\
\          idealC2 := ideal(x_0^5+x_1^5+x_2^5);\\
\          ringC2 := ringP2/idealC2;\\
\          ringP5 := kk[x_0..x_5];\\
\          idealC5 := trim kernel map(ringC2, ringP5, \\
\               gens (ideal vars ringC2)^2);\\
\          ringC5 := ringP5/idealC5;\\
\          use ringC5;\\
\          trim kernel map(ringC5, ringP3,\\
\             matrix\{\{x_0+x_1,x_2,x_3,x_5\}\}))\\
\endOutput
And here is the code of the function {\tt prettyPrint}.
\beginOutput
i101 : code prettyPrint\\
\emptyLine
o101 = -- mystery.m2:15-51\\
\       prettyPrint = f -> (\\
\          -- accept a matrix f and print its entries prettily,\\
\          -- separated by commas\\
\          wid := 74;\\
\          -- page width\\
\          post := (c,s) -> (\\
\             -- This function concatenates string c to end of each\\
\             -- string in list s except the last one\\
\             concatenate {\char`\\} pack_2 between_c s);\\
\          strings := post_"," (toString {\char`\\} flatten entries f);\\
\          -- list of strings, one for each polynomial, with commas\\
\          istate := ("",0);\\
\          -- initial state = (out : output string, col : column number)\\
\          strings = apply(\\
\             strings,\\
\             poly -> first fold(\\
\                -- break each poly into lines\\
\                (state,term) -> (\\
\                   (out,col) -> (\\
\                      if col + #term > wid -- too wide?\\
\                      then (\\
\                         out = out | "{\char`\\}n   "; \\
\                         col = 3;\\
\                         -- insert line break\\
\                         );\\
\                      (out | term, col + #term) -- new state\\
\                      )\\
\                   ) state,\\
\                istate,\\
\                fold( -- separate poly into terms \\
\                   \{"+","-"\},\\
\                   \{poly\},\\
\                   (delimiter,poly) -> flatten( \\
\                      post_delimiter {\char`\\} separate_delimiter {\char`\\} poly\\
\                      ))));\\
\          print stack strings;  -- stack them vertically, then print\\
\          )\\
\endOutput

% \end
\begin{thebibliography}{10}

\bibitem{ak}
Allen Altman and Steven Kleiman:
\newblock {\em Introduction to {G}rothendieck duality theory}.
\newblock Springer-Verlag, Berlin, 1970.
\newblock Lecture Notes in Mathematics, Vol. 146.

\bibitem{adhpr}
Alf Aure, Wolfram Decker, Klaus Hulek, Sorin Popescu, and Kristian Ranestad:
\newblock Syzygies of abelian and bielliptic surfaces in ${\bf {p}}\sp 4$.
\newblock {\em Internat. J. Math.}, 8(7):849--919, 1997.

\bibitem{ds}
Wolfram Decker and Frank-Olaf Schreyer:
\newblock Non-general type surfaces in ${\bf {p}}\sp 4$: some remarks on bounds
  and constructions.
\newblock {\em J. Symbolic Comput.}, 29(4-5):545--582, 2000.
\newblock Symbolic computation in algebra, analysis, and geometry (Berkeley,
  CA, 1998).

\bibitem{eCA}
David Eisenbud:
\newblock {\em Commutative algebra}.
\newblock Springer-Verlag, New York, 1995.
\newblock With a view toward algebraic geometry.

\bibitem{gl}
Mark Green and Robert Lazarsfeld:
\newblock On the projective normality of complete linear series on an algebraic
  curve.
\newblock {\em Invent. Math.}, 83(1):73--90, 1985.

\bibitem{Harris}
Joe Harris:
\newblock {\em Algebraic geometry}.
\newblock Springer-Verlag, New York, 1995.
\newblock A first course, Corrected reprint of the 1992 original.

\bibitem{Hartshorne}
Robin Hartshorne:
\newblock {\em Algebraic geometry}.
\newblock Springer-Verlag, New York, 1977.
\newblock Graduate Texts in Mathematics, No. 52.

\bibitem{s-d}
B.~Saint-Donat:
\newblock On {P}etri's analysis of the linear system of quadrics through a
  canonical curve.
\newblock {\em Math. Ann.}, 206:157--175, 1973.

\bibitem{s}
Frank-Olaf Schreyer:
\newblock Syzygies of canonical curves and special linear series.
\newblock {\em Math. Ann.}, 275(1):105--137, 1986.

\bibitem{s1}
Frank-Olaf Schreyer:
\newblock A standard basis approach to syzygies of canonical curves.
\newblock {\em J. Reine Angew. Math.}, 421:83--123, 1991.

\bibitem{Serre}
Jean-Pierre Serre:
\newblock {\em Alg\`ebre locale. {M}ultiplicit\'es}.
\newblock Springer-Verlag, Berlin, 1965.
\newblock Cours au Coll\`ege de France, 1957--1958, r\'edig\'e par Pierre
  Gabriel. Seconde \'edition, 1965. Lecture Notes in Mathematics, 11.

\end{thebibliography}
  \egroup
 \makeatletter
 \renewcommand\thesection{\@arabic\c@section}
 \makeatother


%%%%%%%%%%%%%%%%%%%%%%%%%%%%%%%%%%%%%%%%%%%%%%%%
%%%%%
%%%%% ../chapters/programming/chapter-m2.tex and ../chapters/programming/chapter-wrapper.bbl
%%%%%
%%%%%%%%%%%%%%%%%%%%%%%%%%%%%%%%%%%%%%%%%%%%%%%%

  \bgroup
\title{Data Types, Functions, and Programming}
\titlerunning{Data Types, Functions, and Programming}
\toctitle{Data Types, Functions, and Programming}
\author{Daniel R. Grayson%
        \thanks{Supported by NSF grant DMS 99-70085.}
        % \inst 1
   \and Michael E. Stillman%
        %\inst 2
        %\fnmsep
        \thanks{Supported by NSF grant 99-70348.}}
\authorrunning{D. R. Grayson and M. E. Stillman}
% \institute{University of Illinois at Urbana-Champaign, Department of
%   Mathematics, Urbana, IL 61801, USA
%       \and Cornell University, Department of Mathematics, Ithaca, NY 14853, USA}

\maketitle

\begin{abstract}
  In this chapter we present an introduction to the structure of \Mtwo
  commands and the writing of functions in the \Mtwo language.  For further details
  see the \Mtwo manual distributed with the program~\cite{M2}.
\end{abstract}

\section{Basic Data Types}

The basic \ie{data types} of \Mtwo include numbers of various types (integers,
rational numbers, floating point numbers, complex numbers), lists (basic
lists, and three types of visible lists, depending on the delimiter used),
hash tables, strings of characters (both 1-dimensional and 2-dimensional),
Boolean values (true and false), symbols, and functions.  Higher level types
useful in mathematics are derived from these basic types using facilities
provided in the \Mtwo language.  Except for the simplest types (integers and
Boolean values), \Mtwo normally displays the type of the output value on a
second labeled output line.

Symbols\index{symbols} have a name which consists of letters, digits, or apostrophes, the
first of which is a letter.  Values can be assigned to symbols and recalled
later.
\beginOutput
i1 : w\\
\emptyLine
o1 = w\\
\emptyLine
o1 : Symbol\\
\endOutput
\beginOutput
i2 : w = 2^100\\
\emptyLine
o2 = 1267650600228229401496703205376\\
\endOutput
\beginOutput
i3 : w\\
\emptyLine
o3 = 1267650600228229401496703205376\\
\endOutput
Multiple values can be assigned in parallel.
\beginOutput
i4 : (w,w') = (33,44)\\
\emptyLine
o4 = (33, 44)\\
\emptyLine
o4 : Sequence\\
\endOutput
\beginOutput
i5 : w\\
\emptyLine
o5 = 33\\
\endOutput
\beginOutput
i6 : w'\\
\emptyLine
o6 = 44\\
\endOutput
Comments are initiated by {\tt --} and extend to the end of the line.
\beginOutput
i7 : (w,w') = (33,   -- this is a comment\\
\               44)\\
\emptyLine
o7 = (33, 44)\\
\emptyLine
o7 : Sequence\\
\endOutput
Strings\index{strings} of characters are delimited by quotation marks.
\beginOutput
i8 : w = "abcdefghij"\\
\emptyLine
o8 = abcdefghij\\
\endOutput
They may be joined horizontally to make longer strings, or vertically to make
a two-dimensional version called a {\sl net}.
\beginOutput
i9 : w | w\\
\emptyLine
o9 = abcdefghijabcdefghij\\
\endOutput
\beginOutput
i10 : w || w\\
\emptyLine
o10 = abcdefghij\\
\      abcdefghij\\
\endOutput
Nets are used in the preparation of two dimensional output for polynomials.

Floating point numbers are distinguished from integers by the presence of a
decimal point, and rational numbers are entered as fractions.
\beginOutput
i11 : 2^100\\
\emptyLine
o11 = 1267650600228229401496703205376\\
\endOutput
\beginOutput
i12 : 2.^100\\
\emptyLine
o12 = 1.26765 10^30\\
\emptyLine
o12 : RR\\
\endOutput
\beginOutput
i13 : (36 + 1/8)^6\\
\emptyLine
\      582622237229761\\
o13 = ---------------\\
\           262144\\
\emptyLine
o13 : QQ\\
\endOutput
Parentheses, braces, and brackets are used as delimiters for the three types
of {\em \ie{visible lists}}: \ie{lists}, \ie{sequences}, and \ie{arrays}.
\beginOutput
i14 : x1 = \{1,a\}\\
\emptyLine
o14 = \{1, a\}\\
\emptyLine
o14 : List\\
\endOutput
\beginOutput
i15 : x2 = (2,b)\\
\emptyLine
o15 = (2, b)\\
\emptyLine
o15 : Sequence\\
\endOutput
\beginOutput
i16 : x3 = [3,c,d,e]\\
\emptyLine
o16 = [3, c, d, e]\\
\emptyLine
o16 : Array\\
\endOutput
Even though they use braces, lists should not be confused with sets, which will be treated later.
A double period can be used to construct a sequence of consecutive elements in various contexts.
\beginOutput
i17 : 1 .. 6\\
\emptyLine
o17 = (1, 2, 3, 4, 5, 6)\\
\emptyLine
o17 : Sequence\\
\endOutput
\beginOutput
i18 : a .. f\\
\emptyLine
o18 = (a, b, c, d, e, f)\\
\emptyLine
o18 : Sequence\\
\endOutput
Lists can be nested.
\beginOutput
i19 : xx = \{x1,x2,x3\}\\
\emptyLine
o19 = \{\{1, a\}, (2, b), [3, c, d, e]\}\\
\emptyLine
o19 : List\\
\endOutput
The number of entries in a list is provided by {\tt \#}.
\beginOutput
i20 : #xx\\
\emptyLine
o20 = 3\\
\endOutput
The entries in a list are numbered starting with 0, and can be recovered with
{\tt \#} used as a binary operator.
\beginOutput
i21 : xx#0\\
\emptyLine
o21 = \{1, a\}\\
\emptyLine
o21 : List\\
\endOutput
\beginOutput
i22 : xx#0#1\\
\emptyLine
o22 = a\\
\emptyLine
o22 : Symbol\\
\endOutput
We can join visible lists and {\sl append} or {\sl prepend} an element to a visible
list.  The output will be the same type of visible list that was provided in
the input: a list, a sequence, or an array; if the arguments are various
types of lists, the output will be same type as the first argument. 
\beginOutput
i23 : join(x1,x2,x3)\\
\emptyLine
o23 = \{1, a, 2, b, 3, c, d, e\}\\
\emptyLine
o23 : List\\
\endOutput
\index{lists!joining}\indexcmd{join}%
\beginOutput
i24 : append(x3,f)\\
\emptyLine
o24 = [3, c, d, e, f]\\
\emptyLine
o24 : Array\\
\endOutput
\index{lists!appending}\indexcmd{append}%
\beginOutput
i25 : prepend(f,x3)\\
\emptyLine
o25 = [f, 3, c, d, e]\\
\emptyLine
o25 : Array\\
\endOutput
\index{lists!prepending}\indexcmd{prepend}%
Use {\tt sum} or {\tt product} to produce the sum or product of all the
elements in a list.
\beginOutput
i26 : sum \{1,2,3,4\}\\
\emptyLine
o26 = 10\\
\endOutput
\indexcmd{sum}%
\beginOutput
i27 : product \{1,2,3,4\}\\
\emptyLine
o27 = 24\\
\endOutput
\indexcmd{product}%


\section{Control Structures}

Commands for later execution are encapsulated in {\sl \ie{functions}}.  A function
is created using the operator {\tt -\char`\>}\indexcmd{->} to separate the parameter or
sequence of parameters from the code to be executed later.  Let's try an
elementary example of a function with two arguments.
\beginOutput
i28 : f = (x,y) -> 1000 * x + y\\
\emptyLine
o28 = f\\
\emptyLine
o28 : Function\\
\endOutput
The parameters {\tt x} and {\tt y} are symbols that will acquire a
value later when the function is executed.  They are {\sl local}\index{variables!local} in the sense that
they are completely different from any symbols with the same name that occur
elsewhere.  Additional local variables for use within the body of a function
can be created by assigning a value to them with {\tt \char`\:\char`\=}
\indexcmd{:=}
(first time only).  We illustrate this by rewriting the function above.
\beginOutput
i29 : f = (x,y) -> (z := 1000 * x; z + y)\\
\emptyLine
o29 = f\\
\emptyLine
o29 : Function\\
\endOutput
Let's apply the function to some arguments.
\beginOutput
i30 : f(3,7)\\
\emptyLine
o30 = 3007\\
\endOutput
The sequence of arguments can be assembled first, and then passed to the
function.
\beginOutput
i31 : s = (3,7)\\
\emptyLine
o31 = (3, 7)\\
\emptyLine
o31 : Sequence\\
\endOutput
\beginOutput
i32 : f s\\
\emptyLine
o32 = 3007\\
\endOutput
As above, functions receiving one argument may be called without parentheses.
\beginOutput
i33 : sin 2.1\\
\emptyLine
o33 = 0.863209\\
\emptyLine
o33 : RR\\
\endOutput
A compact notation for functions makes it convenient to apply them
without naming them first.  For example, we may use \indexcmd{apply}{\tt apply} to apply a
function to every element of a list and to collect the results into a list.
\beginOutput
i34 : apply(1 .. 10, i -> i^3)\\
\emptyLine
o34 = (1, 8, 27, 64, 125, 216, 343, 512, 729, 1000)\\
\emptyLine
o34 : Sequence\\
\endOutput
The function \indexcmd{scan}{\tt scan} will do the same thing, but discard the results.
\beginOutput
i35 : scan(1 .. 5, print)\\
1\\
2\\
3\\
4\\
5\\
\endOutput
Use \indexcmd{if}{\tt if ... then ... else ...} to perform alternative actions
based on the truth of a condition.
\beginOutput
i36 : apply(1 .. 10, i -> if even i then 1000*i else i)\\
\emptyLine
o36 = (1, 2000, 3, 4000, 5, 6000, 7, 8000, 9, 10000)\\
\emptyLine
o36 : Sequence\\
\endOutput
A function can be terminated prematurely with \indexcmd{return}{\tt return}.
\beginOutput
i37 : apply(1 .. 10, i -> (if even i then return 1000*i; -i))\\
\emptyLine
o37 = (-1, 2000, -3, 4000, -5, 6000, -7, 8000, -9, 10000)\\
\emptyLine
o37 : Sequence\\
\endOutput
Loops in a program can be implemented with \indexcmd{while}{\tt while ... do ...}.
\beginOutput
i38 : i = 1; while i < 50 do (print i; i = 2*i)\\
1\\
2\\
4\\
8\\
16\\
32\\
\endOutput
Another way to implement loops is with \indexcmd{for}{\tt for} and
\indexcmd{do}{\tt do} or {\tt list}, with optional
clauses introduced by the keywords {\tt from}, {\tt to}, and {\tt when}.
\beginOutput
i40 : for i from 1 to 10 list i^3\\
\emptyLine
o40 = \{1, 8, 27, 64, 125, 216, 343, 512, 729, 1000\}\\
\emptyLine
o40 : List\\
\endOutput
\beginOutput
i41 : for i from 1 to 4 do print i\\
1\\
2\\
3\\
4\\
\endOutput
A loop can be terminated prematurely with \indexcmd{break}{\tt break}, which accepts an
optional value to return as the value of the loop expression.
\beginOutput
i42 : for i from 2 to 100 do if not isPrime i then break i\\
\emptyLine
o42 = 4\\
\endOutput
If no value needs to be returned, the condition for continuing can be
provided with the keyword {\tt when}; iteration continues only as long as the
predicate following the keyword returns {\tt true}.
\beginOutput
i43 : for i from 2 to 100 when isPrime i do print i\\
2\\
3\\
\endOutput


% if then else, while do, for, break, return,

\section{Input and Output}

The function \indexcmd{print}{\tt print} can be used to display something on the screen.
\beginOutput
i44 : print 2^100\\
1267650600228229401496703205376\\
\endOutput
For example, it could be used to display the elements of a list on separate
lines.
\beginOutput
i45 : (1 .. 5) / print;\\
1\\
2\\
3\\
4\\
5\\
\endOutput
The operator {\tt <<} can be used to display something on the
screen, without the newline character.
\beginOutput
i46 : << 2^100\\
1267650600228229401496703205376\\
o46 = stdio\\
\emptyLine
o46 : File\\
\emptyLine
\  --  the standard input output file\\
\endOutput
Notice the value returned is a {\em file}\index{files}.  A {\em file} in \Mtwo is a data type that
represents a channel through which data can be passed, as input, as
output, or in both directions.  The file \indexcmd{stdio}{\tt stdio} encountered above
corresponds to your shell window or terminal, and is used for two-way
communication between the program and the user.  A file may correspond
to what one usually calls a file, i.e., a sequence of data bytes associated
with a given name and stored on
your disk drive.  A file may also correspond to a {\em socket}, a
channel for communication with other programs over the network.

Files can be used with the binary form
of the operator {\tt \char`\<\char`\<} to display something else on the same
line.
\beginOutput
i47 : << "the value is : " << 2^100\\
the value is : 1267650600228229401496703205376\\
o47 = stdio\\
\emptyLine
o47 : File\\
\emptyLine
\  --  the standard input output file\\
\endOutput
Using \indexcmd{endl}{\tt endl} to represent the new line character or character sequence, we can
produce multiple lines of output.
\beginOutput
i48 : << "A = " << 2^100 << endl << "B = " << 2^200 << endl;\\
A = 1267650600228229401496703205376\\
B = 1606938044258990275541962092341162602522202993782792835301376\\
\endOutput
We can send the same output to a disk file named {\tt foo}, but we must remember to
close it with {\tt close}.
\beginOutput
i49 : "foo" << "A = " << 2^100 << endl << close\\
\emptyLine
o49 = foo\\
\emptyLine
o49 : File\\
\endOutput
The contents of the file can be recovered as a string with \indexcmd{get}{\tt get}.
\beginOutput
i50 : get "foo"\\
\emptyLine
o50 = A = 1267650600228229401496703205376\\
\emptyLine
\endOutput
If the file contains valid \Mtwo commands, as it does in this case, we can
execute those commands with \indexcmd{load}{\tt load}.
\beginOutput
i51 : load "foo"\\
\endOutput
We can verify that the command took effect by evaluating {\tt A}.
\beginOutput
i52 : A\\
\emptyLine
o52 = 1267650600228229401496703205376\\
\endOutput
Alternatively, if we want to see those commands and the output they produce,
we may use \indexcmd{input}{\tt input}.
\beginOutput
i53 : input "foo"\\
\emptyLine
i54 : A = 1267650600228229401496703205376\\
\emptyLine
o54 = 1267650600228229401496703205376\\
\emptyLine
i55 : \endOutput
Let's set up a ring for computation in \Mtwo.
\beginOutput
i56 : R = QQ[x,y,z]\\
\emptyLine
o56 = R\\
\emptyLine
o56 : PolynomialRing\\
\endOutput
\index{ring!making one}%
\beginOutput
i57 : f = (x+y)^3\\
\emptyLine
\       3     2        2    3\\
o57 = x  + 3x y + 3x*y  + y\\
\emptyLine
o57 : R\\
\endOutput
Printing, and printing to files, works for polynomials,
too.\index{printing!to a file}%
\beginOutput
i58 : "foo" << f << close;\\
\endOutput
The two-dimensional output is readable by humans, but is not easy to convert
back into a polynomial.
\beginOutput
i59 : get "foo"\\
\emptyLine
o59 =  3     2        2    3\\
\      x  + 3x y + 3x*y  + y\\
\endOutput
Use \indexcmd{toString}{\tt toString} to create a 1-dimensional form of the polynomial
that can be stored in a file in a format readable by \Mtwo and by other
symbolic algebra programs, such as {\em Mathematica} or {\em Maple}.
\beginOutput
i60 : toString f\\
\emptyLine
o60 = x^3+3*x^2*y+3*x*y^2+y^3\\
\endOutput
Send it to the file.
\beginOutput
i61 : "foo" << toString f << close;\\
\endOutput
Get it back.
\beginOutput
i62 : get "foo"\\
\emptyLine
o62 = x^3+3*x^2*y+3*x*y^2+y^3\\
\endOutput
Convert the string back to a polynomial with \indexcmd{value}{\tt value}, using \indexcmd{oo}{\tt oo} to
recover the value of the expression on the previous line.
\beginOutput
i63 : value oo\\
\emptyLine
\       3     2        2    3\\
o63 = x  + 3x y + 3x*y  + y\\
\emptyLine
o63 : R\\
\endOutput
The same thing works for matrices, and a little more detail is provided by
\indexcmd{toExternalString}{\tt toExternalString}, if needed.
\beginOutput
i64 : vars R\\
\emptyLine
o64 = | x y z |\\
\emptyLine
\              1       3\\
o64 : Matrix R  <--- R\\
\endOutput
\beginOutput
i65 : toString vars R\\
\emptyLine
o65 = matrix \{\{x, y, z\}\}\\
\endOutput
\beginOutput
i66 : toExternalString vars R\\
\emptyLine
o66 = map(R^\{\{0\}\}, R^\{\{-1\}, \{-1\}, \{-1\}\}, \{\{x, y, z\}\})\\
\endOutput

\section{Hash Tables}

Recall how one sets up a quotient ring for computation in \Mtwo.
\beginOutput
i67 : R = QQ[x,y,z]/(x^3-y)\\
\emptyLine
o67 = R\\
\emptyLine
o67 : QuotientRing\\
\endOutput
\beginOutput
i68 : (x+y)^4\\
\emptyLine
\        2 2       3    4           2\\
o68 = 6x y  + 4x*y  + y  + x*y + 4y\\
\emptyLine
o68 : R\\
\endOutput
How does \Mtwo represent a ring like $R$ in the computer?  To answer that,
first think about what sort of information needs to be retained about $R$.
We may need to remember the coefficient ring of $R$, the names of the
variables in $R$, the monoid of monomials in the variables, the degrees of
the variables, the characteristic of the ring, whether the ring is
commutative, the ideal modulo which we are working, and so on.  We also may
need to remember various bits of code: the code for performing the basic
arithmetic operations, such as addition and multiplication, on elements of
$R$; the code for preparing a readable representation of an element of $R$,
either 2-dimensional (with superscripts above the line and subscripts below),
or 1-dimensional.  Finally, we may want to remember certain things that take
a lot of time to compute, such as the Gr\"obner basis of the ideal.

A {\sl \ie{hash table}} is, by definition, a way of representing (in the computer)
a function whose domain is a finite set.  In \Mtwo, hash tables are extremely
flexible: the elements of the domain (or {\sl keys\index{keys of a hash table}}) and the elements of the
range (or {\sl values\index{values of a hash table}}) of the function may be any of the other objects
represented in the computer.  It's easy to come up with uses for functions
whose domain is finite: for example, a monomial can be represented by the
function that associates to a variable its nonzero exponent; a polynomial can
be represented by a function that associates to a monomial its nonzero
coefficient; a set can be represented by any function with that set as its
domain; a (sparse) matrix can be represented as a function from pairs of
natural numbers to the corresponding nonzero entry.

Let's create a hash table and name it.
\beginOutput
i69 : f = new HashTable from \{ a=>444, Daniel=>555, \{c,d\}=>\{1,2,3,4\}\}\\
\emptyLine
o69 = HashTable\{\{c, d\} => \{1, 2, 3, 4\}\}\\
\                a => 444\\
\                Daniel => 555\\
\emptyLine
o69 : HashTable\\
\endOutput
The operator \indexcmd{=>}{\tt \char`\=\char`\>} is used to represent a key-value pair.
We can use the operator \indexcmd{\#}{\tt \#} to recover the value from the key.
\beginOutput
i70 : f#Daniel\\
\emptyLine
o70 = 555\\
\endOutput
\beginOutput
i71 : f#\{c,d\}\\
\emptyLine
o71 = \{1, 2, 3, 4\}\\
\emptyLine
o71 : List\\
\endOutput
If the key is a symbol, we can use the operator \indexcmd{.}{\tt .} instead; this is
convenient if the symbol has a value that we want to ignore.
\beginOutput
i72 : Daniel = a\\
\emptyLine
o72 = a\\
\emptyLine
o72 : Symbol\\
\endOutput
\beginOutput
i73 : f.Daniel\\
\emptyLine
o73 = 555\\
\endOutput
We can use \indexcmd{\#?}{\tt \#?} to test whether a given key occurs in the hash table.
\beginOutput
i74 : f#?a\\
\emptyLine
o74 = true\\
\endOutput
\beginOutput
i75 : f#?c\\
\emptyLine
o75 = false\\
\endOutput
Finite sets are implemented in \Mtwo as hash tables: the elements of the set
are stored as the keys in the hash table, with the accompanying values all
being $1$.  (Multisets are implemented by using values larger than $1$, and
are called {\sl tallies}.)
\beginOutput
i76 : x = set\{1,a,\{4,5\},a\}\\
\emptyLine
o76 = Set \{\{4, 5\}, 1, a\}\\
\emptyLine
o76 : Set\\
\endOutput
\indexcmd{set}%
\beginOutput
i77 : x#?a\\
\emptyLine
o77 = true\\
\endOutput
\beginOutput
i78 : peek x\\
\emptyLine
o78 = Set\{\{4, 5\} => 1\}\\
\          1 => 1\\
\          a => 1\\
\endOutput
\beginOutput
i79 : y = tally\{1,a,\{4,5\},a\}\\
\emptyLine
o79 = Tally\{\{4, 5\} => 1\}\\
\            1 => 1\\
\            a => 2\\
\emptyLine
o79 : Tally\\
\endOutput
\beginOutput
i80 : y#a\\
\emptyLine
o80 = 2\\
\endOutput
We might use \indexcmd{tally}{\tt tally} to tally how often a function attains its various
possible values.  For example, how often does an integer have 3 prime
factors?  Or 4?  Use \indexcmd{factor}{\tt factor} to factor an integer.
\beginOutput
i81 : factor 60\\
\emptyLine
\       2\\
o81 = 2 3*5\\
\emptyLine
o81 : Product\\
\endOutput
Then use {\tt \#} to get the number of factors.
\beginOutput
i82 : # factor 60\\
\emptyLine
o82 = 3\\
\endOutput
Use {\tt apply} to list some values of the function.
\beginOutput
i83 : apply(2 .. 1000, i -> # factor i)\\
\emptyLine
o83 = (1, 1, 1, 1, 2, 1, 1, 1, 2, 1, 2, 1, 2, 2, 1, 1, 2, 1, 2, 2, 2,  $\cdot\cdot\cdot$\\
\emptyLine
o83 : Sequence\\
\endOutput
Finally, use {\tt tally} to summarize the results.
\beginOutput
i84 : tally oo\\
\emptyLine
o84 = Tally\{1 => 193\}\\
\            2 => 508\\
\            3 => 275\\
\            4 => 23\\
\emptyLine
o84 : Tally\\
\endOutput

Hash tables turn out to be convenient entities for storing odd bits and
pieces of information about something in a way that's easy to think about and
use.
In \Mtwo, rings are represented as hash tables, as are ideals, matrices,
modules, chain complexes, and so on.  For example, although it isn't a
documented feature, the key {\tt ideal} is used to preserve the ideal that
was used above to define the quotient ring {\tt R}, as part of the
information stored in {\tt R}.  
\beginOutput
i85 : R.ideal\\
\emptyLine
\             3\\
o85 = ideal(x  - y)\\
\emptyLine
o85 : Ideal of QQ [x, y, z]\\
\endOutput
The preferred and documented way for a user to recover this information is
with the function \indexcmd{ideal}{\tt ideal}. 
\beginOutput
i86 : ideal R\\
\emptyLine
\             3\\
o86 = ideal(x  - y)\\
\emptyLine
o86 : Ideal of QQ [x, y, z]\\
\endOutput
Users who want to introduce a new high-level mathematical
concept to \Mtwo may learn about hash tables by referring to the \Mtwo manual
\cite{M2}.

\section{Methods}

You may use the \texttt{code}\indexcmd{code} command to locate the source code for a given
function, at least if it is one of those functions written in the \Mtwo
language.  For example, here is the code for {\tt demark}, which may be used
to put commas between strings in a list.
\beginOutput
i87 : code demark\\
\emptyLine
o87 = -- ../../../m2/fold.m2:23\\
\      demark = (s,v) -> concatenate between(s,v)\\
\endOutput
The code for tensoring a ring map with a module can be displayed in this way.
\beginOutput
i88 : code(symbol **, RingMap, Module)\\
\emptyLine
o88 = -- ../../../m2/ringmap.m2:294-298\\
\      RingMap ** Module := Module => (f,M) -> (\\
\           R := source f;\\
\           S := target f;\\
\           if R =!= ring M then error "expected module over source ring";\\
\           cokernel f(presentation M));\\
\endOutput
\indexcmd{symbol}
The code implementing the {\tt ideal} function when applied to a
quotient ring can be displayed as follows.
\beginOutput
i89 : code(ideal, QuotientRing)\\
\emptyLine
o89 = -- ../../../m2/quotring.m2:7\\
\      ideal QuotientRing := R -> R.ideal\\
\endOutput
Notice that it uses the key {\tt ideal} to extract the information from the
ring's hash table, as you might have guessed from the previous discussion.
The bit of code displayed above may be called a {\sl method}\index{method} as
a way of indicating that several methods for dealing with various types of
arguments are attached to the function named {\tt ideal}.  New such {\sl
  method functions} may be created with the function {\tt method}\indexcmd{method}.  Let's
illustrate that with an example: we'll write a function called {\tt denom}
which should produce the denominator of a rational number.  When applied to
an integer, it should return 1.  First we create the method function.
\beginOutput
i90 : denom = method();\\
\endOutput
Then we tell it what to do with an argument from the class {\tt QQ} of rational numbers.
\beginOutput
i91 : denom QQ := x -> denominator x;\\
\endOutput
And also what to do with an argument from the class {\tt ZZ} of integers.
\beginOutput
i92 : denom ZZ := x -> 1;\\
\endOutput
Let's test it.
\beginOutput
i93 : denom(5/3)\\
\emptyLine
o93 = 3\\
\endOutput
\beginOutput
i94 : denom 5\\
\emptyLine
o94 = 1\\
\endOutput

\section{Pointers to the Source Code}

A substantial part of \Mtwo is written in the same language provided to the
users.  A good way to learn more about the \Mtwo language is to peruse the
source code that comes with the system in the directory {\tt Macaulay2/m2}.
Use the {\tt code} function, as described in the previous section, for
locating the bit of code you wish to view.

The source code for the interpreter of the \Mtwo language is in the directory
{\tt Macaulay2/d}.  It is written in another language designed to be mostly
type-safe, which is translated into {\tt C} by the translator whose own
{\tt C} source code is in the directory {\tt Macaulay2/c}.  Here is a
sample line of code from the file {\tt Macaulay2/d/tokens.d}, which shows how
the translator provides for allocation and initialization of dynamic data
structures.
\par
\vskip 5 pt
\begingroup
\tteight
\baselineskip=8pt
\lineskip=0pt
\obeyspaces
globalFrame := Frame(dummyFrame,globalScope.seqno,Sequence(nullE));\leavevmode\hss\endgraf
\endgroup
\penalty-1000
\par
\vskip 1 pt
\noindent
And here is the {\tt C} code produced by the translator.
\par
\vskip 5 pt
\begingroup
\tteight
\baselineskip=\outputBaseLineSkip
\lineskip=0pt
\obeyspaces
\obeylines
tokens\char`\_Frame tokens\char`\_globalFrame;
tokens\char`\_Frame tmp\char`\_\char`\_23;
Sequence tmp\char`\_\char`\_24;
tmp\char`\_\char`\_24 = (Sequence) GC\char`\_MALLOC(sizeof(struct S259\char`\_)+(1-1)*sizeof(Expr));
if (0 == tmp\char`\_\char`\_24) outofmem();
tmp\char`\_\char`\_24->len\char`\_ = 1;
tmp\char`\_\char`\_24->array\char`\_[0] = tokens\char`\_nullE;
tmp\char`\_\char`\_23 = (tokens\char`\_Frame) GC\char`\_MALLOC(sizeof(struct S260\char`\_));
if (0 == tmp\char`\_\char`\_23) outofmem();
tmp\char`\_\char`\_23->next = tokens\char`\_dummyFrame;
tmp\char`\_\char`\_23->scopenum = tokens\char`\_globalScope->seqno;
tmp\char`\_\char`\_23->values = tmp\char`\_\char`\_24;
tokens\char`\_globalFrame = tmp\char`\_\char`\_23;
\endgroup
\penalty-1000
\par
\vskip 1 pt

The core algebraic algorithms constitute the {\sl engine} of \Mtwo and are
written in {\tt C\char`\+\char`\+}, with the source files in the directory
{\tt Macaulay2/e}.  In the current version of the program, the interface
between the interpreter and the core algorithms consists of a single
two-directional stream of bytes.  The manual that comes with the system
\cite{M2} describes the engine communication protocol used in that
interface.

%% Mike says we don't need this:

%% One feature of \Mtwo which should be understood by system administrators, at
%% least of Unix systems, is the {\tt dumpdata} routine.  Since so much of the
%% source code is written in the interpreted \Mtwo language, and the task of
%% reading and parsing it can take an appreciable amount of time on slower
%% machines, we perform that task in advance (by the {\tt setup} command) when
%% the program is installed on your system.  The results are saved by writing
%% all of the data areas of the memory of the program to disk.  When a user runs
%% {\tt M2}, the file containing the saved contents is quickly mapped into
%% memory using the {\tt loaddata} command.  The contents of memory depend
%% in a detailed way on the architecture of the computer and the shareable
%% libraries that are installed, so when those things change, the {\tt setup}
%% command may need to be run again.  If those things change too often, the
%% system administrator may disable that feature.


%% Date: Tue, 21 Nov 2000 11:36:16 +0900 (JST)
%% From: Nobuki Takayama <taka@math.kobe-u.ac.jp>
%% To: dan@math.uiuc.edu
%% CC: mike@polygon.math.cornell.edu, taka@math.kobe-u.ac.jp
%% In-reply-to: <200011161511.JAA18405@orion.math.uiuc.edu> (dan@math.uiuc.edu)
%% Subject: Re: software
%% 
%% Dear Dan;
%% I would like to know 
%%   (1) "hints" to read and understand the M2 source code
%% and
%%   (2) what are new ideas in the M2 implementations.
%% The source code contains all ideas in detail, but it is not easy to
%% read and understand it without some hints.
%% 
%% I look forward to reading the section and the source code with the help
%% of the section.
%% Nobuki
%% 
%% >    If you want to add one subsection at the end, it could be something
%% >    like "Developers' notes" or "System Issues". The reader I have in mind
%% >
%% >    chapter is just the place. If you wish to consider this, then I suggest
%% >    that Mike communicate directly with Nobuki about the contents.
%% >
%% >Could you tell us directly whay you have in mind about that?
%% 
\begin{thebibliography}{1}

\bibitem{M2}
Daniel~R. Grayson and Michael~E. Stillman:
\newblock \textsl{Macaulay~2}, a software system for research in algebraic
  geometry and commutative algebra.
\newblock Available in source code form and compiled for various architectures,
  with documentation, at {http://www.math.uiuc.edu/Macaulay2/}.

\end{thebibliography}
  \egroup
 \makeatletter
 \renewcommand\thesection{\@arabic\c@section}
 \makeatother


%%%%%%%%%%%%%%%%%%%%%%%%%%%%%%%%%%%%%%%%%%%%%%%%
%%%%%
%%%%% ../chapters/schemes/chapter-m2.tex and ../chapters/schemes/chapter-wrapper.bbl
%%%%%
%%%%%%%%%%%%%%%%%%%%%%%%%%%%%%%%%%%%%%%%%%%%%%%%

  \bgroup
\title{Teaching the Geometry of Schemes}
\titlerunning{Teaching the Geometry of Schemes}
\toctitle{Teaching the Geometry of Schemes}

\author{Gregory G.~Smith \and Bernd Sturmfels}
\authorrunning{G. G. Smith and B. Sturmfels}
% \institute{Department of Mathematics, University of California,
% Berkeley, California 94720, USA}

\maketitle


%%----------------------------------------------------------
\newtheorem*{problem*}{Problem}{\bfseries\upshape}{\itshape}
\newtheorem*{solution*}{Solution}{\itshape}{\rmfamily}

\newcommand{\Spec}{\operatorname{Spec}}
\newcommand{\Proj}{\operatorname{Proj}}
\newcommand{\codim}{\operatorname{codim}}
%%----------------------------------------------------------


\begin{abstract}
This chapter presents a collection of graduate level problems in
algebraic geometry illustrating the power of \Mtwo as an educational
tool.
\end{abstract}

When teaching an advanced subject, like the language of schemes, we
think it is important to provide plenty of concrete instances of the
theory.  Computer algebra systems, such as \Mtwo, provide students
with an invaluable tool for studying complicated examples.
Furthermore, we believe that the explicit nature of a computational
approach leads to a better understanding of the objects being
examined.  This chapter presents some problems which we feel
illustrate this point of view.

Our examples are selected from the homework of an algebraic geometry
class given at the University of California at Berkeley in the fall of 1999.
This graduate course was taught by the second author with assistance from the
first author.  Our choice of problems, as the title suggests, follows the
material in David Eisenbud and Joe Harris' textbook {\em The Geometry of
  Schemes} \cite{SC:EH}.

%%----------------------------------------------------------
\section{Distinguished Open Sets}

We begin with a simple example involving the Zariski topology of an affine
scheme\index{scheme!affine}. This example also indicates some of the
subtleties involved in working with arithmetic
schemes\index{scheme!arithmetic}.

\begin{problem*}
Let $S = \bbbz[x,y,z]$ and $X = \Spec(S)$.  If $f = x$ and $X_{f}$ is
the corresponding basic open subset in $X$, then establish the
following:
\begin{enumerate}
\item[$(1)$] If $e_{1} = x+y+z$, $e_{2} = xy+xz+yz$ and $e_{3} = xyz$
are the elementary symmetric functions then the set $\{X_{e_{i}}\}_{1
\leq i \leq 3}$ is an open cover of $X_{f}$.
\item[$(2)$] If $p_{1} = x+y+z$, $p_{2} = x^{2}+y^{2}+z^{2}$ and $p_{3}
= x^{3}+y^{3}+z^{3}$ are the power sum symmetric functions then
$\{X_{p_{i}}\}_{1 \leq i \leq 3}$ is {\em not} an open cover of
$X_{f}$.
\end{enumerate}
\end{problem*}

\begin{solution*}
$(1)$ To prove that $\{X_{e_{i}}\}_{1 \leq i \leq 3}$ is an open cover
of $X_{f}$, it suffices to show that $e_{1}$, $e_{2}$ and $e_{3}$
generate the unit ideal in $S_{f}$; see Lemma I-16 in Eisenbud and
Harris~\cite{SC:EH}.  This is equivalent to showing that $x^{m}$
belongs to the $S$-ideal $\langle e_{1}, e_{2}, e_{3} \rangle$ for
some $m \in \bbbn$.  In other words, the saturation\index{saturation}
$\big( \langle e_{1}, e_{2}, e_{3} \rangle : x^{\infty} \big)$ is the
unit ideal if and only if $\{X_{e_{i}}\}_{1 \leq i \leq 3}$ is an open
cover of $X_{f}$.  We verify this in \Mtwo as follows:
\beginOutput
i1 : S = ZZ[x, y, z];\\
\endOutput
\beginOutput
i2 : elementaryBasis = ideal(x+y+z, x*y+x*z+y*z, x*y*z);\\
\emptyLine
o2 : Ideal of S\\
\endOutput
\beginOutput
i3 : saturate(elementaryBasis, x)\\
\emptyLine
o3 = ideal 1\\
\emptyLine
o3 : Ideal of S\\
\endOutput
$(2)$ Similarly, to show that $\{X_{p_{i}}\}_{1 \leq i \leq 3}$ is not
an open cover of $X_{f}$, we prove that $\big( \langle p_{1}, p_{2},
p_{3} \rangle : x^{\infty} \big)$ is not the unit ideal.  Calculating
this saturation, we find
\beginOutput
i4 : powerSumBasis = ideal(x+y+z, x^2+y^2+z^2, x^3+y^3+z^3);\\
\emptyLine
o4 : Ideal of S\\
\endOutput
\beginOutput
i5 : saturate(powerSumBasis, x)\\
\emptyLine
\                            2            2\\
o5 = ideal (6, x + y + z, 2y  + 2y*z + 2z , 3y*z)\\
\emptyLine
o5 : Ideal of S\\
\endOutput
\beginOutput
i6 : clearAll\\
\endOutput
which is not the unit ideal.\qed
\end{solution*}

The fact that $6$ is a generator of the ideal $\big( \langle p_{1},
p_{2}, p_{3} \rangle : x^{\infty} \big)$ indicates that
$\{X_{p_{i}}\}_{1 \leq i \leq 3}$ does not contain the points in $X$
lying over the points $\langle 2 \rangle$ and $\langle 3 \rangle$ in
$\Spec(\bbbz)$.  If we work over a base ring in which $6$ is a unit,
then $\{X_{p_{i}}\}_{1 \leq i \leq 3}$ would, in fact, be an open
cover of $X_{f}$.


%%----------------------------------------------------------
\section{Irreducibility}

The study of complex semisimple Lie algebras gives rise to an
important family of algebraic varieties called nilpotent
orbits\index{nilpotent orbits}.  The next problem examines the
irreducibility\index{scheme!irreducible} of a particular nilpotent
orbit.

\begin{problem*} 
Let $X$ be the set of nilpotent complex $3 \times 3$ matrices.  Show
that $X$ is an irreducible algebraic variety.
\end{problem*}

\begin{solution*}
A $3 \times 3$ matrix $M$ is nilpotent if and only if its minimal
polynomial $p(\sf T)$ equals ${\sf T}^{k}$, for some $k \in \bbbn$.
Since each irreducible factor of the characteristic polynomial of $M$
is also a factor of $p(\sf T)$, it follows that the characteristic
polynomial of $M$ is ${\sf T}^{3}$.  We conclude that the coefficients
of the characteristic polynomial of a generic $3 \times 3$ matrix
define the algebraic variety $X$.

To prove that $X$ is irreducible over $\bbbc$, we construct a rational
parameterization\index{rational parameterization}.  First, observe
that ${\rm GL}_{3}(\bbbc)$ acts on $X$ by conjugation.  Jordan's
canonical form theorem implies that there are exactly three orbits;
one for each of the following matrices:
\[
N_{(1,1,1)} =\left[ \begin{smallmatrix} 0 & 0 & 0 \\ 0 & 0 & 0 \\ 0 &
0 & 0 \end{smallmatrix} \right], \quad
N_{(2,1)} = \left[ \begin{smallmatrix} 0 & 1 & 0 \\ 0 & 0 & 0 \\ 0 & 0
& 0 \end{smallmatrix} \right] \text{ and }
N_{(3)} = \left[ \begin{smallmatrix} 0 & 1 & 0 \\ 0 & 0 & 1 \\ 0 & 0 &
0 \end{smallmatrix} \right] \enspace .
\]
Each orbit is defined by a rational parameterization, so it suffices
to show that the closure of the orbit containing $N_{(3)}$ is the
entire variety $X$.  We demonstrate this as follows:
\beginOutput
i7 : S = QQ[t, y_0 .. y_8, a..i, MonomialOrder => Eliminate 10];\\
\endOutput
\beginOutput
i8 : N3 = (matrix \{\{0,1,0\},\{0,0,1\},\{0,0,0\}\}) ** S\\
\emptyLine
o8 = | 0 1 0 |\\
\     | 0 0 1 |\\
\     | 0 0 0 |\\
\emptyLine
\             3       3\\
o8 : Matrix S  <--- S\\
\endOutput
\beginOutput
i9 : G = genericMatrix(S, y_0, 3, 3)\\
\emptyLine
o9 = | y_0 y_3 y_6 |\\
\     | y_1 y_4 y_7 |\\
\     | y_2 y_5 y_8 |\\
\emptyLine
\             3       3\\
o9 : Matrix S  <--- S\\
\endOutput
To determine the entries in $G \cdot N_{(3)} \cdot G^{-1}$, we use the
classical adjoint\index{classical adjoint} to construct the matrix
$\det(G) \cdot G^{-1}$.
\beginOutput
i10 : classicalAdjoint = (G) -> (\\
\           n := degree target G;\\
\           m := degree source G;\\
\           matrix table(n, n, (i, j) -> (-1)^(i+j) * det(\\
\                     submatrix(G, \{0..j-1, j+1..n-1\}, \\
\                          \{0..i-1, i+1..m-1\}))));\\
\endOutput
\beginOutput
i11 : num = G * N3 * classicalAdjoint(G);\\
\emptyLine
\              3       3\\
o11 : Matrix S  <--- S\\
\endOutput
\beginOutput
i12 : D = det(G);\\
\endOutput
\beginOutput
i13 : M = genericMatrix(S, a, 3, 3);\\
\emptyLine
\              3       3\\
o13 : Matrix S  <--- S\\
\endOutput
The entries in $G \cdot N_{(3)} \cdot G^{-1}$ give a rational
parameterization of the orbit generated by $N_{(3)}$.  Using
elimination theory\index{elimination theory} --- see section~3.3 in
Cox, Little and O`Shea~\cite{SC:CLO} --- we give an ``implicit
representation'' of this variety.
\beginOutput
i14 : elimIdeal = minors(1, (D*id_(S^3))*M - num) + ideal(1-D*t);\\
\emptyLine
o14 : Ideal of S\\
\endOutput
\beginOutput
i15 : closureOfOrbit = ideal selectInSubring(1, gens gb elimIdeal);\\
\emptyLine
o15 : Ideal of S\\
\endOutput

Finally, we verify that this orbit closure equals $X$
scheme-theoretically.  Recall that $X$ is defined by the coefficients
of the characteristic polynomial of a generic $3 \times 3$ matrix {\tt
M}.
%% was X = ideal submatrix( (coeff-icients({0}, det(M - t*id_(S^3))))_1,
%% {1,2,3} ), but 'coeff-icients' is to be redesigned, and 'contract' is
%% more self-explanatory, anyway.
\beginOutput
i16 : X = ideal substitute(\\
\              contract(matrix\{\{t^2,t,1\}\}, det(t-M)),\\
\              \{t => 0_S\})\\
\emptyLine
o16 = ideal (- a - e - i, - b*d + a*e - c*g - f*h + a*i + e*i, c*e*g - $\cdot\cdot\cdot$\\
\emptyLine
o16 : Ideal of S\\
\endOutput
\beginOutput
i17 : closureOfOrbit == X\\
\emptyLine
o17 = true\\
\endOutput
\beginOutput
i18 : clearAll\\
\endOutput
This completes our solution.\qed
\end{solution*}

More generally, Kostant shows that the set of all nilpotent elements
in a complex semisimple Lie algebra\index{Lie algebra} form an
irreducible variety.  We refer the reader to Chriss and
Ginzburg~\cite{SC:CV} for a proof of this result (Corollary~3.2.8) and
a discussion of its applications in representation theory.


%%----------------------------------------------------------
\section{Singular Points}

In our third question, we study the singular locus\index{singular
locus} of a family of elliptic curves\index{elliptic curve}.

\begin{problem*} 
Consider a general form of degree $3$ in $\bbbq[x,y,z]$:
\[
F = ax^{3} + bx^{2}y + cx^{2}z + dxy^{2} + exyz + fxz^{2} + gy^{3} +
hy^{2}z + iyz^{2} + jz^{3} \enspace .
\]
Give necessary and sufficient conditions in terms of $a, \ldots, j$
for the cubic curve $\Proj\big( \bbbq[x,y,z] / \langle F \rangle
\big)$ to have a singular point.
\end{problem*}

\begin{solution*}
The singular locus of $F$ is defined by a polynomial of degree $12$ in
the $10$ variables $a, \dotsc, j$.  We calculate this polynomial in two
different ways.

Our first method is an elementary but time consuming elimination.
Carrying it out in \Mtwo, we have
\beginOutput
i19 : S = QQ[x, y, z, a..j, MonomialOrder => Eliminate 2];\\
\endOutput
\beginOutput
i20 : F = a*x^3+b*x^2*y+c*x^2*z+d*x*y^2+e*x*y*z+f*x*z^2+g*y^3+h*y^2*z+\\
\                   i*y*z^2+j*z^3;\\
\endOutput
\beginOutput
i21 : partials = submatrix(jacobian matrix\{\{F\}\}, \{0..2\}, \{0\})\\
\emptyLine
o21 = \{1\} | 3x2a+2xyb+y2d+2xzc+yze+z2f |\\
\      \{1\} | x2b+2xyd+3y2g+xze+2yzh+z2i |\\
\      \{1\} | x2c+xye+y2h+2xzf+2yzi+3z2j |\\
\emptyLine
\              3       1\\
o21 : Matrix S  <--- S\\
\endOutput
\beginOutput
i22 : singularities = ideal(partials) + ideal(F);\\
\emptyLine
o22 : Ideal of S\\
\endOutput
\beginOutput
i23 : elimDiscr = time ideal selectInSubring(1,gens gb singularities);\\
\     -- used 64.27 seconds\\
\emptyLine
o23 : Ideal of S\\
\endOutput
\beginOutput
i24 : elimDiscr = substitute(elimDiscr, \{z => 1\});\\
\emptyLine
o24 : Ideal of S\\
\endOutput
On the other hand, there is also an elegant and more useful
determinantal formula for this discriminant\index{discriminant}; it is
a specialization of the formula (2.8) in section~3.2 of Cox, Little
and O`Shea~\cite{SC:CLO2}.  To apply this determinantal formula, we
first create the coefficient matrix {\tt A} of the partial derivatives
of $F$.
%% was A = (coeff-icients({0,1,2}, submatrix(jacobian matrix{{F}}, {0..2}, {0})))_1;
%% but 'coeff-icients' is deprecated.
\beginOutput
i25 : A = contract(matrix\{\{x^2,x*y,y^2,x*z,y*z,z^2\}\},\\
\              diff(transpose matrix\{\{x,y,z\}\},F))\\
\emptyLine
o25 = \{1\} | 3a 2b d  2c e  f  |\\
\      \{1\} | b  2d 3g e  2h i  |\\
\      \{1\} | c  e  h  2f 2i 3j |\\
\emptyLine
\              3       6\\
o25 : Matrix S  <--- S\\
\endOutput
We also construct the coefficient matrix {\tt B} of the partial
derivatives of the Hessian\index{hessian} of $F$.
\beginOutput
i26 : hess = det submatrix(jacobian ideal partials, \{0..2\}, \{0..2\});\\
\endOutput
%% was B = (coeff-icients({0,1,2}, submatrix(jacobian matrix{{hess}}, {0..2}, {0})))_1;
%% but 'coeff-icients' is deprecated.
\beginOutput
i27 : B = contract(matrix\{\{x^2,x*y,y^2,x*z,y*z,z^2\}\},\\
\              diff(transpose matrix\{\{x,y,z\}\},hess))\\
\emptyLine
o27 = \{1\} | -24c2d+24bce-18ae2-24b2f+72adf               4be2-16bdf-48 $\cdot\cdot\cdot$\\
\      \{1\} | 2be2-8bdf-24c2g+72afg+16bch-24aeh-8b2i+24adi 4de2-16d2f-48 $\cdot\cdot\cdot$\\
\      \{1\} | 2ce2-8cdf-8c2h+24afh+16bci-24aei-24b2j+72adj 2e3-8def-24cf $\cdot\cdot\cdot$\\
\emptyLine
\              3       6\\
o27 : Matrix S  <--- S\\
\endOutput
To obtain the discriminant, we combine these two matrices and take the
determinant.
\beginOutput
i28 : detDiscr = ideal det (A || B);\\
\emptyLine
o28 : Ideal of S\\
\endOutput
Finally, we check that our two discriminants are equal
\beginOutput
i29 : detDiscr == elimDiscr\\
\emptyLine
o29 = true\\
\endOutput
and examine the generator.
\beginOutput
i30 : detDiscr_0\\
\emptyLine
\            2   4 3 2             5 3 2           6 3 2          2 2 2 $\cdot\cdot\cdot$\\
o30 = 13824c d*e f g  - 13824b*c*e f g  + 13824a*e f g  - 110592c d e  $\cdot\cdot\cdot$\\
\emptyLine
o30 : S\\
\endOutput
\beginOutput
i31 : numgens detDiscr\\
\emptyLine
o31 = 1\\
\endOutput
\beginOutput
i32 : # terms detDiscr_0\\
\emptyLine
o32 = 2040\\
\endOutput
\beginOutput
i33 : clearAll\\
\endOutput
Hence, the singular locus is given by a single polynomial of degree
$12$ with $2040$ terms.\qed
\end{solution*}

For a further discussion of singularities and discriminants see
Section~V.3 in Eisenbud and Harris~\cite{SC:EH}.  For information on
resultants and discriminants see Chapter~2 in Cox, Little and
O`Shea~\cite{SC:CLO2}.


%%----------------------------------------------------------
\section{Fields of Definition}

Schemes\index{scheme!over a number field} over non-algebraically
closed fields arise in number theory.  Our fourth problem looks at one
technique for working with number fields in \Mtwo.

\begin{problem*}[Exercise~II-6 in  \cite{SC:EH}]
An inclusion of fields $K \hookrightarrow L$ induces a map
$\mathbb{A}_{L}^{n} \to \mathbb{A}_{K}^{n}$.  Find the images in
$\mathbb{A}_{\bbbq}^{2}$ of the following points of
$\mathbb{A}_{\overline{\bbbq}}^{2}$ under this map.
\begin{enumerate}
\item[$(1)$] $\langle x - \sqrt{2}, y - \sqrt{2} \rangle ;$
\item[$(2)$] $\langle x - \sqrt{2}, y - \sqrt{3} \rangle ;$
\item[$(3)$] $\langle x - \zeta, y - \zeta^{-1} \rangle$ where $\zeta$
is a $5$-th root of unity $;$
\item[$(4)$] $\langle \sqrt{2}x- \sqrt{3}y \rangle ;$
\item[$(5)$] $\langle \sqrt{2}x- \sqrt{3}y-1 \rangle$.
\end{enumerate}
\end{problem*}

\begin{solution*}
The images can be determined by using the following three step
algorithm: (1) replace the coefficients not contained in $K$ with
indeterminates, (2) add the minimal polynomials of these coefficients
to the given ideal in $\mathbb{A}_{L}^{2}$, and (3) eliminate the new
indeterminates.  Here are the five examples:
\beginOutput
i34 : S = QQ[a,b,x,y, MonomialOrder => Eliminate 2];\\
\endOutput
\beginOutput
i35 : I1 = ideal(x-a, y-a, a^2-2);\\
\emptyLine
o35 : Ideal of S\\
\endOutput
\beginOutput
i36 : ideal selectInSubring(1, gens gb I1)\\
\emptyLine
\                     2\\
o36 = ideal (x - y, y  - 2)\\
\emptyLine
o36 : Ideal of S\\
\endOutput
\beginOutput
i37 : I2 = ideal(x-a, y-b, a^2-2, b^2-3);\\
\emptyLine
o37 : Ideal of S\\
\endOutput
\beginOutput
i38 : ideal selectInSubring(1, gens gb I2)\\
\emptyLine
\              2       2\\
o38 = ideal (y  - 3, x  - 2)\\
\emptyLine
o38 : Ideal of S\\
\endOutput
\beginOutput
i39 : I3 = ideal(x-a, y-a^4, a^4+a^3+a^2+a+1);\\
\emptyLine
o39 : Ideal of S\\
\endOutput
\beginOutput
i40 : ideal selectInSubring(1, gens gb I3)\\
\emptyLine
\                       2    2               3    2\\
o40 = ideal (x*y - 1, x  + y  + x + y + 1, y  + y  + x + y + 1)\\
\emptyLine
o40 : Ideal of S\\
\endOutput
\beginOutput
i41 : I4 = ideal(a*x+b*y, a^2-2, b^2-3);\\
\emptyLine
o41 : Ideal of S\\
\endOutput
\beginOutput
i42 : ideal selectInSubring(1, gens gb I4)\\
\emptyLine
\             2   3  2\\
o42 = ideal(x  - -*y )\\
\                 2\\
\emptyLine
o42 : Ideal of S\\
\endOutput
\beginOutput
i43 : I5 = ideal(a*x+b*y-1, a^2-2, b^2-3);\\
\emptyLine
o43 : Ideal of S\\
\endOutput
\beginOutput
i44 : ideal selectInSubring(1, gens gb I5)\\
\emptyLine
\             4     2 2   9  4    2   3  2   1\\
o44 = ideal(x  - 3x y  + -*y  - x  - -*y  + -)\\
\                         4           2      4\\
\emptyLine
o44 : Ideal of S\\
\endOutput
\beginOutput
i45 : clearAll\\
\endOutput
\qed
\end{solution*}

It is worth noting that the points in $\mathbb{A}_{\bbbq}^{n}$ correspond
to orbits of the action of ${\rm Gal}(\overline{\bbbq}/\bbbq)$ on the
points of $\mathbb{A}_{\overline{\bbbq}}^{n}$.  For more examples and
information, see section~II.2 in Eisenbud and Harris~\cite{SC:EH}.


%%----------------------------------------------------------
\section{Multiplicity}

The multiplicity\index{multiplicity} of a zero-dimensional scheme $X$
at a point $p \in X$ is defined to be the length of the local ring
$\mathcal{O}_{X,p}$.  Unfortunately, we cannot work directly in the
local ring in \Mtwo.  What we can do, however, is to compute the
multiplicity by computing the degree of the component of $X$ supported
at $p$; see page 66 in Eisenbud and Harris~\cite{SC:EH}.

\begin{problem*}
What is the multiplicity of the origin as a zero of the polynomial
equations $x^{5}+y^{3}+z^{3} = x^{3}+y^{5}+z^{3} = x^{3}+y^{3}+z^{5} =
0$?
\end{problem*}

\begin{solution*}
If $I$ is the ideal generated by $x^{5}+y^{3}+z^{3}$,
$x^{3}+y^{5}+z^{3}$ and $x^{3}+y^{3}+z^{5}$ in $\bbbq[x,y,z]$, then
the multiplicity of the origin is
\[
\dim_{\bbbq} \frac{\bbbq[x,y,z]_{\langle x,y,z \rangle}}
{I \bbbq[x,y,z]_{\langle x,y,z \rangle}} \, .
\]
It follows that the multiplicity is the vector space dimension of the
ring $\bbbq[x,y,z] / \varphi^{-1}(I \bbbq[x,y,z]_{\langle x,y,z
\rangle})$ where $\varphi \colon \bbbq[x,y,z] \to
\bbbq[x,y,z]_{\langle x,y,z \rangle}$ is the natural map.  Moreover,
we can express this using ideal quotients:
\[
\varphi^{-1}(I \bbbq[x,y,z]_{\langle x,y,z \rangle}) \,\,= \,\,
\big(I : (I : \langle x,y,z \rangle^{\infty})\big) \, .
\]
Carrying out this calculation in \Mtwo, we obtain:
\beginOutput
i46 : S = QQ[x, y, z];\\
\endOutput
\beginOutput
i47 : I = ideal(x^5+y^3+z^3, x^3+y^5+z^3, x^3+y^3+z^5);\\
\emptyLine
o47 : Ideal of S\\
\endOutput
\beginOutput
i48 : multiplicity = degree(I : saturate(I))\\
\emptyLine
o48 = 27\\
\endOutput
\beginOutput
i49 : clearAll\\
\endOutput
Thus, we conclude that the multiplicity is $27$.\qed
\end{solution*}

There are algorithms (not yet implemented in \Mtwo) for working
directly in the local ring $\bbbq[x,y,z]_{\langle x,y,z \rangle}$.  We
refer the interested reader to Chapter~4 in Cox, Little and
O`Shea~\cite{SC:CLO2}.


%%----------------------------------------------------------
\section{Flat Families}

Non-reduced schemes\index{scheme!non-reduced} arise naturally as flat
limits\index{flat limit} of a family of reduced
schemes\index{scheme!reduced}. Our next problem illustrates how a
family of skew lines in $\bbbp^{3}$ gives rise to a double line with
an embedded point\index{embedded point}.

\begin{problem*}[Exercise~III-68 in \cite{SC:EH}]
Let $L$ and $M $ be the lines in $\bbbp^{3}_{k[t]}$ given by $x=y=0$
and $x-tz = y+t^{2}w =0$ respectively.  Show that the flat limit as $t
\to 0$ of the union $L \cup M$ is the double line $x^{2} = y = 0$ with
an embedded point of degree $1$ located at the point $(0:0:0:1)$.
\end{problem*}

\begin{solution*}
We first find the flat limit by saturating\index{saturation} the
intersection ideal and setting $t = 0$.
\beginOutput
i50 : PP3 = QQ[t, x, y, z, w];\\
\endOutput
\beginOutput
i51 : L = ideal(x, y);\\
\emptyLine
o51 : Ideal of PP3\\
\endOutput
\beginOutput
i52 : M = ideal(x-t*z, y+t^2*w);\\
\emptyLine
o52 : Ideal of PP3\\
\endOutput
\beginOutput
i53 : X = intersect(L, M);\\
\emptyLine
o53 : Ideal of PP3\\
\endOutput
\beginOutput
i54 : Xzero = trim substitute(saturate(X, t), \{t => 0\})\\
\emptyLine
\                   2        2\\
o54 = ideal (y*z, y , x*y, x )\\
\emptyLine
o54 : Ideal of PP3\\
\endOutput
Secondly, we verify that this is the union of a double line and an
embedded point of degree $1$.
\beginOutput
i55 : Xzero == intersect(ideal(x^2, y), ideal(x, y^2, z))\\
\emptyLine
o55 = true\\
\endOutput
\beginOutput
i56 : degree(ideal(x^2, y ) / ideal(x, y^2, z))\\
\emptyLine
o56 = 1\\
\endOutput
\beginOutput
i57 : clearAll\\
\endOutput
\qed
\end{solution*}

Section~III.3.4 in Eisenbud and Harris~\cite{SC:EH} contains several
other interesting limits of various flat families.


%%----------------------------------------------------------
\section{B\'{e}zout's Theorem}

B\'{e}zout's Theorem\index{Bezout's Theorem@B\'ezout's Theorem} --- Theorem~III-78 in
Eisenbud and Harris~\cite{SC:EH} --- may fail without the
Cohen-Macaulay\index{Cohen-Macaulay} hypothesis.  Our seventh problem
is to demonstrate this.

\begin{problem*}[Exercise~III-81 in \cite{SC:EH}]
Find irreducible closed subvarieties $X$ and $Y$ in $\bbbp^{4}$ such
that 
\begin{align*}
\codim(X \cap Y) &= \codim(X) + \codim(Y) \\
\deg(X \cap Y) &> \deg(X) \cdot \deg(Y) \, .
\end{align*}
\end{problem*}

\begin{solution*}
We show that the assertion holds when $X$ is the cone over the
nonsingular rational quartic curve\index{rational quartic curve} in
$\bbbp^{3}$ and $Y$ is a two-plane passing through the vertex of the
cone.  First, recall that the rational quartic curve is given by the
$2 \times 2$ minors of the matrix $\left[ \begin{smallmatrix} a &
b^{2} & bd & c \\ b & ac & c^2 & d \end{smallmatrix} \right]$; see
Exercise~18.8 in Eisenbud~\cite{SC:E}.  Thus, we have
\beginOutput
i58 : S = QQ[a, b, c, d, e];\\
\endOutput
\beginOutput
i59 : IX = trim minors(2, matrix\{\{a, b^2, b*d, c\},\{b, a*c, c^2, d\}\})\\
\emptyLine
\                         3      2     2    2    3    2\\
o59 = ideal (b*c - a*d, c  - b*d , a*c  - b d, b  - a c)\\
\emptyLine
o59 : Ideal of S\\
\endOutput
\beginOutput
i60 : IY = ideal(a, d);\\
\emptyLine
o60 : Ideal of S\\
\endOutput
\beginOutput
i61 : codim IX + codim IY == codim (IX + IY)\\
\emptyLine
o61 = true\\
\endOutput
\beginOutput
i62 : (degree IX) * (degree IY)\\
\emptyLine
o62 = 4\\
\endOutput
\beginOutput
i63 : degree (IX + IY)\\
\emptyLine
o63 = 5\\
\endOutput
which establishes the assertion.\qed
\end{solution*}

To understand how this example works, it is enlightening to express
$Y$ as the intersection of two hyperplanes; one given by $a = 0$ and
the other given by $d = 0$.  Intersecting $X$ with the first
hyperplane yields
\beginOutput
i64 : J = ideal mingens (IX + ideal(a))\\
\emptyLine
\                      3      2   2    3\\
o64 = ideal (a, b*c, c  - b*d , b d, b )\\
\emptyLine
o64 : Ideal of S\\
\endOutput
However, this first intersection has an embedded point;
\beginOutput
i65 : J == intersect(ideal(a, b*c, b^2, c^3-b*d^2), \\
\           ideal(a, d, b*c, c^3, b^3)) -- embedded point\\
\emptyLine
o65 = true\\
\endOutput
\beginOutput
i66 : clearAll\\
\endOutput
The second hyperplane passes through this embedded
point\index{embedded point} which explains the extra intersection.


%%----------------------------------------------------------
\section{Constructing Blow-ups}

The blow-up\index{blow-up} of a scheme $X$ along a subscheme $Y$ can
be constructed from the Rees algebra\index{Rees algebra} associated to
the ideal sheaf of $Y$ in $X$; see Theorem~IV-22 in Eisenbud and
Harris~\cite{SC:EH}.  Gr\"{o}bner basis techniques allow one to
express the Rees algebra in terms of generators and relations.  We
illustrate this method in the next solution.

\begin{problem*}[Exercises~IV-43 \& IV-44 in \cite{SC:EH}]
Find the blow-up $X$ of the affine plane\index{scheme!affine}
$\mathbb{A}^{2} = \Spec\big( \bbbq[x, y] \big)$ along the subscheme
defined by $\langle x^{3}, xy, y^{2} \rangle$.  Show that $X$ is
nonsingular and its fiber over the origin is the union of two copies
of $\bbbp^{1}$ meeting at a point.
\end{problem*}

\begin{solution*}
We first provide a general function which returns the ideal of
relations for the Rees algebra.
\beginOutput
i67 : blowUpIdeal = (I) -> (\\
\           r := numgens I;\\
\           S := ring I;\\
\           n := numgens S;\\
\           K := coefficientRing S;\\
\           tR := K[t, gens S, vars(0..r-1), \\
\                     MonomialOrder => Eliminate 1];\\
\           f := map(tR, S, submatrix(vars tR, \{1..n\}));\\
\           F := f(gens I);\\
\           J := ideal apply(1..r, j -> (gens tR)_(n+j)-t*F_(0,(j-1)));\\
\           L := ideal selectInSubring(1, gens gb J);\\
\           R := K[gens S, vars(0..r-1)];\\
\           g := map(R, tR, 0 | vars R);\\
\           trim g(L));\\
\endOutput
Now, applying the function to our specific case yields: 
\beginOutput
i68 : S = QQ[x, y];\\
\endOutput
\beginOutput
i69 : I = ideal(x^3, x*y, y^2);\\
\emptyLine
o69 : Ideal of S\\
\endOutput
\beginOutput
i70 : J = blowUpIdeal(I)\\
\emptyLine
\                           2         2          3     2\\
o70 = ideal (y*b - x*c, x*b  - a*c, x b - y*a, x c - y a)\\
\emptyLine
o70 : Ideal of QQ [x, y, a, b, c]\\
\endOutput
Therefore, the blow-up of the affine plane along the given subscheme
is
\[
X = \Proj\left( \frac{(\bbbq[x,y])[a,b,c]}{\langle yb-xc, xb^{2}-ac,
x^{2}b-ya, x^{3}c-y^{2}a \rangle} \right) \, .
\]
Using \Mtwo, we can also verify that the scheme $X$ is
nonsingular\index{singular locus};
\beginOutput
i71 : J + ideal jacobian J == ideal gens ring J\\
\emptyLine
o71 = true\\
\endOutput
\beginOutput
i72 : clearAll\\
\endOutput
Since we have
\[
\frac{(\bbbq[x,y])[a,b,c]}{\langle yb-xc, xb^{2}-ac, x^{2}b-ya,
x^{3}c-y^{2}a \rangle} \otimes \frac{\bbbq[x,y]}{\langle x, y \rangle}
\cong \frac{\bbbq[a,b,c]}{\langle ac \rangle} \, ,
\]
the fiber over the origin $\langle x,y \rangle$ in $\mathbb{A}^{2}$ is
clearly a union of two copies of $\bbbp^{1}$ meeting at one point.  In
particular, the exceptional fiber is not a projective space.\qed
\end{solution*}

Many other interesting blow-ups can be found in section~II.2 in
Eisenbud and Harris~\cite{SC:EH}.


%%----------------------------------------------------------
\section{A Classic Blow-up}

We consider the blow-up\index{blow-up} of the projective plane
$\bbbp^{2}$ at a point.

\vbox{
\begin{problem*}
Show that the following varieties are isomorphic.
\begin{enumerate}
\item[$(a)$] the image of the rational map from $\bbbp^{2}$ to
$\bbbp^{4}$ given by
\[
(r:s:t) \mapsto (r^{2}:s^{2}:rs:rt:st) \, ;
\]
\item[$(b)$] the blow-up of the plane $\bbbp^{2}$ at the point
$(0:0:1)$;
\item[$(c)$] the determinantal variety\index{determinantal variety}
defined by the $2 \times 2$ minors of the matrix $\left[
\begin{smallmatrix} a & c & d \\ b & d & e \end{smallmatrix} \right]$
where $\bbbp^{4} = \Proj\big( k[a,b,c,d,e] \big)$.
\end{enumerate}
This surface is called the {\em cubic scroll}\index{cubic scroll} in
$\bbbp^{4}$.
\end{problem*}
}

\begin{solution*}
We find the ideal in part~$(a)$ by elimination
theory\index{elimination theory}.
\beginOutput
i73 : PP4 = QQ[a..e];\\
\endOutput
\beginOutput
i74 : S = QQ[r..t, A..E, MonomialOrder => Eliminate 3];\\
\endOutput
\beginOutput
i75 : I = ideal(A - r^2, B - s^2, C - r*s, D - r*t, E - s*t);\\
\emptyLine
o75 : Ideal of S\\
\endOutput
\beginOutput
i76 : phi = map(PP4, S, matrix\{\{0_PP4, 0_PP4, 0_PP4\}\} | vars PP4)\\
\emptyLine
o76 = map(PP4,S,\{0, 0, 0, a, b, c, d, e\})\\
\emptyLine
o76 : RingMap PP4 <--- S\\
\endOutput
\beginOutput
i77 : surfaceA = phi ideal selectInSubring(1, gens gb I)\\
\emptyLine
\                                          2\\
o77 = ideal (c*d - a*e, b*d - c*e, a*b - c )\\
\emptyLine
o77 : Ideal of PP4\\
\endOutput
Next, we determine the surface in part~$(b)$.  We construct the ideal
defining the blow-up of $\bbbp^{2}$ 
\beginOutput
i78 : R = QQ[t, x, y, z, u, v, MonomialOrder => Eliminate 1];\\
\endOutput
\beginOutput
i79 : blowUpIdeal = ideal selectInSubring(1, gens gb ideal(u-t*x, \\
\           v-t*y))\\
\emptyLine
o79 = ideal(y*u - x*v)\\
\emptyLine
o79 : Ideal of R\\
\endOutput
and embed it in $\bbbp^{2} \times \bbbp^{1}$.
\beginOutput
i80 : PP2xPP1 = QQ[x, y, z, u, v];\\
\endOutput
\beginOutput
i81 : embed = map(PP2xPP1, R, 0 | vars PP2xPP1);\\
\emptyLine
o81 : RingMap PP2xPP1 <--- R\\
\endOutput
\beginOutput
i82 : blowUp = PP2xPP1 / embed(blowUpIdeal);\\
\endOutput
We then map this surface into $\bbbp^{5}$ using the Segre
embedding\index{Segre embedding}.
\beginOutput
i83 : PP5 = QQ[A .. F];\\
\endOutput
\beginOutput
i84 : segre = map(blowUp, PP5, matrix\{\{x*u,y*u,z*u,x*v,y*v,z*v\}\});\\
\emptyLine
o84 : RingMap blowUp <--- PP5\\
\endOutput
\beginOutput
i85 : ker segre\\
\emptyLine
\                                2\\
o85 = ideal (B - D, C*E - D*F, D  - A*E, C*D - A*F)\\
\emptyLine
o85 : Ideal of PP5\\
\endOutput
Note that the image under the Segre map lies on a hyperplane in
$\bbbp^{5}$.  To get the desired surface in $\bbbp^{4}$, we project
\beginOutput
i86 : projection = map(PP4, PP5, matrix\{\{a, c, d, c, b, e\}\})\\
\emptyLine
o86 = map(PP4,PP5,\{a, c, d, c, b, e\})\\
\emptyLine
o86 : RingMap PP4 <--- PP5\\
\endOutput
\beginOutput
i87 : surfaceB = trim projection ker segre\\
\emptyLine
\                                          2\\
o87 = ideal (c*d - a*e, b*d - c*e, a*b - c )\\
\emptyLine
o87 : Ideal of PP4\\
\endOutput
Finally, we compute the surface in part~$(c)$.
\beginOutput
i88 : determinantal = minors(2, matrix\{\{a, c, d\}, \{b, d, e\}\})\\
\emptyLine
\                                          2\\
o88 = ideal (- b*c + a*d, - b*d + a*e, - d  + c*e)\\
\emptyLine
o88 : Ideal of PP4\\
\endOutput
\beginOutput
i89 : sigma = map( PP4, PP4, matrix\{\{d, e, a, c, b\}\});\\
\emptyLine
o89 : RingMap PP4 <--- PP4\\
\endOutput
\beginOutput
i90 : surfaceC = sigma determinantal\\
\emptyLine
\                                          2\\
o90 = ideal (c*d - a*e, b*d - c*e, a*b - c )\\
\emptyLine
o90 : Ideal of PP4\\
\endOutput
By incorporating a permutation of the variables into definition of
{\tt surfaceC}, we obtain the desired isomorphisms
\beginOutput
i91 : surfaceA == surfaceB\\
\emptyLine
o91 = true\\
\endOutput
\beginOutput
i92 : surfaceB == surfaceC\\
\emptyLine
o92 = true\\
\endOutput
\beginOutput
i93 : clearAll\\
\endOutput
which completes the solution.\qed
\end{solution*}

For more information of the geometry of rational normal scrolls, see
Lecture~8 in Harris~\cite{SC:H}.


%%----------------------------------------------------------
\section{Fano Schemes}

Our final example concerns the family of Fano schemes\index{Fano
scheme} associated to a flat family of quadrics.
Recall that the $k$-th Fano scheme $F_{k}(X)$ of a
scheme $X \subseteq \bbbp^{n}$ is the subscheme of
the Grassmannian parametrizing $k$-planes
contained in $X$.

\begin{problem*}[Exercise~IV-69 in \cite{SC:EH}]
Consider the one-parameter family\index{one-parameter family} of
quadrics tending to a double plane with equation
\[ 
Q = V(tx^{2}+ty^{2}+tz^{2}+w^{2}) \subseteq \bbbp^{3}_{\bbbq[t]} =
\Proj\big(\bbbq[t][x,y,z,w]\big) \enspace .
\]
What is the flat limit\index{flat limit} of the Fano schemes
$F_{1}(Q_{t})$?
\end{problem*}

\begin{solution*}
We first compute the ideal defining $F_{1}(Q_{t})$, the scheme
parametrizing lines in $Q$.
\beginOutput
i94 : PP3 = QQ[t, x, y, z, w];\\
\endOutput
\beginOutput
i95 : Q = ideal( t*x^2+t*y^2+t*z^2+w^2 );\\
\emptyLine
o95 : Ideal of PP3\\
\endOutput
To parametrize a line in our projective space, we introduce
indeterminates $u, v$ and $A, \dotsc, H$.
\beginOutput
i96 : R = QQ[t, u, v, A .. H];\\
\endOutput
We then make a map {\tt phi} from {\tt PP3} to {\tt R} sending the
variables to the coordinates of the general point on a line.
\beginOutput
i97 : phi = map(R, PP3, matrix\{\{t\}\} | \\
\              u*matrix\{\{A, B, C, D\}\} + v*matrix\{\{E, F, G, H\}\});\\
\emptyLine
o97 : RingMap R <--- PP3\\
\endOutput
\beginOutput
i98 : imageFamily = phi Q;\\
\emptyLine
o98 : Ideal of R\\
\endOutput
For a line to belong to $Q$, the {\tt imageFamily} must vanish
identically.  In other words, $F_{1}(Q)$ is defined by the
coefficients of the generators of {\tt imageFamily}.
%% removing a final use of 'coefficients'
%% coeffOfFamily = (coefficients ({1,2}, gens imageFamily))_1;
\beginOutput
i99 : coeffOfFamily = contract(matrix\{\{u^2,u*v,v^2\}\}, gens imageFamily)\\
\emptyLine
o99 = | tA2+tB2+tC2+D2 2tAE+2tBF+2tCG+2DH tE2+tF2+tG2+H2 |\\
\emptyLine
\              1       3\\
o99 : Matrix R  <--- R\\
\endOutput
Since we don't need the variables $u$ and $v$, we get rid of them.
\beginOutput
i100 : S = QQ[t, A..H];\\
\endOutput
\beginOutput
i101 : coeffOfFamily = substitute(coeffOfFamily, S);\\
\emptyLine
\               1       3\\
o101 : Matrix S  <--- S\\
\endOutput
\beginOutput
i102 : Sbar = S / (ideal coeffOfFamily);\\
\endOutput
Next, we move to the Grassmannian\index{Grassmannian} $\mathbb{G}(1,3)
\subset \bbbp^{5}$.  Recall the homogeneous coordinates on
$\bbbp^{5}$ correspond to the $2 \times 2$ minors of a $2 \times 4$
matrix.  We obtain these minors using the {\tt exteriorPower} function
in \Mtwo.
\beginOutput
i103 : psi = matrix\{\{t\}\} | exteriorPower(2, \\
\                   matrix\{\{A, B, C, D\}, \{E, F, G, H\}\})\\
\emptyLine
o103 = | t -BE+AF -CE+AG -CF+BG -DE+AH -DF+BH -DG+CH |\\
\emptyLine
\                  1          7\\
o103 : Matrix Sbar  <--- Sbar\\
\endOutput
\beginOutput
i104 : PP5 = QQ[t, a..f];\\
\endOutput
\beginOutput
i105 : fanoOfFamily = trim ker map(Sbar, PP5, psi);\\
\emptyLine
o105 : Ideal of PP5\\
\endOutput
Now, to answer the question, we determine the limit as $t$ tends to $0$.
\beginOutput
i106 : zeroFibre = trim substitute(saturate(fanoOfFamily, t), \{t=>0\})\\
\emptyLine
\                         2   2                   2                     $\cdot\cdot\cdot$\\
o106 = ideal (e*f, d*f, e , f , d*e, a*e + b*f, d , c*d - b*e + a*f, b $\cdot\cdot\cdot$\\
\emptyLine
o106 : Ideal of PP5\\
\endOutput
Let's transpose the matrix of generators so all of its elements are visible
on the printed page.
\beginOutput
i107 : transpose gens zeroFibre\\
\emptyLine
o107 = \{-2\} | ef       |\\
\       \{-2\} | df       |\\
\       \{-2\} | e2       |\\
\       \{-2\} | f2       |\\
\       \{-2\} | de       |\\
\       \{-2\} | ae+bf    |\\
\       \{-2\} | d2       |\\
\       \{-2\} | cd-be+af |\\
\       \{-2\} | bd+ce    |\\
\       \{-2\} | ad-cf    |\\
\       \{-2\} | a2+b2+c2 |\\
\emptyLine
\                 11         1\\
o107 : Matrix PP5   <--- PP5\\
\endOutput
We see that $F_{1}(Q_{0})$ is supported on the plane conic $\langle d,
e, f, a^{2}+b^{2}+c^{2} \rangle$.  However, $F_{1}(Q_{0})$ is not
reduced\index{scheme!non-reduced}; it has
multiplicity\index{multiplicity} two.  On the other hand, the generic
fiber is
\beginOutput
i108 : oneFibre = trim substitute(saturate(fanoOfFamily, t), \{t => 1\})\\
\emptyLine
\                          2    2    2                                  $\cdot\cdot\cdot$\\
o108 = ideal (a*e + b*f, d  + e  + f , c*d - b*e + a*f, b*d + c*e, a*d $\cdot\cdot\cdot$\\
\emptyLine
o108 : Ideal of PP5\\
\endOutput
\beginOutput
i109 : oneFibre == intersect(ideal(c-d, b+e, a-f, d^2+e^2+f^2), \\
\            ideal(c+d, b-e, a+f, d^2+e^2+f^2))\\
\emptyLine
o109 = true\\
\endOutput
Hence, for $t \neq 0$, $F_{1}(Q_{t})$ is the union of two conics lying
in complementary planes and $F_{1}(Q_{0})$ is the double conic
obtained when the two conics move together.\qed
\end{solution*}

% Local Variables:
% mode: latex
% mode: reftex
% tex-main-file: "chapter-wrapper.tex"
% reftex-keep-temporary-buffers: t
% reftex-use-external-file-finders: t
% reftex-external-file-finders: (("tex" . "make FILE=%f find-tex") ("bib" . "make FILE=%f find-bib"))
% End:
\begin{thebibliography}{1}

\bibitem{SC:CV}
Neil Chriss and Victor Ginzburg:
\newblock {\em Representation theory and complex geometry}.
\newblock Birkh\"auser Boston Inc., Boston, MA, 1997.

\bibitem{SC:CLO}
David Cox, John Little, and Donal O'Shea:
\newblock {\em Ideals, varieties, and algorithms}.
\newblock Springer-Verlag, New York, second edition, 1997.
\newblock An introduction to computational algebraic geometry and commutative
  algebra.

\bibitem{SC:CLO2}
David Cox, John Little, and Donal O'Shea:
\newblock {\em Using algebraic geometry}.
\newblock Springer-Verlag, New York, 1998.

\bibitem{SC:E}
David Eisenbud:
\newblock {\em Commutative algebra with a view toward algebraic geometry}.
\newblock Springer-Verlag, New York, 1995.

\bibitem{SC:EH}
David Eisenbud and Joe Harris:
\newblock {\em The geometry of schemes}.
\newblock Springer-Verlag, New York, 2000.

\bibitem{SC:H}
Joe Harris:
\newblock {\em Algebraic geometry, A first course}.
\newblock Springer-Verlag, New York, 1995.

\end{thebibliography}
  \egroup
 \makeatletter
 \renewcommand\thesection{\@arabic\c@section}
 \makeatother


  \part{Mathematical Computations}



%%%%%%%%%%%%%%%%%%%%%%%%%%%%%%%%%%%%%%%%%%%%%%%%
%%%%%
%%%%% ../chapters/monomialIdeals/chapter
%%%%%
%%%%%%%%%%%%%%%%%%%%%%%%%%%%%%%%%%%%%%%%%%%%%%%%

\bgroup
\title{Monomial Ideals}
\titlerunning{Monomial Ideals}
\toctitle{Monomial Ideals}
\author{Serkan Ho\c{s}ten
        % \inst 1
        \and Gregory G.~Smith
        % \inst 2
        }
\authorrunning{S. Ho\c{s}ten and G. G. Smith}
% \institute{Department of Mathematics, San Francisco State University,
% San Francisco, CA 94132, USA\and Department of Mathematics, University of
% California, Berkeley, CA 94720, USA}
\maketitle

\newtheorem*{sproof}{Proof of Proposition}{\itshape}{\rmfamily}
%% old form caused problems:
%%%% \newtheorem*{sproof}{Proof of Proposition~\ref{pro:complexity}}{\itshape}{\rmfamily}

\newcommand{\supp}{\operatorname{supp}}
\newcommand{\depth}{\operatorname{depth}}
\newcommand{\GL}{\operatorname{GL}}
\newcommand{\initial}{\operatorname{in}}
\newcommand{\Hilb}{\operatorname{Hilb}}
\newcommand{\codim}{\operatorname{codim}}
\newcommand{\sg}{{\sf g}}
\newcommand{\IP}{\operatorname{IP}}

\begin{abstract}
Monomial ideals form an important link between commutative algebra and
combinatorics.  In this chapter, we demonstrate how to implement
algorithms in \Mtwo for studying and using monomial ideals.  We
illustrate these methods with examples from combinatorics, integer
programming, and algebraic geometry.
\end{abstract}


%%--------------------------------------------------------------

An ideal $I$ in $S = \bbbq[x_{1}, \dotsc, x_{n}]$ is called a monomial
ideal\index{monomial ideal}\index{ideal!monomial} if it satisfies any
of the following equivalent conditions:
\begin{enumerate}
\item[$(a)$] $I$ is generated by monomials,
\item[$(b)$] if $f = \sum_{\alpha \in \bbbn^{n}} k_{\alpha}
x^{\alpha}$ belongs to $I$ then $x^{\alpha} \in I$ whenever
$k_{\alpha} \neq 0$,
\item[$(c)$] $I$ is torus-fixed; in other words, if $(c_{1}, \dotsc,
c_{n}) \in (\bbbq^{*})^{n}$, then $I$ is fixed under the action $x_{i}
\mapsto c_{i}x_{i}$ for all $i$.
\end{enumerate}
It follows that a monomial ideal is uniquely determined by the
monomials it contains.  Most operations are far simpler for a
monomial ideal than for an ideal generated by arbitrary polynomials.
In particular, many invariants can be effectively determined for
monomial ideals.  As a result, one can solve a broad collection of
problems by reducing to or encoding data in a monomial ideal.  The aim
of this chapter is to develop the computational aspects of monomial
ideals in \Mtwo and demonstrate a range of applications.

This chapter is divided into five sections.  Each section begins with
a discussion of a computational procedure involving monomial ideals.
Algorithms are presented as \Mtwo functions.  We illustrate these
methods by solving problems from various areas of mathematics.
In particular, we include the \Mtwo code for generating interesting
families of monomial ideals.  The first section introduces the basic
functions on monomial ideals in \Mtwo.  To demonstrate these functions,
we use the Stanley-Reisner ideal associated to a simplicial complex to
compute its $f$-vector.  Next, we present two algorithms for finding a
primary decomposition of a monomial ideal.  In a related example, we
use graph ideals to study the complexity of determining the
codimension of a monomial ideal.  The third section focuses on the
standard pairs of a monomial ideal; two methods are given for finding
the set of standard pairs.  As an application, we use standard pairs
to solve integer linear programming problems.  The fourth section
examines Borel-fixed ideals and generic initial ideals.  Combining
these constructions with distractions, we demonstrate that the Hilbert
scheme $\Hilb^{\, 4t+1}(\bbbp^{4})$ is connected.  Finally, we look at
the chains of associated primes in various families of monomial
ideals.


%%----------------------------------------------------------
\section{The Basics of Monomial Ideals}

Creating monomial ideals in \Mtwo is analogous to creating general
ideals.  The monomial ideal\index{monomial
ideal}\index{ideal!monomial} generated by a sequence or list of
monomials can be constructed with the function {\tt monomialIdeal}\indexcmd{monomialIdeal}.
\beginOutput
i1 : S = QQ[a, b, c, d]; \\
\endOutput
\beginOutput
i2 : I = monomialIdeal(a^2, a*b, b^3, a*c)\\
\emptyLine
\                     2        3\\
o2 = monomialIdeal (a , a*b, b , a*c)\\
\emptyLine
o2 : MonomialIdeal of S\\
\endOutput
\beginOutput
i3 : J = monomialIdeal\{a^2, a*b, b^2\}\\
\emptyLine
\                     2        2\\
o3 = monomialIdeal (a , a*b, b )\\
\emptyLine
o3 : MonomialIdeal of S\\
\endOutput
The type {\tt MonomialIdeal}\indexcmd{MonomialIdeal} is the class of all monomial ideals.  If
an entry in the sequence or list is not a single monomial, then {\tt
monomialIdeal} takes only the leading monomial; recall that every
polynomial ring in \Mtwo is equipped with a monomial ordering.
\beginOutput
i4 : monomialIdeal(a^2+a*b, a*b+3, b^2+d)\\
\emptyLine
\                     2        2\\
o4 = monomialIdeal (a , a*b, b )\\
\emptyLine
o4 : MonomialIdeal of S\\
\endOutput

There are also several methods of associating a monomial
ideal to an arbitrary ideal in a polynomial ring.  The most important
of these is the initial ideal\index{initial
ideal}\index{ideal!initial} --- the monomial ideal generated by the
leading monomials of all elements in the given ideal.  When applied to
an {\tt Ideal}, the function {\tt monomialIdeal} returns the initial
ideal.
\beginOutput
i5 : K = ideal(a^2, b^2, a*b+b*c)\\
\emptyLine
\             2   2\\
o5 = ideal (a , b , a*b + b*c)\\
\emptyLine
o5 : Ideal of S\\
\endOutput
\beginOutput
i6 : monomialIdeal K\\
\emptyLine
\                     2        2     2\\
o6 = monomialIdeal (a , a*b, b , b*c )\\
\emptyLine
o6 : MonomialIdeal of S\\
\endOutput
This is equivalent to taking the leading monomials of a Gr\"{o}bner
basis\index{Grobner basis@Gr\"obner basis} for {\tt K}.  In our example, the given
generators for {\tt K} are not a Gr\"{o}bner basis.
\beginOutput
i7 : monomialIdeal gens K\\
\emptyLine
\                     2        2\\
o7 = monomialIdeal (a , a*b, b )\\
\emptyLine
o7 : MonomialIdeal of S\\
\endOutput

One can also test if a general ideal is generated by monomials with
the function {\tt isMonomialIdeal}\indexcmd{isMonomialIdeal}.
\beginOutput
i8 : isMonomialIdeal K\\
\emptyLine
o8 = false\\
\endOutput
\beginOutput
i9 : isMonomialIdeal ideal(a^5, b^2*c, d^11)\\
\emptyLine
o9 = true\\
\endOutput
The usual algebraic operations on monomial ideals are the same as on
general ideals. For example, we have
\beginOutput
i10 : I+J\\
\emptyLine
\                      2        2\\
o10 = monomialIdeal (a , a*b, b , a*c)\\
\emptyLine
o10 : MonomialIdeal of S\\
\endOutput


%%----------------------------------------------------------
\subsection*{Example: Stanley-Reisner Ideals and $f$-vectors} 

Radical monomial ideals --- ideals generated by squarefree monomials
--- have a beautiful combinatorial interpretation in terms of
simplicial complexes\index{simplicial complex}.  More explicitly, a
simplicial complex $\Delta$ on the vertex set $\{ x_{1}, \dotsc, x_{n}
\}$ corresponds to the ideal $I_{\Delta}$ in $S = \bbbq[x_{1}, \dotsc,
x_{n}]$ generated by all monomials $x_{i_{1}} \dotsb x_{i_{p}}$ such
that $\{x_{i_{1}}, \dotsc, x_{i_{p}} \} \not\in \Delta$. The ideal
$I_{\Delta}$ is called the Stanley-Reisner\index{Stanley-Reisner
ideal}\index{ideal!Stanley-Reisner} ideal of $\Delta$.

To illustrate the connections between Stanley-Reisner ideals and
simplicial complexes, we consider the
$f$-vector\index{f-vector@$f$-vector}\index{simplicial complex!$f$-vector}.
Perhaps the most important invariant of a simplicial complex, the
$f$-vector of a $d$-dimensional simplicial complex $\Delta$ is
$(f_{0}, f_{1}, \dotsc, f_{d}) \in \bbbn^{d+1}$, where $f_{i}$ denotes
the number of $i$-dimensional faces in $\Delta$.  From the monomial
ideal point of view, the $f$-vector is encoded in the Hilbert
series\index{Hilbert series} of the quotient ring $S/I_{\Delta}$ as
follows:

\begin{theorem}
If $\Delta$ is a simplicial complex with $f$-vector $(f_{0}, \dotsc,
f_{d})$, then the Hilbert series of $S / I_{\Delta}$ is
\[
H_{S/I_{\Delta}}(t) = \sum_{i=-1}^{d} \frac{f_{i}t^{i+1}}{(1-t)^{i+1}}
\, ,
\]
where $f_{-1} = 1$.
\end{theorem}

\begin{proof}
Following Stanley~\cite{MR98h:05001}, we work with the fine
grading and then
specialize.  The fine grading of $S$ is the $\bbbz^{n}$-grading
defined by $\deg x_{i} = \mathbf{e}_{i} \in \bbbz^{n}$, where
$\mathbf{e}_{i}$ is the $i$-th standard basis vector.  The
support\index{monomial!support} of a monomial $x^{\alpha}$ is defined
to be the set $\supp(x^{\alpha}) = \{ x_{i} : \alpha_{i} > 0 \}$.
Observe that $x^{\alpha} \neq 0$ in $S / I_{\Delta}$ if and only if
$\supp(x^{\alpha}) \in \Delta$.  Moreover, the nonzero monomials
$x^{\alpha}$ form a $\bbbq$-basis of $S / I_{\Delta}$.  By counting
such monomials according to their support, we obtain the following
expression for the Hilbert series with the fine grading:
\[
H_{S/I_{\Delta}}(\mathbf{t}) = \sum_{F \in \Delta}
\sum_{\substack{\alpha \in \bbbn^{n} \\ \supp(x^{\alpha}) = F}}
\mathbf{t}^{\alpha} = \sum_{F \in \Delta} \prod_{x_{i} \in F}
\frac{t_{i}}{1-t_{i}} \, .
\]
Finally, by replacing each $t_{i}$ with $t$, we complete the
proof.\qed
\end{proof}

Since $H_{S/I_{\Delta}}(t)$ is typically expressed in the form
$\frac{h_{0} + h_{1}t + \dotsb + h_{d}t^{d}}{(1-t)^{d+1}}$, we can
obtain the $f$-vector by using the identity $\sum_{i} h_{i}t^{i} =
\sum_{j=0}^{d} f_{j-1}t^{j}(1-t)^{d-j}$.  In particular, we can
compute $f$-vectors from Stanley-Reisner ideals as follows:
\beginOutput
i11 : fvector = I -> (\\
\           R := (ring I)/I;\\
\           d := dim R;\\
\           N := poincare R;\\
\           t := first gens ring N;\\
\           while 0 == substitute(N, t => 1) do N = N // (1-t);\\
\           h := apply(reverse toList(0..d), i -> N_(t^i));\\
\           f := j -> sum(0..j+1, i -> binomial(d-i, j+1-i)*h#(d-i));\\
\           apply(toList(0..d-1), j -> f(j)));\\
\endOutput
For example, we can demonstrate that the $f$-vector of the octahedron
is $(6,12,8)$.
\beginOutput
i12 : S = QQ[x_1 .. x_6];\\
\endOutput
\beginOutput
i13 : octahedron = monomialIdeal(x_1*x_2, x_3*x_4, x_5*x_6)\\
\emptyLine
o13 = monomialIdeal (x x , x x , x x )\\
\                      1 2   3 4   5 6\\
\emptyLine
o13 : MonomialIdeal of S\\
\endOutput
\beginOutput
i14 : fvector octahedron\\
\emptyLine
o14 = \{6, 12, 8\}\\
\emptyLine
o14 : List\\
\endOutput

\begin{figure}
\begin{center}
\epsfysize=1.8in \epsfbox{octahedron.eps}
\end{center}
\caption{The octahedron}
\end{figure}

More generally, we can recursively construct simplicial $2$-spheres
with $f_{0} \geq 4$, starting with the tetrahedron, by pulling a point
in the relative interior of a facet.  This procedure leads to the
following family:
\beginOutput
i15 : simplicial2sphere = v -> ( \\
\           S := QQ[x_1..x_v]; \\
\           if v === 4 then monomialIdeal product gens S \\
\           else ( \\
\                L := \{\};\\
\                scan(1..v-4, i -> L = L | apply(v-i-3, \\
\                          j -> x_i*x_(i+j+4))); \\
\                scan(2..v-3, i -> L = L | \{x_i*x_(i+1)*x_(i+2)\}); \\
\                monomialIdeal L));\\
\endOutput
\beginOutput
i16 : apply(\{4,5,6,7,8\}, j -> fvector simplicial2sphere(j))\\
\emptyLine
o16 = \{\{4, 6, 4\}, \{5, 9, 6\}, \{6, 12, 8\}, \{7, 15, 10\}, \{8, 18, 12\}\}\\
\emptyLine
o16 : List\\
\endOutput
In fact, it follows from Euler's formula that the $f$-vector of any
simplicial $2$-sphere has the form $(v, 3v-6, 2v-4)$ for $v \geq 4$.
The problem of characterizing the $f$-vectors for triangulations of
$d$-spheres is open for $d \geq 3$.  One of the most important results
in this direction is the upper bound theorem for simplicial spheres
(Corollary~5.4.7 Bruns and Herzog~\cite{MR95h:13020}) which states
that the cyclic polytope has the maximal number of $i$-faces for all
$i$.  We point out that Stanley's proof of this theorem depends
heavily on these methods from commutative algebra.

On the other hand, the $f$-vectors for several major classes of
simplicial complexes have been characterized.  The Kruskal-Katona
theorem (Theorem~8.32 in Ziegler~\cite{MR96a:52011}) gives necessary
and sufficient conditions for a sequence of nonnegative integers to be
an $f$-vector of a simplicial complex.  Stanley~\cite{MR98h:05001}
describes the $f$-vectors of pure shellable complexes and
Cohen-Macaulay complexes.  Given the Betti numbers of a simplicial
complex, Bj\"{o}rner and Kalai~\cite{MR89m:52009} specify the
$f$-vectors.  Finally, the $g$-theorem (Theorem~8.35 in
Ziegler~\cite{MR96a:52011}) characterizes the $f$-vectors for boundary
complexes of a simplicial convex polytope.

For a further study of Stanley-Reisner ideals see Bruns and
Herzog~\cite{MR95h:13020} and Stanley~\cite{MR98h:05001}.  For more
information of $f$-vectors, see Ziegler~\cite{MR96a:52011} and
Bj\"{o}rner~\cite{MR96h:05213}.

%%----------------------------------------------------------
\section{Primary Decomposition}

A primary decomposition\index{primary decomposition} of an ideal $I$
is an expression of $I$ as a finite intersection of primary ideals; an
ideal $J$ is called primary\index{ideal!primary} if $r_{1}r_{2} \in J$
implies either $r_{1} \in J$ or $r_{2}^{\ell} \in J$ for some $\ell >
0$.  Providing an algorithm for computing the primary decomposition of
an arbitrary ideal in a polynomial ring is quite difficult.  However,
for monomial ideals, there are two algorithms which are relatively
simple to describe.

We first present a recursive method for generating an irreducible
primary decomposition.  It is based on the following two observations.
\begin{lemma}
Let $I$ be a monomial ideal in $S = \bbbq[x_1, \dotsc, x_n]$.
\begin{enumerate}
\item[$(1)$] If $I$ is generated by pure powers of a subset of the
variables, then it is a primary ideal.
\item[$(2)$] If $r$ is minimal generator of $I$ such that $r =
r_{1}r_{2}$ where $r_{1}$ and $r_{2}$ are relatively prime, then $I =
\big(I + \langle r_{1} \rangle \big) \cap \big(I + \langle r_{2}
\rangle \big)$.
\end{enumerate}
\end{lemma}

\begin{proof}
$(1)$ This follows immediately from the definition of primary.  $(2)$
Since $I$ is a monomial ideal, it is enough to show that $I$ and
$\big( I + \langle r_{1} \rangle \big) \cap \big( I + \langle r_{1}
\rangle \big)$ contain the same monomials.  A monomial $r'$ belongs to
$\big( I + \langle r_{j} \rangle \big)$ if and only if $r' \in I$ or
$r_{j}$ divides $r'$.  Because $r_{1}$ and $r_{2}$ are relative prime,
we have
\[
r' \in \big( I + \langle r_{1} \rangle \big) \cap \big( I + \langle
r_{1} \rangle \big) \Leftrightarrow \text{ $r' \in I$ or $r_{1}r_{2}$
divides $r'$ } \Leftrightarrow \text{ $r' \in I$} \, . \qquad \qed
\]
\end{proof}

The following is an implementation of the resulting algorithm:
\beginOutput
i17 : supp = r -> select(gens ring r, e -> r {\char`\%} e == 0);\\
\endOutput
\beginOutput
i18 : monomialDecompose = method();\\
\endOutput
\beginOutput
i19 : monomialDecompose List := L -> (\\
\           P := select(L, I -> all(first entries gens I, \\
\                     r -> #supp(r) < 2) === false);\\
\           if #P > 0 then (\\
\                I := first P;\\
\                m := first select(first entries gens I, \\
\                     r -> #supp(r) > 1);\\
\                E := first exponents m;\\
\                i := position(E, e -> e =!= 0);\\
\                r1 := product apply(E_\{0..i\}, (gens ring I)_\{0..i\}, \\
\                     (j, r) -> r^j);\\
\                r2 := m // r1;\\
\                monomialDecompose(delete(I, L) | \{I+monomialIdeal(r1),\\
\                          I+monomialIdeal(r2)\}))\\
\           else L);\\
\endOutput
\beginOutput
i20 : monomialDecompose MonomialIdeal := I -> monomialDecompose \{I\};\\
\endOutput
Here is a small example illustrating this method.
\beginOutput
i21 : S = QQ[a,b,c,d];\\
\endOutput
\beginOutput
i22 : I = monomialIdeal(a^3*b, a^3*c, a*b^3, b^3*c, a*c^3, b*c^3)\\
\emptyLine
\                      3      3   3    3      3     3\\
o22 = monomialIdeal (a b, a*b , a c, b c, a*c , b*c )\\
\emptyLine
o22 : MonomialIdeal of S\\
\endOutput
\beginOutput
i23 : P = monomialDecompose I;\\
\endOutput
\beginOutput
i24 : scan(P, J -> << endl << J << endl);\\
\emptyLine
monomialIdeal (b, c)\\
\emptyLine
monomialIdeal (a, c)\\
\emptyLine
\                3   3   3\\
monomialIdeal (a , b , c )\\
\emptyLine
monomialIdeal (a, b)\\
\emptyLine
\                3      3\\
monomialIdeal (a , b, c )\\
\emptyLine
monomialIdeal (a, b)\\
\emptyLine
\                   3   3\\
monomialIdeal (a, b , c )\\
\endOutput
\beginOutput
i25 : I == intersect(P)\\
\emptyLine
o25 = true\\
\endOutput
As we see from this example, this procedure doesn't necessarily
yield an irredundant decomposition.

The second algorithm for finding a primary decomposition of a monomial
ideal $I$ is based on the Alexander dual\index{Alexander
dual}\index{dual, Alexander} of $I$.  The Alexander dual was first
introduced for squarefree monomial ideals.  In this case, it is the
monomial ideal of the dual of the simplicial complex $\Delta$
corresponding to $I$.  By definition the dual complex of $\Delta$ is
$\Delta^{\vee} = \{ F : F^{c} \not\in \Delta \}$, where $F^{c} =
\{x_{1}, \dotsc, x_{n}\} \setminus F$.  The following general
definition appears in Miller~\cite{M},~\cite{MR1779598}.  If
$I \subseteq \bbbq[x_{1},
\dotsc, x_{n}]$ is a monomial ideal and $x^{\lambda}$ is the least
common multiple of the minimal generators of $I$, then the Alexander
dual of $I$ is
\[
I^{\vee} = \left\langle \prod_{\beta_{i} > 0}
x_{i}^{\lambda_{i}+1-\beta_{i}} : \begin{tabular}{l} \text{$\langle
x_{i}^{\beta_{i}} : \beta_{i} \geq 1\rangle$ is an
irredundant} \\ \text{irreducible component of $I$} \end{tabular}
\right\rangle \, .
\]
In particular, the minimal generators of $I^{\vee}$ correspond to the
irredundant irreducible components of $I$.  The next proposition
provides a useful way of computing $I^{\vee}$ given a set of
generators for $I$.

\begin{proposition}  
If $I$ is a monomial ideal and $x^{\lambda}$ is the least common
multiple of the minimal generators of $I$, then the generators for
$I^{\vee}$ are those generators of the ideal $\left( \langle
x_{1}^{\lambda_{1}+1}, \dotsc, x_{n}^{\lambda_{n}+1} \rangle : I
\right)$ that are not divisible by $x_{i}^{\lambda_{i}+1}$ for $1 \leq
i \leq n$.
\end{proposition}

\begin{proof}
See Theorem~2.1 in Miller~\cite{M}. \qed
\end{proof}

Miller's definition of Alexander dual is even more general than the one
above. The resulting algorithm for computing this general Alexander
dual and primary decomposition are implemented in \Mtwo as follows. 
For the Alexander dual we use, the list {\tt a} that appears
as an input argument for {\tt dual}\indexcmd{dual} should be list of exponents of
the least common multiple of the minimal generators of $I$.
\beginOutput
i26 : code(dual, MonomialIdeal, List)\\
\emptyLine
o26 = -- ../../../m2/monideal.m2:260-278\\
\      dual(MonomialIdeal, List) := (I,a) -> ( -- Alexander dual\\
\           R := ring I;\\
\           X := gens R;\\
\           aI := lcmOfGens I;\\
\           if aI =!= a then (\\
\                if #aI =!= #a \\
\                then error (\\
\                     "expected list of length ",\\
\                     toString (#aI));\\
\                scan(a, aI, \\
\                     (b,c) -> (\\
\                          if b<c then\\
\                          error "exponent vector not large enough"\\
\                          ));\\
\                ); \\
\           S := R/(I + monomialIdeal apply(#X, i -> X#i^(a#i+1)));\\
\           monomialIdeal contract(\\
\                lift(syz transpose vars S, R), \\
\                product(#X, i -> X#i^(a#i))))\\
\endOutput
\beginOutput
i27 : code(primaryDecomposition, MonomialIdeal)\\
\emptyLine
o27 = -- ../../../m2/monideal.m2:286-295\\
\      primaryDecomposition MonomialIdeal := (I) -> (\\
\           R := ring I;\\
\           aI := lcmOfGens I;\\
\           M := first entries gens dual I;\\
\           L := unique apply(#M, i -> first exponents M_i);\\
\           apply(L, i -> monomialIdeal apply(#i, j -> ( \\
\                          if i#j === 0 then 0_R \\
\                          else R_j^(aI#j+1-i#j)\\
\                          )))\\
\           )\\
\endOutput
This direct algorithm is more efficient than our recursive algorithm.
In particular, it gives an irredundant decomposition.  For example,
when we use it to determine a primary decomposition for the ideal {\tt
I} above, we obtain
\beginOutput
i28 : L = primaryDecomposition I;\\
\endOutput
\beginOutput
i29 : scan(L, J -> << endl << J << endl);\\
\emptyLine
\                3   3   3\\
monomialIdeal (a , b , c )\\
\emptyLine
monomialIdeal (b, c)\\
\emptyLine
monomialIdeal (a, b)\\
\emptyLine
monomialIdeal (a, c)\\
\endOutput
\beginOutput
i30 : I == intersect L\\
\emptyLine
o30 = true\\
\endOutput

For a family of larger examples, we consider the tree
ideals\index{ideal!tree}:
\[
\left\langle \big({\textstyle \prod_{i \in F}} x_{i}\big)^{n-|F|+1} :
\text{ $\emptyset \neq F \subseteq \{ x_{1}, \dotsc, x_{n} \}$}
\right\rangle \, .
\]
These ideals are so named because their standard monomials (the
monomials not in the ideal) correspond to trees on $n+1$ labeled
vertices.  We determine the number of irredundant irreducible
components as follows:
\beginOutput
i31 : treeIdeal = n -> (\\
\           S = QQ[vars(0..n-1)];\\
\           L := delete(\{\}, subsets gens S);\\
\           monomialIdeal apply(L, F -> (product F)^(n - #F +1)));\\
\endOutput
\beginOutput
i32 : apply(2..6, i -> #primaryDecomposition treeIdeal i)\\
\emptyLine
o32 = (2, 6, 24, 120, 720)\\
\emptyLine
o32 : Sequence\\
\endOutput


%%----------------------------------------------------------
\subsection*{Example: Graph Ideals and Complexity Theory}

Monomial ideals also arise in graph theory.  Given a graph $G$ with
vertices $\{x_{1}, \dotsc, x_{n}\}$, we associate the
ideal\index{ideal!graph} $I_{G}$ in $\bbbq[x_{1}, \dotsc, x_{n}]$
generated by the quadratic monomials $x_{i}x_{j}$ such that $x_{i}$ is
adjacent to $x_{j}$.  The primary decomposition of $I_{G}$ is related to
the graph $G$ as follows.  Recall that a subset $F \subseteq \{x_{1},
\dotsc, x_{n}\}$ is called a {\em vertex cover}\index{vertex
cover}\index{cover} of $G$ if each edge in $G$ is incident to at least
one vertex in $F$.

\begin{lemma} \label{lem:graphideal}
If $G$ is a graph and $\mathcal{C}$ is the set of minimal vertex
covers of $G$ then the irreducible irredundant primary
decomposition\index{primary decomposition} of $I_{G}$ is $\bigcap_{F
\in \mathcal{C}} P_{F^{c}}$, where $P_{F^{c}}$ is the prime ideal
$\langle x_{i} : x_{i} \not\in F^{c} \rangle =
\langle x_{i} : x_{i} \in F \rangle$.
\end{lemma}

\begin{proof}
Since each generator of $I_{G}$ corresponds to an edge in $G$, it
follows from the {\tt monomialDecompose} algorithm that $I_{G}$ has an
irreducible primary decomposition of the form: $I_{G} = \bigcap
P_{F^{c}}$, where $F$ is a vertex cover.  To obtain an
irredundant decomposition, one clearly needs only the minimal vertex
covers. \qed
\end{proof}

As an application of graph ideals, we examine the complexity of
determining the codimension\index{ideal!codimension} of a monomial
ideal.  In fact, following Bayer and Stillman~\cite{MR94f:13018}, we
prove

\begin{proposition} \label{pro:complexity}
The following decision problem is NP-complete: 
\begin{equation} \tag{{\sc Codim}}
\text{\begin{minipage}[c]{200pt}
Given a monomial ideal $I \subseteq \bbbq[x_1, \dotsc, x_n]$
and $m \in \bbbn$, is $\codim I \leq m$?
\end{minipage}}
\end{equation}
\end{proposition}

By definition, a decision problem is NP-complete\index{NP-complete} if
all other problems in the class NP can be reduced to it.  To prove
that a particular problem is NP-complete, it suffices to show: $(1)$
the problem belongs to the class NP; $(2)$ some known NP-complete
problem reduces to the given decision problem (see Lemma~2.3 in Garey
and Johnson~\cite{MR80g:68056}).  One of the ``standard NP-complete''
problems (see section~3.1 in Garey and Johnson~\cite{MR80g:68056}) is
the following:
\begin{equation} \tag{{\sc Vertex Cover}}
\text{\begin{minipage}[c]{200pt} Given a graph $G$ and $m \in \bbbn$,
is there a vertex cover $F$ such that $|F| \leq m$? \end{minipage}}
\end{equation}

\begin{sproof}
$(1)$ Observe that a monomial ideal $I$ has codimension at most $m$ if
and only if $I \subseteq P_{F^{c}}$ for some $F$ with $|F|
\leq m$.  Now, if $I$ has codimension at most $m$, then given an
appropriate choice of $F$, one can verify in polynomial time that $I
\subseteq P_{F^{c}}$ and $|F| \leq m$.  Therefore, the {\sc
Codim} problem belongs to the class NP.

$(2)$ Lemma~\ref{lem:graphideal} implies that $I_{G}$ has codimension
$m$ if and only if $G$ has a vertex cover of size at most $m$.  In
particular, the {\sc Vertex Cover} problem reduces to the {\sc Codim}
problem. \qed
\end{sproof}

Thus, assuming P $\neq$ NP, there is no polynomial time algorithm for
finding the codimension of a monomial ideal.  Nevertheless, we can
effectively compute the codimension for many interesting examples.

To illustrate this point, we consider the following family of
examples.  Let $S = \bbbq[X]$ denote the polynomial ring generated by
the entries of a generic $m \times n$ matrix $X = [x_{i,j}]$.  Let
$I_{k}$\index{ideal!of minors} be the ideal generated by the $k \times
k$ minors of $X$.  Since the Hilbert function of $S / I_{k}$ equals
the Hilbert function of $S / \initial(I_{k})$ (see Theorem~15.26 in
Eisenbud~\cite{MR97a:13001}), we can determine the codimension $I_{k}$
by working with the monomial ideal $\initial(I_{k})$.  Because
Sturmfels~\cite{MR91m:14076} shows that the set of $k \times k$-minors
of $X$ is the reduced Gr\"{o}bner basis of $I_{k}$ with respect to the
lexicographic term order induced from the variable order
\[
x_{1,n} > x_{1,n-1} > \dotsb > x_{1,1} > x_{2,n} > \dotsb > x_{2,1} >
\dotsb > x_{m,n} > \dotsb > x_{m,1} \, ,
\] 
we can easily calculate $\initial(I_k)$.  In particular, in \Mtwo we
have
\beginOutput
i33 : minorsIdeal = (m,n,k) -> (\\
\           S := QQ[x_1..x_(m*n), MonomialOrder => Lex];\\
\           I := minors(k, matrix table(m, n, (i,j) -> x_(i*n+n-j)));\\
\           forceGB gens I;\\
\           I);\\
\endOutput
\beginOutput
i34 : apply(2..8, i -> time codim monomialIdeal minorsIdeal(i,2*i,2))\\
\     -- used 0.02 seconds\\
\     -- used 0.05 seconds\\
\     -- used 0.1 seconds\\
\     -- used 0.36 seconds\\
\     -- used 1.41 seconds\\
\     -- used 5.94 seconds\\
\     -- used 25.51 seconds\\
\emptyLine
o34 = (3, 10, 21, 36, 55, 78, 105)\\
\emptyLine
o34 : Sequence\\
\endOutput
The properties of $I_{k}$ are further developed in chapter~11 of
Sturmfels~\cite{MR97b:13034} and chapter~7 of Bruns and
Herzog~\cite{MR95h:13020}

For more on the relationships between a graph and its associated
ideal, see Villarreal~\cite{MR91b:13031}, Simis, Vasconcelos and
Villarreal~\cite{MR99c:13004}, and Ohsugi and
Hibi~\cite{MR2000a:13010}.

\beginOutput
i35 : erase symbol x;\\
\endOutput

%%----------------------------------------------------------
\section{Standard Pairs}

In this section, we examine a combinatorial object associated to a
monomial ideal.  In particular, we present two algorithms for
computing the standard pairs of a monomial ideal from its minimal
generators.  Before giving the definition of a standard pair, we
consider an example.

\begin{figure} 
\begin{center}
\epsfysize=2.4in \epsfbox{standardpairs-fixed.eps}
\end{center}
\caption{Staircase diagram for $I = \langle xy^3z, xy^2z^2, y^3z^2,
y^2z^3 \rangle$}
\end{figure}

\begin{example} \label{std-ex} 
Let $I = \langle xy^3z, xy^2z^2, y^3z^2, y^2z^3 \rangle$ in
$\bbbq[x,y,z]$.  We identify the monomials in $\bbbq[x,y,z]$
with the lattice points in $\bbbn^3$; see Figure~2.  The
standard monomials of $I$, those monomials which are not in $I$, can
be enumerated as follows: $(i)$ monomials corresponding to lattice
points in the $xy$-plane, $(ii)$ monomials corresponding to lattice
points in the $xz$-plane, $(iii)$ monomials corresponding to lattice
points in the plane parallel to the $xz$-plane containing $(0,1,0)$,
$(iv)$ monomials corresponding to lattice points on the line parallel
to the $y$-axis containing $(0,0,1)$, $(v)$ monomials corresponding to
lattices point on the line parallel to the $x$-axis containing
$(0,2,1)$, and $(vi)$ the monomial $y^{2}z^{2}$.
\end{example}

Following Sturmfels, Trung and Vogel~\cite{MR96i:13029}, we make the
following definitions.  Given a monomial $x^{\alpha}$ and a subset $F
\subseteq \{ x_{1}, \dotsc, x_{n} \}$, we index the set of monomials
of the form $x^{\alpha} \cdot x^{\beta}$ where $\supp(x^{\beta})
\subseteq F$ by the pair $(x^{\alpha}, F)$.  A standard
pair\index{standard pair} of a monomial ideal $I$ is a pair
$(x^{\alpha}, F)$ satisfying the following three conditions:
\begin{enumerate}
\item[$(1)$] $\supp(x^{\alpha}) \cap F = \emptyset$,
\item[$(2)$] all of the monomials represented by this pair are
standard, and
\item[$(3)$] $(x^{\alpha}, F) \not\subseteq (x^{\beta}, G)$ for any
other pair $(x^{\beta}, G)$ satisfying the first two conditions.
\end{enumerate}
Hence, the six standard pairs 
\[ 
(1, \{x,y\}), (1, \{x,z\}), (y,\{x,z\}), (z, \{y\}), (y^{2}z, \{x\}),
(y^2z^2, \emptyset)
\]
in Example~\ref{std-ex} correspond to $(i)$--$(vi)$.

Observe that the set of standard pairs of $I$ gives an irreducible
decomposition\index{primary decomposition} of $I = \bigcap \langle
x_{i}^{\alpha_{i}+1} : \text{ $x_{i} \notin F$} \rangle$, where the
intersection is over all standard pairs $(x^{\alpha}, F)$.  Moreover,
the prime ideal $P_F := \langle x_{i} : \text{ $x_{i} \notin F$}
\rangle$ is an associated prime of $I$ if and only if there exists a
standard pair of the form $(\bullet, F)$; see Sturmfels, Trung and
Vogel~\cite{MR96i:13029} for details.

Our first algorithm for computing the set of standard
pairs\index{standard pair} is taken from Ho\c{s}ten and
Thomas~\cite{MR2000f:13052}.  The ideas behind it are as follows:
given a witness $w_1 = x^{\alpha}$ for the associated prime $P_F :=
\langle x_{i} : \text{ $x_{i} \notin F$} \rangle$, that is $(I :
x^{\alpha}) = P_F$, set $w_2 = \prod_{x_i \in \supp(w_1) \cap F^c}
x_i^{\alpha_{i}}$.  It follows that $(w_2, F)$ is a standard pair of
$I$.  Now, consider the standard pairs of the slightly larger ideal $I
+ \langle w_1 \rangle$.  Clearly $(w_2, F)$ is not a standard pair of
this ideal because $w_1$ ``destroys'' it.  This larger ideal might
have standard pairs which cover standard monomials in $(w_2, F)$ that
are not in the pair $(w_1, F)$.  However, all other standard pairs are
the same as the original ideal $I$.  Thus, the problem of finding all
standard pairs of $I$ reduces to determining if a standard pair of $I
+ \langle w_1 \rangle$ is a standard pair for $I$.  To decide if a
pair $(x^{\beta}, G)$ of $I + \langle w_1 \rangle$ is a standard pair
of $I$, we first check that $P_{F}$ is an associated prime of $I$.  If
this is true, we determine if $(w_{2},F)$ is covered by
$(x^{\beta},G)$.

The \Mtwo version of this algorithm takes the following form:
\beginOutput
i36 : stdPairs = I -> (\\
\           S := ring I;\\
\           X := gens S;\\
\           std := \{\};\\
\           J := I;\\
\           while J != S do (\\
\                w1 := 1_S;\\
\                F := X;\\
\                K := J;\\
\                while K != 0 do (\\
\                     g1 := (ideal mingens ideal K)_0;\\
\                     x := first supp g1;\\
\                     w1 = w1 * g1 // x;\\
\                     F = delete(x, F);\\
\                     K = K : monomialIdeal(g1 // x);\\
\                     L := select(first entries gens K, \\
\                          r -> not member(x, supp r));\\
\                     if #L > 0 then K = monomialIdeal L\\
\                     else K = monomialIdeal 0_S;);\\
\                w2 := w1;\\
\                scan(X, r -> if not member(r, supp w1) or member(r, F)\\
\                     then w2 = substitute(w2, \{r => 1\}));\\
\                P := monomialIdeal select(X, r -> not member(r, F));\\
\                if (I:(I:P) == P) and (all(std, p -> \\
\                          (w2 {\char`\%} (first p) != 0) or not\\
\                          isSubset(supp(w2 // first p) | F, last p)))\\
\                then std = std | \{\{w2, F\}\};\\
\                J = J + monomialIdeal(w1););\\
\           std);\\
\endOutput
We can compute the standard pairs of Example~\ref{std-ex} using this
\Mtwo function:
\beginOutput
i37 : S = QQ[x,y,z];\\
\endOutput
\beginOutput
i38 : I = monomialIdeal(x*y^3*z, x*y^2*z^2, y^3*z^2, y^2*z^3);\\
\emptyLine
o38 : MonomialIdeal of S\\
\endOutput
\beginOutput
i39 : scan(time stdPairs I, P -> << endl << P << endl);\\
\     -- used 0.66 seconds\\
\emptyLine
\{y, \{x, z\}\}\\
\emptyLine
\{1, \{x, z\}\}\\
\emptyLine
\  2 2\\
\{y z , \{\}\}\\
\emptyLine
\{z, \{y\}\}\\
\emptyLine
\  2\\
\{y z, \{x\}\}\\
\emptyLine
\{1, \{x, y\}\}\\
\endOutput

Our second algorithm is taken from section~3.2 of Saito, Sturmfels and
Takayama~\cite{MR1734566}.  The proposition below provides the main
ingredient for this algorithm.  If $I$ is a monomial ideal and $F
\subseteq \{ x_{1},\dotsc, x_{n} \}$, we write $I_F$ for the monomial
ideal in $\bbbq[x_i : \text{ $x_{i} \notin F$}]$ obtained by replacing
each $x_{i} \in F$ with $1$ in every minimal generator of $I$.

\begin{proposition} 
For $(x^{\alpha}, F)$ to be a standard pair of $I$, it is necessary
and sufficient that $(x^{\alpha}, \emptyset)$ be a standard pair of
$I_F$.
\end{proposition}

\begin{proof} 
Lemma 3.1 in Sturmfels, Trung and Vogel \cite{MR96i:13029}.
\end{proof}

The definition of a standard pair\index{standard pair} implies that
$(x^{\alpha}, \emptyset)$ is a standard pair of $I_F$ if and only if
$x^{\alpha}$ is one of the finitely many monomials contained in $(I_F
: P_F^\infty)$ but not contained in $I_F$, where $P_F = \langle x_i:
x_i \notin F \rangle$.  Since ideal quotients and saturations are
implemented in \Mtwo, this reduces the problem to finding a set $D$
which contains $F$ for every associated prime $P_{F}$ of $I$.  One
approach is to simply compute the associated primes of $I$ from a
primary decomposition.

The method {\tt standardPairs}\indexcmd{standardPairs} uses this algorithm to determine the
set of standard pairs for a monomial ideal.
\beginOutput
i40 : code(standardPairs, MonomialIdeal, List)\\
\emptyLine
o40 = -- ../../../m2/monideal.m2:318-341\\
\      standardPairs(MonomialIdeal, List) := (I,D) -> (\\
\           R := ring I;\\
\           X := gens R;\\
\           S := \{\};\\
\           k := coefficientRing R;\\
\           scan(D, L -> ( \\
\                     Y := X;\\
\                     m := vars R;\\
\                     Lset := set L;\\
\                     Y = select(Y, r -> not Lset#?r);\\
\                     m = substitute(m, apply(L, r -> r => 1));\\
\                     -- using monoid to create ring to avoid \\
\                     -- changing global ring.\\
\                     A := k (monoid [Y]);\\
\                     phi := map(A, R, substitute(m, A));\\
\                     J := ideal mingens ideal phi gens I;\\
\                     Jsat := saturate(J, ideal vars A);\\
\                     if Jsat != J then (\\
\                          B := flatten entries super basis (\\
\                               trim (Jsat / J));\\
\                          psi := map(R, A, matrix\{Y\});\\
\                          S = join(S, apply(B, b -> \{psi(b), L\}));\\
\                          )));\\
\           S)\\
\endOutput
\beginOutput
i41 : time standardPairs I;\\
\     -- used 0.83 seconds\\
\endOutput

As an example, we will compute the standard pairs of the permutahedron
ideal\index{ideal!permutahedron}. Let $S = \bbbq[x_1, \dotsc, x_n]$
and let $\mathfrak{S}_n$ be the symmetric group of order $n$.  We
write $\rho$ for the vector $(1, 2, \dotsc, n)$ and $\sigma(\rho)$ for
the vector obtained by applying $\sigma \in \mathfrak{S}_n$ to the
coordinates of $\rho$. The $n$-th permutahedron ideal is $\langle
x^{\sigma(\rho)} : \text{ $\sigma \in \mathfrak{S}_n$} \rangle$.  We
compute the number of standard pairs for $2 \leq n \leq 5$.
\beginOutput
i42 : permutohedronIdeal = n -> (\\
\           S := QQ[X_1..X_n];\\
\           monomialIdeal terms det matrix table(n ,gens S, \\
\                (i,r) -> r^(i+1)));\\
\endOutput
\beginOutput
i43 : L = apply(\{2,3,4,5\}, j -> standardPairs(permutohedronIdeal(j)));\\
\endOutput
\beginOutput
i44 : apply(L, i -> #i)\\
\emptyLine
o44 = \{3, 10, 53, 446\}\\
\emptyLine
o44 : List\\
\endOutput
\beginOutput
i45 : erase symbol x; erase symbol z;\\
\endOutput


%%----------------------------------------------------------
\subsection*{Example: Integer Programming Problems}

As an application of standard pairs\index{standard pair}, we show how
to solve integer linear programming problems\index{integer programming}.
Let $A$ be a $d \times n$ matrix of nonnegative integers, let $\omega
\in \bbbr^{n}$ and fix $\beta \in \bbbz^{d}$.  We focus on the
following optimization problem
\[
\IP_{A,\omega}(\beta) : \,\,\, \text{ minimize $\omega \cdot \alpha$
subject to $A \alpha = \beta$, $\alpha \in \bbbn^n$.}
\] 
We view this integer linear program as a family depending on the
vector $\beta$.  The algorithm we present for solving
$\IP_{A,\omega}(\beta)$ depends on the proposition below.

The toric ideal\index{toric ideal}\index{ideal!toric} $I_{A} \subseteq
S = \bbbq[x_{1}, \dotsc, x_{n}]$ associated to $A$ is the binomial
ideal generated by $x^{\gamma} - x^{\delta}$ where $\gamma$, $\delta
\in \bbbn^{n}$ and $A \gamma = A \delta$.  We write $\initial_{\,
\omega}(I_{A})$ for the initial ideal\index{initial ideal} of $I_A$
with respect to the following order:
\[
x^{\gamma} \prec_{\omega} x^{\delta} \Longleftrightarrow \begin{cases}
\omega \cdot \gamma < \omega \cdot \delta & \text{ or } \\ 
\omega \cdot \gamma = \omega \cdot \delta & \text{ and } x^{\alpha}
\prec_{{\rm rlex}} x^{\gamma} \, .
\end{cases}
\]
For more information on toric ideals and their initial ideals see
Sturmfels~\cite{MR97b:13034}.

\begin{proposition} 
\begin{enumerate}
\item[$(1)$] A monomial $x^{\alpha}$ is a standard monomial of
$\initial_{\, \omega}(I_{A})$ if and only if $\alpha$ is the optimal
solution to the integer program $\IP_{A,\omega}(A\alpha)$.
\item[$(2)$] If $(\bullet, F)$ is a standard pair of
$\initial_{\, \omega}(I_{A})$, then the columns of $A$ corresponding to
$F$ are linearly independent.
\end{enumerate}
\end{proposition} 

\begin{proof} 
See Proposition~2.1 in Ho\c{s}ten and Thomas~\cite{MR2000b:13037} for
the proof of the first statement. The second statement follows from
Corollary~2.9 of the same article.
\qed
\end{proof}

The first statement implies that the standard pairs of $\initial_{\,
\omega}(I_{A})$ cover all optimal solutions to all integer programs
in $\IP_{A,\omega}$.  If $\alpha$ is the optimal solution to
$\IP_{A,\omega}(\beta)$ covered by the standard pair $(x^{\gamma},
F)$, then the second statement guarantees there exists a unique
$\delta \in \bbbn^{n}$ such that $A \delta = \beta - A \gamma$.
Therefore, $\alpha = \delta + \gamma$.  We point out that the
complexity of this algorithm is dominated by determining the set of
standard pairs of $\initial_{\, \omega}(I_{A})$ which depends only on
$A$ and $\omega$.  As a result, this method is particularly well
suited to solving $\IP_{A,\omega}(\beta)$ as $\beta$ varies.


To implement this algorithm in \Mtwo, we need a function which returns
the toric ideal $I_A$.  Following Algorithm~12.3 in
Sturmfels~\cite{MR97b:13034}, we have
\beginOutput
i47 : toBinomial = (b, S) -> (\\
\           pos := 1_S;\\
\           neg := 1_S;\\
\           scan(#b, i -> if b_i > 0 then pos = pos*S_i^(b_i)\\
\                         else if b_i < 0 then neg = neg*S_i^(-b_i));\\
\           pos - neg);\\
\endOutput
\beginOutput
i48 : toricIdeal = (A, omega) -> (\\
\           n := rank source A;\\
\           S = QQ[x_1..x_n, Weights => omega, MonomialSize => 16];\\
\           B := transpose matrix syz A;\\
\           J := ideal apply(entries B, b -> toBinomial(b, S));\\
\           scan(gens S, r -> J = saturate(J, r));\\
\           J);\\
\endOutput
Thus, we can solve $\IP_{A,\omega}(\beta)$ using the following
function.
\beginOutput
i49 : IP = (A, omega, beta) -> (\\
\           std := standardPairs monomialIdeal toricIdeal(A, omega);\\
\           n := rank source A;\\
\           alpha := \{\};\\
\           Q := first select(1, std, P -> (\\
\                F := apply(last P, r -> index r);\\
\                gamma := transpose matrix exponents first P;\\
\                K := transpose syz (submatrix(A,F) | (A*gamma-beta));\\
\                X := select(entries K, k -> abs last(k) === 1);\\
\                scan(X, k -> if all(k, j -> j>=0) or all(k, j -> j<=0)\\
\                     then alpha = apply(n, j -> if member(j, F) \\
\                          then last(k)*k_(position(F, i -> i === j))\\
\                          else 0));\\
\                #alpha > 0));\\
\           if #Q > 0 then (matrix \{alpha\})+(matrix exponents first Q)\\
\           else 0);\\
\endOutput

We illustrate this with some examples.
\beginOutput
i50 : A = matrix\{\{1,1,1,1,1\},\{1,2,4,5,6\}\}\\
\emptyLine
o50 = | 1 1 1 1 1 |\\
\      | 1 2 4 5 6 |\\
\emptyLine
\               2        5\\
o50 : Matrix ZZ  <--- ZZ\\
\endOutput
\beginOutput
i51 : w1 = \{1,1,1,1,1\};\\
\endOutput
\beginOutput
i52 : w2 = \{2,3,5,7,11\};\\
\endOutput
\beginOutput
i53 : b1 = transpose matrix\{\{3,9\}\}\\
\emptyLine
o53 = | 3 |\\
\      | 9 |\\
\emptyLine
\               2        1\\
o53 : Matrix ZZ  <--- ZZ\\
\endOutput
\beginOutput
i54 : b2 = transpose matrix\{\{5,16\}\}\\
\emptyLine
o54 = | 5  |\\
\      | 16 |\\
\emptyLine
\               2        1\\
o54 : Matrix ZZ  <--- ZZ\\
\endOutput
\beginOutput
i55 : IP(A, w1, b1)\\
\emptyLine
o55 = | 1 1 0 0 1 |\\
\emptyLine
\               1        5\\
o55 : Matrix ZZ  <--- ZZ\\
\endOutput
\beginOutput
i56 : IP(A, w2, b1)\\
\emptyLine
o56 = | 1 0 2 0 0 |\\
\emptyLine
\               1        5\\
o56 : Matrix ZZ  <--- ZZ\\
\endOutput
\beginOutput
i57 : IP(A, w1, b2)\\
\emptyLine
o57 = | 2 1 0 0 2 |\\
\emptyLine
\               1        5\\
o57 : Matrix ZZ  <--- ZZ\\
\endOutput
\beginOutput
i58 : IP(A, w2, b2)\\
\emptyLine
o58 = | 2 0 1 2 0 |\\
\emptyLine
\               1        5\\
o58 : Matrix ZZ  <--- ZZ\\
\endOutput

%%----------------------------------------------------------
\section{Generic Initial Ideals}

Gr\"{o}bner basis calculations and initial ideals depend heavily on
the given coordinate system.  By making a generic change of coordinates
before taking the initial ideal, we may eliminate this dependence.
This procedure also endows the resulting monomial ideal with a rich
combinatorial structure.

To describe this structure, we introduce the following definitions and
notation.  Let $S = \bbbq[x_{0}, \dotsc, x_{n}]$.  If $\sg = [g_{i,j}]
\in \GL_{n+1}(\bbbq)$ and $f \in S$, then $\sg \cdot f$ denotes the
standard action of the general linear group on $S$: $x_{i} \mapsto
\sum_{j=0}^{n} g_{i,j} x_{j}$.  For an ideal $I \subseteq S$, we
define $\sg \cdot I = \{ \sg \cdot f | f \in I \}$.  Let $B$ denote
the Borel subgroup of $\GL_{n+1}(\bbbq)$ consisting of upper
triangular matrices.  A monomial ideal $I$ is called
{\em Borel-fixed}\index{Borel-fixed}\index{ideal!Borel-fixed} if it
satisfies any of the following equivalent conditions:
\begin{enumerate}
\item[$(a)$] $\sg \cdot I = I$ for every $\sg \in B$;
\item[$(b)$] if $r$ is a generator of $I$ divisible by $x_{j}$ then
$\frac{rx_{i}}{x_{j}} \in I$ for all $i < j$;
\item[$(c)$] $\initial(\sg \cdot I) = I$ for every $\sg$ is some open
neighborhood of the identity in $B$.
\end{enumerate}
For a proof that these conditions are equivalent, see Propositon~1.25
in Green~\cite{MR99m:13040}.  

In \Mtwo, the function {\tt isBorel}\indexcmd{isBorel} tests whether a monomial ideal is
Borel-fixed.
\beginOutput
i59 : S = QQ[a,b,c,d];\\
\endOutput
\beginOutput
i60 : isBorel monomialIdeal(a^2, a*b, b^2)\\
\emptyLine
o60 = true\\
\endOutput
\beginOutput
i61 : isBorel monomialIdeal(a^2, b^2)\\
\emptyLine
o61 = false\\
\endOutput
The function {\tt borel}\indexcmd{borel} generates the smallest Borel-fixed ideal
containing the given monomial ideal.
\beginOutput
i62 : borel monomialIdeal(b*c)\\
\emptyLine
\                      2        2\\
o62 = monomialIdeal (a , a*b, b , a*c, b*c)\\
\emptyLine
o62 : MonomialIdeal of S\\
\endOutput
\beginOutput
i63 : borel monomialIdeal(a,c^3)\\
\emptyLine
\                         3   2      2   3\\
o63 = monomialIdeal (a, b , b c, b*c , c )\\
\emptyLine
o63 : MonomialIdeal of S\\
\endOutput

The next theorem provides the main source of Borel-fixed ideals.
\begin{theorem}[Galligo] \label{galligo}
Fix a term order on $S = \bbbq[x_{0}, \dotsc, x_{n}]$ such that $x_{0}
> \dotsc > x_{n}$.  If $I$ is a homogeneous ideal in $S$, then there
is a Zariski open subset $U \subseteq \GL_{n+1}(\bbbq)$ such that
\begin{enumerate}
\item[$(1)$] there is a monomial ideal $J \subseteq S$ such that $J =
\initial(\sg \cdot I )$ for all $\sg \in U$;
\item[$(2)$] the ideal $J$ is Borel-fixed.
\end{enumerate}
The ideal $J$ is called the generic initial ideal\index{generic
initial ideal}\index{ideal!generic initial} of $I$.
\end{theorem}

\begin{proof}
See Theorem~1.27 in Green~\cite{MR99m:13040}. \qed
\end{proof}

The following method allows one to compute generic initial ideals.
\beginOutput
i64 : gin = method();\\
\endOutput
\beginOutput
i65 : gin Ideal := I -> (\\
\           S := ring I;\\
\           StoS := map(S, S, random(S^\{0\}, S^\{numgens S:-1\}));\\
\           monomialIdeal StoS I);\\
\endOutput
\beginOutput
i66 : gin MonomialIdeal := I -> gin ideal I;\\
\endOutput
This routine assumes that the random function generates a matrix in
the Zariski open subset $U$.  Since we are working over a field of
characteristic zero this occurs with probability one.
For example, we can determine the generic initial ideal of two generic
homogeneous polynomials of degree $p$ and $q$ in $\bbbq[a,b,c,d]$.
\beginOutput
i67 : genericForms = (p,q) -> ideal(random(p,S), random(q,S));\\
\endOutput
\beginOutput
i68 : gin genericForms(2,2)\\
\emptyLine
\                      2        3\\
o68 = monomialIdeal (a , a*b, b )\\
\emptyLine
o68 : MonomialIdeal of S\\
\endOutput
\beginOutput
i69 : gin genericForms(2,3)\\
\emptyLine
\                      2     2   4\\
o69 = monomialIdeal (a , a*b , b )\\
\emptyLine
o69 : MonomialIdeal of S\\
\endOutput
Although the generic initial ideal is Borel-fixed, some non-generic
initial ideals may also be Borel-Fixed.
\beginOutput
i70 : J = ideal(a^2, a*b+b^2, a*c)\\
\emptyLine
\              2         2\\
o70 = ideal (a , a*b + b , a*c)\\
\emptyLine
o70 : Ideal of S\\
\endOutput
\beginOutput
i71 : ginJ = gin J\\
\emptyLine
\                      2        2     2\\
o71 = monomialIdeal (a , a*b, b , a*c )\\
\emptyLine
o71 : MonomialIdeal of S\\
\endOutput
\beginOutput
i72 : inJ = monomialIdeal J\\
\emptyLine
\                      2        3        2\\
o72 = monomialIdeal (a , a*b, b , a*c, b c)\\
\emptyLine
o72 : MonomialIdeal of S\\
\endOutput
\beginOutput
i73 : isBorel inJ and isBorel ginJ\\
\emptyLine
o73 = true\\
\endOutput
Finally, we show that the generic initial ideal does depend on the
term order by computing lexicographic generic initial ideal for two
generic forms of degree $p$ and $q$ in $\bbbq[a,b,c,d]$
\beginOutput
i74 : S = QQ[a,b,c,d, MonomialOrder => Lex];\\
\endOutput
\beginOutput
i75 : gin genericForms(2,2)\\
\emptyLine
\                      2        4     2\\
o75 = monomialIdeal (a , a*b, b , a*c )\\
\emptyLine
o75 : MonomialIdeal of S\\
\endOutput
\beginOutput
i76 : gin genericForms(2,3)\\
\emptyLine
\                      2     2   6       2     6         2       4\\
o76 = monomialIdeal (a , a*b , b , a*b*c , a*c , a*b*c*d , a*b*d )\\
\emptyLine
o76 : MonomialIdeal of S\\
\endOutput
A more comprehensive treatment of generic initial ideals can be found
in Green~\cite{MR99m:13040}.  The properties of Borel-fixed ideals in
characteristic $p>0$ are discussed in Eisenbud~\cite{MR97a:13001}.


%%----------------------------------------------------------
\subsection*{Example: Connectedness of the Hilbert Scheme}

Generic initial ideals are a powerful tool for studying the structure
of the Hilbert scheme\index{Hilbert scheme}.  Intuitively, the Hilbert
scheme $\Hilb^{\, p(t)}(\bbbp^{n})$ parameterizes subschemes $X
\subseteq \bbbp^{n}$ with Hilbert polynomial $p(t)$.  For an
introduction to Hilbert schemes see Harris and
Morrison~\cite{MR99g:14031}.  The construction of the Hilbert scheme $\Hilb^{\,
p(t)}(\bbbp^{n})$ can be found in Grothendieck's original
article~\cite{MR26:3566} or Altman and Kleiman~\cite{MR81f:14025a}.
While much is known about specific Hilbert schemes, the general
structure remain largely a mystery.  In particular, the component
structure --- the number of irreducible components, their dimensions,
how they intersect and what subschemes they parameterize --- is not
well understood.

Reeves~\cite{MR97g:14003} uses generic initial ideals to establish the
most important theorem to date on the component structure.  The
incidence graph\index{Hilbert scheme!incidence graph} of $\Hilb^{\,
p(t)}(\bbbp^{n})$ is defined as follows: to each irreducible component
we assign a vertex and we connect two vertices if the corresponding
components intersect.  Reeves~\cite{MR97g:14003} proves that the
distance (the number of edges in the shortest path) between any two
vertices in the incidence graph of $\Hilb^{\, p(t)}(\bbbp^{n})$ is at
most $2 \deg p(t)+2$.  Her proof can be divided into three major
steps.

{\em Step I: connect an arbitrary ideal to a Borel-fixed ideal.}
Passing to an initial ideal corresponds to taking the limit in a flat
family, in other words a path on the Hilbert scheme; see Theorem 15.17
in Eisenbud~\cite{MR97a:13001}.  Thus, Theorem~\ref{galligo} shows
that generic initial ideals\index{generic initial
ideal}\index{ideal!generic initial} connect arbitrary
ideals to Borel-fixed ideals.

{\em Step II: connect Borel-fixed ideals by projection.}  For a
homogeneous ideal $I \subseteq S = \bbbq[x_{0}, \dotsc, x_{n}]$, let
$\pi(I)$ denote the ideal obtained by setting $x_{n}=1$ and
$x_{n-1}=1$ in $I$.  With this notation, we have

\begin{theorem}
If $J$ is a Borel-fixed ideal, then the set of Borel-fixed ideals $I$,
with Hilbert polynomial $p(t)$ and $\pi(I) = J$, consists of ideals
defining subschemes of $\bbbp^{n}$ which all lie on a single component
of $\Hilb^{\, p(t)}(\bbbp^{n})$.
\end{theorem}

\begin{proof}
See Theorem~6 in Reeves~\cite{MR97g:14003}. \qed
\end{proof}

\noindent This gives an easy method for partitioning Borel-fixed
ideals into classes, each of which must lie in a single component.

{\em Step III: connect Borel-fixed ideals by distraction.}  Given a
Borel-fixed ideal, we produce a new ideal via a two-step process
called distraction\index{distraction}.  First, one polarizes the
Borel-fixed ideal.  The polarization\index{polarization} of a monomial
ideal $I \subset S$ is defined as:
\[
\left\langle \prod_{i=0}^{n}\prod_{j=1}^{\alpha_{i}} z_{i,j} : \text{
where $x_{0}^{\alpha_{0}}\dotsb x_{n}^{\alpha_{n}}$ is a minimal
generator of $I$} \right\rangle \, .
\]
One then pulls the result back to an ideal in the original variables
by taking a linear section of the polarization.  Theorem~4.10 in
Hartshorne~\cite{MR35:4232} shows that the distraction is connected to
the original Borel-fixed ideal.  Now, taking the lexicographic generic
initial ideal of the distraction yields a second Borel-fixed ideal.
Reeves~\cite{MR97g:14003} proves that repeating this process, at most
$\deg p(t) + 1$ times, one arrives at a distinguished component of
$\Hilb^{\, p(t)}(\bbbp^{n})$ called the lexicographic component.  For
more information on the lexicographic component see Reeves and
Stillman~\cite{MR98m:14003}.

We can implement these operations in \Mtwo as follows:
\beginOutput
i77 : projection = I -> (\\
\           S := ring I;\\
\           n := numgens S;\\
\           X := gens S;\\
\           monomialIdeal mingens substitute(ideal I, \\
\                \{X#(n-2) => 1, X#(n-1) => 1\}));\\
\endOutput
\beginOutput
i78 : polarization = I -> (\\
\           n := numgens ring I;\\
\           u := apply(numgens I, i -> first exponents I_i);\\
\           I.lcm = max {\char`\\} transpose u;\\
\           Z := flatten apply(n, i -> apply(I.lcm#i, j -> z_\{i,j\}));\\
\           R := QQ(monoid[Z]);\\
\           Z = gens R;\\
\           p := apply(n, i -> sum((I.lcm)_\{0..i-1\}));\\
\           monomialIdeal apply(u, e -> product apply(n, i -> \\
\                     product(toList(0..e#i-1), j -> Z#(p#i+j)))));\\
\endOutput
\beginOutput
i79 : distraction = I -> (\\
\           S := ring I;\\
\           n := numgens S;\\
\           X := gens S;\\
\           J := polarization I;\\
\           W := flatten apply(n, i -> flatten apply(I.lcm#i, \\
\                     j -> X#i));\\
\           section := map(S, ring J, apply(W, r -> r - \\
\                     random(500)*X#(n-2) - random(500)*X#(n-1)));     \\
\           section ideal J);\\
\endOutput
For example, we have
\beginOutput
i80 : S = QQ[x_0 .. x_4, MonomialOrder => GLex];\\
\endOutput
\beginOutput
i81 : I = monomialIdeal(x_0^2, x_0*x_1^2*x_3, x_1^3*x_4)\\
\emptyLine
\                      2     2     3\\
o81 = monomialIdeal (x , x x x , x x )\\
\                      0   0 1 3   1 4\\
\emptyLine
o81 : MonomialIdeal of S\\
\endOutput
\beginOutput
i82 : projection I\\
\emptyLine
\                      2     2   3\\
o82 = monomialIdeal (x , x x , x )\\
\                      0   0 1   1\\
\emptyLine
o82 : MonomialIdeal of S\\
\endOutput
\beginOutput
i83 : polarization I\\
\emptyLine
o83 = monomialIdeal (z      z      , z      z      z      z      , z   $\cdot\cdot\cdot$\\
\                      \{0, 0\} \{0, 1\}   \{0, 0\} \{1, 0\} \{1, 1\} \{3, 0\}   \{1 $\cdot\cdot\cdot$\\
\emptyLine
o83 : MonomialIdeal of QQ [z      , z      , z      , z      , z       $\cdot\cdot\cdot$\\
\                            \{0, 0\}   \{0, 1\}   \{1, 0\}   \{1, 1\}   \{1, 2\} $\cdot\cdot\cdot$\\
\endOutput
\beginOutput
i84 : distraction I\\
\emptyLine
\              2                             2                     2    $\cdot\cdot\cdot$\\
o84 = ideal (x  - 398x x  - 584x x  + 36001x  + 92816x x  + 47239x , - $\cdot\cdot\cdot$\\
\              0       0 3       0 4         3         3 4         4    $\cdot\cdot\cdot$\\
\emptyLine
o84 : Ideal of S\\
\endOutput

To illustrate Reeves' method, we show that the incidence graph of the
Hilbert scheme $\Hilb^{4t+1}(\bbbp^{4})$ has diameter at most $2$.
Note that the rational quartic curve\index{rational quartic curve} in
$\bbbp^{4}$ has Hilbert polynomial $4t+1$.
\beginOutput
i85 : m =  matrix table(\{0,1,2\}, \{0,1,2\}, (i,j) -> (gens S)#(i+j))\\
\emptyLine
o85 = | x_0 x_1 x_2 |\\
\      | x_1 x_2 x_3 |\\
\      | x_2 x_3 x_4 |\\
\emptyLine
\              3       3\\
o85 : Matrix S  <--- S\\
\endOutput
\beginOutput
i86 : rationalQuartic = minors(2, m);\\
\emptyLine
o86 : Ideal of S\\
\endOutput
\beginOutput
i87 : H = hilbertPolynomial(S/rationalQuartic);\\
\endOutput
\beginOutput
i88 : hilbertPolynomial(S/rationalQuartic, Projective => false)\\
\emptyLine
o88 = 4\$i + 1\\
\emptyLine
o88 : QQ [\$i]\\
\endOutput
There are $12$ Borel-fixed ideals with Hilbert polynomial $4t+1$; see
Example~1 in Reeves~\cite{MR97g:14003}.
\beginOutput
i89 : L = \{monomialIdeal(x_0^2, x_0*x_1, x_0*x_2, x_1^2, x_1*x_2, x_2^ $\cdot\cdot\cdot$\\
\endOutput
\beginOutput
i90 : scan(#L, i -> << endl << i+1 << " : " << L#i << endl);\\
\emptyLine
\                    2         2               2\\
1 : monomialIdeal (x , x x , x , x x , x x , x )\\
\                    0   0 1   1   0 2   1 2   2\\
\emptyLine
\                    2         2               3\\
2 : monomialIdeal (x , x x , x , x x , x x , x , x x )\\
\                    0   0 1   1   0 2   1 2   2   0 3\\
\emptyLine
\                        2     2   3\\
3 : monomialIdeal (x , x , x x , x , x x x )\\
\                    0   1   1 2   2   1 2 3\\
\emptyLine
\                        2         4   3\\
4 : monomialIdeal (x , x , x x , x , x x )\\
\                    0   1   1 2   2   2 3\\
\emptyLine
\                            5   4 3\\
5 : monomialIdeal (x , x , x , x x )\\
\                    0   1   2   2 3\\
\emptyLine
\                        2         5         4 2\\
6 : monomialIdeal (x , x , x x , x , x x , x x )\\
\                    0   1   1 2   2   1 3   2 3\\
\emptyLine
\                    2         2               5               4\\
7 : monomialIdeal (x , x x , x , x x , x x , x , x x , x x , x x )\\
\                    0   0 1   1   0 2   1 2   2   0 3   1 3   2 3\\
\emptyLine
\                        2         5   4       2\\
8 : monomialIdeal (x , x , x x , x , x x , x x )\\
\                    0   1   1 2   2   2 3   1 3\\
\emptyLine
\                    2         2               4           2\\
9 : monomialIdeal (x , x x , x , x x , x x , x , x x , x x )\\
\                    0   0 1   1   0 2   1 2   2   0 3   1 3\\
\emptyLine
\                         2     2   4             2\\
10 : monomialIdeal (x , x , x x , x , x x x , x x )\\
\                     0   1   1 2   2   1 2 3   1 3\\
\emptyLine
\                         2         4     3\\
11 : monomialIdeal (x , x , x x , x , x x )\\
\                     0   1   1 2   2   1 3\\
\emptyLine
\                             6   5     4 2\\
12 : monomialIdeal (x , x , x , x x , x x )\\
\                     0   1   2   2 3   2 3\\
\endOutput
\beginOutput
i91 : all(L, I -> isBorel I and hilbertPolynomial(S/I) == H)\\
\emptyLine
o91 = true\\
\endOutput
The projection operation partitions the list {\tt L} into $3$ classes:
\beginOutput
i92 : class1 = projection L#0\\
\emptyLine
\                      2         2               2\\
o92 = monomialIdeal (x , x x , x , x x , x x , x )\\
\                      0   0 1   1   0 2   1 2   2\\
\emptyLine
o92 : MonomialIdeal of S\\
\endOutput
\beginOutput
i93 : class2 = projection L#1\\
\emptyLine
\                          2         3\\
o93 = monomialIdeal (x , x , x x , x )\\
\                      0   1   1 2   2\\
\emptyLine
o93 : MonomialIdeal of S\\
\endOutput
\beginOutput
i94 : class3 = projection L#4\\
\emptyLine
\                              4\\
o94 = monomialIdeal (x , x , x )\\
\                      0   1   2\\
\emptyLine
o94 : MonomialIdeal of S\\
\endOutput
\beginOutput
i95 : all(1..3, i -> projection L#i == class2)\\
\emptyLine
o95 = true\\
\endOutput
\beginOutput
i96 : all(4..11, i -> projection L#i == class3)\\
\emptyLine
o96 = true\\
\endOutput
Finally, we use the distraction to connect the classes.
\beginOutput
i97 : all(L, I -> I == monomialIdeal distraction I)\\
\emptyLine
o97 = true\\
\endOutput
\beginOutput
i98 : all(0..3, i -> projection gin distraction L#i == class3)\\
\emptyLine
o98 = true\\
\endOutput
Therefore, the components corresponding to {\tt class1} and {\tt
class2} intersect the one corresponding to {\tt class3}.  Note that
{\tt class3} corresponds to the lexicographic component.

%%----------------------------------------------------------
\section{The Chain Property}


Ho\c{s}ten and Thomas~\cite{MR2000b:13037} recently established that
the initial ideals of a toric ideal have an interesting combinatorial
structure called the chain property.  This structure is on the poset
of associated primes where the partial order is given by inclusion.
Since a monomial ideal $I \subset S = \bbbq[x_{1}, \dotsc, x_{n}]$ is
prime if and only if it is generated by a subset of the variables $\{
x_{1}, \dotsc, x_{n}\}$, the poset of associated primes of $I$ is
contained in the power set of the variables.  We say that a monomial
ideal $I$ has the chain property\index{chain property} if the
following condition holds:
\[
\text{\begin{minipage}[t]{300pt}
For any embedded prime $P_{F} = \langle x_{i} : x_{i} \not\in F
\rangle$ of $I$, there exists an associated prime $P_{G} \subset
P_{F}$ such that $|G| = |F|-1$.
\end{minipage}}
\]
In other words, there is a saturated chain from every embedded prime
to some minimal prime.  
Experimental evidence suggests that, in fact, most initial ideals 
of prime ideals satisfy this saturated chain condition.
Because
of ubiquity and simplicity of this condition, we are interested in
understanding which classes of initial ideals (or more generally
monomial ideals) have the chain property.  

More recently, Miller, Sturmfels and Yanagawa~\cite{MR1769661}
provided a large class of monomial ideals with the chain property.  A
monomial ideal $I$ is called generic\index{generic monomial ideal}
when the following condition holds: if two distinct minimal generators
$r_{1}$ and $r_{2}$ of $I$ have the same positive degree in some
variable $x_{i}$, there is a third generator $r_{3}$ which strictly
divides the least common multiple of $r_{1}$ and $r_{2}$.  In
particular, if no two distinct minimal generators have the same
positive degree in any variable, then the monomial ideal is generic.
Theorem~2.2 in Miller, Sturmfels and Yanagawa~\cite{MR1769661} shows
that generic monomial ideals have the chain property.


%%----------------------------------------------------------
\subsection*{Examples and Counterexamples}


In this final section, we illustrate how to use \Mtwo for further
experimentation and investigation of the chain property.  The
following function determines whether a monomial ideal has the chain
property:
\beginOutput
i99 : hasChainProperty = I -> (\\
\           L := ass I;\\
\           radI := radical I;\\
\           all(L, P -> radI : (radI : P) == P or (\\
\                     gensP := first entries gens P;\\
\                     all(gensP, r -> (\\
\                               Q := monomialIdeal delete(r, gensP);\\
\                               I : (I : Q) == Q)))));\\
\endOutput
Using {\tt hasChainProperty}, we examine the initial ideals of four
interesting classes of ideals related to toric ideals.

\subsubsection*{An Initial Ideal of a Toric Ideal}  As mentioned above, Ho\c{s}ten
and Thomas proved that any initial ideal of a toric ideal satisfies
the saturated chain condition.  The following example demonstrates this
phenomenon.  Consider the matrix $A$:
\beginOutput
i100 : A = matrix\{\{1,1,1,1,1,1,1\}, \{2,0,0,0,1,0,0\}, \{0,2,0,0,0,1,0\}, \{ $\cdot\cdot\cdot$\\
\emptyLine
o100 = | 1 1 1 1 1 1 1 |\\
\       | 2 0 0 0 1 0 0 |\\
\       | 0 2 0 0 0 1 0 |\\
\       | 2 2 0 2 1 1 1 |\\
\emptyLine
\                4        7\\
o100 : Matrix ZZ  <--- ZZ\\
\endOutput
\beginOutput
i101 : IA = toricIdeal(A, \{1,1,1,1,1,1,1\})\\
\emptyLine
\                      2          2          2\\
o101 = ideal (x x  - x , x x  - x , x x  - x )\\
\               3 4    7   2 3    6   1 3    5\\
\emptyLine
o101 : Ideal of S\\
\endOutput
\beginOutput
i102 : inIA = monomialIdeal IA\\
\emptyLine
\                                           2     2     2\\
o102 = monomialIdeal (x x , x x , x x , x x , x x , x x )\\
\                       1 3   2 3   3 4   2 5   4 5   4 6\\
\emptyLine
o102 : MonomialIdeal of S\\
\endOutput
\beginOutput
i103 : hasChainProperty inIA\\
\emptyLine
o103 = true\\
\endOutput

\subsubsection*{An Initial Ideal of a Prime Ideal}  Since toric ideals are
prime, one naturally asks if the initial ideal of any prime ideal has
the chain property.  By modifying the previous example, we can show
that this is not the case.  In particular, making the linear change of
coordinates by $x_{4} \mapsto x_{3}-x_{4}$, we obtain a new prime
ideal $J$.
\beginOutput
i104 : StoS = map(S, S, \{x_1, x_2, x_3, x_3 - x_4, x_5, x_6, x_7\});\\
\emptyLine
o104 : RingMap S <--- S\\
\endOutput
\beginOutput
i105 : J = StoS IA\\
\emptyLine
\               2           2          2          2\\
o105 = ideal (x  - x x  - x , x x  - x , x x  - x )\\
\               3    3 4    7   2 3    6   1 3    5\\
\emptyLine
o105 : Ideal of S\\
\endOutput
Taking the initial ideal with respect to the reverse lexicographic
term order (the default order), we have
\beginOutput
i106 : inJ = monomialIdeal J\\
\emptyLine
\                                   2     2     2       2     2       2 $\cdot\cdot\cdot$\\
o106 = monomialIdeal (x x , x x , x , x x , x x , x x x , x x , x x x  $\cdot\cdot\cdot$\\
\                       1 3   2 3   3   2 5   3 5   1 4 5   3 6   1 4 6 $\cdot\cdot\cdot$\\
\emptyLine
o106 : MonomialIdeal of S\\
\endOutput
\beginOutput
i107 : hasChainProperty inJ\\
\emptyLine
o107 = false\\
\endOutput

\subsubsection*{An $A$-graded Monomial Ideal}  Let $A$ be a $d \times n$ matrix
of nonnegative integers and let $\mathbf{a}_{i}$ denote the $i$-th
column of $A$.  Consider the polynomial ring $S = \bbbq[x_{1}, \dotsc,
x_{n}]$ with the $\bbbz^{d}$-grading defined by $\deg x_{i} =
\mathbf{a}_{i}$.  An ideal $I \subset \bbbq[x_{1}, \dotsc, x_{n}]$ is
called $A$-graded\index{ideal!$A$-graded} provided it is homogeneous with
respect to the $A$-grading and
\[
\dim_{\bbbq} \left( \frac{S}{I} \right)_{\mathbf{b}} = \begin{cases} 1
& \text{if $\mathbf{b} \in \bbbn A$} \\ 0 & \text{otherwise}
\end{cases}
\]
for all $\mathbf{b} \in \bbbn^{d}$.  Remark~10.1 in
Sturmfels~\cite{MR97b:13034} shows that the initial ideal of the toric
ideal $I_{A}$ is $A$-graded.  Altmann~\cite{Altmann} shows that when
$A$ has rank $2$ every $A$-graded monomial ideal has the chain
property.  However, Altmann~\cite{Altmann} also provides a
counterexample when $A$ has rank $3$.  We can verify his example in
\Mtwo as follows:
\beginOutput
i108 : A = matrix\{\{2,0,0,1,0,0,2,1,1,3,2,2,2,3,3,3\},\\
\                  \{0,2,0,0,1,0,1,2,1,2,3,2,3,2,3,3\},\\
\                  \{0,0,2,0,0,1,1,1,2,2,2,3,3,3,2,3\}\};\\
\emptyLine
\                3        16\\
o108 : Matrix ZZ  <--- ZZ\\
\endOutput
In \Mtwo, the first entry in degree vector of each variable must be
positive.  Hence, we append to A the sum of its rows to get a matrix whose
columns will serve as the degrees of the variables.
\beginOutput
i109 : D = A^\{0\}+A^\{1\}+A^\{2\} || A\\
\emptyLine
o109 = | 2 2 2 1 1 1 4 4 4 7 7 7 8 8 8 9 |\\
\       | 2 0 0 1 0 0 2 1 1 3 2 2 2 3 3 3 |\\
\       | 0 2 0 0 1 0 1 2 1 2 3 2 3 2 3 3 |\\
\       | 0 0 2 0 0 1 1 1 2 2 2 3 3 3 2 3 |\\
\emptyLine
\                4        16\\
o109 : Matrix ZZ  <--- ZZ\\
\endOutput
\beginOutput
i110 : D = entries transpose D;\\
\endOutput
\beginOutput
i111 : S = QQ[vars(0..15), Degrees => D, MonomialSize => 16];\\
\endOutput
\beginOutput
i112 : I = monomialIdeal(d*j, d*k, d*l, d*m, d*n, d*o, d*p, e*j, e*k,\\
\           e*l, e*m, e*n, e*o, e*p, f*j, f*k, f*l, f*m, f*n, f*o, f*p,\\
\           g*j, g*k, g*l, g*m, g*n, g*o, g*p, h*j, h*k, h*l, h*m, h*n,\\
\           h*o, h*p, i*j, i*k, i*l, i*m, i*n, i*o, i*p, g^2, g*h, g*i,\\
\           h^2, h*i, i^2, j^2, j*k, j*l, j*m, j*n, j*o, j*p, k^2, k*l,\\
\           k*m, k*n, k*o, k*p, l^2, l*m, l*n, l*o, l*p, m^2, m*n, m*o,\\
\           m*p, n^2, n*o, n*p, o^2, o*p, p^2, d^2, e^2, f^2, d*h, e*i,\\
\           f*g, f*d*i, d*e*g, e*f*h, c*d*g, a*e*h, b*f*i, c*e*g, \\
\           a*f*h, b*d*i, c*d*e, a*e*f, b*f*d, c*b*d, a*c*e, b*a*f, \\
\           c*b*g, a*c*h, b*a*i);\\
\emptyLine
o112 : MonomialIdeal of S\\
\endOutput
To help convince you that $I$ is an $A$-graded ideal, we compute the
$\dim_{\bbbq} \left( \frac{S}{I} \right)_{\mathbf{a}_{i}}$ for $1 \leq
i \leq 16$.
\beginOutput
i113 : apply(D, d -> rank source basis(d, (S^1)/ ideal I))\\
\emptyLine
o113 = \{1, 1, 1, 1, 1, 1, 1, 1, 1, 1, 1, 1, 1, 1, 1, 1\}\\
\emptyLine
o113 : List\\
\endOutput
Finally, we check the chain property.
\beginOutput
i114 : hasChainProperty I\\
\emptyLine
o114 = false\\
\endOutput

\subsubsection*{The Vertex Ideal}  Lastly, we consider a different family of
monomials ideals arising from toric ideals.  The vertex
ideal\index{vertex ideal} $V_{A}$ is defined as intersection all the
monomial initial ideals of the toric ideal $I_{A}$.  Although there
are (in general) infinitely many distinct term orders on a polynomial
ring, an ideal has only finitely many initial ideals; see Theorem~1.2
in Sturmfels~\cite{MR97b:13034}.  In particular, the above
intersection is finite.  Vertex ideals were introduced and studied by
Ho{\c{s}}ten and Maclagan~\cite{Hosten-Maclagan}.  However, the
question ``Does the vertex ideal $V_A$ have the chain property?''
remains open.



% Local Variables:
% mode: latex
% mode: reftex
% tex-main-file: "chapter-wrapper.tex"
% reftex-keep-temporary-buffers: t
% reftex-use-external-file-finders: t
% reftex-external-file-finders: (("tex" . "make FILE=%f find-tex") ("bib" . "make FILE=%f find-bib"))
% End:
\begin{thebibliography}{10}

\bibitem{MR81f:14025a}
Allen~B. Altman and Steven~L. Kleiman:
\newblock Compactifying the {P}icard scheme.
\newblock {\em Adv. in Math.}, 35(1):50--112, 1980.

\bibitem{Altmann}
Klaus Altmann:
\newblock {The chain property for the associated primes of A-graded ideals}.
\newblock {arXiv:math.AG/0004142}.

\bibitem{MR94f:13018}
Dave Bayer and Mike Stillman:
\newblock Computation of {H}ilbert functions.
\newblock {\em J. Symbolic Comput.}, 14(1):31--50, 1992.

\bibitem{MR96h:05213}
Anders Bj{\"o}rner:
\newblock Nonpure shellability, $f$-vectors, subspace arrangements and
  complexity.
\newblock In {\em Formal power series and algebraic combinatorics (New
  Brunswick, NJ, 1994)}, pages 25--53. Amer. Math. Soc., Providence, RI, 1996.

\bibitem{MR89m:52009}
Anders Bj{\"o}rner and Gil Kalai:
\newblock An extended {E}uler-{P}oincar\'e theorem.
\newblock {\em Acta Math.}, 161(3-4):279--303, 1988.

\bibitem{MR95h:13020}
Winfried Bruns and J{\"u}rgen Herzog:
\newblock {\em Cohen-{M}acaulay rings}.
\newblock Cambridge University Press, Cambridge, 1993.

\bibitem{MR97a:13001}
David Eisenbud:
\newblock {\em Commutative algebra with a view toward algebraic geometry}.
\newblock Springer-Verlag, New York, 1995.

\bibitem{MR80g:68056}
Michael~R. Garey and David~S. Johnson:
\newblock {\em Computers and intractability}.
\newblock W. H. Freeman and Co., San Francisco, Calif., 1979.
\newblock A guide to the theory of NP-completeness, A Series of Books in the
  Mathematical Sciences.

\bibitem{MR99m:13040}
Mark~L. Green:
\newblock Generic initial ideals.
\newblock In {\em Six lectures on commutative algebra (Bellaterra, 1996)},
  pages 119--186. Birkh\"auser, Basel, 1998.

\bibitem{MR26:3566}
Alexander Grothendieck:
\newblock {\em Fondements de la g\'eom\'etrie alg\'ebrique. [{E}xtraits du
  {S}\'eminaire {B}ourbaki, 1957--1962.]}.
\newblock Secr\'etariat math\'ematique, Paris, 1962.

\bibitem{MR99g:14031}
Joe Harris and Ian Morrison:
\newblock {\em Moduli of curves}.
\newblock Springer-Verlag, New York, 1998.

\bibitem{MR35:4232}
Robin Hartshorne:
\newblock Connectedness of the {H}ilbert scheme.
\newblock {\em Inst. Hautes \'Etudes Sci. Publ. Math.}, 29:5--48, 1966.

\bibitem{Hosten-Maclagan}
Serkan Ho\c{s}ten and Diane Maclagan:
\newblock The vertex ideal of a lattice.
\newblock 20 pages, (2000), preprint.

\bibitem{MR2000b:13037}
Serkan Ho{\c{s}}ten and Rekha~R. Thomas:
\newblock The associated primes of initial ideals of lattice ideals.
\newblock {\em Math. Res. Lett.}, 6(1):83--97, 1999.

\bibitem{MR2000f:13052}
Serkan Ho{\c{s}}ten and Rekha~R. Thomas:
\newblock Standard pairs and group relaxations in integer programming.
\newblock {\em J. Pure Appl. Algebra}, 139(1-3):133--157, 1999.
\newblock Effective methods in algebraic geometry (Saint-Malo, 1998).

\bibitem{M}
Ezra Miller:
\newblock {Alexander Duality for Monomial Ideals and Their Resolutions}.
\newblock {arXiv:math.AG/9812095}.

\bibitem{MR1779598}
Ezra Miller:
\newblock The {A}lexander duality functors and local duality with monomial
  support.
\newblock {\em J. Algebra}, 231(1):180--234, 2000.

\bibitem{MR1769661}
Ezra Miller, Bernd Sturmfels, and Kohji Yanagawa:
\newblock Generic and cogeneric monomial ideals.
\newblock {\em J. Symbolic Comput.}, 29(4-5):691--708, 2000.
\newblock Symbolic computation in algebra, analysis, and geometry (Berkeley,
  CA, 1998).

\bibitem{MR2000a:13010}
Hidefumi Ohsugi and Takayuki Hibi:
\newblock Normal polytopes arising from finite graphs.
\newblock {\em J. Algebra}, 207(2):409--426, 1998.

\bibitem{MR98m:14003}
Alyson Reeves and Mike Stillman:
\newblock Smoothness of the lexicographic point.
\newblock {\em J. Algebraic Geom.}, 6(2):235--246, 1997.

\bibitem{MR97g:14003}
Alyson~A. Reeves:
\newblock The radius of the {H}ilbert scheme.
\newblock {\em J. Algebraic Geom.}, 4(4):639--657, 1995.

\bibitem{MR1734566}
Mutsumi Saito, Bernd Sturmfels, and Nobuki Takayama:
\newblock {\em Gr\"obner deformations of hypergeometric differential
  equations}.
\newblock Springer-Verlag, Berlin, 2000.

\bibitem{MR99c:13004}
Aron Simis, Wolmer~V. Vasconcelos, and Rafael~H. Villarreal:
\newblock The integral closure of subrings associated to graphs.
\newblock {\em J. Algebra}, 199(1):281--289, 1998.

\bibitem{MR98h:05001}
Richard~P. Stanley:
\newblock {\em Combinatorics and commutative algebra}.
\newblock Birkh\"auser Boston Inc., Boston, MA, second edition, 1996.

\bibitem{MR91m:14076}
Bernd Sturmfels:
\newblock Gr\"obner bases and {S}tanley decompositions of determinantal rings.
\newblock {\em Math. Z.}, 205(1):137--144, 1990.

\bibitem{MR97b:13034}
Bernd Sturmfels:
\newblock {\em Gr\"obner bases and convex polytopes}.
\newblock American Mathematical Society, Providence, RI, 1996.

\bibitem{MR96i:13029}
Bernd Sturmfels, Ng\^o~Vi\^et Trung, and Wolfgang Vogel:
\newblock Bounds on degrees of projective schemes.
\newblock {\em Math. Ann.}, 302(3):417--432, 1995.

\bibitem{MR91b:13031}
Rafael~H. Villarreal:
\newblock Cohen-{M}acaulay graphs.
\newblock {\em Manuscripta Math.}, 66(3):277--293, 1990.

\bibitem{MR96a:52011}
G{\"u}nter~M. Ziegler:
\newblock {\em Lectures on polytopes}.
\newblock Springer-Verlag, New York, 1995.

\end{thebibliography}
\egroup
\makeatletter
\renewcommand\thesection{\@arabic\c@section}
\makeatother



%%%%%%%%%%%%%%%%%%%%%%%%%%%%%%%%%%%%%%%%%%%%%%%%
%%%%%
%%%%% ../chapters/solving/solving
%%%%%
%%%%%%%%%%%%%%%%%%%%%%%%%%%%%%%%%%%%%%%%%%%%%%%%

\bgroup
%solving.tex
%
%   Macaulay 2 Chapter:
%  From Enumerative geometry to solving polynomial systems
%
% Frank Sottile
%
% Begun:                  25 April 1999
% Began new version       27 July
% Preliminary version     27 August
% Started working again    1 July 2000
% Completed & Submitted   20 July 
% arXiv.org/math.AG/0007142  npvkp
%     Title: "An excursion ...(title above)... with Macaulay2" 
% Final revised version   14 November  
% Replaced arXiv version  15 November
% Really Final version    25 January 2000
%
%%%%%%%%%%%%%%%%%%%%%%%%%%%%%%%%%%%%%%%%%%%%%%%%%%%%%%%%%%%%%%
%                                                            %
%   WARNING: * On an AMD K6/2 300MHz w/ 256MB the M2 code    %
%              takes 8:15 minutes to run and uses 82MB       %
%            * On Intel PIII 600MHz w/ 256K cache & 256MB    %
%              the M2 code takes 4:15 minutes                %
%            * On Intel PIII 850MHz w/ 256K cache & 1GB      %
%              the M2 code takes 3:20 minutes                %
%%%%%%%%%%%%%%%%%%%%%%%%%%%%%%%%%%%%%%%%%%%%%%%%%%%%%%%%%%%%%%

\title{From Enumerative Geometry to Solving Systems of Polynomial Equations}

\titlerunning{From Enumerative Geometry to Solving Equations} 
\toctitle{From Enumerative Geometry to Solving Systems of Polynomial Equations} 
\author{Frank Sottile\thanks{Supported in part by NSF grant DMS-0070494.}}
\authorrunning{F. Sottile}
% \institute{Department of Mathematics and Statistics, University of
% Massachusetts, Amherst, MA 01003, USA}
\maketitle

\begin{abstract}
Solving a system of polynomial equations is a ubiquitous problem in
the applications of mathematics. 
Until recently, it has been hopeless to find explicit solutions to such
systems, and mathematics has instead developed deep and
powerful theories about the solutions to polynomial equations.
Enumerative Geometry is concerned with counting the
number of solutions when the polynomials come from a geometric situation and
Intersection Theory\index{intersection theory} gives methods to accomplish the enumeration.

   We use \Mtwo{}\/ to investigate some problems from enumerative geometry,
illustrating some applications of symbolic computation to this important
problem of solving systems of polynomial equations.
Besides enumerating solutions to the resulting polynomial systems, which
include overdetermined, deficient, and improper systems, we address the
important question of real solutions to these geometric problems. 
\end{abstract}

\section{Introduction}
A basic question to ask about a system of polynomial equations is
its number of solutions\index{polynomial equations}.
For this, the fundamental result is the following  
B\'ezout Theorem\index{Bezout's Theorem@B\'ezout's Theorem}.

\begin{theorem}
 The number of isolated solutions to a system of polynomial equations
$$
 f_1(x_1,\ldots,x_n)=f_2(x_1,\ldots,x_n)= \cdots
 =f_n(x_1,\ldots,x_n)=0
$$
 is bounded by $d_1d_2\cdots d_n$, where $d_i:=\deg f_i$.
 If the polynomials are generic, then this bound is attained for 
 solutions in an algebraically closed field\index{field!algebraically closed}.
\end{theorem}

Here, isolated is taken with respect to the algebraic closure.
%There are examples where this result fails without that restriction.
This B\'ezout Theorem is a consequence of the refined B\'ezout Theorem of
Fulton and MacPherson~\cite[\S 1.23]{SO:Fu84a}.

A system of polynomial equations with fewer than this
degree bound or B\'ezout number\index{Bezout number@B\'ezout number} of solutions is called
{\it deficient}\index{polynomial equations!deficient}, 
and there are well-defined classes of deficient systems that satisfy other
bounds.  
For example, fewer monomials lead to fewer solutions, for which polyhedral
bounds~\cite{SO:Bernstein} on the number of solutions are often tighter (and
no weaker than) the B\'ezout number, which applies when
all monomials are present.
When the polynomials come from geometry, determining the
number of solutions is the central problem in enumerative
geometry\index{enumerative geometry}.

Symbolic computation\index{symbolic computation} can help compute the
solutions to a system of equations 
that has only isolated solutions.
In this case, the polynomials generate a zero-dimensional ideal
$I$\index{artinian, {\it see also} ideal,
zero-dimensional}\index{ideal!zero-dimensional}. 
The {\it degree}\index{ideal!degree} of $I$ is $\dim_k k[X]/I$, the dimension of the 
$k$-vector space $k[X]/I$, which is also the
number of standard monomials in any term order.
This degree gives an upper bound on
the number of solutions, which is attained when $I$ is
radical\index{ideal!radical}. 

\begin{example}\label{ex:one}
We illustrate this discussion with an example.
Let $f_1$, $f_2$, $f_3$, and $f_4$ be random quadratic polynomials in the ring  
${\mathbb F}_{101}[y_{11},y_{12},y_{21},y_{22}]$.
%
\beginOutput
i1 : R = ZZ/101[y11, y12, y21, y22];\\
\endOutput
%
\beginOutput
i2 : PolynomialSystem = apply(1..4, i -> \\
\                     random(0, R) + random(1, R) + random(2, R));\\
\endOutput
%
The ideal they generate has dimension 0 and degree $16=2^4$, which is the
B\'ezout number. 
%
\beginOutput
i3 : I = ideal PolynomialSystem;\\
\emptyLine
o3 : Ideal of R\\
\endOutput
%
\beginOutput
i4 : dim I, degree I\\
\emptyLine
o4 = (0, 16)\\
\emptyLine
o4 : Sequence\\
\endOutput
%
If we restrict the monomials which appear in the $f_i$ to be among
$$
  1,\;\ y_{11},\;\ y_{12},\;\ y_{21},\;\ y_{22},\;\  
  y_{11}y_{22},\;\ \mbox{ and }\;\  y_{12}y_{21},
$$
then the ideal they generate again has dimension 0, but its degree is now 4.
%
\beginOutput
i5 : J = ideal (random(R^4, R^7) *  transpose(\\
\             matrix\{\{1, y11, y12, y21, y22, y11*y22, y12*y21\}\}));\\
\emptyLine
o5 : Ideal of R\\
\endOutput
%
\beginOutput
i6 : dim J, degree J\\
\emptyLine
o6 = (0, 4)\\
\emptyLine
o6 : Sequence\\
\endOutput
%
If we further require that the coefficients of the
quadratic terms sum to zero, then the ideal they generate now 
has degree 2.
%
\beginOutput
i7 : K = ideal (random(R^4, R^6) * transpose( \\
\             matrix\{\{1, y11, y12, y21, y22, y11*y22 - y12*y21\}\}));\\
\emptyLine
o7 : Ideal of R\\
\endOutput
%
\beginOutput
i8 : dim K, degree K\\
\emptyLine
o8 = (0, 2)\\
\emptyLine
o8 : Sequence\\
\endOutput
%
In Example~\ref{ex:G22}, we shall see how this last specialization is
geometrically meaningful.
\end{example}


For us, enumerative geometry\index{enumerative geometry} is concerned
with {\sl enumerating geometric figures of some kind having
specified positions with respect to general fixed figures}.
That is, counting the solutions to a geometrically meaningful
system of polynomial equations\index{polynomial equations}.
We use \Mtwo{}\/ to investigate some enumerative geometric
problems\index{enumerative problem} from this point of view.
The problem of enumeration will be solved by computing the
degree\index{ideal!degree} of the 
(0-dimensional) ideal generated by the polynomials.

\section{Solving Systems of Polynomials}
We briefly discuss some aspects of solving systems of polynomial
equations\index{solving polynomial equations}.
For a more complete survey, see the relevant chapters
in~\cite{SO:CCS,SO:CLO92}.

Given an ideal $I$ in a polynomial ring $k[X]$, set 
${\mathcal V}(I):= {\rm Spec}\,k[X]/I$.
When $I$ is generated by the polynomials
$f_1,\ldots,f_N$, ${\mathcal V}(I)$ gives the set of solutions in affine 
space to the system 
\begin{equation}\label{eq:system}
  f_1(X)\ =\ \cdots\ =\ f_N(X)\ =\ 0
\end{equation}
a geometric structure.
These solutions are the {\it roots} of the ideal $I$.
The degree of a zero-dimensional ideal $I$ provides an algebraic count of
its roots.
The degree of its radical counts 
roots in the algebraic closure, ignoring multiplicities.
          
\subsection{Excess Intersection}
Sometimes, only a proper (open) subset of affine space is 
geometrically meaningful, and we want to count only the meaningful roots of
$I$. 
Often the roots ${\mathcal V}(I)$ has positive dimensional components that lie
in the complement of the meaningful subset.
One way to treat this situation of excess or improper intersection is to
saturate\index{saturation} $I$ by a polynomial $f$ vanishing on the extraneous
roots. 
This has the effect of working in $k[X][f^{-1}]$, the coordinate ring of the
complement of ${\mathcal V}(f)$~\cite[Exer.~2.3]{SO:MR97a:13001}.

\begin{example}\label{ex:two}
We illustrate this with an example.
Consider the following ideal in ${\mathbb F}_7[x,y]$.
%
\beginOutput
i9 : R = ZZ/7[y, x, MonomialOrder=>Lex];\\
\endOutput
%
\beginOutput
i10 : I = ideal (y^3*x^2 + 2*y^2*x + 3*x*y,  3*y^2 + x*y - 3*y);\\
\emptyLine
o10 : Ideal of R\\
\endOutput
%
Since the generators have greatest common factor $y$, $I$ defines
finitely many points together with the line $y=0$.
Saturate $I$\/ by the variable $y$ to obtain the ideal $J$ of isolated roots. 
%
\beginOutput
i11 : J = saturate(I, ideal(y))\\
\emptyLine
\              4    3     2\\
o11 = ideal (x  + x  + 3x  + 3x, y - 2x - 1)\\
\emptyLine
o11 : Ideal of R\\
\endOutput
%
The first polynomial factors completely in ${\mathbb F}_7[x]$,
%
\beginOutput
i12 : factor(J_0)\\
\emptyLine
o12 = (x)(x - 2)(x + 2)(x + 1)\\
\emptyLine
o12 : Product\\
\endOutput
%
and so the isolated roots of $I$ are $(2,5),(-1,-1),(0,1)$, and $(-2,-3)$. 
\end{example}

Here, the extraneous roots came from a common factor in both
equations.
A less trivial example of this phenomenon will be seen in
Section~\ref{sec:tangent_lines}. 

\subsection{Elimination, Rationality, and Solving}
Elimination theory\index{elimination theory} can be used to study the
roots of a zero-dimensional ideal $I\subset k[X]$\index{solving polynomial
equations!via elimination}.
A polynomial $h\in k[X]$ defines a map 
$k[y]\rightarrow k[X]$ (by $y\mapsto h$) and a corresponding projection 
$h\colon{\rm Spec}\,k[X]\twoheadrightarrow{\mathbb A}^1$.
The generator $g(y)\in k[y]$ of the 
kernel\index{kernel of a ring map} of the map $k[y]\to k[X]/I$ is called an 
{\it eliminant}\index{eliminant}
and it has the property that ${\mathcal V}(g)=h({\mathcal V}(I))$.
When $h$ is a coordinate function $x_i$, we may consider the eliminant to be
in the polynomial ring $k[x_i]$, and we have 
$\langle g(x_i)\rangle=I\cap k[x_i]$.
The most important result concerning eliminants is the Shape
Lemma\index{Shape Lemma}~\cite{SO:BMMT}. 
\medskip

\noindent{\bf Shape Lemma.} 
{\it
Suppose $h$ is a linear polynomial and $g$ is the corresponding eliminant of
a zero-dimensional ideal $I\subset k[X]$ with $\deg(I)=\deg(g)$. 
Then the roots of $I$ are defined in the splitting
field\index{field!splitting} of $g$ and  
$I$ is radical\index{ideal!radical} if and only if $g$ is square-free.

Suppose further that $h=x_1$ so that $g=g(x_1)$.
Then, in the lexicographic term order 
with $x_1<x_2<\cdots<x_n$, $I$ has a Gr\"obner basis\index{Grobner basis@Gr\"obner basis} of
the form: 
%
\begin{equation}\label{triangular}  
    g(x_1),\ \ x_2-g_2(x_1), \ \ \ldots,\ \ x_n-g_n(x_1)\,,
\end{equation}
%
where $\deg(g)>\deg(g_i)$ for $i=2,\ldots,n$.
}\medskip

When $k$ is infinite and $I$ is radical, an eliminant $g$ given by a generic
linear polynomial $h$ will satisfy $\deg(g)=\deg(I)$.
Enumerative geometry\index{enumerative geometry} counts solutions
when the fixed figures are generic.
We are similarly concerned with the generic situation of 
$\deg(g)=\deg(I)$.
In this case, eliminants provide a useful computational device to study
further questions about the roots of $I$.
For instance, the Shape Lemma holds for the saturated ideal of Example~\ref{ex:two}.
Its eliminant, which is the polynomial {\tt J{\char`\_}0}, factors completely 
over the ground field ${\mathbb F}_7$, so all four solutions are defined
in ${\mathbb F}_7$.
In Section 4.3, we will use eliminants in another way, to show that
an ideal is radical. 

Given a polynomial $h$ in a zero-dimensional ring $k[X]/I$, the
procedure {\tt eliminant(h, k[y])} finds a linear relation modulo $I$ 
among the powers $1, h, h^2, \ldots, h^d$ of $h$ with $d$ minimal
and returns this as a polynomial in $k[y]$.
This procedure is included in the \Mtwo{}\/ package 
{\tt realroots.m2}.
%
\beginOutput
i13 : load "realroots.m2"\\
\endOutput
%
\beginOutput
i14 : code eliminant\\
\emptyLine
o14 = -- realroots.m2:65-80\\
\      eliminant = (h, C) -> (\\
\           Z := C_0;\\
\           A := ring h;\\
\           assert( dim A == 0 );\\
\           F := coefficientRing A;\\
\           assert( isField F );\\
\           assert( F == coefficientRing C );\\
\           B := basis A;\\
\           d := numgens source B;\\
\           M := fold((M, i) -> M || \\
\                     substitute(contract(B, h^(i+1)), F), \\
\                     substitute(contract(B, 1_A), F), \\
\                     flatten subsets(d, d));\\
\           N := ((ker transpose M)).generators;\\
\           P := matrix \{toList apply(0..d, i -> Z^i)\} * N;\\
\                (flatten entries(P))_0)\\
\endOutput
%
Here, {\tt M} is a matrix whose rows are the normal forms of the
powers  $1$, $h$, $h^2$, $\ldots$, $h^d$ of $h$, for $d$ the degree of the ideal.
The columns of the kernel {\tt N} of {\tt transpose M} are a basis of the
linear relations among these powers. 
The matrix {\tt P} converts these relations into polynomials.
Since {\tt N} is in column echelon form, the initial entry of {\tt P} 
is the relation of minimal degree.
(This method is often faster than na\"\i vely computing the kernel of the
map $k[Z]\to A$ given by $Z\mapsto h$, which is implemented by
{\tt  eliminantNaive(h, Z)}.)

Suppose we have an eliminant\index{eliminant} $g(x_1)$ of a zero-dimensional 
ideal  $I\subset k[X]$ with $\deg(g)=\deg(I)$, and we have computed the
lexicographic Gr\"obner basis~(\ref{triangular}). 
Then the roots of $I$ are
%
\begin{equation}\label{tri_roots}
   \{ (\xi_1,g_2(\xi_1), \ldots, g_n(\xi_1))\mid g(\xi_1)=0\}\,.
\end{equation}


Suppose now that $k={\mathbb Q}$ and we seek floating point approximations
for the (complex) roots of $I$.
Following this method, we first compute floating point solutions to
$g(\xi)=0$, which give all the $x_1$-coordinates of the roots of $I$,  and
then use~(\ref{tri_roots}) to find the other coordinates.
The difficulty here is that enough precision may be lost in evaluating
$g_i(\xi_1)$ so that the result is a poor approximation for the other
components $\xi_i$.


\subsection{Solving with Linear Algebra}
We describe another method based upon numerical linear algebra.
When $I\subset k[X]$ is zero-dimensional, $A=k[X]/I$ is a finite-dimensional 
$k$-vector space, and {\it any} Gr\"obner basis for $I$ gives an efficient
algorithm to compute ring operations using linear algebra.
In particular, multiplication by $h\in A$ is a linear transformation 
$m_h:A\to A$ and the command {\tt regularRep(h)} from 
{\tt realroots.m2} gives the matrix of $m_h$ in
terms of the standard basis of $A$.
%
\beginOutput
i15 : code regularRep\\
\emptyLine
o15 = -- realroots.m2:96-100\\
\      regularRep = f -> (\\
\           assert( dim ring f == 0 );\\
\           b := basis ring f;\\
\           k := coefficientRing ring f;\\
\           substitute(contract(transpose b, f*b), k))\\
\endOutput
%

Since the action of $A$ on itself is faithful, the minimal polynomial of 
$m_h$ is the eliminant\index{eliminant} corresponding to $h$.
The procedure {\tt charPoly(h, Z)} in {\tt realroots.m2}
computes the characteristic polynomial 
$\det(Z\cdot\mbox{\it Id} - m_h)$ of $h$.
%
\beginOutput
i16 : code charPoly\\
\emptyLine
o16 = -- realroots.m2:106-113\\
\      charPoly = (h, Z) -> (\\
\           A := ring h;\\
\           F := coefficientRing A;\\
\           S := F[Z];\\
\           Z  = value Z;     \\
\           mh := regularRep(h) ** S;\\
\           Idz := S_0 * id_(S^(numgens source mh));\\
\           det(Idz - mh))\\
\endOutput
%
When this is the minimal polynomial (the situation of the Shape Lemma),
this procedure often computes the eliminant faster than does 
{\tt eliminant}, and for systems of moderate degree, much faster than
na\"\i vely computing the kernel of the map $k[Z]\to A$ given by $Z\mapsto h$.

The eigenvalues and eigenvectors of $m_h$ give another algorithm for finding
the roots of $I$\index{solving polynomial equations!via eigenvectors}.
The engine for this is the following result\index{Stickelberger's Theorem}.
\medskip

\noindent{\bf Stickelberger's Theorem. }
{\it
Let $h\in A$ and $m_h$ be as above.
Then there is a one-to-one correspondence between eigenvectors
${\bf v}_\xi$ of $m_h$ and roots $\xi$ of $I$, the eigenvalue of $m_h$ on 
${\bf v}_\xi$ is the value $h(\xi)$ of $h$ at $\xi$, and the multiplicity
of this eigenvalue (on the eigenvector ${\bf v}_\xi$) is the
multiplicity of the root $\xi$.
}\medskip

Since the linear transformations $m_h$ for $h\in A$ commute, the
eigenvectors ${\bf v}_\xi$ are common to all $m_h$.
Thus we may compute the roots of a zero-dimensional ideal $I\subset k[X]$
by first computing floating-point approximations to the
eigenvectors ${\bf v}_\xi$ of $m_{x_1}$.
Then the root $\xi\ =\ (\xi_1,\ldots,\xi_n)$ of $I$ corresponding to the
eigenvector ${\bf v}_\xi$ has $i$th coordinate satisfying
%
\begin{equation}\label{eigenv}
   m_{x_i}\cdot {\bf v}_\xi\ =\ \xi_i \cdot {\bf v}_\xi\,.
\end{equation}
%
An advantage of this method is that we may use structured numerical linear
algebra after the matrices $m_{x_i}$ are precomputed using exact arithmetic. 
(These matrices are typically sparse and have additional structures which may
be exploited.)
Also, the coordinates $\xi_i$ are {\it linear} functions of the floating
point entries of ${\bf v}_\xi$, which affords greater precision than 
the non-linear evaluations $g_i(\xi_1)$ in the method based upon elimination.
While in principle only one of the $\deg(I)$ components of the vectors
in~(\ref{eigenv}) need be computed, averaging the results from all
components can improve precision.


\subsection{Real Roots}
Determining the real roots of a polynomial system is a challenging problem
with real world applications\index{solving polynomial equations!real solutions}.
When the polynomials come from geometry, this is the main problem of
real enumerative geometry\index{enumerative geometry!real}.
Suppose $k\subset{\mathbb R}$ and $I\subset k[X]$ is zero-dimensional.
If $g$ is an eliminant of $k[X]/I$ 
with $\deg(g)=\deg(I)$, then the real roots of
$g$ are in 1-1 correspondence with the real roots of $I$.
Since there are effective methods for counting the real roots of a univariate
polynomial, eliminants give a na\"\i ve, but useful method for determining the
number of real roots to a polynomial system.
(For some applications of this technique in mathematics,
see~\cite{SO:RS98,SO:So_shap-www,SO:So00b}.) 

The classical symbolic method of Sturm, based upon Sturm sequences,  counts
the number of  real roots of a univariate polynomial in an interval.
When applied to an eliminant satisfying the Shape Lemma, this method counts
the number of real roots of the ideal.
This is implemented in \Mtwo{}\/ via the command
{\tt SturmSequence(f)} of {\tt realroots.m2}
%
\beginOutput
i17 : code SturmSequence\\
\emptyLine
o17 = -- realroots.m2:117-131\\
\      SturmSequence = f -> (\\
\           assert( isPolynomialRing ring f );\\
\           assert( numgens ring f === 1 );\\
\           R := ring f;\\
\           assert( char R == 0 );\\
\           x := R_0;\\
\           n := first degree f;\\
\           c := new MutableList from toList (0 .. n);\\
\           if n >= 0 then (\\
\                c#0 = f;\\
\                if n >= 1 then (\\
\                     c#1 = diff(x,f);\\
\                     scan(2 .. n, i -> c#i = - c#(i-2) {\char`\%} c#(i-1));\\
\                     ));\\
\           toList c)\\
\endOutput
%
The last few lines of {\tt SturmSequence} construct the Sturm
sequence\index{Sturm sequence} of the univariate argument $f$:
This is $(f_0, f_1, f_2,\ldots)$ where $f_0=f$, $f_1=f'$, and 
for $i>1$, $f_i$ is the normal form reduction of $-f_{i-2}$ modulo
$f_{i-1}$.
Given any real number $x$, the {\it variation} of $f$ at $x$ is the number of
changes in sign of the sequence $(f_0(x), f_1(x), f_2(x),\ldots)$ obtained by
evaluating the Sturm sequence of $f$ at $x$.
Then the number of real roots of $f$ over an interval $[x,y]$ is the
difference of the variation of $f$ at $x$ and at $y$.

The \Mtwo{}\/ commands  {\tt numRealSturm} and 
{\tt numPosRoots} (and also {\tt numNegRoots}) use this method to respectively
compute the total number of real roots and the number of positive roots of 
a univariate polynomial. 
%
\beginOutput
i18 : code numRealSturm\\
\emptyLine
o18 = -- realroots.m2:160-163\\
\      numRealSturm = f -> (\\
\           c := SturmSequence f;\\
\           variations (signAtMinusInfinity {\char`\\} c) \\
\               - variations (signAtInfinity {\char`\\} c))\\
\endOutput
%
\beginOutput
i19 : code numPosRoots\\
\emptyLine
o19 = -- realroots.m2:168-171\\
\      numPosRoots = f -> (  \\
\           c := SturmSequence f;\\
\           variations (signAtZero {\char`\\} c) \\
\               - variations (signAtInfinity {\char`\\} c))\\
\endOutput
%
These use the commands {\tt signAt}$*${\tt (f)}, which 
give the sign of ${\tt f}$
at $*$.
(Here, $*$ is one of  {\tt Infinity}, {\tt zero}, or {\tt MinusInfinity}.)
Also {\tt variations(c)} computes the 
number of sign changes in the sequence {\tt c}.
%
\beginOutput
i20 : code variations\\
\emptyLine
o20 = -- realroots.m2:183-191\\
\      variations = c -> (\\
\           n := 0;\\
\           last := 0;\\
\           scan(c, x -> if x != 0 then (\\
\                     if last < 0 and x > 0 or last > 0 \\
\                        and x < 0 then n = n+1;\\
\                     last = x;\\
\                     ));\\
\           n)\\
\endOutput
%


A more sophisticated method to compute the number of real roots which can also
give information about their location uses the rank and
signature\index{bilinear form!signature} of the
symmetric trace form. 
Suppose $I\subset k[X]$ is a zero-dimensional ideal and 
set $A:=k[X]/I$.
For $h\in k[X]$, set $S_h(f,g):={\rm trace}(m_{hfg})$.
It is an easy exercise that $S_h$ is a symmetric bilinear form\index{bilinear
form!symmetric} on $A$.
The procedure {\tt traceForm(h)} in {\tt realroots.m2}
computes this trace form\index{trace form} $S_h$.
%
\beginOutput
i21 : code traceForm\\
\emptyLine
o21 = -- realroots.m2:196-203\\
\      traceForm = h -> (\\
\           assert( dim ring h == 0 );\\
\           b  := basis ring h;\\
\           k  := coefficientRing ring h;\\
\           mm := substitute(contract(transpose b, h * b ** b), k);\\
\           tr := matrix \{apply(first entries b, x ->\\
\                     trace regularRep x)\};\\
\           adjoint(tr * mm, source tr, source tr))\\
\endOutput
%
The value of this construction is the following theorem.

\begin{theorem}[\cite{SO:BW,SO:PRS}]\label{t:PRS}
Suppose $k\subset{\mathbb R}$ and $I$ is a zero-dimensional ideal in
$k[x_1,\ldots,x_n]$ and consider ${\mathcal V}(I)\subset {\mathbb C}^n$. 
Then, for $h\in k[x_1,\ldots,x_n]$, the signature $\sigma(S_h)$ and rank 
$\rho(S_h)$ of the bilinear form $S_h$ satisfy
\begin{eqnarray*}
\sigma(S_h)&=&\#\{a\in{\mathcal V}(I)\cap{\mathbb R}^n:h(a)>0\}
            - \#\{a\in{\mathcal V}(I)\cap{\mathbb R}^n:h(a)<0\}\,\\
\rho(S_h)&=&\#\{a\in{\mathcal V}(I):h(a)\neq0\}\,.
\end{eqnarray*}
\end{theorem}

That is, the rank of $S_h$ counts roots in 
${\mathbb C}^n-{\mathcal V}(h)$, and its signature counts the real roots
weighted by the sign of $h$ (which is $-1$, $0$, or $1$) at each root.
The command {\tt traceFormSignature(h)} in {\tt realroots.m2} returns the
rank and  signature of the trace form $S_h$.
%
\beginOutput
i22 : code traceFormSignature\\
\emptyLine
o22 = -- realroots.m2:208-218\\
\      traceFormSignature = h -> (\\
\           A := ring h;\\
\           assert( dim A == 0 );\\
\           assert( char A == 0 );\\
\           S := QQ[Z];\\
\           TrF := traceForm(h) ** S;\\
\           IdZ := Z * id_(S^(numgens source TrF));\\
\           f := det(TrF - IdZ);\\
\           << "The trace form S_h with h = " << h << \\
\             " has rank " << rank(TrF) << " and signature " << \\
\             numPosRoots(f) - numNegRoots(f) << endl; )\\
\endOutput
%
The \Mtwo{}\/ command {\tt numRealTrace(A)} simply returns the number of
real roots of $I$, given ${\tt A}=k[X]/I$.  
%
\beginOutput
i23 : code numRealTrace\\
\emptyLine
o23 = -- realroots.m2:223-230\\
\      numRealTrace = A -> (\\
\           assert( dim A == 0 );\\
\           assert( char A == 0 );\\
\           S := QQ[Z];\\
\           TrF := traceForm(1_A) ** S;\\
\           IdZ := Z * id_(S^(numgens source TrF));\\
\           f := det(TrF - IdZ);\\
\           numPosRoots(f)-numNegRoots(f))\\
\endOutput
%

\begin{example}
We illustrate these methods on the following polynomial system.
%
\beginOutput
i24 : R = QQ[x, y];\\
\endOutput
%
\beginOutput
i25 : I = ideal (1 - x^2*y + 2*x*y^2,  y - 2*x - x*y + x^2);\\
\emptyLine
o25 : Ideal of R\\
\endOutput
%
The ideal $I$ has dimension zero and degree 5.
%
\beginOutput
i26 : dim I, degree I\\
\emptyLine
o26 = (0, 5)\\
\emptyLine
o26 : Sequence\\
\endOutput
%
We compare the two methods to compute the eliminant of $x$ in 
the ring $R/I$.
%
\beginOutput
i27 : A = R/I;\\
\endOutput
%
\beginOutput
i28 : time g = eliminant(x, QQ[Z])\\
\     -- used 0.09 seconds\\
\emptyLine
\       5     4     3    2\\
o28 = Z  - 5Z  + 6Z  + Z  - 2Z + 1\\
\emptyLine
o28 : QQ [Z]\\
\endOutput
%
\beginOutput
i29 : time g = charPoly(x, Z)\\
\     -- used 0.02 seconds\\
\emptyLine
\       5     4     3    2\\
o29 = Z  - 5Z  + 6Z  + Z  - 2Z + 1\\
\emptyLine
o29 : QQ [Z]\\
\endOutput
%
The eliminant has 3 real roots, which we test in two different ways.
%
\beginOutput
i30 : numRealSturm(g), numRealTrace(A)\\
\emptyLine
o30 = (3, 3)\\
\emptyLine
o30 : Sequence\\
\endOutput
%
We use Theorem~\ref{t:PRS} to isolate these roots in the $x,y$-plane.
%
\beginOutput
i31 : traceFormSignature(x*y);\\
The trace form S_h with h = x*y has rank 5 and signature 3\\
\endOutput
%
Thus all 3 real roots lie in the first and third
quadrants (where $xy>0$).
We isolate these further.
%
\beginOutput
i32 : traceFormSignature(x - 2);\\
The trace form S_h with h = x - 2 has rank 5 and signature 1\\
\endOutput
%
This shows that two roots lie in the first quadrant with $x>2$ and one lies
in the third.
Finally, one of the roots lies in the triangle $y>0$, $x>2$, and $x+y<3$.
%
\beginOutput
i33 : traceFormSignature(x + y - 3);\\
The trace form S_h with h = x + y - 3 has rank 5 and signature -1\\
\endOutput

Figure~\ref{fig:roots} shows these three roots (dots), as well as the
lines $x+y=3$ and $x=2$.
\begin{figure}
$$
  %  40 pt = 1 unit
  %  
  %
  \setlength{\unitlength}{0.8pt}
  \begin{picture}(220,110)(-60,-50)
  % y - axis
   \put(0,-50){\vector(0,1){110}} \put(0,0){\vector(0,-1){50}}
   \put(-11,47){$y$}
   \put(-5,-40){\line(1,0){10}} \put(6,-42.8){$-1$}
   \put(-5,40){\line(1,0){10}}  \put(6, 36.5){$1$}
  % \put(-5,80){\line(1,0){10}}  \put(6, 77){$2$}
  % x - axis
   \put(-60,0){\vector(1,0){220}} \put(0,0){\vector(-1,0){60}}
   \put(-40,-5){\line(0,1){10}} \put(-50,-15){$-1$}
   \put(40,-5){\line(0,1){10}}  \put(37.5,-15){$1$}
   \put(120,-5){\line(0,1){10}} \put(117,-15){$3$}
   \put(-51,5){$x$}
  
  \thicklines
  \put(80,-50){\line(0,1){110}}  \put(85,-35){$x=2$}
  \put(160,-40){\line(-1,1){100}}%\put(15,50){$x+y=3$}
  \put(100,25){$x+y=3$}
  
  \put(-26.2, -42.1){\circle*{2}}
  \put(86.3, 11.8){\circle*{2}}
  \put(112.8, 50.8){\circle*{2}}
  \end{picture}
$$
\caption{Location of roots\label{fig:roots}}
\end{figure}
\end{example}


\subsection{Homotopy Methods}
We describe symbolic-numeric 
{\it homotopy continuation methods}\index{homotopy continuation} 
for finding approximate complex solutions to a system of
equations\index{solving polynomial equations!via numerical homotopy}.
These exploit the traditional principles of conservation of number and 
specialization from enumerative geometry\index{enumerative geometry}.

Suppose we seek the isolated solutions of a system $F(X)=0$
where $F=(f_1,\ldots,f_n)$ are polynomials in the variables
$X=(x_1,\ldots,x_N)$.
First, a {\em homotopy} $H(X,t)$ is found with the following properties:
\begin{enumerate}
  \item $H(X,1)= F(X)$. 
  \item The isolated solutions of the {\it start system} $H(X,0)=0$ are known.
  \item The system $H(X,t)=0$ defines finitely many (complex) curves, 
        and each isolated solution of the original system $F(X)=0$ is
        connected to an isolated solution $\sigma_i(0)$ of $H(X,0)=0$ along
        one of these curves. 
\end{enumerate}
Next, choose a generic smooth path $\gamma(t)$ from 0 to 1 in the complex
plane.
Lifting $\gamma$ to the curves $H(X,t)=0$ gives 
smooth paths $\sigma_i(t)$ connecting each solution
$\sigma_i(0)$ of the start system to a solution of the original system.
The path $\gamma$ must avoid the finitely many points in ${\mathbb C}$ over
which the curves are singular or meet other components of the solution set
$H(X,t)=0$.

Numerical path continuation is used to trace each path
$\sigma_i(t)$ from $t=0$ to $t=1$.
When there are fewer solutions to $F(X)=0$ than to 
$H(X,0)=0$, some paths will diverge or become singular as
$t\rightarrow 1$, and it is expensive to trace such a path.
The homotopy is {\it optimal}\index{homotopy!optimal} when this does not
occur. 

When $N=n$ and the $f_i$ are generic, set
$G(X):=(g_1,\ldots,g_n)$ with $g_i=(x_i-1)(x_i-2)\cdots(x_i-d_i)$
where $d_i:=\deg(f_i)$.
Then the {\it B\'ezout homotopy}\index{homotopy!B\'ezout} 
$$
  H(X,t)\quad :=\quad tF(X)\ +\ 
  (1-t)G(X)
$$
is optimal.
This homotopy furnishes an effective demonstration of
the bound in B\'ezout's Theorem\index{Bezout's Theorem@B\'ezout's Theorem} for the number of
solutions to $F(X)=0$.

When the polynomial system is deficient, the B\'ezout homotopy is not optimal.
When $n>N$ (often the case in geometric examples), 
the B\'ezout homotopy  does not apply.
In either case, a different strategy is needed.
Present optimal homotopies for such systems all exploit some structure of
the systems they are designed to solve.
The current state-of-the-art is described in~\cite{SO:Ver99}.

\begin{example}\label{example:Groebner}
The Gr\"obner homotopy\index{homotopy!Gr\"obner}~\cite{SO:HSS} is an optimal 
homotopy\index{homotopy!optimal} that exploits a square-free initial
ideal\index{initial ideal!square-free}.
Suppose our system has the form
$$
  F\ :=\ g_1(X),\ldots,g_m(X),\ \Lambda_1(X),\ldots,\Lambda_d(X)
$$
where $g_1(X),\ldots,g_m(X)$ form a Gr\"obner basis for an ideal $I$ 
with respect to a given term order $\prec$, $\Lambda_1,\ldots,\Lambda_d$ are
linear forms with $d=\dim({\mathcal V}(I))$, {\it and}\/ we assume that 
the initial ideal ${\rm in}_\prec I$ is square-free.
This last, restrictive, hypothesis occurs for
certain determinantal varieties\index{determinantal variety}.

As in~\cite[Chapter 15]{SO:MR97a:13001}, there exist polynomials
$g_i(X,t)$ interpolating between $g_i(X)$ and their initial terms
${\rm in}_\prec g_i(X)$
$$
  g_i(X;1)\ =\ g_i(X) \qquad\mbox{and}\qquad
  g_i(X;0) \ =\ {\rm in}_\prec g_i(X)
$$
so that $\langle g_1(X,t),\ldots,g_m(X,t)\rangle$ is a flat family
with generic fibre isomorphic to $I$ and special fibre 
${\rm in}_\prec I$.
The {\it Gr\"obner homotopy} is
$$
  H(X,t)\ :=\ 
  g_1(X,t),\ldots,g_m(X,t),\ \Lambda_1(X),\ldots,\Lambda_d(X).
$$
Since ${\rm in}_\prec I$ is square-free, 
${\mathcal V}({\rm in}_\prec I)$ is a union of 
$\deg(I)$-many coordinate $d$-planes.
We solve the start system by linear algebra.
This conceptually simple homotopy is in general not
efficient as it is typically overdetermined.
\end{example}

\section{Some Enumerative Geometry}\label{sec:enumerative}

We use the tools we have developed to explore the enumerative geometric
problems of cylinders meeting 5 general points and lines tangent to
4 spheres\index{enumerative geometry}\index{enumerative problem}. 

\subsection{Cylinders Meeting 5 Points}\label{sec:cylinder}
A {\it cylinder}\index{cylinder} is the locus of points equidistant from a
fixed line in ${\mathbb R}^3$. 
The Grassmannian\index{Grassmannian} of lines in 3-space is 4-dimensional,
which implies that 
the space of cylinders is 5-dimensional, and so we expect that 5 points in  
${\mathbb R}^3$ will determine finitely many cylinders.
That is, there should be finitely many lines equidistant from 5 general points.
The question is: How many cylinders/lines, and how many of them can be real?

Bottema and Veldkamp~\cite{SO:BV77}
show there are 6 {\it complex} cylinders 
and Lichtblau~\cite{SO:Li00} observes that if the 5
points are the vertices of 
a bipyramid consisting of 2 regular tetrahedra sharing a common face, then
all 6 will be real.
We check this reality on a configuration with less symmetry (so the Shape
Lemma holds).

If the axial line has direction ${\bf V}$ and contains the point ${\bf P}$
(and hence has parameterization ${\bf P}+t{\bf V}$), and if $r$ is the squared
radius, then the cylinder\index{cylinder} is the set of points ${\bf X}$
satisfying 
$$
  0 \ =\ r - 
  \left\| {\bf X} - {\bf P} - \frac{{\bf V}\cdot({\bf X} - {\bf P})}%
     {\|{\bf V}\|^2}\,{\bf V} \right\|^2\ .
$$
Expanding and clearing the denominator of $\|{\bf V}\|^2$ yields
%
\begin{equation}\label{eq:cylinder}
  0 \ =\ r \|{\bf V}\|^2 + 
         [{\bf V}\cdot({\bf X} - {\bf P})]^2 - 
        \|{\bf X} - {\bf P}\|^2\, \|{\bf V}\|^2\,.
\end{equation}
%
We consider cylinders containing the following 5 points, which form an
asymmetric bipyramid.
%
\beginOutput
i34 : Points = \{\{2, 2,  0 \}, \{1, -2,  0\}, \{-3, 0, 0\}, \\
\                \{0, 0, 5/2\}, \{0,  0, -3\}\};\\
\endOutput
%
Suppose that ${\bf P}=(0,y_{11},y_{12})$ and ${\bf V}=(1,y_{21},y_{22})$.
%
\beginOutput
i35 : R = QQ[r, y11, y12, y21, y22];\\
\endOutput
%
\beginOutput
i36 : P = matrix\{\{0, y11, y12\}\};\\
\emptyLine
\              1       3\\
o36 : Matrix R  <--- R\\
\endOutput
%
\beginOutput
i37 : V = matrix\{\{1, y21, y22\}\};\\
\emptyLine
\              1       3\\
o37 : Matrix R  <--- R\\
\endOutput
%
We construct the ideal given by evaluating the
polynomial~(\ref{eq:cylinder}) at each of the five points.
%
\beginOutput
i38 : Points = matrix Points ** R;\\
\emptyLine
\              5       3\\
o38 : Matrix R  <--- R\\
\endOutput
%
%
\beginOutput
i39 : I = ideal apply(0..4, i -> (\\
\                X := Points^\{i\};\\
\                r * (V * transpose V)  +\\
\                 ((X - P) * transpose V)^2) -\\
\                 ((X - P) * transpose(X - P)) * (V * transpose V)\\
\                );\\
\emptyLine
o39 : Ideal of R\\
\endOutput
%
This ideal has dimension 0 and degree 6.
%
\beginOutput
i40 : dim I, degree I\\
\emptyLine
o40 = (0, 6)\\
\emptyLine
o40 : Sequence\\
\endOutput
%
There are 6 real roots, and they correspond to real cylinders (with $r>0$).
%
\beginOutput
i41 : A = R/I; numPosRoots(charPoly(r, Z))\\
\emptyLine
o42 = 6\\
\endOutput
%

\subsection{Lines Tangent to 4 Spheres}\label{sec:12lines}
We now ask for the lines having a fixed distance from 4 general points.
Equivalently, these are the lines mutually tangent to 4 spheres\index{sphere} of equal radius.
Since the Grassmannian\index{Grassmannian} of lines is four-dimensional, we
expect there to be only finitely many such lines.
Macdonald, Pach, and Theobald~\cite{SO:MPT00} show that there
are indeed 12 lines, and that all 12 may be real.
This problem makes geometric sense over any field $k$ not of characteristic
2, and the derivation of the number 12 is also valid for algebraically
closed\index{field!algebraically closed} 
fields not of characteristic 2.

A sphere in $k^3$ is given by ${\mathcal V}(q(1,{\bf x}))$, where 
$q$ is some quadratic form\index{quadratic form} on $k^4$. 
Here ${\bf x}\in k^3$ and we note that not all quadratic forms give spheres.
If our field does not have characteristic 2, then there 
is a symmetric $4\times 4$ matrix $M$ such that 
$q({\bf u})={\bf u}M{\bf u}^t$.

A line $\ell$ having direction ${\bf V}$ and containing the point ${\bf P}$ 
is tangent to the sphere defined by $q$ when the univariate polynomial in $s$ 
$$
  q( (1,{\bf P})+s(0,{\bf V}) )\ =\ 
  q(1,{\bf P}) + 2s (1,{\bf P})M (0,{\bf V})^t + s^2q(0,{\bf V})\,,
$$
has a double root.
Thus its discriminant\index{discriminant} vanishes, giving the equation
%
\begin{equation}\label{eq:sphere}
  \left( (1,{\bf P})M(0,{\bf V})^t\right)^2 \ -\ 
   (1,{\bf P})M (1,{\bf P})^t\cdot(0,{\bf V})M (0,{\bf V})^t 
     \ =\ 0\,.
\end{equation}
%

The matrix $M$ of the quadratic form $q$ of the sphere with
center $(a,b,c)$ and squared radius $r$ is constructed by 
{\tt Sphere(a,b,c,r)}.
%
\beginOutput
i43 : Sphere = (a, b, c, r) -> (\\
\              matrix\{\{a^2 + b^2 + c^2 - r ,-a ,-b ,-c \},\\
\                     \{         -a         , 1 , 0 , 0 \},\\
\                     \{         -b         , 0 , 1 , 0 \},\\
\                     \{         -c         , 0 , 0 , 1 \}\}\\
\              );\\
\endOutput
%
If a line $\ell$ contains the point ${\bf P}=(0,y_{11},y_{12})$ 
and $\ell$ has direction ${\bf V} = (1,y_{21},y_{22})$, then 
{\tt tangentTo(M)} is the equation for $\ell$ to be tangent to the 
quadric $uMu^T=0$ determined by the matrix $M$.
%
\beginOutput
i44 : R = QQ[y11, y12, y21, y22];\\
\endOutput
%
\beginOutput
i45 : tangentTo = (M) -> (\\
\           P := matrix\{\{1, 0, y11, y12\}\};\\
\           V := matrix\{\{0, 1, y21, y22\}\};\\
\           (P * M * transpose V)^2 - \\
\             (P * M * transpose P) * (V * M * transpose V)\\
\           );\\
\endOutput
The ideal of lines having distance $\sqrt{5}$ from the four points
$(0,0,0)$, $(4,1,1)$, $(1,4,1)$, and $(1,1,4)$ has dimension zero and degree 12.
%
\beginOutput
i46 : I = ideal (tangentTo(Sphere(0,0,0,5)), \\
\                 tangentTo(Sphere(4,1,1,5)), \\
\                 tangentTo(Sphere(1,4,1,5)), \\
\                 tangentTo(Sphere(1,1,4,5)));\\
\emptyLine
o46 : Ideal of R\\
\endOutput
%
\beginOutput
i47 : dim I, degree I\\
\emptyLine
o47 = (0, 12)\\
\emptyLine
o47 : Sequence\\
\endOutput
%
Thus there are 12 lines whose distance from those 4 points is $\sqrt{5}$.
We check that all 12 are real.
%
\beginOutput
i48 : A = R/I;\\
\endOutput
%
\beginOutput
i49 : numRealSturm(eliminant(y11 - y12 + y21 + y22, QQ[Z]))\\
\emptyLine
o49 = 12\\
\endOutput
%
Since no eliminant\index{eliminant} given by a coordinate function satisfies
the hypotheses of the Shape Lemma, 
we took the eliminant with respect to the linear form
$y_{11} - y_{12} + y_{21} + y_{22}$.

This example is an instance of Lemma~3 of~\cite{SO:MPT00}.
These four points define a regular tetrahedron with volume
$V=9$ where each face has area $A=\sqrt{3^5}/2$ and each edge has length
$e=\sqrt{18}$.
That result guarantees that all 12 lines will be real when 
$e/2<r<A^2/3V$, which is the case above.


\section{Schubert Calculus}
The classical Schubert calculus\index{Schubert calculus} of enumerative
geometry\index{enumerative geometry} concerns linear subspaces having
specified positions with respect to other, fixed subspaces. 
For instance, how many lines in ${\mathbb P}^3$ meet four given
lines? (See Example~\ref{ex:G22}.) 
More generally, let $1<r<n$ and suppose that we are given general linear
subspaces $L_1,\ldots,L_m$ of $k^n$ with $\dim L_i=n-r+1-l_i$.
When $l_1+\cdots+l_m=r(n-r)$, there will be a finite number
$d(r,n;l_1,\ldots,l_m)$ of $r$-planes in
$k^n$ which meet each $L_i$ non-trivially. 
This number may be computed using classical algorithms of Schubert and Pieri
(see~\cite{SO:MR48:2152}).


The condition on $r$-planes to meet a fixed $(n{-}r{+}1{-}l)$-plane
non-trivially is called a {\it (special) Schubert condition}, and we call 
the data $(r,n;l_1,\ldots,l_m)$ {\it (special) Schubert data}.
The {\it (special) Schubert calculus} concerns this class of enumerative
problems\index{Schubert calculus}\index{enumerative problem}.
We give two polynomial formulations of this special Schubert calculus,
consider their solutions over ${\mathbb R}$, and end with a question for
fields of arbitrary characteristic.

\subsection{Equations for the Grassmannian}\label{sec:grass}
The ambient space for the Schubert calculus is the
Grassmannian\index{Grassmannian}  
of $r$-planes in $k^n$, denoted ${\bf G}_{r,n}$.
For $H\in{\bf G}_{r,n}$, the $r$th exterior product of the embedding 
$H \rightarrow k^n$ gives a line
$$
  k\ \simeq\ \wedge^r H\ \longrightarrow\ \wedge^r k^{n}\ \simeq\ 
  k^{\binom{n}{r}}\,.
$$
This induces the Pl\"ucker embedding\index{Plucker embedding@Pl\"ucker embedding}  
${\bf G}_{r,n}\hookrightarrow{\mathbb P}^{\binom{n}{r}-1}$.
If $H$ is the row space of an $r$ by $n$ matrix, also written $H$, then the 
Pl\"ucker embedding sends $H$ to its vector of $\binom{n}{r}$ maximal
minors.
Thus the $r$-subsets of $\{0,\ldots,n{-}1\}$, 
${\mathbb Y}_{r,n}:={\tt subsets(n,r)}$,  
index Pl\"ucker coordinates\index{Plucker coordinate@Pl\"ucker coordinate} of ${\bf G}_{r,n}$.
The Pl\"ucker ideal\index{Plucker ideal@Pl\"ucker ideal} of ${\bf G}_{r,n}$ is therefore the
ideal of algebraic relations among the maximal minors of a generic $r$ by $n$
matrix. 

We create the coordinate ring 
$k[p_\alpha\mid\alpha\in{\mathbb Y}_{2,5}]$ of ${\mathbb P}^9$ and the
Pl\"ucker ideal of ${\bf G}_{2,5}$.
The Grassmannian ${\bf G}_{r,n}$ of $r$-dimensional subspaces of $k^n$ is also
the Grassmannian of $r{-}1$-dimensional affine subspaces of 
${\mathbb P}^{n-1}$.
\Mtwo{} uses this alternative indexing scheme.
%
\beginOutput
i50 : R = ZZ/101[apply(subsets(5,2), i -> p_i )];\\
\endOutput
%
\beginOutput
i51 : I = Grassmannian(1, 4, R)\\
\emptyLine
o51 = ideal (p      p       - p      p       + p      p      , p       $\cdot\cdot\cdot$\\
\              \{2, 3\} \{1, 4\}    \{1, 3\} \{2, 4\}    \{1, 2\} \{3, 4\}   \{2, 3\} $\cdot\cdot\cdot$\\
\emptyLine
o51 : Ideal of R\\
\endOutput
%
This projective variety has dimension 6 and degree 5
%
\beginOutput
i52 : dim(Proj(R/I)), degree(I)\\
\emptyLine
o52 = (6, 5)\\
\emptyLine
o52 : Sequence\\
\endOutput
%

This ideal has an important combinatorial 
structure~\cite[Example 11.9]{SO:Sturmfels_GBCP}. 
We write each $\alpha\in{\mathbb Y}_{r,n}$ as an increasing
sequence $\alpha\colon\alpha_1<\cdots<\alpha_r$. 
Given $\alpha,\beta\in{\mathbb Y}_{r,n}$, consider the 
two-rowed array with $\alpha$ written above $\beta$.
We say $\alpha\leq \beta$ if each column weakly increases.
If we sort the columns of an array with rows $\alpha$ and
$\beta$, then the first row is the {\it meet} $\alpha\wedge\beta$ 
(greatest lower bound) and the
second row the {\it join} $\alpha\vee\beta$ (least upper bound) of $\alpha$
and $\beta$. 
These definitions endow ${\mathbb Y}_{r,n}$ with the structure of a
distributive lattice.
Figure~\ref{fig2} shows ${\mathbb Y}_{2,5}$.
\begin{figure}
$$
 \epsfysize=1.8in \epsfbox{Y25.eps}
$$\caption{${\mathbb Y}_{2,5}$\label{fig2}}
\end{figure}

We give $k[p_\alpha]$ the degree reverse
lexicographic order, where we first order the variables $p_\alpha$ by
lexicographic order on their indices $\alpha$.

\begin{theorem}\label{PluckerIdeal}
The reduced Gr\"obner basis\index{Grobner basis@Gr\"obner basis!reduced} of the Pl\"ucker
ideal with respect to this degree 
reverse lexicographic term order consists of quadratic
polynomials 
$$
  g(\alpha,\beta)\quad=\quad
  p_\alpha\cdot p_\beta \ -\  p_{\alpha\vee\beta}\cdot p_{\alpha\wedge\beta} 
  \ +\ \hbox{lower terms in $\prec$}\,,
$$
for each incomparable pair $\alpha,\beta$ in
${\mathbb Y}_{r,n}$,
and all lower terms $\lambda p_\gamma\cdot p_\delta$ in $g(\alpha,\beta)$
satisfy $\gamma\leq \alpha\wedge\beta$ and $\alpha\vee\beta\leq \delta$.
\end{theorem}

The form of this Gr\"obner basis implies that the standard monomials are 
the sortable monomials, those $p_\alpha p_\beta\cdots p_\gamma$ with 
$\alpha\leq\beta\leq\cdots\leq\gamma$.
Thus the Hilbert function\index{Hilbert function} of ${\bf G}_{r,n}$ may be
expressed in terms of the combinatorics of ${\mathbb Y}_{r,n}$.
For instance, the dimension of ${\bf G}_{r,n}$ is the rank of 
${\mathbb Y}_{r,n}$, and its degree is the number of maximal chains.
From Figure~\ref{fig2}, these are 6 and 5 for ${\mathbb Y}_{2,5}$,
confirming our previous calculations.

Since the generators $g(\alpha,\beta)$ are linearly independent, this
Gr\"obner basis is also a minimal generating set for the ideal.
The displayed generator in {\tt o51}, 
$$
  p_{\{2,3\}}p_{\{1,4\}}\ -\ p_{\{1,3\}}p_{\{2,4\}}\ -\
  p_{\{1,2\}}p_{\{3,4\}}\ ,
$$
is $g(23, 14)$, and corresponds to the underlined incomparable pair in
Figure~\ref{fig2}. 
Since there are 5 such incomparable pairs, the Gr\"obner basis has 5
generators.
As ${\bf G}_{2,5}$ has codimension 3, it is not a complete
intersection\index{Grassmannian!not a complete intersection}\index{complete intersection}. 
This shows how the general enumerative problem from the Schubert calculus
gives rise to an overdetermined system of equations\index{polynomial
equations!overdetermined} in this global 
formulation. 
\medskip

The Grassmannian\index{Grassmannian!local coordinates} has a useful system of
local coordinates given by ${\rm Mat}_{r,n-r}$ as follows
%
\begin{equation}\label{eq:local}
  Y\ \in {\rm Mat}_{r,n-r}\ \longmapsto\ 
  {\rm row space}\ [ I_r : Y ]\ \in\ {\bf G}_{r,n}\,.
\end{equation}
%

Let $L$ be a ($n-r+1-l$)-plane in $k^n$ which is the row space of
a $n-r+1-l$ by $n$ matrix, also written $L$.
Then $L$ meets $X\in{\bf G}_{r,n}$ non-trivially if
$$
  \mbox{maximal minors of }\ 
  \left[\begin{array}{c}L\\X\end{array}\right]\ =\ 0\,.
$$
Laplace expansion of each minor along the rows of $X$ gives a linear
equation in the Pl\"ucker coordinates.
In the local coordinates (substituting $[I_r:Y]$ for $X$), we obtain
multilinear equations of degree $\min\{r,n-r\}$.
These equations generate a prime ideal of codimension $l$.

Suppose each $l_i=1$ in our enumerative problem.
Then in the Pl\"ucker coordinates, we have the Pl\"ucker ideal of 
${\bf G}_{r,n}$
together with $r(n-r)$ linear equations, one for each
$(n{-}r)$-plane $L_i$.
By Theorem~\ref{PluckerIdeal}, the Pl\"ucker ideal has a square-free initial
ideal\index{initial ideal!square-free}, and so the Gr\"obner
homotopy\index{homotopy!Gr\"obner} of 
Example~\ref{example:Groebner} may be 
used to solve this enumerative problem.

\begin{example}\label{ex:G22}
${\bf G}_{2,4}\subset{\mathbb P}^5$ has equation
%
\begin{equation}\label{eq:G22}
   p_{\{1,2\}}p_{\{0,3\}}-p_{\{1,3\}}p_{\{0,2\}}+ p_{\{2,3\}}p_{\{0,1\}}
   \ =\ 0\,.
\end{equation}
%
The condition for $H\in{\bf G}_{2,4}$ to meet a 2-plane $L$ is the
vanishing of 
\begin{equation}\label{eq:hypersurface}
    p_{\{1,2\}}L_{34}-p_{\{1,3\}}L_{24}+p_{\{2,3\}}L_{14} 
  + p_{\{1,4\}}L_{23}-p_{\{2,4\}}L_{13}+p_{\{3,4\}}L_{12}\,,
\end{equation}
where $L_{ij}$ is the $(i,j)$th maximal minor of $L$.

If $l_1=\cdots=l_4=1$, we have 5 equations in ${\mathbb P}^5$, one quadratic
and 4 linear, and so by B\'ezout's Theorem\index{Bezout's Theorem@B\'ezout's Theorem} there are
two 2-planes in $k^4$ that meet 4 general 2-planes non-trivially. 
This means that there are 2 lines in ${\mathbb P}^3$ meeting 4 general lines. 
In local coordinates, (\ref{eq:hypersurface}) becomes
$$
    L_{34}-L_{14}y_{11}+L_{13}y_{12}-L_{24}y_{21} 
  + L_{23}y_{22} + L_{12}(y_{11}y_{22}-y_{12}y_{21})\,.
$$
This polynomial has the form of the last specialization in
Example~\ref{ex:one}. 
\end{example}


\subsection{Reality in the Schubert Calculus}\label{sec:shapiro}
Like the other enumerative problems we have discussed, enumerative problems
in the special Schubert calculus\index{Schubert calculus} are fully
real\index{enumerative problem!fully real} in that all solutions can be 
real~\cite{SO:So99a}. 
That is, given any Schubert data $(r,n;l_1,\ldots,l_m)$, there exist 
subspaces $L_1,\ldots,L_m\subset{\mathbb R}^n$ such that each of the
$d(r,n;l_1,\ldots,l_m)$ $r$-planes that meet each $L_i$ are themselves
real.

This result gives some idea of which choices of the $L_i$ give
all $r$-planes real.
Let $\gamma$ be a fixed rational normal curve in ${\mathbb R}^n$.
Then the $L_i$ are linear subspaces osculating $\gamma$.
More concretely, suppose that $\gamma$ is the standard rational normal
curve\index{rational normal curve},
$\gamma(s) = (1, s, s^2, \ldots, s^{n-1})$.
Then the $i$-plane 
$L_i(s):=\langle \gamma(s),\gamma'(s),\ldots,\gamma^{(i-1)}(s)\rangle$ 
osculating $\gamma$ at $\gamma(s)$ is the row space
of the matrix given by {\tt oscPlane(i, n, s)}.
%
\beginOutput
i53 : oscPlane = (i, n, s) -> (\\
\           gamma := matrix \{toList apply(1..n, i -> s^(i-1))\};\\
\           L := gamma;\\
\           j := 0;\\
\           while j < i-1 do (gamma = diff(s, gamma); \\
\                L = L || gamma;\\
\                j = j+1);\\
\           L);\\
\endOutput
%
\beginOutput
i54 : QQ[s]; oscPlane(3, 6, s)\\
\emptyLine
o55 = | 1 s s2 s3  s4   s5   |\\
\      | 0 1 2s 3s2 4s3  5s4  |\\
\      | 0 0 2  6s  12s2 20s3 |\\
\emptyLine
\                   3            6\\
o55 : Matrix QQ [s]  <--- QQ [s]\\
\endOutput
%
(In {\tt o55}, the exponents of $s$ are displayed in line: $s^2$ is written
{\tt s2}.
\Mtwo{} uses this notational convention to display
matrices efficiently.) 


\begin{theorem}[\cite{SO:So99a}]\label{thm:special-reality}
For any Schubert data $(r,n;l_1,\ldots,l_m)$, {\bf there exist} real numbers
$s_1,s_2,\ldots,s_m$ such that there are $d(r,n;l_1,\ldots,l_m)$
$r$-planes that meet each osculating plane $L_i(s_i)$, and all are real.
\end{theorem}

The inspiration for looking at subspaces osculating the rational normal
curve\index{rational normal curve} to
study real enumerative geometry\index{enumerative geometry!real} for the
Schubert calculus\index{Schubert calculus} is the following very interesting 
conjecture of Boris Shapiro and Michael
Shapiro, or more accurately,
extensive computer experimentation based upon their
conjecture~\cite{SO:RS98,SO:So_shap-www,SO:So00b,SO:Ver00}.
\medskip

\noindent{\bf Shapiros's Conjecture\index{Shapiros's Conjecture}. }
{\it 
For any Schubert data $(r,n;l_1,\ldots,l_m)$ and {\bf for all} real numbers
$s_1,s_2,\ldots,s_m$ there are $d(r,n;l_1,\ldots,l_m)$
$r$-planes that meet each osculating plane $L_i(s_i)$, and all are real.
}\medskip

In addition to Theorem~\ref{thm:special-reality}, (which replaces the 
quantifier {\it for all}\/ by  {\it there exist}), the strongest evidence for
this Conjecture is the following result of Eremenko and
Gabrielov~\cite{SO:EG00}. 

\begin{theorem}
Shapiros's Conjecture is true when either $r$ or $n-r$ is $2$.
\end{theorem}

We test an example of this conjecture for the Schubert data
$(3,6;1^3,2^3)$, (where $a^b$ is $a$ repeated $b$ times).
The algorithms of the Schubert calculus predict that $d(3,6;1^3,2^3)=6$.
The function {\tt spSchub(r, L, P)} computes the ideal of $r$-planes meeting
the row space of $L$ in the Pl\"ucker coordinates $P_\alpha$.
%
\beginOutput
i56 : spSchub = (r, L, P) -> (\\
\           I := ideal apply(subsets(numgens source L, \\
\                            r + numgens target L), S -> \\
\                fold((sum, U) -> sum +\\
\                 fold((term,i) -> term*(-1)^i, P_(S_U) * det(\\
\                  submatrix(L, sort toList(set(S) - set(S_U)))), U), \\
\                     0, subsets(#S, r))));\\
\endOutput
%
We are working in the Grassmannian of 3-planes in 
${\mathbb C}^6$.
%
\beginOutput
i57 : R = QQ[apply(subsets(6,3), i -> p_i )];\\
\endOutput
%
The ideal $I$ consists of the
special Schubert conditions for the 3-planes to meet the 3-planes osculating
the rational normal curve at the points 1, 2, and 3, and to also meet the
2-planes osculating at 4, 5, and 6,
together with the Pl\"ucker ideal {\tt Grassmannian(2, 5, R)}.
Since this is a 1-dimensional homogeneous ideal, we add the linear form 
{\tt p{\char`\_}\char`\{0,1,5{\char`\}} - 1} to make the ideal
zero-dimensional\index{ideal!zero-dimensional}.
As before, {\tt Grassmannian(2, 5, R)} creates the Pl\"ucker ideal of 
${\bf G}_{3,6}$.
%
\beginOutput
i58 : I = fold((J, i) -> J +\\
\            spSchub(3, substitute(oscPlane(3, 6, s), \{s=> 1+i\}), p) +\\
\            spSchub(3, substitute(oscPlane(2, 6, s), \{s=> 4+i\}), p), \\
\            Grassmannian(2, 5, R), \{0,1,2\}) + \\
\           ideal (p_\{0,1,5\} - 1);\\
\emptyLine
o58 : Ideal of R\\
\endOutput
%
This has dimension 0 and degree 6, in agreement with the Schubert calculus.
%
\beginOutput
i59 : dim I, degree I\\
\emptyLine
o59 = (0, 6)\\
\emptyLine
o59 : Sequence\\
\endOutput
%
As expected, all roots are real.
%
\beginOutput
i60 : A = R/I; numRealSturm(eliminant(p_\{2,3,4\}, QQ[Z]))\\
\emptyLine
o61 = 6\\
\endOutput
%
There have been many checked instances of this
conjecture~\cite{SO:So_shap-www,SO:So00b,SO:Ver00}, and it has some
geometrically interesting generalizations~\cite{SO:So_flags}.

The question remains for which numbers $0\leq d\leq d(r,n;l_1,\ldots,l_m)$ do
there exist real planes $L_i$ with $d(r,n;l_1,\ldots,l_m)$
$r$-planes meeting each $L_i$, and exactly $d$ of them are real.
Besides Theorem~\ref{thm:special-reality} and the obvious parity condition,
nothing is known in general.
In every known case, every possibility occurs---which is not the case in all
enumerative problems, even those that are fully real\index{enumerative
problem!fully real}\footnote{For example, of
the 12 rational plane cubics containing 8 real points in ${\mathbb P}^2$,
either 8, 10 or 12 can be real, and there are 8 points with all 12
real~\cite[Proposition 4.7.3]{SO:DeKh00}.}.
Settling this (for $d=0$) has implications for linear systems
theory~\cite{SO:RS98}.\footnote{After this was written, Eremenko and
Gabrielov~\cite{SO:EG-NR} showed that $d$ can be zero for the enumerative 
problems given by data $(2,2n,1^{4n-4})$ and  $(2n-2,2n,1^{4n-4})$.}

\subsection{Transversality in the Schubert Calculus}
A basic principle of the classical Schubert calculus\index{Schubert calculus}
is that the intersection 
number $d(r,n;l_1,\ldots,l_m)$ has enumerative significance---that is, for
general linear subspaces $L_i$, all solutions appear with multiplicity 1.
This basic principle is not known to hold in general.
For fields of characteristic zero, Kleiman's Transversality
Theorem~\cite{SO:MR50:13063} establishes this principle.
When $r$ or $n{-}r$ is 2, then Theorem~E of~\cite{SO:So97a} establishes this
principle in arbitrary characteristic.
We conjecture that this principle holds in general; that is, for arbitrary
infinite fields and any Schubert data, if the planes $L_i$ are in general
position, then the resulting zero-dimensional ideal is
radical\index{ideal!radical}. 

We test this conjecture on the enumerative problem of 
Section~\ref{sec:shapiro}, which is not covered by 
Theorem~E of~\cite{SO:So97a}.
The function {\tt testTransverse(F)} tests transversality
for this enumerative problem, for a given field $F$. 
It does this by first computing the ideal of the enumerative problem using 
random planes $L_i$.
%
\beginOutput
i62 : randL = (R, n, r, l) -> \\
\                matrix table(n-r+1-l, n, (i, j) -> random(0, R));\\
\endOutput
%
and the Pl\"ucker ideal of the Grassmannian ${\bf G}_{3,6}$
 {\tt Grassmannian(2, 5, R)}.)
Then it adds a random (inhomogeneous) linear relation 
 {\tt 1 + random(1, R)} to make the ideal zero-dimensional for generic $L_i$. 
When this ideal is zero dimensional and has degree 6 (the expected degree), it
computes the characteristic polynomial {\tt g} of a generic linear form.
If {\tt g} has no multiple roots, {\tt 1 == gcd(g, diff(Z, g))}, 
then the Shape Lemma\index{Shape Lemma}
guarantees that the ideal was radical.
{\tt testTransverse} exits either when it computes a radical ideal,
or after {\tt limit} iterations (which is set to 5 for these examples), and
prints the return status. 
%
\beginOutput
i63 : testTransverse = F -> (\\
\            R := F[apply(subsets(6, 3), i -> q_i )];\\
\            continue := true;\\
\            j := 0;  \\
\            limit := 5;\\
\            while continue and (j < limit) do (\\
\                 j = j + 1;\\
\                 I := fold((J, i) -> J + \\
\                           spSchub(3, randL(R, 6, 3, 1), q) +\\
\                           spSchub(3, randL(R, 6, 3, 2), q),\\
\                           Grassmannian(2, 5, R) + \\
\                           ideal (1 + random(1, R)),\\
\                           \{0, 1, 2\});\\
\                 if (dim I == 0) and (degree I == 6) then (\\
\                 lin := promote(random(1, R), (R/I));\\
\                 g := charPoly(lin, Z);\\
\                 continue = not(1 == gcd(g, diff(Z, g)));\\
\                 ));\\
\            if continue then << "Failed for the prime " << char F << \\
\               " with " << j << " iterations" << endl;\\
\            if not continue then << "Succeeded for the prime " <<\\
\                char F << " in " << j << " iteration(s)" << endl;\\
\            );\\
\endOutput
%
Since 5 iterations do not show transversality for ${\mathbb F}_2$,
%
\beginOutput
i64 : testTransverse(ZZ/2);\\
Failed for the prime 2 with 5 iterations\\
\endOutput
%
we can test transversality in characteristic 2 using the field with
four elements, ${\mathbb F}_4=$ {\tt GF 4}.
%
\beginOutput
i65 : testTransverse(GF 4);\\
Succeeded for the prime 2 in 3 iteration(s)\\
\endOutput
%
We do find transversality for ${\mathbb F}_7$.
%
\beginOutput
i66 : testTransverse(ZZ/7);\\
Succeeded for the prime 7 in 2 iteration(s)\\
\endOutput

We have tested transversality for all primes less than 100 in every
enumerative problem involving Schubert conditions on 3-planes in $k^6$.
These include the problem above as well as the problem of 42 3-planes meeting
9 general 3-planes.\footnote{After this was written, we discovered an
elementary proof of transversality for the enumerative problems given by data 
$(r,n;1^{r(n-r)})$, where the conditions are all
codimension~1~\cite{SO:So_trans}.}


\section{The 12 Lines: Reprise}
The enumerative problems of Section~\ref{sec:enumerative} were formulated
in local coordinates~(\ref{eq:local}) for the Grassmannian of lines in
${\mathbb P}^3$ (Grassmannian of 2-dimensional subspaces in $k^4$).
When we formulate the problem of Section~\ref{sec:12lines} in the global
Pl\"ucker coordinates\index{Plucker coordinate@Pl\"ucker coordinate} of Section~\ref{sec:grass},
we find some interesting phenomena.
We also consider some related enumerative problems\index{enumerative problem}.

\subsection{Global Formulation}\label{sec:global}
A quadratic form\index{quadratic form} $q$ on a vector space $V$ over a field
$k$ not of characteristic 2 is given by
$q({\bf u})=(\varphi({\bf u}),{\bf u})$, where $\varphi\colon V\to V^*$ is
a {\it symmetric} linear map, that is 
$(\varphi({\bf u}),{\bf v})=(\varphi({\bf v}),{\bf u})$.
Here, $V^*$ is the linear dual of $V$ and $(\,\cdot\;,\,\cdot\,)$ is the
pairing $V\otimes V^*\to k$. 
The map $\varphi$ induces a quadratic form $\wedge^rq$ on the $r$th exterior
power $\wedge^rV$ of $V$ through the symmetric map 
$\wedge^r\varphi\colon \wedge^rV\to\wedge^rV^*=(\wedge^rV)^*$.
The action of $\wedge^rV^*$ on $\wedge^rV$ is given by
%
\begin{equation}\label{eq:wedge}
  ({\bf x}_1\wedge{\bf x}_2\wedge\cdots\wedge{\bf x}_r,\ 
   {\bf y\!}_1\wedge{\bf y}\!_2\wedge\cdots\wedge{\bf y}\!_r)\ =\ 
   \det|({\bf x}_i, {\bf y}\!_j)|\,,
\end{equation}
%
where ${\bf x}_i\in V^*$ and ${\bf y}\!_j\in V$.

When we fix isomorphisms $V\simeq k^n\simeq V^*$, the map $\varphi$ is given
by a symmetric $n\times n$ matrix $M$ as in Section~\ref{sec:12lines}.
Suppose $r=2$.
Then for ${\bf u},{\bf v}\in k^n$,
$$
  \wedge^2q({\bf u}\wedge{\bf v})\ = 
              \det\left[\begin{array}{cc}
                    {\bf u}M{\bf u}^t &{\bf u}M{\bf v}^t\\
                    {\bf v}M{\bf u}^t &{\bf v}M{\bf v}^t
                  \end{array}\right]\ ,
$$
which is Equation~(\ref{eq:sphere}) of  Section~\ref{sec:12lines}.

\begin{proposition}\label{prop:tangent_line}
  A line $\ell$ is tangent to a quadric ${\mathcal V}(q)$ in 
  ${\mathbb P}^{n-1}$ if and only if its Pl\"ucker 
  coordinate\index{Plucker coordinate@Pl\"ucker coordinate}  
  $\wedge^2\ell\in{\mathbb P}^{\binom{n}{2}-1}$ lies on the quadric
  ${\mathcal V}(\wedge^2q)$.
\end{proposition}
       
Thus the Pl\"ucker coordinates for the set of lines tangent to 4 general
quadrics in ${\mathbb P}^3$ satisfy 5 quadratic equations:
The single Pl\"ucker relation~(\ref{eq:G22})
together with one quadratic equation for each quadric.
Thus we expect the B\'ezout number\index{Bezout number@B\'ezout number} of $2^5=32$ such
lines. 
We check this.

The procedure {\tt randomSymmetricMatrix(R, n)}
generates a random symmetric $n\times n$ matrix with entries in 
the base ring of $R$.
%
\beginOutput
i67 : randomSymmetricMatrix = (R, n) -> (\\
\          entries := new MutableHashTable;\\
\          scan(0..n-1, i -> scan(i..n-1, j -> \\
\                       entries#(i, j) = random(0, R)));\\
\          matrix table(n, n, (i, j) -> if i > j then \\
\                       entries#(j, i) else entries#(i, j))\\
\          );\\
\endOutput
%
The procedure {\tt tangentEquation(r, R, M)} gives the equation in Pl\"ucker
coordinates for a point in ${\mathbb P}^{\binom{n}{r}-1}$ to be
isotropic with respect to the bilinear form\index{bilinear form}  $\wedge^rM$
({\tt R} is assumed to be the coordinate ring of 
${\mathbb P}^{\binom{n}{r}-1}$).
This is the equation for an $r$-plane to be tangent to the quadric
associated to $M$.
%
\beginOutput
i68 : tangentEquation = (r, R, M) -> (\\
\           g := matrix \{gens(R)\};\\
\           (entries(g * exteriorPower(r, M) * transpose g))_0_0\\
\           );\\
\endOutput
%
We construct the ideal of lines tangent to 4 general quadrics in
${\mathbb P}^3$.
%
\beginOutput
i69 : R = QQ[apply(subsets(4, 2), i -> p_i )];\\
\endOutput
%
\beginOutput
i70 : I = Grassmannian(1, 3, R) + ideal apply(0..3, i -> \\
\           tangentEquation(2, R, randomSymmetricMatrix(R, 4)));\\
\emptyLine
o70 : Ideal of R\\
\endOutput
%
As expected, this ideal has dimension 0 and degree 32.
%
\beginOutput
i71 : dim Proj(R/I), degree I\\
\emptyLine
o71 = (0, 32)\\
\emptyLine
o71 : Sequence\\
\endOutput
%


\subsection{Lines Tangent to 4 Spheres}\label{sec:tangent_lines}
That calculation raises the following question:
In Section~\ref{sec:12lines}, why did we obtain only 12 lines tangent to 4
spheres\index{sphere}? 
To investigate this, we generate the global ideal of lines tangent to the 
spheres of Section~\ref{sec:12lines}.
%
\beginOutput
i72 : I = Grassmannian(1, 3, R) + \\
\              ideal (tangentEquation(2, R, Sphere(0,0,0,5)),\\
\                     tangentEquation(2, R, Sphere(4,1,1,5)),\\
\                     tangentEquation(2, R, Sphere(1,4,1,5)),\\
\                     tangentEquation(2, R, Sphere(1,1,4,5)));\\
\emptyLine
o72 : Ideal of R\\
\endOutput
%
We compute the dimension and degree of ${\mathcal V}(I)$.
%
\beginOutput
i73 : dim Proj(R/I), degree I\\
\emptyLine
o73 = (1, 4)\\
\emptyLine
o73 : Sequence\\
\endOutput
%
The ideal is not zero dimensional\index{ideal!zero-dimensional}; there is an
extraneous one-dimensional 
component of zeroes with degree 4.
Since we found 12 lines in
Section~\ref{sec:12lines} using the local coordinates~(\ref{eq:local}),
the extraneous component must lie in the complement of that coordinate patch,
which is defined by the vanishing of the first Pl\"ucker coordinate, $p_{\{0,1\}}$.
We saturate\index{saturation} $I$ by $p_{\{0,1\}}$ to obtain the desired lines.
%
\beginOutput
i74 : Lines = saturate(I, ideal (p_\{0,1\}));\\
\emptyLine
o74 : Ideal of R\\
\endOutput
%
This ideal does have dimension 0 and degree 12, so we have recovered the
zeroes of Section~\ref{sec:12lines}.
%
\beginOutput
i75 : dim Proj(R/Lines), degree(Lines)\\
\emptyLine
o75 = (0, 12)\\
\emptyLine
o75 : Sequence\\
\endOutput
%

We investigate the rest of the zeroes, which we obtain 
by taking the ideal
quotient of $I$ and the ideal of lines.
As computed above, this has dimension 1 and degree 4.
%
\beginOutput
i76 : Junk = I : Lines;\\
\emptyLine
o76 : Ideal of R\\
\endOutput
%
\beginOutput
i77 : dim Proj(R/Junk), degree Junk\\
\emptyLine
o77 = (1, 4)\\
\emptyLine
o77 : Sequence\\
\endOutput
%
We find the support of this extraneous component by taking its
radical.
%
\beginOutput
i78 : radical(Junk)\\
\emptyLine
\                                         2         2         2\\
o78 = ideal (p      , p      , p      , p       + p       + p      )\\
\              \{0, 3\}   \{0, 2\}   \{0, 1\}   \{1, 2\}    \{1, 3\}    \{2, 3\}\\
\emptyLine
o78 : Ideal of R\\
\endOutput
%
From this, we see that the extraneous component is supported on an imaginary
conic in the ${\mathbb P}^2$ of lines at infinity.
\smallskip

To understand the geometry behind this computation, observe that
the sphere with radius $r$ and center $(a,b,c)$ has homogeneous equation
$$
  (x-wa)^2+(y-wb)^2+(z-wc)^2\ =\ r^2w^2\,.
$$
At infinity, $w=0$, this has equation
$$
  x^2+y^2+z^2\ =\ 0\,.
$$
The extraneous component is supported on the set of tangent
lines to this imaginary conic.
Aluffi and Fulton~\cite{SO:AF} studied
this problem, using geometry to 
identify the extraneous ideal and the excess intersection
formula~\cite{SO:FM76} to obtain the answer of 12. 
Their techniques show that there will be 12 isolated lines tangent to 4
quadrics which have a smooth conic in common.

When the quadrics are spheres, the conic is the imaginary conic at infinity.
Fulton asked the following question:
Can all 12 lines be real if the (real) four quadrics share a real conic?
We answer his question in the affirmative in the next section.


\subsection{Lines Tangent to Real Quadrics Sharing a Real Conic}
We consider four quadrics in ${\mathbb P}^3_{\mathbb R}$ sharing a
non-singular conic, which we will take to be at infinity so that we may use
local coordinates for ${\bf G}_{2,4}$ in our computations.
The variety ${\mathcal V}(q)\subset{\mathbb P}^3_{\mathbb R}$ of a
nondegenerate quadratic form $q$ is determined up to isomorphism by the
absolute value of the signature\index{bilinear form!signature} $\sigma$ of the
associated bilinear form. 
Thus there are three possibilities, 0, 2, or 4, for $|\sigma|$.

When $|\sigma|=4$, the real quadric ${\mathcal V}(q)$ is empty.
The associated symmetric matrix $M$ is conjugate to the identity
matrix, so $\wedge^2M$ is also conjugate to the identity matrix.
Hence ${\mathcal V}(\wedge^2q)$ contains no real points.
Thus we need not consider quadrics with $|\sigma|=4$.

When $|\sigma|=2$, we have ${\mathcal V}(q)\simeq S^2$, the 2-sphere.
If the conic at infinity is imaginary, then 
${\mathcal V}(q)\subset{\mathbb R}^3$ is an ellipsoid\index{ellipsoid}.
If the conic at infinity is real, then ${\mathcal V}(q)\subset{\mathbb R}^3$ is 
a hyperboloid\index{hyperboloid} of two sheets.
When $\sigma=0$, we have ${\mathcal V}(q)\simeq S^1\times S^1$, a torus.
In this case, ${\mathcal V}(q)\subset{\mathbb R}^3$ is 
a hyperboloid of one sheet and the conic at infinity is real.

Thus either we have 4 ellipsoids sharing an imaginary conic at infinity,
which we studied in Section~\ref{sec:12lines}; or else we have four
hyperboloids sharing a real conic at infinity, and there are five 
possible combinations of hyperboloids of one or two sheets sharing a real
conic at infinity.
This gives six topologically distinct possibilities in all.


\begin{theorem}
For each of the six topologically distinct possibilities of four real
quadrics sharing a smooth conic at infinity, there exist four quadrics
having the property that each of the 12 lines in ${\mathbb C}^3$
simultaneously tangent to the four quadrics is real.
\end{theorem}

\begin{proof}
By the computation in Section~\ref{sec:12lines}, 
we need only check the five possibilities for hyperboloids\index{hyperboloid}.
We fix the conic at infinity to be $x^2+y^2-z^2=0$.
Then the general hyperboloid of two sheets containing this conic has
equation in ${\mathbb R}^3$
%
\begin{equation}\label{eq:twoSheet}
  (x-a)^2+(y-b)^2-(z-c)^2+r\ =\ 0\,,
\end{equation}
%
(with $r>0$).
The command {\tt Two(a,b,c,r)} generates the associated 
symmetric matrix.
%
\beginOutput
i79 : Two = (a, b, c, r) -> (\\
\           matrix\{\{a^2 + b^2 - c^2 + r ,-a ,-b , c \},\\
\                  \{         -a         , 1 , 0 , 0 \},\\
\                  \{         -b         , 0 , 1 , 0 \},\\
\                  \{          c         , 0 , 0 ,-1 \}\}\\
\           );\\
\endOutput
The general hyperboloid of one sheet containing the conic
$x^2+y^2-z^2=0$ at infinity has equation in ${\mathbb R}^3$
%
\begin{equation}\label{eq:oneSheet}
  (x-a)^2+(y-b)^2-(z-c)^2-r\ =\ 0\,,
\end{equation}
%
(with $r>0$).
The command {\tt One(a,b,c,r)} generates the associated 
symmetric matrix.
%
\beginOutput
i80 : One = (a, b, c, r) -> (\\
\           matrix\{\{a^2 + b^2 - c^2 - r ,-a ,-b , c \},\\
\                  \{         -a         , 1 , 0 , 0 \},\\
\                  \{         -b         , 0 , 1 , 0 \},\\
\                  \{          c         , 0 , 0 ,-1 \}\}\\
\           );\\
\endOutput

We consider $i$ quadrics of two sheets~(\ref{eq:twoSheet}) and $4-i$
quadrics of one sheet~(\ref{eq:oneSheet}).
For each of these cases, the table below displays four 4-tuples of data
$(a,b,c,r)$ which give 12 real lines.
(The data for the hyperboloids of one sheet are listed first.)
\begin{center}
\begin{tabular}{|c|l|}\hline
$\;i$\;&\hspace{13em}Data\\\hline
$0$&$\hspace{.915em}(5,3,3,16),\hspace{2.185em}(5,-4,2,1),\hspace{1.27em}
                             (-3,-1,1,1),\hspace{.88em}(2,-7,0,1)$\\\hline
$1$&$\hspace{.385em}(3,-2,-3,6),\hspace{.885em}(-3,-7,-6,7),\hspace{.5em}
                            (-6,3,-5,2),\hspace{.865em}(1,6,-2,5)$\\\hline
$2$&$\hspace{1.165em}(6,4,6,4),\hspace{2.43em}(-1,3,3,6),\hspace{1.265em}
                             (-7,-2,3,3),\hspace{.5em}(-6,7,-2,5)$\\\hline
$3$&$(-1,-4,-1,1),\hspace{.885em}(-3,3,-1,1),\hspace{.885em}
              \hspace{.4em}(-7,6,2,9),\hspace{1.025em}(5,6,-1,12)$\\\hline
$4$&$\hspace{.525em}(5,2,-1,25),\hspace{1.555em}(6,-6,2,25),\hspace{1.03em}
                 \hspace{.4em}(-7,1,6,1),\hspace{1.65em}(3,1,0,1)$\\\hline
\end{tabular}
\end{center}

%\noindent$|\hspace{1em}|$

%\noindent$|\hspace{10pt}|$


We test each of these, using the formulation in local coordinates of
Section~\ref{sec:12lines}. 
%
\beginOutput
i81 : R = QQ[y11, y12, y21, y22];\\
\endOutput
%
\beginOutput
i82 : I = ideal (tangentTo(One( 5, 3, 3,16)), \\
\                 tangentTo(One( 5,-4, 2, 1)),  \\
\                 tangentTo(One(-3,-1, 1, 1)), \\
\                 tangentTo(One( 2,-7, 0, 1)));\\
\emptyLine
o82 : Ideal of R\\
\endOutput
%
\beginOutput
i83 : numRealSturm(charPoly(promote(y22, R/I), Z))\\
\emptyLine
o83 = 12\\
\endOutput
%
\beginOutput
i84 : I = ideal (tangentTo(One( 3,-2,-3, 6)), \\
\                 tangentTo(One(-3,-7,-6, 7)),  \\
\                 tangentTo(One(-6, 3,-5, 2)), \\
\                 tangentTo(Two( 1, 6,-2, 5)));\\
\emptyLine
o84 : Ideal of R\\
\endOutput
%
\beginOutput
i85 : numRealSturm(charPoly(promote(y22, R/I), Z))\\
\emptyLine
o85 = 12\\
\endOutput
%
\beginOutput
i86 : I = ideal (tangentTo(One( 6, 4, 6, 4)),  \\
\                 tangentTo(One(-1, 3, 3, 6)), \\
\                 tangentTo(Two(-7,-2, 3, 3)), \\
\                 tangentTo(Two(-6, 7,-2, 5)));\\
\emptyLine
o86 : Ideal of R\\
\endOutput
%
\beginOutput
i87 : numRealSturm(charPoly(promote(y22, R/I), Z))\\
\emptyLine
o87 = 12\\
\endOutput
%

\beginOutput
i88 : I = ideal (tangentTo(One(-1,-4,-1, 1)),\\
\                 tangentTo(Two(-3, 3,-1, 1)),  \\
\                 tangentTo(Two(-7, 6, 2, 9)), \\
\                 tangentTo(Two( 5, 6,-1,12)));\\
\emptyLine
o88 : Ideal of R\\
\endOutput
%
\beginOutput
i89 : numRealSturm(charPoly(promote(y22, R/I), Z))\\
\emptyLine
o89 = 12\\
\endOutput
%
\beginOutput
i90 : I = ideal (tangentTo(Two( 5, 2,-1,25)), \\
\                 tangentTo(Two( 6,-6, 2,25)), \\
\                 tangentTo(Two(-7, 1, 6, 1)), \\
\                 tangentTo(Two( 3, 1, 0, 1)));\\
\emptyLine
o90 : Ideal of R\\
\endOutput
%
\beginOutput
i91 : numRealSturm(charPoly(promote(y22, R/I), Z))\\
\emptyLine
o91 = 12\\
\endOutput
%
\qed
\end{proof}


In each of these enumerative problems there are 12 complex solutions.
For each, we have done other computations  showing that every possible
number of real solutions (0, 2, 4, 6, 8, 10, or 12) can occur.



\subsection{Generalization to Higher Dimensions}
We consider lines tangent to quadrics in higher dimensions.
First, we reinterpret the action of $\wedge^rV^*$ on $\wedge^rV$
described in~(\ref{eq:wedge}) as follows.
The vectors ${\bf x}_1,\ldots,{\bf x}_r$ and ${\bf y}\!_1,\ldots,{\bf y}\!_r$ 
define maps $\alpha\colon k^r\to V^*$ and $\beta\colon k^r\to V$.
The matrix $[({\bf x}_i,\,{\bf y}\!_j)]$ is the matrix of the bilinear form
on $k^r$ given by 
$\langle{\bf u},\,{\bf v}\rangle:= (\alpha({\bf u}),\,\beta({\bf v}))$.
Thus~(\ref{eq:wedge}) vanishes when the bilinear form 
$\langle\,\cdot\,,\,\cdot\,\rangle$ on $k^r$ is degenerate.

Now suppose that we have a quadratic form $q$ on $V$ given by a symmetric map 
$\varphi\colon V\to V^*$.
This induces a quadratic form and hence a quadric on any $r$-plane $H$ in $V$ 
(with $H\not\subset{\mathcal V}(q)$).
This induced quadric is singular when $H$ is tangent to ${\mathcal V}(q)$.
Since a quadratic form is degenerate only when the associated projective
quadric is singular, we see that
$H$ is tangent to the quadric
${\mathcal V}(q)$ if and only if 
$(\wedge^r\varphi(\wedge^rH),\,\wedge^rH)=0$.
(This includes the case $H\subset{\mathcal V}(q)$.)
We summarize this argument.


\begin{theorem}
Let $\varphi\colon V\to V^*$ be a linear map with resulting bilinear form
$(\varphi({\bf u}),\,{\bf v})$.
Then the locus of $r$-planes in $V$ for which the restriction of this form
is degenerate is the set of $r$-planes $H$ whose Pl\"ucker
coordinates\index{Plucker coordinate@Pl\"ucker coordinate} are
isotropic, $(\wedge^r\varphi(\wedge^rH),\,\wedge^rH)=0$, with respect to the
induced form on $\wedge^rV$.

When $\varphi$ is symmetric, this is the locus of $r$-planes tangent to the
associated quadric in ${\mathbb P}(V)$.
\end{theorem}


We explore the problem of lines tangent to quadrics in ${\mathbb P}^n$.
From the calculations of Section~\ref{sec:global}, we do not expect this to
be interesting if the quadrics are general.
(This is borne out for ${\mathbb P}^4$:
 we find 320 lines in ${\mathbb P}^4$ tangent to 6 general quadrics.
 This is the B\'ezout number\index{Bezout number@B\'ezout number}, as $\deg{\bf G}_{2,5}=5$
 and the condition to be tangent to a quadric has degree 2.)
This problem is interesting if the quadrics 
in ${\mathbb P}^n$ share a quadric in a ${\mathbb P}^{n-1}$.
We propose studying such enumerative problems, both determining the 
number of solutions for general such quadrics, and investigating whether or
not it is possible to have all solutions be real. 

We use \Mtwo{}\/ to compute the expected number of
solutions to this problem when $r=2$ and $n=4$.
We first define some functions for this computation, which will involve
counting the degree of the ideal of lines in ${\mathbb P}^4$ tangent to 6
general spheres\index{sphere}.
Here, $X$ gives local coordinates for the Grassmannian\index{Grassmannian},
$M$ is a symmetric matrix, {\tt tanQuad} gives the equation in $X$ for the
lines tangent to the quadric given by $M$.
%
\beginOutput
i92 : tanQuad = (M, X) -> (\\
\           u := X^\{0\};\\
\           v := X^\{1\};\\
\           (u * M * transpose v)^2 - \\
\           (u * M * transpose u) * (v * M * transpose v)\\
\           );\\
\endOutput
%
{\tt nSphere} gives the matrix $M$ for a sphere with
center {\tt V} and squared radius {\tt r}, and {\tt V} and {\tt r} give random
data for a sphere.
%
\beginOutput
i93 : nSphere = (V, r) -> \\
\               (matrix \{\{r + V * transpose V\}\} || transpose V ) |\\
\               ( V || id_((ring r)^n)\\
\               );\\
\endOutput
%
\beginOutput
i94 : V = () -> matrix table(1, n, (i,j) -> random(0, R));\\
\endOutput
%
\beginOutput
i95 : r = () -> random(0, R);\\
\endOutput
%
We construct the ambient ring, local coordinates, and the ideal of the
enumerative problem of lines in ${\mathbb P}^4$ tangent to 6 random spheres.
%
\beginOutput
i96 : n = 4;\\
\endOutput
%
\beginOutput
i97 : R = ZZ/1009[flatten(table(2, n-1, (i,j) -> z_(i,j)))];\\
\endOutput
%
\beginOutput
i98 : X = 1 | matrix table(2, n-1, (i,j) -> z_(i,j))\\
\emptyLine
o98 = | 1 0 z_(0,0) z_(0,1) z_(0,2) |\\
\      | 0 1 z_(1,0) z_(1,1) z_(1,2) |\\
\emptyLine
\              2       5\\
o98 : Matrix R  <--- R\\
\endOutput
%
\beginOutput
i99 : I = ideal (apply(1..(2*n-2), \\
\                     i -> tanQuad(nSphere(V(), r()), X)));\\
\emptyLine
o99 : Ideal of R\\
\endOutput
%
We find there are 24 lines in ${\mathbb P}^4$ tangent to 6 general spheres.
\beginOutput
i100 : dim I, degree I\\
\emptyLine
o100 = (0, 24)\\
\emptyLine
o100 : Sequence\\
\endOutput
%
The expected numbers of solutions we have obtained in this way are displayed
in the table below.
The numbers in boldface are those which are proven.\footnote{As this was
going to press, the obvious pattern was proven:
There are $3\cdot 2^{n-1}$ complex lines tangent to $2n-2$ 
general spheres in ${\mathbb R}^n$, and all may be real~\cite{SO:STh01}.}

\begin{center}
\begin{tabular}{|c|c|c|c|c|c|}\hline
  $n$&2&3&4&5&6\\\hline
  \# expected\;&\;{\bf 4}\;&\;{\bf 12}\;&\;24\;&\;48\;&\;96\;\\\hline
\end{tabular}
\end{center}

\begin{acknowledgment}
We thank Dan Grayson and 
Bernd Sturmfels:
some of the procedures in this chapter were written by Dan Grayson
and the calculation in Section 5.2 is due to Bernd Sturmfels.
\end{acknowledgment}

%\bibliographystyle{siam}
%\bibliography{../../bibl/bibliography,../../bibl/sottile}

%\end{document}

\begin{thebibliography}{10}

\bibitem{SO:AF}
P.~Aluffi and W.~Fulton:
\newblock Lines tangent to four surfaces containing a curve.
\newblock 2001.

\bibitem{SO:BMMT}
E.~Becker, M.~G. Marinari, T.~Mora, and C.~Traverso:
\newblock The shape of the {S}hape {L}emma.
\newblock In {\em Proceedings ISSAC-94}, pages 129--133, 1993.

\bibitem{SO:BW}
E.~Becker and Th. W{\"o}ermann:
\newblock On the trace formula for quadratic forms.
\newblock In {\em Recent advances in real algebraic geometry and quadratic
  forms (Berkeley, CA, 1990/1991; San Francisco, CA, 1991)}, pages 271--291.
  Amer. Math. Soc., Providence, RI, 1994.

\bibitem{SO:Bernstein}
D.~N. Bernstein:
\newblock The number of roots of a system of equations.
\newblock {\em Funct. Anal. Appl.}, 9:183--185, 1975.

\bibitem{SO:BV77}
O.~Bottema and G.R. Veldkamp:
\newblock On the lines in space with equal distances to $n$ given points.
\newblock {\em Geometrie Dedicata}, 6:121--129, 1977.

\bibitem{SO:CCS}
A.~M. Cohen, H.~Cuypers, and H.~Sterk, editors.
\newblock {\em Some Tapas of Computer Algebra}.
\newblock Springer-Varlag, 1999.

\bibitem{SO:CLO92}
D.~Cox, J.~Little, and D.~O'Shea:
\newblock {\em Ideals, Varieties, Algorithms: An Introduction to Computational
  Algebraic Geometry and Commutative Algebra}.
\newblock UTM. Springer-Verlag, New York, 1992.

\bibitem{SO:DeKh00}
A.~I. Degtyarev and V.~M. Kharlamov:
\newblock Topological properties of real algebraic varieties: {R}okhlin's way.
\newblock {\em Uspekhi Mat. Nauk}, 55(4(334)):129--212, 2000.

\bibitem{SO:MR97a:13001}
D.~Eisenbud:
\newblock {\em Commutative Algebra With a View Towards Algebraic Geometry}.
\newblock Number 150 in GTM. Springer-Verlag, 1995.

\bibitem{SO:EG00}
A.~Eremenko and A.~Gabrielov:
\newblock Rational functions with real critical points and {B}.~and
  {M}.~{S}hapiro conjecture in real enumerative geometry.
\newblock MSRI preprint 2000-002, 2000.

\bibitem{SO:EG-NR}
A.~Eremenko and A.~Gabrielov:
\newblock New counterexamples to pole placement by static output feedback.
\newblock Linear Algebra and its Applications, to appear, 2001.

\bibitem{SO:Fu84a}
W.~Fulton:
\newblock {\em Intersection Theory}.
\newblock Number~2 in Ergebnisse der Mathematik und ihrer Grenzgebiete.
  Springer-Verlag, 1984.

\bibitem{SO:FM76}
W.~Fulton and R.~MacPherson:
\newblock Intersecting cycles on an algebraic variety.
\newblock In P.~Holm, editor, {\em Real and Complex Singularities}, pages
  179--197. Oslo, 1976, Sijthoff and Noordhoff, 1977.

\bibitem{SO:HSS}
B.~Huber, F.~Sottile, and B.~Sturmfels:
\newblock Numerical {S}chubert calculus.
\newblock {\em J.~Symb.~Comp.}, 26(6):767--788, 1998.

\bibitem{SO:MR50:13063}
S.~Kleiman:
\newblock The transversality of a general translate.
\newblock {\em Compositio Math.}, 28:287--297, 1974.

\bibitem{SO:MR48:2152}
S.~Kleiman and D.~Laksov:
\newblock Schubert calculus.
\newblock {\em Amer. Math. Monthly}, 79:1061--1082, 1972.

\bibitem{SO:Li00}
D.~Lichtblau:
\newblock Finding cylinders through 5 points in $\mathbb{R}^3$.
\newblock Mss., email address: {danl@wolfram.com}, 2001.

\bibitem{SO:MPT00}
I.G. Macdonald, J.~Pach, and T.~Theobald:
\newblock Common tangents to four unit balls in $\mathbb{R}^3$.
\newblock To appear in {\em Discrete and Computational Geometry} {\bf 26}:1
  (2001).

\bibitem{SO:PRS}
P.~Pedersen, M.-F. Roy, and A.~Szpirglas:
\newblock Counting real zeros in the multivariate case.
\newblock In {\em Computational {A}lgebraic {G}eometry (Nice, 1992)}, pages
  203--224. Birkh\"auser Boston, Boston, MA, 1993.

\bibitem{SO:RS98}
J.~Rosenthal and F.~Sottile:
\newblock Some remarks on real and complex output feedback.
\newblock {\em Systems Control Lett.}, 33(2):73--80, 1998.
\newblock For a description of the computational aspects, see
  {http://www.nd.edu/\~{}rosen/pole/}.

\bibitem{SO:So97a}
F.~Sottile:
\newblock Enumerative geometry for the real {G}rassmannian of lines in
  projective space.
\newblock {\em Duke Math. J.}, 87(1):59--85, 1997.

\bibitem{SO:So99a}
F.~Sottile:
\newblock The special {S}chubert calculus is real.
\newblock {\em ERA of the AMS}, 5:35--39, 1999.

\bibitem{SO:So_shap-www}
F.~Sottile:
\newblock The conjecture of {S}hapiro and {S}hapiro.
\newblock An archive of computations and computer algebra scripts,
  {http://www.expmath.org/extra/9.2/sottile/}, 2000.

\bibitem{SO:So_trans}
F.~Sottile:
\newblock Elementary transversality in the schubert calculus in any
  characteristic.
\newblock math.AG/0010319, 2000.

\bibitem{SO:So00b}
F.~Sottile:
\newblock Real {S}chubert calculus: Polynomial systems and a conjecture of
  {S}hapiro and {S}hapiro.
\newblock {\em Exper.~Math.}, 9:161--182, 2000.

\bibitem{SO:So_flags}
F.~Sottile:
\newblock Some real and unreal enumerative geometry for flag manifolds.
\newblock {\em Mich. Math. J}, 48:573--592, 2000.

\bibitem{SO:STh01}
F.~Sottile and T.~Theobald:
\newblock Lines tangent to $2n-2$ spheres in $\mathbb{R}^n$.
\newblock math.AG/0105180, 2001.

\bibitem{SO:Sturmfels_GBCP}
B.~Sturmfels:
\newblock {\em Gr{\"o}bner Bases and Convex Polytopes}, volume~8 of {\em
  University Lecture Series}.
\newblock American Math. Soc., Providence, RI, 1996.

\bibitem{SO:Ver99}
J.~Verschelde:
\newblock Polynomial homotopies for dense, sparse, and determinantal systems.
\newblock MSRI preprint 1999-041, 1999.

\bibitem{SO:Ver00}
J.~Verschelde:
\newblock Numerical evidence of a conjecture in real algebraic geometry.
\newblock {\em Exper.~Math.}, 9:183--196, 2000.

\end{thebibliography}
\egroup
\makeatletter
\renewcommand\thesection{\@arabic\c@section}
\makeatother



%%%%%%%%%%%%%%%%%%%%%%%%%%%%%%%%%%%%%%%%%%%%%%%%
%%%%%
%%%%% ../chapters/completeIntersections/ci
%%%%%
%%%%%%%%%%%%%%%%%%%%%%%%%%%%%%%%%%%%%%%%%%%%%%%%

\bgroup
% Dan: 333-6209 (office) 367-6384 (home)

\CompileMatrices

\newtheorem{sRemark}{Remark}{\bfseries}{\rm}
\newtheorem{notation}[theorem]{Notation}{\bfseries}{\rm}
\newtheorem{construction}[theorem]{Construction}{\bfseries}{\rm}
\numberwithin{sRemark}{subsection}
\newtheorem{sExample}[sRemark]{Example}{\bfseries}{\rm}
\newtheorem{sCode}[sRemark]{Code}{\bfseries}{\rm}

\title{Resolutions and Cohomology \\ over Complete Intersections}
\titlerunning{Complete Intersections}
\toctitle{Resolutions and Cohomology over Complete Intersections}

\author{Luchezar L. Avramov
        %\inst 1
   \and Daniel R. Grayson%
        %\inst 2
        %\fnmsep
        \thanks{Authors supported by the NSF, grants DMS 99-70375 and
        DMS 99-70085.}}
\authorrunning{L. L. Avramov and D. R. Grayson}
% \institute{Purdue University, Department of Mathematics
%         % \endgraf {\tt http://www.math.purdue.edu/\char`\~avramov}
%         % \endgraf {\tt avramov\char`\@math.purdue.edu}
%       \and University of Illinois at Urbana-Champaign,
%         Department of Mathematics
%         % \endgraf {\tt http://www.math.uiuc.edu/\char`\~dan}
%         % \endgraf {\tt dan\char`\@math.uiuc.edu}
% }

\hyphenation{quasi-iso-mor-phism}

\newcommand\ssum[1]{{\underset{#1}{\sum{\vphantom\sum}^{\scriptscriptstyle+}}}}

\def\ann{{\operatorname{ann}}}
\def\inv{{\operatorname{inv}}}
\def\reg{{\operatorname{reg}}}
\def\gr{{\operatorname{gr}}}
\def\bu{{\scriptscriptstyle\bullet}}
\def\HH{{\operatorname{H}}}
\def\Tor{\operatorname{Tor}}
\def\Ext{\operatorname{Ext}}
\def\rExt{\operatorname{ext}}
\def\Deg{\operatorname{Deg}}
\def\Poi{{P}}
\def\Ba{{I}}
\def\gen{{G}}
\def\depth{\operatorname{depth}}
\def\pd{\operatorname{pd}}
\def\cx{\operatorname{cx}}
\def\crdeg{\operatorname{crdeg}}
\def\rank{\operatorname{rank}}
\def\var{\operatorname{V}}
\def\lcontract{\operatorname{\lrcorner}}
\def\Hom{\operatorname{Hom}}
\def\Coker{\operatorname{Coker}}
\def\Ker{\operatorname{Ker}}
\def\Ima{\operatorname{Im}}
\def\C{{\mathbb C}}
\def\F{{\mathbb F}}
\def\N{{\mathbb N}}
\def\Z{{\mathbb Z}}
\def\DD{{\mathsf D}}
\def\SS{{\mathsf S}}
\def\Wedge{{\textstyle\bigwedge\limits}}
\def\a{\alpha}
\def\b{\beta}
\def\g{\gamma}
\def\d{\delta}
\def\e{\epsilon}
\def\o{\otimes}

\maketitle
\begin{abstract}
This chapter contains a new proof and new applications of a theorem of
Shamash and Eisenbud, providing a construction of projective
resolutions of modules over a complete intersection.  The duals of
these infinite projective resolutions are finitely generated
differential graded modules over a graded polynomial ring, so they can
be represented in the computer, and can be used to compute $\Ext$
modules simultaneously in all homological degrees.  It is shown how to
write \Mtwo code to implement the construction, and how to use the
computer to determine invariants of modules over complete intersections
that are difficult to obtain otherwise.
\index{complete intersection}
  \end{abstract}

\section*{Introduction}
\label{sec:introduction}

Let $A=K[x_1,\dots,x_e]$ be a polynomial ring with variables of
positive degree over a field $K$, and $B=A/J$ a quotient ring modulo a
homogeneous ideal.

In this paper we consider the case when $B$ is a {\it\ie{graded
complete intersection}\/}, that is, when the defining ideal $J$ is
generated by a homogeneous $A$-regular sequence.  We set up, describe,
and illustrate a routine {\tt Ext}\indexcmd{Ext}, now implemented in
\Mtwo.  For any two finitely generated graded $B$-modules $M$ and $N$
it yields a presentation of $\Ext^\bu_B(M,N)$ as a bigraded module over
an appropriately bigraded polynomial ring $S=A[X_1,\dots,X_c]$.

A novel feature of our routine is that it computes the modules
$\Ext^n_B(M,N)$ {\it simultaneously in all cohomological degrees\/}
$n\ge0$.  This is made possible by the use of {\it cohomology
operations\/}, a technique usually confined to theoretical
considerations.  Another aspect worth noticing is that, although the
result is over a ring $B$ with nontrivial relations, all the
computations are made over the {\it polynomial ring\/} $S$; this may
account for the effectiveness of the algorithm.

To explain the role of the complete intersection hypothesis, we cast it
into the broader context of homological algebra over graded rings.

Numerous results indicate that the high syzygy modules of $M$ exhibit
`similar' properties.  For an outrageous example, assume that $M$ has
finite projective dimension.  Its distant syzygies are then all equal
to $0$, and so---for trivial reasons---display an extremely uniform
behavior.  However, even this case has a highly nontrivial aspect: due
to the Auslander-Buchsbaum Equality asymptotic information is available
after at most $(e+1)$ steps.  This accounts for the effectiveness of
computer constructions of {\it finite\/} free resolutions.

Problems that computers are not well equipped to handle arise
unavoidably when studying asymptotic behavior of {\it infinite\/}
resolutions.  We describe some, using graded Betti numbers
$\b^B_{ns}(M)=\dim_K\Ext^n_B(M,k)_{-s}$, where $k=B/(x_1,\dots,x_e)B$,
and regularity $\reg_B(M)= \sup_{n,s}\{s-n\,|\, \b^B_{ns}(M)\ne0\}$.
\begin{itemize}
\item[$\bullet$]
{\it Irrationality\/}.
There are rings $B$ for which no recurrent relation with constant
coefficients exists among the numbers $\b^B_n(k)=\sum_{s}\b^B_{ns}(k)$,
see \cite{CI:MR86i:55011a}.
\item[$\bullet$]
{\it Irregularity\/}.
For each $r\ge2$ there exists a ring $B(r)$ with $\b^{B(r)}_{ns}(k)=0$
for $s\ne n$ and $0\le n\le r$, but with $\b^{B(r)}_{r,r+1}(k)\ne0$,
see \cite{CI:MR94b:16040}.
\item[$\bullet$]
{\it Span\/}.
If $B$ is generated over $K$ by elements of degree one and $\reg_B(k)
\ne0$, then $\reg_B(k) =\infty$, see \cite{CI:AP}.
\item[$\bullet$]
{\it Size\/}.
There are inequalities $\b^B_n(k)\ge\beta^n$ for all $n\ge0$ and for
some constant $\beta>1$, unless $B$ is a complete intersection, see
\cite{CI:res}.
 \end{itemize}

These obstructions vanish miraculously when $B$ is a graded complete
intersection:  For each $M$ and all $n\gg0$ the number $\b_{n+1}(M)$ is
a linear combination with constant coefficients of $\b^B_{n-2c}(M),
\dots,\b^B_{n}(M)$. If $B$ is generated in degree one, then
$\reg_B(k)=0$ if and only if the ideal $J$ is generated by quadratic
forms.  There are inequalities $\b^B_n(M)\le\beta(M)n^{c-1}$ for all
$n\ge1$ and for some constant $\beta(M)>0$.

The algebra behind the miracle is a theorem of Gulliksen
\cite{CI:MR51:487}, who proves that $\Ext^\bu_B(M,N)$ is a finitely
generated bigraded module over a polynomial {\it ring of cohomology
operators\/} $S=A[X_1,\dots,X_c]$, where each variable $X_i$ has
cohomological degree $2$.  As a consequence of this result, problems in
Homological Algebra can be answered in terms of Commutative Algebra.

Gulliksen's definition of the operators $X_i$ as iterated connecting
homomorphisms is badly suited for use by a computer.  Other definitions
have been given subsequently by several authors, see Remark
\ref{history}.  We take the approach of Eisenbud \cite{CI:Ei}, who
derives the operators from a specific $B$-free resolution of $M$,
obtained by extending a construction of Shamash \cite{CI:Sh}.

The resolution of Shamash and Eisenbud, and Gulliksen's Finiteness
Theorem, are presented with detailed proofs in Section \ref{Cohomology
operators}.  They are obtained through a new construction---that of an
intermediate resolution of $M$ over the polynomial ring---that encodes
$C$ and all the null-homotopies of $C$ corresponding to multiplication
with elements of $J$; this material is contained in Section
\ref{Universal homotopies}.  It needs standard multilinear algebra,
developed {\sl ad hoc\/} in Section \ref{Graded algebras}.  Rules for
juggling several gradings are discussed in an Appendix.

In Section \ref{Computation of Ext modules} we present and illustrate
the code for the routine {\tt Ext}, which runs remarkably close to the
proofs in Sections \ref{Universal homotopies} and \ref{Cohomology
operators}.  Section \ref{Invariants of modules} contains numerous
computations of popular numerical invariants of a graded module, like
its complexity, Poincar\'e series, and Bass series.  They are extracted
from knowledge of the bigraded modules $\Ext^\bu_B(M,k)$ and
$\Ext^\bu_B(k,M)$, whose computation is also illustrated by examples,
and is further used to obtain explicit equations for the cohomology
variety $\var^*_B(M)$ defined in \cite{CI:MR90g:13027}.  For most
invariants we include some short code that automates their
computation.  In Section \ref{Invariants of pairs of modules} we
extend these procedures to invariants of pairs of modules.

\section{Matrix Factorizations}
\label{Matrix factorizations}

We start the discussion of homological algebra over a complete
intersection with a very special case, that can be packaged
attractively in matrix terms.

Let $f$ be a non-zero-divisor in a commutative ring $A$.

Following Eisenbud \cite[Sect.\ 5]{CI:Ei} we say that a pair $(U,V)$ of
matrices with entries in $A$, of sizes $k\times \ell$ and $\ell\times
k$, is a {\it\ie{matrix factorization}\/} of $-f$ if
\[
U\cdot V= -f\cdot I_k \qquad\text{and}\qquad V\cdot U=-f\cdot I_\ell
\]
where $I_m$ denotes the $m\times m$ unit matrix.  Localizing at $f$, one
sees that $-f^{-1}\cdot U$ and $V$ are inverse matrices over $A_f$; as a
consequence $\ell=k$, and each equality above implies the other, for
instance:
\[
V\cdot U=\big(-f^{-1}\cdot U\big)^{-1}\cdot U= -f\cdot U^{-1}\cdot U=
-f\cdot I_k
\]
Here is a familiar example of matrix factorization, with $f=xy-wz$:
\[
\begin{pmatrix} w & x \\ y & z \end{pmatrix}\cdot
\begin{pmatrix} z & -y \\ -x & w \end{pmatrix}=
-(xy-wz)\cdot
\begin{pmatrix} 1 & 0 \\ 0 & 1 \end{pmatrix}=
\begin{pmatrix} z & -y \\ -x & w \end{pmatrix}\cdot
\begin{pmatrix} w & x \\ y & z \end{pmatrix}
\]

Let now $C_1$ and $C_0$ be free $A$-modules of rank $r$, and let
\[
d_1\colon C_1\to C_0 \qquad\text{and}\qquad s_0\colon C_0\to C_1
\]
be $A$-linear homomorphisms defined by the matrices $U$ and $V$,
respectively, after bases have been tacitly chosen.  

The second condition on the matrices $U$ and $V$ implies that $d_1$ is
injective, while the first condition on these matrices shows that
$fC_0$ is contained in $\Ima(d_1)$.  Setting $L=\Coker(d_1)$, one sees
that the chosen matrix factorization defines a commutative diagram with
exact rows
\[
\xymatrixrowsep{3pc}
\xymatrixcolsep{4pc}
\xymatrix{
0
\ar[r]
& C_1
\ar[r]^-{d_1}
\ar[d]_{-f\cdot 1_{C_1}}
& C_0
\ar[r]
\ar[d]^{-f\cdot 1_{C_0}}
\ar[dl]_-{s_0}
& L
\ar[r]
\ar[d]_{0_L=}^{-f\cdot1_L}
& 0
\\
0
\ar[r]
& C_1
\ar[r]^-{d_1}
& C_0
\ar[r]
& L
\ar[r]
& 0
}
\]
which expresses the following facts: $C=\ 0\to C_1\xrightarrow{d_1}
C_0\to 0$ is a free resolution of the $A$-module $L$,  this module is
annihilated by $f$, and $s_0$ is a homotopy between the maps
$-f\cdot 1_C$ and $0_C$, both of which lift $-f\cdot 1_L$.

Conversely, if an $A$-module $L$ annihilated by $f$ has a free
resolution $(C,d_1)$ of length $1$, then $\rank_AC_1=\rank_AC_0$, and
any choice of homotopy $s_0$ between $-f\cdot 1_C$ and $0_C$ provides a
matrix factorization of $-f$.

When we already have an $A$-module $L$ with a presentation matrix $U$
that defines an injective $A$-linear map, we can use \Mtwo to create a
matrix factorization $(U,V)$ of $-f$.

\begin{Example}
\label{familiar}
We revisit the familiar example from a higher perspective.
\beginOutput
i1 : A = QQ[w,x,y,z]\\
\emptyLine
o1 = A\\
\emptyLine
o1 : PolynomialRing\\
\endOutput
\beginOutput
i2 : U = matrix \{\{w,x\},\{y,z\}\}\\
\emptyLine
o2 = | w x |\\
\     | y z |\\
\emptyLine
\             2       2\\
o2 : Matrix A  <--- A\\
\endOutput
\beginOutput
i3 : C = chainComplex U\\
\emptyLine
\      2      2\\
o3 = A  <-- A\\
\             \\
\     0      1\\
\emptyLine
o3 : ChainComplex\\
\endOutput
\beginOutput
i4 : L = HH_0 C\\
\emptyLine
o4 = cokernel | w x |\\
\              | y z |\\
\emptyLine
\                            2\\
o4 : A-module, quotient of A\\
\endOutput
\beginOutput
i5 : f = -det U\\
\emptyLine
o5 = x*y - w*z\\
\emptyLine
o5 : A\\
\endOutput
Let's verify that $f$ annihilates $L$.
\beginOutput
i6 : f * L == 0\\
\emptyLine
o6 = true\\
\endOutput
We use the {\tt nullhomotopy} function.\indexcmd{nullhomotopy}
\beginOutput
i7 : s = nullhomotopy (-f * id_C)\\
\emptyLine
\          2                     2\\
o7 = 1 : A  <----------------- A  : 0\\
\               \{1\} | z  -x |\\
\               \{1\} | -y w  |\\
\emptyLine
o7 : ChainComplexMap\\
\endOutput
Let's verify that $s$ is a null-homotopy for $-f$, using {\tt
C.dd}\indexcmd{dd} to obtain the differential of the chain complex {\tt
C} as a map of graded modules.
\beginOutput
i8 : s * C.dd + C.dd * s == -f\\
\emptyLine
o8 = true\\
\endOutput
We extract the matrix $V$ from the null-homotopy to get our factorization.
\beginOutput
i9 : V = s_0\\
\emptyLine
o9 = \{1\} | z  -x |\\
\     \{1\} | -y w  |\\
\emptyLine
\             2       2\\
o9 : Matrix A  <--- A\\
\endOutput
\end{Example}

For every $f$ and every $r\ge1$ there exists a trivial matrix
factorization of $-f$, namely, $(f\cdot I_k, -I_k)$; it can be obtained
from the $A$-module $L=A^k/fA^k$.  In general, it may not be clear how
to find an $A$-module $L$ with the properties necessary to obtain an
`interesting' matrix factorization of $-f$.

However, in some cases the supply is plentiful.

\begin{Remark}
\label{factorization}
Let $A$ be a graded polynomial ring in $e$ variables of positive degree
over a field $K$, let $f$ be a homogeneous polynomial in $A$, and set
$B=A/(f)$.  Every $B$-module $M$ of infinite projective dimension
{\it\ie{generates}\/} a matrix factorization $(U,V)$ of $-f$, as follows.

Let $(F,d_F)$ be a minimal graded free resolution of $M$ over $B$, and
set $L=\Coker\big(d_F\colon F_{e+1}\to F_e\big)$.  Since $M$ has
infinite projective dimension, we have $L\ne0$.  By the Depth Lemma,
$\depth_BL=\depth B$.  On the other hand, $\depth_BL=\depth_AL$ and
$\depth B=\depth A-1$.  By Hilbert's Syzygy Theorem, the minimal graded
free resolution $(C,d_C)$ of $L$ over $A$ is finite.  By the
Auslander-Buchsbaum Equality, $C_n=0$ for $n>\depth A-\depth_AL=1$.

The minimality of $F$ ensures that all nonzero entries of the
presentation matrix $U$ of $L$ are forms of positive degree.  On
the other hand, by \cite[Sect.~0]{CI:Ei} the module $L$ has no direct
summand isomorphic to $B$: it follows that all nonzero entries of
the homotopy matrix $V$ are forms of positive degree (this is the
reason for choosing $L$ as above---stopping one step earlier in the
resolution $F$ could have produced a module $L$ with a non-zero free
direct summand).
 \end{Remark}

Our reader would have noticed that \Mtwo can read all the data and
perform all the operations needed to construct a module $L$ by the
procedure described in the preceding remark.  Here is how it does it.

\begin{Example}
\label{square}
We produce a matrix factorization of $-f$, where 
\[
f=x^3 + 3y^3 - 2yz^2 + 5z^3 \in\mathbb Q[x,y,z]=A
\]
generated by the module $M=B/{\mathfrak m}^2$, where $B=A/(f)$ and
${\mathfrak m}=(x,y,z)B$.
\beginOutput
i10 : A = QQ[x,y,z];\\
\endOutput
\beginOutput
i11 : f = x^3 + 3*y^3 - 2*y*z^2 + 5*z^3;\\
\endOutput
\beginOutput
i12 : B = A/f;\\
\endOutput
\beginOutput
i13 : m = ideal(x,y,z)\\
\emptyLine
o13 = ideal (x, y, z)\\
\emptyLine
o13 : Ideal of B\\
\endOutput
Let's take the $B$-module $M$ and compute its minimal $B$-free
resolution.
\beginOutput
i14 : M = B^1/m^2;\\
\endOutput
\beginOutput
i15 : F = resolution(M, LengthLimit=>8)\\
\emptyLine
\       1      6      9      9      9      9      9      9      9\\
o15 = B  <-- B  <-- B  <-- B  <-- B  <-- B  <-- B  <-- B  <-- B\\
\                                                               \\
\      0      1      2      3      4      5      6      7      8\\
\emptyLine
o15 : ChainComplex\\
\endOutput
We introduce a function {\tt restrict1 N}  which accepts a $B$-module
$N$ and restricts scalars to produce an $A$-module.
\beginOutput
i16 : restrict1 = N -> coker(lift(presentation N,A) | f);\\
\endOutput
Now make $L$ as described in Remark \ref{factorization}
\beginOutput
i17 : L = restrict1 cokernel F.dd_4;\\
\endOutput
We proceed as in Example \ref{familiar} to get a matrix factorization.
\beginOutput
i18 : C = res L;\\
\endOutput
\beginOutput
i19 : U = C.dd_1;\\
\emptyLine
\              9       9\\
o19 : Matrix A  <--- A\\
\endOutput
\beginOutput
i20 : print U\\
\{4\} | 0  xy x2       y2    0        0        0        yz-5/2z2 0      |\\
\{4\} | 0  x2 -3y2     xy    yz-5/2z2 0        yz-5/2z2 0        0      |\\
\{4\} | x2 0  -2yz+5z2 0     y2-5/2yz yz-5/2z2 -5/2yz   0        0      |\\
\{5\} | 0  0  0        1/3z  0        0        0        1/2y     x      |\\
\{5\} | 0  0  -z       0     1/2y     0        1/2y     -1/2x    0      |\\
\{5\} | 0  -z 0        0     -1/2x    0        -1/2x    0        3y     |\\
\{5\} | 0  0  0        -1/3x 0        1/2y     -1/3z    0        0      |\\
\{5\} | -z y  x        0     0        -1/2x    0        0        0      |\\
\{5\} | y  0  0        0     0        0        1/3x     0        -2y+5z |\\
\endOutput
\beginOutput
i21 : s = nullhomotopy (-f * id_C);\\
\endOutput
\beginOutput
i22 : V = s_0;\\
\emptyLine
\              9       9\\
o22 : Matrix A  <--- A\\
\endOutput
\beginOutput
i23 : print V\\
\{6\} | 0   0  -x  0         0         -2y2+5yz 0         -2yz+5z2 -3y2 |\\
\{6\} | 0   -x 0   0         0         -2yz+5z2 -3xy      -3y2     -3yz |\\
\{6\} | -x  y  0   0         -2yz+5z2  0        0         0        0    |\\
\{6\} | -3y 0  0   6yz-15z2  0         0        3x2       3xy      3xz  |\\
\{6\} | 0   2z -3y -15xz     -15yz     2x2      6yz-15z2  0        3x2  |\\
\{6\} | -2x 0  2z  0         -4yz+10z2 0        -6y2      2x2      0    |\\
\{6\} | 0   0  3y  -6xy+15xz -6y2+15yz 0        -6yz+15z2 0        -3x2 |\\
\{6\} | 2z  0  0   -6y2      2x2       2xy      0         0        0    |\\
\{6\} | 0   0  0   -x2       -xy       -y2      -xz       -yz      -z2  |\\
\endOutput
\beginOutput
i24 : U*V+f==0\\
\emptyLine
o24 = true\\
\endOutput
\beginOutput
i25 : V*U+f==0\\
\emptyLine
o25 = true\\
\endOutput
\end{Example}

The procedure described above can be automated for more pleasant usage.

\begin{code}
\label{factorization code}
The function {\tt matrixFactorization M} produces a matrix factorization
$(U,V)$ of $-f$ generated by a module $M$ over $B=A/(f)$.
\beginOutput
i26 : matrixFactorization = M -> (\\
\         B := ring M;\\
\         f := (ideal B)_0;\\
\         e := numgens B;\\
\         F := resolution(M, LengthLimit => e+1);\\
\         L := restrict1 cokernel F.dd_(e+1);\\
\         C := res L;\\
\         U := C.dd_1;\\
\         s := nullhomotopy (-f * id_C);\\
\         V := s_0;\\
\         assert( U*V + f == 0 );\\
\         assert( V*U + f == 0 );\\
\         return (U,V));\\
\endOutput
We use the {\tt assert}\indexcmd{assert} command to signal an error in
case the matrices found don't satisfy our requirement for a matrix
factorization.
 \end{code}

Let's illustrate the new code with a slightly bigger module $M$ than
before.

\begin{Example}
\label{cube}
With the same $A$, $f$, $B$, and $\mathfrak m$ as in Example
\ref{square}, we produce a matrix factorization generated by the
$B$-module $M=B/{\mathfrak m}^3$.
\beginOutput
i27 : time (U,V) = matrixFactorization(B^1/m^3);\\
\     -- used 0.21 seconds\\
\endOutput
The parallel assignment statement above provides both variables {\tt U}
and {\tt V} with matrix values.  We examine their shapes without
viewing the matrices themselves by appending a semicolon to the
appropriate command.
\beginOutput
i28 : U;\\
\emptyLine
\              15       15\\
o28 : Matrix A   <--- A\\
\endOutput
\beginOutput
i29 : V;\\
\emptyLine
\              15       15\\
o29 : Matrix A   <--- A\\
\endOutput
\end{Example}

Matrix factorizations were introduced to construct resolutions over the
the residue ring $B=A/(f)$, using the following observation.

\begin{Remark}
\label{periodicity}
If $(U,V)$ is a factorization of $-f$ by $k\times k$ matrices and the
maps $d_1\colon C_1\to C_0$ and $s_0\colon C_0\to C_1$ are
homomorphisms of free $A$-modules defined by $U$ and $V$, respectively,
then the sequence
\[
\cdots \to
C_1\otimes_AB \xrightarrow{d_1\o 1_B}
C_0\otimes_AB \xrightarrow{s_0\o 1_B}
C_1\otimes_AB \xrightarrow{d_1\o 1_B}
C_0\otimes_AB\to0
\]
of $B$-linear maps is a free resolution of the $B$-module
$L=\Coker(d_1)$.

Indeed, freeness is clear, and we have a complex because $d_1s_0=
-f\cdot 1_{C_0}$ and $s_0d_1=-f\cdot 1_{C_1}$.  If $x\in C_1$ satisfies
$\big(d_1\o1_B\big)(x\o 1)=0$, then $d_1(x)=fy$ for some $y\in C_0$,
hence $d_1x=d_1s_0(y)$.  As $d_1$ is injective, we get $x=s_0(y)$, so
$\Ker\big(d_1\o1_B\big)\subseteq \Ima\big(s_0\o1_B\big)$; the reverse
inclusion follows by symmetry.
 \end{Remark}

Pooling Remarks \ref{factorization} and \ref{periodicity} we recover
Eisenbud's result \cite[Sect.\ 6]{CI:Ei}.

\begin{theorem}
Let $A$ be a graded polynomial ring in $e$ variables of positive degree
over a field $K$, and $f$ a homogeneous polynomial in $A$.  The minimal
graded free resolution of every finitely generated graded module over
$B=A/(f)$ becomes periodic of period $2$ after at most $e$ steps.  The
periodic part of the resolution is given by a matrix factorization of
$-f$ generated by $M$.
 \end{theorem}

We illustrate the theorem on an already computed example.

\begin{Example}
Let $A$, $f$, $B$, $M$, and $F$ be as in Example \ref{square}.

To verify the periodicity of $F$ we subtract pairs of differentials and
compare the result with $0$: direct comparison of the differentials 
would not work, because the free modules involved have different degrees.
\beginOutput
i30 : F.dd_3 - F.dd_5 == 0\\
\emptyLine
o30 = false\\
\endOutput
\beginOutput
i31 : F.dd_4 - F.dd_6 == 0\\
\emptyLine
o31 = false\\
\endOutput
\beginOutput
i32 : F.dd_5 - F.dd_7 == 0\\
\emptyLine
o32 = true\\
\endOutput
The first two answers above come as a surprise---and suggest a property
of $F$ that is weaker than the one we already know to be true!  

There is an easy explanation: we checked the syzygy modules for {\it
equality\/}, rather than for {\it isomorphism\/}.  We do not know why
\Mtwo didn't produce an equality at the earliest possible stage, nor why it
eventually produced one.  The program has other strategies for computing
resolutions, so let's try one.
\beginOutput
i33 : M = B^1/m^2;\\
\endOutput
\beginOutput
i34 : G = resolution(M, LengthLimit => 8, Strategy => 0)\\
\emptyLine
\       1      6      9      9      9      9      9      9      9\\
o34 = B  <-- B  <-- B  <-- B  <-- B  <-- B  <-- B  <-- B  <-- B\\
\                                                               \\
\      0      1      2      3      4      5      6      7      8\\
\emptyLine
o34 : ChainComplex\\
\endOutput
\beginOutput
i35 : G.dd_3 - G.dd_5 == 0\\
\emptyLine
o35 = true\\
\endOutput
\beginOutput
i36 : G.dd_4 - G.dd_6 == 0\\
\emptyLine
o36 = true\\
\endOutput
\beginOutput
i37 : G.dd_5 - G.dd_7 == 0\\
\emptyLine
o37 = true\\
\endOutput
The strategy paid off, revealing periodicity at the earliest possible
stage.  However, the algorithm used may be a lot slower that the
default algorithm.
 \end{Example}

\section{Graded Algebras}
\label{Graded algebras}

We describe some standard universal algebras over a commutative ring $A$.

Let $Q$ denote a free $A$-module of rank $c$, and set $Q^*=\Hom_A(Q,A)$.
We assign degree $2$ to the elements of $Q$, and degree $-2$ to those
of $Q^*$.  We let $Q^\wedge$ denote a copy of $Q$ whose elements are
assigned degree $1$; if $x$ is an element of $Q$, then $x^\wedge$
denotes the corresponding element of $Q^\wedge$.

We use $\a = (\a_1 , \dots , \a_c ) \in \Z^c$ as a multi-index, set
$|\a| = \sum_i \a_i$, and order $\Z^c$ by the rule: $\a \ge \b$ means
$\a_i \ge \b_i$ for each $i$.  We let $o$ denote the trivial element
of $\Z^c$, and $\e_i$ the $i$'th element of its standard basis.

\begin{construction}
\label{algebras}
For each integer ${m}\ge0$ we form free $A$-modules
\begin{gather*}
\SS^{m}(Q^*) \quad\text{with basis}\quad \big\{X^{\a} : |\a|={m}\big\}\\
\DD^{m}(Q) \quad\text{with basis}\quad \big\{Y^{(\a)} : |\a|={m}\big\}\\
\Wedge^{m}(Q^\wedge) \quad\text{with basis}\quad 
\big\{Y^{\wedge\a} : |\a|={m} \quad\text{and}\quad\a\le(\e_1+\cdots+\e_c)\big\}
\end{gather*}
For $m<0$ we declare the modules $\SS^{m}(Q^*)$, $\DD^{m}(Q)$,
and $\Wedge^{m}(Q^\wedge)$ to be equal to $0$, and define the symbols
$X^{\a}$, $Y^{(\a)}$, and $Y^{\wedge\a}$ accordingly; in addition,
we set $\Wedge^{m}(Q^\wedge)=0$ and $Y^{\wedge\a}=0$ if $|\a|\not\le
(\e_1+\cdots+\e_c)$, and we set
\begin{gather*}
X_i=X^{\e_i}\qquad Y_i=Y^{(\e_i)}\qquad Y_i^\wedge=Y^{\wedge\e_i}
\qquad\text{for}\qquad i=1,\dots,c
\end{gather*}
Taking $\SS^{m}(Q^*)$, $\DD^{m}(Q)$, and $\Wedge{}^{m}(Q^\wedge)$ as
homogeneous components of degree $-2m$, $2m$, and $m$, respectively,
we introduce graded algebras
\[
S = \SS(Q^*)\qquad
D = \DD(Q)\qquad
E = \Wedge(Q^\wedge)
\]
by defining products of basis elements by the formulas
\begin{gather*}
X^{\a}\cdot X^{\b}=X^{\a+\b}\\
Y^{(\a)}\cdot Y^{(\b)}=
\prod_{i=1}^c\frac{(\a_i+\b_i)!}{\a_i!\b_i!}Y^{(\a+\b)}\\
Y^{\wedge\a}\cdot Y^{\wedge\b}=\inv(\a,\b)Y^{\wedge\,\a+\b}
\end{gather*}
where $\inv(\a,\b)$ denotes the number of pairs $(i,j)$ with
$\a_i=\b_j=1$ and $i>j$.  Thus, $S$ is the {\it\ie{symmetric
algebra}\/} of $Q^*$, with $X^{o}=1$, while $D$ is the {\it\ie{divided
powers algebra}\/} of $Q$, with $Y^{(o)}=1$, and $E$ is the
{\it\ie{exterior algebra}\/} of $Q^\wedge$, with $Y^{\wedge\,o}=1$.  We
identify $S$ and the polynomial ring $A[X_1,\dots,X_c]$.
 \end{construction}

A {\it\ie{homogeneous derivation}\/} of a graded $A$-algebra $W$ is a
homogeneous $A$-linear map $d\colon W\to W$ such that the
{\it\ie{Leibniz rule}\/}
\[
d(x y) = d(x) y + (-1)^{\deg x\cdot\deg d} x d(y)
\]
holds for all homogeneous elements $x,y\in W$.

\begin{construction}
\label{koszul}
Each sequence $f_1,\dots,f_c\in A$ yields a {\it\ie{Koszul map}\/}
\[
\begin{gathered}
d_E \colon E \to E
\qquad\text{defined by the formula}\\
d_E(Y^{\wedge\b}) =
\sum_{i=1}^c (-1)^{\b_1+\cdots+\b_{i-1}}f_iY^{\wedge\, \b-\e_i}
\end{gathered}
\]
It is a derivation of degree $-1$ and satisfies $d_E^2 = 0$.
 \end{construction}

\begin{construction}
\label{actions}
For every $X_i\in\SS^1(Q^*)$ and each $Y^{(\b)} \in \DD^{m}(Q)$ we set
\[
X_i \lcontract Y^{(\b)} = Y^{(\b-\e_i)} \in \DD^{{m}-1}(Q)
\]
Extending this formula by $A$-bilinearity, we define $g\lcontract y$ for 
all $g\in\SS^1(Q^*)$ and all $y\in D$.  It is well known, and easily
verified, that the map $g\lcontract\colon y\mapsto g\lcontract y$ is a
graded derivation $D\to D$ of degree $-2$, and that the derivations
associated with arbitrary $g$ and $g'$ commute.  As a consequence, the
formula
\[
X^\a\lcontract Y^{(\b)} =
(X_1\lcontract)^{\a_1}\cdots(X_c\lcontract)^{\a_c}\big(Y^{(\b)}\big)
\in\DD^{|\b-\a|}(Q)
\]
extended $A$-linearly to all $u\in S$, defines on $D$ a structure of
graded $S$-module.

The usual products on $S\otimes_AE$ and $D\otimes_A E$ and the induced
gradings
\begin{gather*}
(S\otimes_AE)_n=
\bigoplus_{\ell-2k=n}\SS^{k}(Q^*)\otimes_A \Wedge{}^{\ell}(Q^\wedge)\\
(D\otimes_AE)_n=
\bigoplus_{\ell+2k=n}\DD^{(k)}(Q)\otimes_A \Wedge{}^{\ell}(Q^\wedge)
\end{gather*}
turn $S\otimes_AE$ and $D\otimes_A E$ into graded algebras.  The
second one is a graded module over the first, for the action
$(u\o z)\cdot(y\o z')=(u\lcontract v)\o(z\cdot z')$.
  \end{construction}

\begin{construction}
\label{cartan}
The element $w=\sum_{i=1}^c X_i\o Y_i^\wedge$ yields a {\it\ie{Cartan
map}\/} 
\[
\begin{gathered}
d_{DE} \colon D \otimes_A E \to D \otimes_A E
\qquad\text{defined by the formula}\\
d_{DE}( y \o z ) = w\cdot(y \o z)
= \sum_{i=1}^c (X_i \lcontract y) \o (Y_i^\wedge\cdot z)
\end{gathered}
\]
It is an $E$-linear derivation of degree $-1$, and $d_{DE}^2 = 0$
because $w^2=0$.  
 \end{construction}

\begin{lemma}
\label{split}
For each integer $s$ define a complex $G^s$ as follows:
\[
\cdots\to \DD^{k}(Q)\otimes_A \Wedge^{s-k}(Q^\wedge)\xrightarrow{w}
\DD^{k-1}(Q)\otimes_A \Wedge^{s-k+1}(Q^\wedge) \to \cdots 
\]
with $\DD^{0}(Q)\otimes_A \Wedge^{s}(Q^\wedge)$ in degree $s$.  If
$s>0$, then $G^s$ is split exact.
\end{lemma}

\begin{proof}
Note that for each $s\in\Z$ there exist isomorphisms of complexes
$\bigoplus_{s=1}^\infty G^s\cong (D \otimes_A E)_{\ge1} \cong
\big(\bigotimes_{i=1}^c G(i)\big)_{\ge1}$, where $G(i)$ is the complex
\[
\cdots\to
AY^{(k+1)}_i\otimes_A A
\xrightarrow{w_i}(AY^{(k)}_i)\otimes(AY^\wedge_i)
\xrightarrow{0}(AY^{(k)}_i)\otimes_A A
\to\cdots
\]
and $w_i$ is left multiplication with $X_i\o Y_i^\wedge$.  This map
bijective, so each complex $G(i)_{\ge1}$ is split exact.  The assertion
follows.  \qed
 \end{proof}

\section{Universal Homotopies}
\label{Universal homotopies}

This section contains the main new mathematical result of the paper.  

We introduce a universal construction, that takes as input a projective
resolution $C$ of an $A$-module $M$ and a finite set $\boldsymbol f$ of
elements annihilating $M$; the output is a new projective resolution of
$M$ over $A$.  If $\boldsymbol f\ne\varnothing$, then the new
resolution is infinite---even when $C$ is finite---because it encodes
additional data: the null-homotopies for $f\cdot1_C$ for all
$f\in\boldsymbol f$, all compositions of such homotopies, and all
relations between those compositions.  This higher-order information
tracks the transformation of the homological properties of $M$ when its
ring of operators is changed from $A$ to $A/({\boldsymbol f})$.

Our construction is motivated by, and is similar to, one due to Shamash
\cite{CI:Sh} and Eisenbud \cite{CI:Ei}: assuming that the elements of
$\boldsymbol f$ form an $A$-regular sequence, they produce a projective
resolution of $M$ over $A/({\boldsymbol f})$.  By contrast, we make no
assumption whatsoever on $\boldsymbol f$.  With the additional
hypothesis, in the next section we quickly recover the original result
from the one below.  As an added benefit, we eliminate the use of
spectral sequences from the proof.

\begin{theorem}
\label{main}
Let $A$ be a commutative ring, let $f_1 , \dots, f_c$ be a sequence of
elements of $A$, let $M$ be an $A$-module annihilated by $f_i$ for
$i=1,\dots,c$, and let $r \colon  C \to M$ be a resolution of $M$ by
projective (respectively, free) $A$-modules.

There exists a family of homogeneous $A$-linear maps 
\[
\{d_\g \colon C \to C\mid \deg(d_\g)=2|\g| - 1\}_{\g \in \N^c}
\]
satisfying the following conditions
\begin{equation}
\label{family}
\begin{aligned}
d_o &= d_C
\quad\text{is the differential of}\quad C\\
[d_o,d_\g] &=
\begin{cases}
-f_i\cdot 1_C &\text{if\quad} \g=\e_i \text{ for } i=1,\dots,c\\
-\ssum{\a+\b=\g}d_\a d_\b      &\text{if\quad} |\g|\ge2
\end{cases}
\end{aligned}
\end{equation}
where $\sum^{\scriptscriptstyle +}$ denotes a summation restricted to
indices in $\N^c\smallsetminus\{o\}$.

Any family $\{d_\g\}_{\g\in\N^c}$ as above defines an $A$-linear map
of degree $-1$, 
\begin{equation}
\begin{gathered}
\label{dCD}
d_{CD} \colon C\otimes_AD \to C\otimes_AD 
\qquad\text{given by}\\
d_{CD}(x\o y) = \sum_{\g \in\N^c} d_\g(x) \o (X^{\g} \lcontract y)
\end{gathered}
\end{equation}
where $D$ is the divided powers algebra defined in Construction
\ref{algebras}, and the action of $X^{\g}$ on $D$ is defined in
Construction \ref{actions}.

With $d_{E}$ and $d_{DE}$ defined in Constructions \ref{koszul} and
\ref{cartan} and the tensor product of maps of graded modules defined
as in Remark \ref{graded-map-tensor},
the map
\begin{equation}
\begin{gathered}
\label{diff}
d\colon C \otimes_A D \otimes_A E \to C \otimes_A D \otimes_A E
\qquad\text{given by}\\
d = d_{CD}\o 1_E + 1_C\o d_{DE} + 1_C\o 1_D\o d_E 
\end{gathered}
\end{equation}
is an $A$-linear differential of degree $-1$, and the map
\begin{gather*}
q\colon C \otimes_A D \otimes_A E \to M
\qquad\text{given by}\\
q(x\o y\o z)=
\begin{cases}
yz\cdot r(x) &\quad\text{if } \deg(y)=\deg(z)=0\\
0      &\quad\text{otherwise}
\end{cases}
\end{gather*}
is a resolution of $M$ by projective (respectively, free) $A$-modules.
 \end{theorem}

For use in the proof, we bring up a few general homological points.

A {\it\ie{bounded filtration}\/} of a chain complex $F$ is a sequence
\[
0=F^0\subseteq F^1\subseteq\cdots\subseteq F^{s-1}\subseteq
F^s\subseteq\cdots
\]
of subcomplexes such that for each $n$ there exists an $s$ with
$F^s_n=F_n$.  As usual, we let $\gr^s(F)$ denote the complex of
$A$-modules $F^s/F^{s-1}$.

\begin{lemma}
\label{filtration}
Let $q\colon F \to F'$ be a morphism of complexes with bounded
filtrations, such that $q(F^s)\subseteq F'{}^s$ for all $s \in \Z$.  If
for each $s$ the induced map $\gr^s(q)\colon\gr^s(F)\to\gr^s(F')$ is a
quasi-isomorphism, then so is $q$.
 \end{lemma}

\begin{proof}
Denoting $q^s$ the restriction of $q$ to $F^s$, we first show by
induction on $s$ that $\HH_n(q^s)$ is bijective for all $n$.  The
assertion is clear for $s=0$, since $F^0=0$ and $F^{\prime\,0} = 0$.
For the inductive step, we assume that $q^{s-1}$ is a quasi-isomorphism
for some $s\ge1$.  We have a commutative diagram of complexes
\[
    \xymatrix{
      0 \ar[r] & F^{s-1} \ar[d]_{q^{s-1}} \ar[r] & F^s \ar[d]_{q^{s}}\ar[r]
                & \gr^s(F) \ar[d]_{\gr^s(q)}\ar[r] & 0 \\
      0 \ar[r] & F^{\prime\,s-1} \ar[r]     & F^{\prime\, s} \ar[r]
                & \gr^s(F') \ar[r] & 0 
      }
\]
By hypothesis and inductive assumption, in the induced diagram
\[  
\xymatrixcolsep{0.65pc}
    \xymatrix{
      \HH_{n+1}(\gr^s(F)) \ar[d]_{\HH_{n+1}(\gr^s(q))}^\cong \ar[r] &
      \HH_n    (F^{s-1})  \ar[d]_{\HH_n(q^{s-1})}^\cong \ar[r] &
      \HH_n    (F^s)  \ar[d]_{\HH_n(q^s)}       \ar[r] &
      \HH_n    (\gr^s(F)) \ar[d]_{\HH_n(\gr^s(q))}^\cong \ar[r] &
      \HH_{n-1}(F^{s-1})  \ar[d]_{\HH_{n-1}(q^{s-1})}^\cong
      \\
      \HH_{n+1}(\gr^s(F'))        \ar[r] &
      \HH_n    (F^{\prime\,s-1})         \ar[r] &
      \HH_n    (F^{\prime\,s})         \ar[r] &
      \HH_n    (\gr^s{F'})         \ar[r] &
      \HH_{n-1}(F^{\prime\,s-1})
      }
\]  
the four outer vertical maps are bijective.  By the Five-Lemma, so is
$\HH_n(q^s)$.

Now we fix an integer $n \in \Z$, and pick $s$ so large that
\[
F^s_{k}=F_{k} \qquad\text{and}\qquad
F^{\prime\, s}_{k}=F'_{k}\qquad\text{hold for}\qquad
k=n-1,n,n+1\,.
\]
The choice of $s$ implies that $\HH_n(F^{s}) = \HH_n(F)$,
$\HH_n(F^{\prime\,s}) = \HH_n(F')$, and $\HH_n(q^s) = \HH_n(q)$.  Since
we have already proved that $\HH_n(q^s)$ is an isomorphism, we conclude
that $\HH_n(q)\colon \HH_n(F)\to \HH_n(F')$ is an isomorphism.
 \qed \end{proof}

\begin{Remark}
\label{commutator}
If $(F,d_F)$ is a complex of $A$-modules, then $\Hom^{\gr}_A(F,F)$
denotes the graded module whose $n$'th component consists of the
$A$-linear maps $g\colon F\to F$ with $g(F_i)\subseteq F_{i+n}$ for all
$i\in\Z$.

If $g,h$ are homogeneous
$A$-linear maps, then their composition $gh$ is homogeneous of degree
$\deg(g)+\deg(h)$, and so is their graded commutator
\[
[g,h] = g h - (-1)^{\deg g\cdot \deg h} h g
\]
Commutation is a graded derivation: for each homogeneous map $h'$ one has
\[
[g,hh']= [g,h]h'+(-1)^{\deg g\cdot \deg h}h[g,h']
\]

The map $h\mapsto [d_F,h]$ has square $0$, and transforms
$\Hom^{\gr}_A(F,F)$ into a complex of $A$-modules; by definition, its
cycles are the chain maps $F\to F$, and its boundaries are the
null-homotopic maps.
 \end{Remark}

\begin{Remark}\label{graded-map-tensor}
If $p\colon F \to F'$ and $q\colon G \to G'$ are graded maps of graded
modules, we define the tensor product $p \otimes q\colon F \otimes F'
\to G \otimes G'$ by the formula $(p \otimes q)(f \otimes g)=(-1)^{\deg
q \cdot\deg f}( p(f) \otimes q(g) )$.  With this convention, when $F=F'$
and $G=G'$, the graded commutator $[1_F \otimes q,p \otimes 1_G]$
vanishes.
 \end{Remark}

\begin{lemma}
\label{quism}
Let $M$ be an $A$-module and let $r\colon C\to M$ be a free resolution.
If $g\colon C\to C$ is an $A$-linear map with $\deg(g)>0$, and
$[d_C,g]=0$, then $g=[d_C,h]$ for some $A$-linear map $h\colon C\to C$
with $\deg(h)=\deg(g)+1$.
 \end{lemma}

\begin{proof}
The augmentation $r\colon C \to M$ defines a chain map of degree zero
\[
\Hom^\gr_A(C,r)\colon \Hom^\gr_A(C,C)\to\Hom^\gr_A(C,M)
\]
The map induced in homology is an isomorphism: to see this, apply the
`comparison theorem for projective resolutions'.  Since $A$-linear maps
$C\to M$ of positive degree are trivial, the conclusion follows from
Remark \ref{commutator}. \qed
 \end{proof}

\begin{proof}[of Theorem \ref{main}]
Recall that $D$ is the divided powers algebra of a free $A$-module $Q$
with basis $Y_1,\dots, Y_c$, that $X_1, \dots,X_c$ is the dual basis of
the free $A$-module $Q^*$, and $S$ for the symmetric algebra of $Q^*$,
see Construction \ref{algebras} for details.  We set $f=\sum_{i=1}^c
f_iX_i \in\SS^{1}(Q^*)$.

We first construct the maps $d_\g$ by induction on $|\g|$.

If $|\g|=0$, then $\g=o$, so $d_o=d_C$ is predefined.  If $|\g|=1$,
then $\g=\e_i$ for some $i$ with $1\le i\le c$.  Since $f_i$
annihilates the $B$-module $M$, the map $-f_i \cdot 1_C$ lifts the
zero map on $M$, hence is null-homotopic.  For each $i$ we take
$d_{\e_i}$ to be a null-homotopy, that is, $[d_o,d_{\e_i}]=-f_i
\cdot 1_C$.  With these choices, the desired formulas hold for all $\g$
with $|\g|\le1$.

Assume by induction that maps $d_\g$ satisfying the conclusion of the
lemma have been chosen for all $\g\in\N^c$ with $|\g|<n$, for some 
$n\ge2$.  Fix $\g\in\N^c$ with $|\g|=n$.  Using Remark \ref{commutator}
and the induction hypothesis, we obtain
\begin{align*}
\bigg[d_o,\ssum{\a+\b=\g} d_\a d_\b\bigg]
&=\ssum{\a+\b=\g}\big([d_o,d_\a] d_\b -d_\a [d_o,d_\b]\big)\\
&=\ssum{\a+\b=\g}\bigg(
\bigg(\ssum{\a'+\a''=\a} d_{\a'} d_{\a''}\bigg)d_\b
-d_\a\bigg(\ssum{\b'+\b''=\b} d_{\b'} d_{\b''}\bigg)\bigg)\\
&=\ssum{\a'+\a''+\b=\g}d_{\a'} d_{\a''} d_{\b}
-\ssum{\a+\b'+\b''=\g}d_{\a} d_{\b'} d_{\b''}\\
&=0
\end{align*}
The map $-\sum^+_{\a+\b=\g} d_\a d_\b$ has degree $2|\g|-2$, so by
Lemma \ref{quism} it is equal to $[d_o,d_\g]$ for some $A$-linear map
$d_\g\colon C\to C$ of degree $2|\g|-1$.  Choosing such a $d_\g$ for
each $\g\in\N^c$ with $|\g|=n$, we complete the step of the induction.

From the definition of $d$ we obtain an expression
\begin{align*}
d^2 =& d_{CD}^2\o 1_E +
1_C\o[d_{DE}\,,\,1_D\o d_E]+
[d_{CD}\o 1_E\,,\,1_C\o d_{DE}]+
\\
&1_C\o d_{DE}^2 +
1_C\o 1_D\o d_E^2 +
[d_{CD}\o 1_E\,,\,1_C\o 1_D\o d_E]
\end{align*}
Constructions \ref{cartan}, \ref{koszul}, and Remark
\ref{graded-map-tensor} show that the maps in the second row are equal
to $0$, so to prove that $d^2=0$ it suffices to establish the
equalities
\begin{align}
\label{first}
        d_{CD}^2 & = -f\cdot 1_{C\o D} \\
\label{second}
        [d_{DE}\,,\,1_D\o d_E] & = f\cdot 1_{D\o E}\\
\label{third}
        [d_{CD}\o 1_E\,,\,1_C\o d_{DE}] & =  0
\end{align}

A direct computation with formula \eqref{family} proves equality
\eqref{first} above:
\begin{align*}
d_{CD}^2(x\o y)
&=d_{CD}\bigg(\sum_{\b\in\N^c} d_\b(x)\o\big(X^{\b}\lcontract y\big)\bigg)\\
&=\sum_{\b\in\N^c}\bigg
(\sum_{\a\in\N^c}d_\a d_\b(x)\o
\big(X^{\a}\lcontract\big(X^{\b}\lcontract y\big)\big)\bigg)\\
&=\sum_{\a+\b\in\N^c}d_\a d_\b(x)\o \big(X^{\a+\b}\lcontract y\big)\\
&=\sum_{i=1}^c -f_ix\o (X_i\lcontract y)\\
&= -f\cdot (x\o y)
\end{align*}

By Constructions \ref{koszul} and \ref{cartan}, the maps $1_D\o d_E$
and $d_{DE}$ are derivations of degree $-1$, so the commutator
$[d_{DE},1_D\o d_E]$ is a derivation of degree $-2$.  Every element of
$D \otimes_A E$ is a product of elements $1\o Y^\wedge_i$ of degree $1$
and $Y^{(k)}_j\o1$ of degree $2k$, so it suffices to check that the map
on either side of \eqref{third} takes the same value on those
elements.  For degree reasons, both sides vanish on $1\o Y^\wedge_i$.
We now complete the proof of equality \eqref{second} as follows:
\begin{align*}
[d_{DE}\,,\,&1_E\o d_E]\big(Y^{(k)}_j \o 1\big)\\
&=d_{DE}\big((1_D\o d_E)\big(Y^{(k)}_j \o 1\big)\big) +
(1_E\o d_E)\big(d_{DE}\big(Y^{(k)}_j \o 1\big)\big)\\
&= (1_E\o d_E)\big(Y^{(k-1)}_j \o Y_j^\wedge\big)\\
&= Y^{(k-1)}_j \o f_j\\
&= f\cdot\big(Y^{(k)}_j \o 1\big)
\end{align*}

To derive equation \eqref{third} we use Constructions \ref{koszul} and
\ref{cartan} once again:
\begin{align*}
\big((d_{CD}\o 1_E)&(1_C\o d_{DE})\big)(x\o y\o z)\\
=&(-1)^{\deg x} d_{CD}\bigg(
\sum_{i=1}^c x\o(X_i\lcontract y)\o(Y^\wedge_i\cdot z)\bigg)\\
=&(-1)^{\deg x} \sum_{i=1}^c \sum_{\g\in\N^c}
d_\g(x)\o\big(X^{\g}\lcontract(X_i\lcontract y)\big)
\o (Y^\wedge_i\cdot z)\\
=&-\sum_{\g\in\N^c} \sum_{i=1}^c 
(-1)^{\deg(d_\g(x))}d_\g(x)\o\big(X_i\lcontract(X^{\g}\lcontract y)\big)
\o (Y^\wedge_i\cdot z)\\
=&-\big(1_C\o d_{DE}\big)
\bigg(\sum_{\g\in\N^c}d_\g(x)\o(X^{\g}\lcontract y) \o z\bigg)\\
=&-\big((1_C\o d_{DE})(d_{CD}\o 1_E)\big)(x\o y\o z)
\end{align*}

It remains to show $q$ is a quasi-isomorphism.  Setting
\[
F^s = \bigoplus_{k+\ell \le s} C \otimes_A \DD^k(Q)
\otimes_A \Wedge{}^\ell (Q^\wedge)
\qquad\text{for}\qquad s\in\Z
\]
we obtain a bounded filtration of the complex $F= (C \otimes_A D
\otimes_A E,d)$.  On the other hand, we let $F'$ denote the complex
with $F'_0=M$ and $F'_n=0$ for $n\ne0$; the filtration defined by
$F^{\prime\,0}=0$ and $F^{\prime\,s}=F'$ for $s\ge1$ is obviously
bounded, and $q(F^s)\subseteq F^{\prime\,s}$ holds for all $s\ge0$.  By
Lemma \ref{filtration} it suffices to show that the induced map
$\gr^s(q)\colon \gr^s(F)\to\gr^s(F')$ is bijective for all $s$.

Inspection of the differential $d$ of $F$ shows that $\gr^s(F)$ is
isomorphic to the tensor product of complexes $C\otimes_A G^s$, where
$G^s$ is the complex defined in Lemma \ref{split}.  It is established
there that $G^s$ is is split exact for $s>0$, hence $\HH_n(C\otimes_A
G^s)=0$ for all $n\in\Z$.  As $G^0=A$ and $\gr^0(q)=r$, we are done.
 \qed \end{proof}

\section{Cohomology Operators}
\label{Cohomology operators}

We present a new approach to the procedure of Shamash \cite{CI:Sh} and
Eisenbud \cite{CI:Ei} for building projective resolutions over a
complete intersection.  We then use this resolution to prove a
fundamental result of Gulliksen \cite{CI:MR51:487} on the structure of
Ext modules over complete intersections.

A set ${\boldsymbol f}=\{f_1 , \dots, f_c\}\subseteq A$ is
{\it\ie{Koszul-regular}\/} if the complex $(E,d_E)$ of Construction
\ref{koszul}, has $\HH_n(E)=0$ for $n>0$.  A sufficient condition for
Koszul-regularity is that the elements of ${\boldsymbol f}$, in some
order, form a regular sequence.

\begin{theorem}
\label{resolution}
Let $A$ be a commutative ring, ${\boldsymbol f}=\{f_1,\dots,f_c\}
\subseteq A$ a subset, $B=A/({\boldsymbol f})$ the residue ring,
$M$ a $B$-module, and $r \colon  C \to M$ a resolution of
$M$ by projective (respectively, free) $A$-modules.

Let $\{d_\g\colon C\to C\}_{\g\in\N^c}$ be a family of $A$-linear maps
provided by Theorem \ref{main}, set $D'=D\otimes_AB$, and $y'=y\o 1$
for $y\in D$.  The map
\begin{equation}
\begin{gathered}
\label{partial}
\partial \colon C\otimes_A D' \to C\otimes_A D'
\qquad\text{given by}\\
\partial(x\o y') = \sum_{\g \in\N^c} d_\g(x) \o (X^{\g} \lcontract y)'
\end{gathered}
\end{equation}
is a $B$-linear differential of degree $-1$.  If ${\boldsymbol f}$ is
Koszul-regular, then the map 
\begin{gather*}
    q'\colon C \otimes_A D' \to M \qquad \text{given by}\\
    q'(x\o y')= \begin{cases}
                    y \cdot r(x) &\quad\text{if $\deg(y')=0$}\\
                    0            &\quad\text{otherwise}
                \end{cases}
\end{gather*}
is a resolution of $M$ by projective (respectively, free) $A$-modules.
 \end{theorem}

\begin{Remark}
\label{hypersurface}
Assume that in the theorem ${\boldsymbol f}=\{f_1\}$.  The module
$D_\ell$ is then trivial if $\ell$ is odd, and is free with basis
consisting of a single element $Y_1^{(\ell/2)}$ if $\ell$ is even.
Thus, the resolution $C\otimes_AD'$ has the form
\[
\cdots\xrightarrow{\partial_{2n+1}}
\bigoplus_{j=0}^\infty C_{2j}\otimes_A BY_1^{(n-j)}
\xrightarrow{\partial_{2n}}
\bigoplus_{j=1}^\infty C_{2j-1}\otimes_A BY_1^{(n-j)}
\xrightarrow{\partial_{2n-1}}
\cdots
\]

The simplest situation occurs when, in addition, $C$ is a free
resolution with $C_n=0$ for $n\ge 2$.  In this case the differential
$d_o$ has a single non-zero component, $d_1\colon C_1\to C_0$, the
homotopy $d_{\e_1}$ between $-f\cdot 1_C$ and $0_C$ has a single
non-zero component, $s_0\colon C_0\to C_1$, and all the maps $d_\g$
with $\g\in\N^1\smallsetminus\{o,\e_1\}$ are trivial for degree
reasons.  It is now easy to see that the complex above coincides with
the one constructed, {\sl ad hoc\/}, in Remark \ref{periodicity}.
 \end{Remark}

\begin{proof}[of the theorem]
In the notation of Theorem \ref{main}, we have equalities
\[
C \otimes_A D'=(C\otimes_A D \otimes_A E)\otimes_E B
\qquad\text{and}\qquad
\partial= d\o 1_B 
\]
It follows that $\partial^2=0$.  For each $s\ge0$ consider the
subcomplexes
\begin{gather*}
F^s = \bigoplus_{k+\ell \le s} C_k \otimes_A D_\ell \otimes_A E
\qquad\text{of}\qquad F=C \otimes_A D \otimes_A E\\
F^{\prime\,s} = \bigoplus_{k+\ell \le s} C_k \otimes_A D'_\ell
\qquad\text{of}\qquad F'=C \otimes_A D'
\end{gather*}
They provide bounded filtrations of the complexes $F$ and $F'$,
respectively, such that the map $p'=1_C\o 1_D\o p\colon F\to F'$ 
satisfies $p'(F^s)\subseteq F^{\prime\,s}$ for all $s\ge0$.
Setting $G_s=\bigoplus_{k+\ell=s}(C_k \otimes_A D_\ell)$, we obtain
equalities $\gr^s(F)=(G_s\otimes_AE,1_{G_s}\o d_E)$ and $\gr^s(F')=
(G_s\otimes_AB,0)$ of complexes of $A$-modules.

If $\boldsymbol f$ is Koszul regular, then $p\colon E\to B$ is a
quasi-isomorphism, hence so is $1_{G_s}\o p=\gr^s(p')$ for each
$s\ge0$.  Lemma \ref{filtration} then shows that $p'$ is a
quasi-isomorphism.  The quasi-isomorphism $q\colon F\to M$ of Theorem
\ref{main} factors as $q=q'(1_C\o 1_D\o p)$, so we see that $q'$ is
a quasi-isomorphism, as desired.
 \qed\end{proof}

Let $M$ and $N$ be $B$-modules, and let $\Ext^\bu_B(M,N)$ denote the
graded $B$-module having $\Ext^n_B(M,N)$ as component of degree $-n$.
To avoid negative numbers, it is customary to regrade $\Ext^\bu_B(M,N)$
by {\it co\/}homological degree, under which the elements of
$\Ext^n_B(M,N)$ are assigned degree $n$; we do not do it here, in order
not to confuse \Mtwo. Of course, these modules can be computed from any
projective resolution of $M$ over $B$.

The next couple of remarks collect a few innocuous observations.  In
hindsight, they provide some of the basic tools for studying cohomology
of modules over complete intersections: see Remark \ref{history} for
some related material.

\begin{Remark}
\label{action}
The resolution $(C\otimes_A D',\partial)$ provided by Theorem
\ref{resolution} is a graded module over the graded algebra $S$, with
action defined by the formula
\[
u\cdot(x\o y')= x\o (u\lcontract y)'
\]
and this action commutes with the differential $\partial$.  The induced
action provides a structure of graded $S$-module on the complex
$\Hom_B(C\otimes_A D',N)$.

The action of $S$ commutes with the differential $\partial^*=
\Hom_B(\partial,N)$ of this complex, hence passes to its homology,
making it a graded a $S$-module.  Thus, each element $u\in
S_{-2k}=\SS^k(Q)$ determines homomorphisms
\[
\Ext^n_B(M,N)\xrightarrow{\ u\ }\Ext^{n+2k}_B(M,N)
\qquad\text{for all}\qquad n\in\Z
\]
For this reason, from now on we refer to the graded ring $S$ as the
{\it\ie{ring of cohomology operators}\/} determined by the
Koszul-regular set $\boldsymbol f$.
 \end{Remark}

\begin{Remark}
\label{canonical}
The canonical isomorphisms of complexes of $A$-modules
\begin{equation*}
\Hom_B(C \otimes_A D', N)=S \otimes_A \Hom_A(C, N)=
S \otimes_A \Hom_A(C, A) \otimes_A N
\end{equation*}
commute with the actions of $S$.
 \end{Remark}

The following fundamental result shows that in many important cases the
action of the cohomology operators is highly nontrivial.

\begin{theorem}
\label{finiteness}
Let $A$ be a commutative ring, let ${\boldsymbol f}$ be a Koszul
regular subset of $A$, and let $S$ be the graded ring of cohomology
operators defined by $\boldsymbol f$.

If $M$ and $N$ are finitely generated modules over $B=A/({\boldsymbol
f})$, and $M$ has finite projective dimension over $A$ (in particular,
if $A$ is regular), then the graded $S$-module $\Ext^\bu_B(M,N)$ is
finitely generated.
 \end{theorem}

\begin{proof}
Choose a resolution $r\colon C\to M$ with $C_n$ a finite projective
$A$-module for each $n$ and $C_n=0$ for all $n\gg0$.  By Remark
\eqref{canonical}, the graded $S$-module $\Hom_B(C \otimes_A D',N)$ is
finitely generated.  Since $S$ is noetherian, so is the submodule
$\Ker(\partial^*)$, and hence the homology module, $\Ext^\bu_B(M,N)$.
 \qed
  \end{proof}

\begin{Remark}
\label{history}
The resolution of Remark \ref{hypersurface} was constructed by Shamash
\cite[Sect.\ 3]{CI:Sh}, that of Theorem \ref{resolution} by Eisenbud
\cite[Sect.\ 7]{CI:Ei}.  The new aspect of our approach is indicated at
the beginning of Section \ref{Universal homotopies}.

As introduced in Remark \ref{action}, the $S$-module structure on Ext
may appear {\sl ad hoc\/}.  In fact, it is independent of all choices
of resolutions and maps, it can be computed from {\it any\/} projective
resolution of $M$ over $B$, it is natural in both module arguments
and---in an appropriate sense---in the ring argument, and it commutes
with Yoneda products from either side.  These properties were proved by
Gulliksen \cite[Sect.~2]{CI:MR51:487}, Mehta \cite[Ch.~2]{CI:Me},
Eisenbud \cite[Sect.\ 4]{CI:Ei}, and Avramov
\cite[Sect.\ 2]{CI:MR90g:13027}.  However, each author used a different
construction of cohomology operators, and comparison of the
different approaches has turned to be an unexpectedly delicate
problem.  It was finally resolved in \cite{CI:MR2000e:13021}, where
complete proofs of the main properties of the operators can be found.

Gulliksen \cite[Sect.~3]{CI:MR51:487} established a stronger form of
Theorem \ref{finiteness}, without finiteness hypotheses on the ring
$A$: If the $A$-module $\Ext^\bu_A(M,N)$ is noetherian, then the
$S$-module $\Ext^\bu_B(M,N)$ is noetherian; this can be obtained from
the complexes of Remark \eqref{canonical} by means of a spectral
sequence, cf.\ \cite[Sect.~6]{CI:MR1774757}.  The converse of
Gulliksen's theorem was proved in \cite[Sect.~4]{CI:MR99c:13033}.
 \end{Remark}

For the rest of the paper we place ourselves in a situation where \Mtwo
operates best---graded modules over positively graded rings.  This
grading is inherited by the various Ext modules, and we keep careful
track of it.  Our conventions and bookkeeping procedures are discussed
in detail in an Appendix, which the reader is invited to consult as
needed.

For ease of reference, we collect some notation.

\begin{notation}
\label{graded stuff}
The following is assumed for the rest of the paper.
\begin{itemize}
\item[$\bullet$]
$K$ is a field.
\item[$\bullet$]
$\{x_h\,\vert\,\deg'(x_h)>0\}_{h=1,\dots,e}$ is a set of
indeterminates over $K$.
\item[$\bullet$]
$A=K[x_1,\dots,x_e]$, graded by $\deg'(a)=0$ for $a\in K$.
\item[$\bullet$]
$f_1,\dots,f_c$ is a homogeneous $A$-regular sequence
in $(x_1,\dots,x_e)^2$.
%%
%% Lucho, why do we want the f's to be of degree at least 2?
%%
\item[$\bullet$]
$r_i=\deg'(f_i)$ for $i=1,\dots,c$.
\item[$\bullet$]
$\{X_i\,\vert\,\Deg X_i=(-2,-r_i)\}_{i=1,\dots,c}$ is a set
of indeterminates over $A$.
\item[$\bullet$]
$S=A[X_1,\dots, X_c]$, bigraded by $\Deg(a)=(0,\deg'(a))$.
\item[$\bullet$]
$B=A/({\boldsymbol f})$, with degree induced by $\deg'$.
\item[$\bullet$]
$M$ and $N$ are finitely generated graded $B$-modules.
\item[$\bullet$]
$S$ acts as bigraded ring of cohomology operators on
$\Ext^\bu_B(M,N)$.
\item[$\bullet$]
$k=B/(x_1,\dots,x_e)B$, with degree induced by $\deg'$.
\item[$\bullet$]
$R=S\otimes_A k\cong K[X_1,\dots,X_c]$, with bidegree induced by
$\Deg$.
\end{itemize}
 \end{notation}

\begin{Remark}
Under the conditions above, it is reasonable to ask when the $B$-free
resolution $G$ of Theorem \ref{resolution}, obtained from a {\it
minimal\/} $A$-free resolution $C$ of $M$, will itself be minimal.
Shamash \cite[Sect.~3]{CI:Sh} proves that $G$ is minimal if
$f_i\in(x_1,\dots,x_e)\ann_A(M)$ for $i=1,\dots,c$.  An obvious example
with non-minimal $G$ occurs when $M$ has finite projective dimension
over $B$: if $c>0$ then $G$ is infinite.  A more interesting failure of
minimality follows.
 \end{Remark}

\begin{Example}
Let $A$, $f$, $B$, and $M$ be as in Example \ref{cube}.
\beginOutput
i38 : M = B^1/m^3;\\
\endOutput
\beginOutput
i39 : F = resolution(M, LengthLimit=>8)\\
\emptyLine
\       1      10      16      15      15      15      15      15      15\\
o39 = B  <-- B   <-- B   <-- B   <-- B   <-- B   <-- B   <-- B   <-- B\\
\                                                                      \\
\      0      1       2       3       4       5       6       7       8\\
\emptyLine
o39 : ChainComplex\\
\endOutput
Thus, the sequence of Betti numbers $\b^B_n(M)$ is 
$(1,10,16,15,15,15,\dots)$.
\beginOutput
i40 : M' = restrict1 M;\\
\endOutput
\beginOutput
i41 : C = res M'\\
\emptyLine
\       1      10      15      6\\
o41 = A  <-- A   <-- A   <-- A  <-- 0\\
\                                     \\
\      0      1       2       3      4\\
\emptyLine
o41 : ChainComplex\\
\endOutput
By Remark \ref{hypersurface}, the sequence $\rank_B F_n$ is 
$(1,10,16,16,16,16,\dots)$.
 \end{Example}

In a graded context, all cohomological entities discussed so far in the
text acquire an extra grading, discussed in detail in the Appendix.
The notions below are used, but not named, in \cite{CI:AB2} in a local
situation.

\begin{Remark}
\label{reduced ext}
We define the {\it\ie{reduced Ext module}\/} for $M$ and $N$ over $B$ 
by
\[
\rExt^\bu_B(M,N)=\Ext^\bu_B(M,N)\otimes_Ak
\]
With the induced bigrading and action, it is a bigraded module over the
bigraded ring $R$, that we call the {\it\ie{reduced ring of cohomology
operators}\/}.

The dimension of the $K$-vector space $\rExt^n_B(M,N)_{s}$ is equal to
the number of generators of bidegree $(-n,s)$ in any minimal set of
generators of the graded $B$-module $\Ext^n_B(M,N)$.  We define the
graded (respectively, ungraded) {\it\ie{Ext-generator series}\/} of $M$
and $N$ to be the formal power series
\begin{gather*}
\gen_B^{M,N}(t,u)=
\sum_{n\in\N\,,\,s\in\Z}\dim_K \rExt^n_B(M,N)_{s}\, t^n u^{-s}
\in\Z[u,u^{-1}][[t]]\\
\gen_B^{M,N}(t)=
\sum_{n=0}^\infty\dim_K \rExt^n_B(M,N)\, t^n
\in\Z[[t]]
\end{gather*}
There is a simple relation between these series: $\gen_B^{M,N}(t)=
\gen_B^{M,N}(t,1)$.
 \end{Remark}

\begin{corollary}
\label{series}
In the notation above, $\rExt^\bu_B(M,N)$ is a finitely generated
bigraded $R$-module, and $\gen_B^{M,N}(t,u)$ represents a rational
function of the form
\begin{equation*}
\frac{g_B^{M,N}(t,u)}{(1-t^2u^{r_1})\cdots(1-t^2u^{r_c})}
\qquad\text{with}\qquad g_B^{M,N}(t,u)\in\Z[t,u,u^{-1}]
\end{equation*}
\end{corollary}

\begin{proof}
The assertion on finite generation results from Theorem
\ref{finiteness} and the one on bigradings from Remark \ref{graded
action}.  The form of the power series is then given by the
Hilbert-Serre Theorem. \qed
 \end{proof}

\section{Computation of Ext Modules}
\label{Computation of Ext modules}

This section contains the main new computational result of the paper.

We discuss, apply, and present an algorithm that computes, for graded
modules $M$ and $N$ over a graded complete intersection ring $B$, the
graded $B$-modules $\Ext^n_B(M,N)$ simultaneously in all degrees $n$,
along with all the cohomology operators defined in Remark
\ref{action}.

More precisely, the input consists of a field $K$, a polynomial ring
$A=K[x_1,\dots,x_e]$ with $\deg'(x_h)>0$, a sequence $f_1,\dots,f_c$ of
elements of $A$, and finitely generated modules $M$, $N$ over
$B=A/(f_1,\dots,f_c)$.  The program checks whether the sequence
consists of homogeneous elements, whether it is regular, and whether
the modules $M$, $N$ are graded, sending the appropriate error message
if any one of these conditions is violated.  If the input data pass
those tests, then the program produces a presentation of the bigraded
module $H=\Ext^\bu_B(M,N)$, where the elements of $\Ext^n_B(M,N)$ have
homological degree $-n$, over the polynomial ring $A[X_1,\dots,X_c]$,
bigraded by $\Deg(a)=(0,\deg'(a))$ and $\Deg(X_i)=(-2,-\deg'(f_i))$.

The algorithm is based on the proofs of Theorems \ref{resolution} and
\ref{finiteness}, and is presented in Code \ref{master} below.  We
start with an informal discussion.

\begin{Remark}
\label{algorithm}
The routine {\tt resolution}\indexcmd{resolution} of \Mtwo finds the matrices $d_{o,n}\colon
C_n\to C_{n-1}$ of the differential $d_C$ of a minimal free resolution
$C$ of $M$ over $A$.  Matrices $d_{\g,n}\colon C_n\to C_{n+2|\g|-1}$
satisfying equation \eqref{family} for $\g\in\N^c$ with $|\g|>0$ are
computed using the routine {\tt nullhomotopy} of the \Mtwo language in
a loop that follows the first part of the proof of Theorem
\ref{resolution}.  

The transposed matrix $d^*_{\g,n}$ yields an endomorphism of the free
bigraded $A$-module $C^*=\bigoplus_{n=0}^e \Hom_A(C_n,A)$ of rank $m$,
where $m=\sum_{n=0}^e\rank_A C_n$.  The $m\times m$ matrix ${\widetilde
d}^*_{\g,n}$ describing this endomorphism is formed using the routines
{\tt transpose} and {\tt sum}.  The $m\times m$ matrix
\[
\Delta = \sum_{n=0}^e(-1)^{n+2|\g|-1}
\sum_{\substack{\g\in\N^c\\ |\g|\le(e-n+1)/2}}
X^{\g}\cdot\widetilde d_{\g,n}^*
\]
with entries in $S=A[X_1,\dots,X_c]$ defines an endomorphism of the
free bigraded $S$-module $S\otimes_A C^*$.  It induces an endomorphism
$\overline\Delta$ of the bigraded $S$-module $S\otimes_A C^*\otimes_A
N$.  The bigraded $S$-module $H=\Ext^\bu_B(M,N)$ is computed as
$H=\Ker(\overline\Delta)/ \Ima(\overline\Delta)$ using the routine {\tt
homology}\indexcmd{homology}.
 \end{Remark}

In the computations we let $H$ denote the bigraded $S$-module
$\Ext^\bu_B(M,N)$.  As the graded ring $S$ is zero in odd homological
degrees, there is a canonical direct sum decomposition $H =
H^{\text{even}} \oplus H^{\text{odd}}$ of bigraded $S$-modules, where
`even' or `odd' refers to the parity of the {\it first\/} degree in
each pair $\Deg(x)$.

We begin with an example in codimension 1, where it is possible to
construct the infinite resolution and the action of $S$ on it by
hand.

\begin{Example}
Consider the ring $A = K[x]$ where the variable $x$ is assigned degree
$5$, and set $B = A/(x^3)$.  The bigraded ring of cohomology operators
then is $S=A[X,x]$, where $\Deg(X)=(-2,-15)$ and $\Deg x=(0,5)$.

For the $B$-modules $M = B/(x^2)$ and $N = B/(x)$, the bigraded
$S$-module $H=\Ext^\bu_B(M,N)$ is described by the isomorphism
\[
   H\cong (S/(x))\oplus (S/(x))[1,10]
\]

A minimal free resolution of $M$ over $B$ is displayed below.
\[
F = \quad \dots \longrightarrow
 B[-30] \xrightarrow{-x\,} B[-25] \xrightarrow{\ x^2\,} 
 B[-15] \xrightarrow{-x\,} B[-10] \xrightarrow{\ x^2\,} 
 B \longrightarrow 0
\] 
This resolution is actually isomorphic to the resolution $C\otimes_A
D'$ described in Remark \ref{hypersurface}, formed from the free
resolution
\[C
= \quad 0 \longrightarrow A[-10] \xrightarrow{\ x^2\,} A
\longrightarrow 0
\]
of $M$ over $A$ and the nullhomotopy $d_{\e_1}$ displayed in the diagram
\[
\xymatrix{
0 \ar[r] & A[-10] \ar[r]^-{x^2} \ar[d]_{-x^3} & A \ar[r] \ar[d]^{-x^3}
        \ar[dl]_-{-x} & 0 \\
0 \ar[r] & A[-10] \ar[r]^-{x^2}              & A \ar[r]                              & 0
}
\]
The isomorphism of $F$ with $C \otimes_A D'$ endows $F$ with a
structure of bigraded module over $S$, where the action of $X$ on
$F$ is the chain map $F \to F$ of homological degree $-2$ and internal
degree $-15$ that corresponds to the identity map of $B$ in each
component.

The bigraded $S$-module $H=\Ext^\bu_B(M,N)$ is the homology of the
complex
\[
\Hom_R(F,N) = \quad 0 \xrightarrow{\ \ }
N \xrightarrow{\ 0\ } N[10] \xrightarrow{\ 0\,}
N[15] \xrightarrow{\ 0\ } N[25] \xrightarrow{\ 0\,} 
N[30] \longrightarrow\cdots
\]
where multiplication by $x\in S$ is the zero map, and for each $i\ge0$
multiplication by $X\in S$ sends $N[15i]$ to $N[15i+15]$
(respectively, $N[10i+10]$ to $N[10i+25]$) by the identity map.  This
description provides the desired isomorphisms of bigraded $S$-modules.

Here is how to compute $H$ with \Mtwo.
 
Create the rings and modules.
\beginOutput
i42 : K = ZZ/103; \\
\endOutput
\beginOutput
i43 : A = K[x,Degrees=>\{5\}];\\
\endOutput
\beginOutput
i44 : B = A/(x^3);\\
\endOutput
\beginOutput
i45 : M = B^1/(x^2);\\
\endOutput
\beginOutput
i46 : N = B^1/(x);\\
\endOutput
Use the function {\tt Ext} to compute $H = \Ext^\bu_B(M,N)$ (the
semicolon at the end of the line will suppress printing until we have
assigned the name {\tt S} to the ring of cohomology operators
constructed by \Mtwo.)
\beginOutput
i47 : H = Ext(M,N);\\
\endOutput
We may look at the ring.
\beginOutput
i48 : ring H\\
\emptyLine
o48 = K [\$X , x, Degrees => \{\{-2, -15\}, \{0, 5\}\}]\\
\           1\\
\emptyLine
o48 : PolynomialRing\\
\endOutput
\Mtwo has assigned the name {\tt \$X\char`\_1} to the variable $X$.
The dollar sign indicates an internal name that cannot be entered from
the keyboard: if necessary, obtain the variable by entering {\tt
S\char`\_0}; notice that indexing in \Mtwo starts with $0$ rather than $1$.
Notice also the appearance of braces rather than
parentheses in \Mtwo's notation for bidegrees. 
\beginOutput
i49 : degree {\char`\\} gens ring H\\
\emptyLine
o49 = \{\{-2, -15\}, \{0, 5\}\}\\
\emptyLine
o49 : List\\
\endOutput
Assign the ring a name.
\beginOutput
i50 : S = ring H;\\
\endOutput
We can now look at the $S$-module $H$.
\beginOutput
i51 : H\\
\emptyLine
o51 = cokernel \{0, 0\}    | 0 x |\\
\               \{-1, -10\} | x 0 |\\
\emptyLine
\                             2\\
o51 : S-module, quotient of S\\
\endOutput
Each row in the display above is labeled with the bidegree of the
corresponding generator of $H$.  This presentation gives the 
isomorphisms of bigraded $S$-modules, already computed by hand
earlier.
\end{Example}

Let's try an example with a complete intersection of codimension 2.  It
is not so easy to do by hand, but can be checked using the theory in
\cite{CI:MR1774757}.

\begin{Example}
Begin by constructing a polynomial ring $A=K[x,y]$.
\beginOutput
i52 : A = K[x,y];\\
\endOutput
Now we produce a complete intersection quotient ring $B=A/(x^3,y^2)$.
\beginOutput
i53 : J = ideal(x^3,y^2);\\
\emptyLine
o53 : Ideal of A\\
\endOutput
\beginOutput
i54 : B = A/J;\\
\endOutput
We take $N$ to be the $B$-module $B/(x^2,xy)$.
\beginOutput
i55 : N = cokernel matrix\{\{x^2,x*y\}\}\\
\emptyLine
o55 = cokernel | x2 xy |\\
\emptyLine
\                             1\\
o55 : B-module, quotient of B\\
\endOutput
Remark \ref{bigrading} shows that $H=\Ext^\bu_B(N,N)$ is a bigraded module
over the bigraded ring $S=A[X_1,X_2]=K[X_1,X_2,x,y]$ where 
\begin{alignat*}{2}
\Deg(X_1)&=(-2,-3)\quad &\Deg(X_2)&=(-2,-2)\\
\Deg(x)&=(0,1)        &\Deg(y)&=(0,1)
\end{alignat*}
Using \Mtwo (below) we obtain an isomorphism of bigraded $S$-modules
\begin{gather*}
H^{\text{even}}\cong \frac{S}{(x^2,xy,y^2,xX_1,yX_1)}
        \oplus \frac{S}{(x,y)}[2,2]\\
H^{\text{odd}} \cong \left(\frac{S}{(x,y,X_1)}
\oplus \frac{S}{(x,y)}\right)^2[1,1]
\end{gather*}
These isomorphisms also yield expressions for the graded $B$-modules:
\begin{gather*}
\Ext_B^{2i}(N,N)\cong
N\cdot X_2^{i}\oplus
\bigoplus_{h=1}^{i} k\cdot X_1^{h}X_2^{i-h}\oplus
\bigoplus_{h=0}^{i-1}k[2]\cdot X_1^{h}X_2^{i-1-h}\\
\Ext_B^{2i+1}(N,N)\cong
\bigg(k[1]\cdot X_2^{i}\oplus
\bigoplus_{h=0}^{i}k[1]\cdot X_1^{i-h}X_2^{h}\bigg)^2
\end{gather*}

Now we follow in detail the computation of the bigraded $S$-module $H$.
\beginOutput
i56 : time H = Ext(N,N);\\
\     -- used 0.2 seconds\\
\endOutput
\beginOutput
i57 : ring H\\
\emptyLine
o57 = K [\$X , \$X , x, y, Degrees => \{\{-2, -2\}, \{-2, -3\}, \{0, 1\}, \{0, 1\}\}]\\
\           1    2\\
\emptyLine
o57 : PolynomialRing\\
\endOutput
\beginOutput
i58 : S = ring H;\\
\endOutput
One might wish to have a better view of the bidegrees of the variables
of the ring $S$.  An easy way to achieve this, with signs reversed, is
to display the transpose of the matrix of variables.
\beginOutput
i59 : transpose vars S\\
\emptyLine
o59 = \{2, 2\}  | \$X_1 |\\
\      \{2, 3\}  | \$X_2 |\\
\      \{0, -1\} | x    |\\
\      \{0, -1\} | y    |\\
\emptyLine
\              4       1\\
o59 : Matrix S  <--- S\\
\endOutput
The internal degrees displayed for the cohomology operators may come
as a surprise.  To understand what is going on, recall that these
degrees are determined by a choice of minimal generators for $J$.
At this point we do not know what is the sequence of generators that
\Mtwo used, so let's compute those generators the way the program did.
\beginOutput
i60 : trim J\\
\emptyLine
\              2   3\\
o60 = ideal (y , x )\\
\emptyLine
o60 : Ideal of A\\
\endOutput
Notice that \Mtwo has reordered the original sequence of generators.
Now we see that our variable $X_1$, which corresponds to $x^3$, is
denoted {\tt X\char`\_2} by \Mtwo, and that $X_2$, which corresponds
to $y^2$ is denoted {\tt X\char`\_1}.  This explains the bidegrees used
by the program.

Display $H$.
\beginOutput
i61 : H\\
\emptyLine
o61 = cokernel \{-2, -2\} | 0 0 0 0 0 0 0 0 0  0  0  y x 0    0    0     $\cdot\cdot\cdot$\\
\               \{-1, -1\} | y 0 0 0 0 x 0 0 0  0  0  0 0 \$X_1 0    0     $\cdot\cdot\cdot$\\
\               \{-1, -1\} | 0 0 0 y 0 0 0 x 0  0  0  0 0 0    \$X_1 0     $\cdot\cdot\cdot$\\
\               \{-1, -1\} | 0 y 0 0 x 0 0 0 0  0  0  0 0 0    0    0     $\cdot\cdot\cdot$\\
\               \{-1, -1\} | 0 0 y 0 0 0 x 0 0  0  0  0 0 0    0    0     $\cdot\cdot\cdot$\\
\               \{0, 0\}   | 0 0 0 0 0 0 0 0 y2 xy x2 0 0 0    0    \$X_1y $\cdot\cdot\cdot$\\
\emptyLine
\                             6\\
o61 : S-module, quotient of S\\
\endOutput
That's a bit large, so we want to look at the even and odd parts separately.

We may compute the even and odd parts of $H$ as the span of the
generators of $H$ with the appropriate parity.  Since the two desired
functions differ only in the predicate to be applied, we can generate
them both by writing a function that accepts the predicate as its
argument and returns a function.
\beginOutput
i62 : partSelector = predicate -> H -> (\\
\         R := ring H;\\
\         H' := prune image matrix \{\\
\             select(\\
\                 apply(numgens H, i -> H_\{i\}),\\
\                 f -> predicate first first degrees source f\\
\                 )\\
\             \};\\
\         H');\\
\endOutput
\beginOutput
i63 : evenPart = partSelector even; oddPart = partSelector odd;\\
\endOutput
Now to obtain the even part, $H^{\text{even}}$, simply type
\beginOutput
i65 : evenPart H\\
\emptyLine
o65 = cokernel \{-2, -2\} | 0  0  0  y x 0     0     |\\
\               \{0, 0\}   | y2 xy x2 0 0 \$X_1y \$X_1x |\\
\emptyLine
\                             2\\
o65 : S-module, quotient of S\\
\endOutput
Do the same thing for the odd part, $H^{\text{odd}}$.
\beginOutput
i66 : oddPart H\\
\emptyLine
o66 = cokernel \{-1, -1\} | 0 0 y 0 0 0 x 0 0    0    |\\
\               \{-1, -1\} | 0 y 0 0 x 0 0 0 0    0    |\\
\               \{-1, -1\} | 0 0 0 y 0 0 0 x 0    \$X_1 |\\
\               \{-1, -1\} | y 0 0 0 0 x 0 0 \$X_1 0    |\\
\emptyLine
\                             4\\
o66 : S-module, quotient of S\\
\endOutput
These presentations yield the desired isomorphism of bigraded $S$-modules.
 \end{Example}

Here is the source code which implements the routine {\tt Ext}.  It is
incorporated into \Mtwo.

\begin{code}
\label{master}
The function {\tt Ext(M,N)} computes $\Ext_B^\bu(M,N)$ for graded modules
$M$, $N$ over a graded complete intersection ring $B$.  The function {\tt
  code}\indexcmd{code} can be used to obtain a copy of the source code.\indexcmd{Ext}
\beginOutput
i67 : print code(Ext,Module,Module)\\
-- ../../../m2/ext.m2:82-171\\
Ext(Module,Module) := Module => (M,N) -> (\\
\  cacheModule := youngest(M,N);\\
\  cacheKey := (Ext,M,N);\\
\  if cacheModule#?cacheKey then return cacheModule#cacheKey;\\
\  B := ring M;\\
\  if B =!= ring N\\
\  then error "expected modules over the same ring";\\
\  if not isCommutative B\\
\  then error "'Ext' not implemented yet for noncommutative rings.";\\
\  if not isHomogeneous B\\
\  then error "'Ext' received modules over an inhomogeneous ring";\\
\  if not isHomogeneous N or not isHomogeneous M\\
\  then error "'Ext' received an inhomogeneous module";\\
\  if N == 0 then B^0\\
\  else if M == 0 then B^0\\
\  else (\\
\    p := presentation B;\\
\    A := ring p;\\
\    I := ideal mingens ideal p;\\
\    n := numgens A;\\
\    c := numgens I;\\
\    if c =!= codim B \\
\    then error "total Ext available only for complete intersections";\\
\    f := apply(c, i -> I_i);\\
\    pM := lift(presentation M,A);\\
\    pN := lift(presentation N,A);\\
\    M' := cokernel ( pM | p ** id_(target pM) );\\
\    N' := cokernel ( pN | p ** id_(target pN) );\\
\    C := complete resolution M';\\
\    X := local X;\\
\    K := coefficientRing A;\\
\    -- compute the fudge factor for the adjustment of bidegrees\\
\    fudge := if #f > 0 then 1 + max(first {\char`\\} degree {\char`\\} f) // 2 else 0;\\
\    S := K(monoid [X_1 .. X_c, toSequence A.generatorSymbols,\\
\      Degrees => \{\\
\        apply(0 .. c-1, i -> \{-2, - first degree f_i\}),\\
\        apply(0 .. n-1, j -> \{ 0,   first degree A_j\})\\
\        \},\\
\      Adjust => v -> \{- fudge * v#0 + v#1, - v#0\},\\
\      Repair => w -> \{- w#1, - fudge * w#1 + w#0\}\\
\      ]);\\
\    -- make a monoid whose monomials can be used as indices\\
\    Rmon := monoid [X_1 .. X_c,Degrees=>\{c:\{2\}\}];\\
\    -- make group ring, so 'basis' can enumerate the monomials\\
\    R := K Rmon;\\
\    -- make a hash table to store the blocks of the matrix\\
\    blks := new MutableHashTable;\\
\    blks#(exponents 1_Rmon) = C.dd;\\
\    scan(0 .. c-1, i -> \\
\         blks#(exponents Rmon_i) = nullhomotopy (- f_i*id_C));\\
\    -- a helper function to list the factorizations of a monomial\\
\    factorizations := (gamma) -> (\\
\      -- Input: gamma is the list of exponents for a monomial\\
\      -- Return a list of pairs of lists of exponents showing the\\
\      -- possible factorizations of gamma.\\
\      if gamma === \{\} then \{ (\{\}, \{\}) \}\\
\      else (\\
\        i := gamma#-1;\\
\        splice apply(factorizations drop(gamma,-1), \\
\          (alpha,beta) -> apply (0..i, \\
\               j -> (append(alpha,j), append(beta,i-j))))));\\
\    scan(4 .. length C + 1, \\
\      d -> if even d then (\\
\        scan( exponents {\char`\\} leadMonomial {\char`\\} first entries basis(d,R), \\
\          gamma -> (\\
\            s := - sum(factorizations gamma,\\
\              (alpha,beta) -> (\\
\                if blks#?alpha and blks#?beta\\
\                then blks#alpha * blks#beta\\
\                else 0));\\
\            -- compute and save the nonzero nullhomotopies\\
\            if s != 0 then blks#gamma = nullhomotopy s;\\
\            ))));\\
\    -- make a free module whose basis elements have the right degrees\\
\    spots := C -> sort select(keys C, i -> class i === ZZ);\\
\    Cstar := S^(apply(spots C,\\
\        i -> toSequence apply(degrees C_i, d -> \{i,first d\})));\\
\    -- assemble the matrix from its blocks.\\
\    -- We omit the sign (-1)^(n+1) which would ordinarily be used,\\
\    -- which does not affect the homology.\\
\    toS := map(S,A,apply(toList(c .. c+n-1), i -> S_i),\\
\      DegreeMap => prepend_0);\\
\    Delta := map(Cstar, Cstar, \\
\      transpose sum(keys blks, m -> S_m * toS sum blks#m),\\
\      Degree => \{-1,0\});\\
\    DeltaBar := Delta ** (toS ** N');\\
\    assert isHomogeneous DeltaBar;\\
\    assert(DeltaBar * DeltaBar == 0);\\
\    -- now compute the total Ext as a single homology module\\
\    cacheModule#cacheKey = prune homology(DeltaBar,DeltaBar)))\\
\endOutput
\end{code}

\begin{Remark}
The bigraded module $\Tor_\bu^B(M,N)$ is the homology of the complex
$(C \otimes_A D') \otimes_B N$, where $C\otimes_A D'$ is the complex
from Theorem \ref{resolution}.  Observations parallel to Remarks
\ref{action} and \ref{bigrading} show that $\Tor_\bu^B(M,N)$ inherits
from $D'$ a structure of bigraded $S$-module.

It would be desirable also to have algorithms to compute
$\Tor_\bu^B(M,N)$ in the spirit of the algorithm presented above for
$\Ext^\bu_B(M,N)$.  If one of the modules has finite length, then
each $\Tor_n^B(M,N)$ is a $B$-module of finite length, and the
computation of $\Tor_\bu^B(M,N)$ can be reduced to a computation of
$\Ext$ by means of Matlis duality, which here can be realized as vector
space duality over the field $K$.  However, in homology there is no
equivalent for the finiteness property described in Remark
\ref{canonical}; it is an {\bf open problem} to devise algorithms that
would compute $\Tor_\bu^B(M,N)$ in general.
 \end{Remark}

\section{Invariants of Modules}
\label{Invariants of modules}

In this section we apply our techniques to develop effective methods
for computation (for graded modules over a graded complete
intersection) of invariants such as cohomology modules, Poincar\'e
series, Bass series, complexity, critical degree, and support
varieties.  For each invariant we produce code that computes it, and
illustrate the action of the code on some explicit example.

Whenever appropriate, we describe {\bf open problems} on which the
computational power of \Mtwo could be unleashed.

Notation \ref{graded stuff} is used consistently throughout the section.

\subsection{Cohomology Modules}
\label{Computation of cohomology}

We call the bigraded $R$-module $P=\Ext^\bu_B(M,k)$ the
{\it\ie{contravariant cohomology module}\/} of $M$ over $B$, and the
bigraded $R$-module $I=\Ext^\bu_B(k,M)$ the {\it\ie{covariant
cohomology module}\/} of $M$.   Codes that display presentations of the
cohomology modules are presented after a detailed discussion of an example.

\begin{sExample} 
\label{random}
Let us create the ring $B=K[x,y,z]/(x^3,y^4,z^5)$.
\beginOutput
i68 : A = K[x,y,z];\\
\endOutput
\beginOutput
i69 : J = trim ideal(x^3,y^4,z^5)\\
\emptyLine
\              3   4   5\\
o69 = ideal (x , y , z )\\
\emptyLine
o69 : Ideal of A\\
\endOutput
\beginOutput
i70 : B = A/J;\\
\endOutput
We trimmed the ideal, so that we know the generators \Mtwo will use.

This time we want a graded $B$-module $M$ about whose homology we know
nothing {\sl a priori\/}.   One way to proceed is to create $M$ as the
cokernel of some random matrix of forms; let's try a 3 by 2 matrix of
quadratic forms.
\beginOutput
i71 : f = random (B^3, B^\{-2,-3\})\\
\emptyLine
o71 = | 27x2+49xy-14y2-23xz-6yz-19z2 38x2y-34xy2+4y3+x2z+16xyz-y2z-5xz $\cdot\cdot\cdot$\\
\      | -5x2+44xy+38y2+40xz+15yz+4z2 -37x2y+51xy2-36y3+26x2z-38xyz-17y $\cdot\cdot\cdot$\\
\      | 21x2-30xy+32y2-47xz+7yz-50z2 -6x2y-14xy2-26y3-7x2z+41xyz+50y2z $\cdot\cdot\cdot$\\
\emptyLine
\              3       2\\
o71 : Matrix B  <--- B\\
\endOutput
We can't read the second column of that matrix, so let's display it
separately.
\beginOutput
i72 : f_\{1\}\\
\emptyLine
o72 = | 38x2y-34xy2+4y3+x2z+16xyz-y2z-5xz2-6yz2+47z3        |\\
\      | -37x2y+51xy2-36y3+26x2z-38xyz-17y2z+17xz2-11yz2+8z3 |\\
\      | -6x2y-14xy2-26y3-7x2z+41xyz+50y2z+26xz2+46yz2-44z3  |\\
\emptyLine
\              3       1\\
o72 : Matrix B  <--- B\\
\endOutput
Now let's make the module $M$.
\beginOutput
i73 : M = cokernel f;\\
\endOutput

We are going to produce isomorphisms of bigraded modules 
\begin{gather*}
P^{\text{even}} \cong R[4,10] \oplus (X_1,X_2)[2,7]
        \oplus\left(\frac{R}{(X_1,X_2,X_3)}\right)^3\oplus R^4[2,7]\\
P^{\text{odd}} \cong \frac{R}{(X_1,X_2,X_3)}[1,2]
        \oplus\left(\frac{R}{(X_1)}\right)^3[3,9]
        \oplus \frac{R}{(X_1,X_2)}[1,3] \oplus R^6[3,9]
\end{gather*}
over the polynomial ring $R=K[X_1,X_2,X_3]$ over $K$, bigraded by
\[
\Deg(X_1)=(-2,-3)\qquad \Deg(X_2)=(-2,-4) \qquad \Deg(X_3)=(-2,-5)
\]

Let's compute $\Ext^\bu_B(M,B/(x,y,z))$ by the routine from Section 
\ref{Computation of Ext modules}.
\beginOutput
i74 : time P = Ext(M,B^1/(x,y,z));\\
\     -- used 1.64 seconds\\
\endOutput
\beginOutput
i75 : S = ring P;\\
\endOutput
Examine the variables of $S$; due to transposing, their bidegrees
are displayed with the {\it opposite\/} signs.
\beginOutput
i76 : transpose vars S\\
\emptyLine
o76 = \{2, 3\}  | \$X_1 |\\
\      \{2, 4\}  | \$X_2 |\\
\      \{2, 5\}  | \$X_3 |\\
\      \{0, -1\} | x    |\\
\      \{0, -1\} | y    |\\
\      \{0, -1\} | z    |\\
\emptyLine
\              6       1\\
o76 : Matrix S  <--- S\\
\endOutput
The variables $x$, $y$, and $z$ of $A$ annihilate $P$, and so appear in
many places in a presentation of $P$.  To reduce the size of such a
presentation, we pass to a ring which eliminates those variables.
\beginOutput
i77 : R = K[X_1..X_3,Degrees => \{\{-2,-3\},\{-2,-4\},\{-2,-5\}\},\\
\              Adjust => S.Adjust, Repair => S.Repair];\\
\endOutput
\beginOutput
i78 : phi = map(R,S,\{X_1,X_2,X_3,0,0,0\})\\
\emptyLine
o78 = map(R,S,\{X , X , X , 0, 0, 0\})\\
\                1   2   3\\
\emptyLine
o78 : RingMap R <--- S\\
\endOutput
\beginOutput
i79 : P = prune (phi ** P);\\
\endOutput
\beginOutput
i80 : transpose vars ring P\\
\emptyLine
o80 = \{2, 3\} | X_1 |\\
\      \{2, 4\} | X_2 |\\
\      \{2, 5\} | X_3 |\\
\emptyLine
\              3       1\\
o80 : Matrix R  <--- R\\
\endOutput
As we planned, the original variables $x$, $y$, $z$, which act
trivially on the cohomology, are no longer present in the ring. 
Next we compute presentations
\beginOutput
i81 : evenPart P\\
\emptyLine
o81 = cokernel \{-4, -10\} | 0   0   0   0   0   0   0   0   0   0    |\\
\               \{-4, -10\} | 0   0   0   0   0   0   0   0   0   -X_2 |\\
\               \{-4, -11\} | 0   0   0   0   0   0   0   0   0   X_1  |\\
\               \{0, 0\}    | 0   0   X_3 0   0   X_2 0   X_1 0   0    |\\
\               \{0, 0\}    | 0   X_3 0   0   X_2 0   X_1 0   0   0    |\\
\               \{0, 0\}    | X_3 0   0   X_2 0   0   0   0   X_1 0    |\\
\               \{-2, -7\}  | 0   0   0   0   0   0   0   0   0   0    |\\
\               \{-2, -7\}  | 0   0   0   0   0   0   0   0   0   0    |\\
\               \{-2, -7\}  | 0   0   0   0   0   0   0   0   0   0    |\\
\               \{-2, -7\}  | 0   0   0   0   0   0   0   0   0   0    |\\
\emptyLine
\                             10\\
o81 : R-module, quotient of R\\
\endOutput
\beginOutput
i82 : oddPart P\\
\emptyLine
o82 = cokernel \{-1, -2\} | X_3 0   X_2 0   0   0   0   X_1 |\\
\               \{-3, -9\} | 0   0   0   0   0   0   X_1 0   |\\
\               \{-3, -9\} | 0   0   0   0   0   X_1 0   0   |\\
\               \{-3, -9\} | 0   0   0   0   X_1 0   0   0   |\\
\               \{-1, -3\} | 0   X_2 0   X_1 0   0   0   0   |\\
\               \{-3, -9\} | 0   0   0   0   0   0   0   0   |\\
\               \{-3, -9\} | 0   0   0   0   0   0   0   0   |\\
\               \{-3, -9\} | 0   0   0   0   0   0   0   0   |\\
\               \{-3, -9\} | 0   0   0   0   0   0   0   0   |\\
\               \{-3, -9\} | 0   0   0   0   0   0   0   0   |\\
\               \{-3, -9\} | 0   0   0   0   0   0   0   0   |\\
\emptyLine
\                             11\\
o82 : R-module, quotient of R\\
\endOutput
These presentations yield the desired isomorphisms of bigraded
$R$-modules.
 \end{sExample}

The procedure above can be automated by installing a method that will
be run when {\tt Ext} is presented with a module $M$ and the residue
field $k$.  It displays a presentation of $\Ext_B^\bu(M,k)$ as a bigraded
$R$-module.

\begin{sCode}
\label{change}
The function {\tt changeRing H} takes an $S$-module $H$ and tensors it
with $R$.  It does this by constructing $R$ and a ring homomorphism
\[
\varphi\colon A[X_1,\dots,X_c] = S \to R = K[X_1,\dots,X_c]
\]
\beginOutput
i83 : changeRing = H -> (\\
\         S := ring H;\\
\         K := coefficientRing S;\\
\         degs := select(degrees source vars S,\\
\              d -> 0 != first d);\\
\         R := K[X_1 .. X_#degs, Degrees => degs,\\
\              Repair => S.Repair, Adjust => S.Adjust];\\
\         phi := map(R,S,join(gens R,(numgens S - numgens R):0));\\
\         prune (phi ** H)\\
\         );\\
\endOutput
\end{sCode}

\begin{sCode}
\label{cohomology}
The function {\tt Ext(M,k)} computes $\Ext_B^\bu(M,k)$ when $B$ is a
graded complete intersection, $M$ a graded $B$-module, and $k$ is the
residue field of $B$.  The result is presented as a module over the
ring $k[X_1,\dots,X_c]$.
\beginOutput
i84 : Ext(Module,Ring) := (M,k) -> (\\
\         B := ring M;\\
\         if ideal k != ideal vars B\\
\         then error "expected the residue field of the module";\\
\         changeRing Ext(M,coker vars B)\\
\         );\\
\endOutput
\end{sCode}

\begin{sExample}
For a test, we run again the computation for $P^{\text{odd}}$.
\beginOutput
i85 : use B;\\
\endOutput
\beginOutput
i86 : k = B/(x,y,z);\\
\endOutput
\beginOutput
i87 : use B;\\
\endOutput
\beginOutput
i88 : P = Ext(M,k);\\
\endOutput
\beginOutput
i89 : time oddPart P\\
\     -- used 0.09 seconds\\
\emptyLine
o89 = cokernel \{-1, -2\} | X_3 0   X_2 0   0   0   0   X_1 |\\
\               \{-3, -9\} | 0   0   0   0   0   0   X_1 0   |\\
\               \{-3, -9\} | 0   0   0   0   0   X_1 0   0   |\\
\               \{-3, -9\} | 0   0   0   0   X_1 0   0   0   |\\
\               \{-1, -3\} | 0   X_2 0   X_1 0   0   0   0   |\\
\               \{-3, -9\} | 0   0   0   0   0   0   0   0   |\\
\               \{-3, -9\} | 0   0   0   0   0   0   0   0   |\\
\               \{-3, -9\} | 0   0   0   0   0   0   0   0   |\\
\               \{-3, -9\} | 0   0   0   0   0   0   0   0   |\\
\               \{-3, -9\} | 0   0   0   0   0   0   0   0   |\\
\               \{-3, -9\} | 0   0   0   0   0   0   0   0   |\\
\emptyLine
\                                                                       $\cdot\cdot\cdot$\\
o89 : K [X , X , X , Degrees => \{\{-2, -3\}, \{-2, -4\}, \{-2, -5\}\}]-module $\cdot\cdot\cdot$\\
\          1   2   3                                                    $\cdot\cdot\cdot$\\
\endOutput
\end{sExample}

We also introduce code for computing the covariant cohomology modules.

\begin{sCode}
\label{covariant-cohomology}
The function {\tt Ext(k,M)} computes $\Ext_B^\bu(k,M)$ when $B$ is a
graded complete intersection, $M$ a graded $B$-module, and $k$ is the
residue field of $B$.  The result is presented as a module over the
ring $k[X_1,\dots,X_c]$.
\beginOutput
i90 : Ext(Ring,Module) := (k,M) -> (\\
\         B := ring M;\\
\         if ideal k != ideal vars B\\
\         then error "expected the residue field of the module";\\
\         changeRing Ext(coker vars B,M)\\
\         );\\
\endOutput
\end{sCode}

Let's see the last code in action.

\begin{sExample}
For $B$ and $M$ from Example \ref{random} we compute the odd part of
the covariant cohomology module $\Ext^\bu_B(k,M)$.
\beginOutput
i91 : time I = Ext(k,M);\\
\     -- used 14.81 seconds\\
\endOutput
\beginOutput
i92 : evenPart I\\
\emptyLine
o92 = cokernel \{0, 6\} | 37X_2  37X_1  |\\
\               \{0, 6\} | -18X_2 -18X_1 |\\
\               \{0, 6\} | -13X_2 -13X_1 |\\
\               \{0, 6\} | -37X_2 -37X_1 |\\
\               \{0, 6\} | 22X_2  22X_1  |\\
\               \{0, 6\} | 0      0      |\\
\               \{0, 6\} | X_2    X_1    |\\
\emptyLine
\                                                                       $\cdot\cdot\cdot$\\
o92 : K [X , X , X , Degrees => \{\{-2, -3\}, \{-2, -4\}, \{-2, -5\}\}]-module $\cdot\cdot\cdot$\\
\          1   2   3                                                    $\cdot\cdot\cdot$\\
\endOutput
\beginOutput
i93 : oddPart I\\
\emptyLine
o93 = cokernel \{-1, 5\} | -48X_3 13X_3  34X_3  3X_3   0     0      0    $\cdot\cdot\cdot$\\
\               \{-1, 5\} | 3X_3   -40X_3 8X_3   8X_3   0     0      0    $\cdot\cdot\cdot$\\
\               \{-1, 5\} | -X_3   37X_3  -13X_3 -35X_3 0     0      0    $\cdot\cdot\cdot$\\
\               \{-1, 4\} | 4X_2   20X_2  3X_2   -47X_2 4X_1  20X_1  3X_1 $\cdot\cdot\cdot$\\
\               \{-1, 4\} | 0      51X_2  0      -30X_2 0     51X_1  0    $\cdot\cdot\cdot$\\
\               \{-1, 4\} | 0      12X_2  0      -3X_2  0     12X_1  0    $\cdot\cdot\cdot$\\
\               \{-1, 4\} | 42X_2  12X_2  46X_2  25X_2  42X_1 12X_1  46X_ $\cdot\cdot\cdot$\\
\               \{-1, 4\} | 45X_2  24X_2  -14X_2 -35X_2 45X_1 24X_1  -14X $\cdot\cdot\cdot$\\
\               \{-1, 4\} | 0      0      X_2    0      0     0      X_1  $\cdot\cdot\cdot$\\
\               \{-1, 4\} | X_2    0      0      0      X_1   0      0    $\cdot\cdot\cdot$\\
\               \{-1, 4\} | 0      -40X_2 0      10X_2  0     -40X_1 0    $\cdot\cdot\cdot$\\
\               \{-1, 4\} | 0      X_2    0      0      0     X_1    0    $\cdot\cdot\cdot$\\
\               \{-1, 3\} | 0      0      0      X_1    0     0      0    $\cdot\cdot\cdot$\\
\emptyLine
\                                                                       $\cdot\cdot\cdot$\\
o93 : K [X , X , X , Degrees => \{\{-2, -3\}, \{-2, -4\}, \{-2, -5\}\}]-module $\cdot\cdot\cdot$\\
\          1   2   3                                                    $\cdot\cdot\cdot$\\
\endOutput
\end{sExample}

\subsection{Poincar\'e Series}
\label{Poincare series}

The {\it\ie{graded Betti number}\/} of $M$ over $B$ is the number
$\b_{ns}^B(M)$ of direct summands isomorphic to the free module
$B[-s]$ in the $n$'th module of a minimal free resolution of $M$ over
$B$.  It can be computed from the equality
\[
\b_{ns}^B(M)=\dim_K\Ext^n_B(M,k)_{s}
\]
The {\it\ie{graded Poincar\'e series}\/} of $M$ over $B$ is the
generating function
\[
\Poi^B_M(t,u)=\sum_{n\in\N\,,\,s\in\Z}\b_{ns}^B(M)\, t^n u^{-s}
\in\Z[u,u^{-1}][[t]]
\]
It is easily computable with \Mtwo from the contravariant cohomology
module, by using the {\tt hilbertSeries}\indexcmd{hilbertSeries} routine.

\begin{sCode}
The function {\tt poincareSeries2 M} computes the graded Poin\-car\'e
series of a graded module $M$ over a graded complete intersection $B$.

First we set up a ring whose elements can serve as Poincar\'e series.
\beginOutput
i94 : T = ZZ[t,u,Inverses=>true,MonomialOrder=>RevLex];\\
\endOutput
\beginOutput
i95 : poincareSeries2 = M -> (\\
\         B := ring M;\\
\         k := B/ideal vars B;\\
\         P := Ext(M,k);\\
\         h := hilbertSeries P;\\
\         T':= degreesRing P;\\
\         substitute(h, \{T'_0=>t^-1,T'_1=>u^-1\})\\
\         );\\
\endOutput
The last line in the code above replaces the variables in the
Poincar\'e series provided by the {\tt hilbertSeries} function with the
variables in our ring {\tt T}.
 \end{sCode}

The $n$th {\it\ie{Betti number}\/} $\b^B_n(M)$ of $M$ over $B$ is the
rank of the $n$th module in a minimal resolution of $M$ by free
$B$-modules.  The {\it\ie{Poincar\'e series}\/} $\Poi^B_M(t)$ is the
generating function of the Betti numbers.  There are expressions
\[
\b^B_n(M)=\sum_{s=0}^\infty\b^B_{ns}(M)
\qquad\text{and}\qquad
\Poi^B_M(t)=\Poi^B_M(t,1)
\]
Accordingly, the code for $\Poi^B_M(t)$ just replaces
in $\Poi^B_M(t,u)$ the variable $u$ by $1$.

\begin{sCode}
The function {\tt poincareSeries1 M} computes the Poincar\'e series of a
graded module $M$ over a graded complete intersection $B$.
\beginOutput
i96 : poincareSeries1 = M -> (\\
\         substitute(poincareSeries2 M, \{u=>1_T\})\\
\         );\\
\endOutput
\end{sCode}

Now let's use these codes in computations.

\begin{sExample}
\label{cosyzygy}
To get a module whose Betti sequence initially decreases, we form
an artinian complete intersection $B'$ and take $M'$ to be a cosyzygy
in a minimal injective resolution of the residue field $k$.  Since $B'$
is self-injective, this can be achieved by taking a syzygy of $k$, then
transposing its presentation matrix.  Of course, we ask \Mtwo to carry
out these steps.
\beginOutput
i97 : A' = K[x,y,z];\\
\endOutput
\beginOutput
i98 : B' = A'/(x^2,y^2,z^3);\\
\endOutput
\beginOutput
i99 : C' = res(B'^1/(x,y,z), LengthLimit => 6)\\
\emptyLine
\        1       3       6       10       15       21       28\\
o99 = B'  <-- B'  <-- B'  <-- B'   <-- B'   <-- B'   <-- B'\\
\                                                          \\
\      0       1       2       3        4        5        6\\
\emptyLine
o99 : ChainComplex\\
\endOutput
\beginOutput
i100 : M' = coker transpose C'.dd_5\\
\emptyLine
o100 = cokernel \{-5\} | -y 0   0  0  z  0 0 0  0 0  0  0 0  0 0 |\\
\                \{-5\} | -x -y  0  0  0  z 0 0  0 0  0  0 0  0 0 |\\
\                \{-5\} | 0  x   -y 0  0  0 z 0  0 0  0  0 0  0 0 |\\
\                \{-5\} | 0  0   x  -y 0  0 0 z  0 0  0  0 0  0 0 |\\
\                \{-5\} | 0  0   0  -x 0  0 0 0  z 0  0  0 0  0 0 |\\
\                \{-5\} | 0  0   0  0  y  0 0 0  0 0  0  0 0  0 0 |\\
\                \{-5\} | 0  0   0  0  -x y 0 0  0 0  0  0 0  0 0 |\\
\                \{-5\} | 0  0   0  0  0  x y 0  0 0  0  0 0  0 0 |\\
\                \{-5\} | 0  0   0  0  0  0 x y  0 0  0  0 0  0 0 |\\
\                \{-5\} | 0  0   0  0  0  0 0 -x y 0  0  0 0  0 0 |\\
\                \{-5\} | 0  0   0  0  0  0 0 0  x 0  0  0 0  0 0 |\\
\                \{-6\} | 0  0   0  0  0  0 0 0  0 -y 0  z 0  0 0 |\\
\                \{-6\} | 0  0   0  0  0  0 0 0  0 x  -y 0 z  0 0 |\\
\                \{-6\} | 0  0   0  0  0  0 0 0  0 0  -x 0 0  z 0 |\\
\                \{-6\} | z2 0   0  0  0  0 0 0  0 0  0  y 0  0 0 |\\
\                \{-6\} | 0  -z2 0  0  0  0 0 0  0 0  0  x y  0 0 |\\
\                \{-6\} | 0  0   z2 0  0  0 0 0  0 0  0  0 -x y 0 |\\
\                \{-6\} | 0  0   0  z2 0  0 0 0  0 0  0  0 0  x 0 |\\
\                \{-7\} | 0  0   0  0  0  0 0 0  0 0  0  0 0  0 z |\\
\                \{-7\} | 0  0   0  0  0  0 0 0  0 z2 0  0 0  0 y |\\
\                \{-7\} | 0  0   0  0  0  0 0 0  0 0  z2 0 0  0 x |\\
\emptyLine
\                                21\\
o100 : B'-module, quotient of B'\\
\endOutput
Compute the Poincar\'e series in two variables $\Poi^{B'}_{M'}(t,u)$.
\beginOutput
i101 : poincareSeries2 M'\\
\emptyLine
\         -7     -6      -5      -6       -5       -4     2 -5      2 - $\cdot\cdot\cdot$\\
\       3u   + 7u   + 11u   + t*u   + 5t*u   + 9t*u   - 6t u   - 14t u  $\cdot\cdot\cdot$\\
o101 = --------------------------------------------------------------- $\cdot\cdot\cdot$\\
\                                                                       $\cdot\cdot\cdot$\\
\                                                                       $\cdot\cdot\cdot$\\
\emptyLine
o101 : Divide\\
\endOutput
\end{sExample}

\begin{sExample}
We compute $\Poi^B_M(t)$ for the module $M$ from Example \ref{random}.
\beginOutput
i102 : p = poincareSeries1 M\\
\emptyLine
\                  2     3      4    5     6    7\\
\       3 + 2t - 5t  + 4t  + 12t  + t  - 4t  - t\\
o102 = -----------------------------------------\\
\                      2       2       2\\
\                (1 - t )(1 - t )(1 - t )\\
\emptyLine
o102 : Divide\\
\endOutput
We have written some rather na\"\i ve code for simplifying rational
functions as above.  It locates factors of the form $1-t^n$ in the
denominator, factors out $1-t$, and factors out $1+t$ if $n$ is even.
Keeping the factors of the denominator separate, it then cancels as
many of them as it can with the numerator.
\beginOutput
i103 : load "simplify.m2"\\
\endOutput
\beginOutput
i104 : simplify p\\
\emptyLine
\                 2     3     4     5    6\\
\       3 - t - 4t  + 8t  + 4t  - 3t  - t\\
o104 = ----------------------------------\\
\                       2       3\\
\                (1 + t) (1 - t)\\
\emptyLine
o104 : Divide\\
\endOutput
In this case, it succeeded in canceling a factor of $1+t$.
\end{sExample}

\begin{sExample}
  We compute some Betti numbers for $M$.  We use the division operation in
  the Euclidean domain $T' = \mathbb Q[t,t^{-1}]$ with the reverse monomial ordering
  to compute power series expansions.
\beginOutput
i105 : T' = QQ[t,Inverses=>true,MonomialOrder=>RevLex];\\
\endOutput
\beginOutput
i106 : expansion = (n,q) -> (\\
\           t := T'_0;\\
\           rho := map(T',T,\{t,1\});\\
\           num := rho value numerator q;\\
\           den := rho value denominator q;\\
\           n = n + first degree den;\\
\           n = max(n, first degree num + 1);\\
\           (num + t^n) // den\\
\           );\\
\endOutput
Now let's expand the Poincar\'e series up to $t^{20}$.
\beginOutput
i107 : expansion(20,p)\\
\emptyLine
\                  2      3      4      5      6      7      8      9   $\cdot\cdot\cdot$\\
o107 = 3 + 2t + 4t  + 10t  + 15t  + 25t  + 32t  + 46t  + 55t  + 73t  + $\cdot\cdot\cdot$\\
\emptyLine
o107 : T'\\
\endOutput
Just to make sure, let's compare the first few coefficients with the
more pedestrian way of doing the computation, one Ext module at a time.
\beginOutput
i108 : psi = map(K,B)\\
\emptyLine
o108 = map(K,B,\{0, 0, 0\})\\
\emptyLine
o108 : RingMap K <--- B\\
\endOutput
\beginOutput
i109 : apply(10, i -> rank (psi ** Ext^i(M,coker vars B)))\\
\emptyLine
o109 = \{3, 2, 4, 10, 15, 25, 32, 46, 55, 73\}\\
\emptyLine
o109 : List\\
\endOutput
Now we restore {\tt t} to its original use.
\beginOutput
i110 : use T;\\
\endOutput
 \end{sExample}

\subsection{Complexity}

The {\it\ie{complexity}\/} of $M$ is the least $d\in\N$ such that
the function
\[
n \mapsto \dim_K \Ext^n_B(M,k)
\]
is bounded above by a polynomial of degree $d-1$ (with the convention
that the zero polynomial has degree $-1$).  This number, denoted
$\cx_B(M)$, was introduced in
\cite{CI:MR90g:13027} to measures on a polynomial scale the rate of
growth of the Betti numbers of $M$.  It is calibrated so that
$\cx_B(M)=0$ if and only if $M$ has finite projective dimension.
Corollary \ref{series} yields
\[
\Poi^B_M(t)=\frac{p^B_M(t)}{(1-t^2)^c}
\qquad\text{for some}\qquad p^B_M(t)\in\Z[t]
\]
Decomposing the right hand side into partial fractions, one sees that
$\cx_R(M)$ equals the order of the pole of $\Poi^B_M(t)$ at $t=1$; in
particular, $\cx_R(M,N)\le c$.  However, since we get $\Poi^B_M(t)$
from a computation of the $R$-module $P=\Ext^\bu_B(M,k)$, it is natural
to obtain $\cx_R(M)$ as the Krull dimension of $P$.

\begin{sCode}
The function {\tt complexity M} yields the complexity of a graded
module $M$ over a graded complete intersection ring $B$.
\beginOutput
i111 : complexity = M -> dim Ext(M,coker vars ring M);\\
\endOutput
 \end{sCode}

\begin{sExample}
We compute $\cx_B(M)$ for $M$ from Example \ref{random}.
\beginOutput
i112 : complexity M\\
\emptyLine
o112 = 3\\
\endOutput
 \end{sExample}

\subsection{Critical Degree}

The {\it\ie{critical degree}\/} of $M$ is the least integer $\ell$ for
which the minimal resolution $F$ of $M$ admits a chain map $g\colon
F\to F$ of degree $m<0$, such that $g_{m+n}\colon F_{m+n}\to F_n$ is
surjective for all $n>\ell$.  This number, introduced in
\cite{CI:MR99c:13033} and denoted $\crdeg_BM$, is meaningful over every
graded ring $B$.  It is equal to the projective dimension whenever the
latter is finite.

When $B$ is a complete intersection it is proved in
\cite[Sect.~7]{CI:MR99c:13033} that $\crdeg_BM$ is finite and yields
important information on the Betti sequence:
\begin{itemize}
\item[$\bullet$]
if $\cx_BM\le1$, then $\b^B_n(M) =\b^B_{n+1}(M)$ for all $n>\crdeg_BM$.
\item[$\bullet$]
if $\cx_BM\ge2$, then $\b^B_n(M)<\b^B_{n+1}(M)$ for all $n>\crdeg_BM$.
\end{itemize}
Thus, it is interesting to know $\crdeg_BM$, or at least to have a good
upper bound.  Here is what is known, in terms of $h=\depth B-\depth_BM$.
\begin{itemize}
\item[$\bullet$]
if $\cx_BM=0$, then $\crdeg_BM=h$.
\item[$\bullet$]
if $\cx_BM=1$, then $\crdeg_BM\le h$.
\item[$\bullet$]
if $\cx_BM=2$, then $\crdeg_BM\le h+1+\max\{2\beta^B_h(M)-1\,,\,
2\beta^B_{h+1}(M)\}$.
\end{itemize}
The first part is the Auslander-Buchsbaum Equality, the second part is
proved in \cite[Sect.~6]{CI:Ei}, the third is established in
\cite[Sect.~7]{CI:MR1774757}.

These upper bounds are realistic:  there exist examples in complexity
$1$ when they are reached, and examples in complexity $2$ when they are
not more than twice the actual value of the critical degree.  If
$\cx_RM\ge3$, then it is an {\bf open problem} whether the critical
degree of $M$ can be bounded in terms that do not depend on the action
of the cohomology operators.

However, in every concrete case $\crdeg_RM$ can be computed explicitly
by using \Mtwo.  Indeed, it is proved in \cite[Sect.~7]{CI:MR99c:13033}
that $\crdeg_RM$ is equal to the highest degree of a non-zero element
in the socle of the $R$-module $\Ext^\bu_B(M,k)$, that is, the
submodule consisting of elements annihilated by $(X_1,\dots,X_c)$.  The
socle is naturally isomorphic to $\Hom_B(k,\Ext^\bu_B(M,k))$, so it can
be obtained by standard \Mtwo routines.

For instance, for the module $M$ from Example \ref{random}, we get
\beginOutput
i113 : k = coker vars ring H;\\
\endOutput
\beginOutput
i114 : prune Hom(k,H)\\
\emptyLine
o114 = 0\\
\emptyLine
o114 : K [\$X , \$X , x, y, Degrees => \{\{-2, -2\}, \{-2, -3\}, \{0, 1\}, \{0,  $\cdot\cdot\cdot$\\
\            1    2\\
\endOutput
The degrees displayed above show that $\crdeg_RM=1$.
\medskip

Of course, one might prefer to see the number $\crdeg_BM$ directly.

\begin{sCode}
The function {\tt criticalDegree M} computes the critical degree of a
graded module $M$ over a graded complete intersection ring $B$.
\beginOutput
i115 : criticalDegree = M -> (\\
\          B := ring M;\\
\          k := B / ideal vars B;\\
\          P := Ext(M,k);\\
\          k  = coker vars ring P;\\
\          - min ( first {\char`\\} degrees source gens prune Hom(k,P))\\
\          );\\
\endOutput
\end{sCode}

Let's test the new code in a couple of cases.

\begin{sExample}
For the module $M$ of Example \ref{random} we have
\beginOutput
i116 : criticalDegree M\\
\emptyLine
o116 = 1\\
\endOutput
in accordance with what was already observed above.

For the module $M'$ of Example \ref{cosyzygy} we obtain
\beginOutput
i117 : criticalDegree M'\\
\emptyLine
o117 = 5\\
\endOutput
\end{sExample}

\subsection{Support Variety}
\label{Support variety}

Let $\overline K$ denote an algebraic closure of $K$.  The
{\it\ie{support variety}\/} $\var^*_B(M)$ is the algebraic set in
${\overline K}{}^c$ defined by the annihilator of $\Ext^\bu_B(M,k)$
over $R=K[X_1,\dots,X_c]$.  This `geometric image' of the
contravariant cohomology module was introduced in \cite{CI:MR90g:13027}
and used to study the minimal free resolution of $M$.  The dimension of
the support variety is equal to the complexity $\cx_R(M)$, that we can
already compute.  There is no need to associate a variety to the
covariant cohomology module, see \ref{Support variety2}.

Since $\var^*_B(M)$ is defined by homogeneous equations, it is a cone
in ${\overline K}{}^c$.  An important {\bf open problem} is whether
every cone in ${\overline K}{}^c$ that can be defined over $K$ is the
variety of some $B$-module $M$.  By \cite[Sect.~6]{CI:MR90g:13027} all
linear subspaces and all hypersurfaces arise in this way, but little
more is known in general.

Feeding our computation of $\Ext^\bu_B(M,k)$ to standard \Mtwo routines
we write code for determining a set of equations defining $\var^*_B(M)$.

\begin{sCode}
The function {\tt supportVarietyIdeal M} yields a set of polynomial
equations with coefficients in $K$, defining the support variety
$\var^*_B(M)$ in ${\overline K}{}^c$ for a graded module $M$ over a
graded complete intersection $B$.
\beginOutput
i118 : supportVarietyIdeal = M -> (\\
\          B := ring M;\\
\          k := B/ideal vars B;\\
\          ann Ext(M,k)\\
\          );\\
\endOutput
 \end{sCode}

As before, we illustrate the code with explicit computations.  In
view of the open problem mentioned above, we fix a ring and a type of
presentation, then change randomly the presentation matrix in the hope
of finding an `interesting' variety.  The result of the experiment
is assessed in Remark \ref{letdown}.

\begin{sExample}
Let $\F_7$ denote the prime field with $7$ elements, and form the
zero-dimensional complete intersection $B''=\F_7[x,y,z]/(x^7,y^7,z^7)$.
\beginOutput
i119 : K'' = ZZ/7;\\
\endOutput
\beginOutput
i120 : A'' = K''[x,y,z];\\
\endOutput
\beginOutput
i121 : J'' = ideal(x^7,y^7,z^7);\\
\emptyLine
o121 : Ideal of A''\\
\endOutput
\beginOutput
i122 : B'' = A''/J'';\\
\endOutput
We apply the code above to search, randomly, for some varieties.  Using
{\tt scan}\indexcmd{scan} we print the results from several runs with one command.
\beginOutput
i123 : scan((1,1) .. (3,3), (r,d) -> (\\
\               V := cokernel random (B''^r,B''^\{-d\});\\
\               << "--------------------------------------------------- $\cdot\cdot\cdot$\\
\               << endl\\
\               << "V = " << V << endl\\
\               << "support variety ideal = "\\
\               << timing supportVarietyIdeal V\\
\               << endl))\\
------------------------------------------------------------------\\
V = cokernel | -2x+3y+2z |\\
support variety ideal = ideal (X  - 2X , X  + X )\\
\                                2     3   1    3\\
\                        -- 0.7 seconds\\
------------------------------------------------------------------\\
V = cokernel | 3x2-2xy+xz-3yz |\\
support variety ideal = ideal(X  + 3X  + 2X )\\
\                               1     2     3\\
\                        -- 0.48 seconds\\
------------------------------------------------------------------\\
V = cokernel | -2x3+3x2y+y3-x2z-3y2z-xz2-3z3 |\\
support variety ideal = 0\\
\                        -- 1.54 seconds\\
------------------------------------------------------------------\\
V = cokernel | -3y+3z |\\
\             | -2x-2y |\\
support variety ideal = ideal(X  + X  - X )\\
\                               1    2    3\\
\                        -- 0.86 seconds\\
------------------------------------------------------------------\\
V = cokernel | -x2+2y2-xz+yz+3z2 |\\
\             | 2xy-3xz-3yz-2z2   |\\
support variety ideal = 0\\
\                        -- 1.31 seconds\\
------------------------------------------------------------------\\
V = cokernel | -x3-2x2y-xy2-2xyz+3y2z+2xz2-yz2-2z3 |\\
\             | 2xy2+3y3-3x2z-2y2z+2xz2+2yz2        |\\
support variety ideal = 0\\
\                        -- 2.21 seconds\\
------------------------------------------------------------------\\
V = cokernel | 3x-y-z   |\\
\             | -3x-y+2z |\\
\             | x-2y+3z  |\\
support variety ideal = 0\\
\                        -- 1.1 seconds\\
------------------------------------------------------------------\\
V = cokernel | 2x2-2xy+2y2+2xz-3z2   |\\
\             | -x2+2xy+y2+3xz+3yz-z2 |\\
\             | -2xz+2yz+2z2          |\\
support variety ideal = 0\\
\                        -- 1.67 seconds\\
------------------------------------------------------------------\\
V = cokernel | 2x3-x2y+2xy2-y3-2xyz+3y2z+xz2+3yz2+z3  |\\
\             | -3x3-3x2y+3xy2+2x2z+3xyz-3y2z-xz2      |\\
\             | -3x3-2x2y-xy2-2y3-2xyz+y2z+xz2+3yz2-z3 |\\
support variety ideal = 0\\
\                        -- 1.92 seconds\\
\endOutput
\end{sExample}

\begin{sRemark}
\label{letdown}
The (admittedly short) search above did not turn up any non-linear
variety.  This should be contrasted with the known result that
{\it every\/} cone in ${\overline \F_7}{}^3$ is the support variety of
some $B''$-module.

Indeed, $B''$ is isomorphic to the group algebra $\F_7[G]$ of the
elementary abelian group $G=\C_7\times\C_7\times\C_7$, where $\C_7$ is
a cyclic group of order $7$.  It is shown in
\cite[Sect.~7]{CI:MR90g:13027} that $\var^*_{B''}(V)$ is equal to a
variety $\var^*_G(V)$, defined in a different way in
\cite{CI:MR85a:20004} by Carlson.  He proves in \cite{CI:MR86b:20009}
that if $K$ is a field of characteristic $p>0$, and $G$ is an
elementary abelian $p$-group of rank $c$, then every cone in
${\overline K}{}^c$ is the rank variety of a finitely generated module
over $K[G]$.
 \end{sRemark}

\subsection{Bass Series}
\label{Bass series}

The {\it\ie{graded Bass number}\/} $\mu^{ns}_B(M)$ of $M$ over $B$ is
the number of direct summands isomorphic to $U[s]$ in the $n$'th module
of a minimal graded injective resolution of $M$ over $B$, where $U$ 
is the injective envelope of $k$.  It satisfies
\[
\mu^{ns}_B(M)=\dim_K\Ext^n_B(k,M)_{s}
\]
The {\it\ie{graded Bass series}\/} of $M$ over $B$ is the generating
function
\[
\Ba^M_B(t,u)=\sum_{n\in\N\,,\,s\in\Z}\mu^{ns}_B(M)\, t^nu^s
\in\Z[u,u^{-1}][[t]]
\]
It is easily computable with \Mtwo from the covariant cohomology module,
by using the {\tt hilbertSeries}\indexcmd{hilbertSeries} routine.

\begin{sCode}
The function {\tt bassSeries2 M} computes the graded Bass series of a
graded module $M$ over a graded complete intersection $B$.
\beginOutput
i124 : bassSeries2 = M -> (\\
\          B := ring M;\\
\          k := B/ideal vars B;\\
\          I := Ext(k,M);\\
\          h := hilbertSeries I;\\
\          T':= degreesRing I;\\
\          substitute(h, \{T'_0=>t^-1, T'_1=>u\})\\
\          );\\
\endOutput
 \end{sCode}

As with Betti numbers and Poincar\'e series, there are ungraded versions
of Bass numbers and Bass series; they are given, respectively, by
\[
\mu_B^n(M)=\sum_{s=0}^\infty\mu_B^{ns}(M)
\qquad\text{and}\qquad
\Ba_B^M(t)=\Ba_B^M(t,1)
\]

\begin{sCode}
The function {\tt bassSeries1 M} computes the Bass series of a graded
module $M$ over a graded complete intersection $B$.
\beginOutput
i125 : bassSeries1 = M -> (\\
\          substitute(bassSeries2 M, \{u=>1_T\})\\
\          );\\
\endOutput
\end{sCode}

Now let's use these codes in computations.

\begin{sExample}
For $k$, the residue field of $B$, the contravariant and covariant
cohomology modules coincide.  For comparison, we compute side by side
the Poincar\'e series and the Bass series of $k$, when
$B=K[x,y,z]/(x^3,y^4,z^5)$ is the ring defined in Example
\ref{random}.
\beginOutput
i126 : use B;\\
\endOutput
\beginOutput
i127 : L = B^1/(x,y,z);\\
\endOutput
\beginOutput
i128 : p = poincareSeries2 L\\
\emptyLine
\                        2 2    3 3\\
\           1 + 3t*u + 3t u  + t u\\
o128 = ------------------------------\\
\             2 3       2 4       2 5\\
\       (1 - t u )(1 - t u )(1 - t u )\\
\emptyLine
o128 : Divide\\
\endOutput
\beginOutput
i129 : b = bassSeries2 L\\
\emptyLine
\                  -1     2 -2    3 -3\\
\          1 + 3t*u   + 3t u   + t u\\
o129 = ---------------------------------\\
\             2 -3       2 -4       2 -5\\
\       (1 - t u  )(1 - t u  )(1 - t u  )\\
\emptyLine
o129 : Divide\\
\endOutput
The reader would have noticed that the two series are different, and
that one is obtained from the other by the substitution $u\mapsto
u^{-1}$.  This underscores the different meanings of the graded Betti
numbers and Bass numbers.
 \end{sExample}

\begin{sExample}
Here we compute the graded and ungraded Bass series of the $B$-module
$M$ of Example \ref{random}.
\beginOutput
i130 : b2 = bassSeries2 M\\
\emptyLine
\         6      3       4       5    2 2    2 3     3     3      3 2   $\cdot\cdot\cdot$\\
\       7u  + t*u  + 9t*u  + 3t*u  - t u  - t u  - 4t  - 3t u - 3t u  + $\cdot\cdot\cdot$\\
o130 = --------------------------------------------------------------- $\cdot\cdot\cdot$\\
\                                    2 -3       2 -4       2 -5\\
\                              (1 - t u  )(1 - t u  )(1 - t u  )\\
\emptyLine
o130 : Divide\\
\endOutput
\beginOutput
i131 : b1 = bassSeries1 M;\\
\endOutput
\beginOutput
i132 : simplify b1\\
\emptyLine
\                  2     3     4\\
\       7 + 6t - 8t  - 2t  + 3t\\
o132 = ------------------------\\
\                  2       3\\
\           (1 + t) (1 - t)\\
\emptyLine
o132 : Divide\\
\endOutput
 \end{sExample}

\section{Invariants of Pairs of Modules}
\label{Invariants of pairs of modules}

In this final section we compute invariants of a pair $(M,N)$ of graded
modules over a graded complete intersection $B$, derived from the
reduced Ext module $\rExt^\bu_B(M,N)$ defined in Remark \ref{reduced
ext}.  The treatment here is parallel to that in Section
\ref{Invariants of modules}.  When one of the modules $M$ or $N$ is
equal to the residue field $k$, the invariants discussed below reduce
to those treated in that section.

\subsection{Reduced Ext Module}

The reduced Ext module $\rExt^\bu_B(M,N)=\Ext^\bu_B(M,N)\otimes_Ak$
defined in Remark \ref{reduced ext} is computed from our basic
routine {\tt Ext(M,N)} by applying the function {\tt changeRing}
defined in Code \ref{change}.

\begin{sCode}
\label{reduced}
The function {\tt ext(M,N)} computes $\rExt_B^\bu(M,N)$ when $M$
and $N$ are graded modules over a graded complete intersection $B$.
\beginOutput
i133 : ext = (M,N) -> changeRing Ext(M,N);\\
\endOutput
 \end{sCode}

\begin{sExample}
\label{new module}
Using the ring $B=K[x,y,z]/(x^3,y^4,z^5)$ and the module $M$ created
in Example \ref{random}, we make new modules 
\[
N=B/(x^2+z^2\,,\,y^3) \qquad\text{and}\qquad
N'=B/(x^2+z^2\,,\,y^3-2z^3)
\]
\beginOutput
i134 : use B;\\
\endOutput
\beginOutput
i135 : N = B^1/(x^2 + z^2,y^3);\\
\endOutput
\beginOutput
i136 : time rH = ext(M,N);\\
\     -- used 15.91 seconds\\
\endOutput
\beginOutput
i137 : evenPart rH\\
\emptyLine
o137 = cokernel \{-4, -9\} | 0   0   0   0   0   0   0   0   0   0   0   $\cdot\cdot\cdot$\\
\                \{0, 2\}   | 0   0   0   0   0   0   0   X_3 0   0   0   $\cdot\cdot\cdot$\\
\                \{0, 2\}   | 0   0   0   0   0   X_3 0   0   0   0   0   $\cdot\cdot\cdot$\\
\                \{0, 2\}   | 0   0   0   0   0   0   X_3 0   0   0   0   $\cdot\cdot\cdot$\\
\                \{0, 2\}   | 0   0   0   0   X_3 0   0   0   0   0   0   $\cdot\cdot\cdot$\\
\                \{0, 2\}   | 0   0   0   X_3 0   0   0   0   0   0   0   $\cdot\cdot\cdot$\\
\                \{0, 2\}   | 0   0   X_3 0   0   0   0   0   0   0   0   $\cdot\cdot\cdot$\\
\                \{0, 2\}   | 0   X_3 0   0   0   0   0   0   0   0   X_2 $\cdot\cdot\cdot$\\
\                \{0, 2\}   | X_3 0   0   0   0   0   0   0   0   X_2 0   $\cdot\cdot\cdot$\\
\                \{-2, -4\} | 0   0   0   0   0   0   0   0   0   0   0   $\cdot\cdot\cdot$\\
\                \{-2, -4\} | 0   0   0   0   0   0   0   0   0   0   0   $\cdot\cdot\cdot$\\
\                \{-2, -4\} | 0   0   0   0   0   0   0   0   0   0   0   $\cdot\cdot\cdot$\\
\                \{-2, -4\} | 0   0   0   0   0   0   0   0   0   0   0   $\cdot\cdot\cdot$\\
\                \{-2, -4\} | 0   0   0   0   0   0   0   0   0   0   0   $\cdot\cdot\cdot$\\
\                \{-2, -4\} | 0   0   0   0   0   0   0   0   0   0   0   $\cdot\cdot\cdot$\\
\                \{-2, -4\} | 0   0   0   0   0   0   0   0   0   0   0   $\cdot\cdot\cdot$\\
\                \{0, 1\}   | 0   0   0   0   0   0   0   0   X_2 0   0   $\cdot\cdot\cdot$\\
\emptyLine
\                                                                       $\cdot\cdot\cdot$\\
o137 : K [X , X , X , Degrees => \{\{-2, -3\}, \{-2, -4\}, \{-2, -5\}\}]-modul $\cdot\cdot\cdot$\\
\           1   2   3                                                   $\cdot\cdot\cdot$\\
\endOutput
\beginOutput
i138 : oddPart rH\\
\emptyLine
o138 = cokernel \{-3, -6\} | 0      0   0   0   0   0   0   0   0   0    $\cdot\cdot\cdot$\\
\                \{-3, -6\} | 0      0   0   0   0   0   0   0   0   0    $\cdot\cdot\cdot$\\
\                \{-3, -6\} | 0      0   0   0   0   0   0   0   0   0    $\cdot\cdot\cdot$\\
\                \{-3, -6\} | 0      0   0   0   0   0   0   0   0   0    $\cdot\cdot\cdot$\\
\                \{-3, -6\} | 0      0   0   0   0   0   0   0   0   0    $\cdot\cdot\cdot$\\
\                \{-1, -1\} | -39X_3 0   0   0   0   0   0   0   X_2 0    $\cdot\cdot\cdot$\\
\                \{-1, -1\} | 31X_3  0   0   0   0   0   0   X_2 0   0    $\cdot\cdot\cdot$\\
\                \{-1, -1\} | -34X_3 0   0   0   0   0   X_2 0   0   0    $\cdot\cdot\cdot$\\
\                \{-1, -1\} | -35X_3 0   0   0   0   X_2 0   0   0   0    $\cdot\cdot\cdot$\\
\                \{-1, -1\} | -29X_3 0   0   0   X_2 0   0   0   0   0    $\cdot\cdot\cdot$\\
\                \{-1, -1\} | 12X_3  0   0   X_2 0   0   0   0   0   0    $\cdot\cdot\cdot$\\
\                \{-1, -1\} | -8X_3  0   X_2 0   0   0   0   0   0   0    $\cdot\cdot\cdot$\\
\                \{-1, -1\} | X_3    X_2 0   0   0   0   0   0   0   0    $\cdot\cdot\cdot$\\
\                \{-3, -7\} | 0      0   0   0   0   0   0   0   0   X_1  $\cdot\cdot\cdot$\\
\emptyLine
\                                                                       $\cdot\cdot\cdot$\\
o138 : K [X , X , X , Degrees => \{\{-2, -3\}, \{-2, -4\}, \{-2, -5\}\}]-modul $\cdot\cdot\cdot$\\
\           1   2   3                                                   $\cdot\cdot\cdot$\\
\endOutput
\beginOutput
i139 : N' = B^1/(x^2 + z^2,y^3 - 2*z^3);\\
\endOutput
\beginOutput
i140 : time rH' = ext(M,N');\\
\     -- used 20.26 seconds\\
\endOutput
\beginOutput
i141 : evenPart rH'\\
\emptyLine
o141 = cokernel \{-4, -8\} | 0   0   0   0   0   0   0   0   0   0   0   $\cdot\cdot\cdot$\\
\                \{-4, -8\} | 0   0   0   0   0   0   0   0   0   0   0   $\cdot\cdot\cdot$\\
\                \{-4, -9\} | 0   0   0   0   0   0   0   0   0   0   0   $\cdot\cdot\cdot$\\
\                \{-4, -9\} | 0   0   0   0   0   0   0   0   0   0   0   $\cdot\cdot\cdot$\\
\                \{-4, -9\} | 0   0   0   0   0   0   0   0   0   0   0   $\cdot\cdot\cdot$\\
\                \{-4, -9\} | 0   0   0   0   0   0   0   0   0   0   0   $\cdot\cdot\cdot$\\
\                \{-4, -9\} | 0   0   0   0   0   0   0   0   0   0   0   $\cdot\cdot\cdot$\\
\                \{-4, -9\} | 0   0   0   0   0   0   0   0   0   0   0   $\cdot\cdot\cdot$\\
\                \{0, 2\}   | 0   0   0   0   0   0   X_3 0   0   0   0   $\cdot\cdot\cdot$\\
\                \{0, 2\}   | 0   0   0   0   0   X_3 0   0   0   0   0   $\cdot\cdot\cdot$\\
\                \{-2, -4\} | 0   0   0   0   0   0   0   0   0   0   0   $\cdot\cdot\cdot$\\
\                \{-2, -4\} | 0   0   0   0   0   0   0   0   0   0   0   $\cdot\cdot\cdot$\\
\                \{-2, -4\} | 0   0   0   0   0   0   0   0   0   0   0   $\cdot\cdot\cdot$\\
\                \{-2, -4\} | 0   0   0   0   0   0   0   0   0   0   0   $\cdot\cdot\cdot$\\
\                \{-2, -4\} | 0   0   0   0   0   0   0   0   0   0   0   $\cdot\cdot\cdot$\\
\                \{-2, -4\} | 0   0   0   0   0   0   0   0   0   0   0   $\cdot\cdot\cdot$\\
\                \{0, 2\}   | 0   0   0   0   X_3 0   0   0   0   0   0   $\cdot\cdot\cdot$\\
\                \{0, 2\}   | 0   0   X_3 0   0   0   0   0   0   0   X_2 $\cdot\cdot\cdot$\\
\                \{-2, -4\} | 0   0   0   0   0   0   0   0   0   0   0   $\cdot\cdot\cdot$\\
\                \{-2, -4\} | 0   0   0   0   0   0   0   0   0   0   0   $\cdot\cdot\cdot$\\
\                \{-2, -4\} | 0   0   0   0   0   0   0   0   0   0   0   $\cdot\cdot\cdot$\\
\                \{-2, -4\} | 0   0   0   0   0   0   0   0   0   0   0   $\cdot\cdot\cdot$\\
\                \{0, 2\}   | 0   0   0   X_3 0   0   0   0   0   0   0   $\cdot\cdot\cdot$\\
\                \{0, 2\}   | 0   X_3 0   0   0   0   0   0   0   X_2 0   $\cdot\cdot\cdot$\\
\                \{-2, -4\} | 0   0   0   0   0   0   0   0   0   0   0   $\cdot\cdot\cdot$\\
\                \{-2, -4\} | 0   0   0   0   0   0   0   0   0   0   0   $\cdot\cdot\cdot$\\
\                \{-2, -4\} | 0   0   0   0   0   0   0   0   0   0   0   $\cdot\cdot\cdot$\\
\                \{-2, -4\} | 0   0   0   0   0   0   0   0   0   0   0   $\cdot\cdot\cdot$\\
\                \{-2, -4\} | 0   0   0   0   0   0   0   0   0   0   0   $\cdot\cdot\cdot$\\
\                \{-2, -4\} | 0   0   0   0   0   0   0   0   0   0   0   $\cdot\cdot\cdot$\\
\                \{0, 2\}   | X_3 0   0   0   0   0   0   0   0   0   0   $\cdot\cdot\cdot$\\
\                \{0, 1\}   | 0   0   0   0   0   0   0   X_2 0   0   0   $\cdot\cdot\cdot$\\
\                \{0, 1\}   | 0   0   0   0   0   0   0   0   X_2 0   0   $\cdot\cdot\cdot$\\
\emptyLine
\                                                                       $\cdot\cdot\cdot$\\
o141 : K [X , X , X , Degrees => \{\{-2, -3\}, \{-2, -4\}, \{-2, -5\}\}]-modul $\cdot\cdot\cdot$\\
\           1   2   3                                                   $\cdot\cdot\cdot$\\
\endOutput
\beginOutput
i142 : oddPart rH'\\
\emptyLine
o142 = cokernel \{-3, -6\} | 0   0   0   0   0   0   0   0   -42X_2 21X_ $\cdot\cdot\cdot$\\
\                \{-3, -6\} | 0   0   0   0   0   0   0   0   -6X_2  -32X $\cdot\cdot\cdot$\\
\                \{-3, -6\} | 0   0   0   0   0   0   0   0   -8X_2  12X_ $\cdot\cdot\cdot$\\
\                \{-3, -6\} | 0   0   0   0   0   0   0   0   26X_2  -36X $\cdot\cdot\cdot$\\
\                \{-3, -6\} | 0   0   0   0   0   0   0   0   50X_2  18X_ $\cdot\cdot\cdot$\\
\                \{-3, -6\} | 0   0   0   0   0   0   0   0   31X_2  7X_2 $\cdot\cdot\cdot$\\
\                \{-3, -7\} | 0   0   0   0   0   0   0   0   0      0    $\cdot\cdot\cdot$\\
\                \{-3, -7\} | 0   0   0   0   0   0   0   0   0      0    $\cdot\cdot\cdot$\\
\                \{-3, -7\} | 0   0   0   0   0   0   0   0   0      0    $\cdot\cdot\cdot$\\
\                \{-3, -7\} | 0   0   0   0   0   0   0   0   0      0    $\cdot\cdot\cdot$\\
\                \{-3, -7\} | 0   0   0   0   0   0   0   0   0      X_1  $\cdot\cdot\cdot$\\
\                \{-3, -7\} | 0   0   0   0   0   0   0   0   X_1    0    $\cdot\cdot\cdot$\\
\                \{-1, -2\} | 0   0   0   X_2 0   0   0   X_1 0      0    $\cdot\cdot\cdot$\\
\                \{-1, -2\} | 0   0   X_2 0   0   0   X_1 0   0      0    $\cdot\cdot\cdot$\\
\                \{-1, -2\} | 0   X_2 0   0   0   X_1 0   0   0      0    $\cdot\cdot\cdot$\\
\                \{-1, -2\} | X_2 0   0   0   X_1 0   0   0   0      0    $\cdot\cdot\cdot$\\
\emptyLine
\                                                                       $\cdot\cdot\cdot$\\
o142 : K [X , X , X , Degrees => \{\{-2, -3\}, \{-2, -4\}, \{-2, -5\}\}]-modul $\cdot\cdot\cdot$\\
\           1   2   3                                                   $\cdot\cdot\cdot$\\
\endOutput
\end{sExample}

\subsection{Ext-generator Series}

The Ext-generator series $\gen^{M,N}_B(t,u)$ defined in Remark
\ref{reduced ext} generalizes both the Poincar\'e series of $M$
and the Bass series of $N$, as seen from the formulas
\[
\Poi^B_M(t,u)=\gen^{M,k}_B(t,u)
\qquad\text{and}\qquad
\Ba^N_B(t,u)=\gen^{k,N}_B(t,u^{-1})
\]
Similar equalities hold for the corresponding series in one variable.
Codes for computing Ext-generator series are easy to produce.

\begin{sCode}
\label{genseries}
The function {\tt extgenSeries2(M,N)} computes $\gen^{M,N}_B(t,u)$ when
$M$ and $N$ are graded modules over a graded complete intersection $B$,
and presents it as a rational function with denominator
$(1-t^2u^{r_1})\cdots(1-t^2u^{r_c})$.
\beginOutput
i143 : extgenSeries2 = (M,N) -> (\\
\          H := ext(M,N);\\
\          h := hilbertSeries H;\\
\          T':= degreesRing H;\\
\          substitute(h, \{T'_0=>t^-1,T'_1=>u^-1\})\\
\          );\\
\endOutput
\end{sCode}

\begin{sCode}
The function {\tt extgenSeries1(M,N)} computes the Ext-genera\-tor series in
one variable for a pair $(M,N)$ of graded modules over a graded complete
intersection $B$.
\beginOutput
i144 : extgenSeries1 = (M,N) -> (\\
\          substitute(extgenSeries2(M,N), \{u=>1_T\})\\
\          );\\
\endOutput
\end{sCode}

\begin{sExample}
For $M$, $N$, and $N'$ as in Example \ref{new module} we obtain
\beginOutput
i145 : time extgenSeries2(M,N)\\
\     -- used 0.44 seconds\\
\emptyLine
\         -2    -1            2      2 2     2 3     2 4     3 4     3  $\cdot\cdot\cdot$\\
\       8u   + u   + 8t*u - 8t u - 9t u  - 9t u  + 7t u  - 8t u  - 8t u $\cdot\cdot\cdot$\\
o145 = --------------------------------------------------------------- $\cdot\cdot\cdot$\\
\                                                                       $\cdot\cdot\cdot$\\
\                                                                       $\cdot\cdot\cdot$\\
\emptyLine
o145 : Divide\\
\endOutput
\beginOutput
i146 : g=time extgenSeries1(M,N)\\
\     -- used 0.13 seconds\\
\emptyLine
\                   2      3      4     5     6    7\\
\       9 + 8t - 19t  - 11t  + 17t  + 4t  - 7t  - t\\
o146 = --------------------------------------------\\
\                       2       2       2\\
\                 (1 - t )(1 - t )(1 - t )\\
\emptyLine
o146 : Divide\\
\endOutput
\beginOutput
i147 : simplify g\\
\emptyLine
\                 2     3    4\\
\       9 - t - 9t  + 6t  + t\\
o147 = ----------------------\\
\                         2\\
\           (1 + t)(1 - t)\\
\emptyLine
o147 : Divide\\
\endOutput
\beginOutput
i148 : time extgenSeries2(M,N')\\
\     -- used 0.15 seconds\\
\emptyLine
\         -2     -1       2     2      2 2     2 3      2 4     3 5     $\cdot\cdot\cdot$\\
\       7u   + 2u   + 4t*u  - 7t u - 9t u  - 9t u  + 16t u  - 4t u  + 2 $\cdot\cdot\cdot$\\
o148 = --------------------------------------------------------------- $\cdot\cdot\cdot$\\
\                                                                       $\cdot\cdot\cdot$\\
\                                                                       $\cdot\cdot\cdot$\\
\emptyLine
o148 : Divide\\
\endOutput
\beginOutput
i149 : g'=time extgenSeries1(M,N')\\
\     -- used 0.18 seconds\\
\emptyLine
\                  2     3     4     5     6\\
\       9 + 4t - 9t  + 4t  + 8t  - 2t  - 2t\\
o149 = ------------------------------------\\
\                   2       2       2\\
\             (1 - t )(1 - t )(1 - t )\\
\emptyLine
o149 : Divide\\
\endOutput
\beginOutput
i150 : simplify g'\\
\emptyLine
\                  2     3     5\\
\       9 - 5t - 4t  + 8t  - 2t\\
o150 = ------------------------\\
\                  2       3\\
\           (1 + t) (1 - t)\\
\emptyLine
o150 : Divide\\
\endOutput
\end{sExample}

\subsection{Complexity}

The {\it\ie{complexity}\/} of a pair of $B$-modules $(M,N)$
is the least $d\in\N$ such that there exists
a polynomial of degree $d-1$ bounding above the function
\[
n \mapsto \dim_K \rExt^n_B(M,N)
\]
It is denoted $\cx_B(M,N)$ and measures on a polynomial scale the rate
of growth of the minimal number of generators of $\Ext^n_B(M,N)$; it
vanishes if and only if $\Ext^n_B(M,N)=0$ for all $n\gg0$.  Corollary
\ref{series} yields
\[
\gen^{M,N}_B(t)=\frac{h(t)}{(1-t^2)^c}
\qquad\text{for some}\qquad h(t)\in\Z[t]
\]
so decomposition into partial fractions shows that $\cx_R(M,N)$ equals
the order of the pole of $\gen^{M,N}_B(t)$ at $t=1$.  Alternatively,
$\cx_R(M,N)$ can be obtained by computing the Krull dimension of a
reduced Ext module over $R$.

\begin{sCode}
The function {\tt complexityPair(M,N)} yields the complexity of a pair
$(M,N)$ of graded modules over a graded complete intersection ring $B$.
\beginOutput
i151 : complexityPair = (M,N) -> dim ext(M,N);\\
\endOutput
 \end{sCode}

\begin{sExample}
For $M$, $N$, and $N'$ as in Example \ref{new module} we have
\beginOutput
i152 : time complexityPair(M,N)\\
\     -- used 0.39 seconds\\
\emptyLine
o152 = 2\\
\endOutput
\beginOutput
i153 : time complexityPair(M,N')\\
\     -- used 0.12 seconds\\
\emptyLine
o153 = 3\\
\endOutput
\end{sExample}

\subsection{Support Variety}
\label{Support variety2}

Let $\overline K$ be an algebraic closure of $K$.  The {\it\ie{support
variety}\/} $\var^*_B(M,N)$ is the algebraic set in ${\overline K}{}^c$
defined by the annihilator of $\rExt^\bu_B(M,N)$ over
$R=K[X_1,\dots,X_c]$.  It is clear from the definition that
$\var^*_B(M,k)$ is equal to the variety $\var^*_B(M)$ defined in
\ref{Support variety}.  One of the main results of
\cite[Sect.~5]{CI:AB2} shows that $\var^*_B(M,N)=\var^*_B(M)\cap
\var^*_B(N)$, and, as a consequence, $\var^*_B(M)=\var^*_B(M,M)=
\var^*_B(k,M)$.  The dimension of $\var^*_B(M,N)$ is equal to the
complexity $\cx_R(M,N)$, already computed above.

Feeding our computation of $\rExt^\bu_B(M,N)$ to standard \Mtwo
routines we write code for determining a set of equations defining
$\var^*_B(M,N)$.

\begin{sCode}
The function {\tt supportVarietyPairIdeal(M,N)} yields a set of polynomial
equations with coefficients in $K$, defining the variety
$\var^*_B(M,N)$ in ${\overline K}{}^c$ for graded modules $M$, $N$ over
a graded complete intersection $B$.
\beginOutput
i154 : supportVarietyPairIdeal = (M,N) -> ann ext(M,N);\\
\endOutput
 \end{sCode}

\begin{sExample}
For $M$, $N$, and $N'$ as in Example \ref{new module} we have
\beginOutput
i155 : time supportVarietyPairIdeal(M,N)\\
\     -- used 0.97 seconds\\
\emptyLine
o155 = ideal X\\
\              1\\
\emptyLine
o155 : Ideal of K [X , X , X , Degrees => \{\{-2, -3\}, \{-2, -4\}, \{-2, -5\}\}]\\
\                    1   2   3\\
\endOutput
\beginOutput
i156 : time supportVarietyPairIdeal(M,N')\\
\     -- used 1.73 seconds\\
\emptyLine
o156 = 0\\
\emptyLine
o156 : Ideal of K [X , X , X , Degrees => \{\{-2, -3\}, \{-2, -4\}, \{-2, -5\}\}]\\
\                    1   2   3\\
\endOutput
\end{sExample}

\appendix

\section{Gradings}
\label{Gradings}

Our purpose here is to set up a context in which the theory of Sections
\ref{Universal homotopies} and \ref{Cohomology operators} translates
into data that \Mtwo can operate with.

A first point is to develop a {\it flexible\/} and {\it consistent\/}
scheme within which to handle the two kinds of degrees we deal
with---the internal gradings of the input, and the homological degrees
created during computations.

A purely technological difficulty arises when our data are presented to
\Mtwo: it only accepts multidegrees whose first component is positive,
which is {\it not\/} the case for rings of cohomology operators.

A final point, mostly notational, tends to generate misunderstanding
and errors if left unaddressed.  On the printed page, the difference
between homological and cohomological conventions is handled
graphically by switching between sub- and super-indices, and reversing
signs; both authors were used to it, but \Mtwo has so far refused to
read \TeX\ printouts.

The {\sl raison d'\^etre\/} of the following remarks was to debug
communications between the three of us.

\begin{Remark}
\label{bigrading}
Only one degree, denoted $\deg$, appears in Section \ref{Graded
algebras}, and anywhere in the main text before Notation \ref{graded
stuff}; when needed, it will be referred to as {\it homological
degree\/}.

Assume that $A=\bigoplus_{h\in\Z}A_h$ is a graded ring.  Any element
$a$ of $A_h$ is said to be homogeneous of {\it\ie{internal degree}\/}
$h$; the notation for this is $\deg' a=h$.  Let ${\boldsymbol f}=\{f_1,
\dots, f_c\}$ be a Koszul-regular set consisting of homogeneous
elements.  We give the ring $B=A/({\boldsymbol f})$ the induced
grading, and extend the notation for internal degree to all graded
$B$-modules $M$.

Let $M$ be a graded $B$-module.  For any integer $e$, we let $M[e]$
denote the graded module with $M[e]_d = M_{d+e}$.  We take a projective
resolution $C$ of $M$ by graded $A$-modules, with differential $d_C$
preserving internal degrees.  Recall that we have been writing $\deg x
= n$ to indicate that $x$ is an element in $C_n$; we refer to this
situation also by saying that $x$ has {\it\ie{homological degree\/}}
$x$.  We combine both degrees in a single {\it\ie{bidegree\/}}, denoted
$\Deg$, as follows:
\[
\Deg x = (\deg x, \deg' x)
\]
For a bigraded module $H$ and pair of integers $(e, e')$, we let
$H[e,e']$ denote the bigraded module with $H[e,e']_{d,d'} =
H_{d+e,d'+e'}$.

Because $\deg Y_i = 2$, the elements of the free $B$-module $Q$ have
homological degree $2$.  We introduce an internal grading $\deg'$ on
$Q$ by setting $\deg' Y_i = r_i$, where $r_i=\deg' f_i$ for
$i=1,\dots,c$.  With this choice, the homomorphism $f\colon Q\to A$
acquires internal degree $0$ (of course, this was the reason behind our
choice of grading in the first place).  The internal grading on $Q$
defines, in the usual way, internal gradings on all symmetric and
exterior powers of $Q$ and $Q^*$; in particular, $\deg'Y^{(\a)}=\sum
\a_i r_i$ and $\deg' Y^{\wedge\b} = \sum \b_i r_i$.  Thus, the ring
$S=A[X_1,\dots,X_c]$ acquires a bigrading defined by
$\Deg a=(0, h)$ for all elements $a\in A_h$ and $\Deg X_i=(-2,-r_i)$
for $i=1,\dots,c$.

In this context, we call $S$ the {\it\ie{bigraded ring of cohomology
operators}\/}.

Since the differential $d_C$ has internal degree $0$, a null-homotopic
chain map $C\to C$ which is homogeneous of internal degree $r$ will
have a null-homotopy that is itself homogeneous of internal degree
$r$.  In the proof of Theorem \ref{main} we construct maps $d_\g$ as
null-homotopies, so we may arrange for them to be homogeneous maps with
$\deg' d_\g = \sum \g_i d_i$.  Our grading assumptions guarantee that
$d$ is homogeneous with $\Deg d=(-1,0)$.

With these data, the $B$-free resolution $C\otimes_A D'$ provided by
Theorem \ref{resolution} becomes one by graded $B$-modules, and
its differential $\partial$ is homogeneous with $\Deg
\partial=(-1,0)$.  For any graded $B$-module $N$, these properties are
transferred to the complex $\Hom_B(C\otimes_A D',N)$ and its
differential.
 \end{Remark}

We sum up the contents of Remarks \ref{action} and \ref{bigrading}.

\begin{Remark}
\label{graded action}
If $A$ is a graded ring, $\{f_1, \dots, f_c\}$ is a Koszul-regular set
consisting of homogeneous elements, $B$ is the residue ring
$A/({\boldsymbol f})$, and $M,N$ are graded $B$-modules, then
$\Ext^\bu_B(M,N)$ is a bigraded module over the ring
$S=A[X_1,\dots,X_c]$, itself bigraded by setting $\Deg a = (0,
\deg'(a))$ for all homogeneous $a\in A$ and $\Deg X_i =
(-2,-\deg'(f_i))$ for $i=1,\dots,c$.
 \end{Remark}

\begin{Remark}
\label{macaulay grading}
The core algorithms of the program can handle multi-graded rings and
modules, but only if each variable in the ring has positive first
component of its multi-degree.  At the moment, a user who needs a
multi-graded ring {\tt R} which violates this requirement must provide
two linear maps: {\tt R.Adjust}, that transforms the desired
multi-degrees into ones satisfying this requirement, as well as its
inverse map, {\tt R.Repair}.  The routine {\tt Ext}, discussed above,
incorporates such adjustments for the rings of cohomology operators it
creates.  When we wish to create related rings with some of the same
multi-degrees, we may use the same adjustment operator.
 \end{Remark}

% Local Variables:
% mode: latex
% mode: reftex
% compile-command: "make ci-wrapper.dvi"
% tex-main-file: "ci-wrapper.tex"
% reftex-keep-temporary-buffers: t
% reftex-use-external-file-finders: t
% reftex-external-file-finders: (("tex" . "make FILE=%f find-tex") ("bib" . "make FILE=%f find-bib"))
% End:
\begin{thebibliography}{10}

\bibitem{CI:MR86i:55011a}
David~J. Anick:
\newblock A counterexample to a conjecture of {S}erre.
\newblock {\em Ann. of Math. (2)}, 115(1):1--33, 1982\ \,and {116}(3):661,
  1982.

\bibitem{CI:MR90g:13027}
Luchezar~L. Avramov:
\newblock Modules of finite virtual projective dimension.
\newblock {\em Invent. Math.}, 96(1):71--101, 1989.

\bibitem{CI:res}
Luchezar~L. Avramov:
\newblock Infinite free resolutions.
\newblock In {\em Six lectures on commutative algebra (Bellaterra, 1996)},
  pages 1--118. Progress in Math., vol 166, Birkh\"auser, Basel, 1998.

\bibitem{CI:MR1774757}
Luchezar~L. Avramov and Ragnar-Olaf Buchweitz:
\newblock Homological algebra modulo a regular sequence with special attention
  to codimension two.
\newblock {\em J. Algebra}, 230(1):24--67, 2000.

\bibitem{CI:AB2}
Luchezar~L. Avramov and Ragnar-Olaf Buchweitz:
\newblock Support varieties and cohomology over complete intersections.
\newblock {\em Invent. Math.}, 142(2):285--318, 2000.

\bibitem{CI:MR99c:13033}
Luchezar~L. Avramov, Vesselin~N. Gasharov, and Irena~V. Peeva:
\newblock Complete intersection dimension.
\newblock {\em Inst. Hautes \'Etudes Sci. Publ. Math.}, 86:67--114, 1997.

\bibitem{CI:AP}
Luchezar~L. Avramov and Irena Peeva:
\newblock Finite regularity and {K}oszul algebras.
\newblock {\em Amer. J. Math}, 123(2):275--281, 2001.

\bibitem{CI:MR2000e:13021}
Luchezar~L. Avramov and Li-Chuan Sun:
\newblock Cohomology operators defined by a deformation.
\newblock {\em J. Algebra}, 204(2):684--710, 1998.

\bibitem{CI:MR85a:20004}
Jon~F. Carlson:
\newblock The varieties and the cohomology ring of a module.
\newblock {\em J. Algebra}, 85(1):104--143, 1983.

\bibitem{CI:MR86b:20009}
Jon~F. Carlson:
\newblock The variety of an indecomposable module is connected.
\newblock {\em Invent. Math.}, 77(2):291--299, 1984.

\bibitem{CI:Ei}
David Eisenbud:
\newblock Homological algebra on a complete intersection, with an application
  to group representations.
\newblock {\em Trans. Amer. Math. Soc.}, 260(1):35--64, 1980.

\bibitem{CI:MR51:487}
Tor~H. Gulliksen:
\newblock A change of ring theorem with applications to {P}oincar\'e series and
  intersection multiplicity.
\newblock {\em Math. Scand.}, 34(1):167--183, 1974.

\bibitem{CI:Me}
Vikram Mehta:
\newblock {\em Endomorphisms of complexes and modules over {G}olod rings}.
\newblock Thesis. University of California, Berkeley, CA, 1975.

\bibitem{CI:MR94b:16040}
Jan-Erik Roos:
\newblock Commutative non-{K}oszul algebras having a linear resolution of
  arbitrarily high order. {A}pplications to torsion in loop space homology.
\newblock {\em C. R. Acad. Sci. Paris S\'er. I Math.}, 316(11):1123--1128,
  1993.

\bibitem{CI:Sh}
Jack Shamash:
\newblock The {P}oincar\'e series of a local ring.
\newblock {\em J. Algebra}, 12:453--470, 1969.

\end{thebibliography}
\egroup
\makeatletter
\renewcommand\thesection{\@arabic\c@section}
\makeatother



%%%%%%%%%%%%%%%%%%%%%%%%%%%%%%%%%%%%%%%%%%%%%%%%
%%%%%
%%%%% ../chapters/toricHilbertScheme/chapter
%%%%%
%%%%%%%%%%%%%%%%%%%%%%%%%%%%%%%%%%%%%%%%%%%%%%%%

\bgroup
\def\A{\mathbb{A}}
\def\k{\mathbb{C}}
\def\N{\mathbb{N}}
\def\R{\mathbb{R}}
\def\P{\mathbb{P}}
\def\ZZ{\mathbb{Z}} 

\title{Algorithms for the Toric Hilbert Scheme}
\titlerunning{Toric Hilbert Schemes}
\toctitle{Algorithms for the Toric Hilbert Scheme}
\author{Michael Stillman
        % \inst 1
         \and Bernd Sturmfels
        % \inst 2
         \and Rekha Thomas 
        % \inst 3
        }
\authorrunning{M. Stillman, B. Sturmfels, and R. Thomas}
% \institute{Cornell University, Department of Mathematics, Ithaca, NY 14853, USA
%         \and UC Berkeley, Department of Mathematics, Berkeley, CA 94720, USA
%         \and University of Washington, Department of Mathematics, Seattle, WA 98195, USA}
\maketitle

\begin{abstract}
The toric Hilbert scheme parametrizes all algebras isomorphic to a
given semigroup algebra as a multigraded vector space. All components
of the scheme are toric varieties, and among them, there is a fairly
well understood coherent component. It is unknown whether
toric Hilbert schemes are always connected. In this chapter we
illustrate the use of \Mtwo for exploring the structure of toric
Hilbert schemes. In the process we will encounter algorithms from
commutative algebra, algebraic geometry, polyhedral theory and
geometric combinatorics.
\end{abstract}

\section*{Introduction}
Consider the multigrading of the polynomial ring $R =
\k[x_1,\ldots,x_n]$ specified by a non-negative integer $d \times
n$-matrix $A = (a_1,\ldots,a_n)$ such that degree $(x_i) = a_i \in
\N^d$. This defines a decomposition $\, R = \bigoplus_{b \in \N A} R_b
$, where $\N A$ is the subsemigroup of $\N^d$ spanned by
$a_1,\ldots,a_n$, and $R_b$ is the $\k$-span of all monomials $\, x^u
= x_1^{u_1}\cdots x_n^{u_n}$ with degree $Au = a_1 u_1 +\cdots + a_n
u_n = b$.  The {\it \ie{toric Hilbert scheme}} $\,Hilb_A 
\,$ parametrizes all $A$-homogeneous ideals $I \subset R$ (ideals that
are homogeneous under the multigrading of $R$ by $\N A$) with the
property that $(R/I)_b$ is a $1$-dimensional $\k$-vector space, for all
$b \in \N A$. We call such an ideal $I$ an $A$-{\em graded}\index{ideal!$A$-graded} ideal.
Equivalently, $I$ is $A$-graded if it is $A$-homogeneous and $R/I$ is
isomorphic as a multigraded vector space to the semigroup algebra $\,
\k [ \N A ] = R/I_A$, where $$I_A := \,\langle x^u - x^v \, : \, Au =
Av \rangle \subset R$$ is the {\it \ie{toric ideal}} of $A$. An $A$-graded
ideal is generated by binomials and monomials in $R$ since, by
definition, any two monomials $x^u$ and $x^v$ of the same degree $Au = 
Av$ must be $\k$-linearly dependent modulo the ideal.

We recommend \cite[\S 4, \S 10]{HS:St2} as an introductory reference for the 
topics in this chapter.
The study of toric Hilbert schemes for $d=1$ goes back to
Arnold \cite{HS:Arn} and Korkina et al.\cite{HS:KPR}, and it was
further developed by Sturmfels  (\cite{HS:St1} and \cite[\S 10]{HS:St2}). 
Peeva and Stillman \cite{HS:PS1} introduced the scheme structure 
that gives the toric Hilbert scheme its universal property,
and from this they derive a formula for the tangent space
of a point on  $\, Hilb_A $. Maclagan recently showed that the 
quadratic binomials in \cite[\S 5]{HS:St1} define the same scheme as the
determinantal equations in \cite{HS:PS1}.
Both of these systems of global equations are 
generally much too big for 
practical computations. Instead, most of our algorithms are based on
the local equations given by Peeva and Stillman in \cite{HS:PS2}
and the combinatorial approach of Maclagan and Thomas in \cite{HS:MT}.

We begin with the computation of a toric ideal using \Mtwo. Our
running example throughout this chapter is the following $2 \times
5$-matrix:
\begin{equation}
\label{OurMatrix}
A = \left( \begin{matrix}
           1 & 1 & 1 & 1 & 1  \\ 
           0 & 1 & 2 & 7 & 8 
\end{matrix} \right),
\end{equation}
which we input to \Mtwo as a list of lists of 
integers.
\beginOutput
i1 : A = \{\{1,1,1,1,1\},\{0,1,2,7,8\}\}; \\
\endOutput
The toric ideal of $A$ lives in the multigraded ring $R := \k [a,b,c,d,e]$.
\beginOutput
i2 : R = QQ[a..e,Degrees=>transpose A]; \\
\endOutput
\beginOutput
i3 : describe R \\
\emptyLine
o3 = QQ [a, b, c, d, e, Degrees => \{\{1, 0\}, \{1, 1\}, \{1, 2\}, \{1, 7\}, \{1 $\cdot\cdot\cdot$\\
\endOutput

We use Algorithm 12.3 in \cite{HS:St2} to compute $I_A$. The first step is
to find a matrix $B$ whose rows generate the lattice $ker_{\ZZ}(A)
:= \{x \in \ZZ^n : Ax = 0 \}$. 

\beginOutput
i4 : B = transpose syz matrix A \\
\emptyLine
o4 = | 1 -2 1  0 0 |\\
\     | 0 5  -6 1 0 |\\
\     | 0 6  -7 0 1 |\\
\emptyLine
\              3        5\\
o4 : Matrix ZZ  <--- ZZ\\
\endOutput

Although in theory any basis of $ker_{\ZZ}(A)$ will suffice, in
practice it is more efficient to use a {\em reduced} basis
\cite[\S 6.2]{HS:Sch}, which can be computed using the {\em \ie{basis
reduction}} package {\tt LLL.m2} in \Mtwo. The command {\tt LLL} 
when applied to the output of {\tt syz matrix A} will return a 
matrix of the same size whose columns form a reduced lattice basis 
for $ker_{\ZZ}(A)$. The output appears in compressed form as follows:

\beginOutput
i5 : load "LLL.m2"; \\
\endOutput
\beginOutput
i6 : LLL syz matrix A \\
\emptyLine
o6 = | 0  1  2  |\\
\     | 1  -1 0  |\\
\     | -1 0  -3 |\\
\     | -1 -1 2  |\\
\     | 1  1  -1 |\\
\emptyLine
\              5        3\\
o6 : Matrix ZZ  <--- ZZ\\
\endOutput

We recompute $B$ using this package to get the following $3 \times 5$ matrix.
\beginOutput
i7 : B = transpose LLL syz matrix A \\
\emptyLine
o7 = | 0 1  -1 -1 1  |\\
\     | 1 -1 0  -1 1  |\\
\     | 2 0  -3 2  -1 |\\
\emptyLine
\              3        5\\
o7 : Matrix ZZ  <--- ZZ\\
\endOutput

The advantage of a reduced basis may not be apparent in small
examples. However, as the size of $A$ increases, it becomes
increasingly important for the termination of Algorithm 12.3 in \cite{HS:St2}. (To
appreciate this, consider the matrix (\ref{non-normal}) from 
Section~4.)

A row $b = b^+ - b^-$ of $B$ is then coded as the binomial
$x^{b^+}-x^{b^-} \in R$, and we let $J$ be the ideal generated by all 
such binomials. 

\beginOutput
i8 : toBinomial = (b,R) -> (\\
\          top := 1_R; bottom := 1_R;\\
\          scan(#b, i -> if b_i > 0 then top = top * R_i^(b_i)\\
\               else if b_i < 0 then bottom = bottom * R_i^(-b_i));\\
\          top - bottom); \\
\endOutput

\beginOutput
i9 : J = ideal apply(entries B, b -> toBinomial(b,R)) \\
\emptyLine
\                                       2 2    3\\
o9 = ideal (- c*d + b*e, - b*d + a*e, a d  - c e)\\
\emptyLine
o9 : Ideal of R\\
\endOutput
The toric ideal equals $(J : (x_1 \cdots x_n)^\infty)$, which is 
computed via $n$ successive saturations as follows:
\beginOutput
i10 : scan(gens ring J, f -> J = saturate(J,f))\\
\endOutput

Putting the above pieces of code together, we get the following
procedure for computing the toric ideal of a matrix $A$.

\beginOutput
i11 : toricIdeal = (A) -> (\\
\          n := #(A_0);  \\
\          R = QQ[vars(0..n-1),Degrees=>transpose A,MonomialSize=>16]; \\
\          B := transpose LLL syz matrix A;\\
\          J := ideal apply(entries B, b -> toBinomial(b,R));\\
\          scan(gens ring J, f -> J = saturate(J,f));\\
\          J\\
\          ); \\
\endOutput

See \cite{HS:BLR}, \cite{HS:HS} and \cite[\S 4, \S 12]{HS:St2} for other
algorithms for computing toric ideals and various ideas for
speeding up the computation.

In our example, $I_A = \langle
cd-be,bd-ae,b^2-ac,a^2d^2-c^3e,c^4-a^3e,bc^3-a^3d,
ad^4-c^2e^3,d^6-ce^5 \rangle$, which we now compute using this
procedure.  

\beginOutput
i12 : I = toricIdeal A; \\
\emptyLine
o12 : Ideal of R\\
\endOutput
 
\beginOutput
i13 : transpose mingens I\\
\emptyLine
o13 = \{-2, -9\}  | cd-be    |\\
\      \{-2, -8\}  | bd-ae    |\\
\      \{-2, -2\}  | b2-ac    |\\
\      \{-4, -14\} | a2d2-c3e |\\
\      \{-4, -8\}  | c4-a3e   |\\
\      \{-4, -7\}  | bc3-a3d  |\\
\      \{-5, -28\} | ad4-c2e3 |\\
\      \{-6, -42\} | d6-ce5   |\\
\emptyLine
\              8       1\\
o13 : Matrix R  <--- R\\
\endOutput

This ideal defines an embedding of
$\P^1$ as a degree $8$ curve into $\P^4$. We will see in Section 3
that its toric Hilbert scheme $Hilb_A$ has a non-reduced component.

This chapter is organized into four sections and two appendices as
follows. The main goal in Section~1 is to describe an algorithm for
generating all monomial $A$-graded ideals for a given $A$. These
monomial ideals are the vertices of the {\em flip graph} of $A$ whose
connectivity is equivalent to the connectivity of $Hilb_A$. We
describe how all neighbors of a given vertex of this graph can be
calculated. In Section~2, we explain the role of polyhedral geometry
in the study of $Hilb_A$. Our first algorithm tests for {\em
coherence} in a monomial $A$-graded ideal. We then show how to compute
the polyhedral complexes supporting $A$-graded ideals, which in turn
relate the flip graph of $A$ to the {\em \ie{Baues graph}} of $A$.  For
unimodular matrices, these two graphs coincide and hence our method of
computing the flip graph can be used to compute the Baues
graph. Section~3 explores the components of $Hilb_A$ via local
equations around the torus fixed points of the scheme. We include a
combinatorial interpretation of these local equations from the point
of view of integer programming.  The scheme $Hilb_A$ has a {\em
coherent} component, which is examined in detail in Section~4. We prove
that this component is, in general, not normal and that its
normalization is the toric variety of the Gr\"obner fan of $I_A$. We
conclude the chapter with two appendices, each containing one large
piece of \Mtwo code that we use in this chapter. Appendix \ref{FMe} displays
code from the \Mtwo file {\tt polarCone.m2} that is used to convert a generator
representation of a polyhedron to an inequality representation and
vice versa. Appendix \ref{Mpor} displays code from the file {\tt minPres.m2} used for computing minimal
presentations of polynomial quotient rings. The main ingredient of
this package is the subroutine {\tt removeRedundantVariables}, which is
what we use in this chapter.

\section{Generating Monomial Ideals}
We start out by computing the {\it \ie{Graver basis}} $Gr_A$, which is the
set of binomials in $I_A$ that are minimal with respect to the
partial order defined by $$\, x^u - x^v \,\leq\, x^{u'} - x^{v'} \quad \iff
\quad \hbox{ $x^u$ divides $x^{u'}$ \ and \ $x^v$ divides $x^{v'}$.}
$$ The set $Gr_A$ is a {\em universal Gr\"obner basis}\index{Grobner basis@Gr\"obner basis!universal} of $I_A$ and
has its origins in the theory of integer programming \cite{HS:Gra}. It
can be computed using \cite[Algorithm 7.2]{HS:St2}, a \Mtwo version of
which is given below.

\beginOutput
i14 : graver = (I) -> (\\
\          R := ring I;\\
\          k := coefficientRing R;\\
\          n := numgens R;\\
\          -- construct new ring S with 2n variables\\
\          S := k[Variables=>2*n,MonomialSize=>16];\\
\          toS := map(S,R,(vars S)_\{0..n-1\});\\
\          toR := map(R,S,vars R | matrix(R, \{toList(n:1)\}));\\
\          -- embed I in S\\
\          m := gens toS I;\\
\          -- construct the toric ideal of the Lawrence \\
\          -- lifting of A\\
\          i := 0;\\
\          while i < n do (\\
\              wts := join(toList(i:0),\{1\},toList(n-i-1:0));\\
\              wts = join(wts,wts);\\
\              m = homogenize(m,S_(n+i),wts);\\
\              i=i+1;\\
\              );\\
\         J := ideal m;\\
\         scan(gens ring J, f -> J = saturate(J,f));\\
\         -- apply the map toR to the minimal generators of J \\
\         f := matrix entries toR mingens J;\\
\         p := sortColumns f;\\
\         f_p) ;  \\
\endOutput
   
   The above piece of code first constructs a new polynomial ring $S$
   in $n$ more variables than $R$. Assume $S = \k [x_1, \ldots, x_n,
   y_1, \ldots, y_n]$. The inclusion map {\tt toS} $: R \rightarrow
   S$ embeds the toric ideal $I$ in $S$ and collects its generators in
   the matrix {\tt m}. A binomial $x^a - x^b$ lies in $Gr_A$ if and only
   if $x^ay^b-x^by^a$ is a minimal generator of the toric ideal in $S$
   of the $(d+n) \times 2n$ matrix $$\Lambda(A) := \left (
     \begin{array}{cc} A & 0 \\ I_n & I_n \end{array} \right),$$ which
   is called the {\em \ie{Lawrence lifting}} of $A$. Since $u \in
   ker_{\ZZ}(A) \Leftrightarrow (u,-u) \in ker_{\ZZ} (\Lambda(A))$, we
   use the {\tt while}\indexcmd{while} loop to homogenize the binomials in {\tt m} with
   respect to $\Lambda(A)$, using the $n$ new variables in $S$. This
   converts a binomial $x^a-x^b \in$ {\tt m} to the binomial
   $x^ay^b-x^by^a$.  The ideal generated by these new binomials is
   labeled $J$. As before, we can now successively saturate $J$
   to get the toric ideal of $\Lambda(A)$ in $S$. The image of the
   minimal generators of this toric ideal under the map {\tt toR}
   $: S \rightarrow R$ such that $x_i \mapsto x_i$ and $y_i \mapsto 1$
   is precisely the Graver basis $Gr_A$. These binomials are the
   entries of the matrix {\tt f} and is output by the program.

In our example $Gr_A$ consists of $42$ binomials.
\beginOutput
i15 : Graver = graver I \\
\emptyLine
o15 = | -cd+be -bd+ae -b2+ac -cd2+ae2 -a2d2+c3e -c4+a2bd -c4+a3e -bc3+ $\cdot\cdot\cdot$\\
\emptyLine
\              1       42\\
o15 : Matrix R  <--- R\\
\endOutput
 

Returning to the general case, an element $b $ of $\N A$ is called a
{\it \ie{Graver degree}} if there exists a binomial $x^u - x^v$ in the
Graver basis $Gr_A$ such that $Au = Av = b$. If $b$ is a Graver degree
then the set of monomials in $R_b$ is the corresponding {\it \ie{Graver
  fiber}}.  In our running example there are $37$ distinct Graver
fibers. We define the {\tt ProductIdeal} of $A$ as $PI := 
\langle x^ax^b : x^a-x^b  \in Gr_A \rangle$. This ideal is contained in
every monomial ideal of $Hilb_A$ and hence no monomial in $PI$ can be
a standard monomial of a monomial $A$-graded ideal. Since our purpose
in constructing Graver fibers is to use them to 
generate all monomial $A$-graded ideals, we will be content with
listing just the monomials in each Graver fiber that do not lie in
$PI$.  Since $R$ is multigraded by $A$, we can obtain such a
presentation of a Graver fiber by simply asking for the basis of $R$
in degree $b$ modulo $PI$.  

\beginOutput
i16 : graverFibers = (Graver) -> (\\
\           ProductIdeal := (I) -> ( trim ideal(\\
\              apply(numgens I, a -> ( \\
\                  f := I_a; leadTerm f * (leadTerm f - f))))); \\
\           PI := ProductIdeal ideal Graver; \\
\           R := ring Graver; \\
\           new HashTable from apply(\\
\               unique degrees source Graver,\\
\               d -> d => compress (basis(d,R) {\char`\%} PI) ));\\
\endOutput

\beginOutput
i17 : fibers = graverFibers Graver \\
\emptyLine
o17 = HashTable\{\{2, 2\} => | ac b2 |                                  \}\\
\                \{2, 8\} => | ae bd |\\
\                \{2, 9\} => | be cd |\\
\                \{3, 16\} => | ae2 bde cd2 |\\
\                \{4, 14\} => | a2d2 c3e |\\
\                \{4, 7\} => | a3d bc3 |\\
\                \{4, 8\} => | a3e a2bd c4 |\\
\                \{5, 10\} => | a3ce a2b2e a2bcd ab3d c5 |\\
\                \{5, 14\} => | a3d2 ac3e b2c2e bc3d |\\
\                \{5, 16\} => | a3e2 a2cd2 ab2d2 c4e |\\
\                \{5, 21\} => | a2d3 bc2e2 c3de |\\
\                \{5, 22\} => | a2d2e abd3 c3e2 |\\
\                \{5, 28\} => | ad4 c2e3 |\\
\                \{5, 7\} => | a4d abc3 b3c2 |\\
\                \{5, 8\} => | a4e a3bd ac4 b2c3 |\\
\                \{6, 12\} => | a3c2e a2bc2d ab4e b5d c6 |\\
\                \{6, 14\} => | a4d2 a2c3e abc3d b4ce b3c2d |\\
\                \{6, 18\} => | a3ce2 a2b2e2 a2c2d2 b4d2 c5e |\\
\                \{6, 21\} => | a3d3 abc2e2 ac3de b3ce2 bc3d2 |\\
\                \{6, 24\} => | a3e3 a2cd2e abcd3 b3d3 c4e2 |\\
\                \{6, 28\} => | a2d4 ac2e3 b2ce3 c3d2e |\\
\                \{6, 30\} => | a2d2e2 acd4 b2d4 c3e3 |\\
\                \{6, 35\} => | ad5 bce4 c2de3 |\\
\                \{6, 36\} => | ad4e bd5 c2e4 |\\
\                \{6, 42\} => | ce5 d6 |\\
\                \{6, 7\} => | a5d a2bc3 b5c |\\
\                \{6, 8\} => | a5e a4bd a2c4 b4c2 |\\
\                \{7, 14\} => | a5d2 a3c3e a2bc3d b6e b5cd c7 |\\
\                \{7, 21\} => | a4d3 a2bc2e2 a2c3de abc3d2 b5e2 b3c2d2 |\\
\                \{7, 28\} => | a3d4 a2c2e3 ac3d2e b4e3 bc3d3 |\\
\                \{7, 35\} => | a2d5 abce4 ac2de3 b3e4 c3d3e |\\
\                \{7, 42\} => | ace5 ad6 b2e5 c2d2e3 |\\
\                \{7, 49\} => | be6 cde5 d7 |\\
\                \{7, 7\} => | a6d a3bc3 b7 |\\
\                \{7, 8\} => | a6e a5bd a3c4 b6c |\\
\                \{8, 56\} => | ae7 bde6 cd2e5 d8 |\\
\                \{8, 8\} => | a7e a6bd a4c4 b8 |\\
\emptyLine
o17 : HashTable\\
\endOutput

For example, the Graver degree $(8,8)$ corresponds to the Graver fiber
$$ \bigl\{\,
\underline{a^7 e}, \, \underline{a^6 b d},\,  \underline{a^4 c^4}, \,
a^3 b^2 c^3,\, a^2 b^4 c^2,\,  a b^6 c, \, \underline{b^8} \,\bigr\}.$$
Our \Mtwo code outputs only the four underlined monomials,
in the format {\tt  | a7e a6bd a4c4 b8 |}. The three non-underlined 
monomials lie in the {\tt ProductIdeal}. Graver degrees are
important because of the following result.

\begin{lemma}[{\cite[Lemma 10.5]{HS:St2}}]
The multidegree of any minimal generator of any ideal 
$I$ in $Hilb_A$ is a Graver degree.
\end{lemma}

The next step in constructing the toric Hilbert scheme is to compute
all its fixed points with respect to the scaling action of the
$n$-dimensional algebraic torus $(\k^*)^n$. (The torus $(\k^{\ast})^n$
acts on $R$ by scaling variables : $\lambda \mapsto \lambda \cdot x :=
(\lambda_1 x_1, \ldots, \lambda_n x_n)$.)  These fixed points are the
monomial ideals $M$ lying on $Hilb_A$.  Every term order $\prec$ on
the polynomial ring $R$ gives such a monomial ideal: $M = in_\prec(I_A
)$, the initial ideal of the toric ideal $I_A$ with respect to
$\prec$. Two ideals $J$ and $J'$ are said to be {\em torus
  isomorphic}\index{ideal!torus isomorphism}
if $J = \lambda \cdot J'$ for some $\lambda \in (\k^{\ast})^n$. Any
monomial $A$-graded ideal that is torus isomorphic to an initial ideal
of $I_A$ is said to be {\em coherent}\index{ideal!coherent}. In particular, the initial
ideals of $I_A$ are coherent and they can be computed by
\cite[Algorithm 3.6]{HS:St2} applied to $I_A$. A refinement and fast
implementation can be found in the software package {\tt TiGERS} by
Huber and Thomas \cite{HS:HT}.

Now we wish to compute all monomial ideals $M$ on $Hilb_A$ regardless
of whether $M$ is coherent or not. For this we use the procedure
{\tt generateAmonos} given below. This procedure takes in the Graver
basis $Gr_A$ and records the numerator of the Hilbert series of $I_A$
in {\tt trueHS}. It then computes the Graver fibers of $A$, sorts them
and calls the subroutine {\tt selectStandard} to generate a
candidate for a monomial ideal on $Hilb_A$.

\beginOutput
i18 : generateAmonos = (Graver) -> (\\
\           trueHS := poincare coker Graver;\\
\           fibers := graverFibers Graver;\\
\           fibers = apply(sort pairs fibers, last);\\
\           monos = \{\};\\
\           selectStandard := (fibers, J) -> (\\
\           if #fibers == 0 then (\\
\              if trueHS == poincare coker gens J\\
\              then (monos = append(monos,flatten entries mingens J));\\
\           ) else (\\
\              P := fibers_0;\\
\              fibers = drop(fibers,1);\\
\              P = compress(P {\char`\%} J);\\
\              nP := numgens source P; \\
\              -- nP is the number of monomials not in J.\\
\              if nP > 0 then (\\
\                 if nP == 1 then selectStandard(fibers,J)\\
\                 else (--remove one monomial from P,take the rest.\\
\                       P = flatten entries P;\\
\                       scan(#P, i -> (\\
\                            J1 := J + ideal drop(P,\{i,i\});\\
\                            selectStandard(fibers, J1)))));\\
\           ));\\
\           selectStandard(fibers, ideal(0_(ring Graver)));\\
\           ) ; \\
\endOutput

The arguments to the subroutine {\tt selectStandard}
are the Graver fibers given as a list of matrices and a monomial 
ideal $J$ that should be included in every $A$-graded ideal 
that we generate. The subroutine then loops through each Graver fiber, 
and at each step selects a standard monomial from that fiber and 
updates the ideal $J$ by adding the other monomials in this fiber 
to $J$. The final $J$ output by the subroutine is the candidate ideal
that is sent back to {\tt generateAmonos}. It is stored by the program 
if its Hilbert series agrees with that of $I_A$. 
All the monomial $A$-graded ideals are stored in the list {\tt monos}.
Below, we ask \Mtwo for the cardinality of {\tt monos} and its 
first ten elements.
\beginOutput
i19 : generateAmonos Graver;\\
\endOutput
\beginOutput
i20 : #monos \\
\emptyLine
o20 = 281\\
\endOutput
\beginOutput
i21 : scan(0..9, i -> print toString monos#i) \\
\{c*d, b*d, b^2, c^3*e, c^4, b*c^3, c^2*e^3, b*c^2*e^2, b*c*e^4, d^6\}\\
\{c*d, b*d, b^2, c^3*e, c^4, b*c^3, c^2*e^3, b*c^2*e^2, c*e^5, b*c*e^4, $\cdot\cdot\cdot$\\
\{c*d, b*d, b^2, c^3*e, c^4, b*c^3, c^2*e^3, b*c^2*e^2, c*e^5, b*c*e^4, $\cdot\cdot\cdot$\\
\{c*d, b*d, b^2, c^3*e, c^4, b*c^3, c^2*e^3, b*c^2*e^2, c*e^5, b*c*e^4, $\cdot\cdot\cdot$\\
\{c*d, b*d, b^2, c^3*e, c^4, b*c^3, c^2*e^3, b*c^2*e^2, d^6, a*d^5\}\\
\{c*d, b*d, b^2, c^3*e, c^4, b*c^3, b*c^2*e^2, a*d^4, d^6\}\\
\{c*d, b*d, b^2, c^3*e, c^4, b*c^3, a*d^4, a^2*d^3, d^6\}\\
\{c*d, b*d, b^2, a^2*d^2, c^4, b*c^3, a*d^4, d^6\}\\
\{c*d, b*d, b^2, a^2*d^2, a^3*d, c^4, a*d^4, d^6\}\\
\{c*d, b*d, b^2, a^3*e, a^2*d^2, a^3*d, a*d^4, d^6\}\\
\endOutput

The monomial ideals (torus-fixed points) on $Hilb_A$ form the vertices
of the {\it \ie{flip graph}} of $A$ whose edges correspond to the
torus-fixed curves on $Hilb_A$. This graph was introduced in \cite{HS:MT}
and provides structural information about $Hilb_A$.  The edges
emanating from a monomial ideal $M$ can be constructed as follows: 
For any minimal generator $x^u$ of $M$, let $x^v$ be the unique
monomial with $x^v \not\in M$ and $Au = Av$. Form the {\it \ie{wall ideal}},
which is generated by $x^u - x^v$ and all minimal generators of $M$
other than $x^u$, and let $M'$ be the initial monomial ideal of the
wall ideal with respect to any term order $\succ$ for which $x^v \succ
x^u$. It can be shown that $M'$ is the unique initial monomial ideal
of the wall ideal that contains $x^v$.  If $M'$ lies on $Hilb_A$ then
$\{M, M'\}$ is an edge of the flip graph. We now illustrate the \Mtwo
procedure for computing all flip neighbors of a monomial $A$-graded
ideal.
 
\beginOutput
i22 : findPositiveVector = (m,s) -> (\\
\           expvector := first exponents s - first exponents m;\\
\           n := #expvector;\\
\           i := first positions(0..n-1, j -> expvector_j > 0);\\
\           splice \{i:0, 1, (n-i-1):0\}\\
\           );\\
\endOutput

\beginOutput
i23 : flips = (M) -> (\\
\           R := ring M;\\
\           -- store generators of M in monoms\\
\           monoms := first entries generators M;\\
\           result := \{\};\\
\           -- test each generator of M to see if it leads to a neighbor \\
\           scan(#monoms, i -> (\\
\             m := monoms_i;\\
\             rest := drop(monoms,\{i,i\});\\
\             b := basis(degree m, R);\\
\             s := (compress (b {\char`\%} M))_(0,0);\\
\             J := ideal(m-s) + ideal rest;\\
\             if poincare coker gens J == poincare coker gens M then (\\
\               w := findPositiveVector(m,s);\\
\               R1 := (coefficientRing R)[generators R, Weights=>w];\\
\               J = substitute(J,R1);\\
\               J = trim ideal leadTerm J;\\
\               result = append(result,J);\\
\               )));\\
\           result\\
\      );\\
\endOutput

The code above inputs a monomial $A$-graded ideal $M$ whose minimal
generators are stored in the list {\tt monoms}. The flip neighbors of
$M$ will be stored in {\tt result}. For each monomial $x^u$ in {\tt
monoms} we need to test whether it yields a flip neighbor of $M$ or
not. At the $i$-th step of this loop, we let {\tt m} be the $i$-th
monomial in {\tt monoms}. The list {\tt rest} contains all monomials
in {\tt monoms} except {\tt m}. We compute the standard monomial {\tt
s} of $M$ of the same degree as $m$.  The wall ideal of $m-s$ is the
binomial ideal $J$ generated by $m-s$ and the monomials in {\tt
rest}. We then check whether $J$ is $A$-graded by comparing its
Hilbert series with that of $M$. (Alternately, one could check whether
$M$ is the initial ideal of the wall ideal with respect to $m \succ
s$.) If this is the case, we use the subroutine {\tt
findPositiveVector} to find a unit vector $w = (0,\ldots,1,\ldots,0)$
such that $w \cdot s > w \cdot m$. The flip neighbor is then the
initial ideal of $J$ with respect to $w$ and it is stored in {\tt
result}. The program outputs the minimal generators of each flip
neighbor. Here is an example.
 
\beginOutput
i24 : R = QQ[a..e,Degrees=>transpose A];\\
\endOutput
\beginOutput
i25 : M = ideal(a*e,c*d,a*c,a^2*d^2,a^2*b*d,a^3*d,c^2*e^3,\\
\                c^3*e^2,c^4*e,c^5,c*e^5,a*d^5,b*e^6);\\
\emptyLine
o25 : Ideal of R\\
\endOutput
\beginOutput
i26 : F = flips M\\
\emptyLine
\                              2 2   3    4   2 3   3 2     5     5     $\cdot\cdot\cdot$\\
o26 = \{ideal (a*e, c*d, a*c, a d , a d, c , c e , c e , a*d , c*e , b* $\cdot\cdot\cdot$\\
\emptyLine
o26 : List\\
\endOutput
\beginOutput
i27 : #F\\
\emptyLine
o27 = 4\\
\endOutput
\beginOutput
i28 : scan(#F, i -> print toString entries mingens F_i)\\
\{\{a*e, c*d, a*c, a^2*d^2, a^3*d, c^4, c^2*e^3, c^3*e^2, a*d^5, c*e^5,  $\cdot\cdot\cdot$\\
\{\{c*d, a*e, a*c, a^2*d^2, a^2*b*d, a^3*d, c^3*e^2, c^4*e, c^5, a*d^4,  $\cdot\cdot\cdot$\\
\{\{a*e, c*d, a*c, a^2*d^2, a^3*d, a^2*b*d, c^2*e^3, c^3*e^2, c^4*e, c^5 $\cdot\cdot\cdot$\\
\{\{a*e, a*c, c*d, a^2*b*d, a^3*d, a^2*d^2, c^2*e^3, c^3*e^2, c^4*e, c^5 $\cdot\cdot\cdot$\\
\endOutput

It is an open problem whether the toric Hilbert scheme $Hilb_A$ is
connected. Recent work in geometric combinatorics \cite{HS:San} suggests
that this is probably false for some $A$. This result and its 
implications for $Hilb_A$ will be discussed further in Section 2.
The following theorem of Maclagan and Thomas \cite{HS:MT} reduces the  
connectivity of $Hilb_A$ to a combinatorial problem.

\begin{theorem} 
The toric Hilbert scheme $Hilb_A$ is connected if and only if the 
flip graph of $A$ is connected.
\end{theorem}

We now have two algorithms for listing monomial ideals on $Hilb_A$.
First, there is the {\it \ie{backtracking algorithm}} whose \Mtwo
implementation was described above.  Second, there is the {\it \ie{flip
  search algorithm}}, which starts with any coherent monomial ideal $M$
and then constructs the connected component of $M$ in the flip graph
of $A$ by carrying out local flips as above.  This procedure is also 
implemented in {\tt TiGERS} \cite{HS:HT}. Clearly, the two algorithms
will produce the same answer if and only if $Hilb_A$ is connected. In
other words, finding an example where $Hilb_A$ is disconnected is
equivalent to finding a matrix $A$ for which the flip search algorithm
produces fewer monomial ideals than the backtracking algorithm.

\section{Polyhedral Geometry}

Algorithms from polyhedral geometry are essential in the study of the
toric Hilbert scheme. Consider the problem of deciding whether or not
a given monomial ideal $M$ in $Hilb_A$ is coherent.  This problem
gives rise to a system of linear inequalities as follows: Let
$x^{u_1}, \ldots, x^{u_r}$ be the minimal generators of $M$, and let
$x^{v_i}$ be the unique standard monomial with $A u_i = A v_i$. Then
$M$ is coherent if and only if there exists a vector $w \in \R^n$ such
that $\,w \cdot (u_i - v_i) > 0\,$ for $i =1,\ldots,r$.  Thus the test
for coherence amounts to solving a {\sl feasibility problem of linear
programming}, and there are many highly efficient algorithms (based on
the simplex algorithms or interior point methods) available for this
task. For our experimental purposes, it is convenient to use the code
{\tt polarCone.m2}, given in Appendix \ref{FMe}, which is based on the
(inefficient but easy-to-implement) {\em \ie{Fourier-Motzkin elimination}}
method (see \cite{HS:Zie} for a description).  This code converts the
generator representation of a polyhedron to its inequality
representation and vice versa. A simple example is given in Appendix
\ref{FMe}. In particular, given a Gr\"obner basis $\mathcal G$ of $I_A$, the
function {\tt polarCone} will compute all the extreme rays of the {\em
Gr\"obner cone\index{Grobner cone@Gr\"obner cone}} $\,\{ w \in \R^n \,: \,w \cdot (u_i - v_i) \geq 0\,$
for each $x^{u_i}-x^{v_i} \in {\mathcal G}\}.$

We now show how to use \Mtwo to decide whether a 
monomial $A$-graded ideal $M$ is coherent. The first step in 
this calculation is to compute all the standard monomials of $M$ 
of the same degree as the minimal generators of $M$. We do this 
using the procedure {\tt stdMonomials}.

\beginOutput
i29 : stdMonomials = (M) -> (\\
\           R := ring M;\\
\           RM := R/M;\\
\           apply(numgens M, i -> (\\
\                 s := basis(degree(M_i),RM); lift(s_(0,0), R)))\\
\           ); \\
\endOutput

As an example, consider the following monomial $A$-graded ideal.

\beginOutput
i30 : R = QQ[a..e,Degrees => transpose A ]; \\
\endOutput
\beginOutput
i31 : M = ideal(a^3*d, a^2*b*d, a^2*d^2, a*b^3*d, a*b^2*d^2, a*b*d^3, \\
\                a*c, a*d^4, a*e, b^5*d, b^4*d^2, b^3*d^3, b^2*d^4, \\
\                b*d^5, b*e, c*e^5); \\
\emptyLine
o31 : Ideal of R\\
\endOutput
\beginOutput
i32 : toString stdMonomials M \\
\emptyLine
o32 = \{b*c^3, c^4, c^3*e, c^5, c^4*e, c^3*e^2, b^2, c^2*e^3, b*d, c^6, $\cdot\cdot\cdot$\\
\endOutput

From the pairs $x^u,x^v$ of minimal generators $x^u$ and
the corresponding standard monomials $x^v$, the function {\tt inequalities}
creates a matrix whose columns are the vectors $u-v$. 

\beginOutput
i33 : inequalities = (M) -> (\\
\              stds := stdMonomials(M);\\
\              transpose matrix apply(numgens M, i -> (\\
\                  flatten exponents(M_i) - \\
\                      flatten exponents(stds_i)))); \\
\endOutput
\beginOutput
i34 : inequalities M\\
\emptyLine
o34 = | 3  2  2  1  1  1  1  1  1  0  0  0  0  0  0  0  |\\
\      | -1 1  0  3  2  1  -2 0  -1 5  4  3  2  1  1  0  |\\
\      | -3 -4 -3 -5 -4 -3 1  -2 0  -6 -5 -4 -3 -2 -1 1  |\\
\      | 1  1  2  1  2  3  0  4  -1 1  2  3  4  5  -1 -6 |\\
\      | 0  0  -1 0  -1 -2 0  -3 1  0  -1 -2 -3 -4 1  5  |\\
\emptyLine
\               5        16\\
o34 : Matrix ZZ  <--- ZZ\\
\endOutput

It is convenient to simplify the output of the next procedure 
using the following program to divide an integer vector 
by the g.c.d. of its components. We also load {\tt polarCone.m2},
which is needed in {\tt decideCoherence} below.

\beginOutput
i35 : primitive := (L) -> (\\
\           n := #L-1; g := L#n;\\
\           while n > 0 do (n = n-1; g = gcd(g, L#n););\\
\           if g === 1 then L else apply(L, i -> i // g));\\
\endOutput

\beginOutput
i36 : load "polarCone.m2" \\
\endOutput

\beginOutput
i37 : decideCoherence = (M) -> (\\
\           ineqs := inequalities M;\\
\           c := first polarCone ineqs;\\
\           m := - sum(numgens source c, i -> c_\{i\});\\
\           prods := (transpose m) * ineqs;\\
\           if numgens source prods != numgens source compress prods\\
\           then false else primitive (first entries transpose m)); \\
\endOutput
 
Let $K$ be the cone $\{x \in {\mathbb R}^n : g \cdot x \leq 0$,
for all columns $g$ of {\tt ineqs} \}. The command {\tt
polarCone ineqs} computes a pair of matrices $P$ and $Q$ such
that $K$ is the sum of the cone generated by the columns of $P$
and the subspace generated by the columns of $Q$. Let {\tt m} be
the negative of the sum of the columns of $P$. Then {\tt m} lies
in the cone $-K$. The entries in the matrix {\tt prods} are the
dot products $g \cdot m$ for each column $g$ of {\tt ineqs}.
Since $M$ is a monomial $A$-graded ideal, it is coherent if and
only if $K$ is full dimensional, which is the case if and only if
no dot product $g \cdot m$ is zero. This is the conditional in
the {\tt if .. then} statement of {\tt decideCoherence}. If $M$
is coherent, the program outputs the primitive representative of
{\tt m} and otherwise returns the boolean {\tt false}. Notice that 
if $M$ is coherent, the cone $-K$ is the Gr\"obner cone corresponding 
to $M$ and the vector {\tt m} is a weight vector $w$ such that
$in_w(I_A) = M$. We now test whether the ideal $M$ from 
line {\tt i29} is coherent.

\beginOutput
i38 : decideCoherence M\\
\emptyLine
o38 = \{0, 0, 1, 15, 18\}\\
\emptyLine
o38 : List\\
\endOutput

Hence, $M$ is coherent: it is the initial ideal with respect to the 
weight vector $w = (0,0,1,15,18)$ of the toric ideal in our running
example (\ref{OurMatrix}). Here is one of the 55 noncoherent
monomial $A$-graded ideals of this matrix.

\beginOutput
i39 : N = ideal(a*e,c*d,a*c,c^3*e,a^3*d,c^4,a*d^4,a^2*d^3,c*e^5,\\
\                 c^2*e^4,d^7);\\
\emptyLine
o39 : Ideal of R\\
\endOutput
\beginOutput
i40 : decideCoherence N\\
\emptyLine
o40 = false\\
\endOutput

In the rest of this section, we study the connection between
$A$-graded ideals and polyhedral complexes defined on $A$.  This study
relates the flip graph of the toric Hilbert scheme to the Baues
graph of the configuration $A$.                  (See \cite{HS:Reiner} for a
survey of the Baues problem and its relatives).  Let $pos(A) := \{ Au
: u \in \R^n, u \geq 0 \}$ be the cone generated by the columns of $A$
in $\R^d$. A {\em \ie{polyhedral subdivision}} $\Delta$ of $A$ is a
collection of full dimensional subcones $pos(A_{\sigma})$ of $pos(A)$
such that the union of these subcones is $pos(A)$ and the intersection
of any two subcones is a face of each.  Here $A_{\sigma} := \{a_j : j
\in \sigma \subseteq \{1,\ldots,n\} \}$.  It is customary to identify 
$\Delta$ with the set of sets $\{ \sigma : pos(A_{\sigma}) \in \Delta
\}$. If every cone in the 
subdivision $\Delta$ is simplicial (the number of extreme rays of the
cone equals the dimension of the cone), we say that $\Delta$ is a {\em
  \ie{triangulation}} of $A$. The simplicial complex corresponding
to a triangulation $\Delta$ is uniquely obtained by including in
$\Delta$ all the subsets of every $\sigma \in \Delta$. We refer the
reader to \cite[\S 8]{HS:St2} for more details.

For each $\sigma \in \Delta$, let $I_{\sigma}$ be the prime ideal 
that is the sum of the toric ideal $I_{A_{\sigma}}$ and the monomial 
ideal $\langle x_j :j \not \in \sigma \rangle$. Recall that two
ideals $J$ and $J'$ are said to be 
{\em torus isomorphic} if $J = \lambda \cdot J'$ for some $\lambda \in 
(\k^{\ast})^n$. The following theorem shows that polyhedral
subdivisions of $A$ are related to $A$-graded ideals via their 
radicals.

\begin{theorem}[Theorem~10.10 {\cite[\S 10]{HS:St2}}]\label{polysubdivisions}
  If $I$ is an $A$-graded ideal, then there exists a polyhedral
  subdivision $\Delta(I)$ of $A$ such that $\sqrt{I} = \cap_{\sigma
    \in \Delta(I)} J_{\sigma}$ where each component $J_{\sigma}$ is a
  prime ideal that is torus isomorphic to $I_{\sigma}$.
\end{theorem}

We say that $\Delta(I)$ supports the $A$-graded ideal $I$.
When $M$ is a monomial $A$-graded ideal, $\Delta(M)$ is a 
triangulation of $A$. In particular, if $M$ is coherent (i.e, $M =
in_w(I_A)$ for some weight vector $w$), then $\Delta(M)$ is the {\em
  regular} or {\em coherent} triangulation\index{triangulation!regular} of $A$ induced by $w$
\cite[\S 8]{HS:St2}. The coherent triangulations of $A$ are in bijection
with the vertices of the {\em \ie{secondary polytope}} of $A$ \cite{HS:BFS},
\cite{HS:GKZ}.  

It is convenient to represent a triangulation $\Delta$ of $A$ by its 
{\em Stanley-Reisner} ideal\index{Stanley-Reisner ideal} $I_{\Delta} := \langle x_{i_1}x_{i_2}
\cdots x_{i_k} : \{ i_1, i_2, \ldots, i_k \}$ is a non-face of  
$\Delta \rangle$. If $M$ is a monomial $A$-graded ideal,
Theorem~\ref{polysubdivisions} implies that $I_{\Delta(M)}$ is the 
radical of $M$. Hence we will represent triangulations 
of $A$ by their Stanley-Reisner ideals. As seen below, the matrix in
our running example has eight distinct triangulations 
corresponding to the eight distinct radicals of the 281 monomial 
$A$-graded ideals computed earlier. All eight are coherent.

\medskip

\begin{tabular}{lll}
{$\{\{1,2\},\{2,3\},\{3,4\},\{4,5\}\}$}
&\,\,\,\,$\leftrightarrow$\,\,\,\,& $\langle ac, ad, ae, bd, be, ce
\rangle$ \\  
{$\{\{1,3\},\{3,4\},\{4,5\}\}$} &\,\,\,\,$\leftrightarrow$\,\,\,\,&
$\langle b, ad, ae, ce \rangle$ \\  
{$\{\{1,2\},\{2,4\},\{4,5\}\}$} &\,\,\,\,$\leftrightarrow$\,\,\,\,&
$\langle c, ad, ae, be \rangle$ \\ 
{$\{\{1,2\},\{2,3\},\{3,5\}\}$} &\,\,\,\,$\leftrightarrow$\,\,\,\,&
$\langle d, ac, ae, be \rangle$ \\  
{$\{\{1,3\},\{3,5\}\}$} &\,\,\,\,$\leftrightarrow$\,\,\,\,& $\langle
b, d, ae \rangle$ \\ 
{$\{\{1,4\},\{4,5\}\}$} &\,\,\,\,$\leftrightarrow$\,\,\,\,& $\langle
b, c, ae \rangle$ \\  
{$\{\{1,2\},\{2,5\}\}$} &\,\,\,\,$\leftrightarrow$\,\,\,\,& $\langle
c, d, ae \rangle$ \\ 
{$\{\{1,5\}\}$} &\,\,\,\,$\leftrightarrow$\,\,\,\,& $\langle b, c, d
\rangle$   
\end{tabular}

\medskip

The Baues graph of $A$ is a graph on all the triangulations of
$A$ in which two triangulations are adjacent if they differ by a
single {\em \ie{bistellar flip}} \cite{HS:Reiner}. The {\em \ie{Baues problem}} from
discrete geometry asked whether the Baues graph of a point
configuration can be disconnected for some $A$. Every edge of the
secondary polytope of $A$ corresponds to a bistellar flip, and hence
the subgraph of the Baues graph that is induced by the coherent
triangulations of $A$ is indeed connected: it is precisely the edge
graph of the secondary polytope of $A$.  The Baues problem was
recently settled by Santos \cite{HS:San} who gave an example of a six
dimensional point configuration with $324$ points for which there is
an isolated (necessarily non-regular) triangulation.

Santos' configuration would also have a disconnected flip graph and hence
a disconnected toric Hilbert scheme if it were true that {\em every} 
triangulation of $A$ supports a monomial $A$-graded
ideal. However, Peeva has shown that this need not be the case
(Theorem~10.13 in \cite[\S 10]{HS:St2}). Hence, the map from the set of
all monomial $A$-graded ideals to the set of all triangulations of
$A$ that sends $M \mapsto \Delta(M)$ is not always
surjective, and it is unknown whether Santos' $6 \times 324$ 
configuration has a disconnected toric Hilbert scheme.  

Thus, even though one cannot in general conclude that the existence of
a disconnected Baues graph implies the existence of a disconnected
flip graph, there is an important special situation in which such a
conclusion is possible. We call an integer matrix $A$ of full row rank
{\em unimodular}\index{matrix!unimodular} if the absolute value of each of its non-zero maximal
minors is the same constant. A matrix $A$ is unimodular if and only if
every monomial $A$-graded ideal is square-free. For a unimodular
matrix $A$, the Baues graph of $A$ coincides with the flip graph of
$A$. As you might expect, Santos' configuration is not unimodular.

\begin{theorem}[Lemma~10.14 {\cite[\S 10]{HS:St2}}]\label{unimodular}
If $A$ is unimodular, then each triangulation of $A$ supports a unique
(square-free) monomial $A$-graded ideal. In this case, a monomial
$A$-graded ideal is coherent if and only if the triangulation
supporting it is coherent.
\end{theorem}

Using Theorem~\ref{unimodular} we can compute all the triangulations
of a unimodular matrix since they are precisely the polyhedral
complexes supporting monomial $A$-graded ideals. Then we could
enumerate the connected component of a coherent monomial $A$-graded
ideal in the flip graph of $A$ to decide whether the Baues/flip graph
is disconnected.

Let $\Delta_r$ be the standard $r$-simplex that 
is the convex hull of the $r+1$ unit vectors in $\R^{r+1}$, and let 
$A(r,s)$ be the $(r+s+2) \times (r+1)(s+1)$ matrix whose columns 
are the products of the vertices of $\Delta_r$ and $\Delta_s$. All 
matrices of type $A(r,s)$ are unimodular. From the
product of two triangles we get $$A(2,2) := 
\left ( \begin{array}{ccccccccc}
1&1&1&0&0&0&0&0&0\\
0&0&0&1&1&1&0&0&0\\
0&0&0&0&0&0&1&1&1\\
1&0&0&1&0&0&1&0&0\\
0&1&0&0&1&0&0&1&0\\
0&0&1&0&0&1&0&0&1 \end{array} \right ).$$
We can now use our algebraic algorithms to compute all
the triangulations of $A(2,2)$. Since \Mtwo requires the first entry 
of the degree of every variable in a ring to be positive, we use 
the following matrix with the same row space as $A(2,2)$ for our 
computation:

\beginOutput
i41 : A22 =\\
\        \{\{1,1,1,1,1,1,1,1,1\},\{0,0,0,1,1,1,0,0,0\},\{0,0,0,0,0,0,1,1,1\},\\
\        \{1,0,0,1,0,0,1,0,0\},\{0,1,0,0,1,0,0,1,0\},\{0,0,1,0,0,1,0,0,1\}\}; \\
\endOutput
\beginOutput
i42 : I22 = toricIdeal A22\\
\emptyLine
o42 = ideal (f*h - e*i, c*h - b*i, f*g - d*i, e*g - d*h, c*g - a*i, b* $\cdot\cdot\cdot$\\
\emptyLine
o42 : Ideal of R\\
\endOutput
The ideal {\tt I22} is generated by the 2 by 2 minors of a 3 by 3
matrix of indeterminates.  This is the ideal of $\P^2 \times \P^2$
embedded in $\P^8$ via the Segre embedding.
\beginOutput
i43 : Graver22 = graver I22;\\
\emptyLine
\              1       15\\
o43 : Matrix R  <--- R\\
\endOutput
\beginOutput
i44 : generateAmonos(Graver22);\\
\endOutput
\beginOutput
i45 : #monos\\
\emptyLine
o45 = 108\\
\endOutput
\beginOutput
i46 : scan(0..9,i->print toString monos#i) \\
\{f*h, c*h, f*g, e*g, c*g, b*g, c*e, c*d, b*d\}\\
\{f*h, d*h, c*h, f*g, c*g, b*g, c*e, c*d, b*d\}\\
\{d*i, f*h, d*h, c*h, c*g, b*g, c*e, c*d, b*d\}\\
\{e*i, c*h, f*g, e*g, c*g, b*g, c*e, c*d, b*d\}\\
\{e*i, d*i, c*h, e*g, c*g, b*g, c*e, c*d, b*d\}\\
\{e*i, d*i, d*h, c*h, c*g, b*g, c*e, c*d, b*d\}\\
\{f*h, c*h, f*g, e*g, c*g, b*g, c*e, a*e, c*d\}\\
\{e*i, c*h, f*g, e*g, c*g, b*g, c*e, a*e, c*d, b*d*i\}\\
\{e*i, c*h, f*g, e*g, c*g, b*g, c*e, a*e, c*d, a*f*h\}\\
\{e*i, d*i, c*h, e*g, c*g, b*g, c*e, a*e, c*d\}\\
\endOutput

Thus there are 108 monomial $A(2,2)$-graded ideals and 
{\tt decideCoherence} will check that all of them 
are coherent. Since $A(2,2)$ is unimodular, each monomial 
$A(2,2)$-graded ideal is square-free and is hence 
radical. These 108 ideals represent the 108 triangulations of 
$A(2,2)$ and we have listed ten of them above.
The flip graph (equivalently, Baues graph) of $A(2,2)$ is connected.
However, it is unknown whether the Baues graph of $A(r,s)$ is 
connected for all values of $(r,s)$.

\section{Local Equations}
Consider the reduced Gr\"obner basis of a toric ideal $I_A$ for a
term order $w$:
\begin{equation}
\label{GrobnerBasis} \bigl\{ \,
 x^{u_1} -  x^{v_1} \, , \,\, x^{u_2} -  x^{v_2} \,, \,\, \ldots \, , \,\,
x^{u_r} -  x^{v_r}\, \bigr\} .
\end{equation}
The initial ideal $\,M = in_w(I_A) = \langle x^{u_1}, x^{u_2}, \ldots,
x^{u_r} \rangle \,$ is a coherent monomial $A$-graded ideal. In
particular, it is a $(\k^*)^n$-fixed point on the toric Hilbert scheme
$Hilb_A$.  We shall explain a method, due to Peeva and Stillman
\cite{HS:PS2}, for computing local equations of $Hilb_A$ around such a
fixed point.  A variant of this method also works for computing the
local equations around a non-coherent monomial ideal $M$, but that
variant involves local algebra, specifically Mora's tangent cone
algorithm, which is not yet fully implemented in \Mtwo. See \cite{HS:PS2}
for details.

We saw how to compute the flip graph of $A$ in Section~1. The vertices
of this graph are the $(\k^*)^n$-fixed points $M$ and its edges
correspond to the $(\k^*)^n$-fixed curves.  By computing and
decomposing the local equations around each $M$, we get a complete
description of the scheme $Hilb_A$.

The first step is to introduce a new variable $ \, z_i \,$ for each 
binomial in our Gr\"obner basis (\ref{GrobnerBasis}) and to consider 
the following $r$ binomials:
\begin{equation}
\label{FlatFamily}
 x^{u_1} -  z_1 \cdot x^{v_1} \,, \,\,
 x^{u_2} - z_2  \cdot x^{v_2} \,,\, \, \ldots \, ,
\,\, x^{u_r} -   z_r \cdot x^{v_r} 
\end{equation}
in the polynomial ring $\k[x,z]$ in  $n+r$ indeterminates.
The term order $w$ can be extended to an elimination term order
in $\k[x,z]$ so that $x^{u_i}$ is the leading term of
$ x^{u_i} -  z_i \cdot x^{v_i} $ for all $i$. 
We compute the minimal first syzygies
of the monomial ideal $M$, and form the
corresponding $S$-pairs of binomials in (\ref{FlatFamily}).
For each $S$-pair
$$
\frac{lcm(x^{u_i},x^{u_j})}{x^{u_i}} \cdot (x^{u_i} - z_i \cdot
x^{v_i} ) \,\,\, - \,\,\, \frac{lcm(x^{u_i},x^{u_j})}{x^{u_j}} \cdot
(x^{u_j} - z_j \cdot x^{v_j}) $$
we compute a normal form with respect
to (\ref{FlatFamily}) using the extended term order $w$.  The result
is a binomial in $\k[x,z]$ that factors as 
$$  x^\alpha \cdot z^\beta \cdot  ( z^\gamma - z^\delta ) , $$
where $\alpha \in \N^n$ and $\beta,\gamma,\delta \in \N^r$.
Note that this normal form is not unique but depends on our
choice of a reduction path.
Let $J_M$ denote the ideal in $\k[z_1 , \ldots, z_r]$ generated by all 
binomials $\, z^\beta \cdot  ( z^\gamma - z^\delta ) \,$
gotten from normal forms of all the $S$-pairs considered above.

\begin{proposition}[\cite{HS:PS2}]\label{localeqns}
The ideal $J_M$ is independent of the reduction paths chosen.
It defines a subscheme of $\k^r$ isomorphic to
an affine open neighborhood of the point $M$ on 
the toric Hilbert scheme $Hilb_A$.
\end{proposition}

We apply this technique to compute a particularly interesting affine
chart of $Hilb_A$ for our running example.
Consider the following set of $13$ binomials:
\begin{eqnarray*}
& \bigl\{ \,a e - z_1 b d ,  \,
 c d - z_2 b e , \,
 a c - z_3 b^2 , \,
 a^2 d^2 - z_4 c^3 e , \,
 a^2 b d - z_5 c^4 , \\ &
 a^3 d - z_6 b c^3 , \,
 c^2 e^3 - z_7 a d^4 , \, 
 c^3 e^2 - z_8 a b d^3 , \,
 c^4 e - z_9 a b^2 d^2 , \\ &
 c^5 - z_{10} a b^3 d , \,
 c e^5 - z_{11} d^6 , \,
 a d^5 - z_{12} b c e^4 , \,
 b e^6 - z_{13} d^7  \, \bigr\}.
\end{eqnarray*}
If we set $\, z_1 = z_2 = \cdots = z_{13} = 1\,$
then we get a generating set for the toric ideal $I_A$.
The $13$ monomials obtained by setting
$\, z_1 = z_2 = \cdots = z_{13} = 0 \,$
generate the initial monomial ideal $ M = in_w (I_A)$
with respect to the weight vector $w = (9, 3, 5, 0, 0)$.
Thus $M$ is one of the $226$ coherent monomial 
$A$-graded ideals of our running example. The above set of 
13 binomials in $\k[x,z]$ give the universal family 
for $Hilb_A$ around this $M$.

The local chart of $Hilb_A$ around the point $M$
is a subscheme of affine space $\k^{13}$ with coordinates 
$z_1, \ldots, z_{13}$, whose
defining equations are obtained as follows: 
Extend the weight vector $w$ by assigning
weight zero to all variables $z_i$, so that
the first term in each of the above $13$ binomials
is the leading term. For each pair of binomials corresponding to a  
minimal syzygy of $M$, form their $S$-pair and then reduce it to a 
normal form with respect to the $13$ binomials above.
For instance,
$$
S \bigl(
 c^5 - z_{10} a b^3 d , 
 c e^5 - z_{11} d^6 \bigr)
\, = \,
 z_{11} c^4 d^6  - z_{10} a b^3 d e^5
\, \longrightarrow \,
b^4 d^2 e^4 \cdot (z_2^4 z_{11} - z_1 z_{10}).
$$
Each such normal form is a monomial in $a,b,c,d,e$ times a binomial in
$z_1, \ldots, z_{13}$.  The set of all these binomials, in the
$z$-variables, generates the ideal $J_M$ of local equations of
$Hilb_A$ around $M$.  In our example, $J_M$ is generated by $27$
nonzero binomials.  This computation can be done in \Mtwo using the
procedure {\tt localCoherentEquations}.

\beginOutput
i47 : localCoherentEquations = (IA) -> (\\
\           -- IA is the toric ideal of A living in a ring equipped\\
\           -- with weight order w, if we are computing the local \\
\           -- equations about the initial ideal of IA w.r.t. w.\\
\           R := ring IA;\\
\           w := (monoid R).Options.Weights;\\
\           M := ideal leadTerm IA;\\
\           S := first entries ((gens M) {\char`\%} IA);\\
\           -- Make the universal family J in a new ring.\\
\           nv := numgens R; n := numgens M;\\
\           T = (coefficientRing R)[generators R, z_1 .. z_n, \\
\                                   Weights => flatten splice\{w, n:0\},\\
\                                   MonomialSize=>16];\\
\           M = substitute(generators M,T);\\
\           S = apply(S, s -> substitute(s,T));\\
\           J = ideal apply(n, i -> \\
\                     M_(0,i) - T_(nv + i) * S_i);\\
\           -- Find the ideal Ihilb of local equations about M:\\
\           spairs := (gens J) * (syz M);\\
\           g := forceGB gens J;\\
\           B = (coefficientRing R)[z_1 .. z_n,MonomialSize=>16];\\
\           Fones := map(B,T, matrix(B,\{splice \{nv:1\}\}) | vars B);\\
\           Ihilb := ideal Fones (spairs {\char`\%} g);\\
\           Ihilb\\
\           );\\
\endOutput
 
     
Suppose we wish to calculate the local equations about $M =
in_w(I_A)$.  The input to {\tt localCoherentEquations} is the
toric ideal $I_A$ living in a polynomial ring equipped with the 
weight order specified by $w$. This is done as follows:

\beginOutput
i48 : IA = toricIdeal A;\\
\emptyLine
o48 : Ideal of R\\
\endOutput
\beginOutput
i49 : Y = QQ[a..e, MonomialSize => 16,\\
\                  Degrees => transpose A, Weights => \{9,3,5,0,0\}];\\
\endOutput
\beginOutput
i50 : IA = substitute(IA,Y);\\
\emptyLine
o50 : Ideal of Y\\
\endOutput

The initial ideal $M$ is calculated in the third line of the
algorithm, and {\tt S} stores the standard monomials of $M$ of the
same degrees as the minimal generators of $M$. We could have
calculated {\tt S} using our old procedure {\tt stdMonomials} but this
involves computing the monomials in $R_b$ for various values of $b$,
which can be slow on large examples. As by-products, {\tt
  localCoherentEquations} also gets {\tt J}, the ideal of the
universal family for $Hilb_A$ about $M$, the ring {\tt T} of this
ideal, and the ring {\tt B} of {\tt Ihilb}, which is the ideal of the
affine patch of $Hilb_A$ about $M$. The matrix {\tt spairs} contains
all the $S$-pairs between generators of {\tt J} corresponding to the
minimal first syzygies of $M$. The command {\tt forceGB} is used to
declare the generators of {\tt J} to be a Gr\"obner basis, and {\tt
  Fones} is the ring map from {\tt T} to {\tt B} that sends each of
$a,b,c,d,e$ to one and the $z$ variables to themselves.  The columns
of the matrix {\tt (spairs \% g)} are the normal forms of the
polynomials in {\tt spairs} with respect to the forced Gr\"obner basis
{\tt g} and the ideal {\tt Ihilb} of local equations is generated by
the image of these normal forms in the ring {\tt B} under the map {\tt
  Fones}.

\beginOutput
i51 : JM = localCoherentEquations(IA)\\
\emptyLine
\                                                                       $\cdot\cdot\cdot$\\
o51 = ideal (z z  - z , z z  - z , - z z  + z , - z z  + z , - z z  +  $\cdot\cdot\cdot$\\
\              1 2    3   1 2    3     4 7    2     5 8    2     1 5    $\cdot\cdot\cdot$\\
\emptyLine
o51 : Ideal of B\\
\endOutput

Removing duplications among the generators:

\smallskip
$J_M = \langle
z_1-z_{10}z_{11},
z_2-z_4z_7,
z_2-z_5z_8,
z_2-z_{11}z_{12},
z_2-z_1z_{11}z_{13},\\
z_3-z_1z_2,
z_3-z_5z_9,
z_4-z_1z_5,
z_6-z_3z_5,
z_6-z_1z_2z_5,
z_7-z_1z_{10},
z_8-z_1z_7,\\
z_9-z_1z_8,
z_{12}-z_1z_{13},
z_1z_2-z_5z_9,
z_1z_2-z_1z_5z_8,
z_1z_2-z_1^2z_4z_{10},
z_1z_2-z_1^2z_5z_7,\\
z_1z_2-z_1z_{11}z_{12},
z_1z_2-z_2z_{10}z_{11},
z_1^3z_4-z_3z_{11},
z_1z_5z_8-z_4z_8,
z_2z_{10}-z_1z_{12},\\
z_3z_4-z_1z_6,
z_3z_7-z_2z_8,
z_3z_8-z_2z_9,
z_3z_{10}-z_2z_7
\rangle$.
\smallskip

Notice that there are many generators of $J_M$ that have a single
variable as one of its terms. Using these generators we can remove
variables from other binomials. This is done in \Mtwo using the
subroutine {\tt removeRedundantVariables}, which is the main ingredient
of the package {\tt minPres.m2} for computing the minimal
presentations of polynomial quotient rings. Both {\tt
  removeRedundantVariables} and {\tt minPres.m2} are explained in
Appendix \ref{Mpor}. The command {\tt removeRedundantVariables} applied to an
ideal in a polynomial ring (not quotient ring) creates a ring map from
the ring to itself that sends the redundant variables to polynomials 
in the non-redundant variables and the non-redundant variables to 
themselves. Applying this to our ideal $J_M$ we obtain the following 
simplifications.

\beginOutput
i52 : load "minPres.m2";\\
\endOutput
\beginOutput
i53 : G = removeRedundantVariables JM\\
\emptyLine
\                          3  2      4  3                  2 4  3    2  $\cdot\cdot\cdot$\\
o53 = map(B,B,\{z  z  , z z  z  , z z  z  , z z  z  , z , z z  z  , z   $\cdot\cdot\cdot$\\
\                10 11   5 10 11   5 10 11   5 10 11   5   5 10 11   10 $\cdot\cdot\cdot$\\
\emptyLine
o53 : RingMap B <--- B\\
\endOutput
\beginOutput
i54 : ideal gens gb(G JM)\\
\emptyLine
\               3  2        2\\
o54 = ideal(z z  z   - z  z  z  )\\
\             5 10 11    10 11 13\\
\emptyLine
o54 : Ideal of B\\
\endOutput

Thus our affine patch of $Hilb_A$ has the coordinate ring 
$$\k[z_1,z_2,\ldots,z_{13}]/J_M \,\, \simeq \,\,
\frac{\k[z_5,z_{10},z_{11},z_{13}]}{ \langle z_5 z_{{10}}^3 z_{11}^2 -
  z_{10}z_{11}^2 z_{13} \rangle} = \frac{\k[z_5,z_{10},z_{11},z_{13}]}
{\langle (z_5 z_{10}^2 -z_{13}) z_{10}z_{11}^2 \rangle}.$$
Hence, we see immediately that there are three
components through the point $M$ on $Hilb_A$. The restriction of the
coherent component to the affine neighborhood of $M$ on $Hilb_A$ is
defined by the ideal quotient $\, (J_M : (z_1 z_2 \cdots
z_{13})^\infty) $ and hence the first of the above components 
is an affine patch of the coherent component. Locally near $M$ it is  
given by the single equation $z_5 z_{10}^2 - z_{13} = 0$ in $\A^4$. 
It is smooth and, as expected, has dimension three. The second
component, $z_{10} = 0$, is also of dimension three and is smooth at $M$.
The third component, given by $z_{11}^2 = 0$ is more interesting.  It
has dimension three as well, but is not reduced.  Thus we have proved
the following result.

\begin{proposition}
The toric Hilbert scheme $Hilb_A$ of the matrix 
$$A = \left( \begin{matrix}
           1 & 1 & 1 & 1 & 1  \\ 
           0 & 1 & 2 & 7 & 8 
\end{matrix} \right)$$
is not reduced.
\end{proposition}

We can use the ring map {\tt G} from above to simplify {\tt J} so as
to involve only the four variables $z_5, z_{10},z_{11}$ and $z_{13}$.

\beginOutput
i55 : CX = QQ[a..e, z_5,z_10,z_11,z_13, Weights =>\\
\            \{9,3,5,0,0,0,0,0,0\}];\\
\endOutput
 
\beginOutput
i56 : F = map(CX, ring J, matrix\{\{a,b,c,d,e\}\} | \\
\                  substitute(G.matrix,CX))\\
\emptyLine
\                                          3  2      4  3               $\cdot\cdot\cdot$\\
o56 = map(CX,T,\{a, b, c, d, e, z  z  , z z  z  , z z  z  , z z  z  , z $\cdot\cdot\cdot$\\
\                                10 11   5 10 11   5 10 11   5 10 11    $\cdot\cdot\cdot$\\
\emptyLine
o56 : RingMap CX <--- T\\
\endOutput
Applying this map to {\tt J} we get the ideal {\tt J1}, 
\beginOutput
i57 : J1 = F J\\
\emptyLine
\                                            3  2          2   4  3     $\cdot\cdot\cdot$\\
o57 = ideal (c*d - b*e*z  z  , a*e - b*d*z z  z  , a*c - b z z  z  , a $\cdot\cdot\cdot$\\
\                        10 11             5 10 11           5 10 11    $\cdot\cdot\cdot$\\
\emptyLine
o57 : Ideal of CX\\
\endOutput

\noindent and adding the ideal $\langle z_{11}^2 \rangle$ to {\tt J1} 
we obtain the universal family for the non-reduced component of
$Hilb_A$ about $M$. 

\beginOutput
i58 : substitute(ideal(z_11^2),CX) + J1\\
\emptyLine
\              2                                  3  2          2   4   $\cdot\cdot\cdot$\\
o58 = ideal (z  , c*d - b*e*z  z  , a*e - b*d*z z  z  , a*c - b z z  z $\cdot\cdot\cdot$\\
\              11             10 11             5 10 11           5 10  $\cdot\cdot\cdot$\\
\emptyLine
o58 : Ideal of CX\\
\endOutput

In the rest of this section, we present an interpretation of
the ideal $J_M$ in terms of the combinatorial theory
of {\it \ie{integer programming}}. See, for instance, 
\cite[\S 4]{HS:St2} or \cite{HS:Tho} for 
the relevant background. Our reduced Gr\"obner basis 
(\ref{GrobnerBasis}) is the {\it \ie{minimal test set}} for
the family of integer programs
\begin{equation}
\label{IP}
{\rm Minimize} \quad
w \cdot u \,\,\quad
{\rm subject} \,\, {\rm to } \,\,\,
A \cdot u = b   \,\,\, {\rm and}
 \,\,\,u \in \N^n, 
\end{equation}
where $A \in \N^{d \times n}$ and $w
\in \ZZ^n$ are fixed and $b$ ranges over $\N^d$.
If $u' \in \N^n$ is any feasible solution
to (\ref{IP}), then the corresponding optimal solution
$u \in \N^n$ is computed as follows: the monomial
$x^u $ is the unique normal form of $x^{u'}$ modulo
the Gr\"obner basis (\ref{GrobnerBasis}).

Suppose we had reduced $x^{u'}$
modulo the binomials (\ref{FlatFamily}) instead of (\ref{GrobnerBasis}).
Then the output has a $z$-factor that depends on 
our choice of reduction path. To be precise, suppose the
reduction path has length $m$ and at the $j$-th step we had used the
reduction $\, x^{u_{\mu_j}} \rightarrow  z_{\mu_j} \cdot x^{v_{\mu_j}}
$. Then we would obtain the normal form
$$ \, z_{\mu_1} z_{\mu_2} z_{\mu_3} \cdots z_{\mu_m} \cdot x^u.$$
Reduction paths can have different lengths. If we take
another  path  that
has length $m'$ and  uses
$\, x^{u_{\nu_j}} \rightarrow  z_{\nu_j} \cdot x^{v_{\nu_j}} \,$
at the $j$-th step, then the output would be
$$ \, z_{\nu_1} z_{\nu_2} z_{\nu_3} \cdots z_{\nu_{m'}} \cdot x^u  .$$

\begin{theorem} \label{paths}
The ideal $J_M$ of local equations
on $Hilb_A$ is generated by the binomials
$$ \, z_{\mu_1} z_{\mu_2} z_{\mu_3} \cdots z_{\mu_m} - 
z_{\nu_1} z_{\nu_2} z_{\nu_3} \cdots z_{\nu_{m'}} $$ each encoding a
pair of distinct reduction sequences from a feasible solution of an
integer program of the type (\ref{IP}) to the corresponding optimal
solution using the minimal test set in (\ref{GrobnerBasis}).
\end{theorem}

\begin{proof}
The given ideal is contained in $J_M$ because its generators
are differences of monomials arising from the possible
reduction paths of $\,{lcm(x^{u_i},x^{u_j})} $,
for $1 \leq i,j \leq r $. Conversely, any reduction
sequence can be transformed into an equivalent reduction sequence
using S-pair reductions. This follows from standard
arguments in the proof of Buchberger's criterion
\cite[\S 2.6, Theorem 6]{HS:CLO}, and it implies that
the binomials  $ \, z_{\mu_1}  \cdots z_{\mu_m} -
 z_{\nu_1}  \cdots z_{\nu_{m'}}  \,$ are $\k[z]$-linear
combinations of the generators of $J_M$.
\qed
\end{proof}

A given feasible solution of an integer program (\ref{IP})
usually has many different reduction paths to the optimal solution
using  the reduced Gr\"obner basis (\ref{GrobnerBasis}). 
For our matrix \ref{OurMatrix} and cost vector 
$w = (9,3,5,0,0)$, the monomial
$\, a^2 b d e^6 \,$  encodes the feasible solution $(2,1,0,1,6)$
of the integer program 
$$ {\rm Minimize} \quad
w \cdot u \,\,\quad
{\rm subject} \,\, {\rm to } \,\,\,
A \cdot u = \binom{10}{56}   \,\,\, {\rm and}
 \,\,\,u \in \N^5.$$
There are $19$ different paths from this feasible solution 
to the optimal solution $(0,3,0,3,4)$ encoded by the monomial 
$\, b^3 d^3 e^4 $. The generating function for these paths is:
\begin{eqnarray*}
& z_1^2 + 3 z_1 z_2^2 z_5 z_7 + 2 z_1 z_2 z_5 z_7^2 z_{12}
 + 2 z_1 z_2 z_5 z_8 \\ & {} + 2 z_1 z_2 z_{12} z_{13} + z_1 z_5 z_9 
 + z_2^3 z_4 z_5 z_7^2 + z_2^3 z_4 z_{13} + z_2^3 z_5 z_{11} \\ & {} 
 + 2 z_2 z_3 z_5 z_7 + z_3 z_5 z_7^2 z_{12} + z_3 z_5 z_8 + z_3 z_{12} z_{13}.
\end{eqnarray*}
The difference of any two monomials in this generating function 
is a valid local equation for the toric Hilbert scheme of
(\ref{OurMatrix}). For instance, the binomial 
$\,  z_3 z_5 z_7^2 z_{12} - z_3 z_{12} z_{13} \,$ lies in $J_M$,
and, conversely, $J_M$ is generated by binomials obtained in this manner.

The scheme structure of $J_M$ encodes obstructions to making certain
reductions when solving our family of integer programs. For instance,
the variable $z_3$ is a zero-divisor modulo $J_M$. If we factor it
out from the binomial $\,  z_3 z_5 z_7^2 z_{12} - z_3 z_{12} z_{13}
\in J_M \,$, we get 
$\, z_5 z_7^2 z_{12} - z_{12} z_{13} \,$,
which does not lie in $J_M$. Thus there is no monomial
$a^{i_1} b^{i_2} c^{i_3} d^{i_4} e^{i_5} \,$
for which both the paths $ z_5 z_7^2 z_{12} $ and
$ z_{12} z_{13} $ are used to reach the optimum.
It would be a worthwhile combinatorial project to
study the path generating functions and their relation
to the ideal $J_M$ in more detail.

It is instructive to note that the binomials
$ \, z_{\mu_1} z_{\mu_2} \cdots z_{\mu_m} \, - \,
 z_{\nu_1} z_{\nu_2}  \cdots z_{\nu_{m'}}  $
in Theorem~\ref{paths} do not form a vector space basis
for the ideal $J_M$. We demonstrate this for the lexicographic
Gr\"obner basis (with $a \succ b \succ c \succ d \succ e$) of 
the toric ideal defining the rational normal curve of degree $4$. 
In this case, we can take $A = \left ( \begin{array}{ccccc}
1 & 1 & 1 & 1 & 1 \\ 0 & 1 & 2 & 3 & 4 \end{array} \right )$ and the 
universal family in question is :
$$ 
\bigl\{
a c    - z_1 b^2, \,
a d    - z_2 b c,\,
a e    - z_3 c^2,\,
b d    - z_4 c^2,\,
b e    - z_5 c d,\,
c e    - z_6 d^2
\bigr\}.
$$ 
The corresponding ideal of local equations is
$J_M = 
\langle z_3 - z_2  z_5, z_2 - z_1  z_4, z_5  - z_4 z_6 \rangle $,
from which we see that $M$ is a smooth point of $Hilb_A$.
The binomial $\,z_1 z_5 - z_1 z_4 z_6 \,$ lies in $J_M$
but there is no monomial that has the reduction path $z_1 z_5$
or $z_5 z_1 $ to optimality.  Indeed, any monomial
that admits the reductions $z_1 z_5$ or $z_5 z_1$ must be
divisible by either $\, a c e \, $ or $\, a b e  $.
The path generating functions for these two monomials are
$$ abe \quad \rightarrow \quad
(z_3 \, + \, z_1 z_4 z_5 \, +\, z_2 z_5) \cdot b c^2 $$
$$ ace \quad \rightarrow \quad
(z_3 \,+ \,z_1 z_4 z_5 \,+\,
z_2 z_4 z_6) \cdot c^3 . $$
Thus every reduction to optimality using $z_1$ and $z_5$ must
also use $z_4$, and we conclude that $\,z_1 z_5 - z_1 z_4 z_6 \,$
is not in the $\k$-span of the binomials listed in
Theorem~\ref{paths}.

\section{The Coherent Component of the Toric Hilbert Scheme}

In this section we study the component of the toric Hilbert scheme
$Hilb_A$ that contains the point corresponding to the toric ideal
$I_A$. An $A$-graded ideal is
coherent if and only if it is isomorphic to an initial ideal of $I_A$
under the action of the torus $(\k^\ast)^n$. All coherent $A$-graded
ideals lie on the same component of $Hilb_A$ as $I_A$.
We will show that this component need not be
normal, and we will describe how its local and global equations can be
computed using \Mtwo.  Every term order for the toric ideal $I_A$ can
be realized by a weight vector that is an element in the lattice $\,N
= Hom_\ZZ( ker_\ZZ(A) , \ZZ) \, \simeq \, \ZZ^{n-d}$.  Two weight
vectors $w$ and $w'$ in $N$ are considered {\it equivalent}\index{weight vectors!equivalent} if they
define the same initial ideal $\,in_w(I_A) = in_{w'}(I_A)$.  These
equivalence classes are the relatively open cones of a projective fan
$\Sigma_A$ called the {\em Gr\"obner fan\index{Grobner fan@Gr\"obner fan}} of $I_A$  
\cite{HS:MR}, \cite{HS:ST}. This fan lies in 
$\mathbb R^{n-d}$, the real vector space spanned by the lattice $N$.

\begin{theorem}
The toric ideal $I_A$ lies on a unique irreducible component of
the toric Hilbert scheme $Hilb_A$, called the coherent component.
The normalization of the coherent
component is the projective toric variety defined by 
the Gr\"obner fan of $I_A$.
\end{theorem}

\begin{proof}
The {\it \ie{divisor at infinity}} on the toric Hilbert scheme $Hilb_A$ 
consists of all points at which at least one of the local coordinates
(around some monomial $A$-graded ideal) is zero.  This is a proper
closed codimension one subscheme of $Hilb_A$, parametrizing
all those $A$-graded ideals that contain at least one monomial.  
The complement of the divisor at infinity in 
$Hilb_A$ consists of precisely  the orbit of $I_A$ 
under the action of the torus $(\k^*)^n $.
This is the content of \cite[Lemma 10.12]{HS:St2}.

The closure of the $(\k^*)^n $-orbit of $I_A$ is a
reduced and irreducible component of  $Hilb_A$.
It is reduced because $I_A$ is a smooth point on $Hilb_A$,
as can be seen from the local equations, and it is irreducible
since $(\k^*)^n$ is a connected group. It is a component
of $Hilb_A$ because its complement lies in a divisor.
We call this irreducible component the
{\it \ie{coherent component}} of $Hilb_A$.

Identifying $(\k^*)^n$ with $Hom_\ZZ(\ZZ^n, \k^*)$, we note
that the stabilizer of $I_A$ consists of those linear forms
$w$ that restrict to zero on the kernel of $A$. Therefore
the coherent component is the closure in $Hilb_A$
of the  orbit of the point $I_A$ under the action of the torus
$\, N \otimes \k^* \, = \,Hom_\ZZ( ker_\ZZ(A), \k^*)$.
The $(N \otimes \k^*)$-fixed points
on this component are precisely the coherent monomial 
$A$-graded ideals, and the same holds for the
toric variety of the Gr\"obner fan. 

Fix a maximal cone $\sigma$ in the Gr\"obner fan $\Sigma_A$,
and let $M = \langle x^{u_1}, \ldots, x^{u_r} \rangle$ 
be the corresponding (monomial) 
initial ideal of $I_A$.  As before we write
$$ \left \{ x^{u_1} -  z_1 \cdot x^{v_1} \,, \,\,
 x^{u_2} - z_2  \cdot x^{v_2} \,,\, \, \ldots \, ,
\,\, x^{u_r} -   z_r \cdot x^{v_r} \right \}$$
for the universal family arising from the corresponding
reduced Gr\"obner basis of $I_A$.  Let $J_M$ be the
ideal in $\k [z_1,z_2,\ldots,z_r]$ defining this family.

The restriction of the coherent component to the 
affine neighborhood of $M$ on $Hilb_A$ is defined
by $\, J_M :  (z_1 z_2 \cdots z_r)^\infty $.
It then follows from our combinatorial description of
the ideal $J_M$ that this ideal quotient is a binomial prime ideal.
In fact, it is the ideal of algebraic relations among the 
Laurent monomials $\, x^{u_1- v_1}, \ldots, x^{u_r-v_r}$.
We conclude that the restriction of the coherent component to the
affine neighborhood of $M$ on $Hilb_A$ equals
\begin{equation}
\label{uv-algebra}
 {\rm Spec} \,\, \k \bigl[
 x^{u_1-v_1},
 x^{u_2-v_2},  \ldots,
 x^{u_r-v_r} 
\bigr] .
\end{equation}

The abelian group generated by the vectors
$\, u_1-v_1, \ldots, u_r-v_r \,$  equals
$\ker_\ZZ(A) = Hom_\ZZ(N,\ZZ)$. This follows from 
\cite[Lemma 12.2]{HS:St1} because the
binomials $x^{u_i} - x^{v_i}$ generate the toric ideal $I_A$.
The cone generated by the vectors $\, u_1-v_1, \ldots, u_r-v_r \,$  is
precisely the polar dual $\sigma^\vee$ to
the Gr\"obner cone $\sigma$. This follows from
 equation (2.6) in \cite{HS:St1}. We conclude that the
normalization of the affine variety
(\ref{uv-algebra}) is the normal affine toric variety
\begin{equation}
\label{normal-uv-algebra}
 {\rm Spec} \,\, \k \bigl[
 \ker_\ZZ(A) \,\cap\, \sigma^\vee \bigr] .
\end{equation}

The normalization morphism from (\ref{normal-uv-algebra}) to
(\ref{uv-algebra}) maps the identity point in the  toric variety
 (\ref{normal-uv-algebra}) 
to the point $I_A$ in the affine chart 
(\ref{uv-algebra}) of the toric Hilbert 
scheme $Hilb_A$. 
Clearly, this normalization morphism is equivariant with respect to the
action by the torus $\, N \otimes \k^* $.
These two properties  hold for every maximal cone $\sigma$ of the 
Gr\"obner fan $\Sigma_A$. Hence there exists a unique 
$\, N \otimes \k^* $-equivariant morphism $\phi$
from the projective toric variety associated with $\Sigma_A$
onto the coherent component of $Hilb_A$, such that $\phi$ maps 
the identity point to the point $I_A$ on $Hilb_A$, and 
$\phi$ restricts to the normalization
morphism (\ref{normal-uv-algebra}) $\rightarrow$ (\ref{uv-algebra}) on each
affine open chart. We conclude that $\phi$
is the desired normalization map from the
projective toric variety associated with the Gr\"obner fan of $I_A$ 
onto the coherent component of the toric Hilbert scheme $Hilb_A$.
\qed
\end{proof}

We now present an example that shows that the coherent component 
of $Hilb_A$ need not be normal. This example is 
derived from the matrix that appears in Example 3.15 of \cite{HS:HM}.
This example is also mentioned in \cite{HS:PS1} without details.
Let $d=4$ and $n=7$ and fix the matrix

\begin{equation}
\label{non-normal}
A = \left( \begin{array}{ccccccc}  
1 & 1 & 1 & 1 & 1 & 1 & 1 \\
0 & 6 & 7 & 5 & 8 & 4 & 3 \\
3 & 7 & 2 & 0 & 7 & 6 & 1 \\
6 & 5 & 2 & 6 & 5 & 0 & 0 \end{array} \right).
\end{equation}

The lattice $\,N =  Hom_\ZZ ( ker_\ZZ(A), \ZZ)$
is three-dimensional. The toric ideal $I_A$ is minimally 
generated by $30$ binomials of total degree between $6$ and $93$.

\beginOutput
i59 : A = \{\{1,1,1,1,1,1,1\},\{0,6,7,5,8,4,3\},\{3,7,2,0,7,6,1\},\\
\         \{6,5,2,6,5,0,0\}\};\\
\endOutput
\beginOutput
i60 : IA = toricIdeal A\\
\emptyLine
\              2 3       3 2   2     4 4    8 4   4 3 6    7 2 4     4  $\cdot\cdot\cdot$\\
o60 = ideal (a c e - b*d f , a c*d*e f  - b g , d e f  - b c g , a*b c $\cdot\cdot\cdot$\\
\emptyLine
o60 : Ideal of R\\
\endOutput

We fix the weight vector $w = (0,0,276,220,0,0,215)$ in $N$ and 
compute the initial ideal $M = in_w(I_A)$. This initial ideal 
has $44$ minimal generators.

\beginOutput
i61 : Y = QQ[a..g, MonomialSize => 16,\\
\                 Weights => \{0,0,276,220,0,0,215\},\\
\                 Degrees =>transpose A];\\
\endOutput
\beginOutput
i62 : IA = substitute(IA,Y);\\
\emptyLine
o62 : Ideal of Y\\
\endOutput
\beginOutput
i63 : M = ideal leadTerm IA\\
\emptyLine
\              2 3    8 4   7 2 4     4 7 3   5 4 3 5   2 6 5 4   3 3 1 $\cdot\cdot\cdot$\\
o63 = ideal (a c e, b g , b c g , a*b c f , b c d f , a b c g , a b c  $\cdot\cdot\cdot$\\
\emptyLine
o63 : Ideal of Y\\
\endOutput

\begin{proposition} The three dimensional affine variety
  (\ref{uv-algebra}), for the initial ideal $M$ with respect to $w =
  (0,0,276,220,0,0,215)$ of the toric ideal of $A$ in
  (\ref{non-normal}), is not normal.
\end{proposition} 

\begin{proof}
The universal family for the toric Hilbert scheme $Hilb_A$ at $M$ is:
\begin{eqnarray*}
\{& a^2e^{15}g^{18}-z_1b^3c^6d^{10}f^{16}, \,\,
b^{13}d^{15}f^{16}-z_2a^8ce^{21}g^{14}, \\ &
c^{59} d^{57} f^{110} - z_3  e^{92} g^{134},
a c^{14} d^{11} f^{23} - z_4  b e^{19} g^{29}, \\ &
b^7 c^2 g^4 - z_5  d^4 e^3 f^6, \,\,
\ldots, \,\,
b c^{34} d^{32} f^{62} - z_{44}  e^{53} g^{76} \}.
\end{eqnarray*}
The semigroup algebra in (\ref{uv-algebra})
is generated by $44$ Laurent  monomials 
gotten from this family. It turns out that the
first four monomials suffice to generate the semigroup.
In other words, for all $j \in \{5,6,\ldots,44\}$
there exist
$ i_1,i_2,i_3, i_4 \in \N $ such that
$\,
z_{j} - 
z_1^{i_1}
z_2^{i_2}
z_3^{i_3}
z_4^{i_4} \in  J_M : (z_1 \cdots z_{44})^\infty $.
Hence the semigroup algebra in (\ref{uv-algebra}) is:
$$ \k \bigl[
 \frac{a^2 e^{15} g^{18}}{ b^3 c^6 d^{10} f^{16}}, 
 \frac{ b^{13} d^{15} f^{16} }{a^8 c e^{21} g^{14}},
 \frac{ c^{59} d^{57} f^{110}} {e^{92} g^{134}}, 
 \frac{ a c^{14} d^{11} f^{23}} {b e^{19} g^{29}}
\bigr] \,\,\, \simeq \,\,\,
\frac{\k[z_1,z_2,z_3,z_4]}{\langle z_1^5  z_2 z_3 - z_4^2 \rangle}.
$$
This algebra is not integrally closed, since
a toric hypersurface is normal if and only if
at least one of the two monomials in the defining equation 
is square-free. Its integral closure 
in $\k[ ker_\ZZ(A) ]$ is generated by the
Laurent monomial
\begin{equation}
\label{witness}
\frac{z_4}{z_1^2} \,\, = \,\, 
( z_1 z_2 z_3)^{\frac{1}{2}}  \,\, = \,\, 
\frac{ b^5 c^{26} d^{31} f^{55}}{a^3 e^{49} g^{65}}.
\end{equation}
Hence the affine chart (\ref{normal-uv-algebra}) of the toric variety
of the Gr\"obner fan of $I_A$ is the spectrum of the normal domain
$  \k[z_1,z_2,z_3,y]/ \langle
z_1  z_2 z_3 - y^2 \rangle$, 
where $y$ maps to (\ref{witness}).
\qed
\end{proof}

We now examine the local equations of $Hilb_A$ about $M$ for this 
example.
\beginOutput
i64 : JM = localCoherentEquations(IA)\\
\emptyLine
\                                                                       $\cdot\cdot\cdot$\\
o64 = ideal (z z  - z , z z  - z , z z  - z , z z  - z , z z  - z , z  $\cdot\cdot\cdot$\\
\              1 2    3   1 2    3   1 5    4   1 3    6   1 3    6   1 $\cdot\cdot\cdot$\\
\emptyLine
o64 : Ideal of B\\
\endOutput
\beginOutput
i65 : G = removeRedundantVariables JM;\\
\emptyLine
o65 : RingMap B <--- B\\
\endOutput
\beginOutput
i66 : toString ideal gens gb(G JM)\\
\emptyLine
o66 = ideal(z_32*z_42^2*z_44-z_37^2*z_42,z_32^3*z_35*z_37^2-z_42^2*z_4 $\cdot\cdot\cdot$\\
\endOutput

This ideal has six generators and decomposing it 
%%% $\langle z_32z_42^2z_44-z_37^2z_42,
%%% z_32^3z_35z_37^2-z_42^2z_44,
%%% z_32^4z_35z_37-z_37z_42,
%%% z_32^2z_35z_37^4z_42-z_42^4z_44^2,
%%% z_32z_35z_37^6z_42-z_42^5z_44^3,
%%% z_35z_37^8z_42-z_42^6z_44^4 \rangle$
we see that there are five components 
through the monomial ideal $M$ on this toric Hilbert scheme. They 
are defined by the ideals: 
\begin{itemize}
\item $\langle z_{32}z_{42}z_{44}-z_{37}^2,z_{32}^4z_{35}-z_{42},
z_{32}^3z_{35}z_{37}^2-z_{42}^2z_{44},
z_{32}^2z_{35}z_{37}^4-z_{42}^3z_{44}^2,\\
\indent \indent z_{32}z_{35}z_{37}^6-z_{42}^4z_{44}^3,
z_{35}z_{37}^8-z_{42}^5z_{44}^4 \rangle$ 
\item $\langle z_{44},z_{37} \rangle$
\item $\langle z_{37},z_{42}^2 \rangle$
\item $\langle z_{42},z_{35} \rangle$ 
\item $\langle z_{42},z_{32}^3 \rangle$.
\end{itemize}
All five components are three
dimensional. The first component is an affine patch of the coherent
component and two of the components are not reduced. Let $K$ be the 
first of these ideals.

\beginOutput
i67 : K = ideal(z_32*z_42*z_44-z_37^2,z_32^4*z_35-z_42,\\
\          z_32^3*z_35*z_37^2-z_42^2*z_44,z_32^2*z_35*z_37^4-z_42^3*z_44^2,\\
\          z_32*z_35*z_37^6-z_42^4*z_44^3,z_35*z_37^8-z_42^5*z_44^4);\\
\emptyLine
o67 : Ideal of B\\
\endOutput

Applying {\tt removeRedundantVariables} to $K$ we see that 
the affine patch of the coherent component is, locally at $M$,
a non-normal hypersurface singularity (agreeing with (\ref{witness})).
The labels on the variables depend on the order of elements in 
the initial ideal $M$ computed by \Mtwo in line {\tt i61}.

\beginOutput
i68 : GG = removeRedundantVariables K;\\
\emptyLine
o68 : RingMap B <--- B\\
\endOutput
\beginOutput
i69 : ideal gens gb (GG K)\\
\emptyLine
\             5           2\\
o69 = ideal(z  z  z   - z  )\\
\             32 35 44    37\\
\emptyLine
o69 : Ideal of B\\
\endOutput

There is a general algorithm due to de Jong \cite{HS:DJ} for 
computing the \ie{normalization} of any affine variety. 
In the toric case, the problem of normalization amounts to  
computing the minimal {\em \ie{Hilbert basis}} of a given convex
rational polyhedral cone \cite{HS:Sch}. An efficient implementation can be 
found in the software package {\tt Normaliz}\indexcmd{Normaliz} by Bruns and
Koch \cite{HS:BK}.

Our computational study of the toric Hilbert scheme in this
chapter was based on local equations rather than
global equations (arising from a projective embedding of  $Hilb_A$),
because the latter system of equations tends to be too large 
for most purposes. Nonetheless, they are interesting.
In the remainder of this section, we present a canonical 
projective embedding of the coherent component of $Hilb_A$.

Let $G_1, G_2, G_3, \ldots, G_s$ denote all the {\it Graver fibers} of
the matrix $A$. In Section 1 we showed how to compute them in 
{\sl Macaulay 2}. Each
set $G_i$ consists of the monomials in $\k[x_1,\ldots,x_n]$ 
that have a fixed Graver degree.  Consider the set $\, {\mathbf G} \,
:= \, G_1 G_2 G_3 \cdots G_s \,$ that consists of all monomials that
are products of monomials, one from each of the distinct Graver
fibers. Let $t$ denote the cardinality of ${\mathbf G}$.  We introduce
an extra indeterminate $z$, and we consider the $\N$-graded semigroup
algebra $\,\k[z {\mathbf G}] $, which is a subalgebra of
$\k[x_1,\ldots,x_n,z]$. The grading of this algebra is $\,deg(z) = 1\,$ and
$\, deg(x_i) = 0$. Labeling the elements of ${\mathbf G}$ with
indeterminates $y_i$, we can write
$$ \k[z {\mathbf G}]   = 
\k[y_1,y_2,\ldots,y_t]/P_A, $$
where $P_A$ is a homogeneous toric ideal
associated with a configuration of $t$ vectors in $\ZZ^{n+1}$.
We note that the torus $(\k^*)^n$ acts naturally on 
$\, \k[z {\mathbf G}]$.

\begin{example} \rm
Let $n=4,d=2$ and 
 $\, A \, = \, \left( \begin{array}{cccc}
3 & 2 & 1 & 0 \\
0 & 1 & 2 & 3 
 \end{array} \right) $,
so that $I_A$ is the ideal of the twisted cubic curve.
There are five Graver fibers:

\beginOutput
i70 : A = \{\{1,1,1,1\},\{0,1,2,3\}\};\\
\endOutput
\beginOutput
i71 : I = toricIdeal A;\\
\emptyLine
o71 : Ideal of R\\
\endOutput
\beginOutput
i72 : Graver = graver I;\\
\emptyLine
\              1       5\\
o72 : Matrix R  <--- R\\
\endOutput
\beginOutput
i73 : fibers = graverFibers Graver;\\
\endOutput
\beginOutput
i74 : peek fibers\\
\emptyLine
o74 = HashTable\{\{2, 2\} => | ac b2 |     \}\\
\                \{2, 3\} => | ad bc |\\
\                \{2, 4\} => | bd c2 |\\
\                \{3, 3\} => | a2d abc b3 |\\
\                \{3, 6\} => | ad2 bcd c3 |\\
\endOutput

The set ${\mathbf G} = G_1 G_2 G_3 G_4 G_5 \,$ consists of
$22$ monomials of degree $14$.

\beginOutput
i75 : G = trim product(values fibers, ideal)\\
\emptyLine
\              5     5   4 3 5   5 3 4   4 2 2 4   3 4   4   2 6 4   4  $\cdot\cdot\cdot$\\
o75 = ideal (a b*c*d , a b d , a c d , a b c d , a b c*d , a b d , a b $\cdot\cdot\cdot$\\
\emptyLine
o75 : Ideal of R\\
\endOutput
\beginOutput
i76 : numgens G\\
\emptyLine
o76 = 22\\
\endOutput

We introduce a polynomial ring in $22$ variables
$y_1,y_2,\ldots,y_{22}$, and we compute the ideal $P_A$.
It is generated by $180$ binomial quadrics. 

\beginOutput
i77 : z = symbol z;\\
\endOutput
\beginOutput
i78 : S = QQ[a,b,c,d,z];\\
\endOutput
\beginOutput
i79 : zG = z ** substitute(gens G, S);\\
\emptyLine
\              1       22\\
o79 : Matrix S  <--- S\\
\endOutput
\beginOutput
i80 : R = QQ[y_1 .. y_22];\\
\endOutput
\beginOutput
i81 : F = map(S,R,zG)\\
\emptyLine
\                5     5    4 3 5    5 3 4    4 2 2 4    3 4   4    2 6 $\cdot\cdot\cdot$\\
o81 = map(S,R,\{a b*c*d z, a b d z, a c d z, a b c d z, a b c*d z, a b  $\cdot\cdot\cdot$\\
\emptyLine
o81 : RingMap S <--- R\\
\endOutput
\beginOutput
i82 : PA = trim ker F\\
\emptyLine
\              2                                                        $\cdot\cdot\cdot$\\
o82 = ideal (y   - y  y  , y  y   - y  y  , y  y   - y  y  , y  y   -  $\cdot\cdot\cdot$\\
\              21    20 22   19 21    18 22   18 21    17 22   17 21    $\cdot\cdot\cdot$\\
\emptyLine
o82 : Ideal of R\\
\endOutput

These equations define a toric surface
of degree $30$ in projective $21$-space.
\beginOutput
i83 : codim PA\\
\emptyLine
o83 = 19\\
\endOutput
\beginOutput
i84 : degree PA\\
\emptyLine
o84 = 30\\
\endOutput

The surface is smooth, but there are too many equations and the
codimension is too large to use the Jacobian criterion for smoothness
\cite[\S 16.6]{HS:Eis} directly. Instead we check smoothness for each
open set $y_i \neq 0$. 

\beginOutput
i85 : Aff = apply(1..22, v -> (\\
\                             K = substitute(PA,y_v => 1);\\
\                             FF = removeRedundantVariables K;\\
\                             ideal gens gb (FF K)));\\
\endOutput
\beginOutput
i86 : scan(Aff, i -> print toString i);\\
ideal()\\
ideal()\\
ideal()\\
ideal(y_1^4*y_5*y_21-1)\\
ideal(y_1^4*y_6^6*y_21-1)\\
ideal()\\
ideal(y_1^2*y_11^2*y_17-1)\\
ideal(y_1^3*y_9^2*y_21^2-1)\\
ideal(y_6^3*y_21-y_10,y_1*y_10^3-y_6^2,y_1*y_6*y_10^2*y_21-1)\\
ideal(y_6*y_15-1,y_2*y_15^2-y_6*y_14,y_6^2*y_14-y_2*y_15)\\
ideal()\\
ideal(y_11*y_13-1,y_1^2*y_21^3-y_13^2)\\
ideal(y_1^2*y_14^3*y_21^3-1)\\
ideal(y_10^2*y_21-1,y_1*y_15^4-y_10^3)\\
ideal()\\
ideal(y_11*y_20-1,y_3*y_20^2-y_11*y_17,y_11^2*y_17-y_3*y_20)\\
ideal(y_11*y_18*y_21-1,y_1*y_21^3-y_11*y_18^2,y_11^2*y_18^3-y_1*y_21^2)\\
ideal(y_1*y_19^4*y_21^4-1)\\
ideal(y_15*y_22-1)\\
ideal()\\
ideal(y_20*y_22-1)\\
ideal()\\
\endOutput

By examining these local equations, we see that $Hilb_A$ is smooth, and also
that there are eight fixed points under the
action of the $2$-dimensional torus. They correspond
to the variables $y_1,y_2,y_3,y_6,y_{11},y_{15},y_{20}$ and $y_{22}$. 
By setting any of these eight variables to $1$ in the
$180$ quadrics above, we obtain an affine variety
isomorphic to the affine plane.
\end{example}

\begin{theorem} \label{isomorphism}
The coherent component of the toric Hilbert scheme $Hilb_A$ is
isomorphic to the projective spectrum $\,Proj \,\k[z {\mathbf G}]\,$
of the algebra $\k[z {\mathbf G}]$.
\end{theorem}

\begin{proof} The first
step is to define a morphism from $\,Hilb_A \,$ to
the $(t-1)$-dimensional projective space
$\P({\mathbf G}) = Proj \, \k[y_1,y_2,\ldots,y_t]$.
Consider any point $I$ on $Hilb_A$. We intersect
the ideal $I$ with the finite-dimensional vector space
$\k G_i$, consisting of all homogeneous polynomials
in $\k[x_1,\ldots,x_n]$ that lie in the $i$-th  Graver degree.
The definition of $A$-graded ideal implies that
$I \cap \k G_i$ is a linear subspace of codimension $1$ in $\k G_i$.
We represent this subspace by an equation
$\, g_i(I) = \sum_{u \in G_i } c_u x^u \,$, which is 
unique up to scaling. Taking the product of these
polynomials for $i=1,\ldots,t$, we get a unique (up to scaling)
polynomial that is supported on ${\mathbf G} = G_1 G_2 \cdots G_t$.
The map $\, I \mapsto g_1(I) g_2(I) \cdots g_t(I)\,$
defines a morphism from $Hilb_A \,$ to $\P({\mathbf G})$.
This morphism is equivariant with respect to the  $(\k^*)^n$-action
on both schemes.

Consider the restriction of this equivariant
 morphism to the coherent component of the toric Hilbert scheme.
It maps the $(\k^*)^n$-orbit of the toric ideal $I_A$
into the subvariety $\,Proj \,\k[z {\mathbf G}]\,$
of $\P({\mathbf G})$. This inclusion
is an isomorphism onto the dense torus, 
as the dimension of the Newton polytope of 
$$ g(I_A) = \prod_{i=1}^t \,(\sum_{u \in G_i }  x^u \,) $$
equals the dimension of the kernel of $A$. Equivalently,
the stabilizer of $g(I_A)$ in $(\k^*)^n$ 
consists only of those one-parameter subgroups
$w$ that restrict to zero on the kernel of $A$.

To show that our morphism is an isomorphism between the coherent component
and  $\,Proj \,\k[z {\mathbf G}]$,
we consider the affine chart around an initial monomial ideal
$M = in_w(I_A)$. The polynomial $g(M)$ is a monomial,
namely, it is the product of all standard monomials whose
degree is a  Graver degree. Moreover, $g(M)$ is the leading monomial
of $g(I_A)$ with respect to the weight vector $w$. The Newton
polytope of $g(I_A)$ is the Minkowski sum of the Newton polytopes 
of the polynomials $\,g_1(I_A),   \ldots, g_t(I_A)$,
and it is a state polytope for $I_A$, by \cite[Theorem 7.5]{HS:St2}.

Let $g(M) = x^q$, and let  $\sigma$  be the cone of the
Gr\"obner fan  $\Sigma_A$ that has $w$ in its interior. 
Then  $\sigma$ coincides with the normal cone at the vertex $q$ of 
the state  polytope described above  \cite[\S 3]{HS:St2}.
Consider the restriction of our morphism to the affine chart around $M$
of the coherent component,  as described in (\ref{uv-algebra}).
This restriction defines an isomorphism onto the variety
\begin{equation}
\label{other-uv-algebra}
Spec \,\,  \k[\, x^{p-q} \, : \,x^p \in {\mathbf G} \, ]
\end{equation}
On the other hand, the semigroup algebra in (\ref{other-uv-algebra})
is isomorphic to that in  (\ref{uv-algebra}) because
each pair of vectors $\{u_i, v_i\}$ seen in the 
reduced Gr\"obner basis lies in one of the Graver fibers $G_j$.  
Hence our morphism restricts to an isomorphism from the
affine chart around $M$ of the coherent component onto (\ref{other-uv-algebra}).
Finally, note that (\ref{other-uv-algebra}) is the principal affine
open subset of $\,Proj \,\k[z {\mathbf G}]\,$ defined by the
coordinate $x^q$. Hence we get an isomorphism between the
coherent component of $Hilb_A$ and $\,Proj \,\k[z {\mathbf G}]$.
\qed
\end{proof}

\appendix

\section{Fourier-Motzkin Elimination}\index{Fourier-Motzkin elimination}\label{FMe}

We now give the \Mtwo code for converting the generator/inequality
representation of a rational convex polyhedron to the other. It is
based on the Fourier-Motzkin elimination procedure for eliminating a
variable from a system of inequalities \cite {HS:Zie}. This code was
written by Greg Smith.

Given any cone $C \subset \R^d$, the polar cone of $C$ is defined to be
$$C^{\vee} = \{ x \in \R^d \mid x \cdot y \leq 0, \mbox{for all\ } y
\in C\}.$$

\noindent For a $d \times n$ matrix $Z$, define
$cone(Z) = \{ Z x \mid x \in \R_{\geq 0}^n \} \subset \R^d,$ and 
$\mathit{affine}(Z) = \{ Z x \mid x \in \R^n \} \subset \R^d.$
For two integer matrices $Z$ and $H$, both having  $d$
rows, {\tt polarCone(Z,H)} returns a list of two integer matrices
{\tt\char`\{A,E\char`\}} such that $$cone(Z) + \mathit{affine}(H) = \{ x \in \R^d \mid A^t
x \leq 0, E^t x = 0\}.$$ 
Equivalently, $(cone(Z) + \mathit{affine}(H))^\vee = cone(A) + \mathit{affine}(E).$

We now describe each routine in the package {\tt polarCone.m2}\indexcmd{polarCone.m2}.  We have
simplified the code for readability, sometimes at the cost of efficiency.
We start with three simple subroutines: {\tt primitive}, {\tt toZZ}, and {\tt
rotateMatrix}. 

\medskip
The routine {\tt primitive} takes a list of integers {\tt L}, and divides
each element of this list by their greatest common denominator.

\beginOutput
i87 : code primitive\\
\emptyLine
o87 = -- polarCone.m2:16-20\\
\      primitive = (L) -> (\\
\           n := #L-1;                    g := L#n;\\
\           while n > 0 do (n = n-1;      g = gcd(g, L#n);\\
\                if g === 1 then n = 0);\\
\           if g === 1 then L else apply(L, i -> i // g));\\
\endOutput

\medskip
The routine {\tt toZZ} converts a list of rational numbers to a list of 
integers, by multiplying by their common denominator.
\beginOutput
i88 : code toZZ\\
\emptyLine
o88 = -- polarCone.m2:28-32\\
\      toZZ = (L) -> (\\
\           d := apply(L, e -> denominator e);\\
\           R := ring d#0;             l := 1_R;\\
\           scan(d, i -> (l = (l*i // gcd(l,i))));    \\
\           apply(L, e -> (numerator(l*e))));\\
\endOutput

\medskip
The routine {\tt rotateMatrix} is a kind of transpose.  Its input is a
matrix, and its output is a matrix of the same shape as the transpose.
It places the matrix in the form so that in the routine {\tt polarCone},
computing a Gr\"obner basis will do the Gaussian elimination that is needed.
\beginOutput
i89 : code rotateMatrix\\
\emptyLine
o89 = -- polarCone.m2:41-43\\
\      rotateMatrix = (M) -> (\\
\           r := rank source M;        c := rank target M;\\
\           matrix table(r, c, (i,j) -> M_(c-j-1, r-i-1)));\\
\endOutput

\medskip
The procedure of Fourier-Motzkin elimination as presented by 
Ziegler in \cite{HS:Zie} is used, together with some heuristics that he
presents as exercises.  The following, which is a kind of $S$-pair
criterion for inequalities, comes from Exercise 2.15(i) in \cite{HS:Zie}.

The routine {\tt isRedundant} determines if a row vector (inequality)
is redundant. Its input argument {\tt V} is the same input that is
used in {\tt fourierMotzkin}: it is a list of sets of integers.  Each
entry contains indices of the original rays that do {\sl not} vanish
at the corresponding row vector.  {\tt vert} is a set of integers; the
original rays for the row vector in question.  A boolean value is
returned.  

\beginOutput
i90 : code isRedundant\\
\emptyLine
o90 = -- polarCone.m2:57-65\\
\      isRedundant = (V, vert) -> (\\
\           -- the row vector is redundant iff 'vert' contains an\\
\           -- entry in 'V'.\\
\           x := 0;            k := 0;\\
\           numRow := #V;      -- equals the number of inequalities\\
\           while x < 1 and k < numRow do (\\
\                if isSubset(V#k, vert) then x = x+1;\\
\                k = k+1;);     \\
\           x === 1);\\
\endOutput

\medskip
The main work horse of {\tt polarCone.m2} is the subroutine 
{\tt fourierMotzkin}, which eliminates the first variable in the
inequalities {\tt A} using the double description version of
Fourier-Motzkin elimination. The set {\tt A} is a list of lists of
integers, each entry corresponding to a row vector in the system of
inequalities.  The argument {\tt V} is a list of sets of integers.
Each entry contains the indices of the original rays that do {\sl
  not} vanish at the corresponding row vector in {\tt A}.  Note that
this set is the {\sl complement} of the set $V_i$ appearing in
exercise 2.15 in \cite{HS:Zie}. The argument {\tt spot} is the integer
index of the variable being eliminated.  

The routine returns a list {\tt \char`\{projA,projV\char`\}} where {\tt projA} is
a list of lists of integers.  Each entry corresponds to a row vector
in the projected system of inequalities.  The list {\tt projV} is a
list of sets of integers.  Each entry contains indices of the original
rays that do {\sl not} vanish at the corresponding row vector in {\tt
  projA}. 

\beginOutput
i91 : code fourierMotzkin\\
\emptyLine
o91 = -- polarCone.m2:89-118\\
\      fourierMotzkin = (A, V, spot) -> (\\
\           -- initializing local variables\\
\           numRow := #A;               -- equal to the length of V\\
\           numCol := #(A#0);           pos := \{\};       \\
\           neg := \{\};                  projA := \{\};     \\
\           projV := \{\};                k := 0;\\
\           -- divide the inequalities into three groups.\\
\           while k < numRow do (\\
\                if A#k#0 < 0 then neg = append(neg, k)\\
\                else if A#k#0 > 0 then pos = append(pos, k)\\
\                else (projA = append(projA, A#k);\\
\                     projV = append(projV, V#k););\\
\                k = k+1;);      \\
\           -- generate new irredundant inequalities.\\
\           scan(pos, i -> scan(neg, j -> (vert := V#i + V#j;\\
\                          if not isRedundant(projV, vert)  \\
\                          then (iRow := A#i;     jRow := A#j;\\
\                               iCoeff := - jRow#0;\\
\                               jCoeff := iRow#0;\\
\                               a := iCoeff*iRow + jCoeff*jRow;\\
\                               projA = append(projA, a);\\
\                               projV = append(projV, vert););)));\\
\           -- don't forget the implicit inequalities '-t <= 0'.\\
\           scan(pos, i -> (vert := V#i + set\{spot\};\\
\                if not isRedundant(projV, vert) then (\\
\                     projA = append(projA, A#i);\\
\                     projV = append(projV, vert););));\\
\           -- remove the first column \\
\           projA = apply(projA, e -> e_\{1..(numCol-1)\});\\
\           \{projA, projV\});   \\
\endOutput

\medskip
As mentioned above, {\tt polarCone} takes two matrices {\tt Z, H},
both having $d$ rows, and outputs a pair of matrices {\tt A, E} 
such that $(\operatorname{cone}(Z) + \operatorname{affine}(H))^\vee =
\operatorname{cone}(A) + \operatorname{affine}(E).$

\beginOutput
i92 : code(polarCone,Matrix,Matrix)\\
\emptyLine
o92 = -- polarCone.m2:137-192\\
\      polarCone(Matrix, Matrix) := (Z, H) -> (\\
\           R := ring source Z;\\
\           if R =!= ring source H then error ("polarCone: " | \\
\                "expected matrices over the same ring");\\
\           if rank target Z =!= rank target H then error (\\
\                "polarCone: expected matrices to have the " |\\
\                "same number of rows");     \\
\           if (R =!= ZZ) then error ("polarCone: expected " | \\
\                "matrices over 'ZZ'");\\
\           -- expressing 'cone(Y)+affine(B)' as '\{x : Ax <= 0\}'\\
\           Y := substitute(Z, QQ);     B := substitute(H, QQ);   \\
\           if rank source B > 0 then Y = Y | B | -B;\\
\           n := rank source Y;         d := rank target Y;     \\
\           A := Y | -id_(QQ^d);\\
\           -- computing the row echelon form of 'A'\\
\           A = gens gb rotateMatrix A;\\
\           L := rotateMatrix leadTerm A;\\
\           A = rotateMatrix A;\\
\           -- find pivots\\
\           numRow = rank target A;                  -- numRow <= d\\
\           i := 0;                     pivotCol := \{\};\\
\           while i < numRow do (j := 0;\\
\                while j < n+d and L_(i,j) =!= 1_QQ do j = j+1;\\
\                pivotCol = append(pivotCol, j);\\
\                i = i+1;);\\
\           -- computing the row-reduced echelon form of 'A'\\
\           A = ((submatrix(A, pivotCol))^(-1)) * A;\\
\           -- converting 'A' into a list of integer row vectors \\
\           A = entries A;\\
\           A = apply(A, e -> primitive toZZ e);\\
\           -- creating the vertex list 'V' for double description\\
\           -- and listing the variables 'T' which remain to be\\
\           -- eliminated\\
\           V := \{\};                    T := toList(0..(n-1));\\
\           scan(pivotCol, e -> (if e < n then (T = delete(e, T);\\
\                          V = append(V, set\{e\});)));\\
\           -- separating inequalities 'A' and equalities 'E'\\
\           eqnRow := \{\};               ineqnRow := \{\};\\
\           scan(numRow, i -> (if pivotCol#i >= n then \\
\                     eqnRow = append(eqnRow, i)\\
\                     else ineqnRow = append(ineqnRow, i);));    \\
\           E := apply(eqnRow, i -> A#i);\\
\           E = apply(E, e -> e_\{n..(n+d-1)\});\\
\           A = apply(ineqnRow, i -> A#i);\\
\           A = apply(A, e -> e_(T | toList(n..(n+d-1)))); \\
\           -- successive projections eliminate the variables 'T'.\\
\           if A =!= \{\} then scan(T, t -> (\\
\                     D := fourierMotzkin(A, V, t);\\
\                     A = D#0;          V = D#1;));\\
\           -- output formating\\
\           A = apply(A, e -> primitive e);\\
\           if A === \{\} then A = map(ZZ^d, ZZ^0, 0)\\
\           else A = transpose matrix A;\\
\           if E === \{\} then E = map(ZZ^d, ZZ^0, 0)\\
\           else E = transpose matrix E;\\
\           (A, E)); \\
\endOutput

If the input matrix $H$ has no columns, it can be omitted.  A sequence of two
matrices is returned, as above.
\beginOutput
i93 : code(polarCone,Matrix)\\
\emptyLine
o93 = -- polarCone.m2:199-200\\
\      polarCone(Matrix) := (Z) -> (\\
\           polarCone(Z, map(ZZ^(rank target Z), ZZ^0, 0)));\\
\endOutput

As a simple example, consider the permutahedron in $\R^3$ 
whose vertices are the following six points. 

\beginOutput
i94 : H = transpose matrix\{\\
\      \{1,2,3\},\\
\      \{1,3,2\},\\
\      \{2,1,3\},\\
\      \{2,3,1\},\\
\      \{3,1,2\},\\
\      \{3,2,1\}\};\\
\emptyLine
\               3        6\\
o94 : Matrix ZZ  <--- ZZ\\
\endOutput

The inequality representation of the permutahedron is obtained 
by calling {\tt polarCone} on $H$: the facet normals of the 
polytope are the columns of the matrix in the first argument of the 
output. The second argument is trivial since our input is a polytope 
and hence there are is no non-trivial affine space contained in it.
If we call {\tt polarCone} on the output, we will get back H as
expected. 

\beginOutput
i95 : P = polarCone H\\
\emptyLine
o95 = (| 1  1  1  -1 -1 -5 |, 0)\\
\       | -1 1  -5 1  -1 1  |\\
\       | -1 -5 1  -1 1  1  |\\
\emptyLine
o95 : Sequence\\
\endOutput
\beginOutput
i96 : Q = polarCone P_0\\
\emptyLine
o96 = (| 1 1 2 2 3 3 |, 0)\\
\       | 2 3 1 3 1 2 |\\
\       | 3 2 3 1 2 1 |\\
\emptyLine
o96 : Sequence\\
\endOutput
 

\section{Minimal Presentation of Rings}\label{Mpor}

Throughout this chapter, we have used on several occasions the simple, yet
useful subroutine {\tt removeRedundantVariables}.
In this appendix, we present \Mtwo code for this routine,
which is the main ingredient for finding minimal
presentations of quotients of polynomial rings.
Our code for this routine is a somewhat simplified, but less
efficient version of a routine in the \Mtwo package, {\tt minPres.m2}\indexcmd{minPres.m2},
written by Amelia Taylor.

The routine {\tt removeRedundantVariables} takes as input an ideal {\tt I} in
a polynomial ring {\tt A}.  It returns a ring map {\tt F} from {\tt A} to
itself that sends redundant variables to polynomials in the non-redundant
variables and sends non-redundant variables to themselves.  For example:
  \beginOutput
i97 : A = QQ[a..e];\\
\endOutput
  \beginOutput
i98 : I = ideal(a-b^2-1, b-c^2, c-d^2, a^2-e^2)\\
\emptyLine
\                2             2         2       2    2\\
o98 = ideal (- b  + a - 1, - c  + b, - d  + c, a  - e )\\
\emptyLine
o98 : Ideal of A\\
\endOutput
  \beginOutput
i99 : F = removeRedundantVariables I\\
\emptyLine
\                8       4   2\\
o99 = map(A,A,\{d  + 1, d , d , d, e\})\\
\emptyLine
o99 : RingMap A <--- A\\
\endOutput
The non-redundant variables are $d$ and $e$.  The image of $I$ under $F$
gives the elements in this smaller set of variables.  We take the ideal of a 
Gr\"obner basis of the image:
  \beginOutput
i100 : I1 = ideal gens gb(F I)\\
\emptyLine
\              16     8    2\\
o100 = ideal(d   + 2d  - e  + 1)\\
\emptyLine
o100 : Ideal of A\\
\endOutput
The original ideal can be written in a cleaner way as
  \beginOutput
i101 : ideal compress (F.matrix - vars A) + I1\\
\emptyLine
\               8           4       2       16     8    2\\
o101 = ideal (d  - a + 1, d  - b, d  - c, d   + 2d  - e  + 1)\\
\emptyLine
o101 : Ideal of A\\
\endOutput
  
  Let us now describe the \Mtwo code.  The subroutine {\tt
    findRedundant} takes a polynomial $f$, and finds a variable $x_i$
  in the ring of $f$ such that $f = c x_i + g$ for a non-zero
  constant $c$ and a polynomial $g$ that does not involve the
  variable $x_i$.  If there is no such variable, {\tt null} is
  returned.  Otherwise, if $x_i$ is the first such variable , the list
  $\{i, c^{-1} g\}$ is returned.

\beginOutput
i102 : code findRedundant\\
\emptyLine
o102 = -- minPres.m2:1-12\\
\       findRedundant=(f)->(\\
\            A := ring(f);\\
\            p := first entries contract(vars A,f);\\
\            i := position(p, g -> g != 0 and first degree g === 0);\\
\            if i === null then\\
\                null\\
\            else (\\
\                 v := A_i;\\
\                 c := f_v;\\
\                 \{i,(-1)*(c^(-1)*(f-c*v))\}\\
\                 )\\
\            )\\
\endOutput

The main function {\tt removeRedundantVariables} requires an ideal in a
polynomial ring (not a quotient ring) as input.  The internal
routine {\tt findnext} finds the first entry of the (one row) matrix {\tt M}
that contains a redundancy.  This redundancy is used to modify the list {\tt
xmap}, which contains the images of the redundant variables.
The matrix {\tt M}, and the list {\tt xmap} are both updated, and 
then we continue to look for more redundancies.

\beginOutput
i103 : code removeRedundantVariables\\
\emptyLine
o103 = -- minPres.m2:14-39\\
\       removeRedundantVariables = (I) -> (\\
\            A := ring I;\\
\            xmap := new MutableList from gens A;       \\
\            M := gens I;\\
\            findnext := () -> (\\
\                 p := null;\\
\                 next := 0;\\
\                 done := false;\\
\                 ngens := numgens source M;\\
\                 while next < ngens and not done do (\\
\                   p = findRedundant(M_(0,next));\\
\                   if p =!= null then\\
\                        done = true\\
\                   else next=next+1;\\
\                 );\\
\                 p);\\
\            p := findnext();\\
\            while p =!= null do (\\
\                 xmap#(p#0) = p#1;\\
\                 F1 := map(A,A,toList xmap);\\
\                 F2 := map(A,A, F1 (F1.matrix));\\
\                 xmap = new MutableList from first entries F2.matrix;\\
\                 M = compress(F2 M);\\
\                 p = findnext();\\
\                 );\\
\            map(A,A,toList xmap));\\
\endOutput
\begin{thebibliography}{10}

\bibitem{HS:Arn}
V.I. Arnold:
\newblock A-graded algebras and continued fractions.
\newblock {\em Communications in Pure and Applied Mathematics}, 42:993--1000,
  1989.

\bibitem{HS:BLR}
A~Bigatti, R.~La~Scala, and L.~Robbiano:
\newblock Computing toric ideals.
\newblock {\em Journal of Symbolic Computation}, 27:351--365, 1999.

\bibitem{HS:BFS}
L.~J. Billera, P.~Filliman, and B.~Sturmfels:
\newblock Constructions and complexity of secondary polytopes.
\newblock {\em Advances in {M}athematics}, 83:155--179, 1990.

\bibitem{HS:BK}
W.~Bruns and R.~Koch:
\newblock Normaliz, a program to compute normalizations of semigroups.
\newblock available by anonymous ftp from
  ftp.mathematik.Uni-Osnabrueck.DE/pub/osm/kommalg/software/.

\bibitem{HS:CLO}
D.~Cox, J.~Little, and D.~O'Shea:
\newblock {\em Ideals, Varieties, and Algorithms. An Introduction to
  Computational Algebraic Geometry and Commutative Algebra}.
\newblock Springer-Verlag, New York, 1997.

\bibitem{HS:DJ}
T.~de~Jong:
\newblock An algorithm for computing the integral closure.
\newblock {\em Journal of Symbolic Computation}, 26:273--277, 1998.

\bibitem{HS:Eis}
D.~Eisenbud:
\newblock {\em Commutative Algebra with a View Toward Algebraic Geometry}.
\newblock Springer-Verlag, New York, 1994.

\bibitem{HS:GKZ}
I.~M. Gel'fand, M.~Kapranov, and A.~Zelevinsky:
\newblock {\em Multidimensional Determinants, Discriminants and Resultants}.
\newblock Birkh{\"a}user, Boston, 1994.

\bibitem{HS:Gra}
J.E. Graver:
\newblock On the foundations of linear and integer programming.
\newblock {\em Mathematical Programming}, 8:207--226, 1975.

\bibitem{HS:HM}
S.~Ho{\c{s}}ten and D.~Maclagan:
\newblock The vertex ideal of a lattice.
\newblock {P}reprint 2000.

\bibitem{HS:HS}
S.~Ho{\c{s}}ten and J.~Shapiro:
\newblock Primary decomposition of lattice basis ideals.
\newblock {\em Journal of Symbolic Computation}, 29:625--639, 2000.

\bibitem{HS:HT}
B.~Huber and R.R. Thomas:
\newblock Computing {G}r\"obner fans of toric ideals.
\newblock {\em Experimental Mathematics}, 9:321--331, 2000.
\newblock Software, {TiGERS}, available at
  {http://www.math.washington.edu/\~{}thomas/programs.html}.

\bibitem{HS:KPR}
E.~Korkina, G.~Post, and M.~Roelofs:
\newblock Classification of generalized ${A}$-graded algebras with $3$
  generators.
\newblock {\em Bulletin de Sciences Math\'ematiques}, 119:267--287, 1995.

\bibitem{HS:MT}
D.~Maclagan and R.R. Thomas:
\newblock Combinatorics of the toric {H}ilbert scheme.
\newblock {\em Discrete and Computational Geometry}.
\newblock To appear.

\bibitem{HS:MR}
T.~Mora and L.~Robbiano:
\newblock The {G}r\"obner fan of an ideal.
\newblock {\em Journal of Symbolic Computation}, 6:183--208, 1998.

\bibitem{HS:PS2}
I.~Peeva and M.~Stillman:
\newblock Local equations for the toric {H}ilbert scheme.
\newblock {\em Advances in Applied Mathematics}.
\newblock To appear.

\bibitem{HS:PS1}
I.~Peeva and M.~Stillman:
\newblock Toric {H}ilbert schemes.
\newblock Preprint 1999.

\bibitem{HS:Reiner}
V.~Reiner:
\newblock The generalized {B}aues problem.
\newblock In L.~Billera, A.~Bj{\"o}rner, C.~Greene, R.~Simion, and R.~Stanley,
  editors, {\em New {P}erspectives in Algebraic Combinatorics}. Cambridge
  University Press, 1999.

\bibitem{HS:San}
F.~Santos:
\newblock A point configuration whose space of triangulations is disconnected.
\newblock {\em Journal of the American Math.~Soc.}, 13:611--637, 2000.

\bibitem{HS:Sch}
A.~Schrijver:
\newblock {\em Theory of Linear and Integer Programming}.
\newblock Wiley-Interscience, Chichester, 1986.

\bibitem{HS:St1}
B.~Sturmfels:
\newblock The geometry of ${A}$-graded algebras.
\newblock math.AG/9410032.

\bibitem{HS:St2}
B.~Sturmfels:
\newblock {\em Gr\"obner Bases and Convex Polytopes}, volume~8.
\newblock American Mathematical Society, University Lectures, 1996.

\bibitem{HS:ST}
B.~Sturmfels and R.R. Thomas:
\newblock Variation of cost functions in integer programming.
\newblock {\em Mathematical Programming}, 77:357--387, 1997.

\bibitem{HS:Tho}
R.R. Thomas:
\newblock Applications to integer programming.
\newblock In D.A. Cox and B.~Sturmfels, editors, {\em Applications of
  Computational Algebraic Geometry}. AMS Proceedings of Symposia in Applied
  Mathematics, 1997.

\bibitem{HS:Zie}
G.~Ziegler:
\newblock {\em Lectures on Polytopes}, volume 152.
\newblock Springer-Verlag, New York, 1995.

\end{thebibliography}
\egroup
\makeatletter
\renewcommand\thesection{\@arabic\c@section}
\makeatother



%%%%%%%%%%%%%%%%%%%%%%%%%%%%%%%%%%%%%%%%%%%%%%%%
%%%%%
%%%%% ../chapters/exterior-algebra/chapter
%%%%%
%%%%%%%%%%%%%%%%%%%%%%%%%%%%%%%%%%%%%%%%%%%%%%%%

\bgroup
\title{Sheaf Algorithms Using the Exterior Algebra}
\titlerunning{Sheaf Algorithms}
\toctitle{Sheaf Algorithms Using the Exterior Algebra}
\author{Wolfram Decker
        % \inst 1
   \and David Eisenbud
        % \inst 2
        }
\authorrunning{W. Decker and D. Eisenbud}
% \institute{FB Mathematik, Universit\"at des Saarlandes, 66041 Saarbr\"ucken, Germany
%            \and
%            1000 Centennial Drive, Mathematical Sciences Research Institute, Berkeley, CA 94720, USA}

%%%%%%%%%%%%% EISENBUDS %%%%%%%%%%%%%%%%%%
%\input begin.tex
%\showlabels
%\showlabelsabove
% \input diagrams.tex
%\vsize=13.6truecm
%\hsize=19truecm
% \overfullrule=0pt
% Gothic fonts from AMSTeX 
\font\tengoth=eufm10  \font\fivegoth=eufm5
\font\sevengoth=eufm7
\newfam\gothfam  \scriptscriptfont\gothfam=\fivegoth 
\textfont\gothfam=\tengoth \scriptfont\gothfam=\sevengoth
\def\goth{\fam\gothfam\tengoth}
%
% Bold italic fonts 
\font\tenbi=cmmib10  \font\fivebi=cmmib5
\font\sevenbi=cmmib7
\newfam\bifam  \scriptscriptfont\bifam=\fivebi 
\textfont\bifam=\tenbi \scriptfont\bifam=\sevenbi
\def\bi{\fam\bifam\tenbi}
%
\font\hd=cmbx10 scaled\magstep1
%%%%%%%%%%%%%%%%%%%%%% EISENBUD ENDE %%%%%%%%%%%%%%%%%%

%\usepackage{amsmath,amscd,amsthm,amssymb,amsxtra,latexsym,epsfig,epic,eepic,graphics}

% \usepackage{amsmath,amscd,amsthm,amssymb,amsxtra,latexsym,epsfig,epic,graphics}

% \usepackage[matrix,arrow,curve]{xy}
%

\def \fix#1 {{\par \bf (( #1 ))\par}}
\def \Box {\hfill\hbox{}\nobreak \vrule width 1.6mm height 1.6mm
depth 0mm  \par \goodbreak \smallskip}
\def \reg {\mathop{\rm regularity}}
\def \coker {\mathop{\rm coker}}
\def \ker {\mathop{\rm ker}}
\def \im {\mathop{\rm im}}
\def \deg  {\mathop{\rm deg}}
\def \depth {\mathop{\rm depth}}
\def \span {\mathop{\rm span}}
\def \socle {\mathop{\rm socle}}
\def \dim{{\rm dim}}
\def \codim{{\rm codim}}
\def \Im  {\mathop{\rm Im}}
\def \ann  {\mathop{\rm ann}}
\def \rank {\mathop{\rm rank}}
\def \sing {\mathop{\rm Sing}}
\def \iso {\cong}
\def \tensor {\otimes}
\def \dsum {\oplus}
\def \intersect {\cap}
\def \Hom {{\mathop{\rm Hom}\nolimits}}
\def \hom {{\mathop{\rm Hom}\nolimits}}
\def \Ext {{\rm Ext}}
\def \ext{{\rm Ext}}
\def \Tor {{\rm Tor}}
\def \tor{{\rm Tor}}
\def \Sym {{\mathop{\rm Sym}\nolimits}}
\def \End {{\mathop{\rm End}\nolimits}}
\def \sym{{\rm Sym}}
\def \GL{{\rm GL}}
\def \Proj{{\rm Proj}}
\def \h {{\rm h}}

\def \coh{{\rm Coh}}
%\def \BGG{{\rm BGG}}
\def \lin{{\rm lin}}

\def \th {{^{\rm th}}}
\def \st {{^{\rm st}}}


\def \A {{\cal A}}
\def \AA {{\bf A}}
\def \B {{\cal B}}
\def \C {{\cal C}}
\def \CC {{\bf C}}
\def \DD  {{\bf D}}
\def \E  {{\cal E}}
\def \F {{\cal F}}
\def \FF {{\bf F}}
\def \G {{\cal G}}
\def \GG {{\bf G}}
\def \K {{\cal K}}
\def \H {{\rm H}}
\def \KK {{\bf K}}
\def \L {{\cal L}}
\def \LL {{\bf L}}
\def \MM{{\bf M}}
\def \N {{\cal N}}
\def \O {{\cal O}}
\def \P {{\bf P}}
\def \PP {{\bf P}}
\def \RR {{\bf R}}
\def \TT {{\bf T}}
\def \Z {{\bf Z}}

\def \gm {{\goth m}}



%
%%%%%%%%%%%%%%%%%%%%%%%%%%%%
%%%The black board font
%%%%%%%%%%%%%%%%%%%%%%%%%%%
%\newcommand{\A}{{\mathbb A}}
%\newcommand{\B}{{\mathbb B}}
%\newcommand{\C}{{\mathbb C}}
%\newcommand{\D}{{\mathbb D}}
%\newcommand{\E}{{\mathbb E}}
%\newcommand{\F}{{\mathbb F}}
%\newcommand{\G}{{\mathbb G}}
%\newcommand{\H}{{\mathbb H}}
%\newcommand{\I}{{\mathbb I}}
%\newcommand{\J}{{\mathbb J}}
%\newcommand{\K}{{\mathbb K}}
%\newcommand{\L}{{\mathbb L}}
%\newcommand{\M}{{\mathbb M}}
%\newcommand{\N}{{\mathbb N}}
%\newcommand{{\mathbb O}}
%\newcommand{{\mathbb P}}
\newcommand{\QQ}{{\mathbb Q}}
%\newcommand{\R}{{\mathbb R}}
%\newcommand�{{\mathbb S}}
%\newcommand{\T}{{\mathbb T}}
%\newcommand{\U}{{\mathbb U}}
%\newcommand{\V}{{\mathbb V}}
%\newcommand{\W}{{\mathbb W}}
%\newcommand{\X}{{\mathbb X}}
%\newcommand{\Y}{{\mathbb Y}}
%\newcommand{\Z}{{\mathbb Z}}

%\newcommand{\LL}{{\mathbb L}}
%\newcommand{\TT}{{\mathbb T}}
%\newcommand{\HH}{{\mathbb H}}

\def\pp#1{{\text{$\text{I\!P}_#1$}}}
%\newcommand{\PP}{{\mathbb P}}

%\DeclareMathOperator{\gin}{gin}

%\newcommand{\Syz}{{\rm{Syz}\;}}
%\newcommand{\sym}{{\rm{Sym}}}
%\newcommand{\SSyz}{{\rm{Syz}}}
%\newcommand{\spoly}{{\rm{spoly}}}
%\newcommand{\Spe}{{\rm{Sp}}}
%\newcommand{\openC}{{\mathbb C}}
%\newcommand{\ms}{{\rm{m}}}
\newcommand{\SL}{{\rm{SL}}}
%\newcommand{\IS}{{\rm{I}}}
%\newcommand{\Loc}{{\rm{Loc}\,}}
%\newcommand{\lcm}{{\rm{lcm}}}
%\newcommand{\lc}{{\rm{lc}}}
%\newcommand{\lm}{{\rm{lm}}}
%\newcommand{\con}{{\rm{c}}}
%\newcommand{\ext}{{\rm{e}}}
%\newcommand{\ec}{{\rm{ec}}}
%\newcommand{\ann}{{\rm{ann}}}
%\newcommand{\Ext}{{\rm{Ext}}}
%\newcommand{\equi}{{\rm{equi}}}
%\newcommand{\rk}{{\rm{rk}}}

%%%%%%%%%%%%%%%%%%%%%%%%%%%%%
%%% new commands for calligraphic characters with amsmath
%%%%%%%%%%%%%%%%%%%%%%%%%%%%

\newcommand{\ka}{{\mathcal A}}
\newcommand{\kb}{{\mathcal B}}
\newcommand{\kc}{{\mathcal C}}
\newcommand{\kd}{{\mathcal D}}
\newcommand{\ke}{{\mathcal E}}
\newcommand{\kf}{{\mathcal F}}
\newcommand{\kg}{{\mathcal G}}
\newcommand{\kh}{{\mathcal H}}
\newcommand{\ki}{{\mathcal I}}
\newcommand{\kj}{{\mathcal J}}
\newcommand{\kk}{{\mathcal K}}
\newcommand{\kl}{{\mathcal L}}
\newcommand{\km}{{\mathcal M}}
\newcommand{\kn}{{\mathcal N}}
\newcommand{\ko}{{\mathcal O}}
\newcommand{\kp}{{\mathcal P}}
\newcommand{\kq}{{\mathcal Q}}
\newcommand{\kr}{{\mathcal R}}
\newcommand{\ks}{{\mathcal S}}
\newcommand{\kt}{{\mathcal T}}
\newcommand{\ku}{{\mathcal U}}
\newcommand{\kv}{{\mathcal V}}
\newcommand{\kw}{{\mathcal W}}
\newcommand{\kx}{{\mathcal X}}
\newcommand{\ky}{{\mathcal Y}}
\newcommand{\kz}{{\mathcal Z}}
%%%%%%%%%%%%%%%%%%%%%%%%%%%%%%
%%%The mathscript for sheaves
%%%%%%%%%%%%%%%%%%%%% %%%%%%%%%
\newcommand{\s}{\mathscr}
\newcommand{\sA}{{\s A}}
\newcommand{\sB}{{\s B}}
\newcommand{\sC}{{\s C}}
\newcommand{\sD}{{\s D}}
\newcommand{\sE}{{\s E}}
\newcommand{\sF}{{\s F}}
\newcommand{\sG}{{\s G}}
\newcommand{\sH}{{\s H}}
\newcommand{\sI}{{\s I}}
\newcommand{\sJ}{{\s J}}
\newcommand{\sK}{{\s K}}
\newcommand{\sL}{{\s L}}
\newcommand{\sM}{{\s M}}
\newcommand{\sN}{{\s N}}
\newcommand{\sO}{{\s O}}
\newcommand{\sP}{{\s P}}
\newcommand{\sQ}{{\s Q}}
\newcommand{\sR}{{\s R}}
\newcommand{\sS}{{\s S}}
\newcommand{\sT}{{\s T}}
\newcommand{\sU}{{\s U}}
\newcommand{\sV}{{\s V}}
\newcommand{\sW}{{\s W}}
\newcommand{\sX}{{\s X}}
\newcommand{\sY}{{\s Y}}
\newcommand{\sZ}{{\s Z}}

\newcommand{\cO}{{\s O}}
\newcommand{\cI}{{\s I}}
% \newcommand{\cL}{{\s L}} %% \cL is used in \xy
% \newcommand{\cR}{{\s R}} %% \cR is used in \xy
\newcommand{\cN}{{\s N}}
\newcommand{\cT}{{\s T}}
\newcommand{\cX}{{\s X}}
%%%%%%%%%%%%%%%%%%%%%%%%%%%%%%%%
%% Arrows
%%%%%%%%%%%%%%%%%%%%%%%%%%%%%%%
\newcommand{\inj}{\hookrightarrow}
\newcommand{\surj}{\lra \lra}
\newcommand{\lra}{\longrightarrow}
\newcommand{\lla}{\longleftarrow}
%%%%%%%%%%%%%%%%%%%%%%%%%%%%%%%%%%%%
%\newcommand{\C}{\C}
\newcommand{\openP}�
\newcommand{\uf}{{\bf F}}
\newcommand{\uc}{{\bf C}}
%\newcommand{\tensor}{\otimes}
\newcommand{\mi}{{\bf m}}
\newcommand{\tX}{\widetilde{X}}
\newcommand{\punkt}{\hspace{-.3ex}\raise.15ex\hbox to1ex{\Huge.}}
\newcommand{\tpunkt}{\hspace{-.3ex}\hbox to1ex{\Huge.}}
\newlength{\br}
\newlength{\ho}
%\DeclareMathOperator{\GL}{GL}
%\DeclareMathOperator{\Aut}{Aut}
%\DeclareMathOperator{\Oo}{O}
%\DeclareMathOperator{\Spec}{Spec}
%\DeclareMathOperator{\Hom}{Hom}
%\DeclareMathOperator{\Tor}{Tor}
%\DeclareMathOperator{\syz}{syz}
%\DeclareMathOperator{\ord}{ord}
%\DeclareMathOperator{\word}{w\,ord}
%\DeclareMathOperator{\supp}{supp}
%\DeclareMathOperator{\Ker}{Ker}
%\DeclareMathOperator{\im}{im}
%\DeclareMathOperator{\wdeg}{w\,deg}
%\DeclareMathOperator{\depth}{depth}
%\DeclareMathOperator{\Coker}{Coker}
%\DeclareMathOperator{\NF}{NF}
%\DeclareMathOperator{\pd}{pd}
%\DeclareMathOperator{\SL}{SL}
%\DeclareMathOperator{\SO}{SO}
%\DeclareMathOperator{\Ort}{O}
%\DeclareMathOperator{\Spez}{Sp}
%\DeclareMathOperator{\PSL}{PSL}
%\DeclareMathOperator{\wdim}{wdim}
%\DeclareMathOperator{\cdim}{cdim}
%\DeclareMathOperator{\cha}{char}
%\DeclareMathOperator{\trdeg}{trdeg}
%\DeclareMathOperator{\codim}{codim}
%\DeclareMathOperator{\kdim}{kdim}
%\DeclareMathOperator{\height}{height}
%\DeclareMathOperator{\Ass}{Ass}
%\DeclareMathOperator{\Lie}{Lie}
\renewcommand{\labelenumi}{(\arabic{enumi})}
\newcommand{\Ndash}{\nobreakdash--}% for pages 1\Ndash 9
\newcommand{\somespace}{\hfill{}\\ \vspace{-0.7cm}}

%%%theosdefinitionen
%newcommand{\gm}{\mathfrak m}
%newcommand{\integer}{\Z}
%newcommand{\proj}
%newcommand{\complex}{\C}
%newcommand{\real}{\mathbb R}
%newcommand{\gp}{\mathfrak p}
%newcommand{\gq}{\mathfrak q}
%newcommand{\scr}{\cal}
%newcommand{\openF}{\F}
%\newcommand{\CC}{\mathbb C}
%newcommand{\ZZ}{\Z}
%newcommand{\QQ}{\Q}
%newcommand{\FF}{\F}

%%%%%%%%%%%%%%%BIBLIOGRAPHY
%newcommand{\by}{}
%newcommand{\paper}{: \begin{it}}
%newcommand{\jour }{, \end{it}}
%newcommand{\vol}{}
%newcommand{\pages}{}
%newcommand{\yr}{}
%\newcommand{\endref}{}
% Local Variables:
% mode: latex
% TeX-master: "tot"
% End:
%%%%%%%%%%%%%%%%%%%%%%%%%%%%%%%%%%%%%%%%%
%%     HOLGERs S�tze
%%%%%%%%%%%%%%%%%%%%%%%%%%%%%%%%%%%%%%%%%


\maketitle

\begin{abstract}
In this chapter we explain constructive methods for computing
the cohomology of a sheaf on a projective variety. We also
give a construction for the Beilinson monad, a tool for
studying the sheaf from partial knowledge of its cohomology.
Finally, we give some examples illustrating the use of the Beilinson
monad. 
\end{abstract}

\section{Introduction}

%\textbf{EISENBUD}

In this chapter $V$ denotes a vector space of finite dimension $n+1$ over
a field $K$ with dual space $W=V^*$, and $S=\sym_K(W)$ is the symmetric
algebra of $W$, isomorphic to the polynomial ring on a basis for $W$.
We write $E$ for the \ie{exterior algebra} on $V$. We grade
$S$ and $E$ by taking elements of $W$ to have
degree 1, and elements of $V$ to have degree $-1$.
We denote the projective space of 1-quotients of $W$
(or of lines in $V$) by $\P^n = \P(W)$.

Serre's sheafification functor $M\mapsto \tilde M$ 
allows one to consider a coherent sheaf on $\P(W)$ as an
equivalence class of 
finitely generated graded $S$-modules, where we identify two such modules
$M$ and $M'$ if, for some $r$, the truncated modules $M_{\geq r}$ and
$M'_{\geq r}$ are isomorphic. A free resolution of $M$,
sheafified, becomes a resolution of $\tilde M$ by sheaves that
are direct sums of line bundles on $\P(W)$ -- that is, a
description of $\tilde M$ in terms of homogeneous matrices
over $S$. Being able to compute
syzygies over
$S$ one can compute the cohomology of $\tilde M$ 
starting from the minimal free resolution of $M$ (see 
\cite{EA:MR1484973:eisenbud}, \cite{smith}
and Remark \ref{cohold} below).

The \ie{Bernstein-Gel'fand-Gel'fand correspondence} (\ie{BGG}) is an isomorphism
between the derived category of bounded complexes of finitely
ge\-nerated $S$-modules and the derived category of 
bounded complexes of finitely generated $E$-modules
or of certain ``\ie{Tate resolution}s'' of $E$-modules.
In this chapter we show how to effectively compute
the Tate resolution $\TT(\mathcal F)$
associated to a sheaf $\mathcal F$, and we use this construction
to give relatively cheap computations of the cohomology of $\mathcal F$.

It turns out that by applying a simple functor to the Tate resolution $\TT(\mathcal F)$
one gets a finite complex of sheaves  whose homology is the sheaf $\mathcal F$
itself. This complex is called a
{\it Beilinson monad\/}
\index{Beilinson monad} \index{monad!Beilinson}
for $\mathcal F$. The Beilinson monad
provides a powerful method for getting information about a sheaf from 
partial knowledge of its cohomology.
It is a representation of the sheaf in terms  of direct sums of
(suitably twisted) bundles of differentials and homomorphisms  between these
bundles, which are given by homogeneous matrices over $E$. 

The following recipe for computing the cohomology of a sheaf
is typical of our methods: Suppose that
$\F=\tilde M$ is the coherent sheaf on $\P(W)$ associated to a
finitely generated graded $S$-module $M=\oplus M_i$. To
compute the cohomology of $\F$ we consider a
sequence of free $E$-modules and maps
$$
\FF(M):\quad \cdots \rTo F^{i-1}\rTo^{\phi_{i-1}} 
F^i\rTo^{\phi_{i}} 
F^{i+1}\rTo\cdots.
$$
Here we set $F^i=M_i \otimes_K E$ and define $\phi_i:F^i\rTo F^{i+1}$ to be the
map taking $m\otimes 1\in M_{i}\otimes_K E$ to
$$
\sum_j x_jm\otimes e_j\in M_{i+1}\otimes V\subset F^{i+1},
$$
where $\{x_j\}$ and $\{e_j\}$ are dual bases of $W$ and $V$
respectively. It turns out that $\FF(M)$
is a complex; that is, $\phi_i\phi_{i-1}=0$ for
every $i$ (the reader may easily check this by direct
computation; a proof without indices is given in
\cite{EA:eis-sch:sheaf}). %Eisenbud and Schreyer [2000].
If we regard $M_i$ as a vector space concentrated
in degree $i$, so that $F^i$ is a direct sum of copies of $E(-i)$,
then these maps are homogeneous of degree 0.

We shall see that if $s$ is a sufficiently large integer then
the truncation of the Tate resolution
$$
F^s\rTo^{\phi_s} F^{s+1}\rTo\cdots
$$ 
is exact and is thus the minimal injective
resolution of the finitely generated graded $E$-module
$P_s=\ker \phi_{s+1}$. (In fact any value of $s$ greater
than the 
Castelnuovo-Mumford regularity of $M$ will do.) 

Because the number of monomials 
in $E$ in any given degree is small compared to the number of monomials of 
that degree in the symmetric algebra, it is relatively cheap to compute a free
resolution of 
$P_s$ over $E$, and thus to 
compute the graded vector
spaces $\Tor^E_t(P_s,K)$. Our algorithm exploits the fact, proved in
\cite{EA:eis-sch:sheaf}, %Eisenbud and Schreyer [2000],
that the $j^\th$ cohomology $\H^j\F$ of $\F$ 
in the Zariski topology is isomorphic to the degree $-n-1$ part of 
$\Tor^E_{s-j}(P_s,K)$; that is,
$$
\H^j\F\cong\Tor^E_{s-j}(P_s,K)_{-n-1}.
$$
In addition, the linear parts of the matrices
in the complex $\TT(\mathcal F)$ determine the
graded
$S$-modules
$$
\H^j_*\mathcal F := \oplus_{i\in\mathbb Z}\, \H^j  \mathcal F(i)\ .
$$
In many cases this is the fastest known method for computing cohomology.

Section 2 of this paper is devoted to a sketch of the
Eisenbud-Fl{\o}ystad-Schreyer approach
to the Bernstein-Gel'fand-Gel'fand correspondence, and 
the computation of cohomology, together with \Mtwo programs
that carry it out, is explained in Section 3. 


The remainder of this paper is devoted to an explanation of
the Beilinson monad, how to compute it in \Mtwo, and what it
is good for.
This technique has played an important role in the construction
and study of vector bundles and varieties. In the typical application
one constructs or classifies monads in order to  construct or classify
sheaves. 

The BGG correspondence and Beilinson's monad were originally
formulated in the language of derived categories, and the
proofs were rather complicated.
The ideas of Eisenbud-Fl{\o}ystad-Schreyer exposed above allow, 
for the first time, an
explanation of these matters on a level that can be understood by 
an advanced undergraduate.

The Beilinson monad is similar in spirit to the technique of
free re\-solutions. That theory essentially describes arbitrary
sheaves by comparing them with direct sums of line bundles.
In the Beilinson technique, one uses a different set of 
``elementary" sheaves, direct sums of exterior powers of
the tautological sub-bundle. Beilinson's remarkable observation
was that this comparison has a much more direct connection
with cohomology than does the free resolution method.

Sections 4 and 5 are introductory in nature. In
Section 4 we begin with a preparatory discussion of the
necessary vector bundles
on projective space and their cohomology. In Section 5
we define monads, a generalization of resolutions. We give
a completely elementary account which constructs the
Beilinson monad in a very special case, following ideas
of Horrocks, and we use this to sketch part of one of the 
first striking applications of monads: the classification of stable
rank 2 vector bundles on the projective plane by
Barth, Hulek and Le Potier.

In Section 6 we give the construction of Eisenbud-Fl{\o}ystad-Schreyer
for the Beilinson monad in general. This is quite suitable for
computation, and we give \Mtwo code that does this job.

A natural question for the student at this point is:
``Why should I bother learning Beilinson's theorem, what is it good
for?" In section 7, we describe two more explicit applications
of the theory developed. In the  first, the classification of
elliptic conic bundles in $\P^4$, computer algebra played 
a  significant role, demonstrating that several published papers
contained serious mistakes by constructing an example
they had excluded! Using the routines developed earlier
in the chapter we give a simpler account of the
crucial computation.

In the second application, the construction of abelian surfaces in
$\P^4$ and the related Horrocks-Mumford bundles, computer algebra
allows one to greatly shorten some of the original arguments made.
As the reader will see, everything follows easily
with computation, once a certain $2\times 5$ matrix of exterior 
monomials, given by Horrocks and Mumford, has been written down.
One might compare the computations here with the original paper of Horrocks and
Mumford \cite{EA:HM} (for the cohomology) and the papers by 
Manolache \cite{EA:nico} and Decker \cite{EA:cam} (for the
syzygies) of the Horrocks-Mumford bundle. A great deal of effort, using
representation theory, was necessary to derive results that can be
computed in seconds using the \Mtwo programs here. Much more theoretical
effort, however, is needed to derive classification results.

Another application of the construction of the Beilinson
complex (in a slightly more general setting) is to compute
\ie{Chow form}s of varieties; see
\cite{EA:Eisenbud-Schreyer:ChowForms}.

Perhaps the
situation is similar to that in the beginning of the 1980's when it
became clear that syzygies could be computed by a machine. Though
syzygies had been used theoretically for many years it took quite a
while until the practical computation of syzygies lead to applications,
too, mostly through the greatly increased ability to study examples. 

A good open problem of this sort is to extend
and make more precise the very useful criterion given in \ref{critsur}:
Can the reader find a necessary and sufficient condition
to replace the necessary condition for surjectivity given there?
How about a criterion for exactness?

\section{Basics of the Bernstein-Gel'fand-Gel'fand Correspondence}

\index{Bernstein-Gel'fand-Gel'fand correspondence}
In this section we describe the basic idea of the BGG correspondence,
introduced in \cite{EA:MR80c:14010a}. %Bernstein, Gel'fand, and Gel'fand [1978]. 
For a more complete treatment along the lines given here, see the first section
of \cite{EA:eis-sch:sheaf}. %Eisenbud and Schreyer [2000] .

As a simple example of the construction given in
%%% Referenz
Section~1, consider the case $M=S=\sym_K(W)$. The associated 
complex, made from the homogeneous components $\sym_i(W)$ of $S$, 
has the form
$$
\FF(S):\quad E\rTo W\otimes E \rTo \Sym_2(W)\otimes E\rTo\cdots,
$$
where we regard $\Sym_iW$ as concentrated in degree $i$.
It is easy to see that the kernel of the first map,
$E\rTo W\otimes E$, is exactly the socle $\bigwedge^{n+1} V\subset E$,
which is a 1-dimensional vector space concentrated in degree $-n-1$.
In fact $\FF(S)$ is the minimal injective resolution of
this vector space.  If we tensor with the dual vector space
$\bigwedge^{n+1} W$ (which is concentrated in degree $n+1$),
we obtain the minimal injective resolution of the
vector space $\bigwedge^{n+1} W\otimes \bigwedge^{n+1} V$,
which may be identified canonically
with the residue field $K$ of $E$.
This resolution is called the {\it Cartan resolution\/}
\index{Cartan resolution}
of $K$.  To write it conveniently, we
set $\omega_E=\bigwedge^{n+1}W\otimes E$. The socle of
$\omega_E$ is $K$. Since $E$ is
injective (as well as projective) as an $E$-module, 
the same goes for $\omega_E$, so $\omega_E$ is the
injective envelope of the residue class field $K$ and
we have $\omega_E=\Hom_K(E,K)$. Thus we
can write the injective resolution of the residue field as
$$
\RR(S):\quad \omega_E\rTo W\otimes \omega_E \rTo 
\Sym_2(W)\otimes \omega_E \rTo\cdots,
$$
or again as
$$
\Hom_K(E,K)\rTo \Hom_K(E,W) \rTo 
\Hom_K(E,\Sym_2(W)) \rTo\cdots.
$$
Taking our cue from this situation, 
our primary object of study in the case of
an arbitrary finitely generated graded $S$-module 
$M = \oplus M_i$ will be the complex
$$
\RR(M):\quad \cdots \rTo M_i\otimes \omega_E
\rTo M_{i+1}\otimes\omega_E\rTo \cdots,
$$ 
which will have a more natural grading than $\FF(M)$;
in any case, it differs from $\FF(M)$
only by tensoring over $K$ with the one-dimensional
$K$-vector space $\bigwedge^{n+1}W$, concentrated in degree $n+1$, and
thus has the same basic properties. 
(Writing $\RR(M)$ in terms of 
$\Hom$ as above suggests that the functor $\RR$
might have a left adjoint, and
indeed there is a left adjoint that produces linear free
complexes over $S$ from graded $E$-modules. $\RR$ and its
left adjoint are used to construct the isomorphisms of 
derived categories in the BGG correspondence; see
\cite{EA:eis-sch:sheaf} %Eisenbud and Schreyer [2000]
for a treatment in this spirit.)

An important fact for us is that the complex $\RR(M)$ is eventually 
exact (and thus 
$$
F^i\rTo^{\phi_i} F^{i+1}\rTo\cdots
$$
is the minimal injective resolution of $\ker \phi_i$ when $i \gg 0$).
It turns out that the point at which exactness sets in is
a well-known invariant, the \ie{Castelnuovo-Mumford regularity} of
$M$, whose definition we briefly recall:

If $M=\oplus M_i$ is a finitely generated graded $S$-module
then for all large integers $r$
the submodule $M_{\geq r}\subset M$ is generated in degree $r$ and has
a {\it linear free resolution\/}; 
\index{linear free resolution}
that is, its first syzygies are
generated in degree $r+1$, its second syzygies in degree $r+2$,
etc. (see \cite[chapter 20]{EA:MR97a:13001}%Eisenbud [1995]
).  The {\it Castelnuovo-Mumford regularity\/} of $M$ is the
least integer $r$ for which this occurs.

\begin{theorem}[\cite{EA:eis-sch:sheaf}]\label{regularity}
% \vskip0.3cm
% \noindent
% \textbf{Theorem 3.1.}
%\theorem{regularity} 
% {\em 
Let $M$ be a finitely generated graded
$S$-module of Cas\-tel\-nuovo-Mumford regularity $r$.
The complex $\RR(M)$ is exact at $\Hom_K(E,M_i)$
for all $i\geq s$ if and only if $s>r$.\qed
%}
%\vskip0.3cm
\end{theorem}

More generally, it is shown in \cite{EA:eis-sch:sheaf} that the components of the 
cohomology of the complex $\RR(M)$ can be identified with the Koszul cohomology
of $M$. An equivalent result was stated in
\cite{EA:MR89g:13005:appendix}. %Buchweitz [1985] .

For instance, it is not hard to show that if 
$M$ is of finite length, then the regularity of $M$
is the largest $i$ such that $M_i\neq 0$. Let us verify Theorem
\ref{regularity} directly in a simple example:

%\begin{Example}
\begin{example}
Let $S=K[x_0,x_1,x_2]$, and let
$M=S/(x_0^2,x_1^2,x_2^2)$. The module
$M_{\geq 3}= K\cdot x_0x_1x_2$ is a trivial $S$-module, and
its resolution is the Koszul complex on $x_0$, $x_1$ and $x_2$, which is linear.
Thus the Castelnuovo-Mumford regularity of $M$
is $\leq 3$. On the other hand $M_{\geq 2}$ is, up to twist,
isomorphic to the dual of $S/(x_0,x_1,x_2)^2$, and it follows that
the resolution of $M_{\geq 2}$ has the form
$$
0\rTo S(-6)\rTo 6\:\! S(-4)\rTo 8\:\! S(-3)\rTo 3\:\! S(-2),
$$
which is not linear, so the Castelnuovo-Mumford regularity of
$M$ is exactly 3. Note that the regularity is larger than the degrees
of the generators and relations of $M$---in general it
can be much larger.

Over $E$ the linear free complex 
corresponding to $M$ has the form
$$
\cdots\to  0\to
M_0\otimes\omega_E
\to 
M_1\otimes\omega_E
\to 
M_2\otimes\omega_E
\to
M_3\otimes\omega_E
\to 0\to
\cdots,
$$
where all the terms not shown are 0. Using the isomorphism
$\omega_E\cong E(-3)$ this
can be written (non-canonically) as
$$
0\rTo E(-3)
\rTo^{ \begin{pmatrix} e_0\\ e_1\\ e_2 \end{pmatrix}} 
3\:\! E(-2)
\rTo^{\begin{pmatrix} 0&e_2&e_1\\ e_2&0&e_0\\ e_1&e_0&0 \end{pmatrix}} 
3\:\! E(-1)
\rTo^{\begin{pmatrix}e_0& e_1& e_2 \end{pmatrix}} 
E\rTo 0.
$$
One checks easily that this complex is inexact at  every non-zero term
(despite its resemblance to a Koszul complex),
verifying Theorem \ref{regularity}.\qed
\end{example}
%\end{example}

Another case in which everything can be checked directly
occurs when $M$ is the homogeneous coordinate ring of a point:


%\begin{Example}
\begin{example} 
Take $M=S/I$ where $I$ is generated
by a codimension 1 space of linear forms in $W$, so that 
$I$ is the homogeneous ideal of a point $p\in \P(W)$.
The free resolution of $M$ is 
the Koszul complex on $n$ linear forms,
so $M$ is 0-regular. As $M_i$
is 1-dimensional for every $i$ the terms of the complex
$\RR(M)$ are all rank 1 free $E$-modules. One
easily checks that $\RR(M)$ takes the form
$$
\RR(M):\quad \omega_E\rTo^a \omega_E(-1)\rTo^a\omega_E(-2)\rTo^a\cdots\ ,
$$
where $a\in V = W^*$ is a linear functional that vanishes on all
the linear forms in $I$; that is, $a$ is a generator of 
the one-dimensional subspace of $V$ corresponding to the point $p$.
As for any linear form in $E$, the annihilator of $a$ 
is generated by $a$, and it follows directly that the complex
$\RR(M)$ is acyclic in this case.\qed
\end{example}
%\end{example}


We present two \Mtwo functions, {\texttt{symExt}} and 
{\texttt{bgg}}, which compute a differential of the complex
$\RR(M)$ for a finitely generated graded module $M$ defined over 
some polynomial ring $S=K[x_0,\dots,x_n]$ with variables $x_i$ of 
degree 1. Both functions expect as an additional input the name of an 
exterior algebra $E$ with the same number $n+1$ of generators, also 
supposed to be of degree 1 (and NOT -1). This convention, which
makes the cohomology diagrams more naturally looking when printed in 
\Mtwo, necessitates the adjustment of degrees in the second half of 
the programs.

The first of the functions, {\texttt{symExt}}, takes as input a
matrix $m$ with linear entries, which we think of as
a presentation matrix for a positively graded $S$-module  
$M = \oplus _{i\geq 0} M_i$, and 
returns a matrix represen\-ting the map
$M_{0}\otimes\omega_E\to M_{1}\otimes\omega_E$
which is the first differential of the complex $\RR(M)$. 
\vskip0.3cm

\beginOutput
i1 : symExt = (m,E) ->(\\
\          ev := map(E,ring m,vars E);\\
\          mt := transpose jacobian m;\\
\          jn := gens kernel mt;\\
\          q  := vars(ring m)**id_(target m);\\
\          ans:= transpose ev(q*jn);\\
\          --now correct the degrees:\\
\          map(E^\{(rank target ans):1\}, E^\{(rank source ans):0\}, \\
\              ans));\\
\endOutput

\vskip0.3cm
\noindent
If $M$ is a module whose presentation is not linear
in the sense above, we can still
apply {\texttt{symExt}} to a high truncation of $M$:
\vskip0.3cm

\beginOutput
i2 : S=ZZ/32003[x_0..x_2];\\
\endOutput
\beginOutput
i3 : E=ZZ/32003[e_0..e_2,SkewCommutative=>true];\\
\endOutput
\beginOutput
i4 : M=coker matrix\{\{x_0^2, x_1^2\}\};\\
\endOutput
\beginOutput
i5 : m=presentation truncate(regularity M,M);\\
\emptyLine
\             4       8\\
o5 : Matrix S  <--- S\\
\endOutput
\beginOutput
i6 : symExt(m,E)\\
\emptyLine
o6 = \{-1\} | e_2 e_1 e_0 0   |\\
\     \{-1\} | 0   e_2 0   e_0 |\\
\     \{-1\} | 0   0   e_2 e_1 |\\
\     \{-1\} | 0   0   0   e_2 |\\
\emptyLine
\             4       4\\
o6 : Matrix E  <--- E\\
\endOutput

\vskip0.3cm
\noindent
The function
{\texttt{symExt}} is a quick-and-dirty tool which requires little
computation. If it is called on two successive truncations
of a module the maps it produces may NOT compose to zero
because the choice of bases is not consistent.
The second function, {\texttt{bgg}}, makes the computation in such a way
that the bases are consistent, but does more computation
to achieve this end. It takes as input an integer $i$ and 
a finitely generated graded $S$-module  $M$, and 
returns the $i^{\text{th}}$ map in $\RR (M)$,
which is an ``adjoint'' of the multiplication
map between $M_i$ and $M_{i+1}$. 
\vskip0.3cm

\beginOutput
i7 : bgg = (i,M,E) ->(\\
\          S :=ring(M);\\
\          numvarsE := rank source vars E;\\
\          ev:=map(E,S,vars E);\\
\          f0:=basis(i,M);\\
\          f1:=basis(i+1,M);\\
\          g :=((vars S)**f0)//f1;\\
\          b:=(ev g)*((transpose vars E)**(ev source f0));\\
\          --correct the degrees (which are otherwise\\
\          --wrong in the transpose)\\
\          map(E^\{(rank target b):i+1\},E^\{(rank source b):i\}, b));\\
\endOutput

\vskip0.3cm
\noindent
For instance, in Example 2.2:
\vskip0.3cm

\beginOutput
i8 : M=cokernel matrix\{\{x_0^2, x_1^2, x_2^2\}\};\\
\endOutput
\beginOutput
i9 : bgg(1,M,E)\\
\emptyLine
o9 = \{-2\} | e_1 e_0 0   |\\
\     \{-2\} | e_2 0   e_0 |\\
\     \{-2\} | 0   e_2 e_1 |\\
\emptyLine
\             3       3\\
o9 : Matrix E  <--- E\\
\endOutput

\vskip0.3cm

%\textbf{EISENBUD ENDE}

%{\bf 4. Cohomology and Tate}



%\textbf{EISENBUD}



\section{The Cohomology and the Tate Resolution of a Sheaf}

\index{sheaf cohomology}\index{cohomology!sheaf}
Given a finitely generated graded $S$-module 
$M$ we construct a (doubly infinite)
$E$-free complex $\TT(M)$ with vanishing homology,
called the {\it Tate resolution}
\index{Tate resolution} 
of $M$, as follows:
Let $r$ be the Castelnuovo-Mumford regularity of $M$. The
truncation $\TT^{>r}(M)$, the part of $\TT(M)$ with cohomological
degree $>r$, is $\RR(M_{>r})$. We complete this to an exact
complex by adjoining a minimal projective resolution of the kernel
of $\Hom_K(E,M_{r+1})\to\Hom_K(E,M_{r+2})$. 

If, for example, $M$ has finite length as in Example 2.2, the Tate 
resolution of $M$ is the complex
$$
\cdots\to 0\to 0\to 0 \to\cdots.
$$
At the opposite extreme, take $M=S$, the free module of rank 1.
Since $S$ has regularity 0, it follows that $\RR(S)$ is an injective 
resolution of the residue field $K$ of $E$. Applying the exact functor
$\Hom_K(\hbox{\bf---}, K)$, and using the fact that it carries
$\omega_E=\Hom_K(E,K)$ back to $E$, we see that the 
Tate resolution $\TT(S)$ is the first row of the diagram
$$
\xymatrix{
\cdots \ar[r]& W^*\otimes E \ar[r]& E\ar[dr] \ar[rr]&& \omega_E
\ar[r]& W\otimes\omega_E  \ar[r]&\cdots\\
&&&K\ar[ur]
}
$$      
Another simple example occurs in the case where $M$ is
the homogeneous coordinate ring of a point $p\in\P(W)$. The complex
$\RR(M)$ constructed in Example 2.3 is periodic, so
it may be simply continued to the left, giving
$$
\TT(M):\quad \cdots \rTo^a \omega_E(i)
\rTo^a\omega_E(i-1)\rTo^a\cdots,
$$
where again $a\in V=W^*$ is a non-zero linear functional
vanishing on the linear forms in the ideal of $p$.

For arbitrary $M$, by the results of the previous section,
$\RR(M_{>r})$ has no
homology in cohomological degree $>r+1$, so $\TT(M)$ could be constructed
by a similar recipe from any truncation 
$\RR(M_{>s})$ with $s\geq r$. Thus
the Tate resolution  depends only on the sheaf $\tilde M$ 
on $\P(W)$ corresponding to $M$.
We sometimes write $\TT(M)$ as $\TT(\tilde M)$ to
emphasize this point.

Using the \Mtwo function {\texttt{symExt}}
of the last section, one can compute any finite piece of the Tate resolution.
\vskip0.3cm

\beginOutput
i10 : tateResolution = (m,E,loDeg,hiDeg)->(\\
\           M := coker m;\\
\           reg := regularity M;\\
\           bnd := max(reg+1,hiDeg-1);\\
\           mt  := presentation truncate(bnd,M);\\
\           o   := symExt(mt,E);\\
\           --adjust degrees, since symExt forgets them\\
\           ofixed   :=  map(E^\{(rank target o):bnd+1\},\\
\                      E^\{(rank source o):bnd\},\\
\                      o);\\
\           res(coker ofixed, LengthLimit=>max(1,bnd-loDeg+1)));\\
\endOutput

\vskip0.3cm
\noindent
{\texttt{tateResolution}} takes as input a presentation matrix $m$ of a 
finitely generated graded module $M$ defined 
over some polynomial ring $S=K[x_0,\dots,x_n]$ with variables $x_i$ of 
degree 1, the name of an exterior algebra $E$ with the same number 
$n+1$ of generators, also supposed to be of degree 1, and two integers,
say $l$ and $h$. If $r$ is the regularity of $M$, then
{\tt{tateResolution(m,E,l,h)}} computes the piece
$$
\TT^{\,l}(M)\to \dots \to \TT^{\,\max(r+2,h)}(M)
$$
of $\TT(M)$. For instance, for the homogeneous coordinate ring of a point in
the projective plane:
\vskip0.3cm

\beginOutput
i11 : m = matrix\{\{x_0,x_1\}\};\\
\emptyLine
\              1       2\\
o11 : Matrix S  <--- S\\
\endOutput
\beginOutput
i12 : regularity coker m\\
\emptyLine
o12 = 0\\
\endOutput
\beginOutput
i13 : T = tateResolution(m,E,-2,4)\\
\emptyLine
\       1      1      1      1      1      1      1\\
o13 = E  <-- E  <-- E  <-- E  <-- E  <-- E  <-- E\\
\                                                 \\
\      0      1      2      3      4      5      6\\
\emptyLine
o13 : ChainComplex\\
\endOutput
\beginOutput
i14 : betti T\\
\emptyLine
o14 = total: 1 1 1 1 1 1 1\\
\         -4: 1 1 1 1 1 1 1\\
\endOutput
\beginOutput
i15 : T.dd_1\\
\emptyLine
o15 = \{-4\} | e_2 |\\
\emptyLine
\              1       1\\
o15 : Matrix E  <--- E\\
\endOutput

\vskip0.3cm

For arbitrary $M$ we have $M_i=\H^0\tilde M(i)$ for
large $i$, so the correspon\-ding term of the complex 
$\TT(\tilde M)$ with cohomological degree $i$ is 
$M_i\otimes \omega_E=\H^0(\tilde M(i))\otimes \omega_E$. 
The following result generalizes 
this to a description of all the terms of the Tate resolution,
and gives the formula for the cohomology described in the
introduction.

\begin{theorem}[\cite{EA:eis-sch:sheaf}]\label{tate}
%\vskip0.3cm
%\noindent
%\textbf{Theorem 4.1.}
%{\em 
Let $M$ be a finitely generated graded $S$-module.
The term of the complex $\TT(M)=\TT(\tilde M)$ with cohomological degree $i$ is 
$$
\oplus_j \H^j\tilde M(i-j)\otimes \omega_E \ ,
$$
where $\H^j\tilde M(i-j)$ is regarded as a vector space
concentrated in degree $i-j$, so that the summand
$\H^j\tilde M(i-j)\otimes \omega_E$ is isomorphic to a direct sum
of copies of $\omega_E(j-i)$.
Moreover the subquotient complex 
$$
\cdots \to \H^j\tilde M(i-j)\otimes \omega_E
\to
\H^j\tilde M(i+1-j)\otimes \omega_E
\to \cdots
$$
is 
$\RR(\H^j_*(\tilde M(-j))) (j)$ (up to twists and shifts it is
$\RR(\H^j_* \tilde M).$ )\qed
%}
\end{theorem}

%\vskip0.3cm
Thus each cohomology group of each twist of the sheaf
$\tilde M$ occurs (exactly once) in
a term of $\TT(M)$. When we compute a part of $\TT(M)$,
we are computing the sheaf cohomology of various twists
of the associated sheaf together with maps
which describe the $S$-module structure of $\H^j_*\tilde M$
in the sense that  the linear maps in this complex
are adjoints of the multiplication maps that determine the
module structure (the multiplication maps themselves could
be computed by a function similar to {\texttt{bgg}}). 
The higher degree maps in the complex $\TT(M)$
determine certain higher cohomology operations, which we understand 
only in very special cases (see \cite{EA:Eisenbud-Schreyer:ChowForms}).
 
If $M = \coker\, m$, then {\tt{betti tateResolution(m,E,l,h)}} prints
the dimensions $\h^j \tilde M(i-j) = \dim\,\H^j \tilde M(i-j)$ for
$\max(r+2, h) \geq i\geq l$, where $r$ is the regularity of $M$.
Truncating the Tate resolution if necessary allows one to restrict 
the size of the output.
\index{sheaf cohomology}\index{cohomology!sheaf}
\vskip0.3cm

\beginOutput
i16 : sheafCohomology = (m,E,loDeg,hiDeg)->(\\
\           T := tateResolution(m,E,loDeg,hiDeg);\\
\           k := length T;\\
\           d := k-hiDeg+loDeg;\\
\           if d > 0 then \\
\              chainComplex apply(d+1 .. k, i->T.dd_(i))\\
\           else T);\\
\endOutput


\vskip0.3cm
\noindent
The expression
{\tt{betti sheafCohomology(m,E,l,h)}} prints a cohomology table for
$\tilde M$ of the form
$$
{\setlength{\arraycolsep}{.4cm}
\begin{array}{llcl}
\h^0\tilde M(h)& \dots &\h^0\tilde M(l)\\
\h^1\tilde M(h-1)& \dots & \h^1\tilde M(l-1)\\
\hfil\vdots&&\hfil\vdots\\
\h^n\tilde M(h-n)& \dots &\h^n\tilde M (l-n)\ .
\end{array}}
$$
As a simple example we consider the cotangent bundle on projective 
3-space (see the next section for the Koszul resolution of this bundle):
\vskip0.3cm

\beginOutput
i17 : S=ZZ/32003[x_0..x_3];\\
\endOutput
\beginOutput
i18 : E=ZZ/32003[e_0..e_3,SkewCommutative=>true];\\
\endOutput
\vskip0.1cm

\noindent
The cotangent bundle is the cokernel of the third
differential of the Koszul complex on the variables of $S$.

\vskip0.1cm
\beginOutput
i19 : m=koszul(3,vars S);\\
\emptyLine
\              6       4\\
o19 : Matrix S  <--- S\\
\endOutput
\beginOutput
i20 : regularity coker m\\
\emptyLine
o20 = 2\\
\endOutput
\beginOutput
i21 : betti tateResolution(m,E,-6,2)\\
\emptyLine
o21 = total: 45 20 6 1 4 15 36 70 120 189 280\\
\         -4: 45 20 6 . .  .  .  .   .   .   .\\
\         -3:  .  . . 1 .  .  .  .   .   .   .\\
\         -2:  .  . . . .  .  .  .   .   .   .\\
\         -1:  .  . . . 4 15 36 70 120 189 280\\
\endOutput
\beginOutput
i22 : betti sheafCohomology(m,E,-6,2)\\
\emptyLine
o22 = total: 6 1 4 15 36 70 120 189 280\\
\         -2: 6 . .  .  .  .   .   .   .\\
\         -1: . 1 .  .  .  .   .   .   .\\
\          0: . . .  .  .  .   .   .   .\\
\          1: . . 4 15 36 70 120 189 280\\
\endOutput
\vskip0.1cm
\noindent
Of course these two results differ only in the precise point of
truncation.

\index{sheaf cohomology}\index{cohomology!sheaf}
\begin{remark}\label{cohold}  There is also a built-in sheaf cohomology
function {\texttt{HH}} in \Mtwo which is based on the algorithms in 
\cite{EA:MR1484973:eisenbud}.  These algorithms are often much 
slower than {\texttt{sheafCohomology}}. 
To access it, first execute
\begin{verbatim}
     M=sheaf coker m;
\end{verbatim}
\noindent
and pick integers $j$ and $d$. Then
\begin{verbatim}
     HH^j(M(>=d))
\end{verbatim}
\noindent
returns the truncated $j^{\text{th}}$ cohomology module 
$\H^j_{i\geq d} \tilde M$.  In the above example of the cotangent
bundle $\mathcal F$ on projective 3-space we obtain the
Koszul presentation of $H^1 \mathcal F \cong K$ considered as an $S$-module 
sitting in degree 0:

\vskip0.3cm

\beginOutput
i23 : M=sheaf coker m;\\
\endOutput
\beginOutput
i24 : HH^1(M(>=0))\\
\emptyLine
o24 = cokernel | x_3 x_2 x_1 x_0 |\\
\emptyLine
\                             1\\
o24 : S-module, quotient of S\\
\endOutput
\qed
\end{remark}

The Tate resolutions of sheaves are, as the reader may easily check,
precisely the doubly infinite, graded, exact complexes of finitely-generated
free $E$-modules which are ``eventually linear'' on the right,
in an obvious sense. What about other doubly exact graded free
complexes? For example what if we take the dual of the Tate
resolution of a sheaf? In general it will not be eventually linear.
What is it?

To explain this we must generalize the construction of $\RR(M)$:
If 
$$
M^\bullet:\quad \cdots\rTo M^{i+1}\rTo M^i\rTo M^{i-1}\rTo\cdots
$$
is a complex of $S$-modules, then applying the functor
$\RR$ gives a complex of free complexes over $E$.
By changing some signs we get a double complex. 
In general the associated total complex is not minimal; but at least if 
$M^\bullet$ is a bounded complex then,
just as one produces the unique minimal free
resolution of a module from any free resolution,
we can construct a unique minimal
complex from it. We call this minimal complex $\RR(M^\bullet)$.
(See \cite{EA:eis-sch:sheaf} %Eisenbud and Schreyer [2000]
for more information. This construction is a necessary part of interpreting
the BGG correspondence as an equivalence of derived categories.)

Again
if $M^\bullet$ is a bounded complex of finitely generated
modules, then as before
one shows
that $\RR(M^\bullet)$ is exact from a certain point on, and
so we can form the Tate resolution $\TT(M^\bullet)$ by adjoining
a free resolution of a kernel. Once again, the Tate resolution 
depends only on the bounded complex of coherent sheaves
$\F^\bullet$ associated
to $M^\bullet$, and we write $\TT(\F^\bullet)=\TT(M^\bullet)$.

A variant of the theorem of Bernstein, Gel'fand and Gel'fand
shows that every minimal graded doubly infinite exact sequence
of finitely generated free $E$-modules is of the form $\TT(\F^\bullet)$
for some complex of coherent sheaves $\F^\bullet$, unique up
to quasi-isomorphism. The terms of the Tate resolution can
be expressed using hypercohomology
by a formula like that of Theorem \ref{tate}.

One way that interesting complexes of sheaves arise is through
duality. 
\index{duality of sheaves}
For simplicity,
write $\O$ for the structure sheaf $\O_{\P(W)}$.
If $\F=\tilde M$ is a sheaf on $\P(W)$ then 
the derived functor $RHom(\F, \O)$ may be computed
by applying the functor $Hom(\hbox{\bf ---},\O)$
to a sheafified free resolution of $M$; it's value is thus
a complex of sheaves rather than an individual sheaf.

We can now identify the dual of the Tate resolution:


\begin{theorem}\label{duality}
%\noindent
%\textbf{Theorem 4.2.}
%{\em 
%\theorem{duality} 
$\Hom_K(\TT(\F), K) \iso \TT(RH{\rm om}(\F, \O))[1]$.\qed
%}
\end{theorem}

Here the $[1]$ denotes a shift by one in cohomological degree.
For example, take $\F=\O$. We have $RH{\rm om}(\O, \O)=\O$. The
Tate resolution is given by
$$
\xymatrix @R=0mm @C=6mm{
\TT(\O):\quad\cdots\ar[r]& E \ar[r]& \omega_E\ar[r]&\cdots\\
&-1&0
}
$$
where the number under each term is its  cohomological degree.
Taking into account $\omega_E=\Hom_K(E,K)$,
the dual of the Tate resolution is thus
$$
\xymatrix @R=0mm{
\Hom_K(\TT(\O),K):\quad\cdots&\ar[l]\omega_E&\ar[l]E&\ar[l]\cdots\\
&1&0
}
$$
which is the same as 
$\TT(\O)[1]$.
A completely analogous computation gives the
proof of Theorem \ref{duality} 
if $\F=\O(a)$ for some $a$, and the general case follows
by taking free resolutions.


%\textbf{EISENBUD ENDE}


\section{Cohomology and Vector Bundles}

\index{vector bundle}
In this section we first recall how vector bundles,
direct sums of line bundles, and  bundles of differentials 
can be characterized among all coherent sheaves on $\PP (W)$ in terms of  
cohomology (as usual we do not distinguish between vector bundles and 
locally free sheaves). Then we describe the homomorphisms between the 
suitably twisted bundles of differentials in terms of the exterior 
algebra $E$. This description plays an important role in the context of 
Beilinson monads.

Vector bundles on $\PP (W)$ are characterized by a criterion of Serre
\index{bundle!Serre's criterion}
\cite{EA:MR16:953c} which can be formulated as follows:
A coherent sheaf $\mathcal F$ on $\PP (W)$ is locally free if and only if 
its module of sections $\H^0_* ({\mathcal F})$ is finitely generated and its 
{\it {intermediate cohomology modules}} \index{cohomology!intermediate}
$\H^j_*\mathcal F$, $1\leq j \leq n-1$, are of finite length.

 From a cohomological point of view, the simplest vector bundles are
the direct sums of line bundles.  Every vector bundle on the projective line
splits into a direct sum of line bundles by Grothendieck's splitting
theorem \index{splitting theorem!of Grothendieck}
(see \cite{EA:MR81b:14001}).  Induction yields Horrocks'
splitting theorem \index{splitting theorem!of Horrocks}
(see \cite{EA:MR80f:14005}): A vector bundle on $\P
(W)$ splits into a direct sum of line bundles if and only if its
intermediate cohomology vanishes (originally, this theorem was proved
as a corollary to a more general result, see \cite{EA:MR30:120} and
\cite{EA:MR99f:14064}).

Just a little bit more complicated are the bundles of differentials.
\index{bundles!of differentials} To fix our notation in this context we write 
$\mathcal O = \mathcal O_{\PP (W)}$, 
$W \otimes \mathcal O$ for the trivial bundle on $\PP (W)$ with fiber $W$,
$U=\Omega_{\P (W)}(1)$ for the cotangent bundle twisted by 1, and
$$
U^i = {\textstyle\bigwedge}^i U = {\textstyle\bigwedge}^i (\Omega_{\P (W)}(1)) = \Omega^i_{\PP (W)}(i)
$$
for the $i^{\text{th}}$ bundle of  differentials twisted by $i$; in particular 
$U^0 = \mathcal O$,  $U^n\cong \mathcal O (-1)$, and $U^i=0$ if $i<0$ or $i>n$.

\begin{remark} \label{dualitybd} For each $0 \leq i \leq n$ the pairing
$$
U^i \otimes U^{n-i} \overset\wedge\longrightarrow U^n \cong \mathcal O (-1)
$$
induces an isomorphism
$$
U^{n-i} \cong (U^{i})^* (-1)\, .\qquad\qed
$$
\end{remark}

The fiber of $U$ at the point of $\PP (W)$ corresponding to the line 
$\langle a \rangle\subset V$ is the subspace $(V/\langle a \rangle)^*\subset W$. 
Thus $U$ fits into the 
short exact sequence
$$
0\rightarrow U \rightarrow W \otimes \mathcal O \rightarrow 
\mathcal O (1) \rightarrow 0\ .
$$
In fact, $U$ is the {\it tautological subbundle\/} 
\index{tautological subbundle}
of $W \otimes \mathcal O$.
Taking exterior powers, we get the short exact sequences
$$
0\rightarrow U^{i+1} \rightarrow {\textstyle\bigwedge}^{i+1}W \otimes \mathcal O \rightarrow 
U^i\otimes \mathcal O (1) \rightarrow 0\ .
$$
Twisting the $i^\th$ sequence by $-i-1$, and gluing them
together we get the exact sequence
$$
\xymatrix@1@C=6mm{
0  \ar[r] & {\bigwedge^{n+1}_{\vbox to 2mm{}}}
W\otimes \mathcal O(-n-1) \ar[r] &\,\cdots\;  \ar[r]& {\bigwedge^{0}_{\vbox to 2mm{}}}
W\otimes \mathcal O \ar[r]& 0\ .
}
$$
This sequence is the sheafification of the Koszul complex, which
is the free resolution of the ``trivial'' graded $S$-module $K=S/(W)$.

\begin{remark} \label{cohbd}
By taking cohomology in the short exact sequences above we find that
$$
\H^j_*U^i =
\begin{cases}
K(i)& j=i,\\
0& j\ne i,
\end{cases}
\qquad 1\leq i,j\leq n-1\ ,
$$
where $K(i)=(S/(W))(i)$. Conversely, every vector bundle $\mathcal F$ on 
$\PP(W)$ with this intermediate cohomo\-logy is {\it {stably equivalent}}
\index{stable equivalence} 
to $U^i$; that is, there exists a direct sum  $\mathcal L$  
of line bundles such that $\mathcal F\cong U^i \oplus\mathcal L.$
This follows by comparing the sheafified Koszul complex with the minimal free resolution
of the dual bundle $\mathcal F^*$.\qed 
\end{remark}

In what follows we describe the homomorphisms bet\-ween the various $U^i$,
$0 \leq i \leq n$. Note that since $U=U^1\subset W\otimes \mathcal O$ each element
of $V=\hom_K(W,K)$ induces a homomorphism $U^1\to U^0$ which is the composite
$$
U^1 \subset W\otimes \mathcal O \to K\otimes\mathcal O = \mathcal O = U^0.
$$
Similarly, using the diagonal map of the exterior algebra
$U^i=\bigwedge^iU \to U\otimes U^{i-1}$, each element of $V$ 
induces a homomorphism $U^i\to U^{i-1}$ which is the composite
$$
U^i \to U\otimes U^{i-1} \to W\otimes U^{i-1} \to K\otimes U^{i-1} = U^{i-1}.
$$
It is not hard to show that these maps induced by elements of $V$
anticommute with each other
(see for example \cite[A2.4.1]{EA:MR97a:13001}).
Thus we get maps $\bigwedge^jV\to \hom(U^i, U^{i-j})$
which together give a graded ring  homomorphism
$\bigwedge V \to \hom(\oplus_i U^i, \oplus_i U^i)$.
In fact this construction gives
all the homomorphisms between the $U^i$:
\noindent
\begin{lemma} \label{hombd} The maps
$$
{\textstyle{\textstyle\bigwedge}}^jV\to \hom(U^i, U^{i-j}),\quad 0 \leq i, i-j \leq n\, ,
$$
described above are isomorphisms. Under these isomorphisms an element
$e\in\bigwedge^jV$ acts by contraction on the fibers of the $U^i$:
$$
\xymatrix{
\bigwedge^i(V/\langle a \rangle)^*\;\ar[d]\ar@{^(->}[r]& \bigwedge^{i} W \ar[d]^{e}\\
\bigwedge^{i-j}(V/\langle a \rangle)^*\;\ar@{^(->}[r] &\bigwedge^{i-j} W\ .
}
$$
\end{lemma}

\begin{proof} Every homomorphism  
$U^i \rightarrow U^{i-j}$ lifts uniquely to a
homomorphism between shifted Koszul complexes:
$$
\xymatrix@1@C=4mm{
0  \ar[r] & {\bigwedge^{n+1}_{\vbox to 2mm{}}}
W\otimes \mathcal O(i-n-1) \ar[d]\ar[r] &\,\cdots\; \ar[r]& {\bigwedge^{j}_{\vbox to 2mm{}}}
W\otimes \mathcal O (i-j) \ar[d]\ar[r]&\cdots\\
 \cdots\,  
\ar[r] & {\bigwedge^{n+1-j}_{\vbox to 2mm{}}}W\otimes \mathcal O(i-n-1) 
\ar[r] &\,\cdots\;  \ar[r]& \mathcal O (i-j) \ar[r]& 0
}
$$
Indeed, the corresponding obstructions vanish by Remarks \ref{dualitybd} and 
\ref{cohbd}. All results follow since the vertical arrows are necessarily 
given by contraction with an element in
$$
\Hom ({\textstyle\bigwedge}^{j} W \otimes \mathcal O (i-j), \mathcal O (i-j))\cong  {\textstyle\bigwedge}^jV\, .\quad\qed
$$
\end{proof} 


In practical terms, these results say that a map
$U^i \overset{e}\longrightarrow U^{i-j}$ is represented as
$$
\xymatrix{
\bigwedge^{i+1} W \otimes \mathcal O (-1) \ar[d]^e \ar@{->>}[r]& U^i \ar[d]\\
\bigwedge^{i-j+1} W \otimes \mathcal O (-1) \ar@{->>}[r]& U^{i-j}\
}
$$
if $0 < i-j \leq i \leq n$, and as the composite
$$
\xymatrix{
\bigwedge^{i+1} W \otimes \mathcal O (-1) \ar@{->>}[r]& U^i\; \ar[d]\ar@{^(->}[r]&
\bigwedge^{i} W \otimes \mathcal O\ar[d]^e\\
&U^0 \ar@{}[r]|{=} &\mathcal O\
}
$$
if $0 = i-j < i \leq n$. 

A map from a sum of copies of various $U^i$ to another
such sum is given by a homogeneous matrix over the exterior algebra $E$. In
general it is an interesting problem to relate properties of
the matrix to properties of the map. Here is one relation
which is easy. We will apply it later on in this chapter.

\noindent
\begin{proposition}\label{critsur} If
$$
r\, U^i\overset B \longrightarrow s\,U^{i-1}
$$
is a homomorphism, that is, if $B$ is an $s\times r$-matrix with entries 
in $V$, then the following condition is 
necessary for $B$ to be surjective: If $(b_1,\dots, b_r)$ is 
a non-trivial linear combination of the rows of $B$, then
$$
\dim\, {\def\span{\operatorname{span}} \span} (b_1,\dots, b_r)\geq i+1.
$$
\end{proposition}
\begin{proof} $B$ is surjective if and only if its dual map is injective on fibers:
$$
s\,{\textstyle\bigwedge}^{i-1}(V/\langle  a \rangle) \overset {\wedge B^t}\longrightarrow r
\,{\textstyle\bigwedge}^{i}(V/\langle  a \rangle)
$$
is injective for any line $\langle a\rangle \subset V$. Consider
a non-trivial linear combination $(b_1,\dots, b_r)^t$ of the columns of ${B^t}$,
and write $d = \dim\, \span (b_1,\dots, b_r)$. 
If $d=i$, then ${B^t}$ is not injective at any point 
of $\P (W)$ corresponding to a vector in ${\def\span{\operatorname{span}} \span} 
(b_1,\dots, b_r)$. If  $d < i$, then ${B^t}$ is not injective at any point of $\P (W)$.\qed
\end{proof}

\section{Cohomology and Monads}

\index{monads!applications of}
The technique of monads provides powerful tools for problems such as the 
construction and classification of coherent sheaves with prescribed invariants. 
This section is an introduction to monads. We demonstrate their usefulness, 
which is not obvious at first glance, by reviewing the
classification of stable rank 2 vector bundles on the projective plane
(see \cite{EA:MR57:324}%Barth [1977]
, \cite{EA:MR80m:14012}%Le Potier [1979]
, and \cite{EA:MR80m:14011}%Hulek [1979]
).  Recall that stable bundles admit moduli (see \cite{EA:MR81h:14014}%Gieseker [1977]
, \cite{EA:MR56:8567}, and \cite{EA:MR82h:14011}%Maruyama [1977, 1978]
).

The basic idea behind monads is to represent arbitrary coherent sheaves
in terms of simpler sheaves such as line
bundles or bundles of differentials, and in terms of homomorphisms
between these simpler sheaves. If $M$ is a finitely generated graded $S$-module,
with associated sheaf $\mathcal F = \tilde M$, then the sheafification of
the minimal free resolution of $M$ is a monad for $\mathcal F$ which involves 
direct sums of line bundles and thus homogeneous matrices over $S$. 
The Beilinson monad for $\mathcal F$, which 
will be considered in the next section, involves direct sums of twisted bundles 
of differentials $U^i$, and thus homogeneous matrices over $E$.


\noindent
\begin{definition}\label{monad}
A {\it{monad}}
\index{monad}
on $\P (W)$ is a bounded complex
$$
\cdots\, \longrightarrow {\mathcal K}^{-1} \longrightarrow {\mathcal K}^{0}
\longrightarrow {\mathcal K}^{1} \longrightarrow \, \cdots
$$
of coherent sheaves on $\P (W)$ which is exact except  at
${\mathcal K}^{0}$. The homology $\mathcal F$ at ${\mathcal K}^{0}$ 
is called the {\it {homology of the monad}}, \index{monad!homology of} 
and the monad is said to be a monad for $\mathcal F$. We say that the 
{\it {type of a monad}} \index{monad!type of} is 
determined if the sheaves ${\mathcal K}^{i}$ are determined.\qed
\end{definition}

\noindent
There are different ways of representing a given sheaf as the homology of 
a monad, and the type of the monad depends on the way chosen.

When constructing or classifying sheaves in a given class via 
monads, one typically proceeds along the following lines.

\vskip0.2cm
\noindent
{\it{Step 1.}}\; Compute cohomological information which determines 
the type of the corresponding monads.
\vskip0.1cm
\noindent
{\it{Step 2.}}\; Construct or classify the differentials of the monads.

\vskip0.2cm
\noindent
There are no general recipes for either step and some cases require
sophisticated ideas and quite a bit of intuition (see Example 7.2 below).
If one wants to classify, say, vector bundles, then a third step is needed:

\vskip0.2cm
\noindent
{\it{Step 3.}}\; Determine which monads lead to isomorphic vector bundles.
\vskip0.2cm

\noindent
One of the first successful applications of this approach was the classification
of (Gieseker-)stable rank 2 vector bundles with even first Chern class $c_1\in\Z$ on 
the complex projective plane by  Barth \cite{EA:MR57:324}, who 
detected geometric properties of the corresponding moduli spaces without giving
an explicit description of the differentials in the second step. The same ideas 
apply in the case  $c_1$ odd which we are going to survey in what follows
(see \cite{EA:MR80m:14012}, \cite{EA:MR80m:14011}, and \cite{EA:MR81b:14001} for
full details and proofs).

In general, rank 2 vector bundles enjoy properties which are not shared by
all vector bundles.

\begin{remark}\label{remark-5.2}
%{{
%  {\bf Remark 5.2.}
Every rank 2 vector bundle $\mathcal F$ on $\PP (W)$ is 
{\it {self-dual}},\index{bundle!self-dual} that is, it admits a symplectic structure. 
Indeed, the  map 
$$\mathcal F \otimes \mathcal F 
\overset\wedge\longrightarrow {\textstyle\bigwedge}^2 \mathcal F \cong \mathcal O_{\PP (W)} (c_1)
$$ 
induces an isomorphism 
$\varphi: \mathcal F \overset{\cong}\to \mathcal F^{\ast} (c_1)$ 
with $\varphi = - \varphi^* (c_1)$ (here $c_1$ is the first Chern class
of $\mathcal F$). In particular there are isomorphisms 
$$
(\H^j \mathcal F (i))^*\cong \H^{n-j} \mathcal F (-i-n-1-c_1)
$$
by Serre duality.\index{bundle!Serre duality} \qed
\end{remark}


We will not give a general definition of stability here. For rank 2 vector bundles
stability can be characterized as follows (see \cite{EA:MR81b:14001}).

\begin{remark}\label{remark-5.3} If $\mathcal F$ is a rank 2 vector bundle 
on $\PP (W)$, then the following hold:
\vskip0.1cm
\noindent
(1) \; $\mathcal F$ is stable \index{bundle!stable}
if and only if $\Hom (\mathcal F, \mathcal F)\cong K$.
In this case the symplectic structure on $\mathcal F$ is uniquely determined up to scalars.
\vskip0.1cm
\noindent
(2)\; By tensoring with a line bundle we can {\it {normalize}} $\mathcal F$
\index{bundle!normalized}
so that its first Chern class is $0$ or $-1$. In this case $\mathcal F$ is stable  
\index{bundle!stable} if and only if it has no global sections.\qed
\end{remark}

\begin{example} By the results of the previous section the twisted cotangent bundle 
$U$ on the projective plane is a stable rank 2 vector bundle with Chern classes
$c_1 = -1$ and $c_2 = 1$.\qed
\end{example}

\begin{remark}\label{remark-5.5}
% \noindent
%{{{\bf Remark 5.3.} 
The generalized theorem of Riemann-Roch yields a polynomial
in $\QQ [c_1, \dots ,  c_r]$ which gives the Euler characteristic 
$\chi \mathcal F = \sum_j (-1)^j \h^j \mathcal F$ for e\-very rank $r$ vector bundle 
$\mathcal F$ on $\PP (W)$ with Chern classes $c_1, \dots , c_r$.
This polynomial can be determined by interpreting the generalized theorem of Riemann-Roch
or by computing the Euler characteristic for enough special bundles of rank $r$ (like direct
sums of line bundles). For a rank 2 vector bundle on the projective plane, 
for example, one obtains 
$$\chi (\mathcal F)=(c_1^2-2c_2+3c_1+4)/2\, .\quad\qed$$
\end{remark}


We now focus on stable rank 2 vector bundles on the complex projective plane
$\PP^2(\CC) = \PP(W)$ with first Chern class $c_1 = -1$.
Let $\mathcal F$ be such a bundle.


\begin{remark}\label{remark-5.6} Since $\mathcal F$  is stable and
normalized its second Chern class $c_2$ must be $\geq 1$. Indeed, 
$$\H^2 \mathcal F (i-2) = \H^0 \mathcal F(-i) = 0 \quad {\text{for}}\quad i\geq 0$$
by Remarks 5.2 and 5.3, and $\chi (\mathcal F (i)) = (i+1)^2 -c_2$ by
Riemann-Roch. In particular the dimensions 
$\h^j \mathcal F (i)$ in the range $-2 \leq i \leq 0$ are 
as in the following cohomology table (a zero is represented by an empty box):
\vspace{0.2cm}
%%
%%%%%%%%%%%%%%%%%%%%%%%%%%%%%%%%%%%%%%%%%%%%%%%%%%%%%%%%%%%%%%%%%%%%%%%%%%%%%%%

% Neue L�ngen f�r Abst�nde horizontal und vertikal
% Nur einmal vor dem ersten Auftreten eines Beilinson-Diagramms
%\newlength{\br}
%\newlength{\ho}
%
%%%%%%%%%%%%%%%%%%%%%%%%%%%%%%%%%%%%%%%%%%%%%%%%%%%%%%%%%%%%%%%%%%%%%%%%%%%%%%%

{
$$ %Diagramm zentrieren
%
% W�hle die Einheiten \br horizontal und \ho vertikal
{
\setlength{\br}{10mm}
\setlength{\ho}{6mm}
\fontsize{10pt}{8pt}
\selectfont
\begin{xy}
%%%%%%%%%%%%%%%%%%%%%%%%%%%%%%%%%%%%%%%
%
% Achsenkreuz im Punkte (0,0)
%
% x-Achse von -3\br bis 1\br
% mit einem "j" an 0.98 der L�nge und 3mm unter der Achse:
%
,<-3.5\br,0\ho>;<1\br,0\ho>**@{-}?>*@{>}
?(0.95)*!/^3mm/{i}
%
% y-Achse von 0\ho bis 4\ho
% mit einem "i" an 0,9 der L�nge und 3mm rechts neben der Achse:
%
,<-1\br,0\ho>;<-1\br,4\ho>**@{-}?>*@{>}
?(0.95)*!/^3mm/{j}
%%%%%%%%%%%%%%%%%%%%%%%%%%%%%%%%%%%%%%%
%
% 5 waagrechte Linien von -3\br bis +0\br
% in den H�hen 1\ho,...,5\ho:
%
,0+<-3\br,1\ho>;<0\br,1\ho>**@{-}
,0+<-3\br,2\ho>;<0\br,2\ho>**@{-}
,0+<-3.5\br,3\ho>;<.5\br,3\ho>**@{-}
%%%%%%%%%%%%%%%%%%%%%%%%%%%%%%%%%%%%%%%
%
% 11 senkrechte Linien von 0\ho bis 3\ho
% in den waagrechten Punkten -3\br,...,+0\br:
%
,0+<-3\br,0\ho>;<-3\br,3\ho>**@{-}
,0+<-2\br,0\ho>;<-2\br,3\ho>**@{-}
,0+<0\br,0\ho>;<0\br,3\ho>**@{-}
%
%%%%%%%%%%%%%%%%%%%%%%%%%%%%%%%%%%%%%%%
%
% Eintr�ge in den Mitten der K�sten. Daher die Koordinaten mit .5
%
,0+<-2.5\br,1.5\ho>*{c_2-1}
,0+<-1.5\br,1.5\ho>*{c_2}
,0+<-0.5\br,1.5\ho>*{c_2-1}
%
%%%%%%%%%%%%%%%%%%%%%%%%%%%%%%%%%%%%%%%
,0+<-2.5\br,-0.6\ho>*{-2}
,0+<-1.5\br,-0.6\ho>*{-1}
,0+<-0.5\br,-0.6\ho>*{0}
%%%%%%%%%%%%%%%%%%%%%%%%%%%%%%%%%%%%%%%
,0+<-4.0\br,2.5\ho>*{2}
,0+<-4.0\br,1.5\ho>*{1}
,0+<-4.0\br,0.5\ho>*{0}
\end{xy}
}
$$
\quad\qed
}
%%%%%%%%%%%%%%%%%%%%%%%%%%%%%%%%%%%%%%%%%%%%%%%%%%%%%%%%%%%%%%%%%%%%%%%%%%%%%%%
%\vskip0.1cm
\noindent
\end{remark}

We abbreviate $\mathcal O = \mathcal O_{\PP^2 (\CC)}$ and go through the 
three steps above.

\vskip0.3cm
\noindent
{\it{Step 1.}}\; In this step we show that $\mathcal F$ is the 
homology of a monad of type
$$
0\rightarrow \H^1 \mathcal F (-2)\otimes U^2 
\rightarrow \H^1 \mathcal F (-1) \otimes U \rightarrow 
\H^1 \mathcal F \otimes \mathcal O \rightarrow 0\ ,
$$
where the middle term occurs in cohomological degree 0.
This actually follows from the general construction of Beilinson monads presented
in the next chapter and the fact that $\H^2 \mathcal F (i-2) = \H^0 \mathcal F(-i) = 0$ 
for $2\geq i\geq 0$ (see Remark \ref{remark-5.6}). Here we derive the existence 
of the monad directly with Horrocks' technique of killing cohomology 
\index{killing cohomology} \cite{EA:MR84j:14026},  which 
requires further cohomological information. Such information is typically obtained 
by restricting the given bundles to linear subspaces. In our case we consider 
the Koszul complex on the equations of a point $p\in \PP^2(\CC)$:
$$
\xymatrix @C=11mm{
0\ar[r]&\ \mathcal O(-2)\ar[r]^{\left(\substack{-x'\\x}\right)}
&2\:\! \mathcal O (-1)\ar[r]^{\quad(x\ x^{\prime})} &\mathcal O
\ar[r]&\mathcal O_p\ar[r]&0\ .
}
$$
\noindent
By tensoring with $\mathcal F (i+1)$ and taking cohomology we find that 
$\H^1 \mathcal F$ generates $\H^1_{\geq 0}\, \mathcal F$. Indeed, the composite map 
$${\begin{pmatrix} x & x^{\prime}\end{pmatrix}}: 2 \\\H^1\mathcal F(i) 
\longrightarrow \H^1 (\mathcal J_p\otimes\mathcal F(i+1))
\longrightarrow \H^1\mathcal F(i+1)$$
is surjective if $i\geq -1$. In particular, if $c_2=1$, then  
$\H^1 \mathcal F(i) =  0$ for $i \neq -1$ (apply Serre duality for the
twists $\leq -2$), so $\mathcal F\cong U$ is the twisted cotangent bundle by Remark 
\ref{cohbd} since both bundles have the same rank and intermediate cohomology. 

If $c_2\geq 2$ then $\H^1 \mathcal F\neq 0$, and the identity in 
$$\Hom (\H^1 \mathcal F, \H^1 \mathcal F)\cong\Ext^1(\H^1 \mathcal F 
\otimes \mathcal O, \mathcal F)$$
 defines an extension
$$
0\rightarrow \mathcal F \rightarrow \mathcal G \rightarrow \H^1 \mathcal F
\otimes \mathcal O\rightarrow 0\ , 
$$
where $\H^1_{\geq 0}\,\mathcal G = 0$, and where $\mathcal G$ is a vector bundle 
(apply Serre's criterion in Section 4). Similarly, by taking Serre duality into
account,  we obtain an extension
$$
0\rightarrow \H^1 \mathcal F (-2)\otimes  U^2 \rightarrow \mathcal H
\rightarrow  \mathcal F\rightarrow 0\ , 
$$
where $\mathcal H$ is a vector bundle with $\H^{1}_{\leq -2} \mathcal H = 0$. 
The two extensions fit into a commutative diagram with exact rows and and columns
%%%%%%%%%%%%%%%%%%%%%%%%%%%%%%%%%%%%%%%%%%%%%%%%%%%%%
$$
\xymatrix{
&&0\ar[d]&0\ar[d]\\
0\ar[r]& \H^1 \mathcal F (-2)\otimes U^2\ar@{}[d]|{\scriptscriptstyle ||}\ar[r]&
\mathcal  H\ar[d]\ar[r]&
\mathcal F\ar[r]\ar[d]& 0\\
0\ar[r]&  \H^1 \mathcal F (-2)\otimes U^2\ar[r]^{\qquad \quad \alpha}& 
\mathcal B \ar[d]^{\beta}\ar[r]&\mathcal G\ar[d]\ar[r]& 0\\
&& \H^1 \mathcal F\otimes \mathcal O \ar[d]\ar@{}[r]|{=}&
\H^1 \mathcal F \otimes \mathcal O \ar[d]\\
&&0&0
}
$$
%%%%%%%%%%%%%%%%%%%%%%%%%%%%%%%%%%%%%%%%%%%%%%%%%%%%
since, for example, the extension in the top row lifts uniquely to an extension 
as in the middle row (the obstructions in the corresponding Ext-sequence vanish). 
Then $\mathcal B \cong \H^1 \mathcal F (-1)\otimes U$ 
since by construction these bundles  have the same rank and intermediate cohomology.
What we have is the {\it display}
\index{monad!display of}
 of (the short exact sequences associated to)
a monad 
$$
0\longrightarrow \H^1 \mathcal F (-2)\otimes U^2 \overset{\alpha}
\longrightarrow \H^1 \mathcal F (-1) \otimes U \overset{\beta}\longrightarrow 
\H^1 \mathcal F \otimes \mathcal O \longrightarrow 0
$$
for $\mathcal F$. 

\vskip0.3cm
\noindent
{\it{Step 2.}}\;
Our task in this step is to describe what maps $\alpha$ and $\beta$ could
be the differentials of a monad as above. In fact we give a description in terms of 
linear algebra for which it is enough to deal with one of the differentials,
say $\alpha$, since the self-duality of $\mathcal F$ and the vanishing
of certain obstructions allows one to represent $\mathcal F$ as the
homology of a ``self-dual'' monad. Let us abbreviate $A=\H^1 \mathcal F (-2)$, 
$B=\H^1 \mathcal F (-1)$ and $A^*\cong \H^1 \mathcal F$. By chasing the displays 
of a monad as above and its dual we see that the symplectic structure on 
$\mathcal F$ lifts to a unique isomorphism of monads
$$
\xymatrix{
0\ar[r]& A\otimes U^2 \ar[d]^{\Phi}\ar[r]^{\alpha}&
B \otimes U \ar[d]^{\Psi}\ar[r]^{\beta}&
A^*\otimes\mathcal O \ar[r] \ar[d]^{-\Phi^*(-1)} & 0\\
0\ar[r]&A\otimes \mathcal O(-1) \ar[r]^{{\beta^*}(-1)} & B^* \otimes U^*(-1) 
\ar[r]^{{\alpha^*}(-1)}& A^*\otimes (U^2)^*(-1) \ar[r]& 0
}
$$
with $\Psi = -\Psi^*(-1)$. Indeed, the corresponding obstructions vanish
(see \cite{EA:MR80f:14005} and \cite[II, 4.1]{EA:MR81b:14001} for a discussion of 
this argument in a general context). $\Psi$ is the tensor product of an isomorphism 
$q: B\rightarrow B^*$ and a symplectic form $\iota\in\Hom(U,U^*(-1))\cong\CC$
on $U$.  Note that $q$ is symmetric since
$-(q\otimes\iota) = (q\otimes\iota)^*(-1)=q^*\otimes\iota^*(-1)=-q^*\otimes\iota$.
We may and will now assume that $\mathcal F$ is the homology of a {\it {self-dual monad}},
\index{monad!self-dual}
where self-dual means  that $\beta = \alpha^d:=\alpha^* (-1)\circ(q\otimes\iota)$. 
The monad conditions

\vskip0.2cm
($\alpha_1$) $\alpha^d\circ\alpha = 0$, and
\vskip0.1cm
($\alpha_2$) $\alpha$ is a vector bundle monomorphism ($\alpha^d$ is an epimorphism)
\vskip0.2cm
\noindent
can be rewritten in terms of linear algebra as follows. 
The identifications in Lemma \ref{hombd} allow one to view 
$$\alpha\in\Hom(A\otimes U^2, B \otimes U) \cong V \otimes \Hom(A, B) $$
as a homomorphism $\alpha: W \rightarrow \Hom (A,B)$ operating by
$\xi \otimes (x \wedge x^{\prime}) \rightarrow \alpha (x)(\xi) \otimes x^{\prime}
- \alpha (x^{\prime})(\xi) \otimes x$ on the fibers of 
$A\otimes U^2$. Similarly we consider $\alpha^d$
as the homomorphism $\alpha^d: W \rightarrow \Hom (B,A^*), x\mapsto \alpha^*(x) \circ q$, 
operating by $\eta \otimes x \rightarrow \alpha^d (x)(\eta)$ on the fibers
of $B \otimes U$. Then

\vskip0.2cm
($\alpha_1^{\prime}$)\, $\alpha^d(x) \circ \alpha(x^{\prime})=
\alpha^d(x^{\prime}) \circ\alpha(x)$ for all $x, x^{\prime}\in W$, and
\vskip0.1cm
($\alpha_2^{\prime}$)\, for every $\xi\in A\setminus \{0\}$
the map $W \rightarrow B$, $x \to \alpha(x)(\xi)$ has rank $\geq 2$.


\noindent
% \noindent
%\begin{Example} 
\begin{example}\label{exc22}
If $c_2=2$, then  the monads can be written (non-canonically) as 
$$
\xymatrix @C=9mm{
0\ar[r]&U^2\ar[r]^{\ \left(\substack{a\\b}\right)}
&2\:\!U\ar[r]^{\,(a\ b)} &\mathcal O\ar[r]&0\ ,
}
$$
where $a, b$ are two vectors in $V$.
In this case ($\alpha_1$) gives no extra condition and 
($\alpha_2$) means that $a$ and $b$ are linearly independent.
If $a$ and $b$ are  explicitly given, then we can compute the
homology of the monad with the help of \Mtwo:

\vskip0.3cm
\beginOutput
i25 : S = ZZ/32003[x_0..x_2];\\
\endOutput
\vskip0.1cm
\noindent
$U$ is obtained from the Koszul complex resolving $S/(x_0, x_1, x_2)$ by
tensoring the cokernel of the differential $\bigwedge^3 W\otimes S(-3) 
\to \bigwedge^2W \otimes S(-2)$ with $S(1)$ (and sheafifying).

\vskip0.1cm
\beginOutput
i26 : U = coker koszul(3,vars S) ** S^\{1\};\\
\endOutput
\vskip0.1cm

\noindent
For representing $\alpha$ and $\alpha^d$ we also need the differential 
$\bigwedge^2W \otimes S(-2) \to W \otimes S(-1)$ of the Koszul complex.

\vskip0.1cm
\beginOutput
i27 : k2 = koszul(2,vars S)\\
\emptyLine
o27 = \{1\} | -x_1 -x_2 0    |\\
\      \{1\} | x_0  0    -x_2 |\\
\      \{1\} | 0    x_0  x_1  |\\
\emptyLine
\              3       3\\
o27 : Matrix S  <--- S\\
\endOutput
\vskip0.1cm

\noindent
The expression
{\tt{koszul(2,vars S)}} computes a matrix representing the differential
with respect to the monomial bases $x_0\wedge x_1, x_0 \wedge x_2, x_1 \wedge x_2$ 
of $\bigwedge^2W$ and $x_0, x_1, x_2$ of $W$.
We pick $(a,b) = (e_1, e_2)$ and represent the corresponding maps $\alpha$ 
and $\alpha^d$ with respect to the monomial bases (see the discussion following
Lemma \ref{hombd}).

\vskip0.1cm
\beginOutput
i28 : alpha = map(U ++ U, S^\{-1\}, transpose\{\{0,-1,0,1,0,0\}\});\\
\emptyLine
o28 : Matrix\\
\endOutput
\beginOutput
i29 : alphad = map(S^1, U ++ U, matrix\{\{0,1,0,0,0,1\}\} * (k2 ++ k2));\\
\emptyLine
o29 : Matrix\\
\endOutput
\vskip0.1cm

\noindent
Prune computes a minimal presentation.

\vskip0.1cm
\beginOutput
i30 : F = prune homology(alphad, alpha);\\
\endOutput
\beginOutput
i31 : betti  F\\
\emptyLine
o31 = relations : total: 3 1\\
\                      1: 2 .\\
\                      2: 1 1\\
\endOutput

\vskip0.3cm

In the next section we will present a more elegant way of 
computing the homology of Beilinson monads. \qed
\end{example}

We go back to the general case and reverse our construction. 
Let $A$ and $B$ be $\CC$-vector spaces of 
the appropriate dimensions, let $q$ be a non-degenerate quadratic form on $B$,
and let 
$$
\widetilde{\cal{M}}=\{\alpha\in\Hom(W, \Hom(A, B)) \mid \alpha {\text { satisfies }} 
(\alpha_1^{\prime}) {\text { and }} (\alpha_2^{\prime})\}\ .
$$
Then every $\alpha\in \widetilde{\cal{M}}$ defines a self-dual
monad as above whose homology is a stable rank 2 vector bundle on $\PP^2(\CC)$ with 
Chern classes $c_1=-1$ and $c_2$. In this way we obtain a description of
the differentials of  the monads which is not as explicit as we might have 
hoped (with the exception of the case $c_2 = 2$). It is, however,  enough for detecting 
geometric properties of the corresponding moduli spaces.

\vskip0.3cm
\noindent
{\it{Step 3.}}\; Constructing the moduli spaces means to parametrize
the isomorphism classes of our bundles in a convenient way. We very roughly 
outline how to do that. Let ${\text{O}}(B)$ be the orthogonal group of
$(B,q)$, and let $G:= \GL (A)\times {\text{O}}(B)$. Then $G$ acts on 
$\widetilde{\cal{M}}$ by $((\Phi, \Psi),\alpha) \mapsto \Psi\alpha\Phi^{-1}$,
where $\Psi\alpha\Phi^{-1}(x) := \Psi\alpha(x)\Phi^{-1}$. We may consider an 
element $(\Phi, \Psi)\in G$ as an isomorphism bet\-ween 
the monad defined by $\alpha$ and the monad defined by $\Psi\alpha\Phi^{-1}$.
By going back and forth between isomorphisms of bundles
and isomorphisms of mo\-nads one shows that the  stabilizer of $G$ in 
each point is $\{\pm 1\}$, and that our construction induces a bijection
between the set of isomorphism classes of stable rank 2 vector bundles on $\PP^2(\CC)$ 
with Chern classes $c_1=-1$ and $c_2$ and ${\cal{M}}:=\widetilde{\cal{M}}/G_0$, 
where $G_0:=G/\{\pm 1\}$. With the help of a universal monad over 
$\PP^2(\CC)\times \widetilde{\cal{M}}$ one proves that the analytic structure on 
$\widetilde{\cal{M}}$ descends to an analytic structure on ${\cal{M}}$ so that 
${\cal{M}}$ is smooth of dimension $\h^1 \mathcal F^* \otimes \mathcal F = 4c_2-4$ 
in each point (the obstructions for smoothness in the point corresponding to 
$\mathcal F$ lie in $\H^2 \mathcal F^* \otimes \mathcal F$ which is zero). 
Moreover the homology of the universal monad tensored by a suitable line bundle
descends to a universal family over ${\cal{M}}$ (here one needs $c_1=-1$). In other 
words, ${\cal{M}}$ is what one calls a fine moduli space for our bundles. Further 
efforts show that ${\cal{M}}$ is irreducible and rational.

\begin{remark} Horrocks' technique of killing cohomology always yields 3-term monads.
In general, the bundle in the middle can be pretty complicated.\qed
\end{remark}


\section{The Beilinson Monad}

%\textbf{EISENBUD}


%\section{beilinson} The Beilinson Monad

\index{monad!Beilinson}
We can use the Tate resolution associated to a sheaf to 
give a construction of a complex first
described by Beilinson \cite{EA:MR80c:14010b}, which gives a 
powerful method for deriving information about a sheaf from
information about a few of its cohomology groups.
The general idea is the following:


Suppose that $\A$ is an additive category and consider
a graded object
$
\oplus_{i=0}^{n+1}U^i
$
in $\A$.
Given a graded ring homomorphism $E\to \End_\A(\oplus_{i=0}^{n+1}U^i)$
we can make an additive functor from the category of 
free $E$-modules to $\A$:
On objects we take
$$
\omega_E(i)\mapsto{ \begin{cases} U^i & \text{for } 0\leq i
\leq {n+1} \text{ and}; \\0 & \text{otherwise.}\end{cases}}
$$
To define the functor on maps, we use 
$$
\begin{aligned}
\Hom_E(\omega_E(i),\omega_E(j))&=
\Hom_E(E(i),E(j))\\
&= E_{j-i}
\longrightarrow\End(\oplus U^i)_{j-i}\longrightarrow \Hom(U^i,U^j)\ .
\end{aligned}
$$
(Note that we could have taken any twist of $E$ in place of 
$\omega_E\cong E(-n-1)$; the choice of $\omega_E$ is made to 
simplify the statement of Theorem \ref{Beilinson-theorem}, below.)
%%%%%%%%% Referenz \ref{Beilinson theorem}, below.)

We shall be interested in the special case
where $\A$ is the category of coherent sheaves on
$\P(W)$ and where $U^i = \Omega_{\P(W)}^i (i)$ as in Section~4.
Further examples may be obtained by taking $U^i$ to be the $i^\th$ exterior
power of the tautological subbundle $U_k$ on the Grassmannian
of $k$-planes in $W$ for any $k$; the case we have taken
here is the case $k=n$. See \cite{EA:Eisenbud-Schreyer:ChowForms} 
for more information on the general case
and applications to the computation of
resultants and more general Chow forms.

Applying the functor just defined to the Tate resolution $\TT(\F)$
of a cohe\-rent sheaf $\F$ on $\P(W)$, 
and using Theorem \ref{tate}, we get 
a complex
$$
\Omega(\F):\quad
\cdots\rTo \oplus_j \H^j\F(i-j)\otimes U^{j-i}\rTo\dots ,
$$
where the term we have written down occurs in cohomological
degree $i$. The resolution $\TT(\F)$ is well-defined up to homotopy, 
so the same is true of $\Omega(\F)$.
Since $U^{k}=0$ unless $0\leq k\leq n$
the only cohomology groups of $\F$ that
are actually involved in $\Omega(\F)$ are $\H^j\F(k)$ with
$-n\leq k\leq 0$; $\Omega(\F)$ is of type
$$
\xymatrix@1@C=3mm{
0\ar[r]& \H^0 \mathcal F (-n)\otimes U^n\ar@{}[d]|{\scriptscriptstyle ||}\ar[r]&
\cdots \ar[r]& 
\oplus_{j=0}^{n}\H^j \mathcal F (-j)\otimes U^j\ar@{}[d]|{\scriptscriptstyle ||}\ar[r]&
\cdots \ar[r]&
\H^n \mathcal F \otimes U^0\ar@{}[d]|{\scriptscriptstyle ||}\ar[r]&
0&\\
0\ar[r]&  \Omega^{-n}(\F)\ar[r]&\cdots \ar[r]& \Omega^0(\F)\ar[r]& 
\cdots \ar[r]& \Omega^n(\F)\ar[r]& 0&.\\
}
$$
For applications it is important to note that instead of working with  
$\Omega(\F)$ one can also work with $\Omega(\F(i))$ for some twist $i$. 
This gives one some freedom in choosing the cohomology groups of $\F$ 
to be involved.


To see a simple example, consider again the structure 
sheaf $\O_p$ of the subvariety consisting of a point
$p\in \P(W)$. Write $I$ for the homogeneous ideal of $p$,
and let $a\in V=W^*$ be a non-zero functional vanishing
on the linear forms in $I$ as before.
The Tate resolution of the homogeneous coordinate ring $S/I$
has already been computed, and we have seen that it depends only
on the sheaf $\widetilde{S/I}=\O_p$. From the computation of
$\TT(S/I)=\TT(\O_p)$ made in Section~3 we see that 
$\Omega(\O_p)$ takes the form
$$
\Omega(\O_p):\quad 0\to U^n\rTo^a U^{n-1}\rTo^a
\cdots \rTo^a U^1\rTo^a U^0\rTo 0\ ,
$$
with $U^i$ in cohomological degree $-i$. 

We have already noted that the map 
$a: U=U^1 \rTo U^0=\O_{\P(W)}$ is the composite of the
tautological embedding $U\subset W\otimes \O_{\P(W)}$ with the
map $a\otimes 1:\ W\otimes \O_{\P(W)} \to \O_{\P(W)}$.
Thus the image of $a: U^1\to \O_{\P(W)}$ is the ideal
sheaf of $p$, and we see that the homology of the complex
$\Omega(\O_p)$ at $U^0$ is $\O_p$. One can check further
that $\Omega(\O_p)$ is the Koszul complex associated with
the map $a: U^1\to \O_{\P(W)}$, and it follows that 
the homology of $\Omega(\O_p)$ at $U^i$ is 0 for $i>0$.
The following result shows that this is typical.

\begin{theorem}[\cite{EA:eis-sch:sheaf}]\label{Beilinson-theorem}
%\theorem{Beilinson theorem} 
%\vskip0.3cm
%\noindent
%\textbf{Theorem 6.1}
%{\em 
If $\F$ is a coherent sheaf on
$\P(W)$, then the only non-vanishing homology of  the
complex $\Omega(\F)$ is 
$$
\H^0(\Omega(\F))=\F\, .\qquad\qed
$$
%}
\end{theorem}

The existence of a complex satisfying the theorem and having the same
terms as $\Omega(\F)$ was first asserted by
Beilinson in \cite{EA:MR80c:14010b},
and thus we will call $\Omega(\F)$ a {\it Beilinson monad\/} 
\index{monad!Beilinson}
\index{Beilinson monad}
for $\F$.
Existence proofs via a somewhat less effective construction than the one
given here may be found in \cite{EA:MR89g:18018} and \cite{EA:MR92g:14013}.

The explicitness of the construction via Tate resolutions allows one to 
detect properties of the differentials of Beilinson monads.
\index{Beilinson monad!differentials of} Let us write
$$
\begin{aligned}
d_{ij}^{(r)}\in \Hom( \H^j \mathcal F (i-j) \otimes U^{j-i}, 
\H^{j-r+1} \mathcal F (i-j+r) \otimes U^{j-i-r})\\
\cong {\textstyle\bigwedge}^r V\otimes  \Hom (\H^j \mathcal F (i-j),  
\H^{j-r+1} \mathcal F (i-j+r))\\
\cong \Hom ({\textstyle\bigwedge}^r W \otimes \H^j \mathcal F (i-j),  
\H^{j-r+1} \mathcal F (i-j+r))
\end{aligned}
$$
for the degree $r$ maps actually occurring in $\Omega(\F)$.

\begin{remark}\label{diff1}\; The constant maps $d_{ij}^{(0)}$ in 
$\Omega(\F)$ are zero since $\TT(\mathcal F)$ is minimal. \qed
\end{remark}

\begin{proposition}[\cite{EA:eis-sch:sheaf}]\label{diff2}\; 
The linear maps $d_{ij}^{(1)}$ in $\Omega(\F)$ 
correspond to the multiplication maps 
$$W \otimes \H^j \mathcal F (i-j)\to \H^{j} \mathcal F (i-j+1)\, .\qquad\qed$$
\end{proposition}

\noindent
This follows from the identification of the linear strands in $\TT (\F)$ 
(see the discussion following Theorem 3.1). The higher degree maps in 
$\TT (\F)$ and $\Omega(\F)$, however, are not yet well-understood.

Since $(\TT(\mathcal F))[1] = \TT(\mathcal F(1))$ we can compare
the differentials in $\Omega(\F)$ with those in $\Omega(\F(1))$:

\begin{proposition}[\cite{EA:eis-sch:sheaf}]\label{diff3}\; If the maps 
$d_{ij}^{(r)}$ in $\Omega(\F)$ and $d_{i-1,j}^{(r)}$ in $\Omega(\F(1))$ 
both actually occur, then they correspond to the same element
in 
$${\textstyle\bigwedge}^r V\otimes  \Hom (\H^j \mathcal F (i-j),  
\H^{j-r+1} \mathcal F (i-j+r))\ .\qquad\qed$$
\end{proposition}


In what follows we present some \Mtwo code for computing Beilinson monads. 
Our  functions {\tt {sortedBasis}}, {\tt {beilinson1}}, {\tt {U}}, and
{\tt {beilinson}} reflect what we did in Example \ref{exc22} . 

The expression
{\tt {sortedBasis(i,E)}} sorts the monomials of degree $i$ in $E$ to match the order 
of the columns of {\tt {koszul(i,vars S)}}, where our conventions with respect to $S$ and $E$
are as in Section 2, and where we suppose that the monomial order on $E$ is
reverse lexicographic, the \Mtwo default order.

\vskip0.3cm
\beginOutput
i32 : sortedBasis = (i,E) -> (\\
\           m := basis(i,E);\\
\           p := sortColumns(m,MonomialOrder=>Descending);\\
\           m_p);\\
\endOutput

\vskip0.1cm
\noindent
For example:
\vskip0.1cm

\beginOutput
i33 : S=ZZ/32003[x_0..x_3];\\
\endOutput
\beginOutput
i34 : E=ZZ/32003[e_0..e_3,SkewCommutative=>true];\\
\endOutput
\beginOutput
i35 : koszul(2,vars S)\\
\emptyLine
o35 = \{1\} | -x_1 -x_2 0    -x_3 0    0    |\\
\      \{1\} | x_0  0    -x_2 0    -x_3 0    |\\
\      \{1\} | 0    x_0  x_1  0    0    -x_3 |\\
\      \{1\} | 0    0    0    x_0  x_1  x_2  |\\
\emptyLine
\              4       6\\
o35 : Matrix S  <--- S\\
\endOutput
\beginOutput
i36 : sortedBasis(2,E)\\
\emptyLine
o36 = | e_0e_1 e_0e_2 e_1e_2 e_0e_3 e_1e_3 e_2e_3 |\\
\emptyLine
\              1       6\\
o36 : Matrix E  <--- E\\
\endOutput

\vskip0.1cm
\noindent
If $e\in E$ is homogeneous of degree $j$, then 
{\tt {beilinson1(e,j,i,S)}} computes the map $U^i \overset{e}\longrightarrow U^{i-j}$
on $\PP^n = \Proj \,S$. If $0 < i-j \leq i \leq n$, then the result is a matrix representing 
the map $\bigwedge^{i+1}W \otimes S(-1)\overset{e\otimes 1} \longrightarrow 
\bigwedge^{i-j+1}W\otimes S(-1)$ defined by contraction with $e$. If $0 = i-j < i \leq n$, 
then the result is a matrix representing the composite
of the map $\bigwedge^{i}W \otimes S \overset{e\otimes 1}
\longrightarrow S$ with the Koszul differential 
$\bigwedge^{i+1}W \otimes S(-1) \rightarrow \bigwedge^{i}W \otimes S$.
Note that the degrees of the result are not set correctly since 
the functions {\tt {U}} and {\tt {beilinson}} below are supposed to do that.
\vskip0.1cm

\beginOutput
i37 : beilinson1=(e,dege,i,S)->(\\
\           E := ring e;\\
\           mi := if i < 0 or i >= numgens E then map(E^1, E^0, 0)\\
\                 else if i === 0 then id_(E^1)\\
\                 else sortedBasis(i+1,E);\\
\           r := i - dege;\\
\           mr := if r < 0 or r >= numgens E then map(E^1, E^0, 0)\\
\                 else sortedBasis(r+1,E);\\
\           s = numgens source mr;\\
\           if i === 0 and r === 0 then\\
\                substitute(map(E^1,E^1,\{\{e\}\}),S)\\
\           else if i>0 and r === i then substitute(e*id_(E^s),S)\\
\           else if i > 0 and r === 0 then\\
\                (vars S) * substitute(contract(diff(e,mi),transpose mr),S)\\
\           else substitute(contract(diff(e,mi), transpose mr),S));\\
\endOutput
 
\vskip0.1cm
\noindent
For example:
\vskip0.1cm

\beginOutput
i38 : beilinson1(e_1,1,3,S)\\
\emptyLine
o38 = \{-3\} | 0 |\\
\      \{-3\} | 0 |\\
\      \{-3\} | 1 |\\
\      \{-3\} | 0 |\\
\emptyLine
\              4       1\\
o38 : Matrix S  <--- S\\
\endOutput
\beginOutput
i39 : beilinson1(e_1,1,2,S)\\
\emptyLine
o39 = \{-2\} | 0  0  0 0 |\\
\      \{-2\} | -1 0  0 0 |\\
\      \{-2\} | 0  0  0 0 |\\
\      \{-2\} | 0  -1 0 0 |\\
\      \{-2\} | 0  0  0 0 |\\
\      \{-2\} | 0  0  0 1 |\\
\emptyLine
\              6       4\\
o39 : Matrix S  <--- S\\
\endOutput
\beginOutput
i40 : beilinson1(e_1,1,1,S)\\
\emptyLine
o40 = | x_0 0 -x_2 0 -x_3 0 |\\
\emptyLine
\              1       6\\
o40 : Matrix S  <--- S\\
\endOutput

\vskip0.1cm
\noindent
The function {\tt {U}} computes the bundles $U^i$ on Proj$\,S$:

\vskip0.1cm
\noindent
\beginOutput
i41 : U = (i,S) -> (\\
\           if i < 0 or i >= numgens S then S^0\\
\           else if i === 0 then S^1\\
\           else cokernel koszul(i+2,vars S) ** S^\{i\});\\
\endOutput

\vskip0.1cm
\noindent
Finally, if $o : \oplus E(-a_i) \to \oplus E(-b_j)$ is a homogeneous 
matrix over $E$, then {\tt {beilinson(o,S)}} computes the 
corresponding map $o : \oplus U^{a_i} \to \oplus U^{b_j}$ on Proj$\,S$
by calling {\tt {beilinson1}} and {\tt {U}}.
\vskip0.3cm

\beginOutput
i42 : beilinson = (o,S) -> (\\
\           coldegs := degrees source o;\\
\           rowdegs := degrees target o;\\
\           mats = table(numgens target o, numgens source o,\\
\                    (r,c) -> (\\
\                         rdeg = first rowdegs#r;\\
\                         cdeg = first coldegs#c;\\
\                         overS = beilinson1(o_(r,c),cdeg-rdeg,cdeg,S);\\
\                         -- overS = substitute(overE,S);\\
\                         map(U(rdeg,S),U(cdeg,S),overS)));\\
\           if #mats === 0 then matrix(S,\{\{\}\})\\
\           else matrix(mats));\\
\endOutput

\vskip 0.1cm
\noindent
With these functions the code in Example \ref{exc22} can be rewritten as follows:
\vskip 0.1cm

\beginOutput
i43 : S=ZZ/32003[x_0..x_2];\\
\endOutput
\beginOutput
i44 : E = ZZ/32003[e_0..e_2,SkewCommutative=>true];\\
\endOutput
\beginOutput
i45 : alphad = map(E^1,E^\{-1,-1\},\{\{e_1,e_2\}\})\\
\emptyLine
o45 = | e_1 e_2 |\\
\emptyLine
\              1       2\\
o45 : Matrix E  <--- E\\
\endOutput
\beginOutput
i46 : alpha = map(E^\{-1,-1\},E^\{-2\},\{\{e_1\},\{e_2\}\})\\
\emptyLine
o46 = \{1\} | e_1 |\\
\      \{1\} | e_2 |\\
\emptyLine
\              2       1\\
o46 : Matrix E  <--- E\\
\endOutput
\beginOutput
i47 : alphad=beilinson(alphad,S);\\
\emptyLine
o47 : Matrix\\
\endOutput
\beginOutput
i48 : alpha=beilinson(alpha,S);\\
\emptyLine
o48 : Matrix\\
\endOutput
\beginOutput
i49 : F = prune homology(alphad,alpha);\\
\endOutput
\beginOutput
i50 : betti  F\\
\emptyLine
o50 = relations : total: 3 1\\
\                      1: 2 .\\
\                      2: 1 1\\
\endOutput

\vskip 0.3cm

\section{Examples}

\index{Beilinson monad!applications of}
In this section we give two examples of explicit constructions of Beilinson 
monads over $\PP^4(\CC) = \PP(W)$ and of classification results based on these monads.
As in Section 5 we proceed in three steps. Let us write $\mathcal O = 
\mathcal O_{\PP^4(\CC)}$.

%\begin{Example}\label{was7.1}
\begin{example}\label{was7.1}
%\noindent
%{\it{Example 7.1.}}\, 
Our first example is taken from the classification
of {\it {conic bundles}} 
\index{conic bundle}
in  $\PP^4(\CC)$, that is, of smooth surfaces 
$X\subset\PP^4(\CC)$ which are ruled in conics in the sense that there exists a 
surjective morphism $\pi:X \rightarrow C$ onto a smooth curve $C$ such that the 
general fiber of $\pi$ is a smooth conic in the given embedding of $X$.
There are precisely three families of such surfaces (see \cite{EA:ES} and 
\cite{EA:BR}). Two families, the Del Pezzo surfaces of 
degree 4 and the  Castelnuovo surfaces, are classical. 
The third family, consisting of {\it {elliptic conic bundles}}
\index{conic bundle!elliptic}
(conic bundles over an elliptic curve) of degree 8, had been falsely 
ruled out in two classification papers in the 1980's
(see  \cite{EA:okgrad8} and \cite{EA:ionescu}). Only recently Abo, Decker, and Sasakura 
\cite{EA:conicbundle} constructed and classified such surfaces  by considering the Beilinson 
monads for the suitably twisted ideal sheaves of the surfaces. Let us explain how
this works.

\vskip0.1cm
\noindent
{\it{Step 1.}}\; In this step we suppose that an elliptic conic
bundle $X$ as above exists, and we determine the type of the Beilinson
monad for the suitably twisted ideal sheaf $\mathcal J_X$.
We know from the classification of smooth surfaces in 
$\PP^4(\CC)$ which are contained in a cubic hypersurface (see \cite{EA:roth} and 
\cite{EA:aure-thesis}) that $\H^0 \mathcal J_X (i) = 0$ for $i\leq 3$. 
It follows from general results such as the theorem of Riemann-Roch that the 
dimensions $\h^j \mathcal J_X(i)$ in range $-2\leq i \leq 3$ are as follows  
(here, again, a zero is represented by an empty box):
\vskip0.2cm
%
%%%%%%%%%%%%%%%%%%%%%%%%%%%%%%%%%%%%%%%%%%%%%%%%%%%%%%%%%%%%%%%%%%%%%%%%%%%%%%%

% Neue L�ngen f�r Abst�nde horizontal und vertikal
% Nur einmal vor dem ersten Auftreten eines Beilinson-Diagramms
% \newlength{\br}
% \newlength{\ho}
%
%%%%%%%%%%%%%%%%%%%%%%%%%%%%%%%%%%%%%%%%%%%%%%%%%%%%%%%%%%%%%%%%%%%%%%%%%%%%%%%

{
$$ %Diagramm zentrieren
%
% W�hle die Einheiten \br horizontal und \ho vertikal
{
\setlength{\br}{9mm}
\setlength{\ho}{6mm}
\fontsize{10pt}{8pt}
\selectfont
\begin{xy}
%%%%%%%%%%%%%%%%%%%%%%%%%%%%%%%%%%%%%%%
%
% Achsenkreuz im Punkte (0,0)
%
% x-Achse von -3\br bis 5\br
% mit einem "j" an 0.95 der L�nge und 3mm unter der Achse:
%
,<-2.5\br,0\ho>;<5\br,0\ho>**@{-}?>*@{>}
?(0.98)*!/^3mm/{i}
%
% y-Achse von 0\ho bis 6\ho
% mit einem "i" an 0,9 der L�nge und 3mm rechts neben der Achse:
%
,<0\br,0\ho>;<0\br,6\ho>**@{-}?>*@{>}
?(0.98)*!/^3mm/{j}
%%%%%%%%%%%%%%%%%%%%%%%%%%%%%%%%%%%%%%%
%
% 5 waagrechte Linien von -2\br bis +4\br
% in den H�hen 1\ho,...,5\ho:
%
,0+<-2\br,1\ho>;<4\br,1\ho>**@{-}
,0+<-2\br,2\ho>;<4\br,2\ho>**@{-}
,0+<-2\br,3\ho>;<4\br,3\ho>**@{-}
,0+<-2\br,4\ho>;<4\br,4\ho>**@{-}
,0+<-2.5\br,5\ho>;<4.5\br,5\ho>**@{-}
%%%%%%%%%%%%%%%%%%%%%%%%%%%%%%%%%%%%%%%
%
% 11 senkrechte Linien von 0\ho bis 5\ho
% in den waagrechten Punkten -2\br,...,+3\br:
%
,0+<-2\br,0\ho>;<-2\br,5\ho>**@{-}
,0+<-1\br,0\ho>;<-1\br,5\ho>**@{-}
%
,0+<1\br,0\ho>;<1\br,5\ho>**@{-}
,0+<2\br,0\ho>;<2\br,5\ho>**@{-}
,0+<3\br,0\ho>;<3\br,5\ho>**@{-}
,0+<4\br,0\ho>;<4\br,5\ho>**@{-}
%
%%%%%%%%%%%%%%%%%%%%%%%%%%%%%%%%%%%%%%%
%
% Eintr�ge in den Mitten der K�sten. Daher die Koordinaten mit .5
%
,0+<-1.5\br,3.5\ho>*{8}
,0+<-0.5\br,3.5\ho>*{4}
%
,0+<0.5\br,2.5\ho>*{1}
,0+<1.5\br,2.5\ho>*{1}
,0+<2.5\br,2.5\ho>*{a}
,0+<3.5\br,2.5\ho>*{b}
,0+<2.5\br,1.5\ho>*{a+1}
,0+<3.5\br,1.5\ho>*{b+1}
%
%%%%%%%%%%%%%%%%%%%%%%%%%%%%%%%%%%%%%%%
,0+<-1.5\br,-0.6\ho>*{-2}
,0+<-0.5\br,-0.6\ho>*{-1}
,0+<0.5\br,-0.6\ho>*{0}
,0+<1.5\br,-0.6\ho>*{1}
,0+<2.5\br,-0.6\ho>*{2}
,0+<3.5\br,-0.6\ho>*{3}
%%%%%%%%%%%%%%%%%%%%%%%%%%%%%%%%%%%%%%%
,0+<-3.0\br,4.5\ho>*{4}
,0+<-3.0\br,3.5\ho>*{3}
,0+<-3.0\br,2.5\ho>*{2}
,0+<-3.0\br,1.5\ho>*{1}
,0+<-3.0\br,0.5\ho>*{0}
\end{xy} 
}
$$
}
%%%%%%%%%%%%%%%%%%%%%%%%%%%%%%%%%%%%%%%%%%%%%%%%%%%%%%%%%%%%%%%%%%%%%%%%%%%%%%%
\vskip0.1cm
\noindent
with $a := \h^2 \mathcal J_X(2)$ and $b := \h^2 \mathcal J_X(3)$ still to
be determined. The Beilinson monad for $\mathcal J_X(2)$ is thus of type
$$
0 \rightarrow 8\:\! \mathcal O (-1) \rightarrow 4\:\! U^3\oplus U^2
\rightarrow  U\oplus (a+1)\:\! \mathcal O \rightarrow a\:\! \mathcal O \rightarrow 0\ ,
$$
where $\,(a+1)\:\! \mathcal O \rightarrow a\:\! \mathcal O\,$ is the zero map
(see Remark \ref{diff1}), 
and where consequently $U$ is mapped surjectively onto $\,a\,\! \mathcal O\,$. By 
Proposition \ref{critsur} this is only possible if $a = 0$. The same idea applied to 
$\mathcal J_X(3)$ shows that then also $b = 0$. 

The cohomological information obtained so far determines the type of the Beilinson 
monad for $\mathcal J_X(2)$ and for $\mathcal J_X(3)$. We decide to concentrate on 
the monad for $\mathcal J_X(3)$ since its differentials are smaller in size than those of the 
monad for $\mathcal J_X(2)$. In order to ease our calculations further we kill
the 4-dimensional space $\H^3 \mathcal J_X(-1)$. Let us write $\omega_X$ for
the dualizing sheaf of $X$. Serre duality on $\PP^4(\CC)$ 
respectively on $X$  yields canonical isomorphisms
$$
\begin{aligned}
Z& :=  \Ext^1(\mathcal J_X(-1),\mathcal O(-5))\\
&\cong (\H^3 \mathcal J_X(-1))^*
\cong (\H^2 \mathcal O_X(-1))^*\cong \H^0(\omega_X(1))\ .
\end{aligned}
$$
The identity in 
$$
\Hom (Z, Z) \cong \Ext^1(\mathcal J_X(-1), Z^* \otimes \mathcal O(-5))
$$
defines an extension which, twisted by 4, can be written as
$$
0 \rightarrow 4\:\! \mathcal O(-1) \rightarrow \mathcal G
\rightarrow \mathcal J_X(3) \rightarrow 0\ .
$$
Let us show that $\mathcal G$ is a vector bundle.
We know from the classification of scrolls in $\PP^4(\CC)$ (see \cite{EA:Lanteri} 
and \cite{EA:aure-thesis}) that $X$ is not a scroll.
Hence adjunction theory implies that $\omega_X(1)$ is generated by the adjoint linear 
system $\H^0(\omega_X(1))$ (see \cite[Corollary 9.2.2]{EA:adj-theory}). It follows by
Serre's criterion (\cite{EA:MR16:953c}, see also \cite[Theorem 2.2]{EA:okreflexiv})
that $\mathcal G$ is locally free. By construction $\mathcal G$ has a 
cohomology table as follows:
\vskip0.2cm
%
%
%%%%%%%%%%%%%%%%%%%%%%%%%%%%%%%%%%%%%%%%%%%%%%%%%%%%%%%%%%%%%%%%%%%%%%%%%%%%%%%

% Neue L�ngen f�r Abst�nde horizontal und vertikal
% Nur einmal vor dem ersten Auftreten eines Beilinson-Diagramms
%\newlength{\br}
%\newlength{\ho}
%
%%%%%%%%%%%%%%%%%%%%%%%%%%%%%%%%%%%%%%%%%%%%%%%%%%%%%%%%%%%%%%%%%%%%%%%%%%%%%%%
{
$$ %Diagramm zentrieren
%
% W�hle die Einheiten \br horizontal und \ho vertikal
{
\setlength{\br}{9mm}
\setlength{\ho}{6mm}
\fontsize{10pt}{8pt}
\selectfont
\begin{xy}
%%%%%%%%%%%%%%%%%%%%%%%%%%%%%%%%%%%%%%%
%
% Achsenkreuz im Punkte (0,0)
%
% x-Achse von -5.5\br bis 1\br
% mit einem "j" an 0.95 der L�nge und 3mm unter der Achse:
%
,<-5.5\br,0\ho>;<1\br,0\ho>**@{-}?>*@{>}
?(0.98)*!/^3mm/{i}
%
% y-Achse von 0\ho bis 6\ho
% mit einem "i" an 0,9 der L�nge und 3mm rechts neben der Achse:
%
,<-1\br,0\ho>;<-1\br,6\ho>**@{-}?>*@{>}
?(0.98)*!/^3mm/{j}
%%%%%%%%%%%%%%%%%%%%%%%%%%%%%%%%%%%%%%%
%
% 5 waagrechte Linien von -5\br bis +0\br
% in den H�hen 1\ho,...,5\ho:
%
,0+<-5\br,1\ho>;<0\br,1\ho>**@{-}
,0+<-5\br,2\ho>;<0\br,2\ho>**@{-}
,0+<-5\br,3\ho>;<0\br,3\ho>**@{-}
,0+<-5\br,4\ho>;<0\br,4\ho>**@{-}
,0+<-5.5\br,5\ho>;<.5\br,5\ho>**@{-}
%%%%%%%%%%%%%%%%%%%%%%%%%%%%%%%%%%%%%%%
%
% 11 senkrechte Linien von 0\ho bis 5\ho
% in den waagrechten Punkten -8\br,...,+3\br:
%
,0+<-5\br,0\ho>;<-5\br,5\ho>**@{-}
,0+<-4\br,0\ho>;<-4\br,5\ho>**@{-}
,0+<-3\br,0\ho>;<-3\br,5\ho>**@{-}
,0+<-2\br,0\ho>;<-2\br,5\ho>**@{-}
,0+<0\br,0\ho>;<0\br,5\ho>**@{-}
%
%%%%%%%%%%%%%%%%%%%%%%%%%%%%%%%%%%%%%%%
%
% Eintr�ge in den Mitten der K�sten. Daher die Koordinaten mit .5
%
,0+<-3.5\br,2.5\ho>*{1}
,0+<-2.5\br,2.5\ho>*{1}
,0+<-1.5\br,1.5\ho>*{1}
,0+<-0.5\br,1.5\ho>*{1}
%
%%%%%%%%%%%%%%%%%%%%%%%%%%%%%%%%%%%%%%%
,0+<-4.5\br,-0.6\ho>*{-4}
,0+<-3.5\br,-0.6\ho>*{-3}
,0+<-2.5\br,-0.6\ho>*{-2}
,0+<-1.5\br,-0.6\ho>*{-1}
,0+<-0.5\br,-0.6\ho>*{0}
%%%%%%%%%%%%%%%%%%%%%%%%%%%%%%%%%%%%%%%
,0+<-6.0\br,4.5\ho>*{4}
,0+<-6.0\br,3.5\ho>*{3}
,0+<-6.0\br,2.5\ho>*{2}
,0+<-6.0\br,1.5\ho>*{1}
,0+<-6.0\br,0.5\ho>*{0}
\end{xy}
}
$$
}
%%%%%%%%%%%%%%%%%%%%%%%%%%%%%%%%%%%%%%%%%%%%%%%%%%%%%%%%%%%%%%%%%%%%%%%%%%%%%%%
\vskip0.1cm
\noindent
So the  Beilinson monad of $\mathcal G$ is of type
$$
0 \rightarrow U^3 \overset\alpha\rightarrow U^2\oplus U
\overset\beta\rightarrow \mathcal O \rightarrow 0\ .
$$

\vskip0.1cm
\noindent
{\it{Step 2.}}\; Now we proceed the other way around. We show that a 
rank 5 bundle $\mathcal G$  as in the first step exists, and that the dependency locus of 
four general sections of $\mathcal G(1)$ is a surface of the desired type. Differentials which 
define a monad as above with a locally free homology  can be easily found.
By Lemma \ref{hombd} $\alpha$ corresponds to a pair of vectors $\alpha=(\alpha_1, \alpha_2)^t
\in V \oplus \bigwedge^2 V$ . By dualizing (see Remark \ref{dualitybd}) we find that it 
is a vector bundle monomorphism  if and only if  
$U^2\oplus U^3\overset{\alpha^t}\longrightarrow U^1$ is an epimorphism. Equivalently, 
$\alpha_1$ is non-zero and $\alpha_2$ considered as a vector in 
$\bigwedge^2 (V/\langle \alpha_1 \rangle)$ is indecomposable (argue as in the proof of
Proposition \ref{critsur}). Taking the other monad conditions into account we see
that we may pick
$$
\alpha=
\begin{pmatrix}
e_4\\
e_0\wedge e_2 + e_1\wedge e_3
\end{pmatrix}
$$
and
$$
\beta=
\begin{pmatrix}
    e_0\wedge e_2 + e_1\wedge e_3\, ,&-e_4
\end{pmatrix} ,
$$
where $e_0 , \dots , e_4$ is a basis of $V$, and that up to isomorphisms of monads 
and up to the choice of the basis this is the only possibility. We fix $\mathcal G$ as
the homology of this monad and compute the syzygies of $\mathcal G$ with \Mtwo.
\vskip0.3cm


\beginOutput
i51 : S = ZZ/32003[x_0..x_4];\\
\endOutput
\beginOutput
i52 : E = ZZ/32003[e_0..e_4,SkewCommutative=>true];\\
\endOutput
\beginOutput
i53 : beta=map(E^1,E^\{-2,-1\},\{\{e_0*e_2+e_1*e_3,-e_4\}\})\\
\emptyLine
o53 = | e_0e_2+e_1e_3 -e_4 |\\
\emptyLine
\              1       2\\
o53 : Matrix E  <--- E\\
\endOutput
\beginOutput
i54 : alpha=map(E^\{-2,-1\},E^\{-3\},\{\{e_4\},\{e_0*e_2+e_1*e_3\}\})\\
\emptyLine
o54 = \{2\} | e_4           |\\
\      \{1\} | e_0e_2+e_1e_3 |\\
\emptyLine
\              2       1\\
o54 : Matrix E  <--- E\\
\endOutput
\beginOutput
i55 : beta=beilinson(beta,S);\\
\emptyLine
o55 : Matrix\\
\endOutput
\beginOutput
i56 : alpha=beilinson(alpha,S);\\
\emptyLine
o56 : Matrix\\
\endOutput
\beginOutput
i57 : G = prune homology(beta,alpha);\\
\endOutput
\beginOutput
i58 : betti res G\\
\emptyLine
o58 = total: 10 9 5 1\\
\          1: 10 4 1 .\\
\          2:  . 5 4 1\\
\endOutput


\vskip0.3cm
\noindent
We see in particular that $\mathcal G (1)$ is globally generated. Hence the
dependency locus of four general sections of $\mathcal G (1)$ is indeed a smooth surface
in $\PP^4(\CC)$ by Kleiman's Bertini-type result \cite{EA:Bertini}. The smoothness can
also be checked with \Mtwo in an example via the built-in Jacobian criterion
(see \cite{EA:DSJSC} for a speedier method).
\vskip0.3cm

\beginOutput
i59 : foursect = random(S^4, S^10) * presentation G;\\
\emptyLine
\              4       9\\
o59 : Matrix S  <--- S\\
\endOutput
\vskip0.1cm

\noindent
The function
{\tt{trim}} computes a minimal presentation.

\vskip0.1cm
\beginOutput
i60 : IX = trim minors(4,foursect);\\
\emptyLine
o60 : Ideal of S\\
\endOutput
 
\beginOutput
i61 : codim IX\\
\emptyLine
o61 = 2\\
\endOutput
\beginOutput
i62 : degree IX\\
\emptyLine
o62 = 8\\
\endOutput
\beginOutput
i63 : codim singularLocus IX\\
\emptyLine
o63 = 5\\
\endOutput

\vskip0.3cm
\noindent
By construction $X$ has the correct invariants and is in fact an elliptic conic bundle
as claimed: Since the adjoint linear system $\H^0(\omega_X(1))$ is base point free 
and 4-dimensional
by what has been  said in the first step, the corresponding adjunction map 
$X \rightarrow \PP^3$ is a morphism which exhibits, as is easy to see, $X$ as a
conic bundle over a smooth elliptic curve in $\PP^3$ 
(see \cite[Proposition 2.1]{EA:conicbundle}). 

\vskip0.1cm
\noindent
{\it{Step 3.}}\; Our discussion in the previous steps gives also a classification
result. Up to projectivities the elliptic conic bundles of degree 8 in $\PP^4(\CC)$ are 
precisely the smooth surfaces arising as the dependency locus of four sections of the 
bundle $\mathcal G(1)$ fixed in Step 2.\qed
%\end{Example}
\end{example}

%\begin{Example}
\begin{example}
%\noindent
%{\it {Example 7.2}}\, 
This example is concerned with the construction and
classification of \ie{abelian surface}s in $\PP^4(\CC)$, and
with the closely related
\ie{Horrocks-Mumford bundle}s \cite{EA:HM}.
\index{bundle!Horrocks-Mumford}
\vskip0.1cm
\noindent
{\it{Step 1.}}\; Horrocks and Mumford  found evidence for the 
existence of a family of abelian surfaces in $\PP^4(\CC)$. 
Suppose that such a surface $X$ exists. Then the dualizing sheaf of $X$ is trivial, 
$\omega_X\cong \mathcal O_X$, and $X$ has degree 10 (see \cite[Example 3.2.15]{EA:fultonit}).
The same arguments as in Example \ref{was7.1} show that $X$ arises as the zero scheme of a 
section of a rank 2 vector bundle: There is an extension 
$$
0 \rightarrow  \mathcal O \rightarrow \mathcal F (3)
\rightarrow \mathcal J_X(5) \rightarrow 0\ ,
$$
where $\mathcal F (3)$ is a rank 2 vector bundle with Chern classes $c_1 = 5$ and 
$c_2 = \deg X = 10$, and where $\mathcal F$ has a cohomology table as
displayed in Figure \ref{cohtable}.
\begin{figure}
%
%
%%%%%%%%%%%%%%%%%%%%%%%%%%%%%%%%%%%%%%%%%%%%%%%%%%%%%%%%%%%%%%%%%%%%%%%%%%%%%%%

% Neue L�ngen f�r Abst�nde horizontal und vertikal
% Nur einmal vor dem ersten Auftreten eines Beilinson-Diagramms
%\newlength{\br}
%\newlength{\ho}
%
%%%%%%%%%%%%%%%%%%%%%%%%%%%%%%%%%%%%%%%%%%%%%%%%%%%%%%%%%%%%%%%%%%%%%%%%%%%%%%%
{
$$ %Diagramm zentrieren
%
% W�hle die Einheiten \br horizontal und \ho vertikal
{
\setlength{\br}{9mm}
\setlength{\ho}{6mm}
\fontsize{10pt}{8pt}
\selectfont
\begin{xy}
%%%%%%%%%%%%%%%%%%%%%%%%%%%%%%%%%%%%%%%
%
% Achsenkreuz im Punkte (0,0)
%
% x-Achse von -5.5\br bis 1\br
% mit einem "j" an 0.95 der L�nge und 3mm unter der Achse:
%
,<-5.5\br,0\ho>;<1\br,0\ho>**@{-}?>*@{>}
?(0.98)*!/^3mm/{i}
%
% y-Achse von 0\ho bis 6\ho
% mit einem "i" an 0,9 der L�nge und 3mm rechts neben der Achse:
%
,<-1\br,0\ho>;<-1\br,6\ho>**@{-}?>*@{>}
?(0.98)*!/^3mm/{j}
%%%%%%%%%%%%%%%%%%%%%%%%%%%%%%%%%%%%%%%
%
% 5 waagrechte Linien von -5\br bis +0\br
% in den H�hen 1\ho,...,5\ho:
%
,0+<-5\br,1\ho>;<0\br,1\ho>**@{-}
,0+<-5\br,2\ho>;<0\br,2\ho>**@{-}
,0+<-5\br,3\ho>;<0\br,3\ho>**@{-}
,0+<-5\br,4\ho>;<0\br,4\ho>**@{-}
,0+<-5.5\br,5\ho>;<.5\br,5\ho>**@{-}
%%%%%%%%%%%%%%%%%%%%%%%%%%%%%%%%%%%%%%%
%
% 11 senkrechte Linien von 0\ho bis 5\ho
% in den waagrechten Punkten -8\br,...,+3\br:
%
,0+<-5\br,0\ho>;<-5\br,5\ho>**@{-}
,0+<-4\br,0\ho>;<-4\br,5\ho>**@{-}
,0+<-3\br,0\ho>;<-3\br,5\ho>**@{-}
,0+<-2\br,0\ho>;<-2\br,5\ho>**@{-}
,0+<0\br,0\ho>;<0\br,5\ho>**@{-}
%
%%%%%%%%%%%%%%%%%%%%%%%%%%%%%%%%%%%%%%%
%
% Eintr�ge in den Mitten der K�sten. Daher die Koordinaten mit .5
%
,0+<-4.5\br,3.5\ho>*{5}
,0+<-2.5\br,2.5\ho>*{2}
,0+<-0.5\br,1.5\ho>*{5}
%
%%%%%%%%%%%%%%%%%%%%%%%%%%%%%%%%%%%%%%%
,0+<-4.5\br,-0.6\ho>*{-4}
,0+<-3.5\br,-0.6\ho>*{-3}
,0+<-2.5\br,-0.6\ho>*{-2}
,0+<-1.5\br,-0.6\ho>*{-1}
,0+<-0.5\br,-0.6\ho>*{0}
%%%%%%%%%%%%%%%%%%%%%%%%%%%%%%%%%%%%%%%
,0+<-6.0\br,4.5\ho>*{4}
,0+<-6.0\br,3.5\ho>*{3}
,0+<-6.0\br,2.5\ho>*{2}
,0+<-6.0\br,1.5\ho>*{1}
,0+<-6.0\br,0.5\ho>*{0}
\end{xy}
}
$$
}
%%%%%%%%%%%%%%%%%%%%%%%%%%%%%%%%%%%%%%%%%%%%%%%%%%%%%%%%%%%%%%%%%%%%%%%%%%%%%%%
\caption{}\label{cohtable}
\end{figure}
In particular $\mathcal F$, which has Chern classes $c_1=-1$ and $c_2=4$, is stable by 
Remark 5.3. A discussion as in Section 5 shows that the Beilinson monad for $\mathcal F$ 
is of type 
$$
\xymatrix{
0\ar[r]& A\otimes \mathcal O(-1)\ar[r]^{\alpha}&
B \otimes U^2 \ar[r]^{\alpha^d}& A^*\otimes\mathcal O \ar[r]  & 0
}\ ,
$$
with $\CC$-vector spaces $A$ and $B$ of dimension 5 and 2 respectively,
and with $\alpha^d = \alpha^*(-1)\circ(q\otimes\iota)$,  where $q$ is
a symplectic form on $B$, and where 
$\iota : U^2 \overset{\cong}\longrightarrow (U^2)^*(-1)$ is induced
by the pairing $U^2 \otimes U^2 \overset\wedge\longrightarrow U^4\cong\mathcal O(-1)$.
By choosing appropriate bases of $A$ and $B$ we may suppose
that $\alpha$ is a $2\times 5$ matrix with entries in $\bigwedge^2 V$ and that
$\alpha^d = \alpha^t\cdot \begin{pmatrix} 0 & 1\\ -1 & 0\end{pmatrix}$.

\vskip0.1cm
\noindent
{\it{Step 2.}}\; As in Example 7.1 we now proceed the other way around.
But this time it is not obvious  how to define $\alpha$.
Horrocks and Mumford remark that up to projectivities
one may suppose that the abelian surfaces in $\P^4(\CC)$ are invariant under
the action of the {\ie Heisenberg group} $H_5$ in its Schr\"odinger representation, and
they use the representation theory of $H_5$ and its normalizer $N_5$ in 
$\SL(5, \CC)$ to find 
$$
\alpha = 
\begin{pmatrix}
e_2\wedge e_3\;&e_3\wedge e_4\;&e_4\wedge e_0\;&e_0\wedge e_1\;&e_1\wedge e_2\\
e_1\wedge e_4\;&e_2\wedge e_0\;&e_3\wedge e_1\;&e_4\wedge e_2\;&e_0\wedge e_3
\end{pmatrix}\ ,
$$
where $e_0 ,\dots , e_4$ is a basis of $V$. A straightforward computation shows
that with this $\alpha$ the desired monad conditions are indeed satisfied. The resulting
Horrocks-Mumford bundle $\mathcal F_{\text{HM}}$ on $\PP^4(\CC)$ is essentially 
the only rank 2 vector bundle known on $\PP^n(\CC)$, $n\geq 4$, which does not 
split as direct sum of two line bundles. Let us compute the syzygies of 
$\mathcal F_{\text{HM}}$ with \Mtwo.
\vskip0.3cm

\beginOutput
i64 : alphad = matrix\{\{e_4*e_1, e_2*e_3\},\{e_0*e_2, e_3*e_4\},\\
\                      \{e_1*e_3, e_4*e_0\},\{e_2*e_4, e_0*e_1\},\\
\                      \{e_3*e_0, e_1*e_2\}\};\\
\emptyLine
\              5       2\\
o64 : Matrix E  <--- E\\
\endOutput
\beginOutput
i65 : alphad=map(E^5,E^\{-2,-2\},alphad)\\
\emptyLine
o65 = | -e_1e_4 e_2e_3  |\\
\      | e_0e_2  e_3e_4  |\\
\      | e_1e_3  -e_0e_4 |\\
\      | e_2e_4  e_0e_1  |\\
\      | -e_0e_3 e_1e_2  |\\
\emptyLine
\              5       2\\
o65 : Matrix E  <--- E\\
\endOutput
\beginOutput
i66 : alpha=syz alphad\\
\emptyLine
o66 = \{2\} | e_2e_3 e_0e_4 e_1e_2 -e_3e_4 e_0e_1  |\\
\      \{2\} | e_1e_4 e_1e_3 e_0e_3 e_0e_2  -e_2e_4 |\\
\emptyLine
\              2       5\\
o66 : Matrix E  <--- E\\
\endOutput
\beginOutput
i67 : alphad=beilinson(alphad,S);\\
\emptyLine
o67 : Matrix\\
\endOutput
\beginOutput
i68 : alpha=beilinson(alpha,S);\\
\emptyLine
o68 : Matrix\\
\endOutput
\beginOutput
i69 : FHM = prune homology(alphad,alpha);\\
\endOutput
\beginOutput
i70 : betti res FHM\\
\emptyLine
o70 = total: 19 35 20 2\\
\          3:  4  .  . .\\
\          4: 15 35 20 .\\
\          5:  .  .  . 2\\
\endOutput
\beginOutput
i71 : regularity FHM\\
\emptyLine
o71 = 5\\
\endOutput
\beginOutput
i72 : betti sheafCohomology(presentation FHM,E,-6,6)\\
\emptyLine
o72 = total: 210 100 37 14 10 5 2 5 10 14 37 100 210\\
\         -6: 210 100 35  4  . . . .  .  .  .   .   .\\
\         -5:   .   .  2 10 10 5 . .  .  .  .   .   .\\
\         -4:   .   .  .  .  . . 2 .  .  .  .   .   .\\
\         -3:   .   .  .  .  . . . 5 10 10  2   .   .\\
\         -2:   .   .  .  .  . . . .  .  4 35 100 210\\
\endOutput
 

\vskip0.3cm
\noindent
Since $\H^0 \mathcal F_{\text{HM}}(i) = 0$ for $i<3$ every non-zero section of 
$\mathcal F_{\text{HM}}(3)$ vanishes along a surface (with the desired invariants).
Horrocks and Mumford need an extra argument to show that the general such surface
is smooth (and thus abelian) since Kleiman's Bertini-type result does not apply
($\mathcal F_{\text{HM}}(3)$ is not globally generated). Our explicit construction 
allows one again to check the smoothness with \Mtwo in an example.
\vskip0.3cm

\beginOutput
i73 : sect =  map(S^1,S^15,0) | random(S^1, S^4);\\
\emptyLine
\              1       19\\
o73 : Matrix S  <--- S\\
\endOutput
\vskip0.1cm

\noindent
We compute the equations of $X$ via a mapping cone.
\vskip0.1cm

\beginOutput
i74 : mapcone = sect || transpose presentation FHM;\\
\emptyLine
\              36       19\\
o74 : Matrix S   <--- S\\
\endOutput
\beginOutput
i75 : fmapcone = res coker mapcone;\\
\endOutput
\beginOutput
i76 : IX =  trim ideal fmapcone.dd_2;\\
\emptyLine
o76 : Ideal of S\\
\endOutput
\beginOutput
i77 : codim IX\\
\emptyLine
o77 = 2\\
\endOutput
\beginOutput
i78 : degree IX\\
\emptyLine
o78 = 10\\
\endOutput
\beginOutput
i79 : codim singularLocus IX\\
\emptyLine
o79 = 5\\
\endOutput


\vskip0.3cm
\noindent
{\it{Step 3.}}\; Horrocks and Mumford showed that up to projectivities
every abelian surface in $\PP^4(\CC)$ arises as the zero scheme of a section
of $\mathcal F_{\text{HM}}(3)$. In fact, one can show much more.
By a careful analysis of possible Beilinson monads and their restrictions
to various linear subspaces Decker \cite{EA:uniquenesshm1} proved that every stable rank 2
vector bundle $\mathcal F$ on $\PP^4(\CC)$ with Chern classes $c_1=-1$ and $c_2=4$
is the homology of a monad of the type as in Step 1. From geometric properties of the 
``variety of unstable planes''  of $\mathcal F$  Decker and Schreyer \cite{EA:uniquenesshm2} 
deduced that up to isomorphisms and projectivities the differentials of the monad coincide 
with those of $\mathcal F_{\text{HM}}$. Together with results from \cite{EA:decker24} 
this implies that the moduli space of our bundles is isomorphic to the homogeneous 
space $\SL(5, \CC)/N_5.$\qed
%\end{Example}
\end{example}

%\textbf{EISENBUD ENDE}

% \section*{References}

% use \cite{EA:MR92g:14013} instead.
% \noindent\textbf{Ancona, V. \& Ottaviani, G.}:
%         {\sl An introduction to derived categories and the theorem of Beilinson},
%         Atti Accademia Peloritana dei Pericolanti, Classe I de Scienze 
%         Fis. Mat. et Nat. LXVII, 99-110 (1989)

% use \cite{EA:MR57:324} instead
% \noindent\textbf{Barth, W.}: 
%         {\sl Moduli of vector bundles on the projective plane},
%         Invent. math. {\bf 42}, 63-91 (1977)


% use \cite{EA:MR80f:14005} instead
%\noindent\textbf{Barth, W. \& Hulek, K.}: 
%        {\sl Monads and Moduli of Vector Bundles}, 
%        manuscripta math. {\bf 25}, 323-347 (1978)

% use \cite{EA:MR80c:14010b} instead
% \noindent\textbf{Beilinson, A}:
%         {\sl Coherent sheaves on $\P^n$ and problems of linear algebra},
%         Funct. Anal. and its Appl. {\bf 12}, 214-216 (1978)

% use \cite{EA:MR80c:14010a} instead
% \noindent\textbf{Bernstein, Gel'fand, and Gel'fand}:
%         {\sl Algebraic bundles on $\P^n$ and problems of linear algebra},
%         Funct. Anal. and its Appl. {\bf 12}, 212-214 (1978)

% use \cite{EA:MR89g:13005:appendix} instead
% \noindent\textbf{Buchweitz, R.-O.}:
%         Appendix to Cohen-Macaulay modules on quadrics, by
%         R.-O. Buchweitz, D. Eisenbud, and J. Herzog. In
%         {\sl Singularities, representation of algebras,
%         and vector bundles} (Lambrecht, 1985),
%         Springer-Verlag Lecture Notes in Math, 1273, 96-116 (1987)


% use \cite{EA:MR97a:13001} instead
% \noindent\textbf{Eisenbud,  David}: 
%         {\sl Commutative Algebra with a View Toward Algebraic Geometry},
%         Springer Verlag, 1995

% use \cite{EA:MR1484973:eisenbud} instead
% \noindent\textbf{Eisenbud,  David}: 
%         {\sl Computing cohomology} in Chapter of 
%         ``Computational methods in Commutative Algebra and Algebraic Geometry'' 
%         by ~W.~Vasconcelos, Springer Verlag, Berlin, 1998

% use \cite{EA:Eisenbud-Schreyer:ChowForms} instead
% \noindent\textbf{Eisenbud, D.  \&  Schreyer, F.-O.}: 
%         {\sl Chow forms and free resolution}, in preparation 2001

% use \cite{EA:eis-sch:sheaf} instead
% \noindent\textbf{Eisenbud, D.  \&  Schreyer, F.-O.}: 
%         {\sl Sheaf Cohomology and Free Resolutions over Exterior Algebras},
%         AG/0005055, 2000

% use \cite{EA:MR81h:14014}% instead
% \noindent\textbf{Gieseker, D.}:
%         {\sl On the moduli of vector bundles on an algebraic surface},
%         Ann. of Math. {\bf 106}, 45-60 (1977)

% use \cite{EA:MR80m:14011}% instead
% \noindent\textbf{Hulek, Klaus}:
%         {\sl Stable Rank-2 Vector Bundles on $\P_2$ with $c_1$ odd},
%         Math. Ann. {\bf 242}, 241-266 (1979)

% use \cite{EA:MR30:120}% instead
% \noindent\textbf{Horrocks, G.}: 
%         {\sl Vector bundles on the punctured spectrum of a local ring},
%         Proc. London Math. Soc. (3), {\bf 14}, 689-713 (1964)

% use \cite{EA:MR84j:14026}% instead
% \noindent\textbf{Horrocks, G.}: 
%         {\sl Construction of bundles on $\PP^n$} in ``Les equations de Yang-Mills'',
%         by A. Douady, J.-L. Verdier (eds.), Asterisque {\bf 71-72}, 197-202 (1980)


% use \cite{EA:MR89g:18018}% instead
% \noindent\textbf{Kapranov, M. M.}:
%         {\sl On the derived categories of coherent sheaves on some 
%         homogeneous spaces}, Invent. Math. {\bf 92}, 479-508 (1988)

% use \cite{EA:MR80m:14012} instead
% \noindent\textbf{Le Potier, J.}:
%         {\sl Fibr\'es stables de rang 2 sur $\P_2 (\CC)$},
%         Math. Ann. {\bf 241}, 217-256 (1979)

% use \cite{EA:MR56:8567}% instead
% \noindent\textbf{Maruyama, M.}:
%         {\sl Moduli of stable sheaves I},
%         J. Math. Kyoto Univ. {\bf 17}, 91-126 (1977)

% use \cite{EA:MR82h:14011}% instead
% \noindent\textbf{Maruyama, M.}:
%         {\sl Moduli of stable sheaves II},
%          J. Math. Kyoto Univ. {\bf 18}, 557-614 (1978)


% use \cite{EA:MR81b:14001}% instead
% \noindent\textbf{Okonek, C., Schneider, M. \& Spindler, H.}:
%          {\sl Vector bundles on complex projective spaces}, Boston, 1980


% use \cite{EA:MR16:953c}% instead
% \noindent\textbf{Serre, J.P.}: 
%         {\sl Faisceaux alg\'ebriques coherents},
%         Ann. of Math. {\bf 61}, 197-278  (1955)

% use \cite{EA:MR99f:14064}% instead
% \noindent\textbf{Walter, Charles H.}:
%          {\sl Pfaffian subschemes}, 
%         J. Algebr. Geom. {\bf 5}, 671-704 (1996)

% % Local Variables:
% % mode: latex
% % mode: reftex
% % tex-main-file: "chapter-wrapper.tex"
% % reftex-keep-temporary-buffers: t
% % reftex-use-external-file-finders: t
% % reftex-external-file-finders: (("tex" . "make FILE=%f find-tex") ("bib" . "make FILE=%f find-bib"))
% % TeX-master: "~/M2BUCH/ComputationsBook/chapters/exterior-alge"
% % End:
\begin{thebibliography}{10}

\bibitem{EA:conicbundle}
H.~Abo, W.~Decker, and N.~Sasakura:
\newblock An elliptic conic bundle on ${\bf {\char 80}}\sp{4}$ arising from a
  stable rank-$3$ vector bundle.
\newblock {\em Math. Z.}, 229:725--741, 1998.

\bibitem{EA:MR92g:14013}
V.~Ancona and G.~Ottaviani:
\newblock An introduction to the derived categories and the theorem of
  {B}eilinson.
\newblock {\em Atti Accad. Peloritana Pericolanti Cl. Sci. Fis. Mat. Natur.},
  67:99--110 (1991), 1989.

\bibitem{EA:aure-thesis}
A.~Aure:
\newblock On surfaces in projective 4-space.
\newblock PhD Thesis, Oslo, 1987.

\bibitem{EA:MR57:324}
W.~Barth:
\newblock Moduli of vector bundles on the projective plane.
\newblock {\em Invent. Math.}, 42:63--91, 1977.

\bibitem{EA:MR80f:14005}
W.~Barth and K.~Hulek:
\newblock Monads and moduli of vector bundles.
\newblock {\em Manuscripta Math.}, 25:323--347, 1978.

\bibitem{EA:MR80c:14010b}
A.~A. Be{\u\i}linson:
\newblock Coherent sheaves on ${\bf {\char 80}}\sp{n}$ and problems in linear
  algebra.
\newblock {\em Functional Anal. Appl.}, 12:214--216, 1978.

\bibitem{EA:adj-theory}
M.C. Beltrametti and A.~Sommese:
\newblock {\em The adjunction theory of complex projective varieties}.
\newblock de Gruyter, Berlin, 1995.

\bibitem{EA:MR80c:14010a}
I.~N. Bern{\v{s}}te{\u\i}n, I.~M. Gel{'}fand, and S.~I. Gel{'}fand:
\newblock Algebraic vector bundles on ${\bf {\char 80}}\sp{n}$\ and problems of
  linear algebra.
\newblock {\em Functional Anal. Appl.}, 12:212--214, 1978.

\bibitem{EA:BR}
R.~Braun and K.~Ranestad:
\newblock Conic bundles in projective fourspace.
\newblock In P.~Newstead, editor, {\em Algebraic geometry. Papers presented for
  the EUROPROJ conferences held in Catania, Italy, September 1993, and
  Barcelona, Spain, September 1994}, pages 331--339. Marcel Dekker, New York,
  1994.

\bibitem{EA:MR89g:13005:appendix}
R.-O. Buchweitz:
\newblock Appendix to {C}ohen-{M}acaulay modules on quadrics.
\newblock In {\em Singularities, representation of algebras, and vector bundles
  (Lambrecht, 1985)}, pages 96--116. Springer, Berlin, 1987.

\bibitem{EA:decker24}
W.~Decker:
\newblock Das {H}orrocks-{M}umford-{B}\"undel und das {M}odul-{S}chema f\"ur
  stabile $2$-{V}ektorb\"undel \"uber ${\bf {\char 80}}_4$ mit $c_1$=-1,
  $c_2$=4.
\newblock {\em Math. Z.}, 188:101--110, 1984.

\bibitem{EA:uniquenesshm1}
W.~Decker:
\newblock Stable rank 2 vector bundles with {C}hern-classes $c_1$=-1, $c_2$=4.
\newblock {\em Math. Ann.}, 275:481--500, 1986.

\bibitem{EA:cam}
W.~Decker:
\newblock Monads and cohomoloy modules of rank 2 vector bundles.
\newblock {\em Compositio Math.}, 76:7--17, 1990.

\bibitem{EA:uniquenesshm2}
W.~Decker and F.-O. Schreyer:
\newblock On the uniqueness of the {H}orrocks-{M}umford-bundle.
\newblock {\em Math. Ann.}, 273:415--443, 1986.

\bibitem{EA:DSJSC}
W.~Decker and F.-O. Schreyer:
\newblock Non-general type surfaces in ${\bf {\char 80}}^4$: Some remarks on
  bounds and constructions.
\newblock {\em J. Symbolic Computation}, 29:545--582, 2000.

\bibitem{EA:MR1484973:eisenbud}
D.~Eisenbud:
\newblock Computing cohomology.
\newblock A chapter in \cite{EA:VAS}.

\bibitem{EA:MR97a:13001}
D.~Eisenbud:
\newblock {\em Commutative algebra with a view toward algebraic geometry}.
\newblock Springer-Verlag, New York, 1995.

\bibitem{EA:eis-sch:sheaf}
D.~Eisenbud, G.~Fl{\o}ystad, and F.-O. Schreyer:
\newblock Sheaf cohomology and free resolutions over exterior algebras.
\newblock Work in progress, 2001.

\bibitem{EA:Eisenbud-Schreyer:ChowForms}
D.~Eisenbud and F.-O. Schreyer:
\newblock Resultants, chow forms, and free resolutions.
\newblock In preparation, 2001.

\bibitem{EA:ES}
P.~Ellia and G.~Sacchiero:
\newblock Smooth surfaces of ${\bf {\char 80}}\sp{4}$\ ruled in conics.
\newblock In P.~Newstead, editor, {\em Algebraic geometry. Papers presented for
  the EUROPROJ conferences held in Catania, Italy, September 1993, and
  Barcelona, Spain, September 1994}, pages 49--62. Marcel Dekker, New York,
  1994.

\bibitem{EA:fultonit}
W.~Fulton:
\newblock {\em Intersection theory}.
\newblock Springer-Verlag, New York, 1984.

\bibitem{EA:MR81h:14014}
D.~Gieseker:
\newblock On the moduli of vector bundles on an algebraic surface.
\newblock {\em Ann. of Math. (2)}, 106:45--60, 1977.

\bibitem{EA:MR30:120}
G.~Horrocks:
\newblock Vector bundles on the punctured spectrum of a local ring.
\newblock {\em Proc. London Math. Soc. (3)}, 14:689--713, 1964.

\bibitem{EA:MR84j:14026}
G.~Horrocks:
\newblock Construction of bundles on ${\bf {\char 80}}\sp{n}$.
\newblock In A.~Douady and J.-L. Verdier, editors, {\em Les \'equations de
  {Y}ang-{M}ills}, pages 197--203. Soci\'et\'e Math\'ematique de France, Paris,
  1980.
\newblock S\'eminaire E. N. S., 1977-1978, Ast\'erisqe 71--72.

\bibitem{EA:HM}
G.~Horrocks and D.~Mumford:
\newblock A rank 2 vector bundle on ${\bf {\char 80}}\sp{4}$\ with 15,000
  symmetries.
\newblock {\em Topology}, 12:63--81, 1973.

\bibitem{EA:MR80m:14011}
Klaus Hulek:
\newblock Stable rank-$2$\ vector bundles on ${\bf {\char 80}}\sb{2}$\ with
  $c\sb{1}$\ odd.
\newblock {\em Math. Ann.}, 242(3):241--266, 1979.

\bibitem{EA:ionescu}
P.~Ionescu:
\newblock Embedded projective varieties of small invariants {I}{I}{I}.
\newblock In A.J. Sommese, A.~Biancofiore, and E.~Livorni, editors, {\em
  Algebraic Geometry (L`A\-qui\-la 1988)}, pages 138--154. Springer, New York,
  1990.

\bibitem{EA:MR89g:18018}
M.~M. Kapranov:
\newblock On the derived categories of coherent sheaves on some homogeneous
  spaces.
\newblock {\em Invent. Math.}, 92(3):479--508, 1988.

\bibitem{EA:Bertini}
S.~Kleiman:
\newblock Geometry on grassmanians and applications to splitting bundles and
  smoothing cycles.
\newblock {\em Publ. Math. I.H.E.S.}, 36:281--297, 1969.

\bibitem{EA:Lanteri}
A.~Lanteri:
\newblock On the existence of scrolls in ${\bf {\char 80}}\sp{4}$.
\newblock {\em Atti Accad. Naz. Lincei, VIII. Ser., Rend., Cl. Sci. Fis. Mat.
  Nat.}, 69:223--227, 1980.

\bibitem{EA:MR80m:14012}
J.~Le~Potier:
\newblock Fibr\'es stables de rang $2$\ sur ${\bf {\char 80}}\sb{2}({\bf {\char
  67}})$.
\newblock {\em Math. Ann.}, 241:217--256, 1979.

\bibitem{EA:nico}
N.~Manolache:
\newblock Syzygies of abelian surfaces embedded in ${\bf {\char 80}}^4({\bf
  {\char 67}})$.
\newblock {\em J. reine angew. Math.}, 384:180--191, 1988.

\bibitem{EA:MR56:8567}
M.~Maruyama:
\newblock Moduli of stable sheaves {I}.
\newblock {\em J. Math. Kyoto Univ.}, 17:91--126, 1977.

\bibitem{EA:MR82h:14011}
M.~Maruyama:
\newblock Moduli of stable sheaves {I}{I}.
\newblock {\em J. Math. Kyoto Univ.}, 18:557--614, 1978.

\bibitem{EA:okreflexiv}
Ch. Okonek:
\newblock Reflexive {G}arben auf ${\bf {\char 80}}\sp{4}$.
\newblock {\em Math. Ann.}, 260:211--237, 1982.

\bibitem{EA:okgrad8}
Ch. Okonek:
\newblock Fl\"achen vom {G}rad 8 im ${\bf {\char 80}}\sp{4}$.
\newblock {\em Math. Z.}, 191:207--223, 1986.

\bibitem{EA:MR81b:14001}
Ch. Okonek, M.~Schneider, and H.~Spindler:
\newblock {\em Vector bundles on complex projective spaces}.
\newblock Birkh\"auser Boston, Mass., 1980.

\bibitem{EA:roth}
L.~Roth:
\newblock On the projective classification of surfaces.
\newblock {\em Proc. of London Math. Soc.}, 42:142--170, 1937.

\bibitem{EA:MR16:953c}
J.-P. Serre:
\newblock Faisceaux alg\'ebriques coh\'erents.
\newblock {\em Ann. of Math. (2)}, 61:197--278, 1955.

\bibitem{smith}
Gregory~G. Smith:
\newblock Computing global extension modules.
\newblock {\em J. Symbolic Comput.}, 29(4-5):729--746, 2000.
\newblock Symbolic computation in algebra, analysis, and geometry (Berkeley,
  CA, 1998).

\bibitem{EA:VAS}
Wolmer~V. Vasconcelos:
\newblock {\em Computational methods in commutative algebra and algebraic
  geometry}.
\newblock Springer-Verlag, Berlin, 1998.
\newblock With chapters by David Eisenbud, Daniel R. Grayson, J\"urgen Herzog
  and Michael Stillman.

\bibitem{EA:MR99f:14064}
C.~H. Walter:
\newblock Pfaffian subschemes.
\newblock {\em J. Algebraic Geom.}, 5:671--704, 1996.

\end{thebibliography}
\egroup
\makeatletter
\renewcommand\thesection{\@arabic\c@section}
\makeatother



%%%%%%%%%%%%%%%%%%%%%%%%%%%%%%%%%%%%%%%%%%%%%%%%
%%%%%
%%%%% ../chapters/constructions/chapter
%%%%%
%%%%%%%%%%%%%%%%%%%%%%%%%%%%%%%%%%%%%%%%%%%%%%%%

\bgroup
% $Source: /home/cvs/M2/Macaulay2/ComputationsBook/chapters/constructions/chapter.tex,v $
% $Revision: 1.30 $
% $Date: 2001/03/16 22:03:06 $

\title{Needles in a Haystack:\\Special Varieties via Small Fields}
\titlerunning{Needles in a Haystack: Special Varieties via Small Fields}
\toctitle{Needles in a Haystack: Special Varieties via Small Fields}
\author{Frank-Olaf Schreyer
        % \inst 1
         \and Fabio Tonoli
        % \inst 2
        }
\authorrunning{F-O. Schreyer and F. Tonoli}
% \institute{Fakult\"at f\"ur Mathematik und Physik,
%         Universit\"at Bayreuth, D-95440 Bayreuth, Germany
%         \and
%         Mathematisches Institut, Georg--August Universit\"at, D-37073
%         G\"ottingen, Germany}

% \newtheorem{lemma}{Lemma}[section]
% \newtheorem{proposition}[lemma]{Proposition}
% \newtheorem{theorem}[lemma]{Theorem}
% \newtheorem{corollary}[lemma]{Corollary}
\newtheorem{conjecture}[theorem]{Conjecture}{\itshape}{\rm}

% \theoremstyle{definition}
% \newtheorem{definition}[theorem]{Definition}
% \newtheorem{remark}{Remark}[section]
% \newtheorem{example}{Example}[section]
% \newtheorem{exercise}{Exercise}[section]
% \newtheorem{algorithm}{Algorithm}[subsection]
% \newtheorem{sub}[subsubsection]{}


\newcommand{\AAA}{{\mathbb A}}
\newcommand{\BB}{{\mathbb B}}
\newcommand{\CC}{{\mathbb C}}
\newcommand{\DD}{{\mathbb D}}
\newcommand{\EE}{{\mathbb E}}
\newcommand{\FF}{{\mathbb F}}
\newcommand{\GG}{{\mathbb G}}
%\newcommand{\HH}{{\mathbb H}}
\newcommand{\II}{{\mathbb I}}
\newcommand{\JJ}{{\mathbb J}}
\newcommand{\KK}{{\mathbb K}}
%\newcommand{\LL}{{\mathbb L}}
\newcommand{\MM}{{\mathbb M}}
\newcommand{\NN}{{\mathbb N}}
%\newcommand{{\mathbb O}}
\newcommand{\PP}{{\mathbb P}}
\newcommand{\QQ}{{\mathbb Q}}
\newcommand{\RR}{{\mathbb R}}
\renewcommand{\SS}{{\mathbb S}}
\newcommand{\Ss}{{\mathbf S}}
\newcommand{\TT}{{\mathbb T}}
\newcommand{\UU}{{\mathbb U}}
\newcommand{\VV}{{\mathbb V}}
\newcommand{\WW}{{\mathbb W}}
\newcommand{\XX}{{\mathbb X}}
\newcommand{\YY}{{\mathbb Y}}
\newcommand{\ZZ}{{\mathbb Z}}

\newcommand{\LL}{{\mathbb L}}
\newcommand{\HH}{{\mathbb H}}
\newcommand{\Syz}{{\rm{Syz}\;}}
\newcommand{\Sym}{{\rm{Sym}\;}}
\newcommand{\SSyz}{{\rm{Syz}}}
\newcommand{\spoly}{{\rm{spoly}}}
\newcommand{\Spe}{{Sp}}
\newcommand{\openC}{{\mathbb C}}
\newcommand{\ms}{{\rm{m}}}
\newcommand{\LS}{{\rm{L}}}
\newcommand{\IS}{{\rm{I}}}
\newcommand{\Loc}{{\rm{Loc}\,}}
\newcommand{\lcm}{{\rm{lcm}}}
\newcommand{\lc}{{\rm{lc}}}
\newcommand{\lm}{{\rm{lm}}}
\newcommand{\con}{{\rm{c}}}
\newcommand{\ext}{{\rm{e}}}
\newcommand{\ec}{{\rm{ec}}}
\newcommand{\ann}{{\rm{ann}}}
% \newcommand{\Ext}{{\rm{Ext}}}
\newcommand{\equi}{{\rm{equi}}}
\newcommand{\Tor}{{\rm{Tor}}}
\newcommand{\rad}{{\rm{rad\;}}}
\newcommand{\ini}{{\rm{in}}}
\newcommand{\Hilb}{{\rm{Hilb}}}
\newcommand{\image}{{\rm{image}}}
\newcommand{\cliff}{{\rm{cliff}}}
\newcommand{\Pic}{{\rm{Pic}}}
\newcommand{\PR}{{\KK [x_1, \dots , x_n]}}
%%%%%%%%%%%%%%%%%%%%%%%%%%%%%
%%% new commands for calligraphic characters with amsmath
%%%%%%%%%%%%%%%%%%%%%%%%%%%%

\newcommand{\ka}{{\mathcal A}}
\newcommand{\kb}{{\mathcal B}}
\newcommand{\kc}{{\mathcal C}}
\newcommand{\kd}{{\mathcal D}}
\newcommand{\ke}{{\mathcal E}}
\newcommand{\kf}{{\mathcal F}}
\newcommand{\kg}{{\mathcal G}}
\newcommand{\kh}{{\mathcal H}}
\newcommand{\ki}{{\mathcal I}}
\newcommand{\kj}{{\mathcal J}}
\newcommand{\kk}{{\mathcal K}}
\newcommand{\kl}{{\mathcal L}}
\newcommand{\km}{{\mathcal M}}
\newcommand{\kn}{{\mathcal N}}
\newcommand{\ko}{{\mathcal O}}
\newcommand{\kp}{{\mathcal P}}
\newcommand{\kq}{{\mathcal Q}}
\newcommand{\kr}{{\mathcal R}}
\newcommand{\ks}{{\mathcal S}}
\newcommand{\kt}{{\mathcal T}}
\newcommand{\ku}{{\mathcal U}}
\newcommand{\kv}{{\mathcal V}}
\newcommand{\kw}{{\mathcal W}}
\newcommand{\kx}{{\mathcal X}}
\newcommand{\ky}{{\mathcal Y}}
\newcommand{\kz}{{\mathcal Z}}
%%%%%%%%%%%%%%%%%%%%%%%%%%%%%%
%%%The mathscript for sheaves
%%%%%%%%%%%%%%%%%%%%% %%%%%%%%%
\newcommand{\s}{\mathscr}
\newcommand{\sA}{{\s A}}
\newcommand{\sB}{{\s B}}
\newcommand{\sC}{{\s C}}
\newcommand{\sD}{{\s D}}
\newcommand{\sE}{{\s E}}
\newcommand{\sF}{{\s F}}
\newcommand{\sG}{{\s G}}
\newcommand{\sH}{{\s H}}
\newcommand{\sI}{{\s I}}
\newcommand{\sJ}{{\s J}}
\newcommand{\sK}{{\s K}}
\newcommand{\sL}{{\s L}}
\newcommand{\sM}{{\s M}}
\newcommand{\sN}{{\s N}}
\newcommand{\sO}{{\s O}}
\newcommand{\sP}{{\s P}}
\newcommand{\sQ}{{\s Q}}
\newcommand{\sR}{{\s R}}
\newcommand{\sS}{{\s S}}
\newcommand{\sT}{{\s T}}
\newcommand{\sU}{{\s U}}
\newcommand{\sV}{{\s V}}
\newcommand{\sW}{{\s W}}
\newcommand{\sX}{{\s X}}
\newcommand{\sY}{{\s Y}}
\newcommand{\sZ}{{\s Z}}



\newcommand{\cO}{{\s O}}
\newcommand{\cI}{{\s I}}
% \newcommand{\cL}{{\s L}}
% \newcommand{\cR}{{\s R}}
\newcommand{\cN}{{\s N}}
\newcommand{\cT}{{\s T}}
\newcommand{\cX}{{\s X}}
%%%%%%%%%%%%%%%%%%%%%%%%%%%%%%%%
%% Arrows
%%%%%%%%%%%%%%%%%%%%%%%%%%%%%%%
\newcommand{\inj}{\hookrightarrow}
%\newcommand{\surj}{\lra}
\newcommand{\lra}{\longrightarrow}
\newcommand{\lla}{\longleftarrow}
\def\lto{\leftarrow}
%%%%%%%%%%%%%%%%%%%%%%%%%%%%%%%%%%%%
%\newcommand{\C}{\C}
%\newcommand{\openP}{\P}
\newcommand{\uf}{{\bf F}}
\newcommand{\uc}{{\bf C}}
\newcommand{\tensor}{\otimes}
\newcommand{\mi}{{\bf m}}
\newcommand{\tX}{\widetilde{X}}
\newcommand{\punkt}{\hspace{-.3ex}\raise.15ex\hbox to1ex{\Huge.}}
\newcommand{\tpunkt}{\hspace{-.3ex}\hbox to1ex{\Huge.}}
% \newlength{\br}
% \newlength{\ho}
\def\DeclareMathOperator#1#2{\def#1{\operatorname{#2}}}
\DeclareMathOperator{\det}{det}
\DeclareMathOperator{\GL}{GL}
\DeclareMathOperator{\Aut}{Aut}
\DeclareMathOperator{\Oo}{O}
\DeclareMathOperator{\Spec}{Spec}
\DeclareMathOperator{\Hom}{Hom}
\DeclareMathOperator{\syz}{syz}
\DeclareMathOperator{\ord}{ord}
\DeclareMathOperator{\word}{w\,ord}
\DeclareMathOperator{\supp}{supp}
\DeclareMathOperator{\Ker}{Ker}
\DeclareMathOperator{\im}{im}
%\DeclareMathOperator{\wdeg}{w\,deg}
\DeclareMathOperator{\depth}{depth}
\DeclareMathOperator{\gin}{gin}
\DeclareMathOperator{\Coker}{Coker}
\DeclareMathOperator{\NF}{NF}
\DeclareMathOperator{\pd}{pd}
\DeclareMathOperator{\SL}{SL}
\DeclareMathOperator{\SO}{SO}
\DeclareMathOperator{\Ort}{O}
\DeclareMathOperator{\Spez}{Sp}
\DeclareMathOperator{\PSL}{PSL}
\DeclareMathOperator{\PGL}{PGL}
\DeclareMathOperator{\wdim}{wdim}
\DeclareMathOperator{\cdim}{cdim}
\DeclareMathOperator{\cha}{char}
\DeclareMathOperator{\trdeg}{trdeg}
\DeclareMathOperator{\codim}{codim}
\DeclareMathOperator{\kdim}{kdim}
\DeclareMathOperator{\height}{height}
\DeclareMathOperator{\Ass}{Ass}
\DeclareMathOperator{\Lie}{Lie}
\DeclareMathOperator{\rk}{rk}


\renewcommand{\labelenumi}{(\arabic{enumi})}
\newcommand{\Ndash}{\nobreakdash--}% for pages 1\Ndash 9
%\newcommand{\somespace}{\hfill{}\\ \vspace{-0.7cm}}
%\def\partitle#1{{\medskip\noindent {\bf #1.\hbox to 12pt{}}}}
\def\partitle{\subsubsection}

%%%theosdefinitionen
\newcommand{\gm}{\mathfrak m}
\newcommand{\gM}{\mathfrak M}
\newcommand{\integer}{\ZZ}
\newcommand{\proj}{\PP}
\newcommand{\complex}{\CC}
\newcommand{\real}{\mathbb R}
\newcommand{\gp}{\mathfrak p}
\newcommand{\gq}{\mathfrak q}
% \newcommand{\go}{\mathfrak so}
%\newcommand{\openF}{\F}

%%%%%%%%%%%%%%%betti table more wide!
\tabcolsep 3pt



%%%%%%%%%%%%%%%BIBLIOGRAPHY
\newcommand{\by}{}
\newcommand{\paper}{: \begin{it}}
\newcommand{\jour }{, \end{it}}
\newcommand{\vol}{\begin{bf} }
\newcommand{\yr}{\end{bf}(}
\newcommand{\pages}{),}

\maketitle

\begin{abstract} In this article we illustrate how picking points
over a finite field at random 
can help to investigate algebraic geometry questions.
In the first part we develop a program that produces random
curves of genus $g \le 14$. In the second part we use the program to test
Green's Conjecture on syzygies of canonical curves and compare 
it with the corresponding statement for Coble self-dual sets of
points.
In the third section we apply our techniques to produce Calabi-Yau
3-folds of degree $17$ in $\PP^6$.     
\end{abstract}

\section*{Introduction}

The advances in speed of modern computers and computer algebra systems gave 
life to the idea of solving equations\index{solving polynomial equations} by guessing a solution.
Suppose $\MM \subset \GG$ is a subvariety of a rational variety
of codimension $c$. 
Then we expect that the probability for a point $p \in \GG(\FF_q)$ to lie in 
$\MM(\FF_q)$ is about $1/q^c$. 
Here $\FF_q$ denotes the field with $q$ elements.  

We will discuss this idea in the following setting: 
$\MM$ will be a parameter space for objects in algebraic geometry, e.g.,
a \ie{Hilbert scheme}, a \ie{moduli space}, or a space dominating such spaces.

The most basic question we might have in this case is whether $\MM$ is 
non-empty and whether an open part of $\MM$ corresponds to smooth objects.

Typically in these cases
we will not have explicit equations for $\MM \subset \GG$
but only an implicit algebraic description of $\MM$, and our approach will
be successful if the time required to check $p \notin \MM(\FF_q)$ is sufficiently
small compared to $q^c$.
The first author applied this method first in \cite{CO:Sch1} to construct some
rational surfaces in $\PP^4$; see \cite{CO:ElPe,CO:DS} for motivation.


In this first section we describe a program that picks curve of genus
$g \le 14$ at random. The moduli spaces $\gM_g$ are known to be unirational
for $g\le 13$; see \cite{CO:Se,CO:CR}.

Our approach based on this result
can viewed as a computer aided proof of the \ie{unirationality}. 
Many people might object 
that this not a proof because we cannot control every single step in the
computation. We however think that such a proof is much more reliable than
a proof based on man-made computations.
A mistake in a computer aided approach most often leads to an output far 
away from our expectation, hence it is easy to spot.  
A substantial improvement of present computers and computer algebra systems 
would give us an explicit unirational parametrization of $\gM_g$ for $g\le 13$.
\medskip

In the second part we apply our ``random curves'' to probe the consequences 
of Green's conjecture on syzygies of canonical curves, 
and compare these results with the corresponding statements for 
``Coble self-dual'' sets of $2g-2$ points in $\PP^{g-2}$. 
\medskip

In the last section we exploit our method to prove the existence 
of three components of the Hilbert scheme of Calabi-Yau 3-folds of degree $17$ 
in $\PP^6$ over the complex numbers. 
This is one of the  main results of the second author's thesis
\cite[Chapter 4]{CO:To}. 
Calabi-Yau threefolds of lower degree in $\PP^6$ are easy to construct, 
using the Pfaffian construction and a study of their Hartshorne-Rao modules.
For degree $17$ the Hartshorne-Rao module has to satisfy a subtle condition. 
Explicit examples of such Calabi-Yau 3-folds are first 
constructed over a finite field by our probabilistic method.
Then a delicate semi-continuity argument gives us the existence of such Calabi-Yau 3-folds
over some number field.


\begin{acknowledgment} 
We thank Hans-Christian v. Bothmer and Dan Gray\-son for valuable discussions and
remarks. 
\end{acknowledgment} 


\partitle{Notation} 
For a finitely generated graded module $M$ over the polynomial ring 
$S=k[x_0,\ldots,x_r]$ we summarize the numerical information of a finite
free resolution
$$ 0 \lto M \lto F_0 \lto F_1 \lto \ldots \lto F_n\lto 0 $$
with $F_i = \oplus_j S(-j)^{\beta_{ij}}$ in a \ie{table of Betti numbers},
whose $ij^{th}$ entry is
$$ \beta_{i,i+j} = \dim \Tor_i^S(M,k)_{i+j}.$$
As in the \Mtwo command {\tt betti} we suppress zeroes.
For example the syzygies of the rational normal curve in $\PP^3$ 
have the following Betti table.
$$\begin{tabular}{|ccc}
\hline
1 & - & - \cr
- & 3 & 2 \cr
\end{tabular}$$
Note that the degrees of the entries of the matrices in the free
resolution can be read off from the relative position of two numbers in 
consecutive columns. A pair of numbers in a line corresponds to linear entries.
Quadratic entries correspond to two numbers of a square. Thus 
$$\begin{tabular}{|cccc}
\hline
1 & - & - & - \cr
- & 5 & 5 & - \cr 
- & - & - & 1 \cr
\end{tabular}$$
corresponds to a 4 term complex with a quadratic, a linear and another
quadratic map. The Grassmannian $\GG(2,5)$ in its Pl\"ucker embedding
has such a free resolution.
 


 

\section{How to Make Random Curves up to Genus $14$}

\index{random curves}
The \ie{moduli space of curves} $\gM_g$ is known to be of general type for 
$g \ge 24$ and has non-negative Kodaira dimension for $g=23$ by work of 
Harris, Mumford and Eisenbud \cite{CO:HM,CO:EH}. 
For genus $g \le 13  $ unirationality is known \cite{CO:CR,CO:Se}.
In this section we present a \Mtwo program that over a finite field $\FF_q$ 
picks a point in $\gM_g(\FF_q)$  for $g \le 14$ at random. 

By Brill-Noether theory \cite{CO:ACGH} every curve of genus $g$ has a linear 
system $g^r_d$ of dimension $r$ and degree $d$, provided that the 
\ie{Brill-Noether number}
$\rho$ satisfies
$$\rho := \rho(g,d,r) := g-(r+1)(g-d+r)\geq 0.$$
We utilize this to find appropriate (birational) models for general curves of genus $g$.



\subsection{Plane Models, $g\le 10$}

This case was known to Severi; see \cite{CO:AC}. 
Choose $d = g+2 - \lfloor g/3 \rfloor$. 
Then $\rho(g,d,2) \ge 0$
i.e., a general curve of genus g has a plane model $C'$ of degree $d$. 
We expect that $C'$ has 
$$\delta= \binom{d-1}{2} - g$$
double points. 
If the double points are in general position, then 
$$s=h^0(\PP^2,\ko(d))-3\delta-1$$
is the expected dimension of the linear system of curves of degree $d$ 
with $\delta$ assigned double points.  
We have the following table:
$$
{\tabcolsep 5pt
\begin{tabular}{l|rrrrr rrrrr rr}
g & 1 & 2 & 3 & 4 & 5 & 6 & 7 & 8 & 9 & 10 & 11 & 12 \cr
\hline
$\rho$      & 1 & 2 & 0 & 1 & 2 & 0 & 1 & 2 & 0 & 1  &  2 & 0 \cr
\hline
$d$        & 3 & 4 & 4 & 5 & 6 & 6 & 7 & 8 & 8 & 9 & 10 & 10 \cr
\hline
$\delta$ & 0 & 1 & 0 & 2 & 5 & 4 & 8 & 13 & 12 & 18 & 25 & 24 \cr
\hline
$s$        & 9 & 11 & 14 & 14 & 12 & 15 & 11 & 5 & 8 & 0 & -10 & -7 \cr
\end{tabular}}
$$
Thus for $g\le 10$ we assume that these double points lie in general position. 
For $g>10$ the double points cannot lie in general position because $s<0$. 
Since it is difficult to describe the special locus 
$H_\delta(g) \subset \Hilb_\delta(\PP^2)$ 
of double points of nodal genus $g$ curves, the plane model approach collapses for $g>10$.

\partitle{Random Points}
\index{random points}
In our program, which picks plane models at random from an Zariski open subspace of $\gM_g$,
we start by picking the nodes. 
However, over a small field $\FF_q$ it is not a good idea to pick points individually, 
because there might be simply too few: $|\PP^2(\FF_q)|=1+q+q^2$.
What we should do is to pick a collection $\Gamma$ of $\delta$ points in $\PP^2(\bar \FF_q)$
that is defined over $\FF_q$.
General points in $\PP^2$ satisfy the minimal resolution condition, 
that is, they have expected Betti numbers.
This follows from the Hilbert-Burch theorem \cite[Theorem 20.15]{CO:Ei}.
If the ideal of such $\Gamma$ has generators in minimal degree $k$,
then 
$\binom{k+1}2 \le \delta < \binom{k+2}2$,
which gives $\delta= \binom{k+1}{2} + \epsilon$  with  $0 \le \epsilon \le k$.
Thus $k=\lceil ({-3+\sqrt{9+8\delta}})/{2} \rceil$.
The Betti table is one of the following two tables:


\medskip 
$2\epsilon \le k :$
\begin{tabular}{c|ccc}\cline{2-4}
0 & 1 & - & - \cr
1 & - & - & - \cr
\vdots & \vdots & \vdots & \vdots \cr
$k-2$ & - & - & - \cr
$k-1$ & - & $k+1-\epsilon$ & $k-2\epsilon$ \cr
$k$ & - & - & \hbox to 20pt{\hfil$\epsilon$\hfil} \cr
\end{tabular}
$\quad2\epsilon \ge k :$
\begin{tabular}{c|ccc}\cline{2-4}
0 & 1 &- & - \cr
1 & - & - & - \cr
\vdots & \vdots & \vdots & \vdots \cr
$k-2$ & - & - & - \cr
$k-1$ & - & $k+1-\epsilon$ & - \cr
$k$ & - & $2\epsilon-k$ & \hbox to 20pt{\hfil$\epsilon$\hfil} \cr
\end{tabular}
\medskip

\noindent
So we can specify a collection $\Gamma$ of $\delta$ points by picking 
the Hilbert-Burch matrix of their resolution; see \cite[Thm 20.15]{CO:Ei}. 
This is a matrix with linear and quadratic entries only, 
whose minors of size $\epsilon$ ($k-\epsilon$ if $2\epsilon\leq k$) 
generate the homogeneous ideal of $\Gamma$.
\beginOutput
i1 : randomPlanePoints = (delta,R) -> (\\
\          k:=ceiling((-3+sqrt(9.0+8*delta))/2);\\
\          eps:=delta-binomial(k+1,2);\\
\          if k-2*eps>=0 \\
\          then minors(k-eps,\\
\               random(R^(k+1-eps),R^\{k-2*eps:-1,eps:-2\}))\\
\          else minors(eps,\\
\               random(R^\{k+1-eps:0,2*eps-k:-1\},R^\{eps:-2\})));\\
\endOutput
In unlucky cases these points might be infinitesimally near.
\beginOutput
i2 : distinctPoints = (J) -> (\\
\          singJ:=minors(2,jacobian J)+J;\\
\          codim singJ == 3);\\
\endOutput

\medskip \noindent
The procedure that returns the ideal of a \ie{random nodal curve} is then straightforward: 
\beginOutput
i3 : randomNodalCurve = method();\\
\endOutput
\beginOutput
i4 : randomNodalCurve (ZZ,ZZ,Ring) := (d,g,R) -> (\\
\          delta:=binomial(d-1,2)-g;\\
\          K:=coefficientRing R;\\
\          if (delta==0) \\
\          then (     --no double points\\
\               ideal random(R^1,R^\{-d\}))\\
\          else (      --delta double points            \\
\               Ip:=randomPlanePoints(delta,R);\\
\               --choose the curve\\
\               Ip2:=saturate Ip^2;\\
\               ideal (gens Ip2 * random(source gens Ip2, R^\{-d\}))));\\
\endOutput
\beginOutput
i5 : isNodalCurve = (I) -> (\\
\          singI:=ideal jacobian I +I;delta:=degree singI;\\
\          d:=degree I;g:=binomial(d-1,2)-delta;\\
\          \{distinctPoints(singI),delta,g\});\\
\endOutput

We next ask if we indeed get in this way points in a parameter space 
that dominates $\gM_g$ for $g \le 10$.
Let $\Hilb_{(d,g)}(\PP^2)$ denote the Hilbert scheme of nodal plane curves
of degree $d$ and geometric genus $g$.
Our construction starts from  a random element in $\Hilb_\delta(\PP^2)$ and picks 
a random curve in the corresponding fiber of 
$\Hilb_{(d,g)}(\PP^2)\to\Hilb_\delta(\PP^2)$:
$$\xymatrix{\Hilb_{(d,g)}(\PP^2) \ar[r] \ar[d] &\gM_g\\ \Hilb_\delta(\PP^2)}.$$
So the question is whether $\Hilb_{(d,g)}(\PP^2)$ dominates $\Hilb_\delta(\PP^2)$.
A naive dimension count suggests that this should be true:
the dimension of our parameter space is given by $2\delta +s$, 
which is $3(g-1)+\rho+\dim\PGL(3)$, as it should be.
To conclude this there is more to verify: 
it could be that the nodal models of general curves have double points 
in special position, 
while all curve constructed above lie over a subvariety of $\gM_g$. 
One way to exclude this is to prove that the variety $G(g,d,2)$ over $\gM_g$,
whose fiber over a curve ${\tilde C}\in\gM_g$ is $G^2_d(\tilde C)=\{g^2_d\text{'s}\}$,
is irreducible or, to put it differently, that the Severi Conjecture holds:

\begin{theorem}[Harris \cite{CO:Ha1}] The space
of nodal degree d genus g curves in $\PP^2$ is irreducible.
\end{theorem}

Another much easier proof for the few $(d,g)$ we are interested in is to
establish that our parameter space $\MM$ of the construction is smooth of 
expected dimension at our random point $p \in \MM$, as in \cite{CO:AC}.
Consider the following diagram:
$$\HH = \Hilb_{(d,g)}(\PP^2)/\Aut(\PP^2) \stackrel{\pi}{\lra} \gM_g.$$
For a given curve ${\tilde C}\in\gM_g$, the inverse image $\pi^{-1} ({\tilde C})$ 
consists of the variety $W^2_d({\tilde C})\subset\Pic^d({\tilde C})$.
Moreover the choice of a divisor $L\in W^2_d({\tilde C})$ is equivalent to the 
choice of $p \in \MM$, modulo $\Aut(\PP^2)$: 
indeed $p$ determines a morphism $\nu\colon {\tilde C} \lra C\subset \PP^2$ 
and a line bundle $L=\nu^{-1}\ko_{\PP^2}(1)$, where ${\tilde C}$ is the 
normalization of the (nodal) curve $C$.
Therefore $\MM$ is smooth of expected dimension $3(g-1)+\rho+\dim \PGL(3)$
at $p \in \MM$ 
if and only if $W^2_d({\tilde C})$ is smooth of expected dimension $\rho$ in $L$.
This is well known to be equivalent to the injectivity of the multiplication map
$\mu_L$
$$
H^0(L)\otimes H^0(K_{\tilde C}\otimes L^{-1}) \stackrel{\mu_L}{\lra}H^0(K_{\tilde C}),
$$
which can be easily checked in our cases,
see \cite[p. 189]{CO:ACGH}.
In our cases $\mu_L$ can be rewritten as 
$$
H^0(\ko_{\PP^2}(1))\otimes H^0(I_\Gamma(d-4)) \stackrel{\mu_L}{\lra}
H^0(I_\Gamma(d-3)).
$$
So we need two conditions: 
\begin{enumerate}
\item $H^0(I_\Gamma(d-5))=0$;
\item there are no linear relations among the generators of $H^0(I_\Gamma(d-4))$ of 
degree $d-3$.
\end{enumerate}
We proceed case by case.
For genus $g\leq 5$ this is clear, since $H^0(I_\Gamma(d-4))=0$ for $g=2,3$ and
$\dim H^0(I_\Gamma(d-3))=1$ for $g=4,5$. 
For $g=6$ we have $\dim H^0(I_\Gamma(d-3))= \dim H^0(I_\Gamma(2))=2$ and 
the Betti numbers of $\Gamma$ 
$$
\begin{tabular}{|ccc}\hline
1 & - & - \cr
- & 2 & - \cr
- & - & 1 \cr
\end{tabular}
$$
shows there are no relations 
with linear coefficients in $H^0(I_\Gamma(2))$.
For $7\leq g \leq 10$ the method is similar:
everything is clear once the Betti table of resolution of the set of nodal points 
$\Gamma$ is computed. As a further example we do here the case $g=10$: we see that
$\dim H^0(I_\Gamma(d-3))=\dim H^0(I_\Gamma(5))=3$ and the Betti numbers of $\Gamma$ are
$$
\begin{tabular}{|ccc}\hline
1 & - & - \cr
- & - & - \cr
- & - & - \cr
- & - & - \cr
- & 3 & - \cr
- & 1 & 3 \cr
\end{tabular}
$$
from which it is clear that there are no linear relations between the quintic generators
of $I_\Gamma$.

%% and that the 
%% Kodaira-Spencer map
%% $$T_p(\MM) \to T_C(\gM_g) = H^0(C,\omega^{\tensor 2})^*$$
%% is surjective. This is a first order deformation problem, which can be easily 
%% solved by linear algebra in Macaulay2:
%% 
%% {\scriptsize
%% \begin{verbatim}
%% Kodaira-SpencerMap = (I) -> ???
%% \end{verbatim}}


\subsection{Space Models and Hartshorne-Rao Modules}
\partitle{The Case of Genus $g=11$}
In this case we have $\rho(11,12,3)=3$. 
Hence every general curve of genus 11 has a space model of degree 12. 
Moreover for a general curve the general space model of this degree 
is linearly normal, because $\rho(11,13,4)=-1$ takes a smaller value. 
If moreover such a curve $C \subset \PP^3$ has \ie{maximal rank},
i.e., for each $m \in \ZZ$ the map 
$$H^0(\PP^3,\ko(m)) \to H^0(C,\ko_C(m))$$
has maximal rank,
then the \ie{Hartshorne-Rao module} $M$, 
defined as
$M=H^1_*(\ki_C)=\oplus_m H^1(\PP^3,\ki_C(m))$, 
has Hilbert function with values $(0,0,4,6,3,0,$ $0,\dots)$. For readers who want to know more about the
Hartshorne-Rao module, we refer to the  pleasant treatment in \cite{CO:MDP}.

Since being of maximal rank is an open condition, we will try a construction
of maximal rank curves.
Consider the  vector bundle $\kg$ on $\PP^3$ associated to the first syzygy 
module of $I_C$: 
$$ 0 \lto \ki_C \lto \oplus_i \ko(-a_i) \lto \kg \lto 0 \leqno(1)$$
In this set-up $H^2_*(\kg)=H^1_*(\ki_C)$. 
Thus $\kg$ is, up to direct sum of line bundles, the sheafified second syzygy 
module of $M$; see e.g., \cite[Prop. 1.5]{CO:DES}.

The expected Betti numbers of $M$ are
$$
\begin{tabular}{|ccccc}\hline
4 & 10 & 3 & - & - \cr
- & - & 8 & 2 & - \cr
- & - & - & 6 & 3 \cr
\end{tabular}
$$
Thus the $\FF$-dual $M^* = \Hom_\FF(M,\FF)$ is presented as 
$\FF[x_0,\dots,x_3]$-module by a 
$3 \times 8$ matrix with linear and quadratic entries, and a general such matrix 
will give a general module
(if the construction works, i.e., if the desired space of modules is non-empty), 
because all conditions we impose are semi-continuous and open.
Thus $M$ depends on 
$$\dim \GG(6,3h^0\ko(1))+\dim \GG(2,3h^0\ko(2)-6h^0\ko(1))-\dim SL(3)=36$$
parameters. 

Assuming that $C$ has minimal possible syzygies:
$$
\begin{tabular}{|cccc}\hline
1 & - & - & - \cr
-&-&-&-\cr
-&-&-&-\cr
-&-&-&-\cr
-&6&2&-\cr
-&-&6&3\cr
\end{tabular}
$$
we obtain, by dualizing the sequence (1), the following exact sequence
$$ \kg^* {\lto  } 6\ko(5) \lto \ko \lto 0.$$
If everything is as expected, 
i.e., the general curve is of maximal rank and its syzygies have minimal possible
Betti numbers, then the entries of the right hand matrix are
homogeneous polynomials that generate $I_C$. 
We will compute $I_C$ by determining $ \ker (\phi \colon 6\ko(5) \to \kg^*)$.
Comparing with the syzygies of $M$ we obtain the following isomorphism
$$\kg^* \cong \ker(2\ko(6)\oplus6\ko(7) \to 3\ko(8)) \cong 
\image(3\ko(4)\oplus8\ko(5) \to 2\ko(6)\oplus6\ko(7)).$$
and $\kg^* \lto 6\ko(5)$ factors over $\kg^* \lto 8\ko(5)\oplus3\ko(4)$.
A general $\phi \in \Hom(6\ko(5),$ $\kg^*)$ gives a point in $\GG(6,8)$
and the Hilbert scheme of  desired curves would have dimension 
$36+12=48=4\cdot12=30+3+15$ as expected, c.f.~\cite{CO:Ha2}.

\medskip
Therefore the computation for obtaining a random space curve of genus $11$ 
is done as follows:
\beginOutput
i6 : randomGenus11Curve = (R) -> (\\
\          correctCodimAndDegree:=false;\\
\          while not correctCodimAndDegree do (\\
\               Mt=coker random(R^\{3:8\},R^\{6:7,2:6\});\\
\               M=coker (transpose (res Mt).dd_4);\\
\               Gt:=transpose (res M).dd_3;\\
\               I:=ideal syz (Gt*random(source Gt,R^\{6:5\}));\\
\               correctCodimAndDegree=(codim I==2 and degree I==12););\\
\          I);\\
\endOutput

\medskip
In general for these problems there is rarely an a priori reason
why such a construction for general choices will give a smooth curve. 
Kleiman's global generation condition \cite{CO:Klei} is much too
strong a hypothesis for many interesting examples. 
But it is easy to check an example over a finite field with a computer:
\beginOutput
i7 : isSmoothSpaceCurve = (I) -> (\\
\          --I generates the ideal sheaf of a pure codim 2 scheme in P3\\
\          singI:=I+minors(2,jacobian I);\\
\          codim singI==4);\\
\endOutput

Hence by semi-continuity this is true over $\QQ$ and the
desired unirationality of $G(11,12,3)/\gM_{11}$ holds for all fields,
except possibly for those whose ground field has characteristic is in some finite set.

A calculation of an example over the integers
would bound the number of exceptional characteristics, 
which then can be ruled out case by case, 
or by considering sufficiently many integer examples.

As in case of nodal curves, to prove \ie{unirationality} of $\gM_{11}$ by computer
aided computations we have to show the injectivity of
$$
H^0(L)\otimes H^0(K_{C}\otimes L^{-1}) \stackrel{\mu_L}{\lra}H^0(K_{C}),
$$
where $L$ is the restriction of $\ko_{\PP^3}(1)$ to the curve $C\subset \PP^3$.
The following few lines do the job:
\beginOutput
i8 : K=ZZ/101;\\
\endOutput
\beginOutput
i9 : R=K[x_0..x_3];\\
\endOutput
\beginOutput
i10 : C=randomGenus11Curve(R);\\
\emptyLine
o10 : Ideal of R\\
\endOutput
\beginOutput
i11 : isSmoothSpaceCurve(C)\\
\emptyLine
o11 = true\\
\endOutput
\beginOutput
i12 : Omega=prune Ext^2(coker gens C,R^\{-4\});\\
\endOutput
\beginOutput
i13 : betti Omega\\
\emptyLine
o13 = relations : total: 5 10\\
\                     -1: 2  .\\
\                      0: 3 10\\
\endOutput

\noindent
We see that there are no linear relations among the two generators
of $H^0_*(\Omega_C)$ in degree -1.

%% to prove uniratioanlity of $\gM_{11}$ by Computer
%% aided computations we have to show that the Kodaira-Spencer map
%%         $$ H^0(C,\kn_C) \to H^1(C,\kt_C) $$
%% is onto in our example. 
%% The following procedure returns the dimension 
%% of the image of the Kodaira-Spencer map for a smooth projective variety
%% (of codimension at least 2,
%% otherwise $\kn_C^*=${\tt I2} is free and {\tt T} is just {\tt coker djt}):
%% 
%% {\scriptsize
%% \begin{verbatim} 
%% kodairaSpencerMap = (I) -> (
%%      S:=R/I;
%%      dIt:=substitute(transpose jacobian I,S);
%%      I2:=substitute(syz gens I,S); --the conormal bundle
%%      T:=prune(kernel transpose I2/image dIt);
%%      hilbertFunction(0,T))
%% \end{verbatim}}
%% THIS SCRIPT FORGETS THE CONTRIBUTION OF H1 IN THE SEQUENCE GIVING T_P|C



\partitle{Betti Numbers for Genus $g=12,13,14,15$}
The approach in these cases is similar to $g=11$. We choose here $d=g$, 
so $\rho(g,g,3)=g-12\ge 0$. 
Under the maximal rank assumption the corresponding space curve has
a Hartshorne-Rao module whose Hilbert function takes values 
$(0,0,g-9,2g-19,3g-34,0,\ldots)$ in case $g=12,13$ 
and $(0,0,g-9,2g-19,3g-34,4g-55,0,\ldots)$ in case $g=14,15$.
Expected syzygies of $M$ have Betti tables:

$$g=12: \;
\begin{tabular}{|ccccc}\hline
3 & 7 & - & - & -  \cr
- & - & 10 & 5 & - \cr
- & - & - & 3 & 2  \cr
\end{tabular}
\qquad
g=13: \;
\begin{tabular}{|ccccc}\hline
4 & 9 & 1 & - & -  \cr
- & - & 6 & - & - \cr
- & - & 6 & 13 & 5  \cr
\end{tabular}
$$
\medskip
$$g=14: \;
\begin{tabular}{|ccccc}\hline
5 & 11 & 2 & - & -  \cr
- & - & 3 & - & - \cr
- & - & 13 & 17 & 4  \cr
- & - & - & - & 1 \cr
\end{tabular}
\qquad
g=15: \;
\begin{tabular}{|ccccc}
\hline
6 & 13 & 3 & - & -  \cr
- & - & 3 & - & - \cr
- & - & 8 & 3 & -  \cr
- & - & - & 9 & 5 \cr
\end{tabular}
$$

\medskip
Comparing with the expected syzygies of $C$ 
$$g=12: \;
\begin{tabular}{|cccc}\hline
 1 & - & - & -  \cr
 - & - & - & -  \cr
 - & - & - & -  \cr
 - & - & - & -  \cr
 - & 7 & 5 & - \cr
 - & - & 3 & 2  \cr
\end{tabular}
\qquad
g=13: \;
\begin{tabular}{|cccc}\hline
 1 & - & - & -  \cr
 - & - & - & -  \cr
 - & - & - & -  \cr
 - & - & - & -  \cr
 - & 3 & - & - \cr
 - & 6 & 13 & 5  \cr
\end{tabular}
$$
\medskip
$$g=14: \;
\begin{tabular}{|cccc}\hline
 1 & - & - & -  \cr
 - & - & - & -  \cr
 - & - & - & -  \cr 
- & - & - & -  \cr 
- & - & - & -  \cr
 - & 13 & 17 & 4  \cr
 - & - & - & 1 \cr
\end{tabular}
\qquad
g=15: \;
\begin{tabular}{|cccc}
\hline
 1 & - & - & -  \cr 
 - & - & - & -  \cr
- & - & - & -  \cr
 - & - & - & -  \cr
 - & - & - & -  \cr
 - & 8 & 3 & -  \cr
 - & - & 9 & 5 \cr
\end{tabular}
$$
we see that given  $M$ the choice of a curve corresponds to a point
in $\GG(7,10)$ or $\GG(3,6)$ for $g=12,13$ respectively, 
while for $g=14,15$ everything is determined by the Hartshorne-Rao module. 
For $g=12$ Kleiman's result guarantees smoothness for general choices,
in contrast to the more difficult cases $g=14,15$. 
So the construction of $M$ is the crucial step. 




\partitle{Construction of Hartshorne-Rao Modules}
In case $g=12$ the construction of $M$ is straightforward. 
It is presented by a sufficiently general matrix of linear forms:
$$0 \lto M \lto 3S(-2) \lto 7S(-3).$$

\noindent
The procedure for obtaining a random genus 12 curve is:
\beginOutput
i14 : randomGenus12Curve = (R) -> (\\
\           correctCodimAndDegree:=false;\\
\           while not correctCodimAndDegree do (\\
\                M:=coker random(R^\{3:-2\},R^\{7:-3\});\\
\                Gt:=transpose (res M).dd_3;\\
\                I:=ideal syz (Gt*random(source Gt,R^\{7:5\}));\\
\                correctCodimAndDegree=(codim I==2 and degree I==12););\\
\           I);\\
\endOutput


\medskip
In case $g=13$ we have to make sure that M has a second linear syzygy. 
Consider the end of the Koszul complex:
$$6R(-2) \stackrel{\kappa}{\lto} 4R(-3) \lto R(-1) \lto 0.$$
Any product of a general map $4R(-2) \stackrel{\alpha}{\lto} 6R(-2)$ with 
the Koszul matrix $\kappa$ yields
$4R(-2) \lto 4R(-3)$ with a linear syzygy, 
and concatenated with a general map $4R(-2) \stackrel{\beta}{\lto} 5R(-3)$
gives a presentation matrix of a module M of desired type:
$$ 0 \lto M \lto 4R(-2) \lto 4R(-3) \oplus 5R(-3).$$

\noindent
The procedure for obtaining a random genus 13 curve is:
\beginOutput
i15 : randomGenus13Curve = (R) -> (\\
\           kappa:=koszul(3,vars R);\\
\           correctCodimAndDegree:=false;\\
\           while not correctCodimAndDegree do (\\
\                test:=false;while test==false do ( \\
\                     alpha:=random(R^\{4:-2\},R^\{6:-2\});\\
\                     beta:=random(R^\{4:-2\},R^\{5:-3\});\\
\                     M:=coker(alpha*kappa|beta);\\
\                     test=(codim M==4 and degree M==16););\\
\                Gt:=transpose (res M).dd_3;\\
\                --up to change of basis we can reduce phi to this form\\
\                phi:=random(R^6,R^3)++id_(R^6);\\
\                I:=ideal syz(Gt_\{1..12\}*phi);\\
\                correctCodimAndDegree=(codim I==2 and degree I==13););\\
\           I);\\
\endOutput


\medskip
The case of genus $g=14$ is about a magnitude more difficult. 
To start with, we can achieve two second linear syzygies by the same method as 
in the case $g=13$. 
A general matrix 
$5R(-2) \stackrel{\alpha}{\lto} 12R(-2)$ composed with 
$12R(-2) \stackrel{\kappa \oplus \kappa}{\lto} 8R(-3)$ 
yields the first component of
$$5R(-2) \lto (8+3)R(-3).$$
For a general choice of the second component 
$5R(-2) \stackrel{\beta}{ \lto} 3R(-3)$ the cokernel
will be a module with Hilbert function $(0,0,5,9,8,0,0,\ldots)$ and syzygies
$$
\begin{tabular}{|ccccc}\hline
5 & 11 & 2 & - & -  \cr
- & - & 2 & - & - \cr
- & - & 17 & 23 & 8  \cr
- & - & - & - & - \cr
\end{tabular}
$$
What we want is to find $\alpha$ and $\beta$ so that $\dim M_5 = 1$ 
and $\dim \Tor_2^R(M,\FF)_5 = 3$.
Taking into account that we ensured $\dim \Tor_2^R(M,\FF)_4 =2$
this amounts to asking that the $100 \times 102$ matrix $m(\alpha,\beta)$ 
obtained from
$$[0 \lto 5R(-2)_5 \lto 11R(-3)_5 \lto 2R(-4)_5 \lto 0] 
\cong [0 \lto 100 \FF \stackrel{m(\alpha,\beta)} \lto 102 \FF \lto 0]$$
drops rank by 1. 
We do not know a systematic approach to produce such 
$m(\alpha,\beta)$'s.
However, we can find such matrices in a probabilistic way.
In the space of the matrices $m(\alpha,\beta)$, 
those which drop rank by 1 have expected codimension 3.
Hence over a finite field $\FF=\FF_q$ we expect to find the desired modules $M$
with a probability of $1/q^3$. 
The code to detect bad modules is rather fast.
\beginOutput
i16 : testModulesForGenus14Curves = (N,p) ->(\\
\           x := local x;\\
\           R := ZZ/p[x_0..x_3];\\
\           i:=0;j:=0;\\
\           kappa:=koszul(3,vars R);\\
\           kappakappa:=kappa++kappa;\\
\           utime:=timing while (i<N) do (\\
\                test:=false;\\
\                alpha:=random(R^\{5:-2\},R^\{12:-2\});\\
\                beta:=random(R^\{5:-2\},R^\{3:-3\});\\
\                M:=coker (alpha*kappakappa|beta);\\
\                fM:=res (M,DegreeLimit =>3);\\
\                if (tally degrees fM_2)_\{5\}==3 then (\\
\                     --further checks to pick up the right module\\
\                     test=(tally degrees fM_2)_\{4\}==2 and\\
\                     codim M==4 and degree M==23;);\\
\                i=i+1;if test==true then (j=j+1;););\\
\           timeForNModules:=utime#0; numberOfGoodModules:=j;\\
\           \{timeForNModules,numberOfGoodModules\});\\
\endOutput
\beginOutput
i17 : testModulesForGenus14Curves(1000,5)\\
\emptyLine
o17 = \{41.02, 10\}\\
\emptyLine
o17 : List\\
\endOutput

\noindent
For timing tests we used a Pentium2 400Mhz with 128Mb of memory running GNU Linux.
On such a machine examples can be tested at a rate 
of $0.04$ seconds per example.
Hence an approximate estimation of the CPU-time required to find a good example 
is $q^3 \cdot 0.04$ seconds.
Comparing this with the time to verify smoothness, 
which is about $12$ seconds for an example of this degree,
we see that up to $|\FF_q|=q \le 13$ we can expect to obtain examples within 
few minutes.
Actually the computations for $q=2$ and $q=3$ take longer than for $q=5$ on average,
because examples of ``good modules'' tend to give singular curves more often. 
Here is a table of statistics which summarizes the situation.
$$
\def\sb#1{\hbox to 22pt{\hfil#1}} %small fixed size box --text mode
\begin{tabular}{l|rrrrrrr}
$q$ &                           \sb2   &\sb3   &\sb5   &\sb7   &\sb{11}  &\sb{13}\cr
\hline
smooth curves &                 100 &100 &100 &100 &100 &100\cr
1-nodal curves &                75  &53  &31  &16  &10  &8\cr
reduced more singular &         1012&142 &24  &11  &2   &0\cr
non reduced curves &            295 &7   &0   &0   &0   &0\cr
\hline
total number of curves &        1482&302 &155 &127 &112 &108\cr
percentage of smooth curves \quad &  %the quad will be divided, ok
                                6.7\% &33\%  &65\%  &79\%  &89\% &93\%\cr
approx. time (in seconds) &     7400&3100&2700&3400&6500&9500\cr
\end{tabular}
$$

\medskip
\goodbreak
\noindent
The procedure for obtaining a random genus 14 curve is
\beginOutput
i18 : randomGenus14Curve = (R) -> (\\
\           kappa:=koszul(3,vars R);\\
\           kappakappa:=kappa++kappa;\\
\           correctCodimAndDegree:=false;\\
\           count:=0;while not correctCodimAndDegree do (\\
\                test:=false;\\
\                t:=timing while test==false do (\\
\                     alpha=random(R^\{5:-2\},R^\{12:-2\});\\
\                     beta=random(R^\{5:-2\},R^\{3:-3\});\\
\                     M:=coker (alpha*kappakappa|beta);\\
\                     fM:=res (M,DegreeLimit =>3);\\
\                     if (tally degrees fM_2)_\{5\}==3 then (\\
\                          --further checks to pick up the right module\\
\                          test=(tally degrees fM_2)_\{4\}==2 and\\
\                          codim M==4 and degree M==23;);\\
\                     count=count+1;);\\
\                Gt:=transpose (res M).dd_3;\\
\                I:=ideal syz (Gt_\{5..17\});\\
\                correctCodimAndDegree=(codim I==2 and degree I==14););\\
\           <<"     -- "<<t#0<<" seconds used for ";\\
\           <<count<<" modules"<<endl;\\
\           I);\\
\endOutput


\medskip
For $g=15$ we do not know a method along these lines that 
would give examples over small fields.



\partitle{Counting Parameters}
For genus $g=12$ clearly the module $M$ depends on 
$\dim\GG(7,3 \cdot h^0\ko(1))-\dim SL(3)= 7\cdot5-8=36$ parameters, 
and the family of curves has dimension $36+\dim\GG(7,10)=48=4\cdot12=33+0+15$,
as expected. 

For genus $g=13$ and $14$ the parameter count is more difficult.
Let us make a careful parameter count for genus $g=14$;
the case $g=13$ is similar and easier.
The choice of $\alpha$ corresponds to a point in $\GG(5,12)$.
Then $\beta$ corresponds to a point $\GG(3,B_\alpha)$ where 
$B_\alpha = U \tensor R_1/{<}\alpha{>}$ where $U$ denotes
the universal subbundle on $\GG(5,12)$ and ${<}\alpha{>}$ 
the subspace generated by the 8 columns of $\alpha\circ(\kappa \oplus \kappa)$. 
So $\dim B_\alpha = 20-8=12$ and $\GG(3,B_\alpha) \to \GG(5,12)$ 
is a Grassmannian bundle with fiber dimension 27 and total dimension 62. 
In this space the scheme of good modules has codimension 3, 
so we get a 59 dimensional family. 
This is larger than the expected dimension $56=4\cdot 14=39+2+15$ 
of the Hilbert scheme, c.f.~\cite{CO:Ha2}. 
Indeed the construction gives a curve together
with a basis of $\Tor_2^R(M,\FF)_4$. 
Subtracting the dimension of the group of the projective coordinate changes 
we arrive at the desired dimension $59-3=56$. 

The unirationality of $\gM_{12}$ and $\gM_{13}$ can be proved by computer 
as in case $\gM_{11}$, 
while in case $g=14$ we don't know the unirationality of 
the parameter space of the modules $M$ with 
$\dim M_5=1$ and $\dim \Tor^R_2(M,\FF)_5=3$.

\section[Comparing Green's Conjecture for Curves and Points]
	{Comparing Green's Conjecture for Curves \\ and Points}

\subsection{Syzygies of Canonical Curves}
\index{Green's conjecture}
\index{Syzygies of canonical curves}
One of the most outstanding conjectures about free resolutions is 
Green's prediction
for the syzygies of canonical curves. 

A \ie{canonical curve}
$C \subset \PP^{g-1}$, i.e., a linearly normal curve with $\ko_C(1) \equiv 
\omega_C$, the canonical line bundle, is projectively normal by a result of
Max Noether, and hence has a Gorenstein homogeneous coordinate ring and is 
3-regular.

Therefore the Betti numbers of the free resolution of a canonical curve 
are symmetric, that is, $\beta_{j,j+1}=\beta_{g-2-j,g-j}$, 
and essentially only two rows of Betti numbers occur.
The situation is summarized in the following table.
$$
\raise 0pt\hbox to 0 pt{\hspace{121.8pt} $\bullet$\hss}
\def\sb#1{\hbox to 5mm{\hfil#1\hfil}} 
\begin{tabular}{|ccccc ccccc cc}
\hline
1 & - & - & - & - & \sb{-} & \sb{-} & - & - & - & - & - \cr
- & $\beta_{1,2}$ & $\beta_{2,3}$ & $\cdots$ & $\beta_{p,p+1}$ & $\cdots$ 
& $\cdots$ & $\beta_{p,p+2}$ & - & - & - & - \cr
- & - & - & - & $\beta_{p,p+2}$ & $\cdots$ 
& $\cdots$ & $\beta_{p,p+1}$ & $\cdots$ & $\beta_{2,3}$ & $\beta_{1,2}$ & - \cr
- & - & - & - & - & - & - & - & - & - & - & 1 \cr
\end{tabular}
$$
%%$$
%%\def\hd#1{\hbox to 14.5mm{\hfil$#1$\hfil}} %for hodge diagrams --math mode
%%\setlength{\unitlength}{1mm}
%%\begin{picture}(116,36)(-58,-18)  
%%\put (-58,7.5){\hd{1}}
%%\put (-43.5,1.5){\hd{\beta_{1,2}}}
%%\put (-29,1.5){\hd{\ldots}}
%%\put (-14.5,1.5){\hd{\ldots}}
%%\put (-14.5,-4.5){\hd{\ldots}} %-0.85=-1+.25-.8
%%\put (0,0){\circle*{2.5}}
%%\put (0,1.5){\hd{\ldots}}
%%\put (0,-4.5){\hd{\ldots}}
%%\put (14.5,-4.5){\hd{\ldots}}
%%\put (29,-4.5){\hd{\beta_{g-3,g-1}}}
%%\put (43.5,-10.5){\hd{1}}
%%%hor lines
%%\put (-58,12) {\line(1,0){116}}
%%\put (-58,6) {\line(1,0){116}}
%%\put (-58,0) {\line(1,0){116}}
%%\put (-58,-6) {\line(1,0){116}}
%%\put (-58,-12) {\line(1,0){116}}
%%%vert lines
%%\put (-58,-12) {\line(0,1){24}}
%%\put (-43.5,-12) {\line(0,1){24}}
%%\put (-29,-12) {\line(0,1){24}}
%%\put (-14.5,-12) {\line(0,1){24}}
%%\put (0,-12) {\line(0,1){24}}
%%\put (14.5,-12) {\line(0,1){24}}
%%\put (29,-12) {\line(0,1){24}}
%%\put (43.5,-12) {\line(0,1){24}}
%%\put (58,-12) {\line(0,1){24}}
%%%mark the resolution
%%\linethickness{1pt}
%%\put (-43.5,6) {\line(1,0){58}}
%%\put (-43.5,0) {\line(1,0){29}} \put (14.5,0) {\line(1,0){29}}
%%\put (-14.5,-6) {\line(1,0){58}}
%%\put (-43.5,0) {\line(0,1){6}}
%%\put (43.5,-6) {\line(0,1){6}}
%%\put (-14.5,-6) {\line(0,1){6}}
%%\put (14.5,0) {\line(0,1){6}}
%%% description
%%\put (-58,13) {$\overbrace{\hspace{116mm}}^{\displaystyle g-1}$}
%%\put (-58,-13) {$\underbrace{\hspace{43.5mm}}_{\displaystyle p}$}
%%\end{picture}   
%%$$

The first $p$ such that $\beta_{p,p+2}\neq 0$ is conjecturally precisely 
the \ie{Clifford index} of the curve.

\begin{conjecture}[Green \cite{CO:Gr2}]\label{GConj} 
Let $C$ be a smooth canonical curve over $\CC$. 
Then $\beta_{p,p+2} \ne 0$ if and only if $\exists L\in \Pic^d(C)$ with
$h^0(C,L),h^1(C,L) \ge 2$ and $\cliff(L):=d-2(h^0(C,L))-1) \le p$.
In particular, $\beta_{j,j+2}= 0$ for $j \le \lfloor \frac{g-3}{2} \rfloor$ 
for a general curve of genus $g$.
\end{conjecture}
 
The ``if'' part is proved by Green and Lazarsfeld in \cite{CO:GL} 
and holds for arbitrary ground fields. 
For some partial results see \cite{CO:Vo,CO:Sch2,CO:Sch3,CO:BaEi,CO:vB,CO:HR,CO:Mu}.
The conjecture is known to be false for some (algebraically closed) fields of
finite characteristic, e.g.,
genus $g=7$ and characteristic $\cha \FF=2$; see \cite{CO:Sch4}.


\subsection{Coble Self-Dual Sets of Points}
\index{Coble self-dual sets of points}
The free resolution of a hyperplane section of a Cohen-Macaulay ring 
has the same Betti numbers. Thus we may ask for a geometric
interpretation of the syzygies of $2g-2$ points in $\PP^{g-2}$ 
(hyperplane section of a canonical curve), 
or syzygies of a graded Artinian Gorenstein algebra with Hilbert function
$(1,g-2,g-2,1,0,\ldots)$ 
(twice a hyperplane section). 
 Any collection of $2g-2$ points obtained as a hyperplane section of a
 canonical curve is special in the sense that it imposes only $2g-3$
 conditions on quadrics.
An equivalent condition for points in linearly uniform position is that 
they are Coble (or Gale) self-dual; see \cite{CO:EiPo}. 
Thus if we distribute the $2g-2$ points into two collections each of $g-1$ points, 
with, say, the first consisting of the coordinate points
and the second corresponding to the rows of a $(g-1) \times (g-1)$ matrix 
$A=(a_{ij})$, then $A$ can be chosen to be an orthogonal matrix, i.e.,
$A^t A = 1$; see \cite{CO:EiPo}.

To see what the analogue of Green's Conjecture for the general curve means for
orthogonal matrices we recall a result of \cite{CO:RS1}.

Set $n=g-2$. We identify the homogeneous coordinate ring of $\PP^n$ with
the ring $S=\FF[\partial_0,\ldots,\partial_n]$ of 
\ie{differential operators} with 
constant coefficients, $\partial_i=\frac{\partial}{\partial x_i}$. 
The ring $S$ acts on $\FF[x_0,\ldots,x_n]$ by differentiation. 
The annihilator of $q=x_0^2+\ldots+x_n^2$ is a homogeneous ideal 
$J \subset S$ such that 
$S/J$ is a \ie{graded Artinian Gorenstein ring} with Hilbert function $(1,n+1,1)$ 
and socle induced by $q$, 
see \cite{CO:RS2}, \cite[Section 21.2 and related exercise 21.7]{CO:Ei}.
The syzygy numbers of S/J are
$$
\begin{tabular}{|ccccc cc}\hline
1 &- &- &- &- &- &-\cr
- &$\frac{n}{n+2}\binom{n+3}{2}$ &$\cdots$ &$\frac{p(n+1-p)}{n+2}\binom{n+3}{p+1}$
&$\cdots$ &$\frac{n}{n+2}\binom{n+3}{n+1}$ &-\cr
- &- &- &- &- &- &1\cr  
\end{tabular}
$$


A collection $H_0,\ldots,H_n$ of hyperplanes in $\PP^n$ is said to 
form a polar simplex to $q$ 
if and only if the collection $\Gamma=\{p_0,\ldots,p_n\} \subset \check \PP^n$ 
of the corresponding points in the dual space has its homogeneous ideal 
$I_\Gamma \subset S$ contained in $J$.

In particular the set $\Lambda$ consisting of the coordinate points correspond 
to a polar simplex, because 
$\partial_i\partial_j$ annihilates $q$ for $i\ne j$.

For any polar collection of points $\Gamma$  the free resolution
$\Ss_\Gamma$ is a subcomplex of the resolution $\Ss_{S/J}$. Green's conjecture
for the generic curve of genus $g=n+2$ would imply: 

\begin{conjecture}\label{PConj}  For a general
$\Gamma$ and the given  $\Lambda$ the corresponding Tor-groups
$$\Tor^S_k(S/I_\Gamma,\FF)_{k+1} \cap \Tor^S_k(S/I_\Lambda,\FF)_{k+1}
\subset \Tor^S_k(S/J,\FF)_{k+1}$$
intersect transversally.
\end{conjecture} 

\begin{proof}
A zero-dimensional non-degenerate scheme $\Gamma\subset\PP^n$ of degree n+1 
has syzygies
$$
\begin{tabular}{|ccccc c}\hline
1 &- &- &- &- &-\cr
- &$\binom{n+1}{2}$ &$\cdots$ &$k\binom{n+1}{k+1}$
&$\cdots$ &$n$
\end{tabular}
$$

Since both Tor groups are contained in $\Tor^S_k(S/J,\FF)_{k+1}$, 
the claim is equivalent to saying that for a general polar simplex $\Gamma$ 
the expected dimension of their intersection is 
$\dim\Tor^S_k(S/_\Gamma,\FF)_{k+1}+\dim\Tor^S_k(S/_\Gamma,\FF)_{k+1}
-\dim\Tor^S_k(S/J,\FF)_{k+1}$, which is
$$
2k\binom{g-1}{k+1}-\frac{k(g-1-k)}{g}\binom{g+1}{k+1}.
$$

On the other hand, $I_{\Gamma\cup\Lambda}=I_\Gamma\cap I_\Lambda$, hence
$$\Tor^S_k(S/I_\Gamma,\FF)_{k+1} \cap \Tor^S_k(S/I_\Lambda,\FF)_{k+1} 
=\Tor^S_k(S/I_{\Gamma\cup\Lambda},\FF)_{k+1},$$
and Green's conjecture would imply
$$
\dim\Tor^S_k(S/I_{\Gamma\cup\Lambda},\FF)_{k+1}
=\beta_{k,k+1}(\Gamma\cup\Lambda)
=k\binom{g-2}{k+1}-(g-1-k)\binom{g-2}{k-2}.
$$
Now a calculation shows that the two dimensions above are equal.
\end{proof}


The family of all polar simplices $V$ is dominated by the family defined  
by the ideal of $2 \times 2$ minors of the matrix
$$\begin{pmatrix} 
\partial_0 & \ldots & \partial_i & \ldots &\partial_n \cr
\sum_j b_{0j} \partial_j & \ldots &\sum_j b_{ij} \partial_j & \ldots & \sum_j b_{nj} \partial_j
\end{pmatrix}$$
depending on     
a symmetric matrix $B=(b_{ij})$, i.e., $b_{ij}=b_{ji}$ as parameters.
For $B$  a general diagonal matrix
we get $\Lambda$ together with a specific element in 
$\Tor^S_n(S/I_\Lambda,\FF)_{n+1}$. 





\subsection{Comparison and Probes}

One of the peculiar consequences of Green's conjecture for odd genus $g=2k+1$ 
is that, if $\beta_{k,k+1} = \beta_{k-1,k+1} \ne 0$, then the curve $C$ lies 
in the closure of the locus of $k+1$-gonal curve. Any $k+1$-gonal curve
lies on a rational normal 
scroll $X$ of codimension $k$ that satisfies 
$\beta_{k,k+1}(X) = k$. Hence
$$\beta_{k,k+1}(C) \ne 0 \Rightarrow \beta_{k,k+1}(C) \ge k$$

We may ask whether a result like this is true for the union of two polar simplices $\Lambda \cup \Gamma \subset \PP^{2k-1}$. Define
$$\tilde D=\{ \Gamma \in V | \Gamma \cup \Lambda \hbox{ is syzygy special} \}$$
where, as above, $V$ denotes the variety of polar simplices and 
$\Lambda$ denotes the coordinate simplex. 
If $\tilde D$ is a proper subvariety, then it is a divisor,
because 
$\beta_{k,k+1}(\Gamma \cup \Lambda) =\beta_{k-1,k+1}(\Gamma \cup \Lambda) $. 


\begin{conjecture}\label{Exceptional locus}  The subscheme 
$\tilde D \subset V$ is an irreducible divisor, 
for $g=n+2=2k+1 \in \{7,9,11\}$. 
The value of $\beta_{k,k+1}$ on a general point of $D$ 
is $3, 2, 1$ respectively.
\end{conjecture} 

We can prove this for $g=7$ by computer algebra.  For
$g=9$ and $g=11$ a proof is
computationally out of reach with our methods, but we can get some
evidence from examples over finite fields.

\partitle{Evidence} 
Since $\tilde D$ is a divisor, we expect that if we pick symmetric
matrices $B$ over $\FF_q$ at random, we will hit points on every component
of $\tilde D$ at a
probability of $1/q$. For a general point on $ \tilde D$  the corresponding Coble
self-dual set of points will have the generic  number of extra syzygies
of that component. Points with even more syzygies will occur in higher 
codimension, hence only with a probability of $1/q^2$. 
Some evidence for irreducibility can be obtained from the Weil formulas:
for sufficiently large $q$ we should see points on $\tilde D$ with
probability
$C q^{-1} + O(q^{-\frac32})$, where $C$ is the number of components.
 
The following tables give for small fields $\FF_q$ the number $s_i$
of examples with $i$ extra syzygies in a test of 1000 examples for $g=9$ and
100 examples for $g=11$. 
The number $s_{\rm tot}= \sum_{i>0} s_i$ is the total number of examples with extra
syzygies.

\noindent
Genus $g=9$:
$$
\def\sb#1{\hbox to .8cm{\hfil$#1$}} %small fixed size box --text mode
\begin{tabular}{r|r|r|r|r|r|r|}
$q$ & $\ {\scriptstyle 1000}/q$ & \sb{s_{\rm tot}} & \sb{s_1} & \sb{s_2} & \sb{s_3} & \sb{s_4}  \cr
\hline
2 & 500 & 925 & 0 & 130 & 0 & 63 \cr
3 & 333 & 782 & 0 & 273 & 0 & 33 \cr
4 & 250 & 521 & 0 & 279 & 0 & 99 \cr
5 & 200 & 350 & 0 & 217 & 0 &74 \cr
7 & 143 & 197 & 0 & 144 & 0 &36 \cr
8 & 125 & 199 & 0 & 147 & 0 & 43 \cr
9 & 111 & 218 & 0 & 98 & 0 & 0 \cr
11 & 91 & 118 & 0 & 102 & 0 & 15 \cr
13 & 77 & 90 & 0 & 79 & 0 & 10 \cr
16 & 62 & 72 & 0 & 68 & 0 & 4 \cr
17 & 59 & 76 & 0 & 69 & 0 & 6 \cr
\end{tabular}$$

\noindent
Genus $g=11$:
$$
\def\sb#1{\hbox to .8cm{\hfil$#1$}} %small fixed size box --text mode
\begin{tabular}{r|r|r|r|r|r|r|}
$q$ & $\ {\scriptstyle 100}/q$ & \sb{s_{\rm tot}} & \sb{s_1} & \sb{s_2} & \sb{s_3} & \sb{s_4}  \cr
\hline
7 & 14.3 & 16 & 14 & 0 & 0 & 0 \cr
17 & 5.9 & 7 & 7 & 0 & 0 & 0 \cr
\end{tabular}$$

\noindent
In view of these numbers, it is more likely that
the set $\tilde D$ of syzygy special Coble points is irreducible 
than that it is reducible. For a more precise statement we refer to \cite{CO:vBS}.

 
 
\medskip
\noindent
{\bf A test of Green's Conjecture for Curves.}
In view of \ref{Exceptional locus} it seems plausible that for a general curve
of odd genus $g \ge 11$ with $\beta_{k,k+1}(C) \ne 0$ the value might
be $\beta_{k,k+1}=1$ contradicting Green's conjecture. 
It is clear that the syzygy 
exceptional locus has codimension 1 in $\gM_g$ for odd genus, if it is proper, 
i.e., if Green's 
conjecture holds for the general curve of that genus. So picking points 
at random we might be able to find such curve over a finite field $\FF_q$
with probability $1/q$, roughly.
%%$$<<\hbox{testGreensConjecture}>>$$ 

Writing code that does this is straightforward. One makes a loop that
picks up randomly a curve, computes its canonical image, and resolves
its ideal, counting the possible values $\beta_{k,k+1}$
until a certain amount of special curves is reached. 
The result for 10 special curves in $\FF_7$ is as predicted:
$$
\def\sb#1{\hbox to .65cm{\hfil$#1$}} %small fixed size box --text mode
\begin{tabular}{r|r|r|r|rrrrrr|}
& &total &\ special &\multicolumn{6}{r}{possible values of $\beta_{k,k+1}$}\vrule\cr
$g$ &\ seconds &\ curves &curves 
&\sb{\le 2}&\sb3&\sb4&\sb5&\sb6&\sb{\ge 7} \cr
\hline
7&148&75&10&0&10&0&0&0&0\cr
9&253&58&10&0&0&9&0&0&1\cr
11&25640&60&10&0&0&0&9&0&1\cr
\end{tabular}
$$
(The test for genus 9 and 11 used about 70
and 120 megabytes of memory, respectively.)


So Green's conjecture passed the test for $g=9,11$. 
Shortly after the first author tried this test for the first time,
a paper of Hirschowitz and Ramanan appeared proving this in general:

\begin{theorem}[\cite{CO:HR}] If the general curve of odd genus $g=2k+1$ 
satisfies Green's conjecture then the syzygy special curves lie on the divisor
$D=\{ C \in \gM_g | W^1_{k+1}(C) \ne \emptyset \}$ 
\end{theorem}

The theorem gives strong evidence for the full Green's conjecture in
view of our study of Coble self-dual sets of points.  

\medskip
Our findings suggest that the variety of points arising as 
hyperplane sections of smooth canonical curves has the strange
property that it intersects the divisor of syzygy special sets of
points $\tilde D$ only in its singular locus. 

\bigskip
The conjecture for general curves is known to us up to $g \le 17$, which is 
as far as a computer allows us to do a ribbon example; see \cite{CO:BaEi}.

%%{\bf The end of the section should be deleted or expanded.}  
%%The following scripts we do these example and allow to verify
%% the generic Green's conjecture
%%up to genus 17 (in that case the tests took 4556 seconds and 113Mbyte of memory).
%%
%%<testGreensConjectureForGivenParams = (a,b,k,l,K) -> (
%%     collectGarbage();
%%     x=symbol x;X=K[x_1..x_a];
%%     mult1Withs2=matrix(apply((k-2)+1,i->(apply((a-1-k)+1,j->x_(i+j+1)))));
%%     mult1Withst=matrix(apply((k-2)+1,i->(apply((a-1-k)+1,j->x_(i+j+2)))));
%%     mult1Witht2=matrix(apply((k-2)+1,i->(apply((a-1-k)+1,j->x_(i+j+3)))));
%%     --the Kozsul matrix \wedge^(k-1) Fa ->\wedge^k Fa ** Fa^ is given by
%%     y=symbol y;Y=K[y_1..y_a];
%%     alpha=transpose koszul(k,vars Y);
%%     --the contraction map
%%     QX=sum(numgens X,i->x_(i+1)^2);
%%     JX=ideal (symmetricPower(2,vars X)*(syz diff(QX, symmetricPower(2,vars X))));  
%%     Xquot=X/JX;XY=X**Y;contractMap=map(Xquot,XY,vars Xquot|vars Xquot);
%%     --Now s^2*(\wedge^(k-1) Fa -> \wedge^k Fa ** Fa^* ** (S_(k-2))^* ** Fa) is symply:
%%     alpha1=substitute(alpha,XY)**substitute(mult1Withs2,XY);
%%     --and we just have to contract the x_i's with the y_i's, which are of bidegree (1,1)
%%     alpha1=substitute(contractMap(alpha1),matrix(K,{toList(numgens X:1)}));
%%     --and the same for the other 2 terms
%%     alpha2=-2*substitute(alpha,XY)**substitute(mult1Withst,XY);
%%     alpha2=substitute(contractMap(alpha2),matrix(K,{toList(numgens X:1)}));
%%     alpha3=substitute(alpha,XY)**substitute(mult1Witht2,XY);
%%     alpha3=substitute(contractMap(alpha3),matrix(K,{toList(numgens X:1)}));
%%     --THE SAME WITH THE S_b, with variables (w,z)
%%     w=symbol w;W=K[w_1..w_b];
%%     mult2Withs2=matrix(apply((l-2)+1,i->(apply((b-1-l)+1,j->w_(i+j+1)))));
%%     mult2Withst=matrix(apply((l-2)+1,i->(apply((b-1-l)+1,j->w_(i+j+2)))));
%%     mult2Witht2=matrix(apply((l-2)+1,i->(apply((b-1-l)+1,j->w_(i+j+3)))));
%%     z=symbol z;Z=K[z_1..z_b];
%%     beta=transpose koszul(l,vars Z);
%%     QW=sum(numgens W,i->w_(i+1)^2);
%%     JW=ideal (symmetricPower(2,vars W)*(syz diff(QW, symmetricPower(2,vars W))));  
%%     Wquot=W/JW;WZ=W**Z;contractMap=map(Wquot,WZ,vars Wquot|vars Wquot);
%%     beta1=substitute(beta,WZ)**substitute(mult2Withs2,WZ);
%%     beta1=substitute(contractMap(beta1),matrix(K,{toList(numgens W:1)}));
%%     beta2=substitute(beta,WZ)**substitute(mult2Withst,WZ);
%%     beta2=substitute(contractMap(beta2),matrix(K,{toList(numgens W:1)}));
%%     beta3=substitute(beta,WZ)**substitute(mult2Witht2,WZ);
%%     beta3=substitute(contractMap(beta3),matrix(K,{toList(numgens W:1)}));
%%     --FINAL EQUATIONS
%%     equat=alpha1**beta1+alpha2**beta2+alpha3**beta3;
%%     <<numgens source equat<<"x"<<numgens target equat<<endl;
%%     rank equat==min(numgens source equat,numgens target equat));>
%%
%%<testGreensConjectureForOddGenera = (g,p) -> (
%%     if p==0 then K=QQ else K=ZZ/p;
%%     test=true;
%%     a:=floor((g-1)/2);
%%     k:=2;while k<=ceiling(a/2) do (
%%          <<k<<","<<a+1-k<<": ";
%%          time test=testGreensConjectureForGivenParams(a,a,k,a+1-k,K); 
%%          if test==false then k=a else k=k+1;);
%%     test);>
%%








\section{Pfaffian Calabi-Yau Threefolds in $\PP^6$}

\ie{Calabi-Yau 3-folds} caught the attention of physicists
because they can serve as the compact factor of the
Kaluza-Klein model of spacetime
in superstring theory.
One of the remarkable things that grows out of the work in physics
is the discovery of mirror symmetry, which associates to a family of
Calabi-Yau 3-folds $(M_\lambda)$, another family $(W_\mu)$ whose Hodge
diamond is the mirror of the Hodge diamond of the original family.

Although there is an enormous amount of evidence at present, the existence 
of a mirror is still a hypothesis for general Calabi-Yau 3-folds. 
The thousands of cases where this was established all are close to toric
geometry, where through the work of Batyrev and others \cite{CO:Ba,CO:CK} 
rigorous mirror
constructions were given and parts of their conjectured properties proved.


\medskip
From a commutative algebra point of view the examples studied so far are 
rather trivial, because nearly all are hypersurfaces or complete 
intersections on toric varieties, or zero loci of sections in 
homogeneous bundles on homogeneous spaces.  

Of course only a few families of Calabi-Yau 3-folds should be of this kind.
Perhaps the easiest examples beyond the toric/homogeneous range are 
Ca\-la\-bi-Yau 3-folds in $\PP^6$. 
Here examples can be obtained by the Pfaffian
construction of Buchsbaum-Eisenbud \cite{CO:BE} with vector bundles; see 
section \ref{Pfaffian complex} below. 
Indeed a recent theorem of Walter \cite{CO:Wa} says 
that any smooth Calabi-Yau in $\PP^6$ can be obtained in this way. 
In this section we report on our construction of such examples. 

As is quite usual in this kind of problem, there is a range where the
construction is still quite easy, 
e.g., for surfaces in $\PP^4$ the work in \cite{CO:DES,CO:Po} 
shows that the construction of nearly all the 50 known families 
of smooth non-general type surfaces is straight forward 
and their Hilbert scheme component unirational. 
Only in very few known examples is the construction more difficult
and the unirationality of the Hilbert scheme component an open problem.

The second author did the first ``non-trivial'' case of a construction of 
Calabi-Yau 3-folds in $\PP^6$. 
Although in the end the families  turned out to be unirational, 
the approach utilized small finite field constructions as a research tool.




\subsection{The Pfaffian Complex}
\index{Pfaffian complex}
Let $\kf$ be a vector bundle of odd rank $\rk \kf = 2r+1$ on a projective
manifold $M$, and let $\kl$ be a line bundle. Let $\varphi 
\in H^0(M,\Lambda^2 \kf \otimes \kl)$ be a section. We can 
think of $\varphi$ as a skew-symmetric twisted homomorphism
$$\kf^* \stackrel{\varphi}{\longrightarrow} \kf \otimes \kl.$$
 
The $r^{th}$ divided power of $\varphi$ is a section
$\varphi^{(r)} = \frac{1}{r!}(\varphi \wedge \dots \wedge \varphi) \in 
H^0(M,$ $\Lambda^{2r}\kf \otimes \kl^r)$.  Wedge product with $\varphi^{(r)}$ 
defines a morphism 
$$ \kf \tensor \kl \stackrel{\psi}{\longrightarrow}  \Lambda^{2r+1} \kf \otimes \kl^{r+1}=\det(\kf)\tensor\kl^{r+1}.$$

The  twisted image 
$\ki = \image(\psi) \tensor \det(\kf^*) \tensor \kl^{-r-1} \subset \ko_M$
is called the {\sl Pfaffian ideal} of $\varphi$, because working locally
with frames, it is given by the ideal generated by the  $2r\times 2r$ principle
Pfaffians of the matrix describing $\varphi$.
Let $\kd$ denote the determinant line bundle $\det(\kf^*)$.

\begin{theorem}[Buchsbaum-Eisenbud \cite{CO:BE}]\label{Pfaffian complex}
With this notation
$$0 \to \kd^2\tensor\kl^{-2r-1}
\stackrel{\psi^t}{\longrightarrow} \kd\tensor\kl^{-r-1}\tensor\kf^* 
\stackrel{\varphi}{\longrightarrow} \kf\tensor\kd\tensor\kl^{-r} 
\stackrel{\psi}{\longrightarrow} \ko_M$$
is a complex. 
$X=V(\ki) \subset M$ has codimension $\le 3$ at every point, and in case equality
holds (everywhere along $X$) then 
this complex is exact and resolves the structure sheaf $\ko_X=\ko_M/\ki$ 
of the  locally Gorenstein subscheme $X$. 
\end{theorem}

We will apply this to construct Calabi-Yau 3-folds in $\PP^6$. 
In that case we want $X$ to be smooth and 
$\det(\kf)^{-2}\tensor\kl^{-2r-1} \cong \omega_\PP\cong \ko(-7)$,
so we may conclude that $\omega_X \cong \ko_X$. 
A result of Walter \cite{CO:Wa} for $\PP^n$ guarantees the existence 
of a Pfaffian presentation in $\PP^6$ for every subcanonical embedded 3-fold. 
Moreover Walter's choice of $\kf\tensor\kd\tensor\kl^{-r}$ for Calabi-Yau 3-folds $X \subset \PP^6$
is the sheafified first syzygy module $H^1_*(\ki_X)$ plus possibly 
a direct sum of line bundles 
(indeed $H^2_*(\ki_X)=0$ because of the Kodaira vanishing theorem). 
Under the maximal rank assumption for
$$H^0(\PP^6,\ko(m)) \to H^0(X,\ko_X(m))$$
the Hartshorne-Rao module is zero for $d=\deg X \in \{12,13,14\}$ and 
an arithmetically Cohen-Macaulay $X$ is readily found. 
For $d \in \{15,16,17,18\}$
the Hartshorne-Rao modules $M$ have Hilbert functions with values
$(0,0,1,0,$ $\ldots)$, $(0,0,2,1,0,\ldots)$, $(0,0,3,5,0,\ldots)$ and 
$(0,0,4,9,0,\ldots)$ respectively.

We do not discuss the cases $d\le 16$ further. 
The construction in those cases is obvious; see \cite{CO:To}.


\subsection{Analysis of the Hartshorne-Rao Module for Degree $17$}

Denote with $\kf_1$ the sheaf $\kf\tensor\kd\tensor\kl^{-r}$.
We try to construct $\kf_1$ as the sheafified first syzygy module of $M$. 
The construction of a module with the desired Hilbert function is straightforward.
The cokernel of $3S(-2) \stackrel{b}{\longleftarrow} 16S(-3)$ for a general 
matrix of linear forms has this property. 
However, for a general $b$ and 
$\kf_1 = \ker( 16\ko(-3) \stackrel{b}{\longrightarrow} 3\ko(-2))$ 
the space of skew-symmetric maps $\Hom_{\rm skew}(\kf_1^*(-7),\kf_1)$ is zero:  
$M$ has syzygies
$$
\begin{tabular}{|cccccccc}
\hline
3 & 16 & 28 & - & - & - & - \cr
- & - & - & 70 & 112 & 84 & 32 & 5 \cr
\end{tabular}
$$
\noindent
Any map $\varphi \colon \kf_1^*(-7) \to \kf_1$ induces a map on the free
resolutions:
$$
\xymatrix{
0 &\kf_1 \ar[l] &28\ko(-4) \ar[l] &70\ko(-6) \ar[l] &112\ko(-7) \ar[l] \\
0 &\kf_1^*(-7) \ar[l] \ar[u]_\varphi &16\ko(-4) \ar[l] \ar[u]_{\varphi_0}
&3\ko(-5) \ar[l] \ar[u]_{\varphi_1} &0 \ar[l]
}
$$
%\begin{diagram}[small] 
%0 &\lTo &\kf &\lTo &28\ko &\lTo &70\ko(-2) &\lTo &112\ko(-3) \cr
%&&\varphi\; \uTo && \varphi_0 \;\uTo && \varphi_1\; \uTo&& \cr
%0&\lTo & \kf^*(1) &\lTo & 16\ko & \lTo &3\ko(-1) &\lTo & 0 \cr 
%\end{diagram}
\noindent
Since $\varphi_1=0$ for degree reasons, $\varphi=0$ as well, and
$\Hom(\kf_1^*(-7),\kf_1)=0$ for a general module $M$.

What we need are special modules $M$ that have extra syzygies 
$$
\begin{tabular}{|cccccccc}
\hline
3 & 16 & 28 & $k$ & - & - & - \cr
- & - & $k$ & 70 & 112 & 84 & 32 & 5 \cr
\end{tabular}
$$
with $k$ at least 3.

In a neighborhood of $o \in \Spec B$, where denotes $B$ the base space of 
a semi-universal deformation of $M$, 
the resolution above would lift to a complex over $B[x_0,\ldots,x_6]$ 
and in the lifted complex there is a $k\times k$ matrix $\Delta$
with entries in the maximal ideal $o \subset B$. 
By the principal ideal theorem we see that Betti numbers stay constant 
in a subvariety of codimension at most $k^2$. 
To check for second linear syzygies on a randomly chosen M 
is a computationally rather easy task.
The following procedure tests the computer speed of this task.
\beginOutput
i19 : testModulesForDeg17CY = (N,k,p) -> (\\
\           x:=symbol x;R:=(ZZ/p)[x_0..x_6];\\
\           numberOfGoodModules:=0;i:=0;\\
\           usedTime:=timing while (i<N) do (\\
\                b:=random(R^3,R^\{16:-1\});\\
\                --we put SyzygyLimit=>60 because we expect \\
\                --k<16 syzygies, so 16+28+k<=60\\
\                fb:=res(coker b, \\
\                     DegreeLimit =>0,SyzygyLimit=>60,LengthLimit =>3);\\
\                if rank fb_3>=k and (dim coker b)==0 then (\\
\                     fb=res(coker b, DegreeLimit =>0,LengthLimit =>4);\\
\                     if rank fb_4==0 \\
\                     then numberOfGoodModules=numberOfGoodModules+1;);\\
\                i=i+1;);\\
\           collectGarbage();\\
\           timeForNModules:=usedTime#0;\\
\           \{timeForNModules,numberOfGoodModules\});\\
\endOutput

Running this procedure we see that it takes not more than 
$0.64$ seconds per example.
Hence we can hope to find examples with $k=3$ within a reasonable time 
for a very small field, say $\FF_3$. 

\medskip
The first surprise is that examples with $k$ extra syzygies 
are found much more often, 
as can be seen by looking at the second value output by the function {\tt testModulesForDeg17CY()}.

This is not only a ``statistical'' remark, in the sense that 
the result is confirmed by computing the semi-universal deformations of these modules.
Indeed define $\MM_k=\{ M \mid \Tor^S_3(M,\FF)_5 \ge k \}$ and 
consider a module $M\in \MM_k$:
``generically'' we obtain $\codim(\MM_k)_M=k$ instead of $k^2$
(and in fact one can diagonalize the matrix $\Delta$ over 
the algebraic closure $\bar{\FF}$).

The procedure is straightforward but a bit long. 
First we pick up an example with $k$-extra syzygies.
\beginOutput
i20 : randomModuleForDeg17CY = (k,R) -> (\\
\           isGoodModule:=false;i:=0;\\
\           while not isGoodModule do (\\
\                b:=random(R^3,R^\{16:-1\});\\
\                --we put SyzygyLimit=>60 because we expect \\
\                --k<16 syzygies, so 16+28+k<=60\\
\                fb:=res(coker b, \\
\                     DegreeLimit =>0,SyzygyLimit=>60,LengthLimit =>3);\\
\                if rank fb_3>=k and (dim coker b)==0 then (\\
\                     fb=res(coker b, DegreeLimit =>0,LengthLimit =>4);\\
\                     if rank fb_4==0 then isGoodModule=true;);\\
\                i=i+1;);\\
\           <<"     -- Trial n. " << i <<", k="<< rank fb_3 <<endl;\\
\           b);\\
\endOutput
Notice that the previous function returns a presentation matrix $b$ of $M$, 
and not $M$.

Next we compute the tangent codimension of $\MM_k$ in the given example 
$M=\Coker b$ by computing the codimension of the space 
of the \ie{infinitesimal deformations} of $M$
that still give an element in $\MM_k$.
Denote with $b_i$ the maps in the linear strand of a minimal free resolution of $M$, 
and with $b_2'$ the quadratic part in the second map of this resolution.
Over $B=\FF[\epsilon]/{\epsilon^2}$ 
let $b_1+\epsilon f_1$ be an infinitesimal deformation of $b_1$. 
Then $f_1$ lifts to a linear map $f_2\colon 28S(-4) \to 16S(-3)$ determined by 
$(b_1+\epsilon f_1)\circ(b_2+\epsilon f_2)=0$, and $f_2$ to a map 
$f_3\oplus\Delta\colon k S(-5) \to 28 S(-4)\oplus k S(-5)$ determined by
$(b_2+b_2')\circ\epsilon (f_3\oplus\Delta)=0$.
Therefore we can determine $\Delta$ as:
\beginOutput
i21 : getDeltaForDeg17CY = (b,f1) -> (\\
\           fb:=res(coker b, LengthLimit =>3);\\
\           k:=numgens target fb.dd_3-28; --# of linear syzygies\\
\           b1:=fb.dd_1;b2:=fb.dd_2_\{0..27\};b2':=fb.dd_2_\{28..28+k-1\};\\
\           b3:=fb.dd_3_\{0..k-1\}^\{0..27\};\\
\           --the equation for f2 is b1*f2+f1*b2=0, \\
\           --so f2 is a lift of (-f1*b2) through b1 \\
\           f2:=-(f1*b2)//b1;\\
\           --the equation for A=(f3||Delta) is -f2*b3 = (b2|b2') * A\\
\           A:=(-f2*b3)//(b2l|b2');\\
\           Delta:=A^\{28..28+k-1\});\\
\endOutput
Now we just parametrize all possible maps $f_1\colon 16S(-3) \to 3S(-2)$,
compute their respective maps $\Delta$,
and find the codimension of the condition that $\Delta$ is the zero map:
\beginOutput
i22 : codimInfDefModuleForDeg17CY = (b) -> (\\
\           --we create a parameter ring for the matrices f1's\\
\           R:=ring b;K:=coefficientRing R;\\
\           u:=symbol u;U:=K[u_0..u_(3*16*7-1)];\\
\           i:=0;while i<3 do (\\
\                <<endl<< " " << i+1 <<":" <<endl;\\
\                j:=0;while j<16 do(\\
\                     << "    " << j+1 <<". "<<endl;\\
\                     k:=0;while k<7 do (\\
\                        l=16*7*i+7*j+k; --index parametrizing the f1's\\
\                        f1:=matrix(R,apply(3,m->apply(16,n->\\
\                             if m==i and n==j then x_k else 0)));\\
\                        Delta:=substitute(getDeltaForDeg17CY(b,f1),U);\\
\                        if l==0 then (equations=u_l*Delta;) else (\\
\                             equations=equations+u_l*Delta;);\\
\                        k=k+1;);\\
\                     collectGarbage(); --frees up memory in the stack\\
\                     j=j+1;);\\
\                i=i+1;);\\
\           codim ideal equations);\\
\endOutput


\medskip
The second surprise is that for $\kf_1 = syz_1(M)$ we find
$$\dim \Hom_{\rm skew}(\kf_1^*(-7),\kf_1) =k= \dim \Tor_3^S(M,\FF)_5.$$
$\Hom_{\rm skew}(\kf_1^*(-7),\kf_1)$ is the vector space of skew-symmetric
linear matrices $\varphi$ such that $b \circ \varphi = 0$.
The following procedure gives a matrix of size
$\binom{16}2\times\dim \Hom_{\rm skew}(\kf_1^*(-7),\kf_1)$
whose $i$-th column gives the entries of a $16\times16$ skew-symmetric matrix 
inducing the $i$-th basis element of the vector space $\Hom_{\rm skew}(\kf_1^*(-7),\kf_1)$.
\beginOutput
i23 : skewSymMorphismsForDeg17CY = (b) -> (\\
\           --we create a parameter ring for the morphisms: \\
\           K:=coefficientRing ring b;\\
\           u:=symbol u;U:=K[u_0..u_(binomial(16,2)-1)];\\
\           --now we compute the equations for the u_i's:\\
\           UU:=U**ring b;\\
\           equationsInUU:=flatten (substitute(b,UU)*\\
\                substitute(genericSkewMatrix(U,u_0,16),UU));\\
\           uu:=substitute(vars U,UU);\\
\           equations:=substitute(\\
\                diff(uu,transpose equationsInUU),ring b);\\
\           syz(equations,DegreeLimit =>0));\\
\endOutput
A morphism parametrized by a column $\tt skewSymMorphism$ is then recovered
by the following code.
\beginOutput
i24 : getMorphismForDeg17CY = (SkewSymMorphism) -> (\\
\           u:=symbol u;U:=K[u_0..u_(binomial(16,2)-1)];\\
\           f=map(ring SkewSymMorphism,U,transpose SkewSymMorphism);\\
\           f genericSkewMatrix(U,u_0,16));\\
\endOutput


\partitle{Rank 1 Linear Syzygies of $M$}
To understand this phenomenon we consider the multiplication tensor of $M$:
$$\mu \colon M_2 \tensor V \to M_3$$
where $V=H^0(\PP^6,\ko(1))$. 

\begin{definition} A decomposable element of $M_2 \tensor V$ in the kernel of 
$\mu$ is called a \ie{rank 1 linear syzygy} of $M$.
The (projective) space of rank 1 syzygies is
$$Y=(\PP^2 \times \PP^6) \cap \PP^{15} \subset \PP^{20}$$
where $\PP^2=\PP(M_2^*),\PP^6=\PP(V^*)$ and $\PP^{15}=\PP(\ker(\mu)^*)$
inside the Segre space $\PP((M_2 \tensor V)^*)\cong \PP^{20}$. 
\end{definition}

Proposition 1.5 of \cite{CO:Gr1} says that, for $\dim M_2 \le j$,
the existence of a $j^{th}$ linear syzygy implies $\dim Y \ge j-1$. 
This is automatically satisfied for $j=3$
in our case: $\dim Y \ge 3$ with equality expected. 

The projection $Y \to \PP^2$ has linear fibers, and the general fiber is a 
$\PP^1$. However, special fibers might have higher dimension. In terms
of the presentation matrix $b$ a special 2-dimensional fiber (defined
over $\FF$) corresponds to a block
$$
b=\begin{pmatrix}
0 & 0 & 0 & * & \ldots \cr
0 & 0 & 0 & * & \ldots \cr
l_1 & l_2 & l_3 & *& \ldots \cr 
\end{pmatrix},
$$
where $l_1,l_2,l_3$ are linear forms, in the $3\times16$ presentation matrix of $M$.
Such a block gives a 
$$
\begin{tabular}{|cccccccc}
\hline
1 & 3 & 3 & 1 & - & - & - &- \cr
- & - & - & - & - & - & - & - \cr
\end{tabular}
$$
subcomplex in the free resolution of $M$ and an element 
$s \in H^0(\PP^6,\Lambda^2 \kf_1 \tensor \ko(7))$
since the syzygy matrix
$$
\begin{pmatrix}
0 & -l_3 & l_2  \cr
l_3 & 0 & -l_1  \cr
-l_2 & l_1 & 0  \cr 
\end{pmatrix}
$$
is skew. 

This answers the questions posed by both surprises: 
we want a module $M$ with at least $k \ge 3$ special
fibers and these satisfy $h^0(\PP^6,\Lambda^2 \kf_1 \tensor \ko(7)) \ge k$, 
if the $k$ sections are linearly independent. 
The condition for $k$ special fibers is of expected codimension $k$ in the
parameter space $\GG(16,3h^0(\PP^6,\ko(1))$ of the presentation matrices.
In a given point $M$ the actual codimension can be readily computed
by a first order deformations and that 
$H^0(\PP^6,\Lambda^2 \kf_1 \tensor \ko(7))$ is $k$-dimensional, and spanned by the
$k$ sections corresponding to the $k$ special fibers can be checked as well. 

First we check that $M$ has $k$ distinct points in $\PP(M_2^*)$ where 
the multiplication map drops rank. 
(Note that this condition is likely to fail over small fields. 
However, the check is computationally easy).
\beginOutput
i25 : checkBasePtsForDeg17CY = b -> (\\
\           --firstly the number of linear syzygies\\
\           fb:=res(coker b, DegreeLimit=>0, LengthLimit =>4);\\
\           k:=#select(degrees source fb.dd_3,i->i==\{3\});\\
\           --then the check\\
\           a=symbol a;A=K[a_0..a_2];\\
\           mult:=(id_(A^7)**vars A)*substitute(\\
\                syz transpose jacobian b,A);\\
\           basePts=ideal mingens minors(5,mult);\\
\           codim basePts==2 and degree basePts==k and distinctPoints(\\
\                basePts));\\
\endOutput
Next we check that $H^0(\PP^6,\Lambda^2 \kf_1 \tensor \ko(7))$ is $k$-dimensional,
by looking at the numbers of columns of {\tt skewSymMorphismsForDeg17CY(b)}.
Finally we do the computationally hard part of the check, which is to
verify that the k special sections corresponding to the k special
fibers of $Y \to \PP^2$ span $H^0(\PP^6,\Lambda^2 \kf_1 \tensor \ko(7))$.
\beginOutput
i26 : checkMorphismsForDeg17CY = (b,skewSymMorphisms) -> (\\
\           --first the number of linear syzygies\\
\           fb:=res(coker b, DegreeLimit=>0, LengthLimit =>4);\\
\           k:=#select(degrees source fb.dd_3,i->i==\{3\});\\
\           if (numgens source skewSymMorphisms)!=k then (\\
\                error "the number of skew-sym morphisms is wrong";);\\
\           --we parametrize the morphisms:\\
\           R:=ring b;K:=coefficientRing R;\\
\           w:=symbol w;W:=K[w_0..w_(k-1)];\\
\           WW:=R**W;ww:=substitute(vars W,WW);\\
\           genericMorphism:=getMorphismForDeg17CY(\\
\                substitute(skewSymMorphisms,WW)*transpose ww);\\
\           --we compute the scheme of the 3x3 morphisms:\\
\           equations:=mingens pfaffians(4,genericMorphism);\\
\           equations=diff(\\
\                substitute(symmetricPower(2,vars R),WW),equations);\\
\           equations=saturate ideal flatten substitute(equations,W);\\
\           CorrectDimensionAndDegree:=(\\
\                dim equations==1 and degree equations==k);\\
\           isNonDegenerate:=#select(\\
\                (flatten degrees source gens equations),i->i==1)==0;\\
\           collectGarbage();\\
\           isOK:=CorrectDimensionAndDegree and isNonDegenerate;\\
\           if isOK then (\\
\                --in this case we also look for a skew-morphism f \\
\                --which is a linear combination of the special \\
\                --morphisms with all coefficients nonzero.\\
\                isGoodMorphism:=false;while isGoodMorphism==false do (\\
\                     evRandomMorphism:=random(K^1,K^k);\\
\                     itsIdeal:=ideal(\\
\                          vars W*substitute(syz evRandomMorphism,W));\\
\                     isGoodMorphism=isGorenstein(\\
\                          intersect(itsIdeal,equations));\\
\                     collectGarbage());\\
\                f=map(R,WW,vars R|substitute(evRandomMorphism,R));\\
\                randomMorphism:=f(genericMorphism);\\
\                \{isOK,randomMorphism\}) else \{isOK\});\\
\endOutput
The code above is structured as follows. 
First we parametrize the skew-symmetric morphisms with new variables.
The ideal of $4\times4$ Pfaffians is generated by forms of bidegree $(2,2)$
over $\PP^6 \times \PP^{k-1}$. We are interested in
points $p \in \PP^{k-1}$ such that the whole fiber 
$\PP^6 \times \{ p \}$ is contained in the zero locus of the Pfaffian
ideal. The next two lines produce the ideal of these points on
$\PP^{k-1}$. Since we already know of $k$ distinct points by the
previous check, it suffices to establish that the set consists of 
collection of k spanning points. Finally, if this is the case, a further
point, i.e., a further skew morphism, is a linear combination with all
coefficients non-zero, if and only if the union with this point is a Gorenstein set
of $k+1$ points in $\PP^{k-1}$.
\beginOutput
i27 : isGorenstein = (I) -> (\\
\           codim I==length res I and rank (res I)_(length res I)==1);\\
\endOutput


It is clear that all 16 relations should take part in the desired
skew homomorphism $\kf_1^*(-7) \stackrel{\varphi}{\longrightarrow} \kf_1$.
Thus we need $k \ge 6$ to have a chance for a Calabi-Yau.
Since $3\cdot5 <16$ it easy to guarantee 5 special fibers by suitable choice
of the presentation matrix. 
So the condition $k \ge 6$ is only of codimension $k-5$ on this subspace, 
and we have a good chance to find a module of the desired type.
\beginOutput
i28 : randomModule2ForDeg17CY = (k,R) -> (\\
\           isGoodModule:=false;i:=0;\\
\           while not isGoodModule do (\\
\                b:=(random(R^1,R^\{3:-1\})++\\
\                     random(R^1,R^\{3:-1\})++\\
\                     random(R^1,R^\{3:-1\})|\\
\                     matrix(R,\{\{1\},\{1\},\{1\}\})**random(R^1,R^\{3:-1\})|\\
\                     random(R^3,R^1)**random(R^1,R^\{3:-1\})|\\
\                     random(R^3,R^\{1:-1\}));\\
\                --we put SyzygyLimit=>60 because we expect \\
\                --k<16 syzygies, so 16+28+k<=60\\
\                fb:=res(coker b, \\
\                     DegreeLimit =>0,SyzygyLimit=>60,LengthLimit =>3);\\
\                if rank fb_3>=k and dim coker b==0 then (\\
\                     fb=res(coker b, DegreeLimit =>0,LengthLimit =>4);\\
\                     if rank fb_4==0 then isGoodModule=true;);\\
\                i=i+1;);\\
\           <<"     -- Trial n. " << i <<", k="<< rank fb_3 <<endl;\\
\           b);\\
\endOutput


\medskip
Some modules $M$ with $k=8,9,11$ lead to smooth examples of 
Calabi-Yau 3-folds $X$ of degree 17. 
To check the smoothness via the Jacobian criterion 
is computationally too heavy for a common computer today.
For a way to speed up this computation considerably 
and to reduce the required amount of memory to a reasonable value (128MB), 
we refer to \cite{CO:To}.

Since $h^0(\PP^6,\Lambda^2 \kf_1 \tensor \ko(7))=k$ 
and $\codim \{ M \mid \Tor^S_3(M,\FF)_5 \ge k \} =k$ all three families
have the same dimension. In particular no family lies in the closure 
of another.

A deformation computation verifies $h^1(X,\kt)=h^1(X,\Omega^2) = 23$. 
Hence a computation of the Hodge numbers $h^q(X,\Omega^p)$ gives the diamond
$$
\begin{matrix}
&&& 1 &&&\cr
&& 0 && 0 &&\cr
& 0 && 1 && 0 &\cr
 1 &&{23} &&{23} && 1\cr
& 0 && 1 && 0 &\cr
&& 0 && 0 &&\cr
&&& 1 &&&\cr 
\end{matrix}
$$

\begin{example}
The following commands give an example of a Calabi-Yau 3-fold in $\PP^6$:
\beginOutput
i29 : K=ZZ/13;\\
\endOutput
\beginOutput
i30 : R=K[x_0..x_6];\\
\endOutput
\beginOutput
i31 : time b=randomModule2ForDeg17CY(8,R);\\
\     -- Trial n. 1757, k=8\\
\     -- used 764.06 seconds\\
\emptyLine
\              3       16\\
o31 : Matrix R  <--- R\\
\endOutput
\beginOutput
i32 : betti res coker b\\
\emptyLine
o32 = total: 3 16 36 78 112 84 32 5\\
\          0: 3 16 28  8   .  .  . .\\
\          1: .  .  8 70 112 84 32 5\\
\endOutput
\beginOutput
i33 : betti (skewSymMorphisms=skewSymMorphismsForDeg17CY b)\\
\emptyLine
o33 = total: 120 8\\
\         -1: 120 8\\
\endOutput
We check whether the base points in $M_0$ are all distinct.
\beginOutput
i34 : checkBasePtsForDeg17CY b\\
\emptyLine
o34 = true\\
\endOutput
Now we check whether the $k$ sections span the morphisms.  
If we get {\tt true} then this is a good module.
\beginOutput
i35 : finalTest=checkMorphismsForDeg17CY(b,skewSymMorphisms);\\
\endOutput
\beginOutput
i36 : finalTest#0\\
\emptyLine
o36 = true\\
\endOutput
We pick up a random morphism involving all $k$ sections.
\beginOutput
i37 : n=finalTest#1;\\
\emptyLine
\              16       16\\
o37 : Matrix R   <--- R\\
\endOutput
 
If all the tests are okay, there should be a high degree syzygy.
\beginOutput
i38 : betti (nn=syz n)\\
\emptyLine
o38 = total: 16 4\\
\          1: 16 3\\
\          2:  . .\\
\          3:  . 1\\
\endOutput
\beginOutput
i39 : n2t=transpose submatrix(nn,\{0..15\},\{3\});\\
\emptyLine
\              1       16\\
o39 : Matrix R  <--- R\\
\endOutput
\beginOutput
i40 : b2:=syz b;\\
\emptyLine
\              16       36\\
o40 : Matrix R   <--- R\\
\endOutput
Finally, compute the ideal of the Calabi-Yau 3-fold in $\PP^6$.
\beginOutput
i41 : j:=ideal mingens ideal flatten(n2t*b2);\\
\emptyLine
o41 : Ideal of R\\
\endOutput
\beginOutput
i42 : degree j\\
\emptyLine
o42 = 17\\
\endOutput
\beginOutput
i43 : codim j\\
\emptyLine
o43 = 3\\
\endOutput
\beginOutput
i44 : betti res j\\
\emptyLine
o44 = total: 1 20 75 113 84 32 5\\
\          0: 1  .  .   .  .  . .\\
\          1: .  .  .   .  .  . .\\
\          2: .  .  .   .  .  . .\\
\          3: . 12  5   .  .  . .\\
\          4: .  8 70 113 84 32 5\\
\endOutput
\end{example}




\subsection{Lift to Characteristic Zero}

\index{lift to characteristic zero}
At this point we have constructed Calabi-Yau 3-folds $X \subset \PP^6$ over
the finite field $\FF_5$ or $\FF_7$. 
However, our main interest is the field of complex numbers $\CC$. 
The existence of a lift to characteristic zero follows by the following argument.

The set
$\MM_k=\{ M \mid \Tor^S_3(M,\FF)_5 \ge k \}$ has codimension at most $k$.
A deformation calculation shows that at our special point
$M^{\rm special} \in \MM(\FF_p)$ the codimension is achieved and that $\MM_k$
is smooth at this point. 
Thus taking a transversal slice defined over $\ZZ$ through this point 
we find a \ie{number field} $K$ and a prime $\gp$ in its ring of integers $O_K$ 
with $O_K/\gp \cong \FF_p$ such that 
$M^{\rm special}$ is the specialization of an $O_{K,\gp}$-valued point of $\MM_k$. 
Over the generic point of $\Spec O_{K,\gp}$ we obtain a $K$-valued point. 
From our computations with {\tt checkBasePtsForDeg17CY()} and 
{\tt checkMorphismsForDeg17CY()},
which explained why $h^0(\PP^6,\Lambda^2 \kf_1^{\rm special} \tensor \ko(7))=k$,
it follows that 
$$H^0(\PP^6_\ZZ\times\Spec O_{K,\gp},\Lambda^2 \kf_1 \tensor \ko(7))$$
is free of rank $k$ over $O_{K,\gp}$.
Hence $\varphi^{\rm special}$ extends to $O_{K,\gp}$ as well,
and by semi-continuity we obtain a smooth Calabi-Yau 3-fold defined over
$K \subset \CC$.

\begin{theorem}[\cite{CO:To}] The Hilbert scheme of smooth Calabi-Yau 3-folds of 
degree $17$ in 
$\PP^6$ has at least 3 components. These three components are reduced  and 
have dimension $23+48$. The corresponding Calabi-Yau 3-folds differ in the
number of quintic generators of their homogeneous ideals,
which are $8$, $9$ and $11$ respectively.   
\end{theorem}

See \cite{CO:To} for more details.
\medskip


Note that we do not give a bound on the degree $[K:\QQ]$ of the number field,
and certainly we are far away from a bound of its discriminant.

This leaves the question open whether these parameter spaces of Calabi-Yau 
3-folds are unirational. Actually they are, as the geometric construction
of modules $M \in \MM_k$ in \cite{CO:To} shows.

A construction of one or several \ie{mirror families} of these Calabi-Yau 3-folds
is an open problem. 

\nocite{CO:vB}




% Local Variables:
% mode: latex
% mode: reftex
% tex-main-file: "chapter-wrapper.tex"
% reftex-keep-temporary-buffers: t
% reftex-use-external-file-finders: t
% reftex-external-file-finders: (("tex" . "make FILE=%f find-tex") ("bib" . "make FILE=%f find-bib"))
% End:
\begin{thebibliography}{10}

\bibitem{CO:AC}
E.~Arbarello and M.~Cornalba:
\newblock {A few remarks about the variety of irreducible plane curve of given
  degree and genus}.
\newblock {\em Ann. Sci. \`Ecole Norm. Sup(4)}, 16:467--488, 1983.

\bibitem{CO:ACGH}
E.~Arbarello, M.~Cornalba, P.~Griffith, and J.~Harris:
\newblock {Geometry of algebraic curves, vol I}.
\newblock {\em Springer Grundlehren}, 267:xvi+386, 1985.

\bibitem{CO:Ba}
V.~Batyrev:
\newblock {Dual polyhedra and mirror symmetry for Calabi-Yau hypersurfaces in
  algebraic tori}.
\newblock {\em J. Alg. Geom.}, 3:493--535, 1994.

\bibitem{CO:BaEi}
D.~Bayer and D.~Eisenbud:
\newblock {Ribbons and their canonical embeddings}.
\newblock {\em Trans. Am. Math. Soc.}, 347,:719--756, 1995.

\bibitem{CO:vB}
H.-C.~v. Bothmer:
\newblock {Geometrische Syzygien kanonischer Kurven}.
\newblock {\em Thesis Bayreuth}, pages 1--123, 2000.

\bibitem{CO:vBS}
H.-C.~v. Bothmer and F.-O. Schreyer:
\newblock {A quick and dirty irreduciblity test}.
\newblock {\em manuscript}, to appear.

\bibitem{CO:BE}
D.~Buchsbaum and D.~Eisenbud:
\newblock {Algebra structure on finite free resolutions and some structure
  theorem for ideals of codimension 3}.
\newblock {\em Am. J. Math.}, 99:447--485, 1977.

\bibitem{CO:CR}
M.-C. Chang and Z.~Ran:
\newblock {Unirationality of the moduli space of curves genus $11$, $13$ (and
  $12$)}.
\newblock {\em Invent. Math.}, 76:41--54, 1984.

\bibitem{CO:CK}
D.~Cox and S.~Katz:
\newblock {Mirror symmetry and algebraic geometry}.
\newblock {\em AMS, Mathematical Surveys and Monographs}, 68:xxii+469, 1999.

\bibitem{CO:DES}
W.~Decker, L.~Ein, and F.-O. Schreyer:
\newblock {Construction of surfaces in ${\mathbb {\char 80}}^4$}.
\newblock {\em J. of Alg. Geom.}, 2:185--237, 1993.

\bibitem{CO:DS}
W.~Decker and F.-O. Schreyer:
\newblock {Non-general type surfaces in ${\mathbb {\char 80}}^4$ - Remarks on
  bounds and constructions}.
\newblock {\em J. Symbolic Comp.}, 29:545--585, 2000.

\bibitem{CO:Ei}
D.~Eisenbud:
\newblock {Commutative Algebra. With a view towards algebraic geometry}.
\newblock {\em Springer Graduate Texts in Mathematics}, 150:xvi+785, 1995.

\bibitem{CO:EH}
D.~Eisenbud and J.~Harris:
\newblock {The Kodaira dimension of the moduli space of curves of genus $g \ge
  23$}.
\newblock {\em Invent. Math.}, 87:495--515, 1987.

\bibitem{CO:EiPo}
D.~Eisenbud and S.~Popescu:
\newblock {Gale duality and free resolutions of ideals of points.}.
\newblock {\em Invent. Math.}, 136:419--449, 1999.

\bibitem{CO:ElPe}
G.~Ellingsrud and C.~Peskine:
\newblock {Sur les surfaces de ${\mathbb {\char 80}}^4$}.
\newblock {\em Invent. Math.}, 95:1--11, 1989.

\bibitem{CO:Gr2}
M.~Green:
\newblock {Koszul cohomology and the geometry of projective varieties}.
\newblock {\em J. Diff. Geom.}, 19:125--171, 1984.

\bibitem{CO:Gr1}
M.~Green:
\newblock {The Eisenbud-Koh-Stillman conjecture on linear syzygies}.
\newblock {\em Invent. Math}, 163:411--418, 1999.

\bibitem{CO:GL}
M.~Green and R.~Lazarsfeld:
\newblock {Appendix to Koszul cohomology and the geometry of projective
  varieties}.
\newblock {\em J. Diff. Geom.}, 19:168--171, 1984.

\bibitem{CO:Ha2}
J.~Harris:
\newblock {Curves in projective space. With collaboration of David Eisenbud}.
\newblock {\em S\'eminaire de Math\'ematiques Sup\'eriores}, 85:Montreal, 138
  pp., 1982.

\bibitem{CO:Ha1}
J.~Harris:
\newblock {On the Severi problem}.
\newblock {\em Invent. Math.}, 84:445--461, 1986.

\bibitem{CO:HM}
J.~Harris and D.~Mumford:
\newblock {On the Kodaira dimension of the moduli space of curves. With an
  appendix by William Fulton}.
\newblock {\em Invent. Math.}, 67:23--88, 1982.

\bibitem{CO:HR}
A.~Hirschowitz and S.~Ramanan:
\newblock {New evidence for Green's conjecture on syzgyies of canonical
  curves}.
\newblock {\em Ann. Sci. \'Ecole Norm. Sup. (4)}, 31:145--152, 1998.

\bibitem{CO:Klei}
S.~Kleiman:
\newblock {Geometry on Grassmannians and application to splitting bundles and
  smoothing cycles}.
\newblock {\em Inst. Hautes \'Etudes Sci. Publ. Math. No.}, 36:281--287, 1969.

\bibitem{CO:MDP}
M.~Martin-Deschamps and D.~Perrin:
\newblock {Sur la classification des courbes gauches}.
\newblock {\em Asterisque}, 184-185:1--208, 1990.

\bibitem{CO:Mu}
S.~Mukai:
\newblock {Curves and symmetric spaces, I.}.
\newblock {\em Amer. J. Math.}, 117:1627--1644, 1995.

\bibitem{CO:Po}
S.~Popescu:
\newblock {Some examples of smooth non general type surfaces in ${\mathbb
  {\char 80}}^4$}.
\newblock {\em Thesis Saarbr\"ucken}, pages ii+123, 1994.

\bibitem{CO:RS2}
K.~Ranestad and F.-O. Schreyer:
\newblock {Varieties of sums of powers}.
\newblock {\em J. Reine Angew. Math.}, 525:147--182, 2000.

\bibitem{CO:RS1}
K.~Ranestad and F.-O. Schreyer:
\newblock {On the variety of polar simplices}.
\newblock {\em manuscript}, to be completed.

\bibitem{CO:Sch4}
F.-O. Schreyer:
\newblock {Syzygies of canonical curves and special linear series}.
\newblock {\em Math. Ann.}, 275:105--137, 1986.

\bibitem{CO:Sch3}
F.-O. Schreyer:
\newblock {Green's conjecture for $p$-gonal curves of large genus. Algebraic
  curves and projective geometry (Trento 1988)}.
\newblock {\em Lecture Notes in Math.}, 1396:254--260, 1989.

\bibitem{CO:Sch2}
F.-O. Schreyer:
\newblock {A standard basis approach to syzygies of canonical curves}.
\newblock {\em J. Reine Angew. Math.}, 421:83--123, 1991.

\bibitem{CO:Sch1}
F.-O. Schreyer:
\newblock Small fields in constructive algebraic geometry.
\newblock In {\em Moduli of vector bundles (Sanda, 1994; Kyoto, 1994)}, pages
  221--228. Dekker, New York, 1996.

\bibitem{CO:Se}
E.~Sernesi:
\newblock {L' unirazionalit\`a della varieta dei moduli delle curve di genere
  dodici}.
\newblock {\em Ann. Scuola Norm. Sup. Pisa Cl. Sci. (4)}, 8:405--439, 1981.

\bibitem{CO:To}
F.~Tonoli:
\newblock {Canonical surfaces in ${\mathbb {\char 80}}^5$ and Calabi-Yau
  threefolds in ${\mathbb {\char 80}}^6$}.
\newblock {\em Thesis Padova}, pages 1--86, 2000.

\bibitem{CO:Vo}
C.~Voisin:
\newblock {Courbes t\'etragonales et cohomologie de Koszul}.
\newblock {\em J. Reine Angew. Math.}, 387:111--121, 1988.

\bibitem{CO:Wa}
C.~Walter:
\newblock {Pfaffian subschemes}.
\newblock {\em J. Alg. Geom.}, 5:671--704, 1996.

\end{thebibliography}
\egroup
\makeatletter
\renewcommand\thesection{\@arabic\c@section}
\makeatother



%%%%%%%%%%%%%%%%%%%%%%%%%%%%%%%%%%%%%%%%%%%%%%%%
%%%%%
%%%%% ../chapters/d-modules/chapter
%%%%%
%%%%%%%%%%%%%%%%%%%%%%%%%%%%%%%%%%%%%%%%%%%%%%%%

\bgroup
\title{$D$-modules and Cohomology of Varieties}
\titlerunning{$D$-modules and Cohomology of Varieties}
\toctitle{$D$-modules and Cohomology of Varieties}
\author{Uli Walther}
\authorrunning{U. Walther}
% \institute{Mathematical Sciences Research Institute, Berkeley, CA 94720, USA,
% and\\
% Department of Mathematics, Purdue University, West
% Lafayette, IN 47907, USA}
% % \email{walther@math.umn.edu}

\maketitle

\newtheorem{alg}[theorem]{Algorithm}{\bfseries}{}
% \newtheorem{hyp}[theorem]{Hypothesis}
% \newtheorem{con}[theorem]{Construction}
% \newtheorem{conj}[theorem]{Conjecture}
% \newtheorem{notn}[theorem]{Notation}

\def\ann{\operatorname{ann}}
\def\ass{\operatorname{ass}}
\def\codim{\operatorname{codim}}
\def\cd{\operatorname{cd}}
\def\depth{\operatorname{depth}}
\def\Der{\operatorname{Der}}
\def\endo{\operatorname{End}}
\def\ext{\operatorname{Ext}}
\def\gr{\operatorname{gr}}
\def\hom{\operatorname{Hom}}
\def\height{\operatorname{ht}}
\def\im{\operatorname{im}}
\def\ini{\operatorname{in}}
\def\lcd{\operatorname{lcd}}
\def\lcm{\operatorname{lcm}}
\def\spec{\operatorname{Spec}}
\def\soc{\operatorname{soc}}
\def\supp{\operatorname{supp}}
\def\var{\operatorname{Var}}

\def\curlN{{\mathcal N}}
\def\del{\partial}
\def\eps{\varepsilon}
\def\m{{\mathfrak m}}
\def\into{\hookrightarrow}
\def\onto{\to\hskip-1.7ex\to}
\def\A{{\mathbb A}}
\def\C{{\mathbb C}}
\def\F{{\mathfrak F}}
\def\M{{\mathcal M}}
%\def\MLf{{{\mathcal M}^L_f}}
\def\N{{\mathbb N}}
\def\OO{{\mathcal O}}
\def\P{{\mathbb P}}
\def\Q{{\mathbb Q}}
\def\R{{\mathbb R}}
\def\Z{{\mathbb Z}}


\def\action{\bullet}
\def\order{\prec}




\def\_#1{\underline{#1}}
%\def\mylabel#1{\label{#1} \marginpar{#1}}
\def\mylabel#1{\label{#1}}
%\def\myindex#1{\index{#1} \marginpar{#1}}
\def\myindex#1{\index{#1}}
\def\h{\hspace*{-3pt}}
\def\bar#1{\overline{#1}}

\numberwithin{equation}{section}

% \setcounter{tocdepth}{2}
% \maketitle
% \tableofcontents

% \input{0.tex}

\begin{abstract}
  In this chapter we introduce the reader to some ideas from the world of
  differential operators. We show how to use these concepts in conjunction
  with \Mtwo to obtain new information about polynomials and their algebraic
  varieties.
\end{abstract}

Gr\"obner bases over polynomial rings have been used for many
years in computational algebra, and the other chapters in this book
bear witness to this fact. 
In the mid-eighties some important steps were made in the theory of
Gr\"obner bases in non-commutative rings, notably in rings of differential
operators. This chapter is about some of the applications of this
theory to problems in commutative algebra and algebraic geometry. 

Our interest in rings of differential operators and $D$-modules stems
from the fact that some very interesting objects in algebraic geometry and
commutative algebra have a {\em finite} module structure
over an appropriate ring of differential operators. The prime example
is the ring of regular functions on the complement of an affine
hypersurface. A more general object is the \v Cech complex associated
to a set of polynomials, and its cohomology, the local
cohomology modules of the variety defined by the vanishing of the
polynomials. More advanced topics are restriction functors and de Rham
cohomology. 

With these goals in mind, we shall study 
applications  of Gr\"obner bases theory 
in the simplest
ring of differential operators, the Weyl algebra, and develop
algorithms that compute various invariants associated to a polynomial
$f$. These include the Bernstein-Sato polynomial $b_f(s)$, the set of
differential  operators $J(f^s)$ 
which annihilate the germ of the function $f^s$
(where $s$ is a new variable), and the ring of regular functions on the
complement of the variety of $f$. 


For a family $f_1,\ldots,f_r$ of polynomials we study the associated \v Cech
complex as a complex in the category of modules over the Weyl
algebra. The algorithms are illustrated with examples.
We
also give an indication what other invariants associated to
polynomials or varieties are known to be computable at this point and
list some open problems in the area.


\begin{acknowledgment}
It is with great pleasure that I acknowledge the help of A.\ Leykin,
M.\ Stillman
and H.\ Tsai while writing this chapter. The $D$-module routines
used or mentioned here have all been written by them and I would like
to thank them for this marvelous job.
I also would like to thank D.\ Grayson for help on \Mtwo and D.\
Eisenbud and B.\ Sturmfels for inviting me to contribute to this volume.
\end{acknowledgment}

%\input{1.tex}


\section{Introduction}

\subsection{Local Cohomology -- Definitions} 
Let $R$ be a commutative Noetherian ring (always associative, 
with identity) and $M$ an 
$R$-module. For $f\in R$ one defines a {\em \v Cech complex}\index{Cech complex@\v
Cech complex} of $R$-modules
\begin{eqnarray}
\check
C^\bullet(f)=(0\to \underbrace{R}_{\text{degree}\ 0}\into 
\underbrace{R[f^{-1}]}_{\text{degree}\ 1}\to 0)
\end{eqnarray}
where
the injection is the natural map sending $g\in R$ to ${g}/{1}\in
R[f^{-1}]$ and ``degree'' refers to  cohomological degree. 
For a family $f_1,\ldots,f_r\in R$ one defines 
\begin{eqnarray}
\check
C^\bullet(f_1,\ldots,f_r)=\bigotimes_{i=1}^r\check C^\bullet(f_i),
\end{eqnarray}
and for
an $R$-module $M$ one sets 
\begin{eqnarray}
\check C^\bullet(M;f_1,\ldots,f_r)=M\otimes_R \check
C^\bullet(f_1,\ldots,f_r).
\end{eqnarray}


The $i$-th (algebraic) {\em local
cohomology functor}\index{local cohomology}
 with respect to $f_1,\ldots,f_r$ is the $i$-th 
cohomology functor of $\check C^\bullet(-;f_1,\ldots,f_r)$. If
$I=R\cdot (f_1,\ldots,f_r)$ then this functor  agrees
with the $i$-th right
derived functor of the functor $H^0_I(-)$ which sends $M$ to the
$I$-torsion $\bigcup_{k=1}^\infty (0:_MI^k)$ of $M$ and is denoted by
$H^i_I(-)$. This means in particular, that $H^\bullet_I(-)$ 
depends only on the (radical of the) ideal generated by
the $f_i$. Local 
cohomology was introduced by A.~Grothendieck \cite{DM:lc-notes}
as an algebraic analog of
(classical) relative cohomology. For instance, if $X$ is a scheme, 
$Y$ is
a closed subscheme and $U=X\setminus Y$ then there is a long exact
sequence 
\[
\cdots\to H^i(X,\F)\to H^i(U,\F)\to H^{i+1}_Y(X,\F)\to\cdots
\]
for all quasi-coherent sheaves $\F$ on $X$. (To make sense of this one
has to generalize the definition of local cohomology to be the right
derived functor of $H^0_Y(-): \F\to(U\to\{f\in\F(U):\supp(f)\subseteq
Y\cap U\})$.)
An introduction to algebraic local
cohomology theory may be found in \cite{DM:B-S}. 

The {\em
cohomological dimension of $I$ in $R$}, \index{cohomological
dimension} denoted by $\cd(R,I)$, is the
smallest integer $c$ such that 
$H^i_I(M)=0$ for all $i>c$ and all $R$-modules $M$. If $R$ is the
coordinate ring of an affine variety $X$ and $I\subseteq R$ is the defining
ideal of the Zariski closed subset $Y\subseteq X$ then the {\em local
cohomological dimension of $Y$ in $X$} \index{local cohomological
dimension}  is defined as $\cd(R,I)$.
It is not hard to show that if $X$ is smooth, then the integer
$\dim(X)-\cd(R,I)$ depends only on $Y$ but neither on $X$ nor on the
embedding $Y\into X$. 
%
\subsection{Motivation} 
As one sees from the definition of local cohomology, the
modules $H^i_I(R)$ carry information about the sections of the
structure sheaf on Zariski open sets, and hence about the topology of
these open sets.
This is illustrated by the following examples. Let $I\subseteq
R$ and $c=\cd(R,I)$. Then 
$I$ cannot be generated by fewer than $c$ elements -- in other words,
$\spec(R)\setminus \var(I)$ cannot be covered by fewer than $c$ affine
open subsets (i.e., $\var(I)$ cannot be cut out by fewer than $c$
hypersurfaces).  In fact, no ideal $J$ 
with the same radical as $I$ will be generated by fewer than $c$
elements, \cite{DM:B-S}.

Let $H^i_{{\rm Sing}}(-;\C)$ stand for the $i$-th singular cohomology
functor with complex coefficients. 
The classical
Lefschetz Theorem \index{Lefschetz Theorem} \cite{DM:Gri-Har}
states that if $X\subseteq \P^n_\C$ is a
 variety in projective $n$-space 
and $Y$ a hyperplane section of $X$ such that $X\setminus Y$
is smooth, 
then $H^i_{{\rm Sing}}(X;\C)\to
H^i_{{\rm Sing}}(Y;\C)$ is an isomorphism for $i<\dim (X)-1$ and injective for
$i=\dim (X)-1$. 
The Lefschetz Theorem has generalizations in terms of local
cohomology, called Theorems of Barth Type.
For example, let $Y\subseteq \P_\C^n$ be Zariski 
closed
and $I\subseteq 
R=\C[x_0,\ldots,x_n]$ the 
defining ideal of $Y$. 
Then $H^i_{{\rm Sing}}(\P^n_\C;\C)\to H^i_{{\rm Sing}}(Y;\C)$ is an isomorphism for $i<
n-\cd(R,I)$ and injective if $i=n-\cd(R,I)$ (\cite{DM:DRCAV}, Theorem III.7.1).

A consequence of the work of Ogus and Hartshorne
(\cite{DM:Og}, 2.2, 2.3 and \cite{DM:DRCAV}, Theorem IV.3.1) is the
following. 
If $I\subseteq
R=\C[x_0,\ldots,x_n]$ is the defining 
ideal of a complex smooth variety $Y\subseteq \P^n_\C$ then, for
$i<n-\codim (Y)$,
\[
\dim_\C\soc_R
(H^0_\m(H^{n-i}_I (R)))=\dim_\C H^i_x(\tilde Y;\C)
\]
where
$H^i_x(\tilde 
Y;\C)$ stands for the $i$-th singular cohomology group of the affine
cone $\tilde Y$ over $Y$ with support in the vertex $x$ of $\tilde Y$ and
with coefficients in $\C$ (and $\soc_R(M)$  denotes the socle
$(0:_M(x_0,\ldots,x_n))\subseteq M$ for any 
$R$-module $M$), \cite{DM:L-Dmod}. 
These iterated local cohomology modules have a
special structure (cf.\ Subsection \ref{subsec-lambda}).

Local cohomology relates to the connectedness of the underlying spaces
as is shown by the following facts. If $Y$ is a complete intersection
of positive dimension in $\P^n_\C$, then $Y$ cannot be disconnected by
the removal of closed subsets of codimension 2 in $Y$ or higher,
\cite{DM:Br-R}. 
This is a
consequence of the so-called Hartshorne-Lichtenbaum vanishing theorem,
see \cite{DM:B-S}.

In a similar spirit one can show that if $(A,\m)$ is a complete
local domain of dimension $n$ and $f_1,\ldots,f_r$ are elements of the
maximal ideal with $r+2\le n$, then
$\var(f_1,\ldots,f_r)\setminus\{\m\}$ 
is connected, \cite{DM:Br-R}.

In fact, as we will discuss to some extent in Section \ref{sec-ausblick}, 
over the complex numbers the complex $\check
C^\bullet(R;f_1,\ldots,f_r)$ for $R=\C[x_1,\ldots,x_n]$ determines
the Betti numbers
$\dim_\C(H^i_{{\rm Sing}}(\C^n\setminus\var(f_1,\ldots,f_r);\C))$.
\subsection{The Master Plan} 
The cohomological dimension has been studied by many authors. For an
extensive list of references and some open questions 
we recommend to consult 
the very nice survey article \cite{DM:Hu}. 

It turns out that for the determination of $\cd(R,I)$ it is in
fact enough to find a test to 
decide whether or not the local cohomology module $H^i_I(R)=0$ for
given $i, R, I$. This is because $H^i_I(R)=0$ for all $i>c$ implies
$\cd(R,I)\le c$ (see \cite{DM:CDAV}, Section 1).  

Unfortunately,
calculations are complicated by the fact that $H^i_I(M)$ is rarely
finitely generated as $R$-module, even for very nice $R$ and $M$. 
In this
chapter we show how in an important class of examples one may still
carry out explicit computations, by enlarging $R$. 

We shall assume that 
$I\subseteq R_n=K[x_1,\ldots,x_n]$ where $K$ is a
computable field
containing the rational numbers. (By a
{\em computable field}\index{computable field} 
we mean a subfield $K$ of $\C$ such that $K$ is
described by a finite set of data and for which addition, subtraction,
multiplication and division as well as the test whether the result of
any of these operations is zero in the field can be executed by the
Turing machine. For example, $K$ could be $\Q[\sqrt 2]$ stored as a
2-dimensional vector space over $\Q$ with an appropriate
multiplication table.)

The ring of $K$-linear differential operators $D(R,K)$ of the
commutative $K$-algebra $R$ is defined inductively: one sets
$D_0(R,K)=R$, and for $i>0$ defines 
\[
D_i(R,K)=\left\{P\in\hom_K(R,R):Pr-rP\in D_{i-1}(R,K) \text{ for all } r\in
R\right\}.
\]
Here, $r\in R$ is interpreted as the endomorphism of $R$ that
multiplies by $r$.

The local cohomology
modules $H^i_I(R_n)$ have a natural
structure of finitely generated left $D(R_n,K)$-modules (see for
example \cite{DM:K2,DM:L-Dmod}).
The basic reason for  this finiteness is that in this case 
$R_n[f^{-1}]$ is a cyclic
$D(R_n,K)$-module, generated by $f^{a}$ for $\Z\ni a\ll 0$
(compare \cite{DM:B}): 
\begin{eqnarray}
\label{eqn-loc-iso}
R_n[f^{-1}]=D(R_n,K)\action f^{a}.
\end{eqnarray}

Using 
this finiteness we employ the theory of
Gr\"obner bases in $D(R_n,K)$
to develop 
algorithms that give a presentation of $H^i_I(R_n)$ and $H^i_\m (H^j_I(R_n))$
for all triples  $i,j\in \N$, $I\subseteq R_n$ in terms of generators and
relations over 
$D(R_n,K)$ (where $\m=R_n\cdot(x_1,\ldots,x_n)$), see Section \ref{sec-lc}. 
This also leads to an
algorithm for the computation of the invariants 
\[
\lambda_{i,j}(R_n/I)=\dim_K\soc_{R_n}(H^i_\m(H^{n-j}_I(R_n)))
\]
introduced in
\cite{DM:L-Dmod}.


At the basis for the computation of local cohomology are algorithms that
compute the localization of a $D(R_n,K)$-module at a hypersurface $f\in
R_n$. That means, if the left module $M={D(R_n,K)}^d/L$ is given by means of a
finite number of generators for the left module $L\subseteq {D(R_n,K)}^d$
then we want to compute a finite number of generators for the left
module $L'\subseteq {D(R_n,K)}^{d'}$ which satisfies

\[
{D(R_n,K)}^{d'}/L'\cong 
({D(R_n,K)}^d/L)\otimes_{R_n}R_n[f^{-1}]
,
\]
which we do in Section \ref{sec-loc}.  

Let $L$ be a left ideal of $D(R_n,K)$. 
The computation of the localization of 
$M=D(R_n,K)/L$ at $f\in R_n$ is closely related to the
$D(R_n,K)[s]$-module $\M_f$ generated by 
\begin{eqnarray}
\bar 1\otimes 1\otimes f^s\in M\otimes_{R_n} R_n[f^{-1},s]\otimes f^s
\end{eqnarray}
and the minimal polynomial $b_f(s)$ 
of $s$ on the quotient of $\M_f$ by its submodule $\M_f\cdot f$
generated over $D(R_n,K)[s]$ by $\bar 1\otimes f\otimes f^s$,  
 cf.\ Section \ref{sec-loc}. 
Algorithms for the computation of these objects have been
established by T.\ Oaku in a sequence of papers
\cite{DM:Oa,DM:Oa3,DM:Oa2}.

Astonishingly, the roots of $b_f(s)$ prescribe the exponents $a$ that
can be used in the isomorphism (\ref{eqn-loc-iso}) 
between $R_n[f^{-1}]$ and the
$D(R_n,K)$-module generated by $f^{a}$. Moreover, any
 good exponent $a$ can be used to transform $\M_f$ into
$M\otimes R_n[f^{-1}]$ by a suitable ``plugging in'' procedure. 

Thus the strategy for the computation of local cohomology will be
to compute $\M_f$ and a good $a$ for each $f\in\{f_1,\ldots,f_r\}$,
and then assemble the \v Cech complex.

\subsection{Outline of the Chapter}
The next section is devoted to a short introduction of results on
the Weyl algebra $D(R_n,K)$ and $D$-modules as they apply to our
work. 
We start with some remarks on the theory of Gr\"obner
bases in the Weyl algebra.

In Section \ref{sec-loc} we investigate Bernstein-Sato polynomials,
localizations and the \v Cech complex.
The purpose of
that section is to find a 
presentation of $M\otimes R_n[f^{-1}]$ as a cyclic $D(R_n,K)$-module 
if $M=D(R_n,K)/L$
is a given holonomic $D$-module (for a definition and some properties
of holonomic modules, see Subsection \ref{subsec-D-modules} below).



In
Section \ref{sec-lc} we describe 
algorithms that for
arbitrary $i,j,k,I$ 
determine the structure of
$H^k_I(R), H^i_\m (H^j_I(R))$ and find $\lambda_{i,j}(R/I)$. 
The final section is devoted to comments on 
implementations, efficiency, discussions of other topics, and open problems.



%\input{2.tex}
\section{The Weyl Algebra and Gr\"obner Bases}
\mylabel{sec-weyl}

$D$-modules, that is, rings or sheaves of differential operators and
modules over these, have been around for several decades and played
prominent roles in representation theory, some parts of analysis and
in algebraic geometry. The founding fathers of the theory are 
M.\ Sato, M.\ Kashiwara, T.\ Kawai, J.\ Bernstein, and A.\ Beilinson.
 The area has also benefited much from the work of P.\ Deligne,
J.-E.\ Bj\"ork, J.-E.\ Roos, 
B.\ Malgrange and  Z.\ Mebkhout. The more computational aspects of the
theory have been initiated by T.\ Oaku and N.\ Takayama.

The simplest example of a ring of differential
operators is given by the Weyl algebra, the ring of $K$-linear
differential operators on $R_n$. In characteristic
zero, this is a finitely generated $K$-algebra that resembles the ring
of polynomials in $2n$ variables but fails to be commutative.
\subsection{Notation} Throughout we shall use the following notation:
$K$ will 
denote a computable field 
of characteristic zero and $R_n=K[x_1,\ldots,x_n]$ the ring of polynomials
over $K$ in $n$ variables. The $K$-linear differential operators on
$R_n$ are
then the elements of  
\[D_n=K\langle
x_1,\del_1,\ldots,x_n,\del_n\rangle,
\]
the {\em $n$-th Weyl algebra}\myindex{Weyl algebra}
 over $K$, where the symbol $x_i$ denotes the operator ``multiply by
 $x_i$'' and $\del_i$ denotes the operator ``take partial
derivative with respect to $x_i$''. We therefore have in $D_n$ the relations
\begin{eqnarray*}
x_ix_j&=&x_jx_i\quad \text{ for all } 1\le i,j\le n,\\ 
\del_i\del_j&=&\del_j\del_i \quad \text{ for all } 1\le i,j\le n,\\
x_i\del_j&=&\del_jx_i \quad \text{ for all } 1\le i\not =j\le n,\\
\text{ and }\, x_i\del_i+1&=&\del_ix_i \quad \text{ for all } 1\le i\le n. 
\end{eqnarray*}
The last relation is nothing but
the {\em product} (or {\em Leibniz})\myindex{Leibniz rule} {\em rule}, 
$xf'+f=(xf)'$. 
We shall use multi-index notation: $x^\alpha\del^\beta$ denotes the
monomial 
\[
{x_1}^{\alpha_1}\cdots {x_n}^{\alpha_n}\cdot
{\del_1}^{\beta_1}\cdots{\del_n}^{\beta_n}
\]
and $|\alpha|=\alpha_1+ \dots +\alpha_n$. 
 
In order to keep the product $\del_i x_i\in D_n$ 
and the application of $\del_i\in D_n$ to $x_i\in R_n$ apart, we shall
write $\del_i\action (g)$ to mean the result of 
the action\myindex{action (of a differential operator)} of $\del_i$ on $g\in
R_n$. So for example, $\del_ix_i=x_i\del_i+1\in D_n$ but
$\del_i\action x_i=1\in R_n$. The action of $D_n$ on $R_n$ takes
precedence over the multiplication in $R_n$ (and is of course
compatible with the multiplication in $D_n$), so for
example $\del_2\action (x_1)x_2=0\cdot x_2=0\in R_n$.

The symbol
$\m$ will stand for the maximal ideal $R_n\cdot (x_1,\ldots,x_n)$ of
$R_n$, $\Delta$ will denote the maximal left ideal $D_n\cdot 
(\del_1,\ldots,\del_n)$
of $D_n$ and  $I$ will stand for the ideal $R_n\cdot (f_1,\ldots,f_r)$ in
$R_n$. Every $D_n$-module becomes an $R_n$-module via the embedding
$R_n\into D_n$ as $D_0(R_n,K)$. 

All tensor products in this chapter will be over $R_n$ and all
$D_n$-modules (resp.\ ideals)\myindex{$D_n$-modules}
 will be left modules (resp.\ left ideals) unless specified otherwise.
%
\subsection{Gr\"obner Bases in $D_n$}
\mylabel{subsec-GB}
This subsection is a severely shortened version of Chapter 1 in
\cite{DM:SST} (and we strongly recommend that the reader take a look at
this book). The 
purpose is to see how Gr\"obner basis theory applies to the Weyl algebra.

The elements in $D_n$ allow a {\em normally ordered
expression}\myindex{normally ordered expression}. 
Namely, if $P\in D_n$ then we can write it as 
\[
P=\sum_{(\alpha,\beta)\in E}c_{\alpha,\beta}x^\alpha\del^\beta
\]
where $E$ is a finite subset of $\N^{2n}$.
Thus, as $K$-vector spaces there is an isomorphism
\[
\Psi:K[x,\xi]\to D_n
\]
(with $\xi=\xi_1,\ldots,\xi_n$) 
sending $x^\alpha\xi^\beta$ to $x^\alpha\del^\beta$. We will assume
that every $P\in D_n$ is normally ordered.

We shall say that $(u,v)\in\R^{2n}$ is a {\em weight vector for
$D_n$}\myindex{weight vector for $D_n$}
if $u+v\geq 0$, that is $u_i+v_i\geq 0$ for all $1\le i\le n$. We
set the {\em weight}\myindex{weight}
 of the monomial $x^\alpha\del^\beta$ under
$(u,v)$ to be $u\cdot\alpha+v\cdot\beta$ (scalar product). The weight
of an operator is then the maximum of the weights of the nonzero
monomials appearing in the normally ordered expression of $P$.
 If $(u,v)$
is a weight vector for $D_n$, there is an associated graded ring
$\gr_{(u,v)}(D_n)$ with 
\[
\gr_{(u,v)}^r(D_n)=\frac{\{P\in D_n : w(P)\le r\}}{\{P\in D_n : w(P)<
r\}}.
\]
So $\gr_{(u,v)}(D_n)$ is 
the
$K$-algebra on the symbols $\{x_i:1\le i\le n\}\cup
\{\del_i:u_i+v_i=0\}\cup\{\xi_i:u_i+v_i>0\}$. Here all variables
commute with each other except $\del_i$ and $x_i$ for which the
Leibniz rule holds.

Each $P\in D_n$ has an {\em initial form}\myindex{initial form}
 or {\em symbol} \myindex{symbol} $\ini_{(u,v)}(P)$ 
in $\gr_{(u,v)}(D_n)$ defined by
taking all monomials in the normally ordered expression for $P$ that
have maximal weight, and replacing all $\del_i$ with $u_i+v_i>0$ by
the corresponding $\xi_i$. 

The inequality $u_i+v_i\geq
0$ is 
needed to assure that the product of the initial forms of two
operators equals the initial form of their product: one would not want
to have $\ini(\del_i\cdot x_i)=\ini(x_i\cdot\del_i +1)=1$.



A weight of particular importance is $-u=v=(1,\ldots,1)$, or more
generally $-u=v=(1,\ldots,1,0,\ldots,0)$. In these cases
$\gr_{(u,v)}(D_n)\cong D_n$. On the other hand, if $u+v$ is
componentwise positive, then $\gr_{(u,v)}(D_n)$ is commutative
(compare the initial forms of $\del_ix_i$ and $x_i\del_i$) and
isomorphic to the polynomial ring in $2n$ variables corresponding to
the symbols of $x_1,\ldots,x_n,\del_1,\ldots,\del_n$. 


If $L$ is a left ideal in $D_n$ 
we write $\ini_{(u,v)}(L)$\myindex{$\ini_{(u,v)}(L)$}
 for $\{\ini_{(u,v)}(P):P\in L\}$. This is a
left ideal in $\gr_{(u,v)}(D_n)$. If $G\subset L$ is a finite set we
call it a {\em $(u,v)$-Gr\"obner basis}\myindex{$(u,v)$-Gr\"obner basis}
 if the left ideal of
$\gr_{(u,v)}(D_n)$ generated by the initial forms of the elements of
$G$ agrees with $\ini_{(u,v)}(L)$.

A {\em multiplicative monomial order on $D_n$}\myindex{monomial order
on $D_n$} 
is a total order $\order$
on the normally ordered monomials such that 
\begin{enumerate}
\item $1\order x_i\del_i$ for all $i$, and
\item $x^\alpha\del^\beta\order x^{\alpha'}\del^{\beta'}$ implies 
$x^{\alpha+\alpha''}\del^{\beta+\beta''}\order
x^{\alpha'+\alpha''}\del^{\beta'+\beta''}$ for all
$\alpha'',\beta''\in \N^n$. 
\end{enumerate}
A multiplicative monomial order is a {\em
term order}\myindex{term order on $D_n$}
 if $1$ is the (unique) smallest monomial. 
Multiplicative monomial orders, and more specifically term orders,
clearly abound.

Multiplicative monomial orders (and hence term orders) allow the
construction of initial forms just like weight vectors. Now, however, the
initial forms are always monomials, and always elements of $K[x,\xi]$
(due to the total order requirement on $\order$). One
defines Gr\"obner bases for multiplicative monomial orders analogously
to the weight vector case.

For our algorithms 
we have need to compute weight vector Gr\"obner bases, and this can be
done as follows. Suppose $(u,v)$ is a weight vector on $D_n$ and
$\order$ a term order. Define a
multiplicative monomial order $\order_{(u,v)}$ as follows:
\begin{eqnarray*}
x^\alpha\del^\beta\order_{(u,v)}
x^{\alpha'}\del^{\beta'}&\Leftrightarrow& \left[(\alpha-\alpha')\cdot
u+(\beta-\beta')\cdot 
v<0\right] \text{ or }\\
&&\left[(\alpha-\alpha')\cdot
u+(\beta-\beta')\cdot 
v=0 \text{ and } x^\alpha\del^\beta\order x^{\alpha'}\del^{\beta'}\right].
\end{eqnarray*}
Note that $\order_{(u,v)}$ is a term order precisely when $(u,v)$ is
componentwise nonnegative.
\begin{theorem}[\cite{DM:SST}, Theorem 1.1.6.]
Let $L$ be a left ideal in $D_n$, $(u,v)$ a weight vector for $D_n$, 
$\order$ a term
order and $G$ a Gr\"obner basis for $L$ with respect to
$\order_{(u,v)}$. Then
\begin{enumerate}
\item $G$ is a Gr\"obner basis for $L$ with respect to $(u,v)$.
\item $\ini_{(u,v)}(G)$ is a Gr\"obner basis for $\ini_{(u,v)}(L)$ with
respect to $\order$.\qed
\end{enumerate}
\end{theorem}
We end this subsection with the remarks that Gr\"obner bases with
respect to multiplicative monomial
 orders can be computed using the Buchberger algorithm
adapted to the non-commutative situation (thus, Gr\"obner bases with
respect to  weight vectors are computable according to the theorem), 
and that the computation of
syzygies, kernels, intersections and preimages in $D_n$ 
works essentially as in the commutative algebra $K[x,\xi]$. For precise
statements of the algorithms we refer the reader to \cite{DM:SST}.

\subsection{$D$-modules} \mylabel{subsec-D-modules}
A good introduction to $D$-modules\myindex{$D$-modules}
 are the book by J.-E.\ Bj\"ork,
\cite{DM:B}, the nice introduction \cite{DM:Coutinho} by S.~Coutinho, 
and the lecture notes by J.~Bernstein
 \cite{DM:Bernstein-notes}. 
In this subsection  we list some properties of
localizations of $R_n$ that are important for module-finiteness over
 $D_n$. Most of this section is taken from Section 1 in \cite{DM:B}.

Let $f\in R_n$. Then the $R_n$-module $R_n[f^{-1}]$ has a
structure as left $D_n$-module via the extension of the action
$\action$\myindex{action (of a differential operator)}: 
\begin{eqnarray*}
x_i\action (\frac{g}{f^k})=\frac{x_ig}{f^k},&\quad&
\del_i\action (\frac{g}{f^k})=\frac{\del_i\action(g)f-
   k\del_i\action(f)g}{f^{k+1}}.
\end{eqnarray*} 
This may be thought of as a special case of localizing a $D_n$-module: if
$M$ is a $D_n$-module and $f\in R_n$ then $M\otimes_{R_n} R_n[f^{-1}]$ 
becomes a
$D_n$-module via the {\em product rule}\myindex{product
rule}\myindex{action}
\begin{eqnarray*}
x_i\action (m\otimes \frac{g}{f^k})=m\otimes (\frac{x_ig}{f^k}),
&\quad&
\del_i\action (m\otimes \frac{g}{f^k})=
 m\otimes \del_i\action(\frac{g}{f^k})+\del_i m\otimes \frac{g}{f^k}.
\end{eqnarray*}

Of particular interest are the {\em
holonomic} modules\myindex{holonomic module}
 which are those finitely generated $D_n$-modules $M$
for which $\ext^j_{D_n}(M,D_n)$ vanishes 
unless $j=n$. 
This innocent looking definition has surprising consequences,
some of which we discuss now. 

The holonomic modules form a full Abelian subcategory of
the category of left $D_n$-modules, closed under the formation of
subquotients. 
Our standard example of a holonomic module is\myindex{$R_n$ as a
$D_n$-module} 
\[
R_n=D_n/\Delta.
\] 
This equality may require some thought -- it pictures $R_n$ as a
$D_n$-module generated by $1\in R_n$. It is particularly noteworthy
that not all elements of $R_n$ are killed by $\Delta$ -- quite
impossible if $D_n$ were commutative. 

Holonomic modules are
always cyclic and of finite length over $D_n$. These fundamental
properties are consequences of the {\em Bernstein
inequality}\myindex{Bernstein inequality}. To
understand this inequality we associate with the $D_n$-module
$M=D_n/L$ the Hilbert function $q_L(k)$ with values in the integers which
counts for each $k\in\N$ the number of monomials $x^\alpha\del^\beta$ with
$|\alpha|+|\beta|\le k$ whose cosets in $M$ are $K$-linearly
independent. The filtration $k\mapsto K\cdot \{x^\alpha\del^\beta\mod L:
|\alpha|+|\beta|\le k\}$ is called the \myindex{Bernstein filtration} 
{\em Bernstein filtration}. The Bernstein inequality states that 
$q_L(k)$ is either identically 
zero (in which case $M=0$) or asymptotically a
polynomial in $k$ of degree between $n$ and $2n$.
This degree is called the {\em dimension of $M$}\myindex{dimension (of a
$D$-module)}.
A holonomic module is one of dimension $n$, the minimal possible value
for a nonzero module.

This characterization of holonomicity can be used quite easily to
check with \Mtwo that $R_n$ is holonomic. Namely, let's say $n=3$. 
Start a \Mtwo session with
\beginOutput
i1 : load "D-modules.m2"\\
\endOutput
\beginOutput
i2 : D = QQ[x,y,z,Dx,Dy,Dz, WeylAlgebra => \{x=>Dx, y=>Dy, z=>Dz\}]\\
\emptyLine
o2 = D\\
\emptyLine
o2 : PolynomialRing\\
\endOutput
\beginOutput
i3 : Delta = ideal(Dx,Dy,Dz)\\
\emptyLine
o3 = ideal (Dx, Dy, Dz)\\
\emptyLine
o3 : Ideal of D\\
\endOutput
The first of these commands loads the $D$-module library 
by A.\ Leykin, M.\ Stillman  and H.\ Tsai, \cite{DM:M2D}. 
The second line defines the
base ring $D_3=\Q\langle x,y,z,\del_x,\del_y,\del_z\rangle$, 
while the third command defines the $D_3$-module $D_3/\Delta\cong
R_3$.

As one can see, \Mtwo thinks of {\tt D} as a ring of polynomials. This is
using the vector space isomorphism $\Psi$ from Subsection
\ref{subsec-GB}. Of course, two elements are multiplied according to
the Leibniz rule. 
To see how \Mtwo uses the map $\Psi$, we enter the following expression.
\beginOutput
i4 : (Dx * x)^2\\
\emptyLine
\      2  2\\
o4 = x Dx  + 3x*Dx + 1\\
\emptyLine
o4 : D\\
\endOutput
All Weyl algebra ideals and modules 
are by default left
ideals and left modules in \Mtwo. 

If we don't explicitly specify a monomial ordering to be used in the Weyl
algebra, then \Mtwo uses graded reverse lex ({\tt GRevLex}), as we can see by
examining the options of the ring.
\beginOutput
i5 : options D\\
\emptyLine
o5 = OptionTable\{Adjust => identity                        \}\\
\                 Degrees => \{\{1\}, \{1\}, \{1\}, \{1\}, \{1\}, \{1\}\}\\
\                 Inverses => false\\
\                 MonomialOrder => GRevLex\\
\                 MonomialSize => 8\\
\                 NewMonomialOrder => \\
\                 Repair => identity\\
\                 SkewCommutative => false\\
\                 VariableBaseName => \\
\                 VariableOrder => \\
\                 Variables => \{x, y, z, Dx, Dy, Dz\}\\
\                 Weights => \{\}\\
\                 WeylAlgebra => \{x => Dx, y => Dy, z => Dz\}\\
\emptyLine
o5 : OptionTable\\
\endOutput

To compute the initial ideal of $\Delta$ with respect to the weight
that associates $1$ to each $\del$ and to each variable, execute
\beginOutput
i6 : DeltaBern = inw(Delta,\{1,1,1,1,1,1\}) \\
\emptyLine
o6 = ideal (Dz, Dy, Dx)\\
\emptyLine
o6 : Ideal of QQ [x, y, z, Dx, Dy, Dz]\\
\endOutput
The command {\tt inw} can be used with any weight vector for $D_n$ as
second argument. 
One notes that the output is not an ideal in a Weyl algebra any more,
but in a ring of polynomials, as it should.
The dimension of $R_3$, which is the dimension of the variety
associated to {\tt DeltaBern}, is computed by
\beginOutput
i7 : dim DeltaBern \\
\emptyLine
o7 = 3\\
\endOutput
As this is equal to $n=3$, the ideal $\Delta$ is holonomic.

\bigskip

Let $R_n[f^{-1},s]\otimes f^s$ be the free $R_n[f^{-1},s]$-module
generated by the symbol $f^s$.  
Using the action $\action$ of $D_n$ on $R_n[f^{-1},s]$ we define an
action\myindex{action (of a differential operator)} $\action$ 
of $D_n[s]$ on 
$R_n[f^{-1},s]\otimes f^s$ by setting
\begin{eqnarray*}
s\action \left(\frac{g(x,s)}{f^k}\otimes f^s\right)&=& 
\frac{sg(x,s)}{f^k}\otimes f^s,\\
x_i\action\left(\frac{g(x,s)}{f^k}\otimes f^s\right)&=&
 \frac{x_ig(x,s)}{f^k}\otimes f^s,\\
\del_i\action\left(\frac{g(x,s)}{f^k}\otimes f^s\right)&=&
 \left(\del_i\action\left(\frac{g(x,s)}{f^k}\right)+s\del_i\action(f)\cdot
 \frac{g(x,s)}{f^{k+1}}\right)\otimes f^s.
\end{eqnarray*}
The last rule justifies the choice for the symbol of the generator.

Writing $M=D_n/L$ and denoting by $\bar 1$ the coset of $1\in D_n$ in
$M$, 
this action extends to an action\myindex{action (of a differential
 operator)}
 of $D_n[s]$ on 
\begin{eqnarray}
\M^L_f=D_n[s]\action(\bar 1\otimes 1\otimes f^s)\subseteq 
M\otimes_{R_n} \left(R_n[f^{-1},s]\otimes f^s\right)
\end{eqnarray}
by the product rule\myindex{product rule} for all left $D_n$-modules $M$.
The interesting bit about $\M^L_f$
is the following fact.
If $M=D_n/L$ is holonomic 
then
there is a nonzero polynomial $b(s)$ in $K[s]$ and an
operator $P(s)\in D_n[s]$ such that 
\begin{eqnarray}
\label{def-b-poly}
P(s)\action(\bar 1\otimes f\otimes f^s)=
\bar 1\otimes b(s)\otimes f^s
\end{eqnarray}
in $\M^L_f$.
This entertaining equality, often written as 
\[
P(s)\left( \bar 1\otimes f^{s+1}\right)=\bar{b(s)}\otimes f^s,
\] 
says that $P(s)$ is roughly equal to
division by $f$.
The unique monic polynomial that divides 
all other polynomials $b(s)$ satisfying an identity of this type is called the
{\em Bernstein} (or also {\em Bernstein-Sato}) {\em
polynomial}\myindex{Bernstein (Bernstein-Sato) polynomial} of $L$ and 
$f$ and denoted by $b_f^L(s)$\myindex{$b_f^L(s)$}.
Any operator $P(s)$ that satisfies (\ref{def-b-poly}) with
$b(s)=b_f(s)$ 
 we shall call a 
{\em Bernstein operator}\myindex{Bernstein operator} and refer
to the roots of $b_f^L(s)$ as {\em Bernstein roots}\myindex{Bernstein
root} 
of $f$ on $D_n/L$.
It is clear from (\ref{def-b-poly}) and the definitions that
$b^L_f(s)$ is the minimal polynomial of $s$ on the quotient of
$\M^L_f$ by $D_n[s]\action(\bar 1\otimes f\otimes f^s)$. 

The Bernstein roots of the polynomial $f$ are somewhat mysterious, but
related to other algebro-geometric invariants as, for example, the
monodromy of $f$ (see \cite{DM:M}), the Igusa zeta function (see \cite{Loeser}), and the log-canonical threshold
(see \cite{DM:Kollar}). For a long time it was also unclear how to
compute $b_f(s)$ for given $f$. In \cite{DM:Yano} many interesting examples of
Bernstein-Sato polynomials are worked out by hand, while in
\cite{DM:AK,DM:Brianconetal,DM:Maisonobe,DM:Satoetal}
algorithms were given that compute $b_f(s)$ under certain conditions
on $f$. The general algorithm we are going to explain was given by
T.\ Oaku.  Here is a classical example.
\begin{example}
Let $f=\sum_{i=1}^n {x_i}^2$ and $M=R_n$ with $L=\Delta$. One can check
that 
\[
\sum_{i=1}^n{\del_i}^2\action(\bar 1\otimes 1\otimes f^{s+1})=\bar 1\otimes
4(s+1)(\frac{n}{2}+s)\otimes f^{s}
\]
and hence that 
$\frac{1}{4}\sum_{i=1}^n{\del_i}^2$ is a Bernstein operator while the
Bernstein roots of $f$ are $-1$ and $-{n}/{2}$ and the Bernstein
polynomial is $(s+1)(s+\frac{n}{2})$.
\end{example}
\begin{example}
Although in the previous example the Bernstein operator looked a lot
like the polynomial $f$, this is not often the case and it is
usually hard to guess Bernstein operators. For example, one has
\[
\left(\frac{1}{27}\,{\del_y}^3+
\frac{y}{6}{\del_x}^2\del_y+\frac{x}{8}{\del_x}^3\right)(x^2+y^3)^{s+1}=
(s+\frac{5}{6})(s+1)(s+\frac{7}{6}) (x^2+y^3)^s.
\]
In the case of non-quasi-homogeneous polynomials, there is usually no
resemblance between $f$ and any Bernstein operator.
\end{example}
A very important property of holonomic modules is the
(somewhat counterintuitive) fact that any localization of a
holonomic module $M=D_n/L$ at a single element 
(and hence at any finite number
of elements) of $R_n$ is holonomic (\cite{DM:B}, 1.5.9) and 
in particular cyclic over $D_n$, generated by $\bar 1\otimes f^{a}$ for
sufficiently small $a\in \Z$. 
As a
special case we note that 
localizations of $R_n$ are holonomic, and hence finitely generated 
over $D_n$. 
Coming back to the \v Cech complex we see that the complex $\check
C^\bullet(M;f_1,\ldots,f_r)$   
consists of holonomic $D_n$-modules whenever $M$ is holonomic. 

As a consequence, local cohomology modules of $R_n$ are 
$D_n$-modules and in fact holonomic. To see this it 
suffices to know that the maps
in the \v Cech complex are
$D_n$-linear, which we will explain in Section \ref{sec-lc}. 
Since the category 
of holonomic $D_n$-modules and their $D_n$-linear maps
is closed under subquotients,
holonomicity of $H^k_I(R_n)$
follows. 

For similar reasons, $H^i_\m (H^j_I(R_n))$ is holonomic for
$i,j\in \N$ (since $H^j_I(R_n)$ is holonomic).  
These modules, investigated in Subsections \ref{subsec-lclc} and
\ref{subsec-lambda}, are 
rather special $R_n$-modules and seem to carry some very interesting
information about $\var(I)$, see \cite{DM:G-S,DM:W-lambda}. 

The fact that $R_n$ is holonomic and every localization of a holonomic
module is as well, provides motivation for us to study this class of
modules. There are, however, more occasions where holonomic modules
show up. One such situation arises in the study of linear partial
differential equations. More specifically, the so-called GKZ-systems
(which we will meet again in the final chapter) provide a very
interesting class of objects with fascinating combinatorial and
analytic properties \cite{DM:SST}. 


%\input{3.tex}
\section{Bernstein-Sato Polynomials and Localization}
\mylabel{sec-loc}
We mentioned in the introduction that for the computation
of local cohomology the following is an
important algorithmic problem to solve.

\begin{problem}
\mylabel{prob}
Given $f\in  
R_n$ and a left ideal $L\subseteq D_n$ such that $M=D_n/L$ is holonomic,  
compute the structure of the module 
$D_n/L\otimes R_n[f^{-1}]$ in terms of generators and relations. 
\end{problem}
This section is about solving Problem \ref{prob}.
\subsection{The Line of Attack}
Recall for a given $D_n$-module $M=D_n/L$ the action of $D_n[s]$ on the
tensor product $M\otimes_{R_n}(R_n[f^{-1},s]\otimes f^s)$ from
Subsection \ref{subsec-D-modules}. 
We begin with defining an ideal of operators:
\begin{definition}
Let  $J^L(f^s)$ stand for the ideal in $D_n[s]$
that kills $\bar 1\otimes 1\otimes f^s\in (D_n/L)\otimes_{R_n}
R_n[f^{-1},s]\otimes f^s$. 
\end{definition}
It turns out that it is very useful to know this ideal.
If $L=\Delta$ then there are some obvious candidates for generators of
$J^L(f^s)$. For example, there are $f\del_i-\del_i\action(f)s$ for all
$i$. However, unless the affine hypersurface defined by $f=0$ is smooth, these will not generate
$J^\Delta(f^s)$. For a more general $L$, there is a similar set of
(somewhat less) obvious candidates, but again finding all elements of
$J^\Delta(f^s)$ is far from elementary, even for smooth $f$.

In order to find $J^L(f^s)$, we will consider the module 
$(D_n/L)\otimes R_n[f^{-1},s]\otimes f^s$ over the ring $D_{n+1}=D_n\langle
t,\del_t\rangle$ by defining an appropriate action of $t$ and $\del_t$
on it. It is then not hard to 
compute the ideal $J^L_{n+1}(f^s)\subseteq D_{n+1}$ consisting of all
operators that kill $\bar 1\otimes 1\otimes f^s$, see Lemma
\ref{lem-malgrange}.  
In Proposition \ref{prop-oaku} 
we will then explain how to compute $J^L(f^s)$ from $J_{n+1}^L(f^s)$. 

This construction 
gives an answer to the question of determining
a presentation of  $D_n\action (\bar 1\otimes f^a)$ for ``most'' $a\in
K$, which we make precise as follows.
\begin{definition}
We say that a property depending on $a\in K^m$ {\em holds for $a$
 in
very general position}, if there is a countable set of hypersurfaces
 in $K^m$ such that the property holds for all $a$ not on any of the
 exceptional hypersurfaces. 
\end{definition}
It will turn out that for $a\in K$ in very general position $J^L(f^s)$
``is'' the annihilator for $f^a$: we shall very explicitly 
identify a countable number of
exceptional values
in $K$ such that if $a$ is not equal to one of them, then $J^L(f^s)$
evaluates under $s\mapsto a$ to the annihilator inside $D_n$ of $\bar 1\otimes f^a$.
  
For $a\in\Z$ we have
of course
$D_n\action (\bar 1\otimes f^a)\subseteq M\otimes 
R_n[f^{-1}]$ but the inclusion may be strict
(e.g., for $L=\Delta$ and $a=0$). 
Proposition \ref{prop-kashiwara}
shows how 
$(D_n/L)\otimes R_n[f^{-1}]$ and
$J^L(f^s)$ are related.
\subsection{Undetermined Exponents}
Consider $D_{n+1}=D_n\langle t,\del_t\rangle$, \myindex{$D_{n+1}$} 
the Weyl algebra in
$x_1,\ldots,x_n$ and the new variable $t$. B.\ Malgrange
\cite{DM:M} 
has defined an\myindex{action} 
action $\action$ of $D_{n+1}$ on 
$(D_n/L)\otimes R_n[f^{-1},s]\otimes f^s$ as follows.
We require that $x_i$ acts as multiplication on the first factor, and
for the other variables we set (with $\bar P\in D_n/L$ and  $g(x,s)\in
R_n[s]$) 
\begin{eqnarray*}
\del_i\action(\bar P\otimes \frac{g(x,s)}{f^k}\otimes f^s)&=
  &\left(\bar P\otimes
  \left(\del_i\action(\frac{g(x,s)}{f^k})+\frac{s\del_i\action(f)g(x,s)}{f^{k+1}}\right)  
\right. \\ & & \left. \quad {}
   +\bar{\del_iP}\otimes \frac{g(x,s)}{f^k}\right)\otimes f^s,\\
t\action(\bar P\otimes \frac{g(x,s)}{f^k}\otimes f^s)&=
  &\bar P\otimes \frac{g(x,s+1)f}{f^k}\otimes f^s,\\
\del_t\action(\bar P\otimes \frac{g(x,s)}{f^k}\otimes f^s)&=
  &\bar P\otimes \frac{-sg(x,s-1)}{f^{k+1}} \otimes f^s.
\end{eqnarray*}
One checks that this actually
defines a left $D_{n+1}$-module structure 
 (i.e., $\del_tt$ acts like $t\del_t+1$) 
and that
$-\del_tt$ acts as 
multiplication by $s$. 

\begin{definition}
We denote by $J^L_{n+1}(f^s)$ the ideal in $D_{n+1}$ that annihilates the
element $\bar 1\otimes 1\otimes f^s$ in $(D_n/L)\otimes
R_n[f^{-1},s]\otimes f^s$ with $D_{n+1}$ acting
as defined above. 
Then we have an induced morphism
of $D_{n+1}$-modules $D_{n+1}/J^L_{n+1}(f^s)\to (D_n/L)\otimes
R_n[f^{-1},s]\otimes f^s$ sending
$P+J^L_{n+1}(f^s)$ to 
$P\action (\bar 1\otimes 1\otimes f^s)$. 
\end{definition}

We say that an ideal $L\subseteq D_n$ is {\em
 $f$-saturated}\myindex{saturated@$f$-saturated}
 if $f\cdot
P\in L$ implies $P\in L$ and we say that $D_n/L$ is {\em $f$-torsion
free}\myindex{torsion free@$f$-torsion free} if $L$ is $f$-saturated. 
$R_n$ and all its localizations are examples of $f$-torsion free
modules for arbitrary $f$.

The
following lemma is a modification of Lemma 4.1 in \cite{DM:M} 
where the special case
$L=D_n\cdot (\del_1,\ldots,\del_n), D_n/L=R_n$ is considered (compare also
\cite{DM:W1}).

\begin{lemma}
\mylabel{lem-malgrange}
Suppose that $L=D_n\cdot (P_1,\ldots,P_r)$ is $f$-saturated.
With the above definitions, $J^L_{n+1}(f^s)$ is the
ideal generated by $f-t$ together with the images of the $P_j$ under
the automorphism $\phi$ of $D_{n+1}$ induced by $x_i\mapsto x_i$ for all
$i$, and $t \mapsto t-f$. 
\end{lemma}

\begin{proof}
The automorphism sends $\del_i$ to $\del_i+\del_i\action(f)\del_t$ 
and $\del_t$ to
$\del_t$. So if we write $P_j$ as a polynomial
$P_j(\del_1,\ldots,\del_n)$
in the $\del_i$ 
with coefficients in
$K[x_1,\ldots,x_n]$, then 
\[
\phi
(P_j)=P_j(\del_1+\del_1\action(f)\del_t,\ldots,\del_n+\del_n\action(f)\del_t).
\]

One checks that $(\del_i+\del_i\action(f)\del_t)\action 
(\bar Q\otimes 1\otimes f^s)=
\bar{\del_i
Q}\otimes 1\otimes f^s$ for all $Q\in D_{n+1}$, so that
$\phi(P_j(\del_1,\ldots,\del_n))\action (
\bar 1\otimes 1\otimes f^s)=\bar{P_j(\del_1,\ldots,\del_n)}\otimes 1
\otimes f^s=0$. By
definition, $f\action
(\bar 1\otimes 1\otimes f^s)=t\action (\bar 1\otimes 1\otimes f^s)$. So $t-f\in
J^L_{n+1}(f^s)$ and $\phi(P_j)\in J^L_{n+1}(f^s)$ for $j=1,\ldots,r$.

Conversely let $P\action (\bar 1\otimes 1\otimes f^s)=0$. The proof that
$P\in\phi(J^L_{n+1}+D_{n+1}\cdot t)$ relies on
an elimination idea and has some Gr\"obner basis flavor. 
We have to show that
$P\in D_{n+1}\cdot (\phi(P_1),\ldots,\phi(P_r),t-f)$. 
We may assume, that $P$ does
not contain 
any power of  $t$ since we can eliminate $t$ using $f-t$. Now rewrite $P$ in
terms of $\del_t$ and the $\del_i+\del_i\action(f)\del_t$. Say, 
$P=\sum_{\alpha,\beta}
\del_t^\alpha
x^\beta
Q_{\alpha,\beta}
(\del_1+\del_1\action(f)\del_t,\ldots,\del_n+\del_n\action(f)\del_t)$, 
where the $Q_{\alpha,\beta}\in K[y_1,\ldots,y_n]$ are polynomial
expressions. 
Then
\[
P\action (\bar 1\otimes 1\otimes f^s)=\sum_{\alpha,\beta} 
\del_t^\alpha\action(
\bar{x^\beta
Q_{\alpha,\beta}(\del_1,\ldots,\del_n)}\otimes 1\otimes f^s).
\] 
Let $\bar\alpha$ be the largest $\alpha\in\N$ for which there is a
nonzero $Q_{\alpha,\beta}$ occurring in $P=\sum_{\alpha,\beta}
\del_t^\alpha
x^\beta Q_{\alpha,\beta}(\del_1+\del_1\action(f)\del_t,\ldots,\del_n+\del_n\action(f)\del_t)$.
We show that the sum of terms that contain
$\del_t^{\bar\alpha}$ is in $D_{n+1}\cdot \phi(L)$ as
follows. In 
order for $P\action 
(\bar 1\otimes 1\otimes f^s)$ to vanish, the sum of terms with the
highest $s$-power, namely $s^{\bar\alpha}$, must vanish. Hence  
$\sum_\beta 
x^\beta Q_{\bar\alpha,\beta}(\del_1,\ldots,\del_n)
\otimes (-1/f)^{\bar\alpha}\otimes f^s
\in
L\otimes R_n[f^{-1},s]\otimes f^s$ as $R_n[f^{-1},s]$ is $R_n[s]$-flat.
It follows that $\sum_\beta x^\beta
Q_{\bar\alpha,\beta}(\del_1,\ldots,\del_n)\in L$ ($L$ is
$f$-saturated!) and hence $\sum_\beta
\del_t^{\bar\alpha}
x^\beta
Q_{\bar\alpha,\beta}
(\del_1+\del_1\action(f)\del_t,\ldots,\del_n+\del_n\action(f)  
\del_t)\in
D_{n+1}\cdot \phi(L)$ as announced.

So by the first part,
$P-\sum_\beta \del_t^{\bar\alpha} x^\beta
Q_{\bar\alpha,\beta}(\del_1+\del_1\action(f)\del_t,\ldots,\del_n+\del_n
\action(f)\del_t)$ kills
$\bar 1\otimes 1\otimes f^s$, but is of 
smaller degree in $\del_t$ than $P$ was.

The claim follows by induction on $\bar\alpha$.\qed
\end{proof}

\bigskip

If we identify $D_n[-\del_tt]$ with $D_n[s]$ then $J^L_{n+1}(f^s)\cap
D_n[-\del_tt]$ is identified with  $J^L(f^s)$ since, as we observed
earlier, $-\del_tt$ multiplies by $s$ on $\M^L_f$. 
As we pointed out in the beginning, the crux of our algorithms is to
calculate $J^L(f^s)=J^L_{n+1}(f^s)\cap D_n[s]$. We shall deal with
this computation now.

\mylabel{subsec-oaku}
In Theorem 19 of \cite{DM:Oa2}, T.~Oaku
showed how to construct a generating set for $J^L(f^s)$ in the case 
$L=D_n\cdot(\del_1,\ldots,\del_n)$. 
%According to Subsection 
%\ref{subsec-special-exp}, $J^L(f^s)$ is the
%intersection of $J^L_{n+1}(f^s)$ with $D_n[-\del_tt]$. 
Using his ideas we explain how one may calculate $J\cap
D_n[-\del_tt]$ whenever $J\subseteq D_{n+1}$ is any given ideal, and as a
corollary develop an algorithm that for $f$-saturated $D_n/L$
computes $J^L(f^s)=J^L_{n+1}(f^s)\cap D_n[-\del_tt]$. 

We first review some work
of Oaku. 
On $D_{n+1}$ we define the weight vector\myindex{weight vector}
 $w$ by  $w(t)=1,
w(\del_t)=-1, w(x_i)=w(\del_i)=0$ and we extend it to
$D_{n+1}[y_1,y_2]$ by $w(y_1)=-w(y_2)=1$. If $P=\sum_i P_i\in
D_{n+1}[y_1,y_2]$ and all $P_i$ are monomials, then we will write
$(P)^h$ for the operator $\sum_i P_i\cdot y_1^{d_i}$ where
$d_i=\max_j(w(P_j))-w(P_i)$ and call it the {\em
$y_1$-homogenization}\myindex{$y_1$-homogenization} 
of $P$.

Note that the
Buchberger algorithm preserves homogeneity 
in the following sense: if a set of generators for an ideal is given
and these generators are homogeneous with respect to the weight above,
then any new generator for the ideal constructed with the classical
Buchberger algorithm will also be homogeneous. (This is a consequence
of the facts that the $y_i$ commute with all other variables and that
$\del_t t=t\del_t+1$ is homogeneous of weight zero.) 
This homogeneity is very important for the following 
result of Oaku:
\begin{proposition}
\mylabel{prop-oaku}
Let $J=D_{n+1}\cdot(Q_1,\ldots,Q_r)$.
Let $I$ be the left ideal in $D_{n+1}[y_1]$ generated by the
$y_1$-homogenizations $(Q_i)^h$ of the $Q_i$, relative to the weight
$w$ above, and set $\tilde
I=D_{n+1}[y_1,y_2]\cdot (I,1-y_1y_2)$. Let $G$ be a Gr\"obner basis
for $\tilde I$ under a monomial order that eliminates $y_1,y_2$. For
each $P\in G\cap D_{n+1}$ 
set $P'=t^{-w(P)}P$ if $w(P)<0$ and $P'=\del_t^{w(P)}P$ if
$w(P)\geq 0$. Set $G_0=\{ P': P\in G\cap D_{n+1}\}$. Then
$G_0\subseteq D_n[-\del_tt]$ generates $J\cap D_n[-\del_tt]$.
\end{proposition}

\begin{proof}
This is in essence Theorem 18 of \cite{DM:Oa2}. (See the remarks 
in Subsection \ref{subsec-GB} on how to compute such Gr\"obner
bases.)\qed 
\end{proof}

As a corollary to this proposition we obtain an algorithm for the
computation of $J^\Delta(f^s)$:

\begin{alg}[Parametric Annihilator]
\mylabel{alg-ann-fs}~

\noindent {\sc Input}: $f\in R_n$; $L\subseteq D_n$ such that $L$ is 
$f$-saturated.

\noindent {\sc Output}: Generators for $J^L(f^s)$\myindex{$J^L(f^s)$,
algorithm for}.

\begin{enumerate}
\item For each generator $Q_i$ of $D_{n+1}\cdot (L,t)$ 
compute the image $\phi(Q_i)$
under $x_i\mapsto
x_i$, $t\mapsto t-f$, $\del_i\mapsto \del_i+\del_i\action(f)\del_t$,
$\del_t\mapsto\del_t$.

\item Homogenize all $\phi(Q_i)$ with respect to the new variable
$y_1$ relative to the weight $w$ introduced before Proposition \ref{prop-oaku}.

\item Compute a Gr\"obner basis for the ideal
\[
D_{n+1}[y_1,y_2]\cdot((\phi(Q_1))^h, \ldots, (\phi(Q_r))^h, 1-y_1y_2)
\]
in $D_{n+1}[y_1,y_2]$
using an order that eliminates $y_1,y_2$.

\item Select the operators $\{ P_j\}_1^b$ in this basis which do not
contain $y_1, y_2$. 

\item For each $P_j$, $1\le j\le b$, if $w(P_j)>0$ replace $P_j$ by
$P_j'=\del_t^{w(P_j)}P_j$. Otherwise replace $P_j$ by
$P_j'=t^{-w(P_j)}P_j$. 

\item Return the new operators $\{P_j'\}_1^b$.
\end{enumerate}
End.
\end{alg}
The output will be operators in $D_n[-\del_tt]$ which is naturally
identified with $D_n[s]$ (including the action on $\M^L_f$).
This algorithm is in effect Proposition 7.1 of \cite{DM:Oa3}.

In \Mtwo, one can compute the parametric annihilator ideal (for
$R_n=\Delta$) by the command {\tt AnnFs}:
\beginOutput
i8 : D = QQ[x,y,z,w,Dx,Dy,Dz,Dw, \\
\            WeylAlgebra => \{x=>Dx, y=>Dy, z=>Dz, w=>Dw\}];\\
\endOutput
\beginOutput
i9 : f = x^2+y^2+z^2+w^2\\
\emptyLine
\      2    2    2    2\\
o9 = x  + y  + z  + w\\
\emptyLine
o9 : D\\
\endOutput
\beginOutput
i10 : AnnFs(f)\\
\emptyLine
\                                                                       $\cdot\cdot\cdot$\\
o10 = ideal (w*Dz - z*Dw, w*Dy - y*Dw, z*Dy - y*Dz, w*Dx - x*Dw, z*Dx  $\cdot\cdot\cdot$\\
\                                                                       $\cdot\cdot\cdot$\\
\emptyLine
o10 : Ideal of QQ [x, y, z, w, Dx, Dy, Dz, Dw, \$s, WeylAlgebra => \{x = $\cdot\cdot\cdot$\\
\endOutput
If we want to compute $J^L(f^s)$ 
for more general $L$, we have to use
the command {\tt AnnIFs}:
\beginOutput
i11 : L=ideal(x,y,Dz,Dw)\\
\emptyLine
o11 = ideal (x, y, Dz, Dw)\\
\emptyLine
o11 : Ideal of D\\
\endOutput
\beginOutput
i12 : AnnIFs(L,f)\\
\emptyLine
\                                1        1\\
o12 = ideal (y, x, w*Dz - z*Dw, -*z*Dz + -*w*Dw - \$s)\\
\                                2        2\\
\emptyLine
o12 : Ideal of QQ [x, y, z, w, Dx, Dy, Dz, Dw, \$s, WeylAlgebra => \{x = $\cdot\cdot\cdot$\\
\endOutput
It should be emphasized that saturatedness of $L$ with respect to $f$ is
a must for {\tt AnnIFs}. 

\subsection{The Bernstein-Sato Polynomial}
Knowing $J^L(f^s)$ allows us to get our hands on the Bernstein-Sato
polynomial of $f$ on $M$:

\begin{corollary}
\mylabel{cor-b-poly}
Suppose $L$ is a holonomic ideal in $D_n$ (i.e., $D_n/L$ is holonomic). 
The
Bernstein polynomial $b_f^L(s)$ of $f$ on $(D_n/L)$ satisfies 
\begin{eqnarray}
(b^L_f(s))=\left(D_n[s]\cdot(J^L(f^s),f)\right)\cap K[s].
\end{eqnarray}
Moreover, if $L$ is $f$-saturated then
$b^L_f(s)$ can be computed with Gr\"obner basis computations.
\end{corollary}

\begin{proof}
By definition of $b^L_f(s)$ we have
$(b_f^L(s)-P_f^L(s)\cdot f)\action (\bar 1\otimes 1\otimes f^s)=0$ for
a suitable $P^L_f(s)\in D_n[s]$. Hence
$b_f^L(s)$ is in $K[s]$ and in 
$D_n[s](J^L(f^s),f)$. Conversely, if $b(s)$ is in this intersection
then $b(s)$ satisfies an equality of the type of (\ref{def-b-poly}) and
hence is a multiple of $b^L_f(s)$.

If we use 
an elimination order for which $\{x_i,\del_i\}_1^n\gg s$ in $D_n[s]$, then
if $J^L(f^s)$ is known, 
$b^L_f(s)$ will be (up to a scalar 
factor) the unique element in the reduced Gr\"obner basis for
$D_n[s]\cdot (J^L(f^s),f)$ that 
contains no $x_i$ nor $\del_i$. Since we assume $L$ to be
$f$-saturated,  $J^L(f^s)$ can be computed
according to Proposition \ref{prop-oaku}. 
\qed
\end{proof}

We therefore arrive at the following algorithm for the Bernstein-Sato
polynomial \cite{DM:Oa}.

\begin{alg}[Bernstein-Sato polynomial]~

\mylabel{alg-b-poly-L}
\noindent {\sc Input}: $f\in R_n$; $ L\subseteq D_n$ such that 
$D_n/L$ is holonomic and
$f$-torsion free. 

\noindent {\sc Output}: The Bernstein polynomial\myindex{Bernstein
polynomial, algorithm for} $b^L_f(s)$.
\begin{enumerate}
\item Determine $J^L(f^s)$ following Algorithm \ref{alg-ann-fs}. 

\item Find a reduced Gr\"obner basis for the ideal
$J^L(f^s)+D_n[s]\cdot f$ 
using an elimination order for $x$ and $\del$. 

\item Pick the unique element $b(s)\in K[s]$ contained in that basis and
return it.
\end{enumerate}
End.
\end{alg}

We illustrate the algorithm with two examples. We first recall $f$
which was defined at the end of the previous subsection.
\beginOutput
i13 : f\\
\emptyLine
\       2    2    2    2\\
o13 = x  + y  + z  + w\\
\emptyLine
o13 : D\\
\endOutput
Now we compute the Bernstein-Sato polynomial. 
\beginOutput
i14 : globalBFunction(f)\\
\emptyLine
\        2\\
o14 = \$s  + 3\$s + 2\\
\emptyLine
o14 : QQ [\$s]\\
\endOutput
The routine {\tt globalBFunction} computes the
Bernstein-Sato polynomial of $f$ on $R_n$. We also take a look at the
Bernstein-Sato polynomial of a cubic:
\beginOutput
i15 : g=x^3+y^3+z^3+w^3\\
\emptyLine
\       3    3    3    3\\
o15 = x  + y  + z  + w\\
\emptyLine
o15 : D\\
\endOutput
\beginOutput
i16 : factorBFunction globalBFunction(g)\\
\emptyLine
\                    7       8               4       5\\
o16 = (\$s + 1)(\$s + -)(\$s + -)(\$s + 2)(\$s + -)(\$s + -)\\
\                    3       3               3       3\\
\emptyLine
o16 : Product\\
\endOutput
In \Mtwo one can also find $b^L_f(s)$ for more general $L$. We will
see in the following remark what the appropriate commands are. 
 
\begin{remark}
\mylabel{rem-nonQ-root}
It is clear that $s+1$ is always a factor of any Bernstein-Sato
polynomial on $R_n$, but this is not necessarily the case if $L\not =
\Delta$. For example, 
$b^L_f(s)=s$ for $n=1$, $f=x$ and $L=x\del_x+1$ (in which case
$D_1/L\cong R_1[x^{-1}]$, generated by ${1}/{x}$). 
In particular, it is not true that
the roots of $b_f^L(s)$ are negative for general holonomic $L$. 

If $L$ is equal to $\Delta$, and if
$f$ is nice, then the Bernstein roots are all between $-n$ and $0$
\cite{DM:Varchenko}.  But for  general $f$ very little is known besides
a famous theorem of Kashiwara
 that states that $b^\Delta_f(s)$ factors over
$\Q$ \cite{DM:K} and all roots are negative.

For $L$ arbitrary, the situation is more complicated.
The Bernstein-Sato polynomial of any polynomial $f$ on the
$D_n$-module generated by $\bar 1\otimes f^a$ with $a\in K$ is
related to that of $f$ on $D_n/L$ by a simple shift, and so the
Bernstein roots of $f$ on the $D_n$-module generated by the function
germ $f^a$, $a\in K$,  
are still all
in $K$  by
\cite{DM:K}. Localizing other modules however can easily lead to
nonrational roots. As an example, consider
\beginOutput
i17 : D1 = QQ[x,Dx,WeylAlgebra => \{x=>Dx\}];\\
\endOutput
\beginOutput
i18 : I1 = ideal((x*Dx)^2+1)\\
\emptyLine
\             2  2\\
o18 = ideal(x Dx  + x*Dx + 1)\\
\emptyLine
o18 : Ideal of D1\\
\endOutput
This is input defined over the rationals.
Even localizing $D_1/I_1$ at a very simple $f$ leads to nonrational roots:
\beginOutput
i19 : f1 = x;\\
\endOutput
\beginOutput
i20 : b=globalB(I1, f1)\\
\emptyLine
\                                   2\\
o20 = HashTable\{Boperator => - x*Dx  + 2Dx*\$s + Dx\}\\
\                                 2\\
\                Bpolynomial => \$s  + 2\$s + 2\\
\emptyLine
o20 : HashTable\\
\endOutput
The routine {\tt globalB} is to be used if a Bernstein-Sato polynomial
is suspected to fail to factor over $\Q$. If $b_f^L(s)$ does factor
over $\Q$, one can also use the routine {\tt DlocalizeAll} to be
discussed below. It
would be very interesting to determine rules that govern the splitting field
of $b^L_f(s)$ in general.


\end{remark}

\subsection{Specializing Exponents}
\mylabel{subsec-special-exp}
In this subsection we investigate the result of substituting $a\in K$
for $s$ in $J^L(f^s)$.
Recall that the Bernstein polynomial $b^L_f(s)$  
will exist (i.e., be nonzero) if $D_n/L$ 
is holonomic. As outlined in the previous subsection, $b^L_f(s)$ can
be computed if $D_n/L$ is holonomic and $f$-torsion free. 
The following proposition 
(Proposition 7.3 in \cite{DM:Oa3}, see also Proposition 6.2 in \cite{DM:K})
 shows that replacing $s$ by an
exponent in very general position 
leads to a solution of the localization problem.

\begin{proposition}
\mylabel{prop-kashiwara}
If $L$ is holonomic and $a\in K$ is such that 
no element of $\{a-1,a-2,\ldots\}$ is a Bernstein
root of $f$ on $L$ then we have $D_n$-isomorphisms 
\begin{equation}
%% \label{eqn-s=a-iso}
(D_n/L)\otimes_{R_n}\left( R_n[f^{-1}]\otimes f^a\right)
\cong \left(D_n[s]/J^L(f^s)\right)|_{s=a}\cong D_n\action
(\bar 1 \otimes 1\otimes f^a).
\end{equation}
\qed
\end{proposition}
One notes in particular that
if any $a\in\Z $ satisfies the conditions of the
proposition, then so does every integer smaller than $a$. This
motivates the following 
\begin{definition}
The {\em stable integral exponent of $f$ on $L$} is the smallest
integral root of $b^L_f(s)$, and denoted $a^L_f$.
\end{definition}
In terms of this definition, 
\[
\left(D_n/J^L(f^s)\right)|_{s=a^L_f}\cong
(D_n/L)\otimes_{R_n}R_n[f^{-1}],
\]
and the presentation corresponds to the generator $\bar 1\otimes
f^{a^L_f}$. 
If $L=\Delta$ then Kashiwara's result tells us that $b^L_f(s)$ will
factor over the rationals, and thus it is very easy to find the stable
integral exponent. If we localize a more general module, 
the roots  may not even be
$K$-rational anymore as we saw at the end of the previous subsection. 

The following lemma deals with the question of finding the smallest
integer root of a polynomial. We let $|s|$ denote the complex absolute
value. 
\begin{lemma}
Suppose that in the situation of Corollary \ref{cor-b-poly}, 
\[
b^L_f(s)=s^d+b_{d-1}s^{d-1}+\dots+b_0,
\]
and define 
$B=\max_{i}\{|b_i|^{1/(d-i)}\}$.
The smallest integer root of $b^L_f(s)$ is an integer between $-2B$
and $2B$. 
If in particular $L=D_n\cdot 
(\del_1,\ldots,\del_n)$, it suffices to check the
integers between $-b_{d-1}$ and $-1$.
\end{lemma}

\begin{proof}
Suppose
$|s_0|=2B\rho$ where $B$ is as defined above and $\rho>1$. 
Assume 
also that $s_0$ is a root of $b_f^L(s)$. We find
\begin{eqnarray*}
(2B\rho)^d=|s_0|^d&=&|-\sum_{i=0}^{d-1}b_i{s_0}^i|
%&\le&\sum_0^{d-1}|b_i|\cdot|s_0|^i \\
\le\sum_{i=0}^{d-1}B^{d-i}|s_0|^i\\
&=&B^d\sum_{i=0}^{d-1}(2\rho)^i
\le B^d((2\rho)^d-1),
\end{eqnarray*}
using $\rho\geq 1$.
By contradiction, $s_0$ is not a root.

The
final claim is a consequence of Kashiwara's work
 \cite{DM:K} where
he proves that if $L=D_n\cdot(\del_1,\ldots,\del_n)$ then  all roots of
$b_f^L(s)$ are rational and negative, and hence 
$-b_{n-1}$ is a 
lower bound for each single root.
\qed
\end{proof}

Combining Proposition \ref{prop-kashiwara} 
with Algorithms 
\ref{alg-ann-fs} 
and  
\ref{alg-b-poly-L} 
we therefore obtain
\begin{alg}[Localization]~

\mylabel{alg-D/L-loc-f}
\myindex{localization!algorithm for}
\noindent {\sc Input}: 
 $f\in R_n$; $L\subseteq D_n$ such that $D_n/L$ is holonomic and
$f$-torsion free. 

\noindent {\sc Output}: 
 Generators for an ideal $J$ such that $(D_n/L)\otimes_{R_n}
 R_n[f^{-1}]\cong D_n/J$.

\begin{enumerate}
\item Determine $J^L(f^s)$ following Algorithm \ref{alg-ann-fs}. 
\item Find the Bernstein polynomial $b_f^L(s)$ using Algorithm
\ref{alg-b-poly-L}. 
\item Find the smallest integer root $a$ of $b_f^L(s)$.
\item Replace $s$ by $a$ in all generators for $J^L(f^s)$ and
return these generators.
\end{enumerate}
End.
\end{alg}
Algorithms \ref{alg-b-poly-L} and \ref{alg-D/L-loc-f} are 
Theorems 6.14 and Proposition 7.3 in \cite{DM:Oa3}.

\begin{example}
For $f=x^2+y^2+z^2+w^2$, we found a stable integral exponent of $-2$
in the previous subsection.
To compute the annihilator of 
$f^{-2}$ using \Mtwo, we use the command {\tt
Dlocalize} which automatically uses the stable integral exponent.
We first change the current ring back to the ring {\tt D} which we used 
in the previous subsection:
\beginOutput
i21 : use D\\
\emptyLine
o21 = D\\
\emptyLine
o21 : PolynomialRing\\
\endOutput
Here is the
module to be localized.
\beginOutput
i22 : R = (D^1/ideal(Dx,Dy,Dz,Dw))\\
\emptyLine
o22 = cokernel | Dx Dy Dz Dw |\\
\emptyLine
\                             1\\
o22 : D-module, quotient of D\\
\endOutput
The localization then is obtained by running
\beginOutput
i23 : ann2 = relations Dlocalize(R,f)\\
\emptyLine
o23 = | wDz-zDw wDy-yDw zDy-yDz wDx-xDw zDx-xDz yDx-xDy xDx+yDy+zDz+wD $\cdot\cdot\cdot$\\
\emptyLine
\              1       10\\
o23 : Matrix D  <--- D\\
\endOutput
The output {\tt ann2} is a $1\times 10$ matrix whose entries generate
$\ann_{D_4}(f^{-2})$. 
\end{example}

\begin{remark}
The computation of the annihilator 
of $f^a$ for values of $a$ such that $a-k$ is a
Bernstein root for some $k\in\N^+$ can be achieved by an appropriate
syzygy computation. For example, we saw above
that the Bernstein-Sato
polynomial of $f=x^2+y^2+z^2+w^2$ on $R_4$ is $(s+1)(s+2)$. So
evaluation of $J^L(f^s)$ at $-1$ does not necessarily yield
$\ann_{D_4}(f^{-1})$, as will be documented in the next remark. 
On the other hand, evaluation at $-2$ gives
$\ann_{D_4}(f^{-2})$. It is not hard to see that
$\ann_{D_4}(f^{-1})=\{P\in D_n:Pf\in \ann_{D_n}(f^{-2})\}$ because
$D_4\action f^{-1}=D_4f\action f^{-2}
\subseteq D_4\action f^{-2}$. So we set:
\beginOutput
i24 : F = matrix\{\{f\}\}\\
\emptyLine
o24 = | x2+y2+z2+w2 |\\
\emptyLine
\              1       1\\
o24 : Matrix D  <--- D\\
\endOutput
To find $\ann_{D_4}(f^{-1})$, we use the command {\tt modulo} which
computes relations: {\tt modulo(M,N)} computes for two matrices $M, N$
the set of (vectors of) operators $P$ such that $P\cdot M\subseteq \im(N)$. 
\beginOutput
i25 : ann1 = gb modulo(F,ann2)\\
\emptyLine
o25 = \{2\} | wDz-zDw wDy-yDw zDy-yDz Dx^2+Dy^2+Dz^2+Dw^2 wDx-xDw zDx-xD $\cdot\cdot\cdot$\\
\emptyLine
o25 : GroebnerBasis\\
\endOutput
The generator $\del_2^2+\del_y^2+\del_z^2+\del_w^2$ is particularly interesting. 
To see the quotient of
$D_4\action f^{-2}$ by $D_4\action f^{-1}$ we execute
\beginOutput
i26 : gb((ideal ann2) + (ideal F))\\
\emptyLine
o26 = | w z y x |\\
\emptyLine
o26 : GroebnerBasis\\
\endOutput
which shows that $D_4\action f^{-2}$ is an extension of
$D_4/D_4(x,y,z,w)$ by $D_4\action f^{-1}$. This is not surprising,
since $(0,0,0,0)$ is the only singularity of $f$ and hence the
difference between $D_4\action f^{-2}$ and $D_4\action f^{-1}$ must be
supported at the origin.

It is perhaps interesting to note that for a more complicated (but still
irreducible) polynomial $f$ the
quotient $({D_n\action f^{a}})/({D_n\action f^{a+1}})$ can
be a nonsimple nonzero $D_n$-module. For example, let
$f=x^3+y^3+z^3+w^3$ and $a=a^\Delta_f=-2$. 
A computation similar to the quadric case above
shows that here $({D_n\action f^{a}})/({D_n\action f^{a+1}})$ is a
$(x,y,z,w)$-torsion module (supported at the singular locus of $f$)
isomorphic to $(D_4/D_4\cdot (x,y,z,w))^6$. The socle elements of the
quotient are the degree 2 polynomials in $x,y,z,w$.
\end{remark}

\begin{example}
\mylabel{ex-cubic}
Here we show how with \Mtwo one can get more information
from the localization
procedure. 
\beginOutput
i27 : D = QQ[x,y,z,Dx,Dy,Dz, WeylAlgebra => \{x=>Dx, y=>Dy, z=>Dz\}];\\
\endOutput
\beginOutput
i28 : Delta = ideal(Dx,Dy,Dz);\\
\emptyLine
o28 : Ideal of D\\
\endOutput
We now define a polynomial and compute the localization of $R_3$
at the
polynomial.
\beginOutput
i29 : f=x^3+y^3+z^3;\\
\endOutput
\beginOutput
i30 : I1=DlocalizeAll(D^1/Delta,f,Strategy=>Oaku)\\
\emptyLine
\                                1        1        1             2      $\cdot\cdot\cdot$\\
o30 = HashTable\{annFS => ideal (-*x*Dx + -*y*Dy + -*z*Dz - \$s, z Dy -  $\cdot\cdot\cdot$\\
\                                3        3        3\\
\                                     2      5       4\\
\                Bfunction => (\$s + 1) (\$s + -)(\$s + -)(\$s + 2)\\
\                                            3       3\\
\                              2       3         2       4      1   2   $\cdot\cdot\cdot$\\
\                Boperator => --*y*z*Dx Dy*Dz - --*y*z*Dy Dz + ---*z Dx $\cdot\cdot\cdot$\\
\                             81                81             243      $\cdot\cdot\cdot$\\
\                GeneratorPower => -2\\
\                LocMap => | x6+2x3y3+y6+2x3z3+2y3z3+z6 |\\
\                LocModule => cokernel | 1/3xDx+1/3yDy+1/3zDz+2 z2Dy-y2 $\cdot\cdot\cdot$\\
\emptyLine
o30 : HashTable\\
\endOutput
\beginOutput
i31 : I2=DlocalizeAll(D^1/Delta,f)\\
\emptyLine
o31 = HashTable\{GeneratorPower => -2                                   $\cdot\cdot\cdot$\\
\                                          2        2      2       1\\
\                IntegrateBfunction => (\$s) (\$s + 1) (\$s + -)(\$s + -)\\
\                                                          3       3\\
\                LocMap => | x6+2x3y3+y6+2x3z3+2y3z3+z6 |\\
\                LocModule => cokernel | xDx+yDy+zDz+6 z2Dy-y2Dz z2Dx-x $\cdot\cdot\cdot$\\
\emptyLine
o31 : HashTable\\
\endOutput

%\begin{verbatim}
%i1 : load "Dloadfile.m2"
%i2 : D = QQ[x,y,z,Dx,Dy,Dz, WeylAlgebra => {x=>Dx, y=>Dy, z=>Dz}]
%i3 : Delta=ideal(Dx,Dy,Dz)
%i4 : f=x^3+y^3+z^3
%i5 : I1=DlocalizeAll(D^1/Delta,f,Strategy=>Oaku)
%i6 : I2=DlocalizeAll(D^1/Delta,f)
%\end{verbatim}
The last two commands  both compute the localization of
$R_3$ 
at $f$ but follow different localization algorithms. The
former uses our Algorithm \ref{alg-D/L-loc-f} while the latter follows
\cite{DM:O-T-W}. 

The output of the command {\tt DlocalizeAll} is a hashtable, because
it contains a variety of data that pertain to the map $R_n\into
R_n[f^{-1}]$. {\tt LocMap} gives the element that induces the map on
the $D_n$-module level (by right multiplication). {\tt LocModule} gives
the localized module as cokernel of the displayed matrix. {\tt
Bfunction} is the Bernstein-Sato polynomial and {\tt annFS} the
generic annihilator $J^L(f^s)$. {\tt Boperator} displays a
Bernstein operator and the stable integral exponent is stored in {\tt
GeneratorPower}. 

  %% The warning {\tt Oaku's \ldots saturated} refers
  %% to the fact that 
Algorithm \ref{alg-D/L-loc-f} requires the ideal $L$ to be
$f$-saturated. This property is not checked by \Mtwo, so the user
needs to make sure it holds. For example, this is always the case if
$D_n/L$ is a localization of $R_n$. One can check the saturation
property in \Mtwo, but it is a rather involved computation. This difficulty can
be circumvented by omitting the option {\tt Strategy=>Oaku}, in
which case the localization algorithm of \cite{DM:O-T-W} is used. In
terms of complexity, using the Oaku strategy is much better behaved.

One can address the entries of a hashtable.  For
example, executing
\beginOutput
i32 : I1.LocModule\\
\emptyLine
o32 = cokernel | 1/3xDx+1/3yDy+1/3zDz+2 z2Dy-y2Dz z2Dx-x2Dz y2Dx-x2Dy |\\
\emptyLine
\                             1\\
o32 : D-module, quotient of D\\
\endOutput
one can see that 
$R_3[f^{-1}]$ is isomorphic to the cokernel of the {\tt LocModule}
entry which (for either localization method) is
\begin{eqnarray*}
D_3/&D_3\cdot(&x\del_x+y\del_y+z\del_z+6,\,\,\,z^2\del_y-y^2\del_z,\,\,\,x^3\del_y+y^3\del_y+y^2z\del_z+6y^2,\\ 
         &&      z^2\del_x-x^2\del_z,\,\,\,y^2\del_x-x^2\del_y,\,\,\, 
               x^3\del_z+y^3\del_z+z^3\del_z+6z^2).
\end{eqnarray*}
The first line of the hashtable {\tt I1} shows 
that $R_3[f^{-1}]$ is generated
by $f^{-2}$ over $D_3$, while {\tt I1.LocMap} shows that the natural inclusion
$D_3/\Delta=R_3\into R_3[f^{-1}]=D_3/J^\Delta(f^s)|_{s=a^\Delta_f}$ 
is given by right multiplication by $f^2$, shown as the third entry
of the hashtable {\tt I1}.
%The Bernstein polynomial {\tt I1.Bfunction} can be computed by
%<<factorBFunction(globalBFunction(f))>>
It is perhaps useful to point out that the fourth entry of hashtable
{\tt I2} is a relative of the Bernstein-Sato polynomial of $f$, and is
used for the computation of the so-called restriction functor (compare
with \cite{DM:O-T1,DM:W2}).
\end{example}

\begin{remark}
Plugging in bad values $a$ for $s$ (such that $a-k$ {\em is} a Bernstein
root for some $k\in \N^+$) can have unexpected results. Consider the
case $n=1$, $f=x$. Then $J^\Delta(f^s)=D_1\cdot(s+1-\del_1x_1)$. Hence
$b^\Delta_f(s)=s+1$ and $-1$ is the unique Bernstein root. According
to Proposition \ref{prop-kashiwara}, 
\[
\left(D_1[s]/J^\Delta(f^s)\right)|_{s=a}
\cong R_1[{x_1}^{-1}]\otimes {x_1}^a\cong D_1\action {x_1}^a
\]
for all $a\in K\setminus\N$. For $a\in\N^+$, we also have
$D_1[s]/J^\Delta(f^s)|_{s=a}\cong D_1\action x^a$, but this is of
course not $R_1[{x_1}^{-1}]$ but just $R_1$. 

For $a=0$ however, $\left(D_1[s]/J^\Delta(f^s)\right)|_{s=a}$ 
has $x_1$-torsion! It
equals in fact what is called the Fourier transform of
$R_1[{x_1}^{-1}]$ and fits into an exact sequence 
\[
0\to H^1_{x_1}(R_1)\to{\mathcal F}(R_1[{x_1}^{-1}])\to R_1\to 0.
\]
\end{remark}



\begin{remark}
If $D_n/L$ is holonomic but has $f$-torsion, then 
$(D_n/L)\otimes R_n[f^{-1}]$ 
and $((D_n/L)/H^0_{(f)}(D_n/L))\otimes R_n[f^{-1}]$ are of course
isomorphic.  So if we
knew how to find $M/H^0_f(M)$ for holonomic modules $M$, our localization
algorithm could be generalized to all holonomic modules. There are two
different approaches to the problem of $f$-torsion, presented in
\cite{DM:O-T1} and in \cite{DM:Ts,DM:Ts0}. The former is based on homological
methods and restriction to the diagonal 
while the latter aims at direct computation of
those $P\in D_n$ for which $f^k P\in L$ for some $k$. 

There is also another direct method for localizing $M=D_n/L$ at $f$
that works in the situation where the nonholonomic locus of $M$ is
contained in the variety of $f$ (irrespective of torsion). 
It was proved by Kashiwara, that
$M[f^{-1}]$ is then holonomic, and in \cite{DM:O-T-W} an algorithm based
on integration is
given that computes a presentation for it.
\end{remark}

%\input{5.tex}
\section{Local Cohomology Computations}
\mylabel{sec-lc}
The purpose of this section is to present algorithms
that compute 
for given $i,j,k\in \N, I\subseteq  R_n$ 
the structure of the
local cohomology modules $H^k_I(R_n)$ and  $H^i_\m(H^j_I(R_n))$, and the
invariants $\lambda_{i,j}(R_n/I)$ associated to $I$.
%
In particular, the algorithms 
detect the vanishing of local
cohomology modules. 
\subsection{Local Cohomology}
\mylabel{subsec-lc}
We will first describe an algorithm that takes a finite
set of polynomials $\{f_1,\ldots,f_r\}\subset R_n$ and
returns a 
presentation of $H^k_I(R_n)$ where $I=R_n\cdot(f_1,\ldots,f_r)$. In particular,
if $H^k_I(R_n)$ is zero, then the algorithm will return the zero
presentation. 

\begin{definition}
Let $\Theta^r_k$\myindex{$\Theta^r_k$}
be the set of $k$-element subsets of $\{1,\ldots,r\}$ and
for $\theta\in \Theta^r_k$ write $F_\theta$ for the product $\prod_{i\in
\theta}f_{i}$.
\end{definition}
Consider the  \v Cech complex $\check C^\bullet=\check
C^\bullet(f_1,\ldots,f_r)$ associated to $f_1,\ldots,f_r$ in
$R_n$, 
\begin{equation}
%% \label{cechcomplex}
0\to R_n\to \bigoplus_{\theta\in\Theta^r_1} R_n[{F_\theta}^{-1}]\to 
 \bigoplus_{\theta\in\Theta^r_2}R_n[{F_\theta}^{-1}]
 \to\cdots\to R_n[{(f_1\cdots f_r)}^{-1}]\to 0.
\end{equation}
Its $k$-th cohomology group is 
$H^k_I(R_n)$.
The map 
\begin{equation}
\label{cechmap}
M_k:\left(\check C^k=\bigoplus\limits_{\theta\in\Theta^r_k}
   R_n[{F_\theta}^{-1}]\right)\to 
\left(\bigoplus\limits_{\theta'\in\Theta^r_{k+1}}
   R_n[{F_{\theta'}}^{-1}]=\check C^{k+1} \right)
\end{equation}
 is the sum of maps
\begin{equation}
\label{cechmap-parts}
R_n[{(f_{i_1}\cdots f_{i_k})}^{-1}]\to R_n[{(f_{j_1}\dotsb
f_{j_{k+1}})}^{-1}]
\end{equation}
which are zero if $\{i_1,\ldots,i_k\}\not\subseteq
\{j_1,\ldots,j_{k+1}\}$, or send $\frac{1}{1}$ to
$\frac{1}{1}$ (up to sign). 
With $D_n/\Delta\cong R_n$, identify
$R_n[{(f_{i_1}\cdots f_{i_k})}^{-1}]$ with
$D_n/J^\Delta((f_{i_1}\cdots f_{i_k})^s)|_{s=a}$ and
$R_n[{(f_{j_1}\cdots f_{j_{k+1}})}^{-1}]$ with
$D_n/J^\Delta((f_{j_1}\cdots f_{j_{k+1}})^s)|_{s=a'}$ where
$a,a'$ are sufficiently small integers. By
Proposition \ref{prop-kashiwara} we may assume that $a=a'\le 0$. Then the map
(\ref{cechmap-parts}) is in the nonzero case multiplication from the right by
$(f_l)^{-a}$ where $l=\{j_1,\ldots,j_{k+1}\}\backslash \{i_1,\ldots,i_k\}$,
again up to sign. For example, consider the inclusion 
\[
D_2/D_2\cdot(\del_xx,\del_y)=R_2[x^{-1}]\into
R_2[(xy)^{-1}]=D_2/D_2\cdot (\del_xx,
\del_yy).
\]
Since $\frac{1}{x}
=\frac{y}{xy}$, the inclusion on the level of $D_2$-modules 
maps $P+\ann(x^{-1})$ to $Py+\ann((xy)^{-1})$.  

It follows that the matrix representing the map $\check C^k\to \check 
C^{k+1}$ in
terms of $D_n$-modules is very easy to write down once the annihilator
ideals and Bernstein polynomials for all $k$- and $(k+1)$-fold products
of the $f_i$ are known: the entries are 0 or $\pm f_l^{-a}$ where
$f_l$ is the new factor. These considerations give the following 

\begin{alg}[Local cohomology\index{local cohomology!algorithm for}]~

\mylabel{alg-lc}

\noindent {\sc Input}: $f_1,\ldots,f_r\in R_n; k\in \N$.

\noindent {\sc Output}: $H_I^k(R_n)$ in terms of generators and relations as finitely
generated $D_n$-module where $I=R_n\cdot(f_1,\ldots,f_r)$.
\begin{enumerate}
\item Compute the annihilator ideal $J^\Delta((F_\theta)^s)$
and the Bernstein 
polynomial $b^\Delta_{F_\theta}(s)$ for all $(k-1)$-, $k$- and $(k+1)$-fold
products $F_\theta$ of 
${f_1},\ldots,{f_r}$ following Algorithms \ref{alg-ann-fs} and \ref{alg-b-poly-L} (so
$\theta$ runs through $\Theta^r_{k-1}\cup \Theta^r_k\cup \Theta^r_{k+1}$).

\item Compute the stable integral exponents $a^\Delta_{F_\theta}$, 
 let $a$
be their minimum
and replace $s$ by $a$ in all the annihilator ideals.

\item Compute the two matrices $M_{k-1},M_k$ representing the
$D_n$-linear maps 
$\check C^{k-1}\to \check C^k$ and $\check C^k\to \check 
C^{k+1}$ as explained above.

\item Compute a Gr\"obner basis $G$ for the kernel of the composition
\[
\bigoplus_{\theta\in\Theta_k^r}D_n\onto \bigoplus_{\theta\in\Theta^r_k} 
D_n/J^\Delta({F_\theta}^s)|_{s=a}\stackrel{M_k}{\longrightarrow}
\bigoplus_{\theta'\in
\Theta^r_{k+1}}D_n/J^\Delta({F_{\theta'}}^s)|_{s=a}. 
\]

\item Compute a Gr\"obner basis $G_0$ for the preimage 
in $\bigoplus_{\theta\in\Theta_k^r}D_n$ of the module 
\[
\im(M_{k-1})
\subseteq \bigoplus_{\theta\in\Theta^r_k} 
D_n/J^\Delta((F_\theta)^s)|_{s=a}\leftarrow\hskip-1.7ex\leftarrow \bigoplus_{\theta\in\Theta_k^r}D_n
\]
under the indicated projection.
\item Compute the remainders of all elements of $G$ with
respect to $G_0$. 

\item Return these remainders and $G_0$.
\end{enumerate}
End.
\end{alg}
The nonzero elements of $G$ generate the quotient $G/G_0\cong
H^k_I(R_n)$ so that in particular
$H^k_I(R_n)=0$ if and only if all returned remainders are zero. 

\begin{example}
\mylabel{ex-minors}
Let $I$ be the ideal in $R_6=K[x,y,z,u,v,w]$ that is generated by the
$2\times 2$ minors $f,g,h$ of the matrix
$\left(\begin{array}{ccc}x&y&z\\u&v&w\end{array}\right)$.
Then $H_I^i(R_6)=0$ for $i<2$ and 
$H^2_I(R_6)\ne 0$ because $I$ is a height 2 prime,
 and $H^i_I(R_6)=0$ for $i>3$
because $I$ is  
3-generated, so the only open case is $H^3_I(R_6)$. This module
 in
fact does 
not vanish, and our algorithm provides a proof of this 
fact by direct calculation. The \Mtwo commands are as follows.
\beginOutput
i33 : D= QQ[x,y,z,u,v,w,Dx,Dy,Dz,Du,Dv,Dw, WeylAlgebra =>\\
\                \{x=>Dx, y=>Dy, z=>Dz, u=>Du, v=>Dv, w=>Dw\}];\\
\endOutput
\beginOutput
i34 : Delta=ideal(Dx,Dy,Dz,Du,Dv,Dw);\\
\emptyLine
o34 : Ideal of D\\
\endOutput
\beginOutput
i35 : R=D^1/Delta;\\
\endOutput
\beginOutput
i36 : f=x*v-u*y;\\
\endOutput
\beginOutput
i37 : g=x*w-u*z;\\
\endOutput
\beginOutput
i38 : h=y*w-v*z;\\
\endOutput
These commands define the relevant rings and polynomials. The
following three
compute the localization of $R_6$ at $f$: 
\beginOutput
i39 : Rf=DlocalizeAll(R,f,Strategy => Oaku)\\
\emptyLine
o39 = HashTable\{annFS => ideal (Dw, Dz, x*Du + y*Dv, y*Dy - u*Du, x*Dy $\cdot\cdot\cdot$\\
\                Bfunction => (\$s + 1)(\$s + 2)\\
\                Boperator => - Dy*Du + Dx*Dv\\
\                GeneratorPower => -2\\
\                LocMap => | y2u2-2xyuv+x2v2 |\\
\                LocModule => cokernel | Dw Dz xDu+yDv yDy-uDu xDy+uDv  $\cdot\cdot\cdot$\\
\emptyLine
o39 : HashTable\\
\endOutput
of $R_6[f^{-1}]$ at
$g$:
\beginOutput
i40 : Rfg=DlocalizeAll(Rf.LocModule,g, Strategy => Oaku)\\
\emptyLine
\                                                                       $\cdot\cdot\cdot$\\
o40 = HashTable\{annFS => ideal (Dz*Dv - Dy*Dw, x*Du + y*Dv + z*Dw, z*D $\cdot\cdot\cdot$\\
\                Bfunction => (\$s + 1)(\$s)\\
\                Boperator => - Dz*Du + Dx*Dw\\
\                GeneratorPower => -1\\
\                LocMap => | -zu+xw |\\
\                LocModule => cokernel | DzDv-DyDw xDu+yDv+zDw zDz-uDu- $\cdot\cdot\cdot$\\
\emptyLine
o40 : HashTable\\
\endOutput
and of $R_6[(fg)^{-1}]$ at $h$:   
\beginOutput
i41 : Rfgh=DlocalizeAll(Rfg.LocModule,h, Strategy => Oaku)\\
\emptyLine
\                                                                       $\cdot\cdot\cdot$\\
o41 = HashTable\{annFS => ideal (x*Du + y*Dv + z*Dw, z*Dz - u*Du - v*Dv $\cdot\cdot\cdot$\\
\                Bfunction => (\$s - 1)(\$s + 1)\\
\                Boperator => - Dz*Dv + Dy*Dw\\
\                GeneratorPower => -1\\
\                LocMap => | -zv+yw |\\
\                LocModule => cokernel | xDu+yDv+zDw zDz-uDu-vDv-2 yDy- $\cdot\cdot\cdot$\\
\emptyLine
o41 : HashTable\\
\endOutput
From the output of these commands
one sees that $R_6[(fgh)^{-1}]$ is generated by
${1}/{f^2gh}$. This follows from considering the stable integral exponents
of the three localization procedures, 
encoded in the hashtable entry stored under the key {\tt GeneratorPower}:
for example, 
\beginOutput
i42 : Rf.GeneratorPower\\
\emptyLine
o42 = -2\\
\endOutput
shows that the generator for $R_6[f^{-1}]$ is $f^{-2}$.
Now we compute the annihilator of $H^3_I(R_6)$.
From the \v Cech complex it follows that 
 $H^3_I(R_6)$ is the quotient of the output of {\tt Rfgh.LocModule}
 (isomorphic to $R_6[(fgh)^{-1}]$)
by the submodules generated by $f^2$, $g$ and $h$. (These submodules
 represent $R_6[(gh)^{-1}]$, $R_6[(fh)^{-1}]$ and $R_6[(fg)^{-1}]$
 respectively.)  
\beginOutput
i43 : Jfgh=ideal relations Rfgh.LocModule;\\
\emptyLine
o43 : Ideal of D\\
\endOutput
\beginOutput
i44 : JH3=Jfgh+ideal(f^2,g,h);\\
\emptyLine
o44 : Ideal of D\\
\endOutput
\beginOutput
i45 : JH3gb=gb JH3\\
\emptyLine
o45 = | w z uDu+vDv+wDw+4 xDu+yDv+zDw yDy-uDu-wDw-1 xDy+uDv uDx+vDy+wD $\cdot\cdot\cdot$\\
\emptyLine
o45 : GroebnerBasis\\
\endOutput
So {\tt JH3} is the ideal of $D_3$ generated by
\begin{eqnarray*}
&w,\,\,\, z,\,\,\, u\del_u+v\del_v+w\del_w+4,\,\,\,
x\del_u+y\del_v+z\del_w,\,\, \, 
y\del_y-u\del_u-w\del_w-1,&\\ 
&x\del_y+u\del_v,\,\,\, 
            u\del_x+v\del_y+w\del_z,\,\,\, y\del_x+v\del_u,\,\,\,
x\del_x-v\del_v-w\del_w-1,&\\ & v^2,\,\,\, uv,\,\, yv,\,\,\,
u^2,\,\,\, yu+xv,\,\, \, 
            xu,\,\,\, y^2,\,\, \,xy,&\\ &x^2,\,\,\, xv\del_v+2x,\,\,\,
v\del_y\del_u+w\del_z\del_u-v\del_x\del_v-w\del_x\del_w-3\del_x&
\end{eqnarray*}
which form a Gr\"obner basis.
This proves that $H^3_I(R)\not =0$, because $1$ is not in the 
Gr\"obner basis of {\tt JH3}.
(There are also algebraic and topological proofs to this account. 
Due to Hochster, and Bruns and
Schw\"anzl, they are quite 
ingenious and work only in rather special situations.)


From our output one can see that $H^3_I(R_6)$ is
$(x,y,z,u,v,w)$-torsion as {\tt JH3} contains $(x,y,z,u,v,w)^2$.
The following sequence of commands defines a procedure {\tt testmTorsion}
which as the name suggests tests a module $D_n/L$ for being $\m$-torsion.
We first replace the generators of $L$ with a Gr\"obner basis.
Then we pick the elements of the Gr\"obner basis not using any $\del_i$.
If now the left over polynomials define an ideal of dimension $0$ in
$R_n$, the ideal was $\m$-torsion and otherwise not.
\beginOutput
i46 : testmTorsion = method();\\
\endOutput
\beginOutput
i47 : testmTorsion Ideal := (L) -> (\\
\           LL = ideal generators gb L;\\
\           n = numgens (ring (LL)) // 2;\\
\           LLL = ideal select(first entries gens LL, f->(\\
\                     l = apply(listForm f, t->drop(t#0,n));\\
\                     all(l, t->t==toList(n:0))       \\
\                     ));\\
\           if dim inw(LLL,toList(apply(1..2*n,t -> 1))) == n\\
\           then true\\
\           else false);\\
\endOutput
If we apply {\tt testmTorsion} to {\tt JH3} we obtain
\beginOutput
i48 : testmTorsion(JH3)\\
\emptyLine
o48 = true\\
\endOutput
Further inspection shows that the ideal
{\tt JH3} is in fact the annihilator of the fraction
${f}/({wzx^2y^2u^2v^2})$ in
$R_6[(xyzuvw)^{-1}]/R_6\cong 
D_6/D_6\cdot(x,y,z,u,v,w)$, and that the fraction
generates $D_6/D_6\cdot(x_1,\dots,x_6)$.
Since $D_6/D_6\cdot(x_1,\dots,x_6)$ is
isomorphic to   
$E_{R_6}(R_6/R_6\cdot(x_1,\dots,x_6))$, the injective hull of 
$R_6/R_6\cdot(x_1,\dots,x_6)=K$ in
the category  
of $R_6$-modules, we conclude that  
$H^3_I(R_6)\cong E_{R_6}(K)$. (In the next subsection we will display
a way to use \Mtwo to find the length of an $\m$-torsion module.)



In contrast, let $I$ be defined as generated by the three
minors, but this time over  a field of finite characteristic. Then 
 $H^3_I(R_6)$ is
zero because Peskine and Szpiro proved using the Frobenius functor 
\cite{DM:P-S} that $R_6/I$
Cohen-Macaulay implies that $H^k_I(R_6)$ is nonzero only if $k=\codim(I)$.

Also opposite to the above example, but in any characteristic, is the
following calculation.  Let $I$ be the ideal in $K[x,y,z,w]$
describing the twisted cubic: $I=R_4\cdot (f,g,h)$ with $f=xz-y^2$, 
$g=yw-z^2$,
$h=xw-yz$. 
The projective variety $V_2$ defined by $I$ is isomorphic to the
projective
line. It is of interest to determine whether $V_2$ and other Veronese
embeddings of the projective line are complete
intersections. The set-theoretic complete intersection property can
occasionally be ruled out with local cohomology techniques: if $V$ is
of codimension $c$ in the affine variety $X$ and
$H^{c+k}_{I(V)}(O(X))\not =0$ for any positive $k$ then $V$ cannot be
a set-theoretic complete intersection. 
In the case of the twisted cubic, it turns out hat $H^3_I(R_4)=0$ as
can be seen from the following computation:
\beginOutput
i49 : D=QQ[x,y,z,w,Dx,Dy,Dz,Dw,WeylAlgebra => \{x=>Dx, y=>Dy, z=>Dz,\\
\      w=>Dw\}];\\
\endOutput
\beginOutput
i50 : f=y^2-x*z;\\
\endOutput
\beginOutput
i51 : g=z^2-y*w;\\
\endOutput
\beginOutput
i52 : h=x*w-y*z;\\
\endOutput
\beginOutput
i53 : Delta=ideal(Dx,Dy,Dz,Dw);\\
\emptyLine
o53 : Ideal of D\\
\endOutput
\beginOutput
i54 : R=D^1/Delta;\\
\endOutput
\beginOutput
i55 : Rf=DlocalizeAll(R,f,Strategy => Oaku)  \\
\emptyLine
\                                                         1             $\cdot\cdot\cdot$\\
o55 = HashTable\{annFS => ideal (Dw, x*Dy + 2y*Dz, y*Dx + -*z*Dy, x*Dx  $\cdot\cdot\cdot$\\
\                                                         2             $\cdot\cdot\cdot$\\
\                                   3\\
\                Bfunction => (\$s + -)(\$s + 1)\\
\                                   2\\
\                             1   2\\
\                Boperator => -*Dy  - Dx*Dz\\
\                             4\\
\                GeneratorPower => -1\\
\                LocMap => | y2-xz |\\
\                LocModule => cokernel | Dw xDy+2yDz yDx+1/2zDy xDx-zDz $\cdot\cdot\cdot$\\
\emptyLine
o55 : HashTable\\
\endOutput
        
\beginOutput
i56 : Rfg=DlocalizeAll(Rf.LocModule,g, Strategy => Oaku);\\
\endOutput
          
\beginOutput
i57 : Rfgh=DlocalizeAll(Rfg.LocModule,h, Strategy => Oaku);\\
\endOutput
          
\beginOutput
i58 : Ifgh=ideal relations Rfgh.LocModule;\\
\emptyLine
o58 : Ideal of D\\
\endOutput
\beginOutput
i59 : IH3=Ifgh+ideal(f,g,h);\\
\emptyLine
o59 : Ideal of D\\
\endOutput
\beginOutput
i60 : IH3gb=gb IH3\\
\emptyLine
o60 = | 1 |\\
\emptyLine
o60 : GroebnerBasis\\
\endOutput
It follows that we cannot conclude from local cohomological
considerations that $V_2$ is not a set-theoretic complete
intersection. This is not an accident but typical, as the second
vanishing theorem 
of Hartshorne, Speiser, Huneke and Lyubeznik shows
\cite{DM:CDAV,DM:H-Sp,DM:Hu-L}: if a homogeneous ideal $I\subseteq R_n$
describes an geometrically connected projective variety of positive
dimension then $H^{n-1}_I(R_n)=H^{n}_I (R_n)=0$.
\end{example}


\subsection{Iterated Local Cohomology}
\mylabel{subsec-lclc}

Recall that $\m=R_n\cdot(x_1,\ldots,x_n)$. 
As a second application of Gr\"ob\-ner basis computations over the
Weyl algebra we
show now how to compute the $\m$-torsion modules $H^i_\m (H^j_I(R_n))$.
Note that we cannot apply Lemma \ref{lem-malgrange} to $D_n/L=H^j_I(R_n)$
since $H^j_I(R_n)$ may well 
contain some torsion.

$\check C^j(R_n;f_1,\ldots,f_r)$ denotes the $j$-th
module in the 
\v Cech complex to $R_n$ and $\{f_1,\ldots,f_r\}$. 
Let $\check C^{\bullet,\bullet}$ be the double complex 

\[
\check C^{i,j}=\check C^i(R_n;x_1,\ldots,x_n)\otimes_{R_n} \check
 C^j(R_n;f_1,\ldots,f_r), 
\] 
with
vertical maps $\phi^{\bullet,\bullet}$ induced by the identity on the
first factor and the 
usual \v Cech maps on the second, and  horizontal maps
$\xi^{\bullet,\bullet} $  induced
by the \v Cech maps on the first factor and the identity on the
second. 
Now $\check C^{i,j}$ is a direct sum of modules $R_n[g^{-1}]$ where 
$g=x_{\alpha_1}\cdots x_{\alpha_i}\cdot
f_{\beta_1}\cdots f_{\beta_j}$. So the whole double complex
can be rewritten in terms of $D_n$-modules and $D_n$-linear maps using
Algorithm \ref{alg-D/L-loc-f}: 
\[
\diagram
{\,\check C^{i-1,j+1}\,}{\rto^{\,\,\xi^{i-1,j+1}}}&
        \check C^{i,j+1}\rto^{\xi^{i,j+1}}&
                \check C^{i+1,j+1}\\
\check C^{i-1,j}\rto^{\xi^{i-1,j}}\uto_{\phi^{i-1,j}}& 
        \check C^{i,j}\rto^{\xi^{i,j}}\uto_{\phi^{i,j}}& 
                \check C^{i+1,j}\uto_{\phi^{i+1,j}}\\
\check C^{i-1,j-1}\rto^{\xi^{i-1,j-1}}\uto_{\phi^{i-1,j-1}}&
        \check C^{i,j-1}\rto^{\xi^{i,j-1}}\uto_{\phi^{i,j-1}}& 
                \check C^{i+1,j-1}\uto_{\phi^{i+1,j-1}}
\enddiagram
\]
Since $\check C^i(R_n;x_1,\ldots,x_n)$ is $R_n$-flat, the column
co\-ho\-mo\-lo\-gy of $\check C^{\bullet,\bullet}$ at $(i,j)$ is
$\check C^i(R_n;x_1,\ldots,x_n)\otimes_{R_n}H^j_I(R_n)$ and the induced horizontal maps
in the $j$-th row are
simply the maps in the \v Cech complex $\check C^\bullet(H^j_I(R_n);x_1,\ldots,x_n)$. 
It follows that
the row cohomology of the column cohomology at $(i_0,j_0)$ is
$H^{i_0}_\m(H^{j_0}_I(R_n))$, the object of our interest.

We have, denoting by $X_{\theta'}$
in analogy to $F_\theta$ the product $\prod_{i\in
\theta'}x_i$,
 the following 
\begin{alg}[Iterated local cohomology]~

\mylabel{alg-lclc}
\myindex{local cohomology!algorithm for}
\myindex{local cohomology!iterated}
\noindent {\sc Input}: $f_1,\ldots,f_r\in R_n; i_0,j_0\in \N$.

\noindent {\sc Output}: 
$H^{i_0}_\m (H^{j_0}_I(R_n))$ in terms of generators and relations as
finitely generated $D_n$-module where $I=R_n\cdot(f_1,\ldots,f_r)$.

\begin{enumerate}
\item For $i=i_0-1, i_0, i_0+1$ and $j=j_0-1,j_0,j_0+1$ compute the
annihilators $J^\Delta((F_\theta\cdot X_{\theta'})^s)$, Bernstein
polynomials $b^\Delta_{F_\theta\cdot X_{\theta'}}(s)$, and stable
integral exponents $a^\Delta_{F_\theta\cdot X_{\theta '}}$  
of $F_\theta\cdot
X_{\theta'}$ 
where $\theta \in \Theta^r_j, \theta'\in \Theta^n_i$.

\item Let $a$ be the minimum of all $a^\Delta_{F_\theta\cdot X_{\theta '}}$
and replace $s$ by $a$ in all the annihilators
computed in the previous step.

\item Compute the matrices to the $D_n$-linear maps
 $\phi^{i,j}:\check C^{i,j}\to 
\check C^{i,j+1}$ 
and $\xi^{k,l}:\check C^{k,l}\to \check C^{k+1,l}$, for
$(i,j)\in\{(i_0,j_0),(i_0+1,j_0-1),(i_0,j_0-1),(i_0-1,j_0)\}$ and 
$(k,l)\in \{(i_0,j_0),(i_0-1,j_0)\}$. 
\item Compute a Gr\"obner basis $G$ for the module 
\[
D_n\cdot G=\ker(\phi^{i_0,j_0})\cap
\left[ 
(\xi^{i_0,j_0})^{-1}(\im(\phi^{i_0+1,j_0-1}))\right]+\im(\phi^{i_0,j_0-1})
\] 
and a Gr\"obner basis $G_0$ for the module
\[
D_n\cdot
G_0=\xi^{i_0-1,j_0}(\ker(\phi^{i_0-1,j_0}))+\im(\phi^{i_0,j_0-1}).
\]
\item Compute the remainders of all elements of $G$ with
respect to $G_0$. 

\item Return these remainders together with $G_0$.
\end{enumerate}
End.
\end{alg}

Note that $(D_n\cdot G)/(D_n\cdot G_0)$ is isomorphic to
\[\frac{
 \ker\left(\frac{\displaystyle\ker(\phi^{i_0,j_0})}
            {\displaystyle\im(\phi^{i_0,j_0-1})}
      \stackrel{\xi^{i_0,j_0}}{\longrightarrow}
      \frac{\displaystyle\ker(\phi^{i_0+1,j_0})}
            {\displaystyle\im(\phi^{i_0+1,j_0-1})}\right)}
 {\xi^{i_0-1,j_0}\left(
      \frac{\displaystyle\ker(\phi^{i_0-1,j_0})}
           {\displaystyle\im(\phi^{i_0-1,j_0-1})}\right)}\cong H^{i_0}_\m(H^{j_0}_I(R_n)).
\]
The elements of $G$ will be generators for $H^{i_0}_\m
(H^{j_0}_I(R_n))$ and 
the elements of $G_0$ generate the extra relations that are not
syzygies.

The algorithm can of course be modified to compute any iterated local
cohomology group $H^j_J(H^i_I(R_n))$ for $J\supseteq I$ by replacing
the generators $x_1,\ldots,x_n$ for $\m$ by those for $J$. Moreover,
the iteration depth can also be increased by considering
``tricomplexes'' etc.\ instead of bicomplexes.

Again we would like to point out that with the methods of \cite{DM:O-T1}
or \cite{DM:O-T-W} one could actually compute first $H^i_I(R_n)$ and from
that $H^j_J(H^i_I(R_n))$, but probably that is quite a bit more
complex a computation. 
%
\subsection{Computation of Lyubeznik Numbers}
\mylabel{subsec-lambda}
G.\ Lyubeznik proved in \cite{DM:L-Dmod} that if $K$ is a field, 
$R=K[x_1,\ldots,x_n]$, 
$I\subseteq R$, $\m=R\cdot (x_1,\ldots,x_n)$ 
and $A=R/I$ then $\lambda_{i,j}(A)=\dim_K\soc_R H^i_\m(H^{n-j}_I(R))$ is
invariant under change of presentation of $A$. 
In other words, it only
depends on $A$ and $i,j$ but not the projection $R\onto A$. 
Lyubeznik proved that $H^i_\m (H^j_I(R_n))$ is in fact an injective
$\m$-torsion $R_n$-module of finite socle dimension $\lambda_{i,n-j}(A)$
and so
isomorphic to $(E_{R_n}(K))^{\lambda_{i,n-j}(A)}$ where $E_{R_n}(K)$ is the
injective hull of $K$ over $R_n$. We
are now in a position to compute these invariants of $R_n/I$ in
characteristic zero..

\begin{alg}[Lyubeznik numbers\index{Lyubeznik numbers}]~

\mylabel{alg-lambda}
\noindent {\sc Input}: $f_1,\ldots,f_r\in R_n; i,j\in \N$.

\noindent {\sc Output}: $\lambda_{i,n-j}(R_n/R_n\cdot(f_1,\ldots,f_r))$.

\begin{enumerate}
\item Using Algorithm \ref{alg-lclc} find $g_1,\ldots,g_l\in {D_n}^d$
and $h_1,\ldots,h_e\in {D_n}^d$ such that $H^i_\m(H^j_I(R_n))$ is
isomorphic to $D_n\cdot (g_1,\ldots,g_l)$ modulo $H=D_n\cdot (h_1,\ldots,h_e)$.
\item Assume that after a suitable renumeration 
$g_1$ is not in $H$. If such a $g_1$ cannot be
chosen, quit. 
\item Find a monomial $m\in R_n$ such that $m\cdot g_1\not\in H$ but
$x_img_1\in H$ for all $x_i$.
\item Replace $H$ by $D_nmg_1+H$ and reenter at Step 2.
\item Return $\lambda_{i,n-j}(R_n/I)$, the number of times Step 3 was
executed. 
\end{enumerate}
End.
\end{alg}
The reason that this works is as follows.
We know that
$(D_n\cdot g_1+H)/H$ is $\m$-torsion (as $H^i_\m(H^j_I(R_n))$ is) 
and so it is possible (with trial
and error, or a suitable syzygy computation) 
to find the monomial $m$ in Step 3.
The element $mg_1 \mod H\in D_n/H$ has annihilator equal
to $\m$ over $R_n$ and therefore generates a $D_n$-module isomorphic to
$D_n/D_n\cdot \m\cong E_{R_n}(K)$. The injection 
\[
(D_n\cdot mg_1+H)/H\into
(D_n\cdot(g_1,\ldots,g_l)+H)/H
\]
splits as map of $R_n$-modules because $E_{R_n}(K)$ is  
injective and so the cokernel 
$D_n\cdot (g_1,\ldots,g_l)/D_n\cdot (mg_1,h_1,\ldots,h_e)$ is isomorphic to
$(E_{R_n}(K))^{\lambda_{i,n-j}(A)-1}$. 

Reduction of the $g_i$ with respect to a Gr\"obner basis of the new
relation module and repetition will lead to
the determination of $\lambda_{i,n-j}(A)$.
%

Assume that $D_n/L$ is an $\m$-torsion module. For example, we could
have $D_n/L\cong H^i_\m(H^j_I(R_n))$.  
Here is a procedure that finds by trial and error 
the monomial socle element $m$ of Step
3 in Algorithm \ref{alg-lclc}.
\beginOutput
i61 : findSocle = method();\\
\endOutput
\beginOutput
i62 : findSocle(Ideal, RingElement):= (L,P) -> (\\
\           createDpairs(ring(L));\\
\           v=(ring L).dpairVars#0;\\
\           myflag = true;\\
\           while myflag do (\\
\                w = apply(v,temp -> temp*P {\char`\%} L);\\
\                if all(w,temp -> temp == 0) then myflag = false\\
\                else (\\
\                     p = position(w, temp -> temp != 0);\\
\                     P = v#p * P;)\\
\                );\\
\           P);\\
\endOutput
For example, if we want to apply this socle search to the ideal
{\tt JH3} describing $H^3_I(R_6)$ of Example \ref{ex-minors} we do
\beginOutput
i63 : D = ring JH3\\
\emptyLine
o63 = D\\
\emptyLine
o63 : PolynomialRing\\
\endOutput
(as {\tt D} was most recently the differential operators on $\Q[x,y,z,w]$)
\beginOutput
i64 : findSocle(JH3,1_D)\\
\emptyLine
o64 = x*v\\
\emptyLine
o64 : D\\
\endOutput
One can then repeat the socle search and kill the newly found element
as suggested in the explanation above:
\beginOutput
i65 : findLength = method();\\
\endOutput
\beginOutput
i66 : findLength Ideal := (I) -> (   \\
\           l = 0;\\
\           while I != ideal 1_(ring I) do (\\
\                l = l + 1;\\
\                s = findSocle(I,1_(ring I));\\
\                I = I + ideal s;);\\
\           l);\\
\endOutput
Applied to {\tt JH3} of the previous subsection this yields
\beginOutput
i67 : findLength JH3\\
\emptyLine
o67 = 1\\
\endOutput
and hence {\tt JH3} does indeed describe a module isomorphic to $E_{R_6}(K)$.
\nocite{DM:W1}

%\input{6.tex}
\section{Implementation, Examples, Questions}
\mylabel{sec-ausblick}
\subsection{Implementations and Optimizing}
The Algorithms \ref{alg-ann-fs}, \ref{alg-b-poly-L} and \ref{alg-D/L-loc-f} 
have first been implemented by
T.\ Oaku
  and N.\ Takayama 
using the package Kan \cite{DM:T} which
is a postscript language for computations in the Weyl algebra and in
polynomial rings. 
In \Mtwo Algorithms  \ref{alg-ann-fs},
\ref{alg-b-poly-L} and \ref{alg-D/L-loc-f} as well as Algorithm 
\ref{alg-lc}
have been implemented by A.~Leykin, M.~Stillman and H.~Tsai. They additionally
implemented a wealth of $D$-module routines that
relate to topics which we cannot all cover in this chapter. These include
homomorphisms between holonomic modules and extension functors,
restriction functors to linear subspaces, integration (de Rham)
functors to quotient spaces and others. For further theoretical 
information 
the reader is referred to \cite{DM:O-T1,DM:O-T2,DM:O-T-T,DM:SST,DM:T-W,DM:W2,DM:W4,DM:W3}. 

\mylabel{efficiency}
Computation of Gr\"obner bases in many variables is in general a time
and space consuming enterprise. In commutative polynomial
rings the worst case performance for the number of elements in reduced
Gr\"obner bases 
is doubly exponential in the number of variables and the degrees of
the generators. In the (relatively) small 
Example \ref{ex-minors} above $R_6$ is of dimension 6,
so that the intermediate ring $D_{n+1}[y_1,y_2]$ contains 16
variables. In view of these facts the following idea 
has proved useful. 

The general context in which Lemma \ref{lem-malgrange} and Proposition
\ref{prop-kashiwara} were stated allows successive localization of
$R_n[(fg)^{-1}]$ 
in the following way. First one computes $R_n[f^{-1}]$ according to
Algorithm \ref{alg-D/L-loc-f} as quotient $D_n/J^\Delta(f^s)|_{s=a}$,
$\Z\ni a\ll 0$.  
Then $R_n[(fg)^{-1}]$ may be
computed using  Algorithm 
\ref{alg-D/L-loc-f} again since $R_n[(fg)^{-1}]\cong
R_n[g^{-1}]\otimes_{R_n} 
D_n/J^\Delta(f^s)|_{s=a}$. (Note that all
localizations of $R_n$ are automatically $f$-torsion free for $f\in R_n$
so that Algorithm \ref{alg-D/L-loc-f} can be used.) This process
may be iterated for products with any finite number of factors. 
Of course the exponents for the various factors might be different. 
This requires some care 
%as the following situations illustrate. Assume
%first that $-1$ is the smallest integer root of the Bernstein The
%Bernstein polynomials
%of $f$ and $g$ (both in $R_n$) with respect to the holonomic module
%$R_n$. Assume further that $R_n[(fg)^{-1}]\cong D_n\action
%( f^{-2}g^{-1})
%\supsetneq
%D_n\action( (fg)^{-1})$. Then $R_n[f^{-1}]\to R_n[(fg)^{-1}]$ can be written as
%$D_n/\ann(f^{-1})\to D_n/\ann(f^{-2}\cdot g^{-1})$ sending $P\in D_n$ to
%$P\cdot f\cdot g$. 
%
%Suppose on the other hand that we are interested in $H^2_I(R_n)$ where
%$I=(f,g,h)$ and we know that $R_n[f^{-1}]=D_n\action
%( f^{-2})\supsetneq 
%D_n\action(
%f^{-1}), R_n[g^{-1}]=D_n\action( g^{-2})$ and $R_n[(fg)^{-1}]=D_n\action( f^{-1}g^{-2})$. (In
%fact, the degree 2 part of the \v Cech complex of Example
%\ref{ex-minors} consists of such localizations.) We cannot write
%the embedding $R_n[f^{-1}]\to R_n[(fg)^{-1}]$ with the use of a
%Bernstein operator for $s=-2$ since $f^{-1}$ is not a
%generator for $R_n[f^{-1}]$. So we must
%write $R_n[(fg)^{-1}]$ as $D_n/\ann((fg)^{-2})$ and then send $P\in
%\ann(f^{-2})$ to $P\cdot g^2$. 
%
%The two examples indicate how to write the \v Cech complex in terms
%of generators and relations over $D_n$ while making 
%
%
when setting up the \v Cech complex. In particular one needs to make 
sure that the maps
$\check C^k\to \check C^{k+1}$ can be made explicit using the $f_i$.
% -- the exponents
%used in $C^k$ have to be at least as big as those in $C^{k-1}$ (for the
%same $f_i$). 
(In our Example \ref{ex-minors}, this is precisely how we proceeded
when we found {\tt Rfgh}.)
\begin{remark}
\label{remark}
One might hope that for all holonomic $fg$-torsion free 
modules $M=D_n/L$ 
we have  (with $M\otimes R_n[g^{-1}]\cong D_n/L'$):
\begin{eqnarray}
\label{eq-ineq}
a^L_f=\min\{s\in\Z:b_f^L(s)=0\}\le \min\{s\in\Z:b_f^{L'}(s)=0\}=a^{L'}_f.
\end{eqnarray}
This hope is unfounded. Let $R_5=K[x_1,\ldots,x_5]$,
$f=x_1^2+x_2^2+x_3^2+x_4^2+x_5^2$. One may check that then
$b^\Delta_f(s)=(s+1)(s+5/2)$. Hence $R_5[f^{-1}]=D_5\action f^{-1}$, let
$L=\ker(D_5\to D_5\action f^{-1})$. Set $g=x_1$. Then
$b^\Delta_g(s)=s+1$, let $L'=\ker(D_5\to D_5\action g^{-1})$.

Then
$b^{L'}_f(s)=(s+1)(s+2)(s+5/2)$ and $b^L_g(s)=(s+1)(s+3)$ because of 
the following computations.
\beginOutput
i68 : erase symbol x; erase symbol Dx;\\
\endOutput
These two commands essentially clear the history of the variables {\tt
x} and {\tt Dx} and make them available for future computations.
\beginOutput
i70 : D = QQ[x_1..x_5, Dx_1..Dx_5, WeylAlgebra =>\\
\           apply(toList(1..5), i -> x_i => Dx_i)];\\
\endOutput
\beginOutput
i71 : f = x_1^2 + x_2^2 + x_3^2 + x_4^2 +x_5^2;\\
\endOutput
\beginOutput
i72 : g = x_1;\\
\endOutput
\beginOutput
i73 : R = D^1/ideal(Dx_1,Dx_2,Dx_3,Dx_4,Dx_5);\\
\endOutput
As usual, these commands defined the base ring, two polynomials and
the $D_5$-module $R_5$. Now we compute the respective localizations.
\beginOutput
i74 : Rf =DlocalizeAll(R,f,Strategy => Oaku);\\
\endOutput
\beginOutput
i75 : Bf = Rf.Bfunction\\
\emptyLine
\            5\\
o75 = (\$s + -)(\$s + 1)\\
\            2\\
\emptyLine
o75 : Product\\
\endOutput
\beginOutput
i76 : Rfg = DlocalizeAll(Rf.LocModule,g,Strategy => Oaku);\\
\endOutput
\beginOutput
i77 : Bfg = Rfg.Bfunction\\
\emptyLine
o77 = (\$s + 1)(\$s + 3)\\
\emptyLine
o77 : Product\\
\endOutput
\beginOutput
i78 : Rg = DlocalizeAll(R,g,Strategy => Oaku);\\
\endOutput
\beginOutput
i79 : Bg = Rg.Bfunction\\
\emptyLine
o79 = (\$s + 1)\\
\emptyLine
o79 : Product\\
\endOutput
\beginOutput
i80 : Rgf = DlocalizeAll(Rg.LocModule,f,Strategy => Oaku);\\
\endOutput
\beginOutput
i81 : Bgf = Rgf.Bfunction\\
\emptyLine
\                            5\\
o81 = (\$s + 2)(\$s + 1)(\$s + -)\\
\                            2\\
\emptyLine
o81 : Product\\
\endOutput

%\begin{verbatim}
%i20 : D = QQ[x_1..x_5, Dx_1..Dx_5, WeylAlgebra => 
%    apply(toList(1..5), i -> x_i => Dx_i)]
%i21 : f = x_1^2 + x_2^2 + x_3^2 + x_4^2 +x_5^2
%i22 : g = x_1
%i23 : R = D^1/ideal(Dx_1,Dx_2,Dx_3,Dx_4,Dx_5)
%i24 : M1 =DlocalizeAll(R,f,Strategy => Oaku)
%i25 : M12 = DlocalizeAll(M1.LocModule,g,Strategy => Oaku)
%i26 : M2 = DlocalizeAll(R,g,Strategy => Oaku)
%i27 : M21 = DlocalizeAll(M2.LocModule,f,Strategy => Oaku)
%\end{verbatim}
The output shows
that $R_n[(fg)^{-1}]$ is generated by $f^{-2}g^{-1}$ or $f^{-1}g^{-3}$ but not
by $f^{-1}g^{-2}$ and in particular not by $f^{-1}g^{-1}$. 
This can be seen from the various Bernstein-Sato polynomials: as for
example the smallest integral root of {\tt Bf} is $-1$ and that of
{\tt Bfg} is $-3$, $R_3[f^{-1}]$ is generated by $f^{-1}$ and
$R_3[(fg)^{-1}]$ by $f^{-1}g^{-3}$.
This example not only disproves the above inequality (\ref{eq-ineq})
but also shows
the inequality to be wrong if $\Z$ is replaced by $\R$ (as
$-3<\min(-5/2,-1)$). 
\end{remark}


Nonetheless, localizing $R_n[(fg)^{-1}]$ as $(R_n[f^{-1}])[g^{-1}]$ is
heuristically advantageous, apparently for two reasons. For 
one, it allows the exponents of the various factors to be distinct
which is useful for the subsequent cohomology computation: it helps
to keep the degrees of the maps small. So in
Example \ref{ex-minors} we can write $R_6[(fg)^{-1}]$ as $D_6\action 
(f^{-1} g^{-2})$ instead
of $D_6\action (fg)^{-2}$. 
Secondly,  
since the computation of Gr\"obner bases is potentially 
doubly exponential it
seems to be advantageous to break a big problem (localization at a
product) into several ``easy'' problems (successive localization).

An interesting case of this behavior is our Example \ref{ex-minors}. If we
compute $R_n[(fgh)^{-1}]$ as $((R_n[f^{-1}])[g^{-1}])[h^{-1}]$, 
the calculation uses
approximately 6MB and lasts a few seconds 
using \Mtwo. If one tries to localize $R_n$ at the
product of the three generators at once, \Mtwo 
runs out of memory on all machines the author has 
tried this computation on.
\subsection{Projects for the Future}
This is a list of theoretical and implementational questions that the
author finds 
important and interesting.
\subsubsection{Prime Characteristic} In \cite{DM:L-Fmod},
G.~Lyubeznik gave an
algorithm for deciding whether or not 
$H^i_I(R)=0$ for any given  $I\subseteq R=K[x_1,\ldots,x_n]$ where
$K$ is a computable field of positive characteristic. His algorithm is built on
entirely different methods than the ones used in this chapter 
and relies on the Frobenius functor. The
implementation of this algorithm would be quite worthwhile.
\subsubsection{Ambient Spaces Different from ${\mathbb A}^n_K$}
\mylabel{singular-spaces}
If $A$ equals $K[x_1,\ldots,x_n]$, 
$I\subseteq A$, $X=\spec (A)$ and $Y=\spec(A/I)$,
knowledge of $H^i_I(A)$ for all $i\in \N$ answers of course the
question about the local cohomological dimension of $Y$ in $X$. 
If $W\subseteq X$ is a smooth variety containing $Y$
then Algorithm \ref{alg-lc} for the computation of $H^i_I(A)$ also
leads to a determination of the local cohomological dimension of $Y$
in $W$. Namely, if $J$ stands for the
defining ideal of $W$ in $X$ so that $R=A/J$ is the affine
coordinate ring of $W$  and if we set $c=\height(J)$, then it can be
shown that 
$H^{i-c}_{I}(R)=\hom_A(R,H^i_I(A))$ for all $i\in\N$.
As $H^i_I(A)$ is 
$I$-torsion (and hence $J$-torsion), $\hom_A(R,H^i_I(A))$ is zero if
and only if 
$H^i_I(A)=0$. It follows that the local cohomological dimension of $Y$
in $W$ equals $\cd(A,I)-c$ and in fact $\{i\in \N:H^i_I(A)\not =0\}=\{i\in
\N:H^{i-c}_I(R)\not =0\}$. 

If however $W=\spec(R)$ is not smooth, no algorithms for the computation of
either $H^i_I(R)$ or $\cd(R,I)$ are known, irrespective of the
characteristic of the base field. It would be very interesting to have
even partial ideas for computations in that case.
\subsubsection{De Rham Cohomology} In \cite{DM:O-T1,DM:W2} algorithms are given
 to compute de Rham (in this case equal to singular) 
cohomology of complements of complex affine
hypersurfaces and more general varieties. In \cite{DM:W3} an algorithm
is given to compute the multiplicative (cup product) structure, and in
\cite{DM:W4} the computation of the de Rham cohomology of open and
closed sets in projective
space is explained. Some of these
algorithms have been implemented while others are still waiting. 

For
example, de Rham cohomology of complements of hypersurfaces, and
partially the cup product routine,  are implemented.
\begin{example}
Let $f=x^3+y^3+z^3$ in $R_3$. One can compute with \Mtwo
the de Rham cohomology of the complement of $\var(f)$, and 
it turns out that the
cohomology in degrees 0 and 1 is 1-dimensional, in degrees 3 and 4
2-dimensional and zero otherwise -- here is the input:
\beginOutput
i82 : erase symbol x;\\
\endOutput
Once {\tt x} gets used as a subscripted
variable, it's hard to use it as a nonsubscripted variable.  So let's just
erase it.
\beginOutput
i83 : R = QQ[x,y,z];\\
\endOutput
\beginOutput
i84 : f=x^3+y^3+z^3;\\
\endOutput
\beginOutput
i85 : H=deRhamAll(f);\\
\endOutput
%\begin{verbatim}
%i28 ; R = QQ[x,y,z]
%i29 ; f=x^3+y^3+z^3
%i30 : deRhamAll(f)
%\end{verbatim}
{\tt H} is a hashtable with the entries 
{\tt Bfunction}, 
{\tt LocalizeMap}, 
{\tt VResolution}, 
{\tt TransferCycles}, 
{\tt PreCycles}, 
{\tt OmegaRes} and 
{\tt CohomologyGroups}. For example,
we have
\beginOutput
i86 : H.CohomologyGroups\\
\emptyLine
\                       1\\
o86 = HashTable\{0 => QQ \}\\
\                       1\\
\                1 => QQ\\
\                       2\\
\                2 => QQ\\
\                       2\\
\                3 => QQ\\
\emptyLine
o86 : HashTable\\
\endOutput
showing that the dimensions are as claimed above. One can also extract
information on the generator of $R_3[f^{-1}]$ used to represent the
cohomology classes by
\beginOutput
i87 : H.LocalizeMap\\
\emptyLine
o87 = | \$x_1^6+2\$x_1^3\$x_2^3+\$x_2^6+2\$x_1^3\$x_3^3+2\$x_2^3\$x_3^3+\$x_3^6 |\\
\emptyLine
o87 : Matrix\\
\endOutput
which proves that the generator in question is $f^{-2}$. The cohomology
classes that \Mtwo computes are differential forms:
\beginOutput
i88 : H.TransferCycles\\
\emptyLine
o88 = HashTable\{0 => | -1/12\$x_1^4\$x_2^3\$D_1-1/3\$x_1\$x_2^6\$D_1-1/12\$x_ $\cdot\cdot\cdot$\\
\                1 => | 2/3\$x_1^5+2/3\$x_1^2\$x_2^3+2/3\$x_1^2\$x_3^3  |\\
\                     | -2/3\$x_1^3\$x_2^2-2/3\$x_2^5-2/3\$x_2^2\$x_3^3 |\\
\                     | 2/3\$x_1^3\$x_3^2+2/3\$x_2^3\$x_3^2+2/3\$x_3^5  |\\
\                2 => | 48\$x_1\$x_2\$x_3^2 600\$x_3^4     |\\
\                     | 48\$x_1\$x_2^2\$x_3 600\$x_2\$x_3^3 |\\
\                     | 48\$x_1^2\$x_2\$x_3 600\$x_1\$x_3^3 |\\
\                3 => | -\$x_1\$x_2\$x_3 -\$x_3^3 |\\
\emptyLine
o88 : HashTable\\
\endOutput
So, for example, the left column of the three rows that correspond to
$H^2_{{\rm dR}}(\C^3\setminus\var(f),\C)$ represent the form
$({xyz(48zdxdy+48ydzdx+48xdydz)})/{f^2}$. 
If we apply the above elements of $D_3$ to $f^{-2}$ and equip the
results with appropriate differentials one arrives (after dropping
unnecessary integral factors) at 
the results displayed in the table in Figure \ref{Figure}.
\begin{figure}
\[
\def\sp#1{\vcenter{\vskip 2pt \hbox{$#1$}\vskip 2pt}}
\begin{array}{|c|c|c|}
\hline
\text{Group}&\text{Dimension}&\text{Generators}\\
\hline
\hline
H^0_{{\rm dR}}&1&\sp{e:=\frac{\displaystyle f^2}{\displaystyle f^2}}\\
\hline
H^1_{{\rm dR}}&1&\sp{o:=\frac{\displaystyle (x^2dx-y^2dy+z^2dz)f}{\displaystyle f^2}}\\
\hline
H^2_{{\rm dR}}&2&\sp{t_1:=\frac{\displaystyle xyz(zdxdy+ydzdx+xdydz)}{\displaystyle f^2}}\\
        & &\sp{t_2:=\frac{\displaystyle (zdxdy+ydzdx+xdydz)z^3}{\displaystyle f^2}}\\
\hline
H^3_{{\rm dR}}&2&\sp{d_1:=\frac{\displaystyle xyzdxdydz}{\displaystyle f^2}}\\
        & &\sp{d_2:=\frac{\displaystyle z^3dxdydz}{\displaystyle f^2}}\\
\hline
\end{array}
\]
\caption{}\label{Figure}
\end{figure}
In terms of de Rham cohomology, the cup product is given by the wedge
product of differential forms. So 
from the table one reads off the cup  product
relations
$o\cup t_1=d_1$, $o\cup t_2=d_2$, $o\cup
o=0$, and that $e$ operates as the identity. 
\end{example}
For more involved examples, and the algorithms in \cite{DM:W4}, an
actual implementation would be  necessary since paper and pen are
insufficient tools then.
\begin{remark}
The reader should be warned: if $f^{-a}$ generates $R_n[f^{-1}]$ over
$D_n$, then it is not necessarily the case that each de Rham
cohomology class of $U=\C^n\setminus\var(f)$ can be written as a form
with a pole of order at most $a$. A counterexample is given by
$f=(x^3+y^3+xy)xy$, where $H^1_{{\rm dR}}(U;\C)$ has a class that requires a
third order pole, although $-2$ is the smallest integral Bernstein root
of $f$ on $R_2$.
\end{remark}

\subsubsection{Hom and Ext}
In \cite{DM:O-T-T,DM:Ts0,DM:T-W} algorithms are explained that compute homomorphisms
between holonomic systems. In particular, rational and polynomial
solutions can be found because, for example, a polynomial solution to the system
$\{P_1,\ldots,P_r\}\in D_n$ corresponds to an element of
$\hom_{D_n}(D_n/I,R_n)$ where $I=D_n\cdot(P_1,\ldots,P_r)$.
\begin{example}
Consider the GKZ system in 2 variables associated to the matrix
$(1,2)\in \Z^{1\times 2}$ and the parameter vector $(5)\in\C^1$. 
Named after
Gelfand-Kapranov-Zelevinski \cite{DM:GKZ}, this 
is the following system of
differential equations:
\begin{eqnarray*}
(x\del_x+y\del_y)\action f&=&5f,\\
(\del_x^2-\del_y)\action f&=&0.
\end{eqnarray*}
With \Mtwo one can solve systems of this sort as follows:
\beginOutput
i89 : I = gkz(matrix\{\{1,2\}\}, \{5\})\\
\emptyLine
\              2\\
o89 = ideal (D  - D , x D  + 2x D  - 5)\\
\              1    2   1 1     2 2\\
\emptyLine
o89 : Ideal of QQ [x , x , D , D , WeylAlgebra => \{x  => D , x  => D \}]\\
\                    1   2   1   2                   1     1   2     2\\
\endOutput
This is a simple command to set up the GKZ-ideal associated to a
matrix and a parameter vector. The polynomial solutions are obtained
by
\beginOutput
i90 : PolySols I\\
\emptyLine
\        5      3          2\\
o90 = \{x  + 20x x  + 60x x \}\\
\        1      1 2      1 2\\
\emptyLine
o90 : List\\
\endOutput

This means that there is exactly one polynomial solution to the given
GKZ-system, and it is 
\[
x^5+20x^3y+60xy^2.
\]
\end{example}
The algorithm for $\hom_{D_n}(M,N)$ is implemented and
can be used to check whether two given $D$-modules are
isomorphic. 
Moreover, there are algorithms (not implemented yet) 
to compute the ring structure of
$\endo_D(M)$ for a given $D$-module $M$ of finite holonomic
rank which can be used to split a given holonomic module into its
direct summands. 
Perhaps an adaptation of these methods can be used to construct
Jordan-H\"older sequences for holonomic $D$-modules.

\subsubsection{Finiteness and Stratifications}
Lyubeznik pointed out in \cite{DM:L-Bpoly} the following curious
fact. 
\begin{theorem}
Let $P(n,d;K)$ denote the set of polynomials of degree at most $d$ in
at most $n$ variables over the field $K$ of characteristic zero. Let
$B(n,d;K)$ denote the set  of Bernstein-Sato polynomials
\[
B(n,d;K)=\{b_f(s):f\in P(n,d;K)\}.
\]
Then $B(n,d;K)$ is finite.\qed
\end{theorem}
So $P(n,d;K)$ has a finite decomposition into strata with constant
Bernstein-Sato polynomial.
A.\ Leykin proved in \cite{DM:Ley} that this decomposition is independent
of $K$ and computable
in the sense that membership in  each stratum can be tested by the
vanishing of a
finite set of algorithmically computable polynomials over $\Q$ in the
coefficients of the given polynomial in $P(n,d;K)$. In particular,
the stratification is algebraic and for each $K$ induced by base
change from $\Q$ to $K$. It makes thus sense to define $B(n,d)$ which
is the finite set of Bernstein polynomials that can occur for $f\in
P(n,d;K)$ (where $K$ is in fact irrelevant).
\begin{example}
Consider $P(2,2;K)$, the set of quadratic binary forms over $K$. With
\Mtwo, Leykin showed that there are
precisely 4 different Bernstein polynomials possible:
\def\labelitemi{$\bullet$}
\begin{itemize}
\item $b_{f}(s)=1$
iff $f\in V_{1}=V_{1}'\setminus V_{1}''$,
where $V_{1}'=V(a_{1,1},a_{0,1},a_{0,2},a_{1,0},a_{2,0}) $, while
$V_{1}''=V(a_{0,0})$
\item $ b_{f}(s)=s+1 $
iff $ f\in V_{2}=(V_{2}'\setminus V_{2}'')\cup (V_{3}'\setminus V_{3}'') $,
where $ V_{2}'=V(0) $, $ V_{2}''=V(\gamma _{1}) $,
$V_{3}'=V\left( \gamma _{2},\gamma _{3},\gamma _{4}\right)$, 
$V_{3}''=V\left( \gamma _{3},\gamma _{4},\gamma _{5},\gamma _{6},\gamma _{7},\gamma _{8}\right)  $,
\item $ b_{f}(s)=(s+1)^{2} $
iff $ f\in V_{4}'\setminus V_{4}'' $,
where $ V_{4}'=V(\gamma _{1}) $, 
$ V_{4}''=V\left( \gamma _{2},\gamma _{3},\gamma _{4}\right)  $, 
\item $ b_{f}(s)=(s+1)(s+\frac{1}{2}) $
iff $ f\in V_{5}'\setminus V_{5}'' $,
where $ V_{5}'=V\left( \gamma _{3},\gamma _{4},\gamma _{5},\gamma _{6},\gamma _{7},\gamma _{8}\right)  $,
$ V_{5}''=V(a_{1,1},a_{0,1},a_{0,2},a_{1,0},a_{2,0}) $.
\end{itemize}
Here we have used the abbreviations
\begin{itemize}
\item $ \gamma_{1}=
a_{0,2}a_{1,0}^{2}-a_{0,1}a_{1,0}a_{1,1}+a_{0,0}a_{1,1}^{2}+ 
a_{0,1}^{2}a_{2,0}-4a_{0,0}a_{0,2}a_{2,0}$, 
\item$ \gamma _{2}=2a_{0,2}a_{1,0}-a_{0,1}a_{1,1} $,

\item $ \gamma _{3}=a_{1,0}a_{1,1}-2a_{0,1}a_{2,0} $, 

\item $ \gamma _{4}=a_{1,1}^{2}-4a_{0,2}a_{2,0} $,

\item $ \gamma _{5}=2a_{0,2}a_{1,0}-a_{0,1}a_{1,1} $,

\item $ \gamma _{6}=a_{0,1}^{2}-4a_{0,0}a_{0,2} $,

\item $ \gamma _{7}=a_{0,1}a_{1,0}-2a_{0,0}a_{1,1} $, 

\item $ \gamma _{8}=a_{1,0}^{2}-4a_{0,0}a_{2,0} $.
\end{itemize}
Similarly, Leykin shows that there are 9 possible Bernstein
polynomials for $f\in B(2,3;K)$:
\begin{eqnarray*}
B(2,3)&=&\left\{\,
 (s+1)^{2}(s+\frac{2}{3})(s+\frac{4}{3}),\,\,\,(s+1)^{2}(s+\frac{1}{2}),\,\,\,(s+1),\,\,\,1,\right.\\ 
 &  &\phantom{x}(s+1)(s+\frac{2}{3})(s+\frac{1}{3}),\,\,\,(s+1)^{2},\,\,\,(s+1)(s+\frac{1}{2}),\\
 &  &\phantom{x}\left.(s+1)(s+\frac{7}{6})(s+\frac{5}{6}),\,\,\,(s+1)^{2}(s+\frac{3}{4})(s+\frac{5}{4})\right\}.
\end{eqnarray*}
\end{example}
It would be very interesting to study the nature of the stratification
in larger cases, and its restriction to hyperplane arrangements.

A generalization of this stratification result is obtained in
\cite{DM:W5}. There it is shown that there is an algorithm to give
$P(n,d;K)$ an algebraic
stratification defined over $\Q$ 
such that the algebraic de Rham cohomology groups of the
complement of $\var(f)$ do not vary on the stratum in a rather strong
sense. 
Again, the study
and explicit computation of this stratification should be very
interesting.

\subsubsection{Hodge Numbers}
If $Y$ is a projective variety in $\P^n_\C$ then algorithms outlined in \cite{DM:W4}
show how to compute the dimensions not only of the de Rham cohomology
groups of $\P^n_\C\setminus Y$ but also of $Y$ itself. 
Suppose now that $Y$ is in fact a smooth  projective variety.  
An interesting set of invariants are the Hodge numbers, defined by
$h^{p,q}=\dim H^p(Y,\Omega^q)$, where $\Omega^q$ denotes the sheaf of
$\C$-linear differential $q$-forms with coefficients in $\OO_Y$. 
At present we do not know how to
compute them. Of course there is a spectral sequence $
H^p(Y,\Omega^q)\Rightarrow H^{p+q}_{{\rm dR}}(Y,\C)$ and we know the abutment
(or at least its dimensions), but the technique for computing the
abutment does not seem to be usable 
to compute the $E^1$ term because on an affine
patch $H^p(Y,\Omega^q)$ is either zero or an infinite dimensional
vector space.

Hodge structures and Bernstein-Sato polynomials are related as is for
example shown in \cite{DM:Varchenko}. 

\subsection{Epilogue}
In this chapter we have only touched a few highlights of the theory of
computations in $D$-modules, most of them related to homology and
topology. Despite this we hardly touched on the topics of integration
and restriction, which are the $D$-module versions of a pushforward
and pullback, \cite{DM:K2,DM:Mebkhout,DM:O-T1,DM:W2}. 

A very different aspect of $D$-modules is discussed in
\cite{DM:SST} where at the center  of investigations is the combinatorics of
solutions of hypergeometric differential equations. The combinatorial
structure is used to find series solutions for the differential
equations which are polynomial in certain logarithmic functions and 
power series with respect to the variables.

Combinatorial elements can also be found in the work of Assi, Castro
and Granger, see \cite{DM:ACG1,DM:ACG2}, 
on Gr\"obner fans in rings of differential
operators. An important (open) question in this direction is the
determination of the set of ideals in $D_n$ that are initial ideals
under {\em some} weight.



Algorithmic $D$-module theory promises to be an active
area of research for many years to come, and to have interesting
applications to various other parts of mathematics.


%\begin{example}
%In the case $n=d=2$ one has clearly 2 cases of the de Rham cohomology ring
%$DR(f)$ in $P_h(n,d;K)$:
%\begin{itemize}
%\item $DR(f)=K\oplus K\cdot e$ with $e^2=0$ iff $4a_{2,0}a_{02}=a_{1,1}^2$,
%\item $DR(f)=K\oplus K\cdot e_1\oplus K\cdot e_2$ with
%$e_1^2=e_2^2=e_1e_2=0$ iff $4a_{20}a_{0,2}\not =a_{1,1}^2$.
%\end{itemize}
%\end{example}


% Local Variables:
% mode: latex
% mode: reftex
% tex-main-file: "chapter-wrapper.tex"
% reftex-keep-temporary-buffers: t
% reftex-use-external-file-finders: t
% reftex-external-file-finders: (("tex" . "make FILE=%f find-tex") ("bib" . "make FILE=%f find-bib"))
% End:
\begin{thebibliography}{10}

\bibitem{DM:AK}
A.G.\ Aleksandrov and V.L.\ Kistlerov:
\newblock Computer method in calculating $b$-functions of non-isolated
  singularities.
\newblock {\em Contemp.\ Math.}, 131:319--335, 1992.

\bibitem{DM:ACG1}
A.~Assi, F.J. Castro-Jim\'enez, and M.~Granger:
\newblock How to calculate the slopes of a $\mathcal {D}$-module..
\newblock {\em Compositio Math.}, 104(2):107--123, 1996.

\bibitem{DM:ACG2}
A.~Assi, F.J. Castro-Jim\'enez, and M.~Granger:
\newblock The {G}r\"obner fan of an ${A}\sb n$-module..
\newblock {\em J.\ Pure Appl.\ Algebra}, 150(1):27--39, 2000.

\bibitem{DM:Bernstein-notes}
J.~Bernstein:
\newblock Lecture notes on the theory of ${D}$-modules.
\newblock Unpublished.

\bibitem{DM:B}
J.-E. Bj\"ork:
\newblock {\em Rings of Differential Operators}.
\newblock North Holland, New York, 1979.

\bibitem{DM:Brianconetal}
J.~Brian{\c c}on, M.\ Granger, Ph.\ Maisonobe, and M.\ Miniconi:
\newblock Algorithme de calcul du polyn\^ome de {B}ernstein: cas
  nond\'eg\'en\'er\'e.
\newblock {\em Ann.\ Inst.\ Fourier}, 39:553--610, 1989.

\bibitem{DM:Br-R}
M.~Brodmann and J.~Rung:
\newblock Local cohomology and the connectedness dimension in algebraic
  varieties.
\newblock {\em Comment.\ Math.\ Helvetici}, 61:481--490, 1986.

\bibitem{DM:B-S}
M.P.\ Brodmann and R.Y.\ Sharp:
\newblock {\em Local Cohomology: an algebraic introduction with geometric
  applications.}.
\newblock Cambridge Studies in Advanced Mathematics, 60. Cambridge University
  Press, 1998.

\bibitem{DM:Coutinho}
S.C.\ Coutinho:
\newblock {\em A primer of algebraic ${D}$-modules}.
\newblock London Mathematical Society Student Texts 33. Cambridge University
  Press, Cambridge, 1995.

\bibitem{DM:G-S}
R.~Garcia and C.~Sabbah:
\newblock Topological computation of local cohomology multiplicities.
\newblock {\em Coll.\ Math.}, XLIX(2-3):317--324, 1998.

\bibitem{DM:GKZ}
I.\ Gelfand, M.\ Kapranov, and A.\ Zelevinski:
\newblock {\em Discriminants, resultants, and multidimensional determinants}.
\newblock Mathematics: Theory \& Applications. Birkh\"auser Boston, Inc.,
  Boston, MA, 1994.

\bibitem{DM:Gri-Har}
P.\ Griffiths and J.\ Harris:
\newblock {\em Principles of Algebraic Geometry}.
\newblock A Wiley Interscience Publication. WILEY and Sons, 1978.

\bibitem{DM:lc-notes}
R.\ Hartshorne:
\newblock {\em Local Cohomology. A seminar given by A.{} Grothendieck},
  volume~41 of {\em Lecture Notes in Mathematics}.
\newblock Springer Verlag, 1967.

\bibitem{DM:CDAV}
R.~Hartshorne:
\newblock {C}ohomological {D}imension of {A}lgebraic {V}arieties.
\newblock {\em Ann.~Math.}, 88:403--450, 1968.

\bibitem{DM:H-Sp}
R.~Hartshorne and R.~Speiser:
\newblock Local cohomological dimension in characteristic $p$.
\newblock {\em Ann.~Math.(2)}, 105:45--79, 1977.

\bibitem{DM:DRCAV}
Robin Hartshorne:
\newblock On the {D}e {R}ham cohomology of algebraic varieties.
\newblock {\em Inst. Hautes \'Etudes Sci. Publ. Math.}, 45:5--99, 1975.

\bibitem{DM:Hu}
C.\ Huneke:
\newblock Problems on local cohomology.
\newblock {\em Res.\ Notes Math.}, 2:93--108, 1992.

\bibitem{DM:Hu-L}
C.~Huneke and G.~Lyubeznik:
\newblock On the vanishing of local cohomology modules.
\newblock {\em Invent.~Math.}, 102:73--93, 1990.

\bibitem{DM:K}
M.~Kashiwara:
\newblock ${B}$-functions and holonomic systems, rationality of
  ${B}$-functions..
\newblock {\em Invent.~Math.}, 38:33--53, 1976.

\bibitem{DM:K2}
M.~Kashiwara:
\newblock On the holonomic systems of linear partial differential equations,
  {II}.
\newblock {\em Invent.~Math.}, 49:121--135, 1978.

\bibitem{DM:Kollar}
J.~Koll\'ar:
\newblock Singularities of pairs, {\em in} {A}lgebraic {G}eometry, {S}anta
  {C}ruz, 1995.
\newblock {\em Proc.\ Symp. Pure Math.\ Amer.\ Math.\ Soc.}, 62:221--287, 1997.

\bibitem{DM:Ley}
A.\ Leykin:
\newblock Towards a definitive computation of {B}ernstein-{S}ato polynomials.
\newblock {\em {math.AG/0003155}\footnote{ The webpage {http://xxx.lanl.gov/}
  is a page designed for the storage of preprints, and allows posting and
  downloading free of charge.}}, 2000.

\bibitem{DM:M2D}
A.~Leykin, M.\ Stillman, and H.~Tsai:
\newblock The ${D}$-module package for {{{\sl Macaulay~2\/}\xispace}}.
\newblock {http://www.math.umn.edu/\~{}leykin/}.

\bibitem{Loeser}
Fran{\c{c}}ois Loeser:
\newblock Fonctions d'{I}gusa $p$-adiques, polyn\^omes de {B}ernstein, et
  poly\`edres de {N}ewton.
\newblock {\em J. Reine Angew. Math.}, 412:75--96, 1990.

\bibitem{DM:L-Dmod}
G.~Lyubeznik:
\newblock Finiteness {P}roperties of {L}ocal {C}ohomology {M}odules: an
  {A}pplication of ${D}$-modules to {C}ommutative {A}lgebra.
\newblock {\em Invent.~Math.}, 113:41--55, 1993.

\bibitem{DM:L-Fmod}
G.~Lyubeznik:
\newblock ${F}$-modules: applications to local cohomology and ${D}$-modules in
  characteristic $p>0$.
\newblock {\em Journal f\"ur die Reine und Angewandte Mathematik}, 491:65--130,
  1997.

\bibitem{DM:L-Bpoly}
G.~Lyubeznik:
\newblock On {B}ernstein-{S}ato polynomials.
\newblock {\em Proc.\ Amer.\ Math.\ Soc.}, 125(7):1941--1944, 1997.

\bibitem{DM:Maisonobe}
Ph.\ Maisonobe:
\newblock ${D}$-modules: an overview towards effectivity, in: {E}.\ {T}ournier,
  ed., {C}omputer {A}lgebra and {D}ifferential {E}quations.
\newblock {\em London Math. Soc. Lecture Note Ser., 193, Cambridge Univ.\
  Press, Cambridge}, pages 21--55, 1994.

\bibitem{DM:M}
B.~Malgrange:
\newblock Le polyn\^ome de {B}ernstein d'une singularit\'e isol\'ee.
\newblock {\em Lecture Notes in Mathematics, Springer Verlag}, 459:98--119,
  1975.

\bibitem{DM:Mebkhout}
Z.\ Mebkhout:
\newblock {\em Le formalisme des six op\'erations de Grothendieck pour les
  ${D_X}$-modules coh\'erents.}.
\newblock Travaux en Cours. Hermann, Paris, 1989.

\bibitem{DM:Oa}
T.~Oaku:
\newblock An algorithm for computing ${B}$-functions.
\newblock {\em Duke Math.~Journal}, 87:115--132, 1997.

\bibitem{DM:Oa3}
T.~Oaku:
\newblock {A}lgorithms for ${b}$-functions, restrictions, and algebraic local
  cohomology groups of ${D}$-modules.
\newblock {\em Advances in Appl.\ Math.}, 19:61--105, 1997.

\bibitem{DM:Oa2}
T.~Oaku:
\newblock {A}lgorithms for the $b$-function and ${D}$-modules associated with a
  polynomial.
\newblock {\em J.\ Pure Appl.\ Algebra}, 117-118:495--518, 1997.

\bibitem{DM:O-T2}
T.\ Oaku and N.\ Takayama:
\newblock An algorithm for de {R}ham cohomology groups of the complement of an
  affine variety via ${D}$-module computation.
\newblock {\em J.\ Pure Appl.\ Algebra}, 139:201--233, 1999.

\bibitem{DM:O-T1}
T.\ Oaku and N.\ Takayama:
\newblock Algorithms for ${D}$-modules -- restriction, tensor product,
  localization and local cohomology groups.
\newblock {\em Journal of Pure and Applied Algebra}, 156(2-3):267--308,
  February 2001.

\bibitem{DM:O-T-T}
T.\ Oaku, N.\ Takayama, and H.\ Tsai:
\newblock Polynomial and rational solutions of holonomic systems.
\newblock {\em {math.AG/0001064}}, 1999.

\bibitem{DM:O-T-W}
T.\ Oaku, N.\ Takayama, and U.\ Walther:
\newblock A localization algorithm for {D}-modules.
\newblock {\em J.\ Symb.\ Comp.}, 29(4/5):721--728, May 2000.

\bibitem{DM:Og}
A.~Ogus:
\newblock Local cohomological dimension of algebraic varieties.
\newblock {\em Ann.~Math. (2)}, 98:327--365, 1973.

\bibitem{DM:P-S}
C.~Peskine and L.~Szpiro:
\newblock Dimension projective finie et cohomologie locale.
\newblock {\em Inst.~Hautes \'Etudes Sci.~Publ.~Math.}, 42:47--119, 1973.

\bibitem{DM:SST}
M.~Saito, {B.\ Sturmfels}, and {N.\ Takayama}:
\newblock {\em Gr\"obner deformations of hypergeometric differential
  equations}.
\newblock Algorithms and Computation in Mathematics, 6. Springer Verlag, 1999.

\bibitem{DM:Satoetal}
M.\ Sato, M.\ Kashiwara, T.\ Kimura, and T.\ Oshima:
\newblock Micro-local analysis of prehomogeneous vector spaces.
\newblock {\em Inv.\ Math.}, 62:117--179, 1980.

\bibitem{DM:T}
N.\ Takayama:
\newblock Kan: A system for computation in algebraic analysis. {S}ource code
  available at {{www.math.kobe-u.ac.jp/{K}{A}{N}}}.
\newblock {\em Version 1 (1991), Version 2 (1994)}, the latest Version is
  2.990914 (1999).

\bibitem{DM:Ts}
H.\ Tsai:
\newblock Weyl closure, torsion, and local cohomology of ${D}$-modules.
\newblock Preprint, 1999.

\bibitem{DM:Ts0}
H.\ Tsai:
\newblock Algorithms for algebraic analysis.
\newblock {\em Thesis, University of California at Berkeley}, 2000.

\bibitem{DM:T-W}
H.\ Tsai and U.\ Walther:
\newblock Computing homomorphisms between holonomic ${D}$-modules.
\newblock {\em {math.RA/0007139}}, 2000.

\bibitem{DM:Varchenko}
A.\ Varchenko:
\newblock Asymptotic {H}odge structure in the vanishing cohomology.
\newblock {\em Math.\ USSR Izvestija}, 18(3):469--512, 1982.

\bibitem{DM:W1}
U.\ Walther:
\newblock Algorithmic {C}omputation of {L}ocal {C}ohomology {M}odules and the
  {L}ocal {C}ohomological {D}imension of {A}lgebraic {V}arieties.
\newblock {\em J.\ Pure Appl.\ Algebra}, 139:303--321, 1999.

\bibitem{DM:W2}
U.\ Walther:
\newblock Algorithmic {C}omputation of de {R}ham {C}ohomology of {C}omplements
  of {C}omplex {A}ffine {V}arieties.
\newblock {\em J.\ Symb.\ Comp.}, 29(4/5):795--839, May 2000.

\bibitem{DM:W5}
U.\ Walther:
\newblock Homotopy {T}ype, {S}tratifications and {G}r\"obner bases.
\newblock In preparation, 2000.

\bibitem{DM:W4}
U.\ Walther:
\newblock {C}ohomology with {R}ational {C}oefficients of {C}omplex {V}arieties.
\newblock {\em Contemp.\ Math.}, to appear.

\bibitem{DM:W3}
U.\ Walther:
\newblock The {C}up {P}roduct {S}tructure for {C}omplements of {C}omplex
  {A}ffine {V}arieties.
\newblock {\em J.\ Pure Appl.\ Algebra}, to appear.

\bibitem{DM:W-lambda}
U.\ Walther:
\newblock On the {L}yubeznik numbers of a local ring.
\newblock {\em Proc.\ Amer. Math.\ Soc.}, to appear.

\bibitem{DM:Yano}
T.\ Yano:
\newblock On the theory of $b$-functions.
\newblock {\em Publ.\ RIMS, Kyoto Univ.}, 14:111--202, 1978.

\end{thebibliography}
\egroup
\makeatletter
\renewcommand\thesection{\@arabic\c@section}
\makeatother


  \vfill\eject
  \rhpage
  \addtocontents{toc}{\protect\vskip 8pt }
  \addcontentsline{toc}{title}{Index}
  \markboth{Index}{Index}
  \renewcommand{\indexname}{Index}
  \threecolindex
  \printindex
 \end{document}
