%%%%%%%%%%%%%%%%%%%%%%%%%%%%%%%%%%%%%%%%%%%%%

%Format: LaTeX
%%
\documentclass[draft,twoside,12pt]{article}
%

%%%%%%%%%%%%% EISENBUDS %%%%%%%%%%%%%%%%%%
%\input begin.tex
%\showlabels
%\showlabelsabove
\input diagrams.tex
%\vsize=13.6truecm
%\hsize=19truecm
\overfullrule=0pt
% Gothic fonts from AMSTeX 
\font\tengoth=eufm10  \font\fivegoth=eufm5
\font\sevengoth=eufm7
\newfam\gothfam  \scriptscriptfont\gothfam=\fivegoth 
\textfont\gothfam=\tengoth \scriptfont\gothfam=\sevengoth
\def\goth{\fam\gothfam\tengoth}
%
% Bold italic fonts 
\font\tenbi=cmmib10  \font\fivebi=cmmib5
\font\sevenbi=cmmib7
\newfam\bifam  \scriptscriptfont\bifam=\fivebi 
\textfont\bifam=\tenbi \scriptfont\bifam=\sevenbi
\def\bi{\fam\bifam\tenbi}
%
\font\hd=cmbx10 scaled\magstep1
%%%%%%%%%%%%%%%%%%%%%% EISENBUD ENDE %%%%%%%%%%%%%%%%%%

%\usepackage{amsmath,amscd,amsthm,amssymb,amsxtra,latexsym,epsfig,epic,eepic,graphics}

\usepackage{amsmath,amscd,amsthm,amssymb,amsxtra,latexsym,epsfig,epic,graphics}

\usepackage[matrix,arrow,curve]{xy}
%


\def \fix#1 {{\hfill\break \bf (( #1 ))\hfill\break}}
\def \Box {\hfill\hbox{}\nobreak \vrule width 1.6mm height 1.6mm
depth 0mm  \par \goodbreak \smallskip}
\def \reg {\mathop{\rm regularity}}
\def \coker {\mathop{\rm coker}}
\def \ker {\mathop{\rm ker}}
\def \im {\mathop{\rm im}}
\def \deg  {\mathop{\rm deg}}
\def \depth {\mathop{\rm depth}}
\def \span {\mathop{\rm span}}
\def \socle {\mathop{\rm socle}}
\def \dim{{\rm dim}}
\def \codim{{\rm codim}}
\def \Im  {\mathop{\rm Im}}
\def \ann  {\mathop{\rm ann}}
\def \rank {\mathop{\rm rank}}
\def \sing {\mathop{\rm Sing}}
\def \iso {\cong}
\def \tensor {\otimes}
\def \dsum {\oplus}
\def \intersect {\cap}
\def \Hom {{\mathop{\rm Hom}\nolimits}}
\def \hom {{\mathop{\rm Hom}\nolimits}}
\def \Ext {{\rm Ext}}
\def \ext{{\rm Ext}}
\def \Tor {{\rm Tor}}
\def \tor{{\rm Tor}}
\def \Sym {{\mathop{\rm Sym}\nolimits}}
\def \End {{\mathop{\rm End}\nolimits}}
\def \sym{{\rm Sym}}
\def \h {{\rm h}}

\def \coh{{\rm Coh}}
%\def \BGG{{\rm BGG}}
\def \lin{{\rm lin}}

\def \th {{^{\rm th}}}
\def \st {{^{\rm st}}}


\def \A {{\cal A}}
\def \AA {{\bf A}}
\def \B {{\cal B}}
\def \C {{\cal C}}
\def \DD  {{\bf D}}
\def \E  {{\cal E}}
\def \F {{\cal F}}
\def \FF {{\bf F}}
\def \G {{\cal G}}
\def \GG {{\bf G}}
\def \K {{\cal K}}
\def \H {{\rm H}}
\def \KK {{\bf K}}
\def \L {{\cal L}}
\def \LL {{\bf L}}
\def \MM{{\bf M}}
\def \N {{\cal N}}
\def \O {{\cal O}}
\def \P {{\bf P}}
\def \RR {{\bf R}}
\def \TT {{\bf T}}
\def \Z {{\bf Z}}

\def \gm {{\goth m}}



%
%%%%%%%%%%%%%%%%%%%%%%%%%%%%
%%%The black board font
%%%%%%%%%%%%%%%%%%%%%%%%%%%
%\newcommand{\A}{{\mathbb A}}
%\newcommand{\B}{{\mathbb B}}
%\newcommand{\C}{{\mathbb C}}
%\newcommand{\D}{{\mathbb D}}
%\newcommand{\E}{{\mathbb E}}
%\newcommand{\F}{{\mathbb F}}
%\newcommand{\G}{{\mathbb G}}
%\newcommand{\H}{{\mathbb H}}
%\newcommand{\I}{{\mathbb I}}
%\newcommand{\J}{{\mathbb J}}
%\newcommand{\K}{{\mathbb K}}
%\newcommand{\L}{{\mathbb L}}
%\newcommand{\M}{{\mathbb M}}
%\newcommand{\N}{{\mathbb N}}
%\newcommand{{\mathbb O}}
%\newcommand{{\mathbb P}}
\newcommand{\QQ}{{\mathbb Q}}
%\newcommand{\R}{{\mathbb R}}
%\newcommand�{{\mathbb S}}
%\newcommand{\T}{{\mathbb T}}
%\newcommand{\U}{{\mathbb U}}
%\newcommand{\V}{{\mathbb V}}
%\newcommand{\W}{{\mathbb W}}
%\newcommand{\X}{{\mathbb X}}
%\newcommand{\Y}{{\mathbb Y}}
%\newcommand{\Z}{{\mathbb Z}}

%\newcommand{\LL}{{\mathbb L}}
%\newcommand{\TT}{{\mathbb T}}
%\newcommand{\HH}{{\mathbb H}}

\def\pp#1{{\text{$\text{I\!P}_#1$}}}
\newcommand{\PP}{{\mathbb P}}

%\DeclareMathOperator{\gin}{gin}

%\newcommand{\Syz}{{\rm{Syz}\;}}
%\newcommand{\sym}{{\rm{Sym}}}
%\newcommand{\SSyz}{{\rm{Syz}}}
%\newcommand{\spoly}{{\rm{spoly}}}
%\newcommand{\Spe}{{\rm{Sp}}}
%\newcommand{\openC}{{\mathbb C}}
%\newcommand{\ms}{{\rm{m}}}
%\newcommand{\LS}{{\rm{L}}}
%\newcommand{\IS}{{\rm{I}}}
%\newcommand{\Loc}{{\rm{Loc}\,}}
%\newcommand{\lcm}{{\rm{lcm}}}
%\newcommand{\lc}{{\rm{lc}}}
%\newcommand{\lm}{{\rm{lm}}}
%\newcommand{\con}{{\rm{c}}}
%\newcommand{\ext}{{\rm{e}}}
%\newcommand{\ec}{{\rm{ec}}}
%\newcommand{\ann}{{\rm{ann}}}
%\newcommand{\Ext}{{\rm{Ext}}}
%\newcommand{\equi}{{\rm{equi}}}
%\newcommand{\rk}{{\rm{rk}}}

%%%%%%%%%%%%%%%%%%%%%%%%%%%%%
%%% new commands for calligraphic characters with amsmath
%%%%%%%%%%%%%%%%%%%%%%%%%%%%

\newcommand{\ka}{{\mathcal A}}
\newcommand{\kb}{{\mathcal B}}
\newcommand{\kc}{{\mathcal C}}
\newcommand{\kd}{{\mathcal D}}
\newcommand{\ke}{{\mathcal E}}
\newcommand{\kf}{{\mathcal F}}
\newcommand{\kg}{{\mathcal G}}
\newcommand{\kh}{{\mathcal H}}
\newcommand{\ki}{{\mathcal I}}
\newcommand{\kj}{{\mathcal J}}
\newcommand{\kk}{{\mathcal K}}
\newcommand{\kl}{{\mathcal L}}
\newcommand{\km}{{\mathcal M}}
\newcommand{\kn}{{\mathcal N}}
\newcommand{\ko}{{\mathcal O}}
\newcommand{\kp}{{\mathcal P}}
\newcommand{\kq}{{\mathcal Q}}
\newcommand{\kr}{{\mathcal R}}
\newcommand{\ks}{{\mathcal S}}
\newcommand{\kt}{{\mathcal T}}
\newcommand{\ku}{{\mathcal U}}
\newcommand{\kv}{{\mathcal V}}
\newcommand{\kw}{{\mathcal W}}
\newcommand{\kx}{{\mathcal X}}
\newcommand{\ky}{{\mathcal Y}}
\newcommand{\kz}{{\mathcal Z}}
%%%%%%%%%%%%%%%%%%%%%%%%%%%%%%
%%%The mathscript for sheaves
%%%%%%%%%%%%%%%%%%%%% %%%%%%%%%
\newcommand{\s}{\mathscr}
\newcommand{\sA}{{\s A}}
\newcommand{\sB}{{\s B}}
\newcommand{\sC}{{\s C}}
\newcommand{\sD}{{\s D}}
\newcommand{\sE}{{\s E}}
\newcommand{\sF}{{\s F}}
\newcommand{\sG}{{\s G}}
\newcommand{\sH}{{\s H}}
\newcommand{\sI}{{\s I}}
\newcommand{\sJ}{{\s J}}
\newcommand{\sK}{{\s K}}
\newcommand{\sL}{{\s L}}
\newcommand{\sM}{{\s M}}
\newcommand{\sN}{{\s N}}
\newcommand{\sO}{{\s O}}
\newcommand{\sP}{{\s P}}
\newcommand{\sQ}{{\s Q}}
\newcommand{\sR}{{\s R}}
\newcommand{\sS}{{\s S}}
\newcommand{\sT}{{\s T}}
\newcommand{\sU}{{\s U}}
\newcommand{\sV}{{\s V}}
\newcommand{\sW}{{\s W}}
\newcommand{\sX}{{\s X}}
\newcommand{\sY}{{\s Y}}
\newcommand{\sZ}{{\s Z}}

\newcommand{\cO}{{\s O}}
\newcommand{\cI}{{\s I}}
\newcommand{\cL}{{\s L}}
\newcommand{\cR}{{\s R}}
\newcommand{\cN}{{\s N}}
\newcommand{\cT}{{\s T}}
\newcommand{\cX}{{\s X}}
%%%%%%%%%%%%%%%%%%%%%%%%%%%%%%%%
%% Arrows
%%%%%%%%%%%%%%%%%%%%%%%%%%%%%%%
\newcommand{\inj}{\hookrightarrow}
\newcommand{\surj}{\lra \lra}
\newcommand{\lra}{\longrightarrow}
\newcommand{\lla}{\longleftarrow}
%%%%%%%%%%%%%%%%%%%%%%%%%%%%%%%%%%%%
%\newcommand{\C}{\C}
\newcommand{\openP}�
\newcommand{\uf}{{\bf F}}
\newcommand{\uc}{{\bf C}}
%\newcommand{\tensor}{\otimes}
\newcommand{\mi}{{\bf m}}
\newcommand{\tX}{\widetilde{X}}
\newcommand{\punkt}{\hspace{-.3ex}\raise.15ex\hbox to1ex{\Huge.}}
\newcommand{\tpunkt}{\hspace{-.3ex}\hbox to1ex{\Huge.}}
\newlength{\br}
\newlength{\ho}
\newcommand{\GL}{GL} %\DeclareMathOperator{\GL}{GL}
%\DeclareMathOperator{\Aut}{Aut}
%\DeclareMathOperator{\Oo}{O}
%\DeclareMathOperator{\Spec}{Spec}
%\DeclareMathOperator{\Hom}{Hom}
%\DeclareMathOperator{\Tor}{Tor}
%\DeclareMathOperator{\syz}{syz}
%\DeclareMathOperator{\ord}{ord}
%\DeclareMathOperator{\word}{w\,ord}
%\DeclareMathOperator{\supp}{supp}
%\DeclareMathOperator{\Ker}{Ker}
%\DeclareMathOperator{\im}{im}
%\DeclareMathOperator{\wdeg}{w\,deg}
%\DeclareMathOperator{\depth}{depth}
%\DeclareMathOperator{\Coker}{Coker}
%\DeclareMathOperator{\NF}{NF}
%\DeclareMathOperator{\pd}{pd}
%\DeclareMathOperator{\SL}{SL}
%\DeclareMathOperator{\SO}{SO}
%\DeclareMathOperator{\Ort}{O}
%\DeclareMathOperator{\Spez}{Sp}
%\DeclareMathOperator{\PSL}{PSL}
%\DeclareMathOperator{\wdim}{wdim}
%\DeclareMathOperator{\cdim}{cdim}
%\DeclareMathOperator{\cha}{char}
%\DeclareMathOperator{\trdeg}{trdeg}
%\DeclareMathOperator{\codim}{codim}
%\DeclareMathOperator{\kdim}{kdim}
%\DeclareMathOperator{\height}{height}
%\DeclareMathOperator{\Ass}{Ass}
%\DeclareMathOperator{\Lie}{Lie}
\renewcommand{\labelenumi}{(\arabic{enumi})}
\newcommand{\Ndash}{\nobreakdash--}% for pages 1\Ndash 9
\newcommand{\somespace}{\hfill{}\\ \vspace{-0.7cm}}

%%%theosdefinitionen
%newcommand{\gm}{\mathfrak m}
%newcommand{\integer}{\Z}
%newcommand{\proj}�
%newcommand{\complex}{\C}
%newcommand{\real}{\mathbb R}
%newcommand{\gp}{\mathfrak p}
%newcommand{\gq}{\mathfrak q}
%newcommand{\scr}{\cal}
%newcommand{\openF}{\F}
\newcommand{\CC}{\mathbb C}
%newcommand{\ZZ}{\Z}
%newcommand{\QQ}{\Q}
%newcommand{\FF}{\F}

%%%%%%%%%%%%%%%BIBLIOGRAPHY
%newcommand{\by}{}
%newcommand{\paper}{: \begin{it}}
%newcommand{\jour }{, \end{it}}
%newcommand{\vol}{}
%newcommand{\pages}{}
%newcommand{\yr}{}
%\newcommand{\endref}{}
% Local Variables:
% mode: latex
% TeX-master: "tot"
% End:
%%%%%%%%%%%%%%%%%%%%%%%%%%%%%%%%%%%%%%%%%
%%     HOLGERs S�tze
%%%%%%%%%%%%%%%%%%%%%%%%%%%%%%%%%%%%%%%%%




%%%%%%%%%%%%%%%%%%%%%%%%%%%%%%%%%%%%%%%%%
\textwidth14cm
\textheight22cm
\oddsidemargin.8cm
\evensidemargin1cm
%%



%%%%% Titel &&&&&&&&&&&
\title{Sheaf algorithms using the exterior algebra}
\author{David Eisenbud\\ 1000 Centiennial Drive\\Mathematical Sciences Research Institute\\
        Berkeley, CA 94720\\ USA 
        \and 
        Wolfram Decker\\ FR Mathematik\\ Universit\"at des Saarlandes\\
        66041 Saarbr\"ucken\\ Germany}
\date{\today}



\begin{document}
%
%\setcounter{page}{1}
%

\maketitle


\noindent
\section{Introduction}

%\textbf{EISENBUD}\\


In this paper $W$ denotes a vector space of finite dimension $v$ over
a field $K$, and $S=\sym_K(W)$ is the symmetric algebra of $W$,
isomorphic to the polynomial ring on a basis for $W$.
We write $V=W^*$ for the vector space dual of $W$
and $E$ for the exterior algebra on $V$. We grade
$S$ and $E$ by taking elements of $W$ to have
degree 1, and elements of $V$ to have degree $-1$.
We denote the projective space of 1-quotients of $W$
(or of lines in $V$) by $\P(W)$, and write $n=v-1$ for its
dimension..

Serre's sheafification functor $M\rightarrow \tilde M$ 
allows to consider a coherent sheaf on $\mathbb P(W)$ as a
finitely generated graded $S$-module where we identify two such modules
$M$ amd $M'$ if for some $r$ the truncated modules $M_{\geq r}$ and
$M'_{\geq r}$ are isomorphic. If in this way $\mathcal F=\tilde M$ is 
associated to a finitely generated graded $S$-module $M$, then the sheafified 
minimal free resolution of $M$ is a representation of $\mathcal F$ in terms 
of direct sums of line bundles and homomorphisms between these bundles, that 
is, in terms of graded matrices over $S$. Being able to compute syzygies over
$S$ one can compute the higher cohomology of $\mathcal F$ 
starting from the minimal free resolution of $M$ (see Eisenbud [1998]).


The Bernstein-Gel'fand-Gel'fand correspondence (BGG) is an isomorphism
between the derived category of bounded complexes of finitely
generated $S$-modules and the derived category of 
bounded complexes of finitely generated $E$-modules
or of certain ``Tate resolutions'' of $E$-modules.


In this paper we show how to use BGG
to give practical computations of the cohomology of coherent sheaves
on the projective space $\P(W)$, and of the Beilinson monad of a sheaf
(sometimes referred to as the Beilinson spectral sequence), which
provides a powerful method of getting information about a sheaf from
its cohomology.

The following recipe is typical of our methods: Suppose that
$\F=\tilde M$ is the coherent sheaf on $\P(W)$ associated to a
finitely generated graded module $M=\oplus M_i$ over $S$. To
compute the cohomology of $\F$ we consider the sequence 
of free $E$-modules and maps
$$
\FF(M):\quad \cdots \rTo F^{j-1}\rTo^{\phi_{j-1}} 
F^j\rTo^{\phi_{j}} 
F^{j+1}\rTo\cdots.
$$
Here $\phi_j:F^j\rTo F^{j+1}$ is defined to be the
map taking a generator $m\in M_{j}\subset F^j$ to
$$
\sum_i x_im\otimes e_i\in M_{j+1}\otimes V\subset F^{j+1},
$$
where $\{x_i\}$ and $\{e_i\}$ are dual bases of $W$ and $V$
respectively. It turns out that $\FF(M)$
is a complex; that is, $\phi_j\phi_{j-1}=0$ for
every $j$ (the reader may easily check this by direct
computation. A proof without indices is given in
Eisenbud and Schreyer [2000]
If we regard $M_j$ as a vector space concentrated
in degree $j$, so that $F^j$ is a direct sum of copies of $E(-j)$,
then these maps are homogeneous of degree 0.

Let $s$ be a large integer (anything greater than the
Castelnuovo-Mumford regularity of $M$ will do) and let $P_s=\ker
\phi_{s+1}$; thus $P$ is a finitely generated graded $E$-module.  
It is relatively easy to compute a free resolution of
$P_s$ over $E$, simply because the number of monomials in $E$ in any
given degree is small compared to the number of monomials of that
degree in the symmetric algebra. We can thus compute the graded vector
spaces $\Tor^E_t(P_s,K)$. Our algorithm exploits the fact, proved in
Eisenbud-Schreyer [2000] that the $i^\th$ cohomology $\H^i\F$ of $\F$ 
in the Zariski topology is isomorphic to the degree $-v$ part of 
$\Tor^E_{s-i}(P_s,K)$; that is,
$$
\H^i\F\cong\Tor^E_{s-i}(P_s,K)_{-v}.
$$



In many cases this is the fastest way of computing cohomology known. For our
practical computation of Beilinson monads along the same lines, however,
we have no really striking application yet. Whereas we start with information
on the sheaf in order to construct the monad, in the typical applications
of Beilinson's theorem people construct or classify monads in order
to  construct or classify sheaves (we will give some examples of how
this works). Let us be bold and compare this with the situation in the beginning
1980's when it became clear that syzygies can be computed by a machine.
Though syzygies had been used theoretically for many years it took quite
a while until the practical computation of syzygies lead to applications, too.\\



%\textbf{EISENBUD ENDE}\\

\noindent
\section{Coherent sheaves and cohomology}


This section contains some background material on coherent sheaves
$\mathcal F$ on $\PP (W)$ and their graded cohomology modules
$$
H^j_*\mathcal F := \bigoplus\limits_{i\in\mathbb Z} H^j (\mathbb
P (W), \mathcal F(i))
$$
over $S$. From a cohomological point of view line bundles (locally free sheaves of rank 1)
are the simplest sheaves we can think of. Just a little bit more complicated are the bundles 
of differentials. Whereas homomorphisms between line bundles correspond to
homogeneuos e\-le\-ments in $S$, homomorphisms between twisted bundles of differentials
$\Omega_{\mathbb P (W)}^i(i)$ are given by homogeneuos elements in $E$.

Our starting point is Serre's sheafification functor $M\rightarrow \tilde M$ 
which induces an equivalence of categories between the category of finitely generated 
graded $S$-modules $M$ modulo the subcategory of modules of finite length and the category 
of coherent sheaves $\mathcal F$ on $\mathbb P(W)$. The inverse functor is induced by 
associating to $\mathcal F$ its module of sections $H^0_*\mathcal F$. More
precisely, since $H^0_*\mathcal F$ need not be finitely generated, 
we consider instead the  truncated modules $H^0_{\geq r}\mathcal F$. These
truncated modules are finitely generated. Indeed, if $\mathcal F =\tilde M$,
and if $L\rightarrow M \rightarrow 0$ is a free presentation of $M$, then
sheafification yields an exact sequence
$$
0 \rightarrow \mathcal K \rightarrow \tilde L \rightarrow \mathcal F
\rightarrow 0
$$
with kernel $\mathcal K$. Our claim follows by taking cohomology since 
$H^1 \mathcal K (i) = 0$ for $i >> 0$ by Serre's theorem A.

Let us next recall how line bundles are characterized in terms of cohomo\-lo\-gy. 
First of all a  theorem of Serre [1955] asserts that $\mathcal F$ is 
a vector bundle if and only if the  cohomoloy modules $H^i_*\mathcal F$, 
$0\leq i \leq n-1$, are finitely generated. In other words, by Serre's theorem A,
$\mathcal F$ is a vector bundle if and only if 
$H^0_* ({\mathcal F})$ is finitely generated and if the intermediate cohomology 
modules $H^i_*\mathcal F$, $1\leq i \leq n-1$, are of finite length.
Every vector bundle on the projective line splits into a direct sum of line bundles
by Grothendieck's splitting theorem (see Okonek et al. [1980]). Induction yields Horrock's splitting theorem 
(see Barth and Hulek [1978]): a vector bundle on $\mathbb P (W)$ splits into a direct sum of line bundles 
if and only if its intermediate cohomology vanishes. (This theorem was originally proved as a 
corollary to a more general result, see Horrocks [1964] and Walter [1996].)

Finally we review the cohomology of bundles of differentials.
Recall that the twisted cotangent bundle $U:=\Omega_{\mathbb P (W)}(1)$ fits as
the tautological subbundle of $W \otimes \mathcal O$ into the exact 
sequence
$$
0\rightarrow U \rightarrow W \otimes \mathcal O \rightarrow 
\mathcal O (1) \rightarrow 0\ .
$$
We set
$$
U^i:= \wedge^i U = \wedge^i (\Omega_{\mathbb P (W)})(1)) = \Omega^i(i)\ .
$$
Taking exterior powers gives the exact sequences
$$
0\rightarrow U^{i+1} \rightarrow \wedge^{i+1}W \otimes \mathcal O \rightarrow 
U^i\otimes \mathcal O (1) \rightarrow 0\ ,
$$
which, twisted by $-i-1$, glue to  the exact sequence
$$
\xymatrix@1@C=6mm{
0  \ar[r] & {\wedge^{v}_{\vbox to 2mm{}}}
W\otimes \mathcal O(-n-1) \ar[r] &\,\dots\;  \ar[r]& {\wedge^{0}_{\vbox to 2mm{}}}
W\otimes \mathcal O \ar[r]& 0\ .
}
$$
This sequence is the sheafification of the  Koszul complex resolving $K$ which we
consider as a graded $S$-module sitting in degree 0. It follows that
$$
H^j_*U^i =
\begin{cases}
K& j=i,\\
0& j\ne i,
\end{cases}
\qquad 1\leq i,j\leq n-1\ .
$$
Conversely, every vector bundle $\mathcal F$ on $\PP(W)$ with this intermediate cohomology
is stably equivalent to $U^i$; that is, there exists a direct sum  $\mathcal L$  of line bundles
such that $\mathcal F\cong U^i \oplus\mathcal L.$
This follows by comparing the sheafified Koszul complex with the minimal free resolution
of the dual bundle $\mathcal F^*$. More generally this argument shows that
a vector bundle $\mathcal F$ with just one non-vanishing intermediate cohomology module
$H^j_* \mathcal F$ is stably equivalent to the $j$th syzygy bundle of 
$H^j_* \mathcal F$.


For our purposes it is important to understand the homomorphisms bet\-ween the bundles
$U^i$ and $U^j$. From the tautological subbundle sequence we see that
$\Hom(U, \mathcal O) \cong \Hom(W \otimes \mathcal O, \mathcal O) \cong V$
where an element $\rho\in V$ corresponds to the
homomorphism $U \rightarrow  W\otimes \mathcal O 
\overset{\rho\otimes{\text id}}\longrightarrow \mathcal O$. By associating to $\rho\in V$
the homomorphism $U^i \rightarrow U \otimes U^{i-1} \overset{\rho\otimes
{\text id}}\longrightarrow  U^{i-1}$ one obtains in fact an isomorphism
$\Hom (U^i, U^{i-1})\cong V$ (see Eisenbud and Schreyer [2000]). Similarly 
and more generally,
$$
\Hom (U^i, U^j)\cong
\bigwedge^{i-j} V\,,\qquad 0\leq i,j\leq n\ ,
$$
where an element $\rho\in \bigwedge^{i-j} V$ corresponds to the
homomorphism $U^i \rightarrow U^{i-j} \otimes U^j \overset{\rho\otimes
{\text id}}\longrightarrow \otimes U^j$.\\


\noindent
%{\bf 3. Basics of BGG}\\

%\textbf{EISENBUD}\\


\section{Basics of the Bernstein-Gel'fand-Gel'fand correspondence}

In this section we describe the basic idea of the BGG correspondence,
introduced in Bernstein, Gel'fand, and Gel'fand [1978].
For a more complete treatment along the lines given here, see the first
section of Eisenbud and Schreyer [2000].

As a simple example of the construction given in
%%% Referenz
Section~1, consider the case $M=S$. The associated 
complex is then
$$
\FF(S):\quad E\rTo W\otimes E \rTo \Sym_2(W)\otimes E\rTo\cdots,
$$
where we regard $\Sym_iW$ as concentrated in degree $i$.
It is easy to see that the kernel of the first map,
$E\rTo W\otimes E$, is exactly the socle $\wedge^v V\subset E$,
which is a 1-dimensional vector space concentrated in degree $-v$.
In fact $\FF(S)$ is the minimal injective resolution of
this vector space.  If we tensor with the dual vector space
$\wedge^v W$ (which is concentrated in degree $v$),
we obtain the minimal injective resolution of
$\wedge^v W\otimes \wedge^v V$,
which may be identified canonically
with the residue field $K$ of $E$.
This resolution is called the {\it Cartan resolution\/}
of $K$.  To write it conveniently, we
set $\omega_E=\wedge^vW\otimes E$. The socle of
$\omega_E$ is $K$. Since $E$ is
injective (as well as projective) as an $E$-module, 
the same goes for $\omega_E$, so $\omega_E$ is the
injective envelope of the residue class field $K$ and
we have $\omega_E=\Hom_K(E,K)$. Thus we
can write the injective resolution of the residue field as
$$
\RR(S):\quad \omega_E\rTo W\otimes \omega_E \rTo 
\Sym_2(W)\otimes \omega_E \rTo\cdots,
$$
or again as
$$
\Hom_K(E,K)\rTo \Hom_K(E,W) \rTo 
\Hom_K(E,\Sym_2(W)) \rTo\cdots.
$$

Taking our cue from this situation, it is
natural to hope that the complex 
$$
\RR(M):\quad \cdots \rTo M_i\otimes \omega_E
\rTo M_{i+1}\otimes\omega_E\rTo \cdots
$$ 
will have more natural grading than $\FF(M)$;
in any case, it differs from $\FF(M)$
only by tensoring over $K$ with the one-dimensional
$K$-vector space $\wedge^vW$, concentrated in degree $v$, and
thus has the same basic properties. 
(Writing $\RR(M)$ it in terms of 
$\Hom$ as above suggests that the functor $\RR$
might have a left adjoint, and
indeed there is a left adjoint that produces linear free
complexes over $S$ from graded $E$-modules. $\RR$ and its
left adjoint is used to construct the isomorphisms of 
derived categories in the BGG correspondence; see
Eisenbud and Schreyer [2000] for a treatment in this spirit.)

The following Macaulay2 programs take as input a
linear presentation
matrix for a graded module $M$
defined over some polynomial ring $S=k[x_1,\dots,x_v]$ with
variables $x_i$ if degree 1,
and the name of an exterior algebra $E$ with the same number of
generators, also supposed to be of degree 1.
They returns a matrix representing the map
$M_{0}\otimes\omega_E\to M_{1}\otimes\omega_E$
which is the first map of the complex $\RR(M)$. The adjustment
of degrees in the second half of the programs is necessitated
by the unnatural convention that $E$ is generated in degree 1
instead of $-1$. The first of the two programs,
symExt, is a quick-and-dirty tool which requires little
computation. If it is called on two successive truncations
of a module the maps it produces may NOT compose to zero
because the choice of bases is not consistent.
The second, bgg, makes the computation in such a way
that the bases are consistent, but does much more computation
to achieve this end.


\medskip\hrule\smallskip
\fix{insert formatted code for symExt and bgg here}
\hrule\medskip

An important fact for us is that the complex $\RR(M)$ is eventually 
exact (and thus a high truncation 
$$
F^i\rTo^\phi_i F^{i+1}\rTo\cdots
$$
is the minimal injective resolution of $\ker \phi_i$.
It turns out that the point at which exactness sets in is
a well-known invariant, the Castelnuovo-Mumford regularity of
$M$, whose definition we briefly recall:

If $M=\oplus M_i$ is a finitely generated graded $S$-module
then for all large integers $r$
the submodule $M_{\geq r}\subset M$ is generated in degree $r$ and has
a {\it linear free resolution\/}; that is, its first syzygies are
generated in degree $r+1$, its second syzygies in degree $r+2$,
etc. (see Eisenbud [1995] chapter 20). 
We define the {\it Castelnuovo-Mumford regularity\/} of $M$ to be
the least integer $r$ for which this occurs. 

\vskip0.3cm
\noindent
\textbf{Theorem 3.1.}
%\theorem{regularity} 
{\em 
Let $M$ be a finitely generated graded
$S$-module of Castelnuovo-Mumford regularity $r$.
The complex $\RR(M)$ is exact at $\Hom_K(E,M_i)$
if and only if $i>r$.
}
\vskip0.3cm

For a proof see Eisenbud and Schreyer [2000].
More generally, they show that
the components of the cohomology of the complex $\RR(M)$ 
can be identified with the Koszul cohomology of $M$.
An equivalent result was stated in Buchweitz [1985].

For instance, it is not hard to show that if 
$M$ is of finite length, then the regularity of $M$
is the largest $i$ such that $M_i\neq 0$.
A simple example is given as follows:
Let $S=k[x,y,z]$, and let
$M=S/(x^2,y^2,z^2)$. The module
$M_{\geq 3}= K\cdot xyz$ is a trivial $S$-module, and
its resolution is the Koszul complex. which is linear.
Thus the Castelnuovo-Mumford regularity of $M$
is $\leq 3$. On the other hand $M_{\geq 2}$ is, up to twist,
isomorphic to the dual of $S/(x,y,z)^2$, and it follows that
the resolution of $M_{\geq 2}$ has the form
$$
0\rTo S(-6)\rTo S^6(-4)\rTo S^8(-3)\rTo S^3(-2),
$$
which is not linear, so the Castelnuovo-Mumford regularity of
$M$ is exactly 3. Note that the regularity is larger than the degrees
of the generators and relations of $M$---in general it
can be much larger.

Over $E$ the linear free complex 
corresponding to $M$ has the form
$$
\cdots 0\to
M_0\otimes\omega_E
\to 
M_1\otimes\omega_E
\to 
M_2\otimes\omega_E
\to
M_3\otimes\omega_E
\to 0
\cdots,
$$
where all the terms not shown are 0. Using the isomorphism
$\omega_E\cong E(-3)$ this
can be written (noncanonically) as
$$
0\rTo E(-3)
\rTo^{ \begin{pmatrix} x_1\\ x_2\\ x_3 \end{pmatrix}} 
E(-2)^3
\rTo^{\begin{pmatrix} 0&x_3&x_2\\ x_3&0&x_1\\ x_2&x_1&0 \end{pmatrix}} 
E(-1)^3
\rTo^{\begin{pmatrix}x_1& x_2& x_3  \end{pmatrix}} 
E\rTo 0.
$$
One checks easily that
this complex is inexact at  every nonzero term,
%%% Referenz
verifying Theorem~3.1.%%%%%%%\ref{regularity}.

For another example take $M=S/I$ where $I$ is generated
by a codimension 1 space of linear forms in $W$---that is,
$I$ is the homogeneous ideal of a point $p\in \P(W)$.
The free resolution of $M$ is 
the Koszul complex on $n=v-1$ linear forms,
so $M$ is 0-regular. As $M_i$
is 1-dimensional for every $i$ the complex
$\RR(M)$ takes the form
$$
\RR(M):\quad \omega\rTo^\rho \omega(-1)\rTo^\rho\omega(-2)\rTo^\rho\cdots
$$
where $\rho\in V$ is a linear functional that vanishes on all
the linear forms in $I$. 
As for any linear form in $E$, the annihilator of $\rho$ 
is generated by $\rho$, and it follows directly that the complex
$\RR(M)$ is acyclic in this case.



%\textbf{EISENBUD ENDE}\\

\noindent
%{\bf 4. Cohomology and Tate}\\



%\textbf{EISENBUD}\\



\section{Cohomology and Tate resolution of a sheaf}

Given a finitely generated graded $S$-module 
$M$ we construct a (doubly infinite)
$E$-free complex $\TT(M)$ with vanishing homology
called the {\it Tate resolution} of $M$, as follows:
Let $r$ be the Castelnuovo-Mumford regularity of $M$. The
truncation $\TT^{>r}(M)$, the part of $\TT(M)$ with cohomological
degree $>r$, is $\RR(M_{>r})$. We complete this to an exact
complex by adjoing a minimal free resolution of the kernel
of $\Hom_K(E,M_{r+1})\to\Hom_K(E,M_{r+2})$. 
Using the Macaulay2 program at the end of the last section
one can compute any finite piece of the Tate resolution;
we give the code below after explaining the significance
of what we are computing.

Thus for example if $M$ has finite length, the Tate 
resolution of $M$ is the complex
$$
\cdots 0\to 0\to 0 \cdots.
$$
At the opposite extreme, take $M=S$, the free module of rank 1.
We know that $\RR(S)$ is an injective resolution
of the residue field $K$ of $E$. Applying the exact functor
$\Hom_K(\hbox{\bf---}, K)$, and using the fact that it carries
$\omega_E=\Hom_K(E,K)$ back to $E$, we see that the 
Tate resolution $\TT(S)$ is the first row of the diagram

$$
\xymatrix{
\cdots \ar[r]& W^*\otimes E \ar[r]& E\ar[dr] \ar[rr]&& \omega_E
\ar[r]& W\otimes\omega_E  \ar[r]&\cdots\\
&&&K\ar[ur]
}
$$      
A further simple example occurs in the case where $M$ is
the homogeneous coordinate ring of a point $p\in\P(W)$. The complex
$\RR(M)$ constructed in the last section is periodic, so
it may be simply continued to the left, giving
$$
\TT(M):\quad \cdots \rTo^\rho \omega_E(i)
\rTo^\rho\omega_E(i-1)\rTo^\rho\cdots,
$$
where again $\rho\in V=W$ is a nonzero linear functional
vanishing on the linear forms in the ideal of $p$.

By the results of the previous section
$\RR(M_{>r})$ has no homology in cohomological degree $>r+1$,
and thus $\TT(M)$ could be constructed
by a similar recipe from any truncation 
$\RR(M_{>s})$ with $s\geq r$. Further, for
large $i$ we have $M_i=\H^0\tilde M(i)$. Putting these
things together we see that
the Tate resolution  depends only on the sheaf on $\P(W)$
corresponding to $M$, which we write as $\tilde M$.
We sometimes write $\TT(M)$ as $\TT(\tilde M)$ to
emphasize this point.

For large $i$ 
the term of the complex 
$\TT(\tilde M)$ with cohomological degree $i$ is 
$M_i\otimes \omega_E=\H^0(\tilde M(i))\otimes \omega_E$. 
The following result generalizes 
this to a description of all the terms of the Tate resolution,
and gives the formula for the cohomology described in the
introduction.

\vskip0.3cm
\noindent
\textbf{Theorem 4.1.}
{\em 
 Let $M$ be a finitely generated graded $S$-module,
and let $\F$ be the corresponding sheaf on $\P(W)$.
The term of the complex $\TT(M)=\TT(\F)$ with cohomological degree $i$ is 
$$
\oplus_j \H^j\tilde M(i-j)\otimes \omega_E
$$
where $\H^j\tilde M(i-j)$ is regarded as a vector space
concentrated in degree $i-j$, so that the summand
$\H^j\tilde M(i-j)\otimes \omega_E$ is isomorphic to a direct sum
of copies of $\omega_E(j-i)$.
Moreover the subquotient complex 
$$
\cdots \to \H^j\tilde M(i-j)\otimes \omega_E
\to
\H^j\tilde M(i+1-j)\otimes \omega_E
\to \cdots
$$
is 
$(\RR(\H^j_*(M)(-j)))(j)$ (up to twists and shifts it is
$(\RR(\H^j_*(M))).$ )
}

\vskip0.3cm
Thus each cohomology group of each twist of the sheaf
$\tilde M$ occurs (exactly once) as the generators of 
a term of $\TT(M)$. When we compute a part of the $\TT(M)$,
we are computing the sheaf cohomology of various twists
of the associated sheaf together with maps
which describe the $S$-module structure of $\oplus_i\H^j\tilde M(i)$
for each $j$, and also some higher cohomology operations, which
we understand only in very special cases (see Eisenbud and Schreyer
[2001]). Here is a program which makes this computation. In 
many cases we have tested, it is the fastest way we know to
compute the cohomology of a given sheaf:\\

\medskip\hrule\smallskip
\fix{insert formatted code for sheafCohomology here}
\smallskip\hrule\medskip


The Tate resolutions of sheaves are, as the reader may easily check,
precisely the doubly infinite, graded, exact complexes of finitely-generated
free $E$-modules which are ``eventually linear'' on the right,
in an obvious sense. What about other doubly exact graded free
complexes? For example what if we take the dual of the Tate
resolution of a sheaf? In general it will not be eventually linear.
What is it?

To explain this we must generalize the construction of $\RR(M)$:
if 
$$
M^\bullet:\quad \cdots M^{i+1}\rTo M^i\rTo M^{i-1}\rTo\cdots
$$
is a complex of $S$-modules, then applying the functor
$\RR$ we get a complex of free complexes over $E$. Changing
some signs we get a double complex, and we may
form the total complex of the associated double complex.
In general this complex is not minimal; but at least if $M^\bullet$
is a bounded complex then,
just as one produces the unique minimal free
resolution of a module from any free resolution,
we can construct a unique minimal
complex from it. We call this minimal complex $\RR(M^\bullet)$.
(See Eisenbud and Schreyer [2000] for more information. This
construction is a necessary part of interpreting the BGG correspondence
as an equivalence of derived categories.) 

Again
if $M^\bullet$ is a bounded complex of finitely generated
modules, then as before
one shows
that $\RR(M^\bullet)$ is exact from a certain point on, and
so we can form the Tate resolution $\TT(M^\bullet)$ by adjoining
a free resolution of a kernel. Once again, the Tate resolution 
depends only on the bounded complex of coherent sheaves
$\F^\bullet$ associated
to $M^\bullet$, and we write $\TT(\F^\bullet)=\TT(M^\bullet)$

A variant of the theorem of Bernstein, Gel'fand and Gel'fand
shows that every minimal graded doubly infinite exact sequence
of finitely generated free $E$-modules is of the form $\TT(\F^\bullet)$
for some complex of coherent sheaves $\F^\bullet$, unique up
to quasi-isomorphism. The terms of the Tate resolution can
be expressed using hypercohomology
by a formula like that of Theorem~4.1.%%\ref{tate}.
%%%%%%%%%%%%% REFERENZ

One way that interesting complexes of sheaves arise is through
duality. For simplicity,
write $\O$ for the structure sheaf $\O_{\P(W)}$.
If $\F$ is a sheaf on $\P(W)$ then 
the derived functor $RHom(\F, \O)$
may be computed by applying the functor $Hom(\hbox{\bf ---},\O)$
to a the complex of sheaves made from a free resolution of 
a module whose associated sheaf is $\F$; it's value is thus
a complex of sheaves rather than an individual sheaf.

We can now identify the dual of the Tate resolution:\\


\noindent
\textbf{Theorem 4.2.}
{\em 
%\theorem{duality} 
$\Hom_K(\TT(\F), K) \iso \TT(RH{\rm om}(\F, \O))[1]$.
}

Here the $[1]$ denotes a shift by one in cohomological degree.
For example, take $\F=\O$. We have $RH{\rm om}(\O, \O)=\O$. The
Tate resolution is given by
$$\diagram
\TT(\O)\quad &\cdots&\rTo &E&\rTo &\omega_E&\rTo&\cdots\\
             &      &     &-1&    &0
\enddiagram
$$
where the number under each term is its the cohomological degree.
Taking into account $\omega_E=\Hom_K(E,K)$,
the dual of the Tate resolution is thus
$$\diagram
\Hom_K(\TT(\O),K):\quad &\cdots&\lTo &\omega_E&\lTo &E&\lTo&\cdots\\
                        &      &     &1       &     &0
\enddiagram
$$
which is the same as 
$\TT(\O))[1]$.
A completely analogous computation gives the
proof of Theorem~4.2 %\ref{duality} 
if $\F=\O(a)$ for some $a$, and the general case follows
by taking free resolutions.


%\textbf{EISENBUD ENDE}\\


\noindent
\section{Cohomology and monads}




This section is an interlude on monads. The technique of monads is an
important tool for problems such as the construction and classification of 
coherent sheaves with prescribed invariants. We demonstrate the usefulness
of monads, which is not obvious at first glance, by reviewing the
classification of stable rank-2 vector bundles on the projective plane
(see Barth [1977], Le Potier [1979], and Hulek [1979]). 
The reason for considering stable bundles is that these bundles admit
moduli (see Gieseker [1977], and Maruyama [1977, 1978]). 

The basic idea behind monads is to represent arbitrary coherent sheaves
in terms of direct sums of simpler sheaves such as line
bundles or bundles of differentials, and in terms of homomorphisms
between these simpler sheaves. The minimal free resolution of a coherent
sheaf $\mathcal F$ associated to a finitely generated, graded $S$-module is
a monad for $\mathcal F$ which involves direct sums of line bundles
and thus graded matrices over $S$. The Beilinson monad for $\mathcal F$, which 
will be considered in the next section, involves direct sums of twisted bundles 
of differentials, and thus graded matrices over $E$.\\

\noindent
{\bf Definition 5.1.}\; {\em A} monad {\em on $\mathbb P (W)$ is a complex
$$
\dots\; \rightarrow {\mathcal K}^{-1} \rightarrow {\mathcal K}^{0}
\rightarrow {\mathcal K}^{1} \rightarrow \; \dots
$$
of coherent sheaves on $\mathbb P (W)$ which is exact except  at
${\mathcal K}^{0}$. The homology $\mathcal F$ at ${\mathcal K}^{0}$ is called the
homology of the monad, and we say that the complex is a monad
for $\mathcal F$.}\\

For the classification of a given class of sheaves via monads one typically 
proceeds in three steps.
\begin{itemize}
\item[(1)] Use cohomological information to determine the shapes of 
the monads, that is, to determine the sheaves ${\mathcal K}^{i}$.
\item[(2)] Classify  the maps of the monads.
\item[(3)] Determine which monads lead to isomorphic sheaves.
\end{itemize}
\noindent

One of the first succesful applications of this approach was the classification
of (Gieseker-)stable rank-2 vector bundles with even first Chern class $c_1$ on 
the projective 
plane $\PP^2(\CC)$ by Barth [1977]. The same ideas apply in the case  $c_1$ odd
(see Le Potier [1979], and Hulek [1979]).\\

\noindent
{{{\bf Remark 5.2.} Let $\mathcal F$ be a rank-2 vector bundle on $\PP (W)$ with
first Chern class $c_1$. Then the following holds (see Okonek et al. [1980]).
\vskip0.1cm
\noindent
(1)\; $\mathcal F$ admits a symplectic structure.
Indeed, the cano\-nical map $\mathcal F \otimes \mathcal F 
\rightarrow \bigwedge^2 \mathcal F \cong \mathcal O (c_1)$ induces an isomorphism
$\varphi: \mathcal F \overset{\cong}\to \mathcal F^{\ast} (c_1)$ 
with $\varphi = - \varphi^* (c_1)$. In particular we have canonical isomorphisms 
$$
H^j(\PP (W), \mathcal F (i))^*\cong H^{n-j}(\PP (W), \mathcal F (-i-n-1-c_1)).
$$
by Serre duality.
\vskip0.1cm
\noindent
(2) \; $\mathcal F$ is stable iff $\Hom (\mathcal F, \mathcal F)\cong K$.
In this case the symplectic structure is uniquely determined up to scalars.
\vskip0.1cm
\noindent
(3)\; By tensoring with a line bundle we can for our purposes assume that 
$c_1 \in \{0,-1\}$. In this case $\mathcal F$ is stable iff it has no 
global sections. \quad\qed\\ 

\noindent
{{{\bf Remark 5.3.} By the  theorem of Riemann-Roch there exists a polynomial in 
$\QQ [c_1, c_2]$ which gives the Euler characteristic for every rank-2 vector bundle
$\mathcal F$ on $\PP (W)$ with Chern classes $c_1$ and $c_2$. This polynomial can be 
determined by computing the Euler characteristic for enough bundles of type 
$\mathcal O(a) \oplus \mathcal O(b)$. For the projective plane, for example, we
obtain $\chi (\mathcal F)=(c_1^2-2c_2+3c_1+4)/2.$\quad\qed\\ 

Let us now quickly go through the three steps above in the case of stable rank-2 vector 
bundles $\mathcal F$ on  $\PP^2(\CC)$ with $c_1 = -1$ (see Le Potier [1979] and Hulek
[1979] for details and proofs).

\vskip0.3cm
\noindent
{\bf\underline{Step 1.}}\; Our goal is to show that $\mathcal F$ is the homology of a 
monad of type
$$
0\rightarrow H^1 \mathcal F (-2)\otimes\mathcal O(-1) \rightarrow H^1 \mathcal F (-1)
\otimes U \rightarrow H^1 \mathcal F \otimes \mathcal O \rightarrow 0\ .
$$
This monad is actually the Beilinson monad whose construction will be presented
in the next chapter. We derive it here directly with Horrocks' technique of cohomology
(see Horrocks [1980]). Let us first compute some of the dimensions
$h^j \mathcal F (i) := \dim H^j \mathcal F (i)$. By stability and Serre duality
$h^0 \mathcal F(-i) = h^2 \mathcal F (i-2) = 0$ for all 
$i\geq 0$. Moreover $\chi (\mathcal F (i)) = (i+1)^2 -c_2$ by the theorem of
Riemann-Roch. Thus the dimensions $h^j \mathcal F (i)$ 
in range $-2 \leq i \leq 0$ are as follows (a zero is represented by an empty box):
\vspace{1cm}
%%%%%%%%%%%%%%%%%%%%%%%%%%%%%%%%%%%%%%%%%%%%%%%%%%%%%%%%%%%%%%%%%%%%%%%%%%%%%%%%%%

{
$$ %Diagramm zentrieren
%
% W�hle die Einheiten \br horizontal und \ho vertikal
{
\setlength{\br}{10mm}
\setlength{\ho}{8mm}
%\fontsize{10pt}{11pt}
%\selectfont
\begin{xy}
%%%%%%%%%%%%%%%%%%%%%%%%%%%%%%%%%%%%%%%
%
% Achsenkreuz im Punkte (0,0)
%
% x-Achse von -5\br bis +1\br
% mit einem "m" an 0.95 der L�nge und 3mm unter der Achse:
%
,<-3\br,0\ho>;<1\br,0\ho>**@{-}?>*@{>}
?(0.95)*!/^3mm/{i}
%
% y-Achse von 0\ho bis 5\ho
% mit einem "i" an 0,9 der L�nge und 3mm rechts neben der Achse:
%
,<0\br,0\ho>;<0\br,4\ho>**@{-}?>*@{>}
?(0.9)*!/^3mm/{j}
%%%%%%%%%%%%%%%%%%%%%%%%%%%%%%%%%%%%%%%
%
% 4 waagrechte Linien von -5\br bis +0\br
% in den H�hen 1\ho,...,5\ho:
%
,0+<-3\br,1\ho>;<0\br,1\ho>**@{-}
,0+<-3\br,2\ho>;<0\br,2\ho>**@{-}
,0+<-3\br,3\ho>;<0\br,3\ho>**@{-}
%,0+<-5\br,4\ho>;<0\br,4\ho>**@{-}
%%%%%%%%%%%%%%%%%%%%%%%%%%%%%%%%%%%%%%%
%
% 5 senkrechte Linien von 0\ho bis 4\ho
% in den waagrechten Punkten -5\br,...,-1\br:
%
%,0+<-5\br,0\ho>;<-5\br,3\ho>**@{-}
%,0+<-4\br,0\ho>;<-4\br,3\ho>**@{-}
,0+<-3\br,0\ho>;<-3\br,3\ho>**@{-}
,0+<-2\br,0\ho>;<-2\br,3\ho>**@{-}
,0+<-1\br,0\ho>;<-1\br,3\ho>**@{-}
%
%%%%%%%%%%%%%%%%%%%%%%%%%%%%%%%%%%%%%%%
%
% Eintr�ge in den Mitten der K�sten. Daher die Koordinaten mit .5
%
,0+<-2.5\br,1.5\ho>*{c_2-1}
%
,0+<-1.5\br,1.5\ho>*{c_2}
%
,0+<-0.5\br,1.5\ho>*{c_2-1}
%
%%%%%%%%%%%%%%%%%%%%%%%%%%%%%%%%%%%%%%%
\end{xy}
}
$$
}
%%%%%%%%%%%%%%%%%%%%%%%%%%%%%%%%%%%%%%%%%%%%%%%%%%%%%%%%%%%%%%%%%%%%%%%%%%%%%%%
\vskip0.2cm
\noindent
In particular we see that stability implies $c_2\geq 1$. Further cohomological 
information is typically obtained by restricting the given bundles to linear subspaces. 
In our case we consider the Koszul complex of a point $P\in \PP^2(\CC)$:
$$
0 \rightarrow \mathcal O (-2) \overset{\begin{pmatrix} -x^{\prime}\\ x\end{pmatrix}}
\longrightarrow 2 \mathcal O (-1) \overset{\begin{pmatrix} x & x^{\prime}\end{pmatrix}}
\longrightarrow \mathcal O   \rightarrow \mathcal O_P  \rightarrow 0\ .
$$
By tensoring with $\mathcal F (i)$ and taking cohomology we find that $H^1_{\geq 0} \mathcal F$ 
is generated by $H^1 \mathcal F$. In particular, $\mathcal F\cong U$ in case $c_2=1$ by
what has been said on syzygy bundles in section 2. Indeed,
both bundles have the same rank and intermediate cohomology.
If $c_2\geq 2$ then the identy in $\Ext^1(H^1 \mathcal F \otimes \mathcal O, \mathcal F)
\cong (H^1 \mathcal F)^* \otimes H^1 \mathcal F$ defines an extension
$$
0\rightarrow \mathcal F \rightarrow \mathcal G \rightarrow H^1 \mathcal F
\otimes \mathcal O\rightarrow 0 
$$
with $H^1_{\geq 0}\mathcal E = 0$. Similarly we obtain an extension
$$
0\rightarrow H^1 \mathcal F (-2)\otimes\mathcal O(-1) \rightarrow \mathcal H
\rightarrow  \mathcal F\rightarrow 0 
$$
with $H^{1}_{\leq -2} \mathcal H = 0$. The two extensions fit into a commutative
diagram with exact rows and and columns
%%%%%%%%%%%%%%%%%%%%%%%%%%%%%%%%%%%%%%%%%%%%%%%%%%%%%
$$
\xymatrix{
&&0\ar[d]&0\ar[d]\\
0\ar[r]& H^1 \mathcal F (-2)\otimes\mathcal O(-1)\ar@{}[d]|{\scriptscriptstyle ||}\ar[r]&
\mathcal  H\ar[d]\ar[r]&
\mathcal F\ar[r]\ar[d]& 0\\
0\ar[r]&  H^1 \mathcal F (-2)\otimes\mathcal O(-1)\ar[r]& \mathcal B \ar[d]\ar[r]&
\mathcal G\ar[d]\ar[r]& 0\\
&& H^1 \mathcal F\otimes \mathcal O \ar[d]\ar@{}[r]|{=}&H^1 \mathcal F \otimes \mathcal O \ar[d]\\
&&0&0
}
$$
%%%%%%%%%%%%%%%%%%%%%%%%%%%%%%%%%%%%%%%%%%%%%%%%%%%%
since, for example, the extension in the top row lifts uniquely to an extension 
as in the middle row. Then $\mathcal B \cong H^1 \mathcal F (-1)\otimes U$ 
since by construction these bundles  have the same rank and intermediate cohomology.
What we got is actually the {\it display} of (the short exact sequences associated to)
a 3-term monad 
$$
0\rightarrow H^1 \mathcal F (-2)\otimes\mathcal O(-1) \rightarrow H^1 \mathcal F (-1)
\otimes U \rightarrow H^1 \mathcal F \otimes \mathcal O \rightarrow 0
$$
for $\mathcal F$. 

\vskip0.3cm
\noindent
{\bf\underline{Step 2.}}\;
Let us abbreviate $A:=H^1 \mathcal F (-2)$, $B:=H^1 \mathcal F (-1)$ and 
$A^*\cong H^1 \mathcal F$. By chasing the displays of the monads it follows that
 the symplectic structure on $\mathcal F$ lifts to a unique isomorphism of monads
$$
\xymatrix{
0\ar[r]& A\otimes \mathcal O(-1)\ar[d]^{\Phi}\ar[r]^{\alpha}&
B \otimes U \ar[d]^{\Psi}\ar[r]^{\beta}&
A^*\otimes\mathcal O \ar[r] \ar[d]^{-\Phi^*(-1)} & 0\\
0\ar[r]&A\otimes \mathcal O(-1) \ar[r]^{{\beta^*}(-1)} & B^* \otimes U^*(-1) 
\ar[r]^{\quad\;{\alpha^*}(-1)}& A^*\otimes \mathcal O \ar[r]& 0
}
$$
with $\Psi = -\Psi^*(-1)$ since the corresponding obstruction groups vanish
(compare Barth and Hulek [1978]). Since $U$ is stable $\Hom(U,U^*(-1))\cong\CC$
by Remark 5.2. So $\Psi$ is the tensor product of an isomorphism $q: B\rightarrow
B^*$ with the symplectic structure $\iota$ on $U$. Note that $q$ is symmetric since
$-(q\otimes\iota) = (q\otimes\iota)^*(-1)=q^*\otimes\iota^*(-1)=-q^*\otimes\iota$.
We may and will now assume that $\mathcal F$ is the cohomology of a self-dual monad, that is, 
that $\beta = \alpha^t:=\alpha^* (-1)\circ(q\otimes\iota)$. The monad conditions
\begin{itemize}
\item[($\alpha_1$)] $\alpha^t\circ\alpha = 0$, and
\item[($\alpha_2$)] $\alpha$ is a vector bundle monomorphism (equivalently,
$\alpha^t$ is surjective)
\end{itemize}
can be equivalently expressed in terms of linear algebra as follows. Consider 
$$\alpha\in\Hom(A\otimes \mathcal O(-1), B \otimes U) \cong V \otimes A^*\otimes B$$
as a homomorphism $\alpha: W \rightarrow \Hom (A,B)$. Then
\begin{itemize}
\item[($\alpha_1^{\prime}$)] $\alpha^t(x) \circ \alpha(x^{\prime})=
\alpha^t(x^{\prime}) \circ\alpha(x)$ for all $x, x^{\prime}\in W$, and
\item[($\alpha_2^{\prime}$)] for every $\xi\in A\setminus \{0\}$
the map $W \rightarrow B$, $x \to \alpha(x)(\xi)$ has rank $\geq 2$.
\end{itemize}

\noindent
{\bf Example 5.4.} If $c_2=2$ then the monads are given by two vectors $a, b \in V$:
$$
0 \rightarrow \mathcal O (-1) \overset{\begin{pmatrix} a\\ b\end{pmatrix}}\longrightarrow 2 U 
\overset{\begin{pmatrix} a & b\end{pmatrix}}\longrightarrow \mathcal O \rightarrow  0\ .
$$
In this case ($\alpha_1$) gives no condition and 
($\alpha_2$) means that $a$ and $b$ are linearly independent.\quad\qed\\

Our construction can now easily be reversed. Let $A$ and $B$ be vector spaces of 
the appropriate dimensions, let $q$ be a non-degenerate quadratic form on $B$,
and let ${\text{O}}(B)$ be the corresponding orthogonal group.
We set
$$
\tilde M�=\{\alpha\in\Hom(W, \Hom(A, B)) \mid \alpha {\text { satisfies }} 
(\alpha_1^{\prime}) {\text { and }} (\alpha_2^{\prime})\}\ .
$$
Then every $\alpha\in \tilde M$ defines a monad as above whose homology is
a stable rank-2 vector bundle on $\PP^2(\CC)$ with Chern classes $c_1=-1$
and $c_2$. Our description does not really give the maps in the monads
explicitly (with the exception of the case $c_2=2$). It is, however, enough for 
detecting geometric properties of the corresponding moduli spaces.

\vskip0.3cm
\noindent
{\bf\underline{Step 3.}}\;
We consider the group $G:= \GL (A)\times {\text{O}}(B)$. Elements of $G$ 
define isomorphisms of self-dual monads as above and this action on the monads
induces the canonical action $((\Phi, \Psi), \alpha)\to \Psi\alpha\Phi^{-1}$
of $G$ on $\tilde M$. By going back and force between ismorpisms of bundles
and isomorphisms of monads as above one shows that the  stabilizer of $G$ in 
each point is $\{\pm 1\}$, and that our construction induces a bijection
between the isomorphism classes of stable rank-2 vector bundle on $\PP^2(\CC)$ 
with Chern classes $c_1=-1$ and $c_2$ onto $M:=\tilde M/G_0$. Here
$G_0:=G/\{\pm 1\}$. With the help of a a universal monad over $\tilde M$ one proofs 
that the analytic structure on $\tilde M$ descends to an analytic structure on
$M$ so that $M$ is smooth of dimension $h^1 \mathcal F^* \otimes \mathcal F
=4c-4$ in each point (the obstructions for smoothness in the point corresponding to
$\mathcal F$ lie in $H^2 \mathcal F^* \otimes \mathcal F = 0$). 
Moreover, the universal family over $\tilde M$ defined by the universal monad descends to 
a universal family over $M$ (here we need $c_1=-1$. In other words, $M$ is a fine moduli 
space for our bundles. Further efforts show that $M$ is irreducible and rational.\\

\noindent
\section{The Beilinson  monad}

%\textbf{EISENBUD}\\


%\section{beilinson} The Beilinson Monad

We can use the Tate resolution associated to a sheaf to 
give a construction of a complex first
described in Beilinson [1978], which gives a 
powerful method for deriving information about a sheaf from
information about a few of its cohomology groups.
The general idea is the following:

Suppose that $\A$ is an additive category and consider
a graded object
$
\oplus_{i=0}^vU^i
$
in $\A$.
Given a graded ring homomorphism $E\to \End_\A(\oplus_{i=0}^vU^i)$
we can make an additive functor from the category of 
free $E$-modules to $\A$:
On objects we take
$$
\omega_E(i)\mapsto{ \begin{cases} U^i & \text{for } 0\leq i\leq v \text{ and}; \\
                     0 & \text{otherwise.}\end{cases}}
$$
To define the functor on maps, we use 
$$
\Hom_E(\omega_E(i),\omega_E(j))=
\Hom_E(E(i),E(j))=
E_{j-i} %\wedge^{i-j}V
\rTo\End(\oplus U^i)_{j-i}\rTo \Hom(U^i,U^j)
$$
(Note that we could have taken any twist of $E$ in place of 
$\omega_E\cong E(-v)$; the choice of $\omega_E$ is made to 
simplify the statement of Theorem~6.1, below.)
%%%%%%%%% Referenz \ref{Beilinson theorem}, below.)
We will apply this functor
to complexes of free $E$-modules; specifically, to Tate resolutions
of sheaves on $\P(W)$.

We shall be interested in a special case
where $\A$ is the category of coherent sheaves on
$\P(W)$ and where $U^i = \Omega^i (i)$ as in section~2.


(Further examples
may be obtained by taking $U^i$ to be the $i^\th$ exterior
power of the tautological subbundle $U_k$ on the Grassmannian
of $k$-planes in $W$ for any $k$; the case we have taken
here is the case $k=n$. See Eisenbud and Schreyer [2001]
for more information on the general case
and applications to the computation of
resultants and more general Chow forms.)

Applying the functor just defined to the Tate resolution $\TT(\F)$
of a coherent sheaf $\F$ on $\P(W)$, 
and using Theorem~4.1, we get 
%%%%%%%%%% Referenz %\ref{tate}, we get 
a complex
$$
\Omega(\F):\quad
\cdots\rTo \oplus_j \H^j\F(i-j)\otimes U^{j-i}\rTo\dots ,
$$
where the term we have written down occurs in cohomological
degree $i$. We will call $\Omega(\F)$ a {\it Beilinson complex\/} 
for $\F$.
The resolution $\TT(\F)$ is well-defined up to homotopy, 
so the same is true of $\Omega(\F)$.

Of course $U^{k}=0$ unless
$0\leq k\leq n$, where $n=v-1$ is the rank of $U$ (equal to the 
dimension of $\P(W)$). Thus
the only cohomology groups of $\F$ that
are actually involved in $\Omega(\F)$ are $\H^j\F(k)$ with
$-n\leq k\leq 0$.

To see a simple example, consider the structure sheaf $\O_P$ 
of the subvariety consisting of a point
$p\in \P(W)$. Write $I$ for the homogeneous ideal of $P$,
and again let $\rho\in V=W^*$ be a nonzero functional vanishing
on the linear forms in $I$.
The Tate resolution of the homogeneous coordinate ring $S/I$
has already been computed, and we have seen that it depends only
on the sheaf $\tilde{S/I}=\O_P$. From the computation of
$\TT(S/I)=\TT(\O_P)$ made in the last section we see that 
$\Omega(\O_P)$ takes the form
$$
\Omega(\O_P):\quad 0\to U^n\rTo^\rho U^{n-1}\rTo^\rho
\cdots \rTo^\rho U^1\rTo^\rho U^0\rTo 0,
$$
with $U^i$ in cohomological degree $i$. Note that $U^0=\O_{\P(W)}$.

We have already noted
that the map 
$\rho: U=U^1 \rTo U^0=\O_{\P(W)}$ is the composite of the
tautological embedding $U\subset W\otimes \O_{\P(W)}$ with the
map $\rho\otimes 1:\ W\otimes \O_{\P(W)} \to \O_{\P(W)}$.
Thus the image of $\rho: U^1\to \O_{\P(W)}$ is the ideal
sheaf of $p$, and we see that the homology of the complex
$\Omega(\O_p)$ at $U^0$ is $\O_p$. One can check further
that $\Omega(\O_p)$ is the Koszul complex associated with
the map $\rho: U^1\to \O_{\P(W)}$, and it follows that 
the homology of $\Omega(\O_p)$ at $U^i$ is 0 for $i>0$.
The following result 
from Eisenbud and Schreyer [2000] shows that this is typical:

%\theorem{Beilinson theorem} 

\vskip0.3cm
\noindent
\textbf{Theorem 6.1}
{\em 
If $\F$ is a coherent sheaf on
$\P(W)$, then the only nonvanishing homology of  the
complex $\Omega(\F)$ is 
$$
\H^0(\Omega(\F))=\F.
$$
}

The existence of a complex satisfying the theorem and having the same
terms as $\Omega(\F)$ was first asserted by Beilinson in [1978],
and thus we will call $\Omega(\F)$ a 
{\it Beilinson complex\/} for $\F$. Existence proofs
via a somewhat less effective construction than the one given here
may be found in Kapranov [1988] and Ancona and Ottaviani [1989].

\medskip\hrule\smallskip
\fix{add code, further examples}
\hrule\medskip


%\textbf{EISENBUD ENDE}\\


\section*{References}

\noindent\textbf{Ancona, V. \& Ottaviani, G.}:
        {\sl An introduction to derived categories and the theorem of Beilinson},
        Atti Accademia Peloritana dei Pericolanti, Classe I de Scienze 
        Fis. Mat. et Nat. LXVII, 99-110 (1989) \\

\noindent\textbf{Barth, W.}: 
        {\sl Moduli of vector bundles on the projective plane},
        Invent. math. {\bf 42}, 63-91 (1977)\\


\noindent\textbf{Barth, W. \& Hulek, K.}: 
        {\sl Monads and Moduli of Vector Bundles}, 
        manuscripta math. {\bf 25}, 323-347 (1978)\\

\noindent\textbf{Beilinson, A}:
        {\sl Coherent sheaves on $\P^n$ and problems of linear algebra},
        Funct. Anal. and its Appl. {\bf 12}, 214-216 (1978)\\
        
\noindent\textbf{Bernstein, Gel'fand, and Gel'fand}:
        {\sl Algebraic bundles on $\P^n$ and problems of linear algebra},
        Funct. Anal. and its Appl. {\bf 12}, 212-214 (1978)\\

\noindent\textbf{Buchweitz, R.-O.}:
         Appendix to Cohen-Macaulay modules on quadrics, by
        R.-O. Buchweitz, D. Eisenbud, and J. Herzog. In
        {\sl Singularities, representation of algebras,
        and vector bundles} (Lambrecht, 1985),
        Springer-Verlag Lecture Notes in Math, 1273, 96-116 (1987)\\
        

\noindent\textbf{Eisenbud,  David}: 
        {\sl Commutative Algebra with a View Toward Algebraic Geometry},
        Springer Verlag, 1995\\

\noindent\textbf{Eisenbud,  David}: 
        {\sl Computing cohomology} in Chapter of 
        ``Computational methods in Commutative Algebra and Algebraic Geometry'' 
        by ~W.~Vasconcelos, Springer Verlag, Berlin, 1998\\

\noindent\textbf{Eisenbud, D.  \&  Schreyer, F.-O.}: 
        {\sl Chow forms and free resolution}, in preparation 2001\\

\noindent\textbf{Eisenbud, D.  \&  Schreyer, F.-O.}: 
        {\sl Sheaf Cohomology ans Free Resolutions over Exterior Algebras},
        AG/0005055, 2000\\

\noindent\textbf{Gieseker, D.}:
        {\sl On the moduli of vector bundles on an algebraic surface},
        Ann. of Math. {\bf 106}, 45-60 (1977)\\

\noindent\textbf{Hulek, Klaus}:
        {\sl Stable Rank-2 Vector Bundles on $\P_2$ with $c_1$ odd},
        Math. Ann. {\bf 242}, 241-266 (1979)\\

\noindent\textbf{Horrocks, G.}: 
        {\sl Vector bundles on the punctured spectrum of a local ring},
        Proc. London Math. Soc. (3), {\bf 14}, 689-713 (1964)\\

\noindent\textbf{Horrocks, G.}: 
        {\sl Construction of bundles on $\PP^n$} in ``Les equations de Yang-Mills'',
        by A. Douady, J.-L. Verdier (eds.), Asterisque {\bf 71-72}, 197-202 (1980)\\

        
\noindent\textbf{Kapranov, M. M.}:
        {\sl On the derived categories of coherent sheaves on some 
        homogeneous spaces}, Invent. Math. {\bf 92}, 479-508 (1988)\\

\noindent\textbf{Le Potier, J.}:
        {\sl Fibr\'es stables de rang 2 sur $\P_2 (\CC)$},
        Math. Ann. {\bf 241}, 217-256 (1979)\\

\noindent\textbf{Maruyama, M.}:
        {\sl Moduli of stable sheaves I},
        J. Math. Kyoto Univ. {\bf 17}, 91-126 (1977)\\


\noindent\textbf{Maruyama, M.}:
        {\sl Moduli of stable sheaves II},
         J. Math. Kyoto Univ. {\bf 18}, 557-614 (1978)\\


\noindent\textbf{Okonek, C., Schneider, M. \& Spindler, H.}:
         {\sl Vector bundles on complex projective spaces}, Boston, 1980\\


\noindent\textbf{Serre, J.P.}: 
        {\sl Faisceaux alg\'ebriques coherents},
        Ann. of Math. {\bf 61}, 197-278  (1955)\\

\noindent\textbf{Walter, Charles H.}:
         {\sl Pfaffian subschemes}, 
        J. Algebr. Geom. {\bf 5}, 671-704 (1996)\\





\end{document}

