\title{Author instructions}
\author{Daniel R. Grayson\thanks{Supported by the NSF}}
\authorrunning{Grayson}
\institute{University of Illinois at Urbana-Champaign, Department of Mathematics}
\maketitle

\def\lll{{\tt\char`\<\char`\<\char`\<}}
\def\rrr{{\tt\char`\>\char`\>\char`\>}}
\def\slash{\char`\\}

\begin{abstract}
The abstract should summarize the contents of the paper
using at least 70 and at most 150 words. It will be set in 9-point
font size and be inset 1.0 cm from the right and left margins.
There will be two blank lines before and after the Abstract.
\end{abstract}

\section*{Introduction}
Run TeX on the files {\tt lncse-doc.tex} and {\tt lncse-dem.tex} in the
directory {\tt lncse-macros} to produce the documentation and a demonstration
of the macro package {\tt lncse} provided to us by Springer.  It's almost
compatible with AMS-Latex and Latex-2e, but multiple authors are handled with
a single {\tt \slash author} macro, the macros {\tt \slash email} and {\tt
  \slash urladdr} are missing, the macro {\tt \slash address} is replaced by
{\tt \slash institute}, and the macro {\tt \slash thanks} should be inserted
where the corresponding footnote indicator is needed.  Another difference is
that the command {\tt \slash theoremstyle} does not exist, and the command
{\tt \slash newtheorem} has been modified to take a few extra arguments.

To create your own chapter, copy the entire directory {\tt TEMPLATE} to
create a directory for your own chapter, and edit the files there.  For
example, the command
\par\smallskip
\centerline{\tt
cp -a TEMPLATE foo
}
\smallskip\noindent
will copy the files to a new directory named {\tt foo}.  Then you should also
add a line $$\text{\tt \slash getChapter\char`\{foo/chapter\char`\}}$$ to the file
{\tt book/book.tex}, and a pair of dependency lines to the top of {\tt
book/Makefile}.

The file {\tt foo/chapter.tex} is the one you where you should insert your
text, after deleting the TeX code for these paragraphs.  It is not a complete
TeX file, because it gets eventually input into TeX by the file {\tt
book/book.tex} or the file {\tt foo/chapter-wrapper.tex}.  It is
intentionally missing the $$\text{\tt \slash documentclass\char`\{lncse\char`\}}$$ and
$$\text{\tt \slash begin\char`\{document\char`\}}$$ macros at the top, and the
$$\text{\tt \slash bibliography\char`\{papers\char`\}}$$ and $$\text{\tt \slash
end\char`\{document\char`\}}$$ macros at the bottom, so don't add them; these things have
to be done in a special way when the chapter is incorporated into the book.

If you want to change the name of the file {\tt foo/chapter.tex}, you should
edit {\tt book/Makefile}, {\tt book/book.tex}, the top line of {\tt
foo/Makefile}, and in the value of {\tt tex-main-file} at the bottom of {\tt
foo/chapter.tex}.

We include a file $$\text{\tt inputs/book-macros.tex},$$ which authors should not
modify, which provides macros and commands which can be used throughout the
book.  It loads the {\tt amsmath}, {\tt amscd}, {\tt amssymb}, {\tt
latexsym} packages.

We can produce an entry in the index for the whole book using the macro {\tt
\slash index}, illustrated here with \index{indexing} $$\text{\tt
\slash index\char`\{indexing\char`\}}.$$

Use the macro $${\tt \slash Mtwo}$$ to incorporate the name \Mtwo into
your text.  This ensures the name will be uniformly italicized.  Spaces after
the macro will be obeyed, and an italic correction will be inserted.

References to the literature \cite{MR47:3318} are created as usual with {\tt
\slash cite}.  Make sure you put the corresponding bibtex entries,
obtainable from MathSciNet, in the file {\tt papers.bib} in the same
directory as your chapter.  Bibtex entries that duplicate those in other
chapters will cause harmless warning messages.

The file {\tt Makefile} contains the information needed for GNU make to run
TeX and various auxiliary programs in the correct sequence.  To make sure you
have GNU make installed, execute the command {\tt make -v}.  GNU make will
identify itself with a first line like this:
\par\smallskip
\centerline{\tt
GNU Make version 3.77, by Richard Stallman and Roland McGrath.
}
\smallskip\noindent
You can type {\tt make} in your chapter's directory to process just your
chapter, or you can type {\tt make} in the top level directory to create the
whole book.  It runs TeX in scroll mode, so you can get many error messages
at once; you can change that if you'd like it to be more interactive.

If {\tt make} runs successfully, your chapter can be viewed in the file $$\text{\tt
foo/chapter-wrapper.dvi},$$ and the book can be viewed in the file $$\text{\tt
book/book.dvi}.$$

\section{Incorporating \Mtwo code in your chapter.}

To incorporate \Mtwo code into your text, surround each statement to be
executed with \lll{} and \rrr, and the result will be automatically run
through \Mtwo for you.
\beginOutput
i1 : A = QQ[x,y]\\
\emptyLine
o1 = A\\
\emptyLine
o1 : PolynomialRing\\
\endOutput
For example, the code above was inserted by the following line.
\par\smallskip
\centerline{\tt
\lll A = QQ[x,y]\rrr
}
\smallskip\noindent
A single statement can span multiple lines, but try to keep it short.
\beginOutput
i2 : N = cokernel random (\\
\             A^1, \\
\             A^\{-2,-2\}\\
\             )\\
\emptyLine
o2 = cokernel | 5/2xy+y2 -8/9xy-2/5y2 |\\
\emptyLine
\                            1\\
o2 : A-module, quotient of A\\
\endOutput
\beginOutput
i3 : res N\\
\emptyLine
\      1      2      1\\
o3 = A  <-- A  <-- A  <-- 0\\
\                           \\
\     0      1      2      3\\
\emptyLine
o3 : ChainComplex\\
\endOutput
If the output from \Mtwo is too wide, dots will appear indicating
where it was truncated. 
\beginOutput
i4 : 100!\\
\emptyLine
o4 = 93326215443944152681699238856266700490715968264381621468592963895 $\cdot\cdot\cdot$\\
\endOutput

Make sure you always have the most recent version of \Mtwo installed, for
you will be running other authors' code through it.

\section{Running \Mtwo in emacs}

Because some answers can be very wide, it is a good idea to run \Mtwo in
a window which does not wrap output lines and allows the user to scroll
horizontally to see the rest of the output.  We provide a package for {\tt
{}emacs} which implements this, in {\tt {}emacs/M2.el} (the directory {\tt
emacs} is in the \Mtwo distribution).  It also provides for dynamic
completion of symbols in the language.

There is an ASCII version of this section of the documentation distributed
in the file {\tt {}emacs/emacs.hlp}.  It might be useful for you to visit
that file with emacs now, thereby avoiding having to cut and paste bits of
text into emacs buffers for the demonstrations below.

If you are a newcomer to emacs, start up emacs with the command 
{\tt {}emacs} and then start up the emacs tutorial with the keystrokes 
{\tt {}C-H\ t}.  (The notation {\tt {}C-H} indicates that you should type 
{\tt {}Control-H}, by holding down the control key, 
and pressing {\tt {}H}.)  The emacs tutorial will introduce you to the
basic keystrokes useful with emacs.  After running through that you will want
to examine the online emacs manual which can be read with {\tt {}info}
mode; you may enter or re-enter that mode with the keystrokes {\tt {}C-H\ i}.  
You may also want to purchase (or print out) the emacs manual.  It is cheap,
comprehensive and informative.  Once you have spent an hour with the emacs
tutorial and manual, come back and continue from this point.

Edit your {\tt {}.emacs} initialization file, located in your home directory,
creating one if necessary.  (Under Windows, this file is called {\tt {}\_emacs}.)
Insert into it the following lines of emacs-lisp code.

\medskip

{\scriptsize\ttfamily\obeylines
(setq~auto-mode-alist~(append~auto-mode-alist~'(("{\tt\char`\\}{\tt\char`\\}.m2\$"~.~M2-mode))))
(autoload~'M2-mode~"M2-mode.el"~"Macaulay~2~editing~mode"~t)
(global-set-key~"{\tt\char`\\}{\char94}Cm"~'M2)~(global-set-key~[~f12~]~'M2)
(global-set-key~"{\tt\char`\\}{\char94}Cm"~'M2)~(global-set-key~[~SunF37~]~'M2)
(autoload~'M2~"M2.el"~"Run~Macaulay~2~in~a~buffer."~t)
(setq~load-path~(cons~"/usr/local/Macaulay2/emacs"~load-path))
(make-variable-buffer-local~'transient-mark-mode)
(add-hook~'M2-mode-hook~'(lambda~()~(setq~transient-mark-mode~t)))
(add-hook~'comint-M2-hook~'(lambda~()~(setq~transient-mark-mode~t)))
}

\smallskip

The first two lines cause emacs to enter a special mode for editing \Mtwo
code whenever a file whose name has the form {\tt {}*.m2} is encountered.  
The next three lines provide a special mode for running \Mtwo in an emacs buffer.
The sixth line tells emacs where to find the emacs-lisp files provided in the
\Mtwo emacs directory - you must edit the string in that line to
indicate the correct path on your system to the \Mtwo emacs directory.
The files needed from that directory are {\tt {}M2-mode.el},
{\tt {}M2-symbols.el}, and {\tt {}M2.el}.  The seventh line sets
the variable {\tt {}transient-mark-mode} so that it can
have a different value in each buffer.  The eighth and ninth lines set
hooks so that {\tt {}transient-mark-mode} will be set to {\tt {}t} 
in M2 buffers.  The effect of this is that the mark is only active occasionally,
and then emacs functions which act on a region of text will refuse to proceed 
unless the mark is active.  The {\tt {}set-mark} function or the
{\tt {}exchange-point-and-mark} function will activate the mark, and it
will remain active until some change occurs to the buffer.  The only reason
we recommend the use of this mode is so the same key can be used to evaluate 
a line or a region of code, depending on whether the region is active.

Exit and restart emacs with your new initialization file.  
If you are reading this file with emacs, then use the keystrokes
{\tt {}C-x\ 2} to divide the buffer containing this file into two windows.
Then press the {\tt {}F12} function key to start up 
\Mtwo in a buffer named {\tt {}*M2*}.

If this doesn't start up \Mtwo, one reason may be that your function
keys are not operable.  In that case press {\tt {}C-C\ m} instead.  (The 
notation {\tt {}C-C} is standard emacs notation for Control-C.)  Another
reason may be that you have not installed \Mtwo properly - the startup
script ({\tt {}M2} or {\tt {}M2.bat}) should be on your path.
A third reason may be that you are in Windows-98 and are using anti-virus 
software such as {\tt {}Dr.\ Solomon's}, which can interfere with emacs 
when it tries to run a subprocess.

You may use {\tt {}C-x\ o} freely to switch from one window to the other.
Verify that \Mtwo is running by entering a command such as {\tt {}2+2}.  
Now paste the following text into a buffer, unless you have the ASCII
version of this documentation in an emacs buffer already, position
the cursor on the first line of code, and press the {\tt {}F11} function 
key (or {\tt {}C-C\ s}) repeatedly to present each line to \Mtwo.

\smallskip

{\ttfamily\obeylines
i1~=~R~=~ZZ/101[x,y,z]
i2~=~f~=~symmetricPower(2,vars~R)
i3~=~M~=~cokernel~f
i4~=~C~=~resolution~M
i5~=~betti~C
}

\smallskip

Notice that the input prompts are not submitted to \Mtwo.

Here is a way to conduct a demo of \Mtwo in which the code to be
submitted is not visible on the screen.  Paste the following text into
an emacs buffer.

\smallskip

{\ttfamily\obeylines
~~~~20!
~~~~4~+~5~2{\char94}20
~~~~--~that's~all~folks!
}

\smallskip

Press {\tt {}M-F11} with your cursor in this buffer to designate it as
the source for the \Mtwo commands.  (The notation {\tt {}M-F11} means 
that while holding the {\tt {}Meta} key down, you should press the {\tt {}F11} 
key.  The Meta key is the Alt key on some keyboards, or it can be simulated by 
pressing Escape (just once) and following that with the key you wanted to press 
while the meta key was held down.)  Then position your cursor (and thus the 
emacs point) within the line containing {\tt {}20!}.  Now press {\tt {}M-F12}
to open up a new frame called {\tt {}DEMO} for the {\tt {}*M2*} window with
a large font suitable for use with a projector, and with your cursor in that
frame, press {\tt {}F11} a few times to conduct the demo.  (If the font or frame is the
wrong size, you may have to create a copy of the file {\tt {}M2.el}
with a version of the function {\tt {}M2-demo} modified to fit your screen.)

One press of {\tt {}F11} brings the next line of code forward into the
{\tt {}*M2*} buffer, and the next press executes it.  Use {\tt {}C-x\ 5\ 0} 
when you want the demo frame to go away.

There is a way to send a region of text to \Mtwo: simply select a region
of text, making sure the mark is active (as described above) and press {\tt {}F11}.
Try that on the list below; put it into an emacs buffer, move your cursor to the 
start of the list, press {\tt {}M-C-@} or {\tt {}M-C-space} to mark the list, 
and then press {\tt {}F11} to send it to \Mtwo.  (The notation {\tt {}M-C-@} 
means: while holding down the Meta key and the Control key press the {\tt {}@} key, 
for which you'll also need the shift key.)

\smallskip

{\ttfamily\obeylines
{\tt\char`\{}a,b,c,d,e,f,
g,h,i,j,k,l,
m,n{\tt\char`\}}
}

\smallskip

We have developed a system for incorporating \Mtwo interactions into TeX
files.  Here is an example of how that looks.  Paste the following text,
representing something you might see in a TeX file, into an emacs buffer.

\medskip

{\ttfamily\obeylines\scriptsize
~~~The~answer,~4,~is~displayed~after~the~output~label~``{\tt\char`\{}{\tt\char`\\}tt~o1{\tt\char`\\}~={\tt\char`\}}''.~~
~~~Multiplication~is~indicated~with~the~traditional~{\tt\char`\{}{\tt\char`\\}tt~*{\tt\char`\}}.
~~~{\tt \char`\<}{\tt\char`\<}{\tt\char`\<}1*2*3*4{\tt\char`\>}{\tt\char`\>}{\tt\char`\>}
~~~Powers~are~obtained~as~follows.
~~~{\tt\char`\<}{\tt\char`\<}{\tt\char`\<}2{\char94}100{\tt\char`\>}{\tt\char`\>}{\tt\char`\>}
}

\smallskip

The bits in brackets can be submitted to \Mtwo easily.  Position your
cursor at the top of the buffer and press {\tt {}F10.}  The cursor will move 
just past the first {\tt {<}<<}, and the emacs mark will be positioned just 
before the {\tt {}>>>}.  Thus {\tt {}1*2*3*4} is the region, and it will
even be highlighted if you have set the emacs variable {\tt {}transient-mark-mode}
to {\tt {}t} for this buffer.  Pressing {\tt {}F11} will send {\tt {}1*2*3*4} 
to \Mtwo for execution: try it now.  A sequence of such \Mtwo commands 
can be executed by alternately pressing {\tt {}F10} and {\tt {}F11}.  You may
also use {\tt {}M-F10} to move backward to the previous bracketed expression.

Now let's see how we can handle wide and tall \Mtwo output.  Execute the
following line of code.

\smallskip

{\ttfamily\obeylines
random(R{\char 94}20,R{\char94}{\tt\char`\{}6:-2{\tt\char`\}})
}

\smallskip

Notice that the long lines in the \Mtwo window, instead of being wrapped
around to the next line, simply disappear off the right side of the screen,
as indicated by the dollar signs in the rightmost column.  Switch to the
other window and practice scrolling up and down with {\tt {}M-v} and {\tt {}C-v}, 
and scrolling left and right with the function key {\tt {}F3} (or {\tt {}C-C\ <}) 
and the function key {\tt {}F4} (or {\tt {}C-C\ >}).  Notice how the use of
{\tt {}C-E} to go to the end of the line
sends the cursor to the dollar sign at the right hand side of the screen;
that's where the cursor will appear whenever you go to a position off the
screen to the right.  Then use the {\tt {}F2} function key (or {\tt {}C-C\ .}) to 
scroll the text so the cursor appears at the center of the screen.  Use {\tt {}C-A} to 
move to the beginning of the line and then the {\tt {}F2} function key 
(or {\tt {}C-C\ .}) to bring the left margin back into view.

You may use the {\tt {}F5} function key or (or {\tt {}C-C\ ?}) to 
toggle whether long lines are truncated or wrapped; initially they are truncated.

Now go to the very end of the {\tt {}*M2*} buffer with {\tt {}M->} and 
experiment with keyword completion.  Type {\tt {}reso} and then press the 
{\tt {}TAB} key.  Notice how the word is completed to {\tt {}resolution}
for you.  Delete the word with {\tt {}M-DEL}, type {\tt {}res}
and then press the {\tt {}TAB} key.  The possible completions are displayed 
in a window.  Switch to it with the {\tt {}F8} key, move to the desired 
completion, select it with the {\tt {}RETURN} key, and then return to the 
{\tt {}*M2*} buffer with {\tt {}C-X\ o}.  Alternatively, if you have a
mouse, use the middle button to select the desired completion.

Experiment with command line history in the {\tt {}*M2*} buffer.  Position 
your cursor at the end of the buffer, and then use {\tt {}M-p} and {\tt {}M-n} 
to move to the previous and next line of input remembered in the history.  When you 
get to one you'd like to run again, simply press return to do so.  Or edit it
slightly to change it before pressing return.

\section{Editing \Mtwo code with emacs}

In this section we learn how to use emacs to edit \Mtwo code.  Assuming you
have set up your emacs init file as described in {\tt {}running\ Macaulay\ 2\ in\ emacs}
when you visit a file whose name ends with {\tt {}.m2} 
you will see on the mode line the name {\tt {}Macaulay\ 2} in
parentheses, indicating that the file is being edited in \Mtwo mode.  (Make
sure that the file {\tt {}emacs/M2-mode.el} is on your {\tt {}load-path}.)

To see how electric parentheses, electric semicolons, and indentation work,
move to a blank line of this file and type the following text.

\smallskip

{\ttfamily\obeylines
f~=~()~-{\tt\char`\>}~(
~~~~~a~:=~4;
~~~~~b~:=~{\tt\char`\{}6,7{\tt\char`\}};
~~~~~a+b)
}

\smallskip

Observe carefully how matching left parentheses are indicated briefly when a
right parenthesis is typed.

Now position your cursor in between the 6 and 7.  Notice how
pressing {\tt {}M-C-u} moves you up out of the list to its left.  Do it 
again.  Experiment with {\tt {}M-C-f} and {\tt {}M-C-b} to move forward
and back over complete parenthesized
expressions.  (In the emacs manual a complete parenthesized expression is
referred to as an sexp, which is an abbreviation for S-expression.)  Try out
{\tt {}C-U\ 2\ M-C-@} as a way of marking the next two complete parenthesized
expression, and see how to use {\tt {}C-W} to kill them and {\tt {}C-Y} to yank 
them back.  Experiment with {\tt {}M-C-K} to kill the next complete parenthesized 
expression.

Position your cursor on the 4 and observe how {\tt {}M-;} will start a comment 
for you with two hyphens, and position the cursor at the point where commentary
may be entered.

Type {\tt {}res} somewhere and then press {\tt {}C-C\ TAB} to bring up the
possible completions of the word to documented \Mtwo symbols.

Finally, notice how {\tt {}C-H\ m} will display the keystrokes peculiar to 
the mode in a help window.

The text below is for testing purposes:

    \def\f{\edef\g{\the\lastskip}{\tt<----- \g}\par}

    a: \f
    a. \f
    a? \f
    a; \f

% Local Variables:
% mode: latex
% mode: reftex
% tex-main-file: "chapter-wrapper.tex"
% reftex-keep-temporary-buffers: t
% reftex-use-external-file-finders: t
% reftex-external-file-finders: (("tex" . "make FILE=%f find-tex") ("bib" . "make FILE=%f find-bib"))
% End:
