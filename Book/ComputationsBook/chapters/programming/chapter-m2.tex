\title{Data Types, Functions, and Programming}
\titlerunning{Data Types, Functions, and Programming}
\toctitle{Data Types, Functions, and Programming}
\author{Daniel R. Grayson%
        \thanks{Supported by NSF grant DMS 99-70085.}
        % \inst 1
   \and Michael E. Stillman%
        %\inst 2
        %\fnmsep
        \thanks{Supported by NSF grant 99-70348.}}
\authorrunning{D. R. Grayson and M. E. Stillman}
% \institute{University of Illinois at Urbana-Champaign, Department of
%   Mathematics, Urbana, IL 61801, USA
%       \and Cornell University, Department of Mathematics, Ithaca, NY 14853, USA}

\maketitle

\begin{abstract}
  In this chapter we present an introduction to the structure of \Mtwo
  commands and the writing of functions in the \Mtwo language.  For further details
  see the \Mtwo manual distributed with the program~\cite{M2}.
\end{abstract}

\section{Basic Data Types}

The basic \ie{data types} of \Mtwo include numbers of various types (integers,
rational numbers, floating point numbers, complex numbers), lists (basic
lists, and three types of visible lists, depending on the delimiter used),
hash tables, strings of characters (both 1-dimensional and 2-dimensional),
Boolean values (true and false), symbols, and functions.  Higher level types
useful in mathematics are derived from these basic types using facilities
provided in the \Mtwo language.  Except for the simplest types (integers and
Boolean values), \Mtwo normally displays the type of the output value on a
second labeled output line.

Symbols\index{symbols} have a name which consists of letters, digits, or apostrophes, the
first of which is a letter.  Values can be assigned to symbols and recalled
later.
\beginOutput
i1 : w\\
\emptyLine
o1 = w\\
\emptyLine
o1 : Symbol\\
\endOutput
\beginOutput
i2 : w = 2^100\\
\emptyLine
o2 = 1267650600228229401496703205376\\
\endOutput
\beginOutput
i3 : w\\
\emptyLine
o3 = 1267650600228229401496703205376\\
\endOutput
Multiple values can be assigned in parallel.
\beginOutput
i4 : (w,w') = (33,44)\\
\emptyLine
o4 = (33, 44)\\
\emptyLine
o4 : Sequence\\
\endOutput
\beginOutput
i5 : w\\
\emptyLine
o5 = 33\\
\endOutput
\beginOutput
i6 : w'\\
\emptyLine
o6 = 44\\
\endOutput
Comments are initiated by {\tt --} and extend to the end of the line.
\beginOutput
i7 : (w,w') = (33,   -- this is a comment\\
\               44)\\
\emptyLine
o7 = (33, 44)\\
\emptyLine
o7 : Sequence\\
\endOutput
Strings\index{strings} of characters are delimited by quotation marks.
\beginOutput
i8 : w = "abcdefghij"\\
\emptyLine
o8 = abcdefghij\\
\endOutput
They may be joined horizontally to make longer strings, or vertically to make
a two-dimensional version called a {\sl net}.
\beginOutput
i9 : w | w\\
\emptyLine
o9 = abcdefghijabcdefghij\\
\endOutput
\beginOutput
i10 : w || w\\
\emptyLine
o10 = abcdefghij\\
\      abcdefghij\\
\endOutput
Nets are used in the preparation of two dimensional output for polynomials.

Floating point numbers are distinguished from integers by the presence of a
decimal point, and rational numbers are entered as fractions.
\beginOutput
i11 : 2^100\\
\emptyLine
o11 = 1267650600228229401496703205376\\
\endOutput
\beginOutput
i12 : 2.^100\\
\emptyLine
o12 = 1.26765 10^30\\
\emptyLine
o12 : RR\\
\endOutput
\beginOutput
i13 : (36 + 1/8)^6\\
\emptyLine
\      582622237229761\\
o13 = ---------------\\
\           262144\\
\emptyLine
o13 : QQ\\
\endOutput
Parentheses, braces, and brackets are used as delimiters for the three types
of {\em \ie{visible lists}}: \ie{lists}, \ie{sequences}, and \ie{arrays}.
\beginOutput
i14 : x1 = \{1,a\}\\
\emptyLine
o14 = \{1, a\}\\
\emptyLine
o14 : List\\
\endOutput
\beginOutput
i15 : x2 = (2,b)\\
\emptyLine
o15 = (2, b)\\
\emptyLine
o15 : Sequence\\
\endOutput
\beginOutput
i16 : x3 = [3,c,d,e]\\
\emptyLine
o16 = [3, c, d, e]\\
\emptyLine
o16 : Array\\
\endOutput
Even though they use braces, lists should not be confused with sets, which will be treated later.
A double period can be used to construct a sequence of consecutive elements in various contexts.
\beginOutput
i17 : 1 .. 6\\
\emptyLine
o17 = (1, 2, 3, 4, 5, 6)\\
\emptyLine
o17 : Sequence\\
\endOutput
\beginOutput
i18 : a .. f\\
\emptyLine
o18 = (a, b, c, d, e, f)\\
\emptyLine
o18 : Sequence\\
\endOutput
Lists can be nested.
\beginOutput
i19 : xx = \{x1,x2,x3\}\\
\emptyLine
o19 = \{\{1, a\}, (2, b), [3, c, d, e]\}\\
\emptyLine
o19 : List\\
\endOutput
The number of entries in a list is provided by {\tt \#}.
\beginOutput
i20 : #xx\\
\emptyLine
o20 = 3\\
\endOutput
The entries in a list are numbered starting with 0, and can be recovered with
{\tt \#} used as a binary operator.
\beginOutput
i21 : xx#0\\
\emptyLine
o21 = \{1, a\}\\
\emptyLine
o21 : List\\
\endOutput
\beginOutput
i22 : xx#0#1\\
\emptyLine
o22 = a\\
\emptyLine
o22 : Symbol\\
\endOutput
We can join visible lists and {\sl append} or {\sl prepend} an element to a visible
list.  The output will be the same type of visible list that was provided in
the input: a list, a sequence, or an array; if the arguments are various
types of lists, the output will be same type as the first argument. 
\beginOutput
i23 : join(x1,x2,x3)\\
\emptyLine
o23 = \{1, a, 2, b, 3, c, d, e\}\\
\emptyLine
o23 : List\\
\endOutput
\index{lists!joining}\indexcmd{join}%
\beginOutput
i24 : append(x3,f)\\
\emptyLine
o24 = [3, c, d, e, f]\\
\emptyLine
o24 : Array\\
\endOutput
\index{lists!appending}\indexcmd{append}%
\beginOutput
i25 : prepend(f,x3)\\
\emptyLine
o25 = [f, 3, c, d, e]\\
\emptyLine
o25 : Array\\
\endOutput
\index{lists!prepending}\indexcmd{prepend}%
Use {\tt sum} or {\tt product} to produce the sum or product of all the
elements in a list.
\beginOutput
i26 : sum \{1,2,3,4\}\\
\emptyLine
o26 = 10\\
\endOutput
\indexcmd{sum}%
\beginOutput
i27 : product \{1,2,3,4\}\\
\emptyLine
o27 = 24\\
\endOutput
\indexcmd{product}%


\section{Control Structures}

Commands for later execution are encapsulated in {\sl \ie{functions}}.  A function
is created using the operator {\tt -\char`\>}\indexcmd{->} to separate the parameter or
sequence of parameters from the code to be executed later.  Let's try an
elementary example of a function with two arguments.
\beginOutput
i28 : f = (x,y) -> 1000 * x + y\\
\emptyLine
o28 = f\\
\emptyLine
o28 : Function\\
\endOutput
The parameters {\tt x} and {\tt y} are symbols that will acquire a
value later when the function is executed.  They are {\sl local}\index{variables!local} in the sense that
they are completely different from any symbols with the same name that occur
elsewhere.  Additional local variables for use within the body of a function
can be created by assigning a value to them with {\tt \char`\:\char`\=}
\indexcmd{:=}
(first time only).  We illustrate this by rewriting the function above.
\beginOutput
i29 : f = (x,y) -> (z := 1000 * x; z + y)\\
\emptyLine
o29 = f\\
\emptyLine
o29 : Function\\
\endOutput
Let's apply the function to some arguments.
\beginOutput
i30 : f(3,7)\\
\emptyLine
o30 = 3007\\
\endOutput
The sequence of arguments can be assembled first, and then passed to the
function.
\beginOutput
i31 : s = (3,7)\\
\emptyLine
o31 = (3, 7)\\
\emptyLine
o31 : Sequence\\
\endOutput
\beginOutput
i32 : f s\\
\emptyLine
o32 = 3007\\
\endOutput
As above, functions receiving one argument may be called without parentheses.
\beginOutput
i33 : sin 2.1\\
\emptyLine
o33 = 0.863209\\
\emptyLine
o33 : RR\\
\endOutput
A compact notation for functions makes it convenient to apply them
without naming them first.  For example, we may use \indexcmd{apply}{\tt apply} to apply a
function to every element of a list and to collect the results into a list.
\beginOutput
i34 : apply(1 .. 10, i -> i^3)\\
\emptyLine
o34 = (1, 8, 27, 64, 125, 216, 343, 512, 729, 1000)\\
\emptyLine
o34 : Sequence\\
\endOutput
The function \indexcmd{scan}{\tt scan} will do the same thing, but discard the results.
\beginOutput
i35 : scan(1 .. 5, print)\\
1\\
2\\
3\\
4\\
5\\
\endOutput
Use \indexcmd{if}{\tt if ... then ... else ...} to perform alternative actions
based on the truth of a condition.
\beginOutput
i36 : apply(1 .. 10, i -> if even i then 1000*i else i)\\
\emptyLine
o36 = (1, 2000, 3, 4000, 5, 6000, 7, 8000, 9, 10000)\\
\emptyLine
o36 : Sequence\\
\endOutput
A function can be terminated prematurely with \indexcmd{return}{\tt return}.
\beginOutput
i37 : apply(1 .. 10, i -> (if even i then return 1000*i; -i))\\
\emptyLine
o37 = (-1, 2000, -3, 4000, -5, 6000, -7, 8000, -9, 10000)\\
\emptyLine
o37 : Sequence\\
\endOutput
Loops in a program can be implemented with \indexcmd{while}{\tt while ... do ...}.
\beginOutput
i38 : i = 1; while i < 50 do (print i; i = 2*i)\\
1\\
2\\
4\\
8\\
16\\
32\\
\endOutput
Another way to implement loops is with \indexcmd{for}{\tt for} and
\indexcmd{do}{\tt do} or {\tt list}, with optional
clauses introduced by the keywords {\tt from}, {\tt to}, and {\tt when}.
\beginOutput
i40 : for i from 1 to 10 list i^3\\
\emptyLine
o40 = \{1, 8, 27, 64, 125, 216, 343, 512, 729, 1000\}\\
\emptyLine
o40 : List\\
\endOutput
\beginOutput
i41 : for i from 1 to 4 do print i\\
1\\
2\\
3\\
4\\
\endOutput
A loop can be terminated prematurely with \indexcmd{break}{\tt break}, which accepts an
optional value to return as the value of the loop expression.
\beginOutput
i42 : for i from 2 to 100 do if not isPrime i then break i\\
\emptyLine
o42 = 4\\
\endOutput
If no value needs to be returned, the condition for continuing can be
provided with the keyword {\tt when}; iteration continues only as long as the
predicate following the keyword returns {\tt true}.
\beginOutput
i43 : for i from 2 to 100 when isPrime i do print i\\
2\\
3\\
\endOutput


% if then else, while do, for, break, return,

\section{Input and Output}

The function \indexcmd{print}{\tt print} can be used to display something on the screen.
\beginOutput
i44 : print 2^100\\
1267650600228229401496703205376\\
\endOutput
For example, it could be used to display the elements of a list on separate
lines.
\beginOutput
i45 : (1 .. 5) / print;\\
1\\
2\\
3\\
4\\
5\\
\endOutput
The operator {\tt <<} can be used to display something on the
screen, without the newline character.
\beginOutput
i46 : << 2^100\\
1267650600228229401496703205376\\
o46 = stdio\\
\emptyLine
o46 : File\\
\emptyLine
\  --  the standard input output file\\
\endOutput
Notice the value returned is a {\em file}\index{files}.  A {\em file} in \Mtwo is a data type that
represents a channel through which data can be passed, as input, as
output, or in both directions.  The file \indexcmd{stdio}{\tt stdio} encountered above
corresponds to your shell window or terminal, and is used for two-way
communication between the program and the user.  A file may correspond
to what one usually calls a file, i.e., a sequence of data bytes associated
with a given name and stored on
your disk drive.  A file may also correspond to a {\em socket}, a
channel for communication with other programs over the network.

Files can be used with the binary form
of the operator {\tt \char`\<\char`\<} to display something else on the same
line.
\beginOutput
i47 : << "the value is : " << 2^100\\
the value is : 1267650600228229401496703205376\\
o47 = stdio\\
\emptyLine
o47 : File\\
\emptyLine
\  --  the standard input output file\\
\endOutput
Using \indexcmd{endl}{\tt endl} to represent the new line character or character sequence, we can
produce multiple lines of output.
\beginOutput
i48 : << "A = " << 2^100 << endl << "B = " << 2^200 << endl;\\
A = 1267650600228229401496703205376\\
B = 1606938044258990275541962092341162602522202993782792835301376\\
\endOutput
We can send the same output to a disk file named {\tt foo}, but we must remember to
close it with {\tt close}.
\beginOutput
i49 : "foo" << "A = " << 2^100 << endl << close\\
\emptyLine
o49 = foo\\
\emptyLine
o49 : File\\
\endOutput
The contents of the file can be recovered as a string with \indexcmd{get}{\tt get}.
\beginOutput
i50 : get "foo"\\
\emptyLine
o50 = A = 1267650600228229401496703205376\\
\emptyLine
\endOutput
If the file contains valid \Mtwo commands, as it does in this case, we can
execute those commands with \indexcmd{load}{\tt load}.
\beginOutput
i51 : load "foo"\\
\endOutput
We can verify that the command took effect by evaluating {\tt A}.
\beginOutput
i52 : A\\
\emptyLine
o52 = 1267650600228229401496703205376\\
\endOutput
Alternatively, if we want to see those commands and the output they produce,
we may use \indexcmd{input}{\tt input}.
\beginOutput
i53 : input "foo"\\
\emptyLine
i54 : A = 1267650600228229401496703205376\\
\emptyLine
o54 = 1267650600228229401496703205376\\
\emptyLine
i55 : \endOutput
Let's set up a ring for computation in \Mtwo.
\beginOutput
i56 : R = QQ[x,y,z]\\
\emptyLine
o56 = R\\
\emptyLine
o56 : PolynomialRing\\
\endOutput
\index{ring!making one}%
\beginOutput
i57 : f = (x+y)^3\\
\emptyLine
\       3     2        2    3\\
o57 = x  + 3x y + 3x*y  + y\\
\emptyLine
o57 : R\\
\endOutput
Printing, and printing to files, works for polynomials,
too.\index{printing!to a file}%
\beginOutput
i58 : "foo" << f << close;\\
\endOutput
The two-dimensional output is readable by humans, but is not easy to convert
back into a polynomial.
\beginOutput
i59 : get "foo"\\
\emptyLine
o59 =  3     2        2    3\\
\      x  + 3x y + 3x*y  + y\\
\endOutput
Use \indexcmd{toString}{\tt toString} to create a 1-dimensional form of the polynomial
that can be stored in a file in a format readable by \Mtwo and by other
symbolic algebra programs, such as {\em Mathematica} or {\em Maple}.
\beginOutput
i60 : toString f\\
\emptyLine
o60 = x^3+3*x^2*y+3*x*y^2+y^3\\
\endOutput
Send it to the file.
\beginOutput
i61 : "foo" << toString f << close;\\
\endOutput
Get it back.
\beginOutput
i62 : get "foo"\\
\emptyLine
o62 = x^3+3*x^2*y+3*x*y^2+y^3\\
\endOutput
Convert the string back to a polynomial with \indexcmd{value}{\tt value}, using \indexcmd{oo}{\tt oo} to
recover the value of the expression on the previous line.
\beginOutput
i63 : value oo\\
\emptyLine
\       3     2        2    3\\
o63 = x  + 3x y + 3x*y  + y\\
\emptyLine
o63 : R\\
\endOutput
The same thing works for matrices, and a little more detail is provided by
\indexcmd{toExternalString}{\tt toExternalString}, if needed.
\beginOutput
i64 : vars R\\
\emptyLine
o64 = | x y z |\\
\emptyLine
\              1       3\\
o64 : Matrix R  <--- R\\
\endOutput
\beginOutput
i65 : toString vars R\\
\emptyLine
o65 = matrix \{\{x, y, z\}\}\\
\endOutput
\beginOutput
i66 : toExternalString vars R\\
\emptyLine
o66 = map(R^\{\{0\}\}, R^\{\{-1\}, \{-1\}, \{-1\}\}, \{\{x, y, z\}\})\\
\endOutput

\section{Hash Tables}

Recall how one sets up a quotient ring for computation in \Mtwo.
\beginOutput
i67 : R = QQ[x,y,z]/(x^3-y)\\
\emptyLine
o67 = R\\
\emptyLine
o67 : QuotientRing\\
\endOutput
\beginOutput
i68 : (x+y)^4\\
\emptyLine
\        2 2       3    4           2\\
o68 = 6x y  + 4x*y  + y  + x*y + 4y\\
\emptyLine
o68 : R\\
\endOutput
How does \Mtwo represent a ring like $R$ in the computer?  To answer that,
first think about what sort of information needs to be retained about $R$.
We may need to remember the coefficient ring of $R$, the names of the
variables in $R$, the monoid of monomials in the variables, the degrees of
the variables, the characteristic of the ring, whether the ring is
commutative, the ideal modulo which we are working, and so on.  We also may
need to remember various bits of code: the code for performing the basic
arithmetic operations, such as addition and multiplication, on elements of
$R$; the code for preparing a readable representation of an element of $R$,
either 2-dimensional (with superscripts above the line and subscripts below),
or 1-dimensional.  Finally, we may want to remember certain things that take
a lot of time to compute, such as the Gr\"obner basis of the ideal.

A {\sl \ie{hash table}} is, by definition, a way of representing (in the computer)
a function whose domain is a finite set.  In \Mtwo, hash tables are extremely
flexible: the elements of the domain (or {\sl keys\index{keys of a hash table}}) and the elements of the
range (or {\sl values\index{values of a hash table}}) of the function may be any of the other objects
represented in the computer.  It's easy to come up with uses for functions
whose domain is finite: for example, a monomial can be represented by the
function that associates to a variable its nonzero exponent; a polynomial can
be represented by a function that associates to a monomial its nonzero
coefficient; a set can be represented by any function with that set as its
domain; a (sparse) matrix can be represented as a function from pairs of
natural numbers to the corresponding nonzero entry.

Let's create a hash table and name it.
\beginOutput
i69 : f = new HashTable from \{ a=>444, Daniel=>555, \{c,d\}=>\{1,2,3,4\}\}\\
\emptyLine
o69 = HashTable\{\{c, d\} => \{1, 2, 3, 4\}\}\\
\                a => 444\\
\                Daniel => 555\\
\emptyLine
o69 : HashTable\\
\endOutput
The operator \indexcmd{=>}{\tt \char`\=\char`\>} is used to represent a key-value pair.
We can use the operator \indexcmd{\#}{\tt \#} to recover the value from the key.
\beginOutput
i70 : f#Daniel\\
\emptyLine
o70 = 555\\
\endOutput
\beginOutput
i71 : f#\{c,d\}\\
\emptyLine
o71 = \{1, 2, 3, 4\}\\
\emptyLine
o71 : List\\
\endOutput
If the key is a symbol, we can use the operator \indexcmd{.}{\tt .} instead; this is
convenient if the symbol has a value that we want to ignore.
\beginOutput
i72 : Daniel = a\\
\emptyLine
o72 = a\\
\emptyLine
o72 : Symbol\\
\endOutput
\beginOutput
i73 : f.Daniel\\
\emptyLine
o73 = 555\\
\endOutput
We can use \indexcmd{\#?}{\tt \#?} to test whether a given key occurs in the hash table.
\beginOutput
i74 : f#?a\\
\emptyLine
o74 = true\\
\endOutput
\beginOutput
i75 : f#?c\\
\emptyLine
o75 = false\\
\endOutput
Finite sets are implemented in \Mtwo as hash tables: the elements of the set
are stored as the keys in the hash table, with the accompanying values all
being $1$.  (Multisets are implemented by using values larger than $1$, and
are called {\sl tallies}.)
\beginOutput
i76 : x = set\{1,a,\{4,5\},a\}\\
\emptyLine
o76 = Set \{\{4, 5\}, 1, a\}\\
\emptyLine
o76 : Set\\
\endOutput
\indexcmd{set}%
\beginOutput
i77 : x#?a\\
\emptyLine
o77 = true\\
\endOutput
\beginOutput
i78 : peek x\\
\emptyLine
o78 = Set\{\{4, 5\} => 1\}\\
\          1 => 1\\
\          a => 1\\
\endOutput
\beginOutput
i79 : y = tally\{1,a,\{4,5\},a\}\\
\emptyLine
o79 = Tally\{\{4, 5\} => 1\}\\
\            1 => 1\\
\            a => 2\\
\emptyLine
o79 : Tally\\
\endOutput
\beginOutput
i80 : y#a\\
\emptyLine
o80 = 2\\
\endOutput
We might use \indexcmd{tally}{\tt tally} to tally how often a function attains its various
possible values.  For example, how often does an integer have 3 prime
factors?  Or 4?  Use \indexcmd{factor}{\tt factor} to factor an integer.
\beginOutput
i81 : factor 60\\
\emptyLine
\       2\\
o81 = 2 3*5\\
\emptyLine
o81 : Product\\
\endOutput
Then use {\tt \#} to get the number of factors.
\beginOutput
i82 : # factor 60\\
\emptyLine
o82 = 3\\
\endOutput
Use {\tt apply} to list some values of the function.
\beginOutput
i83 : apply(2 .. 1000, i -> # factor i)\\
\emptyLine
o83 = (1, 1, 1, 1, 2, 1, 1, 1, 2, 1, 2, 1, 2, 2, 1, 1, 2, 1, 2, 2, 2,  $\cdot\cdot\cdot$\\
\emptyLine
o83 : Sequence\\
\endOutput
Finally, use {\tt tally} to summarize the results.
\beginOutput
i84 : tally oo\\
\emptyLine
o84 = Tally\{1 => 193\}\\
\            2 => 508\\
\            3 => 275\\
\            4 => 23\\
\emptyLine
o84 : Tally\\
\endOutput

Hash tables turn out to be convenient entities for storing odd bits and
pieces of information about something in a way that's easy to think about and
use.
In \Mtwo, rings are represented as hash tables, as are ideals, matrices,
modules, chain complexes, and so on.  For example, although it isn't a
documented feature, the key {\tt ideal} is used to preserve the ideal that
was used above to define the quotient ring {\tt R}, as part of the
information stored in {\tt R}.  
\beginOutput
i85 : R.ideal\\
\emptyLine
\             3\\
o85 = ideal(x  - y)\\
\emptyLine
o85 : Ideal of QQ [x, y, z]\\
\endOutput
The preferred and documented way for a user to recover this information is
with the function \indexcmd{ideal}{\tt ideal}. 
\beginOutput
i86 : ideal R\\
\emptyLine
\             3\\
o86 = ideal(x  - y)\\
\emptyLine
o86 : Ideal of QQ [x, y, z]\\
\endOutput
Users who want to introduce a new high-level mathematical
concept to \Mtwo may learn about hash tables by referring to the \Mtwo manual
\cite{M2}.

\section{Methods}

You may use the \texttt{code}\indexcmd{code} command to locate the source code for a given
function, at least if it is one of those functions written in the \Mtwo
language.  For example, here is the code for {\tt demark}, which may be used
to put commas between strings in a list.
\beginOutput
i87 : code demark\\
\emptyLine
o87 = -- ../../../m2/fold.m2:23\\
\      demark = (s,v) -> concatenate between(s,v)\\
\endOutput
The code for tensoring a ring map with a module can be displayed in this way.
\beginOutput
i88 : code(symbol **, RingMap, Module)\\
\emptyLine
o88 = -- ../../../m2/ringmap.m2:294-298\\
\      RingMap ** Module := Module => (f,M) -> (\\
\           R := source f;\\
\           S := target f;\\
\           if R =!= ring M then error "expected module over source ring";\\
\           cokernel f(presentation M));\\
\endOutput
\indexcmd{symbol}
The code implementing the {\tt ideal} function when applied to a
quotient ring can be displayed as follows.
\beginOutput
i89 : code(ideal, QuotientRing)\\
\emptyLine
o89 = -- ../../../m2/quotring.m2:7\\
\      ideal QuotientRing := R -> R.ideal\\
\endOutput
Notice that it uses the key {\tt ideal} to extract the information from the
ring's hash table, as you might have guessed from the previous discussion.
The bit of code displayed above may be called a {\sl method}\index{method} as
a way of indicating that several methods for dealing with various types of
arguments are attached to the function named {\tt ideal}.  New such {\sl
  method functions} may be created with the function {\tt method}\indexcmd{method}.  Let's
illustrate that with an example: we'll write a function called {\tt denom}
which should produce the denominator of a rational number.  When applied to
an integer, it should return 1.  First we create the method function.
\beginOutput
i90 : denom = method();\\
\endOutput
Then we tell it what to do with an argument from the class {\tt QQ} of rational numbers.
\beginOutput
i91 : denom QQ := x -> denominator x;\\
\endOutput
And also what to do with an argument from the class {\tt ZZ} of integers.
\beginOutput
i92 : denom ZZ := x -> 1;\\
\endOutput
Let's test it.
\beginOutput
i93 : denom(5/3)\\
\emptyLine
o93 = 3\\
\endOutput
\beginOutput
i94 : denom 5\\
\emptyLine
o94 = 1\\
\endOutput

\section{Pointers to the Source Code}

A substantial part of \Mtwo is written in the same language provided to the
users.  A good way to learn more about the \Mtwo language is to peruse the
source code that comes with the system in the directory {\tt Macaulay2/m2}.
Use the {\tt code} function, as described in the previous section, for
locating the bit of code you wish to view.

The source code for the interpreter of the \Mtwo language is in the directory
{\tt Macaulay2/d}.  It is written in another language designed to be mostly
type-safe, which is translated into {\tt C} by the translator whose own
{\tt C} source code is in the directory {\tt Macaulay2/c}.  Here is a
sample line of code from the file {\tt Macaulay2/d/tokens.d}, which shows how
the translator provides for allocation and initialization of dynamic data
structures.
\par
\vskip 5 pt
\begingroup
\tteight
\baselineskip=8pt
\lineskip=0pt
\obeyspaces
globalFrame := Frame(dummyFrame,globalScope.seqno,Sequence(nullE));\leavevmode\hss\endgraf
\endgroup
\penalty-1000
\par
\vskip 1 pt
\noindent
And here is the {\tt C} code produced by the translator.
\par
\vskip 5 pt
\begingroup
\tteight
\baselineskip=\outputBaseLineSkip
\lineskip=0pt
\obeyspaces
\obeylines
tokens\char`\_Frame tokens\char`\_globalFrame;
tokens\char`\_Frame tmp\char`\_\char`\_23;
Sequence tmp\char`\_\char`\_24;
tmp\char`\_\char`\_24 = (Sequence) GC\char`\_MALLOC(sizeof(struct S259\char`\_)+(1-1)*sizeof(Expr));
if (0 == tmp\char`\_\char`\_24) outofmem();
tmp\char`\_\char`\_24->len\char`\_ = 1;
tmp\char`\_\char`\_24->array\char`\_[0] = tokens\char`\_nullE;
tmp\char`\_\char`\_23 = (tokens\char`\_Frame) GC\char`\_MALLOC(sizeof(struct S260\char`\_));
if (0 == tmp\char`\_\char`\_23) outofmem();
tmp\char`\_\char`\_23->next = tokens\char`\_dummyFrame;
tmp\char`\_\char`\_23->scopenum = tokens\char`\_globalScope->seqno;
tmp\char`\_\char`\_23->values = tmp\char`\_\char`\_24;
tokens\char`\_globalFrame = tmp\char`\_\char`\_23;
\endgroup
\penalty-1000
\par
\vskip 1 pt

The core algebraic algorithms constitute the {\sl engine} of \Mtwo and are
written in {\tt C\char`\+\char`\+}, with the source files in the directory
{\tt Macaulay2/e}.  In the current version of the program, the interface
between the interpreter and the core algorithms consists of a single
two-directional stream of bytes.  The manual that comes with the system
\cite{M2} describes the engine communication protocol used in that
interface.

%% Mike says we don't need this:

%% One feature of \Mtwo which should be understood by system administrators, at
%% least of Unix systems, is the {\tt dumpdata} routine.  Since so much of the
%% source code is written in the interpreted \Mtwo language, and the task of
%% reading and parsing it can take an appreciable amount of time on slower
%% machines, we perform that task in advance (by the {\tt setup} command) when
%% the program is installed on your system.  The results are saved by writing
%% all of the data areas of the memory of the program to disk.  When a user runs
%% {\tt M2}, the file containing the saved contents is quickly mapped into
%% memory using the {\tt loaddata} command.  The contents of memory depend
%% in a detailed way on the architecture of the computer and the shareable
%% libraries that are installed, so when those things change, the {\tt setup}
%% command may need to be run again.  If those things change too often, the
%% system administrator may disable that feature.


%% Date: Tue, 21 Nov 2000 11:36:16 +0900 (JST)
%% From: Nobuki Takayama <taka@math.kobe-u.ac.jp>
%% To: dan@math.uiuc.edu
%% CC: mike@polygon.math.cornell.edu, taka@math.kobe-u.ac.jp
%% In-reply-to: <200011161511.JAA18405@orion.math.uiuc.edu> (dan@math.uiuc.edu)
%% Subject: Re: software
%% 
%% Dear Dan;
%% I would like to know 
%%   (1) "hints" to read and understand the M2 source code
%% and
%%   (2) what are new ideas in the M2 implementations.
%% The source code contains all ideas in detail, but it is not easy to
%% read and understand it without some hints.
%% 
%% I look forward to reading the section and the source code with the help
%% of the section.
%% Nobuki
%% 
%% >    If you want to add one subsection at the end, it could be something
%% >    like "Developers' notes" or "System Issues". The reader I have in mind
%% >
%% >    chapter is just the place. If you wish to consider this, then I suggest
%% >    that Mike communicate directly with Nobuki about the contents.
%% >
%% >Could you tell us directly whay you have in mind about that?
%% 
