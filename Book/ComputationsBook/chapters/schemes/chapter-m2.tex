\title{Teaching the Geometry of Schemes}
\titlerunning{Teaching the Geometry of Schemes}
\toctitle{Teaching the Geometry of Schemes}

\author{Gregory G.~Smith \and Bernd Sturmfels}
\authorrunning{G. G. Smith and B. Sturmfels}
% \institute{Department of Mathematics, University of California,
% Berkeley, California 94720, USA}

\maketitle


%%----------------------------------------------------------
\newtheorem*{problem*}{Problem}{\bfseries\upshape}{\itshape}
\newtheorem*{solution*}{Solution}{\itshape}{\rmfamily}

\newcommand{\Spec}{\operatorname{Spec}}
\newcommand{\Proj}{\operatorname{Proj}}
\newcommand{\codim}{\operatorname{codim}}
%%----------------------------------------------------------


\begin{abstract}
This chapter presents a collection of graduate level problems in
algebraic geometry illustrating the power of \Mtwo as an educational
tool.
\end{abstract}

When teaching an advanced subject, like the language of schemes, we
think it is important to provide plenty of concrete instances of the
theory.  Computer algebra systems, such as \Mtwo, provide students
with an invaluable tool for studying complicated examples.
Furthermore, we believe that the explicit nature of a computational
approach leads to a better understanding of the objects being
examined.  This chapter presents some problems which we feel
illustrate this point of view.

Our examples are selected from the homework of an algebraic geometry
class given at the University of California at Berkeley in the fall of 1999.
This graduate course was taught by the second author with assistance from the
first author.  Our choice of problems, as the title suggests, follows the
material in David Eisenbud and Joe Harris' textbook {\em The Geometry of
  Schemes} \cite{SC:EH}.

%%----------------------------------------------------------
\section{Distinguished Open Sets}

We begin with a simple example involving the Zariski topology of an affine
scheme\index{scheme!affine}. This example also indicates some of the
subtleties involved in working with arithmetic
schemes\index{scheme!arithmetic}.

\begin{problem*}
Let $S = \bbbz[x,y,z]$ and $X = \Spec(S)$.  If $f = x$ and $X_{f}$ is
the corresponding basic open subset in $X$, then establish the
following:
\begin{enumerate}
\item[$(1)$] If $e_{1} = x+y+z$, $e_{2} = xy+xz+yz$ and $e_{3} = xyz$
are the elementary symmetric functions then the set $\{X_{e_{i}}\}_{1
\leq i \leq 3}$ is an open cover of $X_{f}$.
\item[$(2)$] If $p_{1} = x+y+z$, $p_{2} = x^{2}+y^{2}+z^{2}$ and $p_{3}
= x^{3}+y^{3}+z^{3}$ are the power sum symmetric functions then
$\{X_{p_{i}}\}_{1 \leq i \leq 3}$ is {\em not} an open cover of
$X_{f}$.
\end{enumerate}
\end{problem*}

\begin{solution*}
$(1)$ To prove that $\{X_{e_{i}}\}_{1 \leq i \leq 3}$ is an open cover
of $X_{f}$, it suffices to show that $e_{1}$, $e_{2}$ and $e_{3}$
generate the unit ideal in $S_{f}$; see Lemma I-16 in Eisenbud and
Harris~\cite{SC:EH}.  This is equivalent to showing that $x^{m}$
belongs to the $S$-ideal $\langle e_{1}, e_{2}, e_{3} \rangle$ for
some $m \in \bbbn$.  In other words, the saturation\index{saturation}
$\big( \langle e_{1}, e_{2}, e_{3} \rangle : x^{\infty} \big)$ is the
unit ideal if and only if $\{X_{e_{i}}\}_{1 \leq i \leq 3}$ is an open
cover of $X_{f}$.  We verify this in \Mtwo as follows:
\beginOutput
i1 : S = ZZ[x, y, z];\\
\endOutput
\beginOutput
i2 : elementaryBasis = ideal(x+y+z, x*y+x*z+y*z, x*y*z);\\
\emptyLine
o2 : Ideal of S\\
\endOutput
\beginOutput
i3 : saturate(elementaryBasis, x)\\
\emptyLine
o3 = ideal 1\\
\emptyLine
o3 : Ideal of S\\
\endOutput
$(2)$ Similarly, to show that $\{X_{p_{i}}\}_{1 \leq i \leq 3}$ is not
an open cover of $X_{f}$, we prove that $\big( \langle p_{1}, p_{2},
p_{3} \rangle : x^{\infty} \big)$ is not the unit ideal.  Calculating
this saturation, we find
\beginOutput
i4 : powerSumBasis = ideal(x+y+z, x^2+y^2+z^2, x^3+y^3+z^3);\\
\emptyLine
o4 : Ideal of S\\
\endOutput
\beginOutput
i5 : saturate(powerSumBasis, x)\\
\emptyLine
\                            2            2\\
o5 = ideal (6, x + y + z, 2y  + 2y*z + 2z , 3y*z)\\
\emptyLine
o5 : Ideal of S\\
\endOutput
\beginOutput
i6 : clearAll\\
\endOutput
which is not the unit ideal.\qed
\end{solution*}

The fact that $6$ is a generator of the ideal $\big( \langle p_{1},
p_{2}, p_{3} \rangle : x^{\infty} \big)$ indicates that
$\{X_{p_{i}}\}_{1 \leq i \leq 3}$ does not contain the points in $X$
lying over the points $\langle 2 \rangle$ and $\langle 3 \rangle$ in
$\Spec(\bbbz)$.  If we work over a base ring in which $6$ is a unit,
then $\{X_{p_{i}}\}_{1 \leq i \leq 3}$ would, in fact, be an open
cover of $X_{f}$.


%%----------------------------------------------------------
\section{Irreducibility}

The study of complex semisimple Lie algebras gives rise to an
important family of algebraic varieties called nilpotent
orbits\index{nilpotent orbits}.  The next problem examines the
irreducibility\index{scheme!irreducible} of a particular nilpotent
orbit.

\begin{problem*} 
Let $X$ be the set of nilpotent complex $3 \times 3$ matrices.  Show
that $X$ is an irreducible algebraic variety.
\end{problem*}

\begin{solution*}
A $3 \times 3$ matrix $M$ is nilpotent if and only if its minimal
polynomial $p(\sf T)$ equals ${\sf T}^{k}$, for some $k \in \bbbn$.
Since each irreducible factor of the characteristic polynomial of $M$
is also a factor of $p(\sf T)$, it follows that the characteristic
polynomial of $M$ is ${\sf T}^{3}$.  We conclude that the coefficients
of the characteristic polynomial of a generic $3 \times 3$ matrix
define the algebraic variety $X$.

To prove that $X$ is irreducible over $\bbbc$, we construct a rational
parameterization\index{rational parameterization}.  First, observe
that ${\rm GL}_{3}(\bbbc)$ acts on $X$ by conjugation.  Jordan's
canonical form theorem implies that there are exactly three orbits;
one for each of the following matrices:
\[
N_{(1,1,1)} =\left[ \begin{smallmatrix} 0 & 0 & 0 \\ 0 & 0 & 0 \\ 0 &
0 & 0 \end{smallmatrix} \right], \quad
N_{(2,1)} = \left[ \begin{smallmatrix} 0 & 1 & 0 \\ 0 & 0 & 0 \\ 0 & 0
& 0 \end{smallmatrix} \right] \text{ and }
N_{(3)} = \left[ \begin{smallmatrix} 0 & 1 & 0 \\ 0 & 0 & 1 \\ 0 & 0 &
0 \end{smallmatrix} \right] \enspace .
\]
Each orbit is defined by a rational parameterization, so it suffices
to show that the closure of the orbit containing $N_{(3)}$ is the
entire variety $X$.  We demonstrate this as follows:
\beginOutput
i7 : S = QQ[t, y_0 .. y_8, a..i, MonomialOrder => Eliminate 10];\\
\endOutput
\beginOutput
i8 : N3 = (matrix \{\{0,1,0\},\{0,0,1\},\{0,0,0\}\}) ** S\\
\emptyLine
o8 = | 0 1 0 |\\
\     | 0 0 1 |\\
\     | 0 0 0 |\\
\emptyLine
\             3       3\\
o8 : Matrix S  <--- S\\
\endOutput
\beginOutput
i9 : G = genericMatrix(S, y_0, 3, 3)\\
\emptyLine
o9 = | y_0 y_3 y_6 |\\
\     | y_1 y_4 y_7 |\\
\     | y_2 y_5 y_8 |\\
\emptyLine
\             3       3\\
o9 : Matrix S  <--- S\\
\endOutput
To determine the entries in $G \cdot N_{(3)} \cdot G^{-1}$, we use the
classical adjoint\index{classical adjoint} to construct the matrix
$\det(G) \cdot G^{-1}$.
\beginOutput
i10 : classicalAdjoint = (G) -> (\\
\           n := degree target G;\\
\           m := degree source G;\\
\           matrix table(n, n, (i, j) -> (-1)^(i+j) * det(\\
\                     submatrix(G, \{0..j-1, j+1..n-1\}, \\
\                          \{0..i-1, i+1..m-1\}))));\\
\endOutput
\beginOutput
i11 : num = G * N3 * classicalAdjoint(G);\\
\emptyLine
\              3       3\\
o11 : Matrix S  <--- S\\
\endOutput
\beginOutput
i12 : D = det(G);\\
\endOutput
\beginOutput
i13 : M = genericMatrix(S, a, 3, 3);\\
\emptyLine
\              3       3\\
o13 : Matrix S  <--- S\\
\endOutput
The entries in $G \cdot N_{(3)} \cdot G^{-1}$ give a rational
parameterization of the orbit generated by $N_{(3)}$.  Using
elimination theory\index{elimination theory} --- see section~3.3 in
Cox, Little and O`Shea~\cite{SC:CLO} --- we give an ``implicit
representation'' of this variety.
\beginOutput
i14 : elimIdeal = minors(1, (D*id_(S^3))*M - num) + ideal(1-D*t);\\
\emptyLine
o14 : Ideal of S\\
\endOutput
\beginOutput
i15 : closureOfOrbit = ideal selectInSubring(1, gens gb elimIdeal);\\
\emptyLine
o15 : Ideal of S\\
\endOutput

Finally, we verify that this orbit closure equals $X$
scheme-theoretically.  Recall that $X$ is defined by the coefficients
of the characteristic polynomial of a generic $3 \times 3$ matrix {\tt
M}.
%% was X = ideal submatrix( (coeff-icients({0}, det(M - t*id_(S^3))))_1,
%% {1,2,3} ), but 'coeff-icients' is to be redesigned, and 'contract' is
%% more self-explanatory, anyway.
\beginOutput
i16 : X = ideal substitute(\\
\              contract(matrix\{\{t^2,t,1\}\}, det(t-M)),\\
\              \{t => 0_S\})\\
\emptyLine
o16 = ideal (- a - e - i, - b*d + a*e - c*g - f*h + a*i + e*i, c*e*g - $\cdot\cdot\cdot$\\
\emptyLine
o16 : Ideal of S\\
\endOutput
\beginOutput
i17 : closureOfOrbit == X\\
\emptyLine
o17 = true\\
\endOutput
\beginOutput
i18 : clearAll\\
\endOutput
This completes our solution.\qed
\end{solution*}

More generally, Kostant shows that the set of all nilpotent elements
in a complex semisimple Lie algebra\index{Lie algebra} form an
irreducible variety.  We refer the reader to Chriss and
Ginzburg~\cite{SC:CV} for a proof of this result (Corollary~3.2.8) and
a discussion of its applications in representation theory.


%%----------------------------------------------------------
\section{Singular Points}

In our third question, we study the singular locus\index{singular
locus} of a family of elliptic curves\index{elliptic curve}.

\begin{problem*} 
Consider a general form of degree $3$ in $\bbbq[x,y,z]$:
\[
F = ax^{3} + bx^{2}y + cx^{2}z + dxy^{2} + exyz + fxz^{2} + gy^{3} +
hy^{2}z + iyz^{2} + jz^{3} \enspace .
\]
Give necessary and sufficient conditions in terms of $a, \ldots, j$
for the cubic curve $\Proj\big( \bbbq[x,y,z] / \langle F \rangle
\big)$ to have a singular point.
\end{problem*}

\begin{solution*}
The singular locus of $F$ is defined by a polynomial of degree $12$ in
the $10$ variables $a, \dotsc, j$.  We calculate this polynomial in two
different ways.

Our first method is an elementary but time consuming elimination.
Carrying it out in \Mtwo, we have
\beginOutput
i19 : S = QQ[x, y, z, a..j, MonomialOrder => Eliminate 2];\\
\endOutput
\beginOutput
i20 : F = a*x^3+b*x^2*y+c*x^2*z+d*x*y^2+e*x*y*z+f*x*z^2+g*y^3+h*y^2*z+\\
\                   i*y*z^2+j*z^3;\\
\endOutput
\beginOutput
i21 : partials = submatrix(jacobian matrix\{\{F\}\}, \{0..2\}, \{0\})\\
\emptyLine
o21 = \{1\} | 3x2a+2xyb+y2d+2xzc+yze+z2f |\\
\      \{1\} | x2b+2xyd+3y2g+xze+2yzh+z2i |\\
\      \{1\} | x2c+xye+y2h+2xzf+2yzi+3z2j |\\
\emptyLine
\              3       1\\
o21 : Matrix S  <--- S\\
\endOutput
\beginOutput
i22 : singularities = ideal(partials) + ideal(F);\\
\emptyLine
o22 : Ideal of S\\
\endOutput
\beginOutput
i23 : elimDiscr = time ideal selectInSubring(1,gens gb singularities);\\
\     -- used 64.27 seconds\\
\emptyLine
o23 : Ideal of S\\
\endOutput
\beginOutput
i24 : elimDiscr = substitute(elimDiscr, \{z => 1\});\\
\emptyLine
o24 : Ideal of S\\
\endOutput
On the other hand, there is also an elegant and more useful
determinantal formula for this discriminant\index{discriminant}; it is
a specialization of the formula (2.8) in section~3.2 of Cox, Little
and O`Shea~\cite{SC:CLO2}.  To apply this determinantal formula, we
first create the coefficient matrix {\tt A} of the partial derivatives
of $F$.
%% was A = (coeff-icients({0,1,2}, submatrix(jacobian matrix{{F}}, {0..2}, {0})))_1;
%% but 'coeff-icients' is deprecated.
\beginOutput
i25 : A = contract(matrix\{\{x^2,x*y,y^2,x*z,y*z,z^2\}\},\\
\              diff(transpose matrix\{\{x,y,z\}\},F))\\
\emptyLine
o25 = \{1\} | 3a 2b d  2c e  f  |\\
\      \{1\} | b  2d 3g e  2h i  |\\
\      \{1\} | c  e  h  2f 2i 3j |\\
\emptyLine
\              3       6\\
o25 : Matrix S  <--- S\\
\endOutput
We also construct the coefficient matrix {\tt B} of the partial
derivatives of the Hessian\index{hessian} of $F$.
\beginOutput
i26 : hess = det submatrix(jacobian ideal partials, \{0..2\}, \{0..2\});\\
\endOutput
%% was B = (coeff-icients({0,1,2}, submatrix(jacobian matrix{{hess}}, {0..2}, {0})))_1;
%% but 'coeff-icients' is deprecated.
\beginOutput
i27 : B = contract(matrix\{\{x^2,x*y,y^2,x*z,y*z,z^2\}\},\\
\              diff(transpose matrix\{\{x,y,z\}\},hess))\\
\emptyLine
o27 = \{1\} | -24c2d+24bce-18ae2-24b2f+72adf               4be2-16bdf-48 $\cdot\cdot\cdot$\\
\      \{1\} | 2be2-8bdf-24c2g+72afg+16bch-24aeh-8b2i+24adi 4de2-16d2f-48 $\cdot\cdot\cdot$\\
\      \{1\} | 2ce2-8cdf-8c2h+24afh+16bci-24aei-24b2j+72adj 2e3-8def-24cf $\cdot\cdot\cdot$\\
\emptyLine
\              3       6\\
o27 : Matrix S  <--- S\\
\endOutput
To obtain the discriminant, we combine these two matrices and take the
determinant.
\beginOutput
i28 : detDiscr = ideal det (A || B);\\
\emptyLine
o28 : Ideal of S\\
\endOutput
Finally, we check that our two discriminants are equal
\beginOutput
i29 : detDiscr == elimDiscr\\
\emptyLine
o29 = true\\
\endOutput
and examine the generator.
\beginOutput
i30 : detDiscr_0\\
\emptyLine
\            2   4 3 2             5 3 2           6 3 2          2 2 2 $\cdot\cdot\cdot$\\
o30 = 13824c d*e f g  - 13824b*c*e f g  + 13824a*e f g  - 110592c d e  $\cdot\cdot\cdot$\\
\emptyLine
o30 : S\\
\endOutput
\beginOutput
i31 : numgens detDiscr\\
\emptyLine
o31 = 1\\
\endOutput
\beginOutput
i32 : # terms detDiscr_0\\
\emptyLine
o32 = 2040\\
\endOutput
\beginOutput
i33 : clearAll\\
\endOutput
Hence, the singular locus is given by a single polynomial of degree
$12$ with $2040$ terms.\qed
\end{solution*}

For a further discussion of singularities and discriminants see
Section~V.3 in Eisenbud and Harris~\cite{SC:EH}.  For information on
resultants and discriminants see Chapter~2 in Cox, Little and
O`Shea~\cite{SC:CLO2}.


%%----------------------------------------------------------
\section{Fields of Definition}

Schemes\index{scheme!over a number field} over non-algebraically
closed fields arise in number theory.  Our fourth problem looks at one
technique for working with number fields in \Mtwo.

\begin{problem*}[Exercise~II-6 in  \cite{SC:EH}]
An inclusion of fields $K \hookrightarrow L$ induces a map
$\mathbb{A}_{L}^{n} \to \mathbb{A}_{K}^{n}$.  Find the images in
$\mathbb{A}_{\bbbq}^{2}$ of the following points of
$\mathbb{A}_{\overline{\bbbq}}^{2}$ under this map.
\begin{enumerate}
\item[$(1)$] $\langle x - \sqrt{2}, y - \sqrt{2} \rangle ;$
\item[$(2)$] $\langle x - \sqrt{2}, y - \sqrt{3} \rangle ;$
\item[$(3)$] $\langle x - \zeta, y - \zeta^{-1} \rangle$ where $\zeta$
is a $5$-th root of unity $;$
\item[$(4)$] $\langle \sqrt{2}x- \sqrt{3}y \rangle ;$
\item[$(5)$] $\langle \sqrt{2}x- \sqrt{3}y-1 \rangle$.
\end{enumerate}
\end{problem*}

\begin{solution*}
The images can be determined by using the following three step
algorithm: (1) replace the coefficients not contained in $K$ with
indeterminates, (2) add the minimal polynomials of these coefficients
to the given ideal in $\mathbb{A}_{L}^{2}$, and (3) eliminate the new
indeterminates.  Here are the five examples:
\beginOutput
i34 : S = QQ[a,b,x,y, MonomialOrder => Eliminate 2];\\
\endOutput
\beginOutput
i35 : I1 = ideal(x-a, y-a, a^2-2);\\
\emptyLine
o35 : Ideal of S\\
\endOutput
\beginOutput
i36 : ideal selectInSubring(1, gens gb I1)\\
\emptyLine
\                     2\\
o36 = ideal (x - y, y  - 2)\\
\emptyLine
o36 : Ideal of S\\
\endOutput
\beginOutput
i37 : I2 = ideal(x-a, y-b, a^2-2, b^2-3);\\
\emptyLine
o37 : Ideal of S\\
\endOutput
\beginOutput
i38 : ideal selectInSubring(1, gens gb I2)\\
\emptyLine
\              2       2\\
o38 = ideal (y  - 3, x  - 2)\\
\emptyLine
o38 : Ideal of S\\
\endOutput
\beginOutput
i39 : I3 = ideal(x-a, y-a^4, a^4+a^3+a^2+a+1);\\
\emptyLine
o39 : Ideal of S\\
\endOutput
\beginOutput
i40 : ideal selectInSubring(1, gens gb I3)\\
\emptyLine
\                       2    2               3    2\\
o40 = ideal (x*y - 1, x  + y  + x + y + 1, y  + y  + x + y + 1)\\
\emptyLine
o40 : Ideal of S\\
\endOutput
\beginOutput
i41 : I4 = ideal(a*x+b*y, a^2-2, b^2-3);\\
\emptyLine
o41 : Ideal of S\\
\endOutput
\beginOutput
i42 : ideal selectInSubring(1, gens gb I4)\\
\emptyLine
\             2   3  2\\
o42 = ideal(x  - -*y )\\
\                 2\\
\emptyLine
o42 : Ideal of S\\
\endOutput
\beginOutput
i43 : I5 = ideal(a*x+b*y-1, a^2-2, b^2-3);\\
\emptyLine
o43 : Ideal of S\\
\endOutput
\beginOutput
i44 : ideal selectInSubring(1, gens gb I5)\\
\emptyLine
\             4     2 2   9  4    2   3  2   1\\
o44 = ideal(x  - 3x y  + -*y  - x  - -*y  + -)\\
\                         4           2      4\\
\emptyLine
o44 : Ideal of S\\
\endOutput
\beginOutput
i45 : clearAll\\
\endOutput
\qed
\end{solution*}

It is worth noting that the points in $\mathbb{A}_{\bbbq}^{n}$ correspond
to orbits of the action of ${\rm Gal}(\overline{\bbbq}/\bbbq)$ on the
points of $\mathbb{A}_{\overline{\bbbq}}^{n}$.  For more examples and
information, see section~II.2 in Eisenbud and Harris~\cite{SC:EH}.


%%----------------------------------------------------------
\section{Multiplicity}

The multiplicity\index{multiplicity} of a zero-dimensional scheme $X$
at a point $p \in X$ is defined to be the length of the local ring
$\mathcal{O}_{X,p}$.  Unfortunately, we cannot work directly in the
local ring in \Mtwo.  What we can do, however, is to compute the
multiplicity by computing the degree of the component of $X$ supported
at $p$; see page 66 in Eisenbud and Harris~\cite{SC:EH}.

\begin{problem*}
What is the multiplicity of the origin as a zero of the polynomial
equations $x^{5}+y^{3}+z^{3} = x^{3}+y^{5}+z^{3} = x^{3}+y^{3}+z^{5} =
0$?
\end{problem*}

\begin{solution*}
If $I$ is the ideal generated by $x^{5}+y^{3}+z^{3}$,
$x^{3}+y^{5}+z^{3}$ and $x^{3}+y^{3}+z^{5}$ in $\bbbq[x,y,z]$, then
the multiplicity of the origin is
\[
\dim_{\bbbq} \frac{\bbbq[x,y,z]_{\langle x,y,z \rangle}}
{I \bbbq[x,y,z]_{\langle x,y,z \rangle}} \, .
\]
It follows that the multiplicity is the vector space dimension of the
ring $\bbbq[x,y,z] / \varphi^{-1}(I \bbbq[x,y,z]_{\langle x,y,z
\rangle})$ where $\varphi \colon \bbbq[x,y,z] \to
\bbbq[x,y,z]_{\langle x,y,z \rangle}$ is the natural map.  Moreover,
we can express this using ideal quotients:
\[
\varphi^{-1}(I \bbbq[x,y,z]_{\langle x,y,z \rangle}) \,\,= \,\,
\big(I : (I : \langle x,y,z \rangle^{\infty})\big) \, .
\]
Carrying out this calculation in \Mtwo, we obtain:
\beginOutput
i46 : S = QQ[x, y, z];\\
\endOutput
\beginOutput
i47 : I = ideal(x^5+y^3+z^3, x^3+y^5+z^3, x^3+y^3+z^5);\\
\emptyLine
o47 : Ideal of S\\
\endOutput
\beginOutput
i48 : multiplicity = degree(I : saturate(I))\\
\emptyLine
o48 = 27\\
\endOutput
\beginOutput
i49 : clearAll\\
\endOutput
Thus, we conclude that the multiplicity is $27$.\qed
\end{solution*}

There are algorithms (not yet implemented in \Mtwo) for working
directly in the local ring $\bbbq[x,y,z]_{\langle x,y,z \rangle}$.  We
refer the interested reader to Chapter~4 in Cox, Little and
O`Shea~\cite{SC:CLO2}.


%%----------------------------------------------------------
\section{Flat Families}

Non-reduced schemes\index{scheme!non-reduced} arise naturally as flat
limits\index{flat limit} of a family of reduced
schemes\index{scheme!reduced}. Our next problem illustrates how a
family of skew lines in $\bbbp^{3}$ gives rise to a double line with
an embedded point\index{embedded point}.

\begin{problem*}[Exercise~III-68 in \cite{SC:EH}]
Let $L$ and $M $ be the lines in $\bbbp^{3}_{k[t]}$ given by $x=y=0$
and $x-tz = y+t^{2}w =0$ respectively.  Show that the flat limit as $t
\to 0$ of the union $L \cup M$ is the double line $x^{2} = y = 0$ with
an embedded point of degree $1$ located at the point $(0:0:0:1)$.
\end{problem*}

\begin{solution*}
We first find the flat limit by saturating\index{saturation} the
intersection ideal and setting $t = 0$.
\beginOutput
i50 : PP3 = QQ[t, x, y, z, w];\\
\endOutput
\beginOutput
i51 : L = ideal(x, y);\\
\emptyLine
o51 : Ideal of PP3\\
\endOutput
\beginOutput
i52 : M = ideal(x-t*z, y+t^2*w);\\
\emptyLine
o52 : Ideal of PP3\\
\endOutput
\beginOutput
i53 : X = intersect(L, M);\\
\emptyLine
o53 : Ideal of PP3\\
\endOutput
\beginOutput
i54 : Xzero = trim substitute(saturate(X, t), \{t => 0\})\\
\emptyLine
\                   2        2\\
o54 = ideal (y*z, y , x*y, x )\\
\emptyLine
o54 : Ideal of PP3\\
\endOutput
Secondly, we verify that this is the union of a double line and an
embedded point of degree $1$.
\beginOutput
i55 : Xzero == intersect(ideal(x^2, y), ideal(x, y^2, z))\\
\emptyLine
o55 = true\\
\endOutput
\beginOutput
i56 : degree(ideal(x^2, y ) / ideal(x, y^2, z))\\
\emptyLine
o56 = 1\\
\endOutput
\beginOutput
i57 : clearAll\\
\endOutput
\qed
\end{solution*}

Section~III.3.4 in Eisenbud and Harris~\cite{SC:EH} contains several
other interesting limits of various flat families.


%%----------------------------------------------------------
\section{B\'{e}zout's Theorem}

B\'{e}zout's Theorem\index{Bezout's Theorem@B\'ezout's Theorem} --- Theorem~III-78 in
Eisenbud and Harris~\cite{SC:EH} --- may fail without the
Cohen-Macaulay\index{Cohen-Macaulay} hypothesis.  Our seventh problem
is to demonstrate this.

\begin{problem*}[Exercise~III-81 in \cite{SC:EH}]
Find irreducible closed subvarieties $X$ and $Y$ in $\bbbp^{4}$ such
that 
\begin{align*}
\codim(X \cap Y) &= \codim(X) + \codim(Y) \\
\deg(X \cap Y) &> \deg(X) \cdot \deg(Y) \, .
\end{align*}
\end{problem*}

\begin{solution*}
We show that the assertion holds when $X$ is the cone over the
nonsingular rational quartic curve\index{rational quartic curve} in
$\bbbp^{3}$ and $Y$ is a two-plane passing through the vertex of the
cone.  First, recall that the rational quartic curve is given by the
$2 \times 2$ minors of the matrix $\left[ \begin{smallmatrix} a &
b^{2} & bd & c \\ b & ac & c^2 & d \end{smallmatrix} \right]$; see
Exercise~18.8 in Eisenbud~\cite{SC:E}.  Thus, we have
\beginOutput
i58 : S = QQ[a, b, c, d, e];\\
\endOutput
\beginOutput
i59 : IX = trim minors(2, matrix\{\{a, b^2, b*d, c\},\{b, a*c, c^2, d\}\})\\
\emptyLine
\                         3      2     2    2    3    2\\
o59 = ideal (b*c - a*d, c  - b*d , a*c  - b d, b  - a c)\\
\emptyLine
o59 : Ideal of S\\
\endOutput
\beginOutput
i60 : IY = ideal(a, d);\\
\emptyLine
o60 : Ideal of S\\
\endOutput
\beginOutput
i61 : codim IX + codim IY == codim (IX + IY)\\
\emptyLine
o61 = true\\
\endOutput
\beginOutput
i62 : (degree IX) * (degree IY)\\
\emptyLine
o62 = 4\\
\endOutput
\beginOutput
i63 : degree (IX + IY)\\
\emptyLine
o63 = 5\\
\endOutput
which establishes the assertion.\qed
\end{solution*}

To understand how this example works, it is enlightening to express
$Y$ as the intersection of two hyperplanes; one given by $a = 0$ and
the other given by $d = 0$.  Intersecting $X$ with the first
hyperplane yields
\beginOutput
i64 : J = ideal mingens (IX + ideal(a))\\
\emptyLine
\                      3      2   2    3\\
o64 = ideal (a, b*c, c  - b*d , b d, b )\\
\emptyLine
o64 : Ideal of S\\
\endOutput
However, this first intersection has an embedded point;
\beginOutput
i65 : J == intersect(ideal(a, b*c, b^2, c^3-b*d^2), \\
\           ideal(a, d, b*c, c^3, b^3)) -- embedded point\\
\emptyLine
o65 = true\\
\endOutput
\beginOutput
i66 : clearAll\\
\endOutput
The second hyperplane passes through this embedded
point\index{embedded point} which explains the extra intersection.


%%----------------------------------------------------------
\section{Constructing Blow-ups}

The blow-up\index{blow-up} of a scheme $X$ along a subscheme $Y$ can
be constructed from the Rees algebra\index{Rees algebra} associated to
the ideal sheaf of $Y$ in $X$; see Theorem~IV-22 in Eisenbud and
Harris~\cite{SC:EH}.  Gr\"{o}bner basis techniques allow one to
express the Rees algebra in terms of generators and relations.  We
illustrate this method in the next solution.

\begin{problem*}[Exercises~IV-43 \& IV-44 in \cite{SC:EH}]
Find the blow-up $X$ of the affine plane\index{scheme!affine}
$\mathbb{A}^{2} = \Spec\big( \bbbq[x, y] \big)$ along the subscheme
defined by $\langle x^{3}, xy, y^{2} \rangle$.  Show that $X$ is
nonsingular and its fiber over the origin is the union of two copies
of $\bbbp^{1}$ meeting at a point.
\end{problem*}

\begin{solution*}
We first provide a general function which returns the ideal of
relations for the Rees algebra.
\beginOutput
i67 : blowUpIdeal = (I) -> (\\
\           r := numgens I;\\
\           S := ring I;\\
\           n := numgens S;\\
\           K := coefficientRing S;\\
\           tR := K[t, gens S, vars(0..r-1), \\
\                     MonomialOrder => Eliminate 1];\\
\           f := map(tR, S, submatrix(vars tR, \{1..n\}));\\
\           F := f(gens I);\\
\           J := ideal apply(1..r, j -> (gens tR)_(n+j)-t*F_(0,(j-1)));\\
\           L := ideal selectInSubring(1, gens gb J);\\
\           R := K[gens S, vars(0..r-1)];\\
\           g := map(R, tR, 0 | vars R);\\
\           trim g(L));\\
\endOutput
Now, applying the function to our specific case yields: 
\beginOutput
i68 : S = QQ[x, y];\\
\endOutput
\beginOutput
i69 : I = ideal(x^3, x*y, y^2);\\
\emptyLine
o69 : Ideal of S\\
\endOutput
\beginOutput
i70 : J = blowUpIdeal(I)\\
\emptyLine
\                           2         2          3     2\\
o70 = ideal (y*b - x*c, x*b  - a*c, x b - y*a, x c - y a)\\
\emptyLine
o70 : Ideal of QQ [x, y, a, b, c]\\
\endOutput
Therefore, the blow-up of the affine plane along the given subscheme
is
\[
X = \Proj\left( \frac{(\bbbq[x,y])[a,b,c]}{\langle yb-xc, xb^{2}-ac,
x^{2}b-ya, x^{3}c-y^{2}a \rangle} \right) \, .
\]
Using \Mtwo, we can also verify that the scheme $X$ is
nonsingular\index{singular locus};
\beginOutput
i71 : J + ideal jacobian J == ideal gens ring J\\
\emptyLine
o71 = true\\
\endOutput
\beginOutput
i72 : clearAll\\
\endOutput
Since we have
\[
\frac{(\bbbq[x,y])[a,b,c]}{\langle yb-xc, xb^{2}-ac, x^{2}b-ya,
x^{3}c-y^{2}a \rangle} \otimes \frac{\bbbq[x,y]}{\langle x, y \rangle}
\cong \frac{\bbbq[a,b,c]}{\langle ac \rangle} \, ,
\]
the fiber over the origin $\langle x,y \rangle$ in $\mathbb{A}^{2}$ is
clearly a union of two copies of $\bbbp^{1}$ meeting at one point.  In
particular, the exceptional fiber is not a projective space.\qed
\end{solution*}

Many other interesting blow-ups can be found in section~II.2 in
Eisenbud and Harris~\cite{SC:EH}.


%%----------------------------------------------------------
\section{A Classic Blow-up}

We consider the blow-up\index{blow-up} of the projective plane
$\bbbp^{2}$ at a point.

\vbox{
\begin{problem*}
Show that the following varieties are isomorphic.
\begin{enumerate}
\item[$(a)$] the image of the rational map from $\bbbp^{2}$ to
$\bbbp^{4}$ given by
\[
(r:s:t) \mapsto (r^{2}:s^{2}:rs:rt:st) \, ;
\]
\item[$(b)$] the blow-up of the plane $\bbbp^{2}$ at the point
$(0:0:1)$;
\item[$(c)$] the determinantal variety\index{determinantal variety}
defined by the $2 \times 2$ minors of the matrix $\left[
\begin{smallmatrix} a & c & d \\ b & d & e \end{smallmatrix} \right]$
where $\bbbp^{4} = \Proj\big( k[a,b,c,d,e] \big)$.
\end{enumerate}
This surface is called the {\em cubic scroll}\index{cubic scroll} in
$\bbbp^{4}$.
\end{problem*}
}

\begin{solution*}
We find the ideal in part~$(a)$ by elimination
theory\index{elimination theory}.
\beginOutput
i73 : PP4 = QQ[a..e];\\
\endOutput
\beginOutput
i74 : S = QQ[r..t, A..E, MonomialOrder => Eliminate 3];\\
\endOutput
\beginOutput
i75 : I = ideal(A - r^2, B - s^2, C - r*s, D - r*t, E - s*t);\\
\emptyLine
o75 : Ideal of S\\
\endOutput
\beginOutput
i76 : phi = map(PP4, S, matrix\{\{0_PP4, 0_PP4, 0_PP4\}\} | vars PP4)\\
\emptyLine
o76 = map(PP4,S,\{0, 0, 0, a, b, c, d, e\})\\
\emptyLine
o76 : RingMap PP4 <--- S\\
\endOutput
\beginOutput
i77 : surfaceA = phi ideal selectInSubring(1, gens gb I)\\
\emptyLine
\                                          2\\
o77 = ideal (c*d - a*e, b*d - c*e, a*b - c )\\
\emptyLine
o77 : Ideal of PP4\\
\endOutput
Next, we determine the surface in part~$(b)$.  We construct the ideal
defining the blow-up of $\bbbp^{2}$ 
\beginOutput
i78 : R = QQ[t, x, y, z, u, v, MonomialOrder => Eliminate 1];\\
\endOutput
\beginOutput
i79 : blowUpIdeal = ideal selectInSubring(1, gens gb ideal(u-t*x, \\
\           v-t*y))\\
\emptyLine
o79 = ideal(y*u - x*v)\\
\emptyLine
o79 : Ideal of R\\
\endOutput
and embed it in $\bbbp^{2} \times \bbbp^{1}$.
\beginOutput
i80 : PP2xPP1 = QQ[x, y, z, u, v];\\
\endOutput
\beginOutput
i81 : embed = map(PP2xPP1, R, 0 | vars PP2xPP1);\\
\emptyLine
o81 : RingMap PP2xPP1 <--- R\\
\endOutput
\beginOutput
i82 : blowUp = PP2xPP1 / embed(blowUpIdeal);\\
\endOutput
We then map this surface into $\bbbp^{5}$ using the Segre
embedding\index{Segre embedding}.
\beginOutput
i83 : PP5 = QQ[A .. F];\\
\endOutput
\beginOutput
i84 : segre = map(blowUp, PP5, matrix\{\{x*u,y*u,z*u,x*v,y*v,z*v\}\});\\
\emptyLine
o84 : RingMap blowUp <--- PP5\\
\endOutput
\beginOutput
i85 : ker segre\\
\emptyLine
\                                2\\
o85 = ideal (B - D, C*E - D*F, D  - A*E, C*D - A*F)\\
\emptyLine
o85 : Ideal of PP5\\
\endOutput
Note that the image under the Segre map lies on a hyperplane in
$\bbbp^{5}$.  To get the desired surface in $\bbbp^{4}$, we project
\beginOutput
i86 : projection = map(PP4, PP5, matrix\{\{a, c, d, c, b, e\}\})\\
\emptyLine
o86 = map(PP4,PP5,\{a, c, d, c, b, e\})\\
\emptyLine
o86 : RingMap PP4 <--- PP5\\
\endOutput
\beginOutput
i87 : surfaceB = trim projection ker segre\\
\emptyLine
\                                          2\\
o87 = ideal (c*d - a*e, b*d - c*e, a*b - c )\\
\emptyLine
o87 : Ideal of PP4\\
\endOutput
Finally, we compute the surface in part~$(c)$.
\beginOutput
i88 : determinantal = minors(2, matrix\{\{a, c, d\}, \{b, d, e\}\})\\
\emptyLine
\                                          2\\
o88 = ideal (- b*c + a*d, - b*d + a*e, - d  + c*e)\\
\emptyLine
o88 : Ideal of PP4\\
\endOutput
\beginOutput
i89 : sigma = map( PP4, PP4, matrix\{\{d, e, a, c, b\}\});\\
\emptyLine
o89 : RingMap PP4 <--- PP4\\
\endOutput
\beginOutput
i90 : surfaceC = sigma determinantal\\
\emptyLine
\                                          2\\
o90 = ideal (c*d - a*e, b*d - c*e, a*b - c )\\
\emptyLine
o90 : Ideal of PP4\\
\endOutput
By incorporating a permutation of the variables into definition of
{\tt surfaceC}, we obtain the desired isomorphisms
\beginOutput
i91 : surfaceA == surfaceB\\
\emptyLine
o91 = true\\
\endOutput
\beginOutput
i92 : surfaceB == surfaceC\\
\emptyLine
o92 = true\\
\endOutput
\beginOutput
i93 : clearAll\\
\endOutput
which completes the solution.\qed
\end{solution*}

For more information of the geometry of rational normal scrolls, see
Lecture~8 in Harris~\cite{SC:H}.


%%----------------------------------------------------------
\section{Fano Schemes}

Our final example concerns the family of Fano schemes\index{Fano
scheme} associated to a flat family of quadrics.
Recall that the $k$-th Fano scheme $F_{k}(X)$ of a
scheme $X \subseteq \bbbp^{n}$ is the subscheme of
the Grassmannian parametrizing $k$-planes
contained in $X$.

\begin{problem*}[Exercise~IV-69 in \cite{SC:EH}]
Consider the one-parameter family\index{one-parameter family} of
quadrics tending to a double plane with equation
\[ 
Q = V(tx^{2}+ty^{2}+tz^{2}+w^{2}) \subseteq \bbbp^{3}_{\bbbq[t]} =
\Proj\big(\bbbq[t][x,y,z,w]\big) \enspace .
\]
What is the flat limit\index{flat limit} of the Fano schemes
$F_{1}(Q_{t})$?
\end{problem*}

\begin{solution*}
We first compute the ideal defining $F_{1}(Q_{t})$, the scheme
parametrizing lines in $Q$.
\beginOutput
i94 : PP3 = QQ[t, x, y, z, w];\\
\endOutput
\beginOutput
i95 : Q = ideal( t*x^2+t*y^2+t*z^2+w^2 );\\
\emptyLine
o95 : Ideal of PP3\\
\endOutput
To parametrize a line in our projective space, we introduce
indeterminates $u, v$ and $A, \dotsc, H$.
\beginOutput
i96 : R = QQ[t, u, v, A .. H];\\
\endOutput
We then make a map {\tt phi} from {\tt PP3} to {\tt R} sending the
variables to the coordinates of the general point on a line.
\beginOutput
i97 : phi = map(R, PP3, matrix\{\{t\}\} | \\
\              u*matrix\{\{A, B, C, D\}\} + v*matrix\{\{E, F, G, H\}\});\\
\emptyLine
o97 : RingMap R <--- PP3\\
\endOutput
\beginOutput
i98 : imageFamily = phi Q;\\
\emptyLine
o98 : Ideal of R\\
\endOutput
For a line to belong to $Q$, the {\tt imageFamily} must vanish
identically.  In other words, $F_{1}(Q)$ is defined by the
coefficients of the generators of {\tt imageFamily}.
%% removing a final use of 'coefficients'
%% coeffOfFamily = (coefficients ({1,2}, gens imageFamily))_1;
\beginOutput
i99 : coeffOfFamily = contract(matrix\{\{u^2,u*v,v^2\}\}, gens imageFamily)\\
\emptyLine
o99 = | tA2+tB2+tC2+D2 2tAE+2tBF+2tCG+2DH tE2+tF2+tG2+H2 |\\
\emptyLine
\              1       3\\
o99 : Matrix R  <--- R\\
\endOutput
Since we don't need the variables $u$ and $v$, we get rid of them.
\beginOutput
i100 : S = QQ[t, A..H];\\
\endOutput
\beginOutput
i101 : coeffOfFamily = substitute(coeffOfFamily, S);\\
\emptyLine
\               1       3\\
o101 : Matrix S  <--- S\\
\endOutput
\beginOutput
i102 : Sbar = S / (ideal coeffOfFamily);\\
\endOutput
Next, we move to the Grassmannian\index{Grassmannian} $\mathbb{G}(1,3)
\subset \bbbp^{5}$.  Recall the homogeneous coordinates on
$\bbbp^{5}$ correspond to the $2 \times 2$ minors of a $2 \times 4$
matrix.  We obtain these minors using the {\tt exteriorPower} function
in \Mtwo.
\beginOutput
i103 : psi = matrix\{\{t\}\} | exteriorPower(2, \\
\                   matrix\{\{A, B, C, D\}, \{E, F, G, H\}\})\\
\emptyLine
o103 = | t -BE+AF -CE+AG -CF+BG -DE+AH -DF+BH -DG+CH |\\
\emptyLine
\                  1          7\\
o103 : Matrix Sbar  <--- Sbar\\
\endOutput
\beginOutput
i104 : PP5 = QQ[t, a..f];\\
\endOutput
\beginOutput
i105 : fanoOfFamily = trim ker map(Sbar, PP5, psi);\\
\emptyLine
o105 : Ideal of PP5\\
\endOutput
Now, to answer the question, we determine the limit as $t$ tends to $0$.
\beginOutput
i106 : zeroFibre = trim substitute(saturate(fanoOfFamily, t), \{t=>0\})\\
\emptyLine
\                         2   2                   2                     $\cdot\cdot\cdot$\\
o106 = ideal (e*f, d*f, e , f , d*e, a*e + b*f, d , c*d - b*e + a*f, b $\cdot\cdot\cdot$\\
\emptyLine
o106 : Ideal of PP5\\
\endOutput
Let's transpose the matrix of generators so all of its elements are visible
on the printed page.
\beginOutput
i107 : transpose gens zeroFibre\\
\emptyLine
o107 = \{-2\} | ef       |\\
\       \{-2\} | df       |\\
\       \{-2\} | e2       |\\
\       \{-2\} | f2       |\\
\       \{-2\} | de       |\\
\       \{-2\} | ae+bf    |\\
\       \{-2\} | d2       |\\
\       \{-2\} | cd-be+af |\\
\       \{-2\} | bd+ce    |\\
\       \{-2\} | ad-cf    |\\
\       \{-2\} | a2+b2+c2 |\\
\emptyLine
\                 11         1\\
o107 : Matrix PP5   <--- PP5\\
\endOutput
We see that $F_{1}(Q_{0})$ is supported on the plane conic $\langle d,
e, f, a^{2}+b^{2}+c^{2} \rangle$.  However, $F_{1}(Q_{0})$ is not
reduced\index{scheme!non-reduced}; it has
multiplicity\index{multiplicity} two.  On the other hand, the generic
fiber is
\beginOutput
i108 : oneFibre = trim substitute(saturate(fanoOfFamily, t), \{t => 1\})\\
\emptyLine
\                          2    2    2                                  $\cdot\cdot\cdot$\\
o108 = ideal (a*e + b*f, d  + e  + f , c*d - b*e + a*f, b*d + c*e, a*d $\cdot\cdot\cdot$\\
\emptyLine
o108 : Ideal of PP5\\
\endOutput
\beginOutput
i109 : oneFibre == intersect(ideal(c-d, b+e, a-f, d^2+e^2+f^2), \\
\            ideal(c+d, b-e, a+f, d^2+e^2+f^2))\\
\emptyLine
o109 = true\\
\endOutput
Hence, for $t \neq 0$, $F_{1}(Q_{t})$ is the union of two conics lying
in complementary planes and $F_{1}(Q_{0})$ is the double conic
obtained when the two conics move together.\qed
\end{solution*}

% Local Variables:
% mode: latex
% mode: reftex
% tex-main-file: "chapter-wrapper.tex"
% reftex-keep-temporary-buffers: t
% reftex-use-external-file-finders: t
% reftex-external-file-finders: (("tex" . "make FILE=%f find-tex") ("bib" . "make FILE=%f find-bib"))
% End:
