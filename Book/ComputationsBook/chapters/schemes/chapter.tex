\title{Teaching the Geometry of Schemes}
\titlerunning{Teaching the Geometry of Schemes}
\toctitle{Teaching the Geometry of Schemes}

\author{Gregory G.~Smith \and Bernd Sturmfels}
\authorrunning{G. G. Smith and B. Sturmfels}
% \institute{Department of Mathematics, University of California,
% Berkeley, California 94720, USA}

\maketitle


%%----------------------------------------------------------
\newtheorem*{problem*}{Problem}{\bfseries\upshape}{\itshape}
\newtheorem*{solution*}{Solution}{\itshape}{\rmfamily}

\newcommand{\Spec}{\operatorname{Spec}}
\newcommand{\Proj}{\operatorname{Proj}}
\newcommand{\codim}{\operatorname{codim}}
%%----------------------------------------------------------


\begin{abstract}
This chapter presents a collection of graduate level problems in
algebraic geometry illustrating the power of \Mtwo as an educational
tool.
\end{abstract}

When teaching an advanced subject, like the language of schemes, we
think it is important to provide plenty of concrete instances of the
theory.  Computer algebra systems, such as \Mtwo, provide students
with an invaluable tool for studying complicated examples.
Furthermore, we believe that the explicit nature of a computational
approach leads to a better understanding of the objects being
examined.  This chapter presents some problems which we feel
illustrate this point of view.

Our examples are selected from the homework of an algebraic geometry
class given at the University of California at Berkeley in the fall of 1999.
This graduate course was taught by the second author with assistance from the
first author.  Our choice of problems, as the title suggests, follows the
material in David Eisenbud and Joe Harris' textbook {\em The Geometry of
  Schemes} \cite{SC:EH}.

%%----------------------------------------------------------
\section{Distinguished Open Sets}

We begin with a simple example involving the Zariski topology of an affine
scheme\index{scheme!affine}. This example also indicates some of the
subtleties involved in working with arithmetic
schemes\index{scheme!arithmetic}.

\begin{problem*}
Let $S = \bbbz[x,y,z]$ and $X = \Spec(S)$.  If $f = x$ and $X_{f}$ is
the corresponding basic open subset in $X$, then establish the
following:
\begin{enumerate}
\item[$(1)$] If $e_{1} = x+y+z$, $e_{2} = xy+xz+yz$ and $e_{3} = xyz$
are the elementary symmetric functions then the set $\{X_{e_{i}}\}_{1
\leq i \leq 3}$ is an open cover of $X_{f}$.
\item[$(2)$] If $p_{1} = x+y+z$, $p_{2} = x^{2}+y^{2}+z^{2}$ and $p_{3}
= x^{3}+y^{3}+z^{3}$ are the power sum symmetric functions then
$\{X_{p_{i}}\}_{1 \leq i \leq 3}$ is {\em not} an open cover of
$X_{f}$.
\end{enumerate}
\end{problem*}

\begin{solution*}
$(1)$ To prove that $\{X_{e_{i}}\}_{1 \leq i \leq 3}$ is an open cover
of $X_{f}$, it suffices to show that $e_{1}$, $e_{2}$ and $e_{3}$
generate the unit ideal in $S_{f}$; see Lemma I-16 in Eisenbud and
Harris~\cite{SC:EH}.  This is equivalent to showing that $x^{m}$
belongs to the $S$-ideal $\langle e_{1}, e_{2}, e_{3} \rangle$ for
some $m \in \bbbn$.  In other words, the saturation\index{saturation}
$\big( \langle e_{1}, e_{2}, e_{3} \rangle : x^{\infty} \big)$ is the
unit ideal if and only if $\{X_{e_{i}}\}_{1 \leq i \leq 3}$ is an open
cover of $X_{f}$.  We verify this in \Mtwo as follows:
<<<S = ZZ[x, y, z];>>>
<<<elementaryBasis = ideal(x+y+z, x*y+x*z+y*z, x*y*z);>>>
<<<saturate(elementaryBasis, x)>>>
$(2)$ Similarly, to show that $\{X_{p_{i}}\}_{1 \leq i \leq 3}$ is not
an open cover of $X_{f}$, we prove that $\big( \langle p_{1}, p_{2},
p_{3} \rangle : x^{\infty} \big)$ is not the unit ideal.  Calculating
this saturation, we find
<<<powerSumBasis = ideal(x+y+z, x^2+y^2+z^2, x^3+y^3+z^3);>>>
<<<saturate(powerSumBasis, x)>>>
<<<clearAll>>>
which is not the unit ideal.\qed
\end{solution*}

The fact that $6$ is a generator of the ideal $\big( \langle p_{1},
p_{2}, p_{3} \rangle : x^{\infty} \big)$ indicates that
$\{X_{p_{i}}\}_{1 \leq i \leq 3}$ does not contain the points in $X$
lying over the points $\langle 2 \rangle$ and $\langle 3 \rangle$ in
$\Spec(\bbbz)$.  If we work over a base ring in which $6$ is a unit,
then $\{X_{p_{i}}\}_{1 \leq i \leq 3}$ would, in fact, be an open
cover of $X_{f}$.


%%----------------------------------------------------------
\section{Irreducibility}

The study of complex semisimple Lie algebras gives rise to an
important family of algebraic varieties called nilpotent
orbits\index{nilpotent orbits}.  The next problem examines the
irreducibility\index{scheme!irreducible} of a particular nilpotent
orbit.

\begin{problem*} 
Let $X$ be the set of nilpotent complex $3 \times 3$ matrices.  Show
that $X$ is an irreducible algebraic variety.
\end{problem*}

\begin{solution*}
A $3 \times 3$ matrix $M$ is nilpotent if and only if its minimal
polynomial $p(\sf T)$ equals ${\sf T}^{k}$, for some $k \in \bbbn$.
Since each irreducible factor of the characteristic polynomial of $M$
is also a factor of $p(\sf T)$, it follows that the characteristic
polynomial of $M$ is ${\sf T}^{3}$.  We conclude that the coefficients
of the characteristic polynomial of a generic $3 \times 3$ matrix
define the algebraic variety $X$.

To prove that $X$ is irreducible over $\bbbc$, we construct a rational
parameterization\index{rational parameterization}.  First, observe
that ${\rm GL}_{3}(\bbbc)$ acts on $X$ by conjugation.  Jordan's
canonical form theorem implies that there are exactly three orbits;
one for each of the following matrices:
\[
N_{(1,1,1)} =\left[ \begin{smallmatrix} 0 & 0 & 0 \\ 0 & 0 & 0 \\ 0 &
0 & 0 \end{smallmatrix} \right], \quad
N_{(2,1)} = \left[ \begin{smallmatrix} 0 & 1 & 0 \\ 0 & 0 & 0 \\ 0 & 0
& 0 \end{smallmatrix} \right] \text{ and }
N_{(3)} = \left[ \begin{smallmatrix} 0 & 1 & 0 \\ 0 & 0 & 1 \\ 0 & 0 &
0 \end{smallmatrix} \right] \enspace .
\]
Each orbit is defined by a rational parameterization, so it suffices
to show that the closure of the orbit containing $N_{(3)}$ is the
entire variety $X$.  We demonstrate this as follows:
<<<S = QQ[t, y_0 .. y_8, a..i, MonomialOrder => Eliminate 10];>>>
<<<N3 = (matrix {{0,1,0},{0,0,1},{0,0,0}}) ** S>>>
<<<G = genericMatrix(S, y_0, 3, 3)>>>
To determine the entries in $G \cdot N_{(3)} \cdot G^{-1}$, we use the
classical adjoint\index{classical adjoint} to construct the matrix
$\det(G) \cdot G^{-1}$.
<<<classicalAdjoint = (G) -> (
     n := degree target G;
     m := degree source G;
     matrix table(n, n, (i, j) -> (-1)^(i+j) * det(
               submatrix(G, {0..j-1, j+1..n-1}, 
                    {0..i-1, i+1..m-1}))));>>>
<<<num = G * N3 * classicalAdjoint(G);>>>
<<<D = det(G);>>>
<<<M = genericMatrix(S, a, 3, 3);>>>
The entries in $G \cdot N_{(3)} \cdot G^{-1}$ give a rational
parameterization of the orbit generated by $N_{(3)}$.  Using
elimination theory\index{elimination theory} --- see section~3.3 in
Cox, Little and O`Shea~\cite{SC:CLO} --- we give an ``implicit
representation'' of this variety.
<<<elimIdeal = minors(1, (D*id_(S^3))*M - num) + ideal(1-D*t);>>>
<<<closureOfOrbit = ideal selectInSubring(1, gens gb elimIdeal);>>>

Finally, we verify that this orbit closure equals $X$
scheme-theoretically.  Recall that $X$ is defined by the coefficients
of the characteristic polynomial of a generic $3 \times 3$ matrix {\tt
M}.
%% was X = ideal submatrix( (coeff-icients({0}, det(M - t*id_(S^3))))_1,
%% {1,2,3} ), but 'coeff-icients' is to be redesigned, and 'contract' is
%% more self-explanatory, anyway.
<<<X = ideal substitute(
        contract(matrix{{t^2,t,1}}, det(t-M)),
        {t => 0_S})>>>
<<<closureOfOrbit == X>>>
<<<clearAll>>>
This completes our solution.\qed
\end{solution*}

More generally, Kostant shows that the set of all nilpotent elements
in a complex semisimple Lie algebra\index{Lie algebra} form an
irreducible variety.  We refer the reader to Chriss and
Ginzburg~\cite{SC:CV} for a proof of this result (Corollary~3.2.8) and
a discussion of its applications in representation theory.


%%----------------------------------------------------------
\section{Singular Points}

In our third question, we study the singular locus\index{singular
locus} of a family of elliptic curves\index{elliptic curve}.

\begin{problem*} 
Consider a general form of degree $3$ in $\bbbq[x,y,z]$:
\[
F = ax^{3} + bx^{2}y + cx^{2}z + dxy^{2} + exyz + fxz^{2} + gy^{3} +
hy^{2}z + iyz^{2} + jz^{3} \enspace .
\]
Give necessary and sufficient conditions in terms of $a, \ldots, j$
for the cubic curve $\Proj\big( \bbbq[x,y,z] / \langle F \rangle
\big)$ to have a singular point.
\end{problem*}

\begin{solution*}
The singular locus of $F$ is defined by a polynomial of degree $12$ in
the $10$ variables $a, \dotsc, j$.  We calculate this polynomial in two
different ways.

Our first method is an elementary but time consuming elimination.
Carrying it out in \Mtwo, we have
<<<S = QQ[x, y, z, a..j, MonomialOrder => Eliminate 2];>>>
<<<F = a*x^3+b*x^2*y+c*x^2*z+d*x*y^2+e*x*y*z+f*x*z^2+g*y^3+h*y^2*z+
             i*y*z^2+j*z^3;>>>
<<<partials = submatrix(jacobian matrix{{F}}, {0..2}, {0})>>>
<<<singularities = ideal(partials) + ideal(F);>>>
<<<elimDiscr = time ideal selectInSubring(1,gens gb singularities);>>>
<<<elimDiscr = substitute(elimDiscr, {z => 1});>>>
On the other hand, there is also an elegant and more useful
determinantal formula for this discriminant\index{discriminant}; it is
a specialization of the formula (2.8) in section~3.2 of Cox, Little
and O`Shea~\cite{SC:CLO2}.  To apply this determinantal formula, we
first create the coefficient matrix {\tt A} of the partial derivatives
of $F$.
%% was A = (coeff-icients({0,1,2}, submatrix(jacobian matrix{{F}}, {0..2}, {0})))_1;
%% but 'coeff-icients' is deprecated.
<<<A = contract(matrix{{x^2,x*y,y^2,x*z,y*z,z^2}},
        diff(transpose matrix{{x,y,z}},F))>>>
We also construct the coefficient matrix {\tt B} of the partial
derivatives of the Hessian\index{hessian} of $F$.
<<<hess = det submatrix(jacobian ideal partials, {0..2}, {0..2});>>>
%% was B = (coeff-icients({0,1,2}, submatrix(jacobian matrix{{hess}}, {0..2}, {0})))_1;
%% but 'coeff-icients' is deprecated.
<<<B = contract(matrix{{x^2,x*y,y^2,x*z,y*z,z^2}},
        diff(transpose matrix{{x,y,z}},hess))>>>
To obtain the discriminant, we combine these two matrices and take the
determinant.
<<<detDiscr = ideal det (A || B);>>>
Finally, we check that our two discriminants are equal
<<<detDiscr == elimDiscr>>>
and examine the generator.
<<<detDiscr_0>>>
<<<numgens detDiscr>>>
<<<# terms detDiscr_0>>>
<<<clearAll>>>
Hence, the singular locus is given by a single polynomial of degree
$12$ with $2040$ terms.\qed
\end{solution*}

For a further discussion of singularities and discriminants see
Section~V.3 in Eisenbud and Harris~\cite{SC:EH}.  For information on
resultants and discriminants see Chapter~2 in Cox, Little and
O`Shea~\cite{SC:CLO2}.


%%----------------------------------------------------------
\section{Fields of Definition}

Schemes\index{scheme!over a number field} over non-algebraically
closed fields arise in number theory.  Our fourth problem looks at one
technique for working with number fields in \Mtwo.

\begin{problem*}[Exercise~II-6 in  \cite{SC:EH}]
An inclusion of fields $K \hookrightarrow L$ induces a map
$\mathbb{A}_{L}^{n} \to \mathbb{A}_{K}^{n}$.  Find the images in
$\mathbb{A}_{\bbbq}^{2}$ of the following points of
$\mathbb{A}_{\overline{\bbbq}}^{2}$ under this map.
\begin{enumerate}
\item[$(1)$] $\langle x - \sqrt{2}, y - \sqrt{2} \rangle ;$
\item[$(2)$] $\langle x - \sqrt{2}, y - \sqrt{3} \rangle ;$
\item[$(3)$] $\langle x - \zeta, y - \zeta^{-1} \rangle$ where $\zeta$
is a $5$-th root of unity $;$
\item[$(4)$] $\langle \sqrt{2}x- \sqrt{3}y \rangle ;$
\item[$(5)$] $\langle \sqrt{2}x- \sqrt{3}y-1 \rangle$.
\end{enumerate}
\end{problem*}

\begin{solution*}
The images can be determined by using the following three step
algorithm: (1) replace the coefficients not contained in $K$ with
indeterminates, (2) add the minimal polynomials of these coefficients
to the given ideal in $\mathbb{A}_{L}^{2}$, and (3) eliminate the new
indeterminates.  Here are the five examples:
<<<S = QQ[a,b,x,y, MonomialOrder => Eliminate 2];>>>
<<<I1 = ideal(x-a, y-a, a^2-2);>>>
<<<ideal selectInSubring(1, gens gb I1)>>>
<<<I2 = ideal(x-a, y-b, a^2-2, b^2-3);>>>
<<<ideal selectInSubring(1, gens gb I2)>>>
<<<I3 = ideal(x-a, y-a^4, a^4+a^3+a^2+a+1);>>>
<<<ideal selectInSubring(1, gens gb I3)>>>
<<<I4 = ideal(a*x+b*y, a^2-2, b^2-3);>>>
<<<ideal selectInSubring(1, gens gb I4)>>>
<<<I5 = ideal(a*x+b*y-1, a^2-2, b^2-3);>>>
<<<ideal selectInSubring(1, gens gb I5)>>>
<<<clearAll>>>
\qed
\end{solution*}

It is worth noting that the points in $\mathbb{A}_{\bbbq}^{n}$ correspond
to orbits of the action of ${\rm Gal}(\overline{\bbbq}/\bbbq)$ on the
points of $\mathbb{A}_{\overline{\bbbq}}^{n}$.  For more examples and
information, see section~II.2 in Eisenbud and Harris~\cite{SC:EH}.


%%----------------------------------------------------------
\section{Multiplicity}

The multiplicity\index{multiplicity} of a zero-dimensional scheme $X$
at a point $p \in X$ is defined to be the length of the local ring
$\mathcal{O}_{X,p}$.  Unfortunately, we cannot work directly in the
local ring in \Mtwo.  What we can do, however, is to compute the
multiplicity by computing the degree of the component of $X$ supported
at $p$; see page 66 in Eisenbud and Harris~\cite{SC:EH}.

\begin{problem*}
What is the multiplicity of the origin as a zero of the polynomial
equations $x^{5}+y^{3}+z^{3} = x^{3}+y^{5}+z^{3} = x^{3}+y^{3}+z^{5} =
0$?
\end{problem*}

\begin{solution*}
If $I$ is the ideal generated by $x^{5}+y^{3}+z^{3}$,
$x^{3}+y^{5}+z^{3}$ and $x^{3}+y^{3}+z^{5}$ in $\bbbq[x,y,z]$, then
the multiplicity of the origin is
\[
\dim_{\bbbq} \frac{\bbbq[x,y,z]_{\langle x,y,z \rangle}}
{I \bbbq[x,y,z]_{\langle x,y,z \rangle}} \, .
\]
It follows that the multiplicity is the vector space dimension of the
ring $\bbbq[x,y,z] / \varphi^{-1}(I \bbbq[x,y,z]_{\langle x,y,z
\rangle})$ where $\varphi \colon \bbbq[x,y,z] \to
\bbbq[x,y,z]_{\langle x,y,z \rangle}$ is the natural map.  Moreover,
we can express this using ideal quotients:
\[
\varphi^{-1}(I \bbbq[x,y,z]_{\langle x,y,z \rangle}) \,\,= \,\,
\big(I : (I : \langle x,y,z \rangle^{\infty})\big) \, .
\]
Carrying out this calculation in \Mtwo, we obtain:
<<<S = QQ[x, y, z];>>>
<<<I = ideal(x^5+y^3+z^3, x^3+y^5+z^3, x^3+y^3+z^5);>>>
<<<multiplicity = degree(I : saturate(I))>>>
<<<clearAll>>>
Thus, we conclude that the multiplicity is $27$.\qed
\end{solution*}

There are algorithms (not yet implemented in \Mtwo) for working
directly in the local ring $\bbbq[x,y,z]_{\langle x,y,z \rangle}$.  We
refer the interested reader to Chapter~4 in Cox, Little and
O`Shea~\cite{SC:CLO2}.


%%----------------------------------------------------------
\section{Flat Families}

Non-reduced schemes\index{scheme!non-reduced} arise naturally as flat
limits\index{flat limit} of a family of reduced
schemes\index{scheme!reduced}. Our next problem illustrates how a
family of skew lines in $\bbbp^{3}$ gives rise to a double line with
an embedded point\index{embedded point}.

\begin{problem*}[Exercise~III-68 in \cite{SC:EH}]
Let $L$ and $M $ be the lines in $\bbbp^{3}_{k[t]}$ given by $x=y=0$
and $x-tz = y+t^{2}w =0$ respectively.  Show that the flat limit as $t
\to 0$ of the union $L \cup M$ is the double line $x^{2} = y = 0$ with
an embedded point of degree $1$ located at the point $(0:0:0:1)$.
\end{problem*}

\begin{solution*}
We first find the flat limit by saturating\index{saturation} the
intersection ideal and setting $t = 0$.
<<<PP3 = QQ[t, x, y, z, w];>>>
<<<L = ideal(x, y);>>>
<<<M = ideal(x-t*z, y+t^2*w);>>>
<<<X = intersect(L, M);>>>
<<<Xzero = trim substitute(saturate(X, t), {t => 0})>>>
Secondly, we verify that this is the union of a double line and an
embedded point of degree $1$.
<<<Xzero == intersect(ideal(x^2, y), ideal(x, y^2, z))>>>
<<<degree(ideal(x^2, y ) / ideal(x, y^2, z))>>>
<<<clearAll>>>
\qed
\end{solution*}

Section~III.3.4 in Eisenbud and Harris~\cite{SC:EH} contains several
other interesting limits of various flat families.


%%----------------------------------------------------------
\section{B\'{e}zout's Theorem}

B\'{e}zout's Theorem\index{Bezout's Theorem@B\'ezout's Theorem} --- Theorem~III-78 in
Eisenbud and Harris~\cite{SC:EH} --- may fail without the
Cohen-Macaulay\index{Cohen-Macaulay} hypothesis.  Our seventh problem
is to demonstrate this.

\begin{problem*}[Exercise~III-81 in \cite{SC:EH}]
Find irreducible closed subvarieties $X$ and $Y$ in $\bbbp^{4}$ such
that 
\begin{align*}
\codim(X \cap Y) &= \codim(X) + \codim(Y) \\
\deg(X \cap Y) &> \deg(X) \cdot \deg(Y) \, .
\end{align*}
\end{problem*}

\begin{solution*}
We show that the assertion holds when $X$ is the cone over the
nonsingular rational quartic curve\index{rational quartic curve} in
$\bbbp^{3}$ and $Y$ is a two-plane passing through the vertex of the
cone.  First, recall that the rational quartic curve is given by the
$2 \times 2$ minors of the matrix $\left[ \begin{smallmatrix} a &
b^{2} & bd & c \\ b & ac & c^2 & d \end{smallmatrix} \right]$; see
Exercise~18.8 in Eisenbud~\cite{SC:E}.  Thus, we have
<<<S = QQ[a, b, c, d, e];>>>
<<<IX = trim minors(2, matrix{{a, b^2, b*d, c},{b, a*c, c^2, d}})>>>
<<<IY = ideal(a, d);>>>
<<<codim IX + codim IY == codim (IX + IY)>>>
<<<(degree IX) * (degree IY)>>>
<<<degree (IX + IY)>>>
which establishes the assertion.\qed
\end{solution*}

To understand how this example works, it is enlightening to express
$Y$ as the intersection of two hyperplanes; one given by $a = 0$ and
the other given by $d = 0$.  Intersecting $X$ with the first
hyperplane yields
<<<J = ideal mingens (IX + ideal(a))>>>
However, this first intersection has an embedded point;
<<<J == intersect(ideal(a, b*c, b^2, c^3-b*d^2), 
     ideal(a, d, b*c, c^3, b^3)) -- embedded point>>>
<<<clearAll>>>
The second hyperplane passes through this embedded
point\index{embedded point} which explains the extra intersection.


%%----------------------------------------------------------
\section{Constructing Blow-ups}

The blow-up\index{blow-up} of a scheme $X$ along a subscheme $Y$ can
be constructed from the Rees algebra\index{Rees algebra} associated to
the ideal sheaf of $Y$ in $X$; see Theorem~IV-22 in Eisenbud and
Harris~\cite{SC:EH}.  Gr\"{o}bner basis techniques allow one to
express the Rees algebra in terms of generators and relations.  We
illustrate this method in the next solution.

\begin{problem*}[Exercises~IV-43 \& IV-44 in \cite{SC:EH}]
Find the blow-up $X$ of the affine plane\index{scheme!affine}
$\mathbb{A}^{2} = \Spec\big( \bbbq[x, y] \big)$ along the subscheme
defined by $\langle x^{3}, xy, y^{2} \rangle$.  Show that $X$ is
nonsingular and its fiber over the origin is the union of two copies
of $\bbbp^{1}$ meeting at a point.
\end{problem*}

\begin{solution*}
We first provide a general function which returns the ideal of
relations for the Rees algebra.
<<<blowUpIdeal = (I) -> (
     r := numgens I;
     S := ring I;
     n := numgens S;
     K := coefficientRing S;
     tR := K[t, gens S, vars(0..r-1), 
               MonomialOrder => Eliminate 1];
     f := map(tR, S, submatrix(vars tR, {1..n}));
     F := f(gens I);
     J := ideal apply(1..r, j -> (gens tR)_(n+j)-t*F_(0,(j-1)));
     L := ideal selectInSubring(1, gens gb J);
     R := K[gens S, vars(0..r-1)];
     g := map(R, tR, 0 | vars R);
     trim g(L));>>>
Now, applying the function to our specific case yields: 
<<<S = QQ[x, y];>>>
<<<I = ideal(x^3, x*y, y^2);>>>
<<<J = blowUpIdeal(I)>>>
Therefore, the blow-up of the affine plane along the given subscheme
is
\[
X = \Proj\left( \frac{(\bbbq[x,y])[a,b,c]}{\langle yb-xc, xb^{2}-ac,
x^{2}b-ya, x^{3}c-y^{2}a \rangle} \right) \, .
\]
Using \Mtwo, we can also verify that the scheme $X$ is
nonsingular\index{singular locus};
<<<J + ideal jacobian J == ideal gens ring J>>>
<<<clearAll>>>
Since we have
\[
\frac{(\bbbq[x,y])[a,b,c]}{\langle yb-xc, xb^{2}-ac, x^{2}b-ya,
x^{3}c-y^{2}a \rangle} \otimes \frac{\bbbq[x,y]}{\langle x, y \rangle}
\cong \frac{\bbbq[a,b,c]}{\langle ac \rangle} \, ,
\]
the fiber over the origin $\langle x,y \rangle$ in $\mathbb{A}^{2}$ is
clearly a union of two copies of $\bbbp^{1}$ meeting at one point.  In
particular, the exceptional fiber is not a projective space.\qed
\end{solution*}

Many other interesting blow-ups can be found in section~II.2 in
Eisenbud and Harris~\cite{SC:EH}.


%%----------------------------------------------------------
\section{A Classic Blow-up}

We consider the blow-up\index{blow-up} of the projective plane
$\bbbp^{2}$ at a point.

\vbox{
\begin{problem*}
Show that the following varieties are isomorphic.
\begin{enumerate}
\item[$(a)$] the image of the rational map from $\bbbp^{2}$ to
$\bbbp^{4}$ given by
\[
(r:s:t) \mapsto (r^{2}:s^{2}:rs:rt:st) \, ;
\]
\item[$(b)$] the blow-up of the plane $\bbbp^{2}$ at the point
$(0:0:1)$;
\item[$(c)$] the determinantal variety\index{determinantal variety}
defined by the $2 \times 2$ minors of the matrix $\left[
\begin{smallmatrix} a & c & d \\ b & d & e \end{smallmatrix} \right]$
where $\bbbp^{4} = \Proj\big( k[a,b,c,d,e] \big)$.
\end{enumerate}
This surface is called the {\em cubic scroll}\index{cubic scroll} in
$\bbbp^{4}$.
\end{problem*}
}

\begin{solution*}
We find the ideal in part~$(a)$ by elimination
theory\index{elimination theory}.
<<<PP4 = QQ[a..e];>>>
<<<S = QQ[r..t, A..E, MonomialOrder => Eliminate 3];>>>
<<<I = ideal(A - r^2, B - s^2, C - r*s, D - r*t, E - s*t);>>>
<<<phi = map(PP4, S, matrix{{0_PP4, 0_PP4, 0_PP4}} | vars PP4)>>>
<<<surfaceA = phi ideal selectInSubring(1, gens gb I)>>>
Next, we determine the surface in part~$(b)$.  We construct the ideal
defining the blow-up of $\bbbp^{2}$ 
<<<R = QQ[t, x, y, z, u, v, MonomialOrder => Eliminate 1];>>>
<<<blowUpIdeal = ideal selectInSubring(1, gens gb ideal(u-t*x, 
     v-t*y))>>>
and embed it in $\bbbp^{2} \times \bbbp^{1}$.
<<<PP2xPP1 = QQ[x, y, z, u, v];>>>
<<<embed = map(PP2xPP1, R, 0 | vars PP2xPP1);>>>
<<<blowUp = PP2xPP1 / embed(blowUpIdeal);>>>
We then map this surface into $\bbbp^{5}$ using the Segre
embedding\index{Segre embedding}.
<<<PP5 = QQ[A .. F];>>>
<<<segre = map(blowUp, PP5, matrix{{x*u,y*u,z*u,x*v,y*v,z*v}});>>>
<<<ker segre>>>
Note that the image under the Segre map lies on a hyperplane in
$\bbbp^{5}$.  To get the desired surface in $\bbbp^{4}$, we project
<<<projection = map(PP4, PP5, matrix{{a, c, d, c, b, e}})>>>
<<<surfaceB = trim projection ker segre>>>
Finally, we compute the surface in part~$(c)$.
<<<determinantal = minors(2, matrix{{a, c, d}, {b, d, e}})>>>
<<<sigma = map( PP4, PP4, matrix{{d, e, a, c, b}});>>>
<<<surfaceC = sigma determinantal>>>
By incorporating a permutation of the variables into definition of
{\tt surfaceC}, we obtain the desired isomorphisms
<<<surfaceA == surfaceB>>>
<<<surfaceB == surfaceC>>>
<<<clearAll>>>
which completes the solution.\qed
\end{solution*}

For more information of the geometry of rational normal scrolls, see
Lecture~8 in Harris~\cite{SC:H}.


%%----------------------------------------------------------
\section{Fano Schemes}

Our final example concerns the family of Fano schemes\index{Fano
scheme} associated to a flat family of quadrics.
Recall that the $k$-th Fano scheme $F_{k}(X)$ of a
scheme $X \subseteq \bbbp^{n}$ is the subscheme of
the Grassmannian parametrizing $k$-planes
contained in $X$.

\begin{problem*}[Exercise~IV-69 in \cite{SC:EH}]
Consider the one-parameter family\index{one-parameter family} of
quadrics tending to a double plane with equation
\[ 
Q = V(tx^{2}+ty^{2}+tz^{2}+w^{2}) \subseteq \bbbp^{3}_{\bbbq[t]} =
\Proj\big(\bbbq[t][x,y,z,w]\big) \enspace .
\]
What is the flat limit\index{flat limit} of the Fano schemes
$F_{1}(Q_{t})$?
\end{problem*}

\begin{solution*}
We first compute the ideal defining $F_{1}(Q_{t})$, the scheme
parametrizing lines in $Q$.
<<<PP3 = QQ[t, x, y, z, w];>>>
<<<Q = ideal( t*x^2+t*y^2+t*z^2+w^2 );>>>
To parametrize a line in our projective space, we introduce
indeterminates $u, v$ and $A, \dotsc, H$.
<<<R = QQ[t, u, v, A .. H];>>>
We then make a map {\tt phi} from {\tt PP3} to {\tt R} sending the
variables to the coordinates of the general point on a line.
<<<phi = map(R, PP3, matrix{{t}} | 
        u*matrix{{A, B, C, D}} + v*matrix{{E, F, G, H}});>>>
<<<imageFamily = phi Q;>>>
For a line to belong to $Q$, the {\tt imageFamily} must vanish
identically.  In other words, $F_{1}(Q)$ is defined by the
coefficients of the generators of {\tt imageFamily}.
%% removing a final use of 'coefficients'
%% coeffOfFamily = (coefficients ({1,2}, gens imageFamily))_1;
<<<coeffOfFamily = contract(matrix{{u^2,u*v,v^2}}, gens imageFamily)>>>
Since we don't need the variables $u$ and $v$, we get rid of them.
<<<S = QQ[t, A..H];>>>
<<<coeffOfFamily = substitute(coeffOfFamily, S);>>>
<<<Sbar = S / (ideal coeffOfFamily);>>>
Next, we move to the Grassmannian\index{Grassmannian} $\mathbb{G}(1,3)
\subset \bbbp^{5}$.  Recall the homogeneous coordinates on
$\bbbp^{5}$ correspond to the $2 \times 2$ minors of a $2 \times 4$
matrix.  We obtain these minors using the {\tt exteriorPower} function
in \Mtwo.
<<<psi = matrix{{t}} | exteriorPower(2, 
            matrix{{A, B, C, D}, {E, F, G, H}})>>>
<<<PP5 = QQ[t, a..f];>>>
<<<fanoOfFamily = trim ker map(Sbar, PP5, psi);>>>
Now, to answer the question, we determine the limit as $t$ tends to $0$.
<<<zeroFibre = trim substitute(saturate(fanoOfFamily, t), {t=>0})>>>
Let's transpose the matrix of generators so all of its elements are visible
on the printed page.
<<<transpose gens zeroFibre>>>
We see that $F_{1}(Q_{0})$ is supported on the plane conic $\langle d,
e, f, a^{2}+b^{2}+c^{2} \rangle$.  However, $F_{1}(Q_{0})$ is not
reduced\index{scheme!non-reduced}; it has
multiplicity\index{multiplicity} two.  On the other hand, the generic
fiber is
<<<oneFibre = trim substitute(saturate(fanoOfFamily, t), {t => 1})>>>
<<<oneFibre == intersect(ideal(c-d, b+e, a-f, d^2+e^2+f^2), 
     ideal(c+d, b-e, a+f, d^2+e^2+f^2))>>>
Hence, for $t \neq 0$, $F_{1}(Q_{t})$ is the union of two conics lying
in complementary planes and $F_{1}(Q_{0})$ is the double conic
obtained when the two conics move together.\qed
\end{solution*}

% Local Variables:
% mode: latex
% mode: reftex
% tex-main-file: "chapter-wrapper.tex"
% reftex-keep-temporary-buffers: t
% reftex-use-external-file-finders: t
% reftex-external-file-finders: (("tex" . "make FILE=%f find-tex") ("bib" . "make FILE=%f find-bib"))
% End:
