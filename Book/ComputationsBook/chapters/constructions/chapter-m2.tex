% $Source: /home/cvs/M2/Macaulay2/ComputationsBook/chapters/constructions/chapter.tex,v $
% $Revision: 1.30 $
% $Date: 2001/03/16 22:03:06 $

\title{Needles in a Haystack:\\Special Varieties via Small Fields}
\titlerunning{Needles in a Haystack: Special Varieties via Small Fields}
\toctitle{Needles in a Haystack: Special Varieties via Small Fields}
\author{Frank-Olaf Schreyer
        % \inst 1
         \and Fabio Tonoli
        % \inst 2
        }
\authorrunning{F-O. Schreyer and F. Tonoli}
% \institute{Fakult\"at f\"ur Mathematik und Physik,
%         Universit\"at Bayreuth, D-95440 Bayreuth, Germany
%         \and
%         Mathematisches Institut, Georg--August Universit\"at, D-37073
%         G\"ottingen, Germany}

% \newtheorem{lemma}{Lemma}[section]
% \newtheorem{proposition}[lemma]{Proposition}
% \newtheorem{theorem}[lemma]{Theorem}
% \newtheorem{corollary}[lemma]{Corollary}
\newtheorem{conjecture}[theorem]{Conjecture}{\itshape}{\rm}

% \theoremstyle{definition}
% \newtheorem{definition}[theorem]{Definition}
% \newtheorem{remark}{Remark}[section]
% \newtheorem{example}{Example}[section]
% \newtheorem{exercise}{Exercise}[section]
% \newtheorem{algorithm}{Algorithm}[subsection]
% \newtheorem{sub}[subsubsection]{}


\newcommand{\AAA}{{\mathbb A}}
\newcommand{\BB}{{\mathbb B}}
\newcommand{\CC}{{\mathbb C}}
\newcommand{\DD}{{\mathbb D}}
\newcommand{\EE}{{\mathbb E}}
\newcommand{\FF}{{\mathbb F}}
\newcommand{\GG}{{\mathbb G}}
%\newcommand{\HH}{{\mathbb H}}
\newcommand{\II}{{\mathbb I}}
\newcommand{\JJ}{{\mathbb J}}
\newcommand{\KK}{{\mathbb K}}
%\newcommand{\LL}{{\mathbb L}}
\newcommand{\MM}{{\mathbb M}}
\newcommand{\NN}{{\mathbb N}}
%\newcommand{{\mathbb O}}
\newcommand{\PP}{{\mathbb P}}
\newcommand{\QQ}{{\mathbb Q}}
\newcommand{\RR}{{\mathbb R}}
\renewcommand{\SS}{{\mathbb S}}
\newcommand{\Ss}{{\mathbf S}}
\newcommand{\TT}{{\mathbb T}}
\newcommand{\UU}{{\mathbb U}}
\newcommand{\VV}{{\mathbb V}}
\newcommand{\WW}{{\mathbb W}}
\newcommand{\XX}{{\mathbb X}}
\newcommand{\YY}{{\mathbb Y}}
\newcommand{\ZZ}{{\mathbb Z}}

\newcommand{\LL}{{\mathbb L}}
\newcommand{\HH}{{\mathbb H}}
\newcommand{\Syz}{{\rm{Syz}\;}}
\newcommand{\Sym}{{\rm{Sym}\;}}
\newcommand{\SSyz}{{\rm{Syz}}}
\newcommand{\spoly}{{\rm{spoly}}}
\newcommand{\Spe}{{Sp}}
\newcommand{\openC}{{\mathbb C}}
\newcommand{\ms}{{\rm{m}}}
\newcommand{\LS}{{\rm{L}}}
\newcommand{\IS}{{\rm{I}}}
\newcommand{\Loc}{{\rm{Loc}\,}}
\newcommand{\lcm}{{\rm{lcm}}}
\newcommand{\lc}{{\rm{lc}}}
\newcommand{\lm}{{\rm{lm}}}
\newcommand{\con}{{\rm{c}}}
\newcommand{\ext}{{\rm{e}}}
\newcommand{\ec}{{\rm{ec}}}
\newcommand{\ann}{{\rm{ann}}}
% \newcommand{\Ext}{{\rm{Ext}}}
\newcommand{\equi}{{\rm{equi}}}
\newcommand{\Tor}{{\rm{Tor}}}
\newcommand{\rad}{{\rm{rad\;}}}
\newcommand{\ini}{{\rm{in}}}
\newcommand{\Hilb}{{\rm{Hilb}}}
\newcommand{\image}{{\rm{image}}}
\newcommand{\cliff}{{\rm{cliff}}}
\newcommand{\Pic}{{\rm{Pic}}}
\newcommand{\PR}{{\KK [x_1, \dots , x_n]}}
%%%%%%%%%%%%%%%%%%%%%%%%%%%%%
%%% new commands for calligraphic characters with amsmath
%%%%%%%%%%%%%%%%%%%%%%%%%%%%

\newcommand{\ka}{{\mathcal A}}
\newcommand{\kb}{{\mathcal B}}
\newcommand{\kc}{{\mathcal C}}
\newcommand{\kd}{{\mathcal D}}
\newcommand{\ke}{{\mathcal E}}
\newcommand{\kf}{{\mathcal F}}
\newcommand{\kg}{{\mathcal G}}
\newcommand{\kh}{{\mathcal H}}
\newcommand{\ki}{{\mathcal I}}
\newcommand{\kj}{{\mathcal J}}
\newcommand{\kk}{{\mathcal K}}
\newcommand{\kl}{{\mathcal L}}
\newcommand{\km}{{\mathcal M}}
\newcommand{\kn}{{\mathcal N}}
\newcommand{\ko}{{\mathcal O}}
\newcommand{\kp}{{\mathcal P}}
\newcommand{\kq}{{\mathcal Q}}
\newcommand{\kr}{{\mathcal R}}
\newcommand{\ks}{{\mathcal S}}
\newcommand{\kt}{{\mathcal T}}
\newcommand{\ku}{{\mathcal U}}
\newcommand{\kv}{{\mathcal V}}
\newcommand{\kw}{{\mathcal W}}
\newcommand{\kx}{{\mathcal X}}
\newcommand{\ky}{{\mathcal Y}}
\newcommand{\kz}{{\mathcal Z}}
%%%%%%%%%%%%%%%%%%%%%%%%%%%%%%
%%%The mathscript for sheaves
%%%%%%%%%%%%%%%%%%%%% %%%%%%%%%
\newcommand{\s}{\mathscr}
\newcommand{\sA}{{\s A}}
\newcommand{\sB}{{\s B}}
\newcommand{\sC}{{\s C}}
\newcommand{\sD}{{\s D}}
\newcommand{\sE}{{\s E}}
\newcommand{\sF}{{\s F}}
\newcommand{\sG}{{\s G}}
\newcommand{\sH}{{\s H}}
\newcommand{\sI}{{\s I}}
\newcommand{\sJ}{{\s J}}
\newcommand{\sK}{{\s K}}
\newcommand{\sL}{{\s L}}
\newcommand{\sM}{{\s M}}
\newcommand{\sN}{{\s N}}
\newcommand{\sO}{{\s O}}
\newcommand{\sP}{{\s P}}
\newcommand{\sQ}{{\s Q}}
\newcommand{\sR}{{\s R}}
\newcommand{\sS}{{\s S}}
\newcommand{\sT}{{\s T}}
\newcommand{\sU}{{\s U}}
\newcommand{\sV}{{\s V}}
\newcommand{\sW}{{\s W}}
\newcommand{\sX}{{\s X}}
\newcommand{\sY}{{\s Y}}
\newcommand{\sZ}{{\s Z}}



\newcommand{\cO}{{\s O}}
\newcommand{\cI}{{\s I}}
% \newcommand{\cL}{{\s L}}
% \newcommand{\cR}{{\s R}}
\newcommand{\cN}{{\s N}}
\newcommand{\cT}{{\s T}}
\newcommand{\cX}{{\s X}}
%%%%%%%%%%%%%%%%%%%%%%%%%%%%%%%%
%% Arrows
%%%%%%%%%%%%%%%%%%%%%%%%%%%%%%%
\newcommand{\inj}{\hookrightarrow}
%\newcommand{\surj}{\lra}
\newcommand{\lra}{\longrightarrow}
\newcommand{\lla}{\longleftarrow}
\def\lto{\leftarrow}
%%%%%%%%%%%%%%%%%%%%%%%%%%%%%%%%%%%%
%\newcommand{\C}{\C}
%\newcommand{\openP}{\P}
\newcommand{\uf}{{\bf F}}
\newcommand{\uc}{{\bf C}}
\newcommand{\tensor}{\otimes}
\newcommand{\mi}{{\bf m}}
\newcommand{\tX}{\widetilde{X}}
\newcommand{\punkt}{\hspace{-.3ex}\raise.15ex\hbox to1ex{\Huge.}}
\newcommand{\tpunkt}{\hspace{-.3ex}\hbox to1ex{\Huge.}}
% \newlength{\br}
% \newlength{\ho}
\def\DeclareMathOperator#1#2{\def#1{\operatorname{#2}}}
\DeclareMathOperator{\det}{det}
\DeclareMathOperator{\GL}{GL}
\DeclareMathOperator{\Aut}{Aut}
\DeclareMathOperator{\Oo}{O}
\DeclareMathOperator{\Spec}{Spec}
\DeclareMathOperator{\Hom}{Hom}
\DeclareMathOperator{\syz}{syz}
\DeclareMathOperator{\ord}{ord}
\DeclareMathOperator{\word}{w\,ord}
\DeclareMathOperator{\supp}{supp}
\DeclareMathOperator{\Ker}{Ker}
\DeclareMathOperator{\im}{im}
%\DeclareMathOperator{\wdeg}{w\,deg}
\DeclareMathOperator{\depth}{depth}
\DeclareMathOperator{\gin}{gin}
\DeclareMathOperator{\Coker}{Coker}
\DeclareMathOperator{\NF}{NF}
\DeclareMathOperator{\pd}{pd}
\DeclareMathOperator{\SL}{SL}
\DeclareMathOperator{\SO}{SO}
\DeclareMathOperator{\Ort}{O}
\DeclareMathOperator{\Spez}{Sp}
\DeclareMathOperator{\PSL}{PSL}
\DeclareMathOperator{\PGL}{PGL}
\DeclareMathOperator{\wdim}{wdim}
\DeclareMathOperator{\cdim}{cdim}
\DeclareMathOperator{\cha}{char}
\DeclareMathOperator{\trdeg}{trdeg}
\DeclareMathOperator{\codim}{codim}
\DeclareMathOperator{\kdim}{kdim}
\DeclareMathOperator{\height}{height}
\DeclareMathOperator{\Ass}{Ass}
\DeclareMathOperator{\Lie}{Lie}
\DeclareMathOperator{\rk}{rk}


\renewcommand{\labelenumi}{(\arabic{enumi})}
\newcommand{\Ndash}{\nobreakdash--}% for pages 1\Ndash 9
%\newcommand{\somespace}{\hfill{}\\ \vspace{-0.7cm}}
%\def\partitle#1{{\medskip\noindent {\bf #1.\hbox to 12pt{}}}}
\def\partitle{\subsubsection}

%%%theosdefinitionen
\newcommand{\gm}{\mathfrak m}
\newcommand{\gM}{\mathfrak M}
\newcommand{\integer}{\ZZ}
\newcommand{\proj}{\PP}
\newcommand{\complex}{\CC}
\newcommand{\real}{\mathbb R}
\newcommand{\gp}{\mathfrak p}
\newcommand{\gq}{\mathfrak q}
% \newcommand{\go}{\mathfrak so}
%\newcommand{\openF}{\F}

%%%%%%%%%%%%%%%betti table more wide!
\tabcolsep 3pt



%%%%%%%%%%%%%%%BIBLIOGRAPHY
\newcommand{\by}{}
\newcommand{\paper}{: \begin{it}}
\newcommand{\jour }{, \end{it}}
\newcommand{\vol}{\begin{bf} }
\newcommand{\yr}{\end{bf}(}
\newcommand{\pages}{),}

\maketitle

\begin{abstract} In this article we illustrate how picking points
over a finite field at random 
can help to investigate algebraic geometry questions.
In the first part we develop a program that produces random
curves of genus $g \le 14$. In the second part we use the program to test
Green's Conjecture on syzygies of canonical curves and compare 
it with the corresponding statement for Coble self-dual sets of
points.
In the third section we apply our techniques to produce Calabi-Yau
3-folds of degree $17$ in $\PP^6$.     
\end{abstract}

\section*{Introduction}

The advances in speed of modern computers and computer algebra systems gave 
life to the idea of solving equations\index{solving polynomial equations} by guessing a solution.
Suppose $\MM \subset \GG$ is a subvariety of a rational variety
of codimension $c$. 
Then we expect that the probability for a point $p \in \GG(\FF_q)$ to lie in 
$\MM(\FF_q)$ is about $1/q^c$. 
Here $\FF_q$ denotes the field with $q$ elements.  

We will discuss this idea in the following setting: 
$\MM$ will be a parameter space for objects in algebraic geometry, e.g.,
a \ie{Hilbert scheme}, a \ie{moduli space}, or a space dominating such spaces.

The most basic question we might have in this case is whether $\MM$ is 
non-empty and whether an open part of $\MM$ corresponds to smooth objects.

Typically in these cases
we will not have explicit equations for $\MM \subset \GG$
but only an implicit algebraic description of $\MM$, and our approach will
be successful if the time required to check $p \notin \MM(\FF_q)$ is sufficiently
small compared to $q^c$.
The first author applied this method first in \cite{CO:Sch1} to construct some
rational surfaces in $\PP^4$; see \cite{CO:ElPe,CO:DS} for motivation.


In this first section we describe a program that picks curve of genus
$g \le 14$ at random. The moduli spaces $\gM_g$ are known to be unirational
for $g\le 13$; see \cite{CO:Se,CO:CR}.

Our approach based on this result
can viewed as a computer aided proof of the \ie{unirationality}. 
Many people might object 
that this not a proof because we cannot control every single step in the
computation. We however think that such a proof is much more reliable than
a proof based on man-made computations.
A mistake in a computer aided approach most often leads to an output far 
away from our expectation, hence it is easy to spot.  
A substantial improvement of present computers and computer algebra systems 
would give us an explicit unirational parametrization of $\gM_g$ for $g\le 13$.
\medskip

In the second part we apply our ``random curves'' to probe the consequences 
of Green's conjecture on syzygies of canonical curves, 
and compare these results with the corresponding statements for 
``Coble self-dual'' sets of $2g-2$ points in $\PP^{g-2}$. 
\medskip

In the last section we exploit our method to prove the existence 
of three components of the Hilbert scheme of Calabi-Yau 3-folds of degree $17$ 
in $\PP^6$ over the complex numbers. 
This is one of the  main results of the second author's thesis
\cite[Chapter 4]{CO:To}. 
Calabi-Yau threefolds of lower degree in $\PP^6$ are easy to construct, 
using the Pfaffian construction and a study of their Hartshorne-Rao modules.
For degree $17$ the Hartshorne-Rao module has to satisfy a subtle condition. 
Explicit examples of such Calabi-Yau 3-folds are first 
constructed over a finite field by our probabilistic method.
Then a delicate semi-continuity argument gives us the existence of such Calabi-Yau 3-folds
over some number field.


\begin{acknowledgment} 
We thank Hans-Christian v. Bothmer and Dan Gray\-son for valuable discussions and
remarks. 
\end{acknowledgment} 


\partitle{Notation} 
For a finitely generated graded module $M$ over the polynomial ring 
$S=k[x_0,\ldots,x_r]$ we summarize the numerical information of a finite
free resolution
$$ 0 \lto M \lto F_0 \lto F_1 \lto \ldots \lto F_n\lto 0 $$
with $F_i = \oplus_j S(-j)^{\beta_{ij}}$ in a \ie{table of Betti numbers},
whose $ij^{th}$ entry is
$$ \beta_{i,i+j} = \dim \Tor_i^S(M,k)_{i+j}.$$
As in the \Mtwo command {\tt betti} we suppress zeroes.
For example the syzygies of the rational normal curve in $\PP^3$ 
have the following Betti table.
$$\begin{tabular}{|ccc}
\hline
1 & - & - \cr
- & 3 & 2 \cr
\end{tabular}$$
Note that the degrees of the entries of the matrices in the free
resolution can be read off from the relative position of two numbers in 
consecutive columns. A pair of numbers in a line corresponds to linear entries.
Quadratic entries correspond to two numbers of a square. Thus 
$$\begin{tabular}{|cccc}
\hline
1 & - & - & - \cr
- & 5 & 5 & - \cr 
- & - & - & 1 \cr
\end{tabular}$$
corresponds to a 4 term complex with a quadratic, a linear and another
quadratic map. The Grassmannian $\GG(2,5)$ in its Pl\"ucker embedding
has such a free resolution.
 


 

\section{How to Make Random Curves up to Genus $14$}

\index{random curves}
The \ie{moduli space of curves} $\gM_g$ is known to be of general type for 
$g \ge 24$ and has non-negative Kodaira dimension for $g=23$ by work of 
Harris, Mumford and Eisenbud \cite{CO:HM,CO:EH}. 
For genus $g \le 13  $ unirationality is known \cite{CO:CR,CO:Se}.
In this section we present a \Mtwo program that over a finite field $\FF_q$ 
picks a point in $\gM_g(\FF_q)$  for $g \le 14$ at random. 

By Brill-Noether theory \cite{CO:ACGH} every curve of genus $g$ has a linear 
system $g^r_d$ of dimension $r$ and degree $d$, provided that the 
\ie{Brill-Noether number}
$\rho$ satisfies
$$\rho := \rho(g,d,r) := g-(r+1)(g-d+r)\geq 0.$$
We utilize this to find appropriate (birational) models for general curves of genus $g$.



\subsection{Plane Models, $g\le 10$}

This case was known to Severi; see \cite{CO:AC}. 
Choose $d = g+2 - \lfloor g/3 \rfloor$. 
Then $\rho(g,d,2) \ge 0$
i.e., a general curve of genus g has a plane model $C'$ of degree $d$. 
We expect that $C'$ has 
$$\delta= \binom{d-1}{2} - g$$
double points. 
If the double points are in general position, then 
$$s=h^0(\PP^2,\ko(d))-3\delta-1$$
is the expected dimension of the linear system of curves of degree $d$ 
with $\delta$ assigned double points.  
We have the following table:
$$
{\tabcolsep 5pt
\begin{tabular}{l|rrrrr rrrrr rr}
g & 1 & 2 & 3 & 4 & 5 & 6 & 7 & 8 & 9 & 10 & 11 & 12 \cr
\hline
$\rho$      & 1 & 2 & 0 & 1 & 2 & 0 & 1 & 2 & 0 & 1  &  2 & 0 \cr
\hline
$d$        & 3 & 4 & 4 & 5 & 6 & 6 & 7 & 8 & 8 & 9 & 10 & 10 \cr
\hline
$\delta$ & 0 & 1 & 0 & 2 & 5 & 4 & 8 & 13 & 12 & 18 & 25 & 24 \cr
\hline
$s$        & 9 & 11 & 14 & 14 & 12 & 15 & 11 & 5 & 8 & 0 & -10 & -7 \cr
\end{tabular}}
$$
Thus for $g\le 10$ we assume that these double points lie in general position. 
For $g>10$ the double points cannot lie in general position because $s<0$. 
Since it is difficult to describe the special locus 
$H_\delta(g) \subset \Hilb_\delta(\PP^2)$ 
of double points of nodal genus $g$ curves, the plane model approach collapses for $g>10$.

\partitle{Random Points}
\index{random points}
In our program, which picks plane models at random from an Zariski open subspace of $\gM_g$,
we start by picking the nodes. 
However, over a small field $\FF_q$ it is not a good idea to pick points individually, 
because there might be simply too few: $|\PP^2(\FF_q)|=1+q+q^2$.
What we should do is to pick a collection $\Gamma$ of $\delta$ points in $\PP^2(\bar \FF_q)$
that is defined over $\FF_q$.
General points in $\PP^2$ satisfy the minimal resolution condition, 
that is, they have expected Betti numbers.
This follows from the Hilbert-Burch theorem \cite[Theorem 20.15]{CO:Ei}.
If the ideal of such $\Gamma$ has generators in minimal degree $k$,
then 
$\binom{k+1}2 \le \delta < \binom{k+2}2$,
which gives $\delta= \binom{k+1}{2} + \epsilon$  with  $0 \le \epsilon \le k$.
Thus $k=\lceil ({-3+\sqrt{9+8\delta}})/{2} \rceil$.
The Betti table is one of the following two tables:


\medskip 
$2\epsilon \le k :$
\begin{tabular}{c|ccc}\cline{2-4}
0 & 1 & - & - \cr
1 & - & - & - \cr
\vdots & \vdots & \vdots & \vdots \cr
$k-2$ & - & - & - \cr
$k-1$ & - & $k+1-\epsilon$ & $k-2\epsilon$ \cr
$k$ & - & - & \hbox to 20pt{\hfil$\epsilon$\hfil} \cr
\end{tabular}
$\quad2\epsilon \ge k :$
\begin{tabular}{c|ccc}\cline{2-4}
0 & 1 &- & - \cr
1 & - & - & - \cr
\vdots & \vdots & \vdots & \vdots \cr
$k-2$ & - & - & - \cr
$k-1$ & - & $k+1-\epsilon$ & - \cr
$k$ & - & $2\epsilon-k$ & \hbox to 20pt{\hfil$\epsilon$\hfil} \cr
\end{tabular}
\medskip

\noindent
So we can specify a collection $\Gamma$ of $\delta$ points by picking 
the Hilbert-Burch matrix of their resolution; see \cite[Thm 20.15]{CO:Ei}. 
This is a matrix with linear and quadratic entries only, 
whose minors of size $\epsilon$ ($k-\epsilon$ if $2\epsilon\leq k$) 
generate the homogeneous ideal of $\Gamma$.
\beginOutput
i1 : randomPlanePoints = (delta,R) -> (\\
\          k:=ceiling((-3+sqrt(9.0+8*delta))/2);\\
\          eps:=delta-binomial(k+1,2);\\
\          if k-2*eps>=0 \\
\          then minors(k-eps,\\
\               random(R^(k+1-eps),R^\{k-2*eps:-1,eps:-2\}))\\
\          else minors(eps,\\
\               random(R^\{k+1-eps:0,2*eps-k:-1\},R^\{eps:-2\})));\\
\endOutput
In unlucky cases these points might be infinitesimally near.
\beginOutput
i2 : distinctPoints = (J) -> (\\
\          singJ:=minors(2,jacobian J)+J;\\
\          codim singJ == 3);\\
\endOutput

\medskip \noindent
The procedure that returns the ideal of a \ie{random nodal curve} is then straightforward: 
\beginOutput
i3 : randomNodalCurve = method();\\
\endOutput
\beginOutput
i4 : randomNodalCurve (ZZ,ZZ,Ring) := (d,g,R) -> (\\
\          delta:=binomial(d-1,2)-g;\\
\          K:=coefficientRing R;\\
\          if (delta==0) \\
\          then (     --no double points\\
\               ideal random(R^1,R^\{-d\}))\\
\          else (      --delta double points            \\
\               Ip:=randomPlanePoints(delta,R);\\
\               --choose the curve\\
\               Ip2:=saturate Ip^2;\\
\               ideal (gens Ip2 * random(source gens Ip2, R^\{-d\}))));\\
\endOutput
\beginOutput
i5 : isNodalCurve = (I) -> (\\
\          singI:=ideal jacobian I +I;delta:=degree singI;\\
\          d:=degree I;g:=binomial(d-1,2)-delta;\\
\          \{distinctPoints(singI),delta,g\});\\
\endOutput

We next ask if we indeed get in this way points in a parameter space 
that dominates $\gM_g$ for $g \le 10$.
Let $\Hilb_{(d,g)}(\PP^2)$ denote the Hilbert scheme of nodal plane curves
of degree $d$ and geometric genus $g$.
Our construction starts from  a random element in $\Hilb_\delta(\PP^2)$ and picks 
a random curve in the corresponding fiber of 
$\Hilb_{(d,g)}(\PP^2)\to\Hilb_\delta(\PP^2)$:
$$\xymatrix{\Hilb_{(d,g)}(\PP^2) \ar[r] \ar[d] &\gM_g\\ \Hilb_\delta(\PP^2)}.$$
So the question is whether $\Hilb_{(d,g)}(\PP^2)$ dominates $\Hilb_\delta(\PP^2)$.
A naive dimension count suggests that this should be true:
the dimension of our parameter space is given by $2\delta +s$, 
which is $3(g-1)+\rho+\dim\PGL(3)$, as it should be.
To conclude this there is more to verify: 
it could be that the nodal models of general curves have double points 
in special position, 
while all curve constructed above lie over a subvariety of $\gM_g$. 
One way to exclude this is to prove that the variety $G(g,d,2)$ over $\gM_g$,
whose fiber over a curve ${\tilde C}\in\gM_g$ is $G^2_d(\tilde C)=\{g^2_d\text{'s}\}$,
is irreducible or, to put it differently, that the Severi Conjecture holds:

\begin{theorem}[Harris \cite{CO:Ha1}] The space
of nodal degree d genus g curves in $\PP^2$ is irreducible.
\end{theorem}

Another much easier proof for the few $(d,g)$ we are interested in is to
establish that our parameter space $\MM$ of the construction is smooth of 
expected dimension at our random point $p \in \MM$, as in \cite{CO:AC}.
Consider the following diagram:
$$\HH = \Hilb_{(d,g)}(\PP^2)/\Aut(\PP^2) \stackrel{\pi}{\lra} \gM_g.$$
For a given curve ${\tilde C}\in\gM_g$, the inverse image $\pi^{-1} ({\tilde C})$ 
consists of the variety $W^2_d({\tilde C})\subset\Pic^d({\tilde C})$.
Moreover the choice of a divisor $L\in W^2_d({\tilde C})$ is equivalent to the 
choice of $p \in \MM$, modulo $\Aut(\PP^2)$: 
indeed $p$ determines a morphism $\nu\colon {\tilde C} \lra C\subset \PP^2$ 
and a line bundle $L=\nu^{-1}\ko_{\PP^2}(1)$, where ${\tilde C}$ is the 
normalization of the (nodal) curve $C$.
Therefore $\MM$ is smooth of expected dimension $3(g-1)+\rho+\dim \PGL(3)$
at $p \in \MM$ 
if and only if $W^2_d({\tilde C})$ is smooth of expected dimension $\rho$ in $L$.
This is well known to be equivalent to the injectivity of the multiplication map
$\mu_L$
$$
H^0(L)\otimes H^0(K_{\tilde C}\otimes L^{-1}) \stackrel{\mu_L}{\lra}H^0(K_{\tilde C}),
$$
which can be easily checked in our cases,
see \cite[p. 189]{CO:ACGH}.
In our cases $\mu_L$ can be rewritten as 
$$
H^0(\ko_{\PP^2}(1))\otimes H^0(I_\Gamma(d-4)) \stackrel{\mu_L}{\lra}
H^0(I_\Gamma(d-3)).
$$
So we need two conditions: 
\begin{enumerate}
\item $H^0(I_\Gamma(d-5))=0$;
\item there are no linear relations among the generators of $H^0(I_\Gamma(d-4))$ of 
degree $d-3$.
\end{enumerate}
We proceed case by case.
For genus $g\leq 5$ this is clear, since $H^0(I_\Gamma(d-4))=0$ for $g=2,3$ and
$\dim H^0(I_\Gamma(d-3))=1$ for $g=4,5$. 
For $g=6$ we have $\dim H^0(I_\Gamma(d-3))= \dim H^0(I_\Gamma(2))=2$ and 
the Betti numbers of $\Gamma$ 
$$
\begin{tabular}{|ccc}\hline
1 & - & - \cr
- & 2 & - \cr
- & - & 1 \cr
\end{tabular}
$$
shows there are no relations 
with linear coefficients in $H^0(I_\Gamma(2))$.
For $7\leq g \leq 10$ the method is similar:
everything is clear once the Betti table of resolution of the set of nodal points 
$\Gamma$ is computed. As a further example we do here the case $g=10$: we see that
$\dim H^0(I_\Gamma(d-3))=\dim H^0(I_\Gamma(5))=3$ and the Betti numbers of $\Gamma$ are
$$
\begin{tabular}{|ccc}\hline
1 & - & - \cr
- & - & - \cr
- & - & - \cr
- & - & - \cr
- & 3 & - \cr
- & 1 & 3 \cr
\end{tabular}
$$
from which it is clear that there are no linear relations between the quintic generators
of $I_\Gamma$.

%% and that the 
%% Kodaira-Spencer map
%% $$T_p(\MM) \to T_C(\gM_g) = H^0(C,\omega^{\tensor 2})^*$$
%% is surjective. This is a first order deformation problem, which can be easily 
%% solved by linear algebra in Macaulay2:
%% 
%% {\scriptsize
%% \begin{verbatim}
%% Kodaira-SpencerMap = (I) -> ???
%% \end{verbatim}}


\subsection{Space Models and Hartshorne-Rao Modules}
\partitle{The Case of Genus $g=11$}
In this case we have $\rho(11,12,3)=3$. 
Hence every general curve of genus 11 has a space model of degree 12. 
Moreover for a general curve the general space model of this degree 
is linearly normal, because $\rho(11,13,4)=-1$ takes a smaller value. 
If moreover such a curve $C \subset \PP^3$ has \ie{maximal rank},
i.e., for each $m \in \ZZ$ the map 
$$H^0(\PP^3,\ko(m)) \to H^0(C,\ko_C(m))$$
has maximal rank,
then the \ie{Hartshorne-Rao module} $M$, 
defined as
$M=H^1_*(\ki_C)=\oplus_m H^1(\PP^3,\ki_C(m))$, 
has Hilbert function with values $(0,0,4,6,3,0,$ $0,\dots)$. For readers who want to know more about the
Hartshorne-Rao module, we refer to the  pleasant treatment in \cite{CO:MDP}.

Since being of maximal rank is an open condition, we will try a construction
of maximal rank curves.
Consider the  vector bundle $\kg$ on $\PP^3$ associated to the first syzygy 
module of $I_C$: 
$$ 0 \lto \ki_C \lto \oplus_i \ko(-a_i) \lto \kg \lto 0 \leqno(1)$$
In this set-up $H^2_*(\kg)=H^1_*(\ki_C)$. 
Thus $\kg$ is, up to direct sum of line bundles, the sheafified second syzygy 
module of $M$; see e.g., \cite[Prop. 1.5]{CO:DES}.

The expected Betti numbers of $M$ are
$$
\begin{tabular}{|ccccc}\hline
4 & 10 & 3 & - & - \cr
- & - & 8 & 2 & - \cr
- & - & - & 6 & 3 \cr
\end{tabular}
$$
Thus the $\FF$-dual $M^* = \Hom_\FF(M,\FF)$ is presented as 
$\FF[x_0,\dots,x_3]$-module by a 
$3 \times 8$ matrix with linear and quadratic entries, and a general such matrix 
will give a general module
(if the construction works, i.e., if the desired space of modules is non-empty), 
because all conditions we impose are semi-continuous and open.
Thus $M$ depends on 
$$\dim \GG(6,3h^0\ko(1))+\dim \GG(2,3h^0\ko(2)-6h^0\ko(1))-\dim SL(3)=36$$
parameters. 

Assuming that $C$ has minimal possible syzygies:
$$
\begin{tabular}{|cccc}\hline
1 & - & - & - \cr
-&-&-&-\cr
-&-&-&-\cr
-&-&-&-\cr
-&6&2&-\cr
-&-&6&3\cr
\end{tabular}
$$
we obtain, by dualizing the sequence (1), the following exact sequence
$$ \kg^* {\lto  } 6\ko(5) \lto \ko \lto 0.$$
If everything is as expected, 
i.e., the general curve is of maximal rank and its syzygies have minimal possible
Betti numbers, then the entries of the right hand matrix are
homogeneous polynomials that generate $I_C$. 
We will compute $I_C$ by determining $ \ker (\phi \colon 6\ko(5) \to \kg^*)$.
Comparing with the syzygies of $M$ we obtain the following isomorphism
$$\kg^* \cong \ker(2\ko(6)\oplus6\ko(7) \to 3\ko(8)) \cong 
\image(3\ko(4)\oplus8\ko(5) \to 2\ko(6)\oplus6\ko(7)).$$
and $\kg^* \lto 6\ko(5)$ factors over $\kg^* \lto 8\ko(5)\oplus3\ko(4)$.
A general $\phi \in \Hom(6\ko(5),$ $\kg^*)$ gives a point in $\GG(6,8)$
and the Hilbert scheme of  desired curves would have dimension 
$36+12=48=4\cdot12=30+3+15$ as expected, c.f.~\cite{CO:Ha2}.

\medskip
Therefore the computation for obtaining a random space curve of genus $11$ 
is done as follows:
\beginOutput
i6 : randomGenus11Curve = (R) -> (\\
\          correctCodimAndDegree:=false;\\
\          while not correctCodimAndDegree do (\\
\               Mt=coker random(R^\{3:8\},R^\{6:7,2:6\});\\
\               M=coker (transpose (res Mt).dd_4);\\
\               Gt:=transpose (res M).dd_3;\\
\               I:=ideal syz (Gt*random(source Gt,R^\{6:5\}));\\
\               correctCodimAndDegree=(codim I==2 and degree I==12););\\
\          I);\\
\endOutput

\medskip
In general for these problems there is rarely an a priori reason
why such a construction for general choices will give a smooth curve. 
Kleiman's global generation condition \cite{CO:Klei} is much too
strong a hypothesis for many interesting examples. 
But it is easy to check an example over a finite field with a computer:
\beginOutput
i7 : isSmoothSpaceCurve = (I) -> (\\
\          --I generates the ideal sheaf of a pure codim 2 scheme in P3\\
\          singI:=I+minors(2,jacobian I);\\
\          codim singI==4);\\
\endOutput

Hence by semi-continuity this is true over $\QQ$ and the
desired unirationality of $G(11,12,3)/\gM_{11}$ holds for all fields,
except possibly for those whose ground field has characteristic is in some finite set.

A calculation of an example over the integers
would bound the number of exceptional characteristics, 
which then can be ruled out case by case, 
or by considering sufficiently many integer examples.

As in case of nodal curves, to prove \ie{unirationality} of $\gM_{11}$ by computer
aided computations we have to show the injectivity of
$$
H^0(L)\otimes H^0(K_{C}\otimes L^{-1}) \stackrel{\mu_L}{\lra}H^0(K_{C}),
$$
where $L$ is the restriction of $\ko_{\PP^3}(1)$ to the curve $C\subset \PP^3$.
The following few lines do the job:
\beginOutput
i8 : K=ZZ/101;\\
\endOutput
\beginOutput
i9 : R=K[x_0..x_3];\\
\endOutput
\beginOutput
i10 : C=randomGenus11Curve(R);\\
\emptyLine
o10 : Ideal of R\\
\endOutput
\beginOutput
i11 : isSmoothSpaceCurve(C)\\
\emptyLine
o11 = true\\
\endOutput
\beginOutput
i12 : Omega=prune Ext^2(coker gens C,R^\{-4\});\\
\endOutput
\beginOutput
i13 : betti Omega\\
\emptyLine
o13 = relations : total: 5 10\\
\                     -1: 2  .\\
\                      0: 3 10\\
\endOutput

\noindent
We see that there are no linear relations among the two generators
of $H^0_*(\Omega_C)$ in degree -1.

%% to prove uniratioanlity of $\gM_{11}$ by Computer
%% aided computations we have to show that the Kodaira-Spencer map
%%         $$ H^0(C,\kn_C) \to H^1(C,\kt_C) $$
%% is onto in our example. 
%% The following procedure returns the dimension 
%% of the image of the Kodaira-Spencer map for a smooth projective variety
%% (of codimension at least 2,
%% otherwise $\kn_C^*=${\tt I2} is free and {\tt T} is just {\tt coker djt}):
%% 
%% {\scriptsize
%% \begin{verbatim} 
%% kodairaSpencerMap = (I) -> (
%%      S:=R/I;
%%      dIt:=substitute(transpose jacobian I,S);
%%      I2:=substitute(syz gens I,S); --the conormal bundle
%%      T:=prune(kernel transpose I2/image dIt);
%%      hilbertFunction(0,T))
%% \end{verbatim}}
%% THIS SCRIPT FORGETS THE CONTRIBUTION OF H1 IN THE SEQUENCE GIVING T_P|C



\partitle{Betti Numbers for Genus $g=12,13,14,15$}
The approach in these cases is similar to $g=11$. We choose here $d=g$, 
so $\rho(g,g,3)=g-12\ge 0$. 
Under the maximal rank assumption the corresponding space curve has
a Hartshorne-Rao module whose Hilbert function takes values 
$(0,0,g-9,2g-19,3g-34,0,\ldots)$ in case $g=12,13$ 
and $(0,0,g-9,2g-19,3g-34,4g-55,0,\ldots)$ in case $g=14,15$.
Expected syzygies of $M$ have Betti tables:

$$g=12: \;
\begin{tabular}{|ccccc}\hline
3 & 7 & - & - & -  \cr
- & - & 10 & 5 & - \cr
- & - & - & 3 & 2  \cr
\end{tabular}
\qquad
g=13: \;
\begin{tabular}{|ccccc}\hline
4 & 9 & 1 & - & -  \cr
- & - & 6 & - & - \cr
- & - & 6 & 13 & 5  \cr
\end{tabular}
$$
\medskip
$$g=14: \;
\begin{tabular}{|ccccc}\hline
5 & 11 & 2 & - & -  \cr
- & - & 3 & - & - \cr
- & - & 13 & 17 & 4  \cr
- & - & - & - & 1 \cr
\end{tabular}
\qquad
g=15: \;
\begin{tabular}{|ccccc}
\hline
6 & 13 & 3 & - & -  \cr
- & - & 3 & - & - \cr
- & - & 8 & 3 & -  \cr
- & - & - & 9 & 5 \cr
\end{tabular}
$$

\medskip
Comparing with the expected syzygies of $C$ 
$$g=12: \;
\begin{tabular}{|cccc}\hline
 1 & - & - & -  \cr
 - & - & - & -  \cr
 - & - & - & -  \cr
 - & - & - & -  \cr
 - & 7 & 5 & - \cr
 - & - & 3 & 2  \cr
\end{tabular}
\qquad
g=13: \;
\begin{tabular}{|cccc}\hline
 1 & - & - & -  \cr
 - & - & - & -  \cr
 - & - & - & -  \cr
 - & - & - & -  \cr
 - & 3 & - & - \cr
 - & 6 & 13 & 5  \cr
\end{tabular}
$$
\medskip
$$g=14: \;
\begin{tabular}{|cccc}\hline
 1 & - & - & -  \cr
 - & - & - & -  \cr
 - & - & - & -  \cr 
- & - & - & -  \cr 
- & - & - & -  \cr
 - & 13 & 17 & 4  \cr
 - & - & - & 1 \cr
\end{tabular}
\qquad
g=15: \;
\begin{tabular}{|cccc}
\hline
 1 & - & - & -  \cr 
 - & - & - & -  \cr
- & - & - & -  \cr
 - & - & - & -  \cr
 - & - & - & -  \cr
 - & 8 & 3 & -  \cr
 - & - & 9 & 5 \cr
\end{tabular}
$$
we see that given  $M$ the choice of a curve corresponds to a point
in $\GG(7,10)$ or $\GG(3,6)$ for $g=12,13$ respectively, 
while for $g=14,15$ everything is determined by the Hartshorne-Rao module. 
For $g=12$ Kleiman's result guarantees smoothness for general choices,
in contrast to the more difficult cases $g=14,15$. 
So the construction of $M$ is the crucial step. 




\partitle{Construction of Hartshorne-Rao Modules}
In case $g=12$ the construction of $M$ is straightforward. 
It is presented by a sufficiently general matrix of linear forms:
$$0 \lto M \lto 3S(-2) \lto 7S(-3).$$

\noindent
The procedure for obtaining a random genus 12 curve is:
\beginOutput
i14 : randomGenus12Curve = (R) -> (\\
\           correctCodimAndDegree:=false;\\
\           while not correctCodimAndDegree do (\\
\                M:=coker random(R^\{3:-2\},R^\{7:-3\});\\
\                Gt:=transpose (res M).dd_3;\\
\                I:=ideal syz (Gt*random(source Gt,R^\{7:5\}));\\
\                correctCodimAndDegree=(codim I==2 and degree I==12););\\
\           I);\\
\endOutput


\medskip
In case $g=13$ we have to make sure that M has a second linear syzygy. 
Consider the end of the Koszul complex:
$$6R(-2) \stackrel{\kappa}{\lto} 4R(-3) \lto R(-1) \lto 0.$$
Any product of a general map $4R(-2) \stackrel{\alpha}{\lto} 6R(-2)$ with 
the Koszul matrix $\kappa$ yields
$4R(-2) \lto 4R(-3)$ with a linear syzygy, 
and concatenated with a general map $4R(-2) \stackrel{\beta}{\lto} 5R(-3)$
gives a presentation matrix of a module M of desired type:
$$ 0 \lto M \lto 4R(-2) \lto 4R(-3) \oplus 5R(-3).$$

\noindent
The procedure for obtaining a random genus 13 curve is:
\beginOutput
i15 : randomGenus13Curve = (R) -> (\\
\           kappa:=koszul(3,vars R);\\
\           correctCodimAndDegree:=false;\\
\           while not correctCodimAndDegree do (\\
\                test:=false;while test==false do ( \\
\                     alpha:=random(R^\{4:-2\},R^\{6:-2\});\\
\                     beta:=random(R^\{4:-2\},R^\{5:-3\});\\
\                     M:=coker(alpha*kappa|beta);\\
\                     test=(codim M==4 and degree M==16););\\
\                Gt:=transpose (res M).dd_3;\\
\                --up to change of basis we can reduce phi to this form\\
\                phi:=random(R^6,R^3)++id_(R^6);\\
\                I:=ideal syz(Gt_\{1..12\}*phi);\\
\                correctCodimAndDegree=(codim I==2 and degree I==13););\\
\           I);\\
\endOutput


\medskip
The case of genus $g=14$ is about a magnitude more difficult. 
To start with, we can achieve two second linear syzygies by the same method as 
in the case $g=13$. 
A general matrix 
$5R(-2) \stackrel{\alpha}{\lto} 12R(-2)$ composed with 
$12R(-2) \stackrel{\kappa \oplus \kappa}{\lto} 8R(-3)$ 
yields the first component of
$$5R(-2) \lto (8+3)R(-3).$$
For a general choice of the second component 
$5R(-2) \stackrel{\beta}{ \lto} 3R(-3)$ the cokernel
will be a module with Hilbert function $(0,0,5,9,8,0,0,\ldots)$ and syzygies
$$
\begin{tabular}{|ccccc}\hline
5 & 11 & 2 & - & -  \cr
- & - & 2 & - & - \cr
- & - & 17 & 23 & 8  \cr
- & - & - & - & - \cr
\end{tabular}
$$
What we want is to find $\alpha$ and $\beta$ so that $\dim M_5 = 1$ 
and $\dim \Tor_2^R(M,\FF)_5 = 3$.
Taking into account that we ensured $\dim \Tor_2^R(M,\FF)_4 =2$
this amounts to asking that the $100 \times 102$ matrix $m(\alpha,\beta)$ 
obtained from
$$[0 \lto 5R(-2)_5 \lto 11R(-3)_5 \lto 2R(-4)_5 \lto 0] 
\cong [0 \lto 100 \FF \stackrel{m(\alpha,\beta)} \lto 102 \FF \lto 0]$$
drops rank by 1. 
We do not know a systematic approach to produce such 
$m(\alpha,\beta)$'s.
However, we can find such matrices in a probabilistic way.
In the space of the matrices $m(\alpha,\beta)$, 
those which drop rank by 1 have expected codimension 3.
Hence over a finite field $\FF=\FF_q$ we expect to find the desired modules $M$
with a probability of $1/q^3$. 
The code to detect bad modules is rather fast.
\beginOutput
i16 : testModulesForGenus14Curves = (N,p) ->(\\
\           x := local x;\\
\           R := ZZ/p[x_0..x_3];\\
\           i:=0;j:=0;\\
\           kappa:=koszul(3,vars R);\\
\           kappakappa:=kappa++kappa;\\
\           utime:=timing while (i<N) do (\\
\                test:=false;\\
\                alpha:=random(R^\{5:-2\},R^\{12:-2\});\\
\                beta:=random(R^\{5:-2\},R^\{3:-3\});\\
\                M:=coker (alpha*kappakappa|beta);\\
\                fM:=res (M,DegreeLimit =>3);\\
\                if (tally degrees fM_2)_\{5\}==3 then (\\
\                     --further checks to pick up the right module\\
\                     test=(tally degrees fM_2)_\{4\}==2 and\\
\                     codim M==4 and degree M==23;);\\
\                i=i+1;if test==true then (j=j+1;););\\
\           timeForNModules:=utime#0; numberOfGoodModules:=j;\\
\           \{timeForNModules,numberOfGoodModules\});\\
\endOutput
\beginOutput
i17 : testModulesForGenus14Curves(1000,5)\\
\emptyLine
o17 = \{41.02, 10\}\\
\emptyLine
o17 : List\\
\endOutput

\noindent
For timing tests we used a Pentium2 400Mhz with 128Mb of memory running GNU Linux.
On such a machine examples can be tested at a rate 
of $0.04$ seconds per example.
Hence an approximate estimation of the CPU-time required to find a good example 
is $q^3 \cdot 0.04$ seconds.
Comparing this with the time to verify smoothness, 
which is about $12$ seconds for an example of this degree,
we see that up to $|\FF_q|=q \le 13$ we can expect to obtain examples within 
few minutes.
Actually the computations for $q=2$ and $q=3$ take longer than for $q=5$ on average,
because examples of ``good modules'' tend to give singular curves more often. 
Here is a table of statistics which summarizes the situation.
$$
\def\sb#1{\hbox to 22pt{\hfil#1}} %small fixed size box --text mode
\begin{tabular}{l|rrrrrrr}
$q$ &                           \sb2   &\sb3   &\sb5   &\sb7   &\sb{11}  &\sb{13}\cr
\hline
smooth curves &                 100 &100 &100 &100 &100 &100\cr
1-nodal curves &                75  &53  &31  &16  &10  &8\cr
reduced more singular &         1012&142 &24  &11  &2   &0\cr
non reduced curves &            295 &7   &0   &0   &0   &0\cr
\hline
total number of curves &        1482&302 &155 &127 &112 &108\cr
percentage of smooth curves \quad &  %the quad will be divided, ok
                                6.7\% &33\%  &65\%  &79\%  &89\% &93\%\cr
approx. time (in seconds) &     7400&3100&2700&3400&6500&9500\cr
\end{tabular}
$$

\medskip
\goodbreak
\noindent
The procedure for obtaining a random genus 14 curve is
\beginOutput
i18 : randomGenus14Curve = (R) -> (\\
\           kappa:=koszul(3,vars R);\\
\           kappakappa:=kappa++kappa;\\
\           correctCodimAndDegree:=false;\\
\           count:=0;while not correctCodimAndDegree do (\\
\                test:=false;\\
\                t:=timing while test==false do (\\
\                     alpha=random(R^\{5:-2\},R^\{12:-2\});\\
\                     beta=random(R^\{5:-2\},R^\{3:-3\});\\
\                     M:=coker (alpha*kappakappa|beta);\\
\                     fM:=res (M,DegreeLimit =>3);\\
\                     if (tally degrees fM_2)_\{5\}==3 then (\\
\                          --further checks to pick up the right module\\
\                          test=(tally degrees fM_2)_\{4\}==2 and\\
\                          codim M==4 and degree M==23;);\\
\                     count=count+1;);\\
\                Gt:=transpose (res M).dd_3;\\
\                I:=ideal syz (Gt_\{5..17\});\\
\                correctCodimAndDegree=(codim I==2 and degree I==14););\\
\           <<"     -- "<<t#0<<" seconds used for ";\\
\           <<count<<" modules"<<endl;\\
\           I);\\
\endOutput


\medskip
For $g=15$ we do not know a method along these lines that 
would give examples over small fields.



\partitle{Counting Parameters}
For genus $g=12$ clearly the module $M$ depends on 
$\dim\GG(7,3 \cdot h^0\ko(1))-\dim SL(3)= 7\cdot5-8=36$ parameters, 
and the family of curves has dimension $36+\dim\GG(7,10)=48=4\cdot12=33+0+15$,
as expected. 

For genus $g=13$ and $14$ the parameter count is more difficult.
Let us make a careful parameter count for genus $g=14$;
the case $g=13$ is similar and easier.
The choice of $\alpha$ corresponds to a point in $\GG(5,12)$.
Then $\beta$ corresponds to a point $\GG(3,B_\alpha)$ where 
$B_\alpha = U \tensor R_1/{<}\alpha{>}$ where $U$ denotes
the universal subbundle on $\GG(5,12)$ and ${<}\alpha{>}$ 
the subspace generated by the 8 columns of $\alpha\circ(\kappa \oplus \kappa)$. 
So $\dim B_\alpha = 20-8=12$ and $\GG(3,B_\alpha) \to \GG(5,12)$ 
is a Grassmannian bundle with fiber dimension 27 and total dimension 62. 
In this space the scheme of good modules has codimension 3, 
so we get a 59 dimensional family. 
This is larger than the expected dimension $56=4\cdot 14=39+2+15$ 
of the Hilbert scheme, c.f.~\cite{CO:Ha2}. 
Indeed the construction gives a curve together
with a basis of $\Tor_2^R(M,\FF)_4$. 
Subtracting the dimension of the group of the projective coordinate changes 
we arrive at the desired dimension $59-3=56$. 

The unirationality of $\gM_{12}$ and $\gM_{13}$ can be proved by computer 
as in case $\gM_{11}$, 
while in case $g=14$ we don't know the unirationality of 
the parameter space of the modules $M$ with 
$\dim M_5=1$ and $\dim \Tor^R_2(M,\FF)_5=3$.

\section[Comparing Green's Conjecture for Curves and Points]
	{Comparing Green's Conjecture for Curves \\ and Points}

\subsection{Syzygies of Canonical Curves}
\index{Green's conjecture}
\index{Syzygies of canonical curves}
One of the most outstanding conjectures about free resolutions is 
Green's prediction
for the syzygies of canonical curves. 

A \ie{canonical curve}
$C \subset \PP^{g-1}$, i.e., a linearly normal curve with $\ko_C(1) \equiv 
\omega_C$, the canonical line bundle, is projectively normal by a result of
Max Noether, and hence has a Gorenstein homogeneous coordinate ring and is 
3-regular.

Therefore the Betti numbers of the free resolution of a canonical curve 
are symmetric, that is, $\beta_{j,j+1}=\beta_{g-2-j,g-j}$, 
and essentially only two rows of Betti numbers occur.
The situation is summarized in the following table.
$$
\raise 0pt\hbox to 0 pt{\hspace{121.8pt} $\bullet$\hss}
\def\sb#1{\hbox to 5mm{\hfil#1\hfil}} 
\begin{tabular}{|ccccc ccccc cc}
\hline
1 & - & - & - & - & \sb{-} & \sb{-} & - & - & - & - & - \cr
- & $\beta_{1,2}$ & $\beta_{2,3}$ & $\cdots$ & $\beta_{p,p+1}$ & $\cdots$ 
& $\cdots$ & $\beta_{p,p+2}$ & - & - & - & - \cr
- & - & - & - & $\beta_{p,p+2}$ & $\cdots$ 
& $\cdots$ & $\beta_{p,p+1}$ & $\cdots$ & $\beta_{2,3}$ & $\beta_{1,2}$ & - \cr
- & - & - & - & - & - & - & - & - & - & - & 1 \cr
\end{tabular}
$$
%%$$
%%\def\hd#1{\hbox to 14.5mm{\hfil$#1$\hfil}} %for hodge diagrams --math mode
%%\setlength{\unitlength}{1mm}
%%\begin{picture}(116,36)(-58,-18)  
%%\put (-58,7.5){\hd{1}}
%%\put (-43.5,1.5){\hd{\beta_{1,2}}}
%%\put (-29,1.5){\hd{\ldots}}
%%\put (-14.5,1.5){\hd{\ldots}}
%%\put (-14.5,-4.5){\hd{\ldots}} %-0.85=-1+.25-.8
%%\put (0,0){\circle*{2.5}}
%%\put (0,1.5){\hd{\ldots}}
%%\put (0,-4.5){\hd{\ldots}}
%%\put (14.5,-4.5){\hd{\ldots}}
%%\put (29,-4.5){\hd{\beta_{g-3,g-1}}}
%%\put (43.5,-10.5){\hd{1}}
%%%hor lines
%%\put (-58,12) {\line(1,0){116}}
%%\put (-58,6) {\line(1,0){116}}
%%\put (-58,0) {\line(1,0){116}}
%%\put (-58,-6) {\line(1,0){116}}
%%\put (-58,-12) {\line(1,0){116}}
%%%vert lines
%%\put (-58,-12) {\line(0,1){24}}
%%\put (-43.5,-12) {\line(0,1){24}}
%%\put (-29,-12) {\line(0,1){24}}
%%\put (-14.5,-12) {\line(0,1){24}}
%%\put (0,-12) {\line(0,1){24}}
%%\put (14.5,-12) {\line(0,1){24}}
%%\put (29,-12) {\line(0,1){24}}
%%\put (43.5,-12) {\line(0,1){24}}
%%\put (58,-12) {\line(0,1){24}}
%%%mark the resolution
%%\linethickness{1pt}
%%\put (-43.5,6) {\line(1,0){58}}
%%\put (-43.5,0) {\line(1,0){29}} \put (14.5,0) {\line(1,0){29}}
%%\put (-14.5,-6) {\line(1,0){58}}
%%\put (-43.5,0) {\line(0,1){6}}
%%\put (43.5,-6) {\line(0,1){6}}
%%\put (-14.5,-6) {\line(0,1){6}}
%%\put (14.5,0) {\line(0,1){6}}
%%% description
%%\put (-58,13) {$\overbrace{\hspace{116mm}}^{\displaystyle g-1}$}
%%\put (-58,-13) {$\underbrace{\hspace{43.5mm}}_{\displaystyle p}$}
%%\end{picture}   
%%$$

The first $p$ such that $\beta_{p,p+2}\neq 0$ is conjecturally precisely 
the \ie{Clifford index} of the curve.

\begin{conjecture}[Green \cite{CO:Gr2}]\label{GConj} 
Let $C$ be a smooth canonical curve over $\CC$. 
Then $\beta_{p,p+2} \ne 0$ if and only if $\exists L\in \Pic^d(C)$ with
$h^0(C,L),h^1(C,L) \ge 2$ and $\cliff(L):=d-2(h^0(C,L))-1) \le p$.
In particular, $\beta_{j,j+2}= 0$ for $j \le \lfloor \frac{g-3}{2} \rfloor$ 
for a general curve of genus $g$.
\end{conjecture}
 
The ``if'' part is proved by Green and Lazarsfeld in \cite{CO:GL} 
and holds for arbitrary ground fields. 
For some partial results see \cite{CO:Vo,CO:Sch2,CO:Sch3,CO:BaEi,CO:vB,CO:HR,CO:Mu}.
The conjecture is known to be false for some (algebraically closed) fields of
finite characteristic, e.g.,
genus $g=7$ and characteristic $\cha \FF=2$; see \cite{CO:Sch4}.


\subsection{Coble Self-Dual Sets of Points}
\index{Coble self-dual sets of points}
The free resolution of a hyperplane section of a Cohen-Macaulay ring 
has the same Betti numbers. Thus we may ask for a geometric
interpretation of the syzygies of $2g-2$ points in $\PP^{g-2}$ 
(hyperplane section of a canonical curve), 
or syzygies of a graded Artinian Gorenstein algebra with Hilbert function
$(1,g-2,g-2,1,0,\ldots)$ 
(twice a hyperplane section). 
 Any collection of $2g-2$ points obtained as a hyperplane section of a
 canonical curve is special in the sense that it imposes only $2g-3$
 conditions on quadrics.
An equivalent condition for points in linearly uniform position is that 
they are Coble (or Gale) self-dual; see \cite{CO:EiPo}. 
Thus if we distribute the $2g-2$ points into two collections each of $g-1$ points, 
with, say, the first consisting of the coordinate points
and the second corresponding to the rows of a $(g-1) \times (g-1)$ matrix 
$A=(a_{ij})$, then $A$ can be chosen to be an orthogonal matrix, i.e.,
$A^t A = 1$; see \cite{CO:EiPo}.

To see what the analogue of Green's Conjecture for the general curve means for
orthogonal matrices we recall a result of \cite{CO:RS1}.

Set $n=g-2$. We identify the homogeneous coordinate ring of $\PP^n$ with
the ring $S=\FF[\partial_0,\ldots,\partial_n]$ of 
\ie{differential operators} with 
constant coefficients, $\partial_i=\frac{\partial}{\partial x_i}$. 
The ring $S$ acts on $\FF[x_0,\ldots,x_n]$ by differentiation. 
The annihilator of $q=x_0^2+\ldots+x_n^2$ is a homogeneous ideal 
$J \subset S$ such that 
$S/J$ is a \ie{graded Artinian Gorenstein ring} with Hilbert function $(1,n+1,1)$ 
and socle induced by $q$, 
see \cite{CO:RS2}, \cite[Section 21.2 and related exercise 21.7]{CO:Ei}.
The syzygy numbers of S/J are
$$
\begin{tabular}{|ccccc cc}\hline
1 &- &- &- &- &- &-\cr
- &$\frac{n}{n+2}\binom{n+3}{2}$ &$\cdots$ &$\frac{p(n+1-p)}{n+2}\binom{n+3}{p+1}$
&$\cdots$ &$\frac{n}{n+2}\binom{n+3}{n+1}$ &-\cr
- &- &- &- &- &- &1\cr  
\end{tabular}
$$


A collection $H_0,\ldots,H_n$ of hyperplanes in $\PP^n$ is said to 
form a polar simplex to $q$ 
if and only if the collection $\Gamma=\{p_0,\ldots,p_n\} \subset \check \PP^n$ 
of the corresponding points in the dual space has its homogeneous ideal 
$I_\Gamma \subset S$ contained in $J$.

In particular the set $\Lambda$ consisting of the coordinate points correspond 
to a polar simplex, because 
$\partial_i\partial_j$ annihilates $q$ for $i\ne j$.

For any polar collection of points $\Gamma$  the free resolution
$\Ss_\Gamma$ is a subcomplex of the resolution $\Ss_{S/J}$. Green's conjecture
for the generic curve of genus $g=n+2$ would imply: 

\begin{conjecture}\label{PConj}  For a general
$\Gamma$ and the given  $\Lambda$ the corresponding Tor-groups
$$\Tor^S_k(S/I_\Gamma,\FF)_{k+1} \cap \Tor^S_k(S/I_\Lambda,\FF)_{k+1}
\subset \Tor^S_k(S/J,\FF)_{k+1}$$
intersect transversally.
\end{conjecture} 

\begin{proof}
A zero-dimensional non-degenerate scheme $\Gamma\subset\PP^n$ of degree n+1 
has syzygies
$$
\begin{tabular}{|ccccc c}\hline
1 &- &- &- &- &-\cr
- &$\binom{n+1}{2}$ &$\cdots$ &$k\binom{n+1}{k+1}$
&$\cdots$ &$n$
\end{tabular}
$$

Since both Tor groups are contained in $\Tor^S_k(S/J,\FF)_{k+1}$, 
the claim is equivalent to saying that for a general polar simplex $\Gamma$ 
the expected dimension of their intersection is 
$\dim\Tor^S_k(S/_\Gamma,\FF)_{k+1}+\dim\Tor^S_k(S/_\Gamma,\FF)_{k+1}
-\dim\Tor^S_k(S/J,\FF)_{k+1}$, which is
$$
2k\binom{g-1}{k+1}-\frac{k(g-1-k)}{g}\binom{g+1}{k+1}.
$$

On the other hand, $I_{\Gamma\cup\Lambda}=I_\Gamma\cap I_\Lambda$, hence
$$\Tor^S_k(S/I_\Gamma,\FF)_{k+1} \cap \Tor^S_k(S/I_\Lambda,\FF)_{k+1} 
=\Tor^S_k(S/I_{\Gamma\cup\Lambda},\FF)_{k+1},$$
and Green's conjecture would imply
$$
\dim\Tor^S_k(S/I_{\Gamma\cup\Lambda},\FF)_{k+1}
=\beta_{k,k+1}(\Gamma\cup\Lambda)
=k\binom{g-2}{k+1}-(g-1-k)\binom{g-2}{k-2}.
$$
Now a calculation shows that the two dimensions above are equal.
\end{proof}


The family of all polar simplices $V$ is dominated by the family defined  
by the ideal of $2 \times 2$ minors of the matrix
$$\begin{pmatrix} 
\partial_0 & \ldots & \partial_i & \ldots &\partial_n \cr
\sum_j b_{0j} \partial_j & \ldots &\sum_j b_{ij} \partial_j & \ldots & \sum_j b_{nj} \partial_j
\end{pmatrix}$$
depending on     
a symmetric matrix $B=(b_{ij})$, i.e., $b_{ij}=b_{ji}$ as parameters.
For $B$  a general diagonal matrix
we get $\Lambda$ together with a specific element in 
$\Tor^S_n(S/I_\Lambda,\FF)_{n+1}$. 





\subsection{Comparison and Probes}

One of the peculiar consequences of Green's conjecture for odd genus $g=2k+1$ 
is that, if $\beta_{k,k+1} = \beta_{k-1,k+1} \ne 0$, then the curve $C$ lies 
in the closure of the locus of $k+1$-gonal curve. Any $k+1$-gonal curve
lies on a rational normal 
scroll $X$ of codimension $k$ that satisfies 
$\beta_{k,k+1}(X) = k$. Hence
$$\beta_{k,k+1}(C) \ne 0 \Rightarrow \beta_{k,k+1}(C) \ge k$$

We may ask whether a result like this is true for the union of two polar simplices $\Lambda \cup \Gamma \subset \PP^{2k-1}$. Define
$$\tilde D=\{ \Gamma \in V | \Gamma \cup \Lambda \hbox{ is syzygy special} \}$$
where, as above, $V$ denotes the variety of polar simplices and 
$\Lambda$ denotes the coordinate simplex. 
If $\tilde D$ is a proper subvariety, then it is a divisor,
because 
$\beta_{k,k+1}(\Gamma \cup \Lambda) =\beta_{k-1,k+1}(\Gamma \cup \Lambda) $. 


\begin{conjecture}\label{Exceptional locus}  The subscheme 
$\tilde D \subset V$ is an irreducible divisor, 
for $g=n+2=2k+1 \in \{7,9,11\}$. 
The value of $\beta_{k,k+1}$ on a general point of $D$ 
is $3, 2, 1$ respectively.
\end{conjecture} 

We can prove this for $g=7$ by computer algebra.  For
$g=9$ and $g=11$ a proof is
computationally out of reach with our methods, but we can get some
evidence from examples over finite fields.

\partitle{Evidence} 
Since $\tilde D$ is a divisor, we expect that if we pick symmetric
matrices $B$ over $\FF_q$ at random, we will hit points on every component
of $\tilde D$ at a
probability of $1/q$. For a general point on $ \tilde D$  the corresponding Coble
self-dual set of points will have the generic  number of extra syzygies
of that component. Points with even more syzygies will occur in higher 
codimension, hence only with a probability of $1/q^2$. 
Some evidence for irreducibility can be obtained from the Weil formulas:
for sufficiently large $q$ we should see points on $\tilde D$ with
probability
$C q^{-1} + O(q^{-\frac32})$, where $C$ is the number of components.
 
The following tables give for small fields $\FF_q$ the number $s_i$
of examples with $i$ extra syzygies in a test of 1000 examples for $g=9$ and
100 examples for $g=11$. 
The number $s_{\rm tot}= \sum_{i>0} s_i$ is the total number of examples with extra
syzygies.

\noindent
Genus $g=9$:
$$
\def\sb#1{\hbox to .8cm{\hfil$#1$}} %small fixed size box --text mode
\begin{tabular}{r|r|r|r|r|r|r|}
$q$ & $\ {\scriptstyle 1000}/q$ & \sb{s_{\rm tot}} & \sb{s_1} & \sb{s_2} & \sb{s_3} & \sb{s_4}  \cr
\hline
2 & 500 & 925 & 0 & 130 & 0 & 63 \cr
3 & 333 & 782 & 0 & 273 & 0 & 33 \cr
4 & 250 & 521 & 0 & 279 & 0 & 99 \cr
5 & 200 & 350 & 0 & 217 & 0 &74 \cr
7 & 143 & 197 & 0 & 144 & 0 &36 \cr
8 & 125 & 199 & 0 & 147 & 0 & 43 \cr
9 & 111 & 218 & 0 & 98 & 0 & 0 \cr
11 & 91 & 118 & 0 & 102 & 0 & 15 \cr
13 & 77 & 90 & 0 & 79 & 0 & 10 \cr
16 & 62 & 72 & 0 & 68 & 0 & 4 \cr
17 & 59 & 76 & 0 & 69 & 0 & 6 \cr
\end{tabular}$$

\noindent
Genus $g=11$:
$$
\def\sb#1{\hbox to .8cm{\hfil$#1$}} %small fixed size box --text mode
\begin{tabular}{r|r|r|r|r|r|r|}
$q$ & $\ {\scriptstyle 100}/q$ & \sb{s_{\rm tot}} & \sb{s_1} & \sb{s_2} & \sb{s_3} & \sb{s_4}  \cr
\hline
7 & 14.3 & 16 & 14 & 0 & 0 & 0 \cr
17 & 5.9 & 7 & 7 & 0 & 0 & 0 \cr
\end{tabular}$$

\noindent
In view of these numbers, it is more likely that
the set $\tilde D$ of syzygy special Coble points is irreducible 
than that it is reducible. For a more precise statement we refer to \cite{CO:vBS}.

 
 
\medskip
\noindent
{\bf A test of Green's Conjecture for Curves.}
In view of \ref{Exceptional locus} it seems plausible that for a general curve
of odd genus $g \ge 11$ with $\beta_{k,k+1}(C) \ne 0$ the value might
be $\beta_{k,k+1}=1$ contradicting Green's conjecture. 
It is clear that the syzygy 
exceptional locus has codimension 1 in $\gM_g$ for odd genus, if it is proper, 
i.e., if Green's 
conjecture holds for the general curve of that genus. So picking points 
at random we might be able to find such curve over a finite field $\FF_q$
with probability $1/q$, roughly.
%%$$<<\hbox{testGreensConjecture}>>$$ 

Writing code that does this is straightforward. One makes a loop that
picks up randomly a curve, computes its canonical image, and resolves
its ideal, counting the possible values $\beta_{k,k+1}$
until a certain amount of special curves is reached. 
The result for 10 special curves in $\FF_7$ is as predicted:
$$
\def\sb#1{\hbox to .65cm{\hfil$#1$}} %small fixed size box --text mode
\begin{tabular}{r|r|r|r|rrrrrr|}
& &total &\ special &\multicolumn{6}{r}{possible values of $\beta_{k,k+1}$}\vrule\cr
$g$ &\ seconds &\ curves &curves 
&\sb{\le 2}&\sb3&\sb4&\sb5&\sb6&\sb{\ge 7} \cr
\hline
7&148&75&10&0&10&0&0&0&0\cr
9&253&58&10&0&0&9&0&0&1\cr
11&25640&60&10&0&0&0&9&0&1\cr
\end{tabular}
$$
(The test for genus 9 and 11 used about 70
and 120 megabytes of memory, respectively.)


So Green's conjecture passed the test for $g=9,11$. 
Shortly after the first author tried this test for the first time,
a paper of Hirschowitz and Ramanan appeared proving this in general:

\begin{theorem}[\cite{CO:HR}] If the general curve of odd genus $g=2k+1$ 
satisfies Green's conjecture then the syzygy special curves lie on the divisor
$D=\{ C \in \gM_g | W^1_{k+1}(C) \ne \emptyset \}$ 
\end{theorem}

The theorem gives strong evidence for the full Green's conjecture in
view of our study of Coble self-dual sets of points.  

\medskip
Our findings suggest that the variety of points arising as 
hyperplane sections of smooth canonical curves has the strange
property that it intersects the divisor of syzygy special sets of
points $\tilde D$ only in its singular locus. 

\bigskip
The conjecture for general curves is known to us up to $g \le 17$, which is 
as far as a computer allows us to do a ribbon example; see \cite{CO:BaEi}.

%%{\bf The end of the section should be deleted or expanded.}  
%%The following scripts we do these example and allow to verify
%% the generic Green's conjecture
%%up to genus 17 (in that case the tests took 4556 seconds and 113Mbyte of memory).
%%
%%<testGreensConjectureForGivenParams = (a,b,k,l,K) -> (
%%     collectGarbage();
%%     x=symbol x;X=K[x_1..x_a];
%%     mult1Withs2=matrix(apply((k-2)+1,i->(apply((a-1-k)+1,j->x_(i+j+1)))));
%%     mult1Withst=matrix(apply((k-2)+1,i->(apply((a-1-k)+1,j->x_(i+j+2)))));
%%     mult1Witht2=matrix(apply((k-2)+1,i->(apply((a-1-k)+1,j->x_(i+j+3)))));
%%     --the Kozsul matrix \wedge^(k-1) Fa ->\wedge^k Fa ** Fa^ is given by
%%     y=symbol y;Y=K[y_1..y_a];
%%     alpha=transpose koszul(k,vars Y);
%%     --the contraction map
%%     QX=sum(numgens X,i->x_(i+1)^2);
%%     JX=ideal (symmetricPower(2,vars X)*(syz diff(QX, symmetricPower(2,vars X))));  
%%     Xquot=X/JX;XY=X**Y;contractMap=map(Xquot,XY,vars Xquot|vars Xquot);
%%     --Now s^2*(\wedge^(k-1) Fa -> \wedge^k Fa ** Fa^* ** (S_(k-2))^* ** Fa) is symply:
%%     alpha1=substitute(alpha,XY)**substitute(mult1Withs2,XY);
%%     --and we just have to contract the x_i's with the y_i's, which are of bidegree (1,1)
%%     alpha1=substitute(contractMap(alpha1),matrix(K,{toList(numgens X:1)}));
%%     --and the same for the other 2 terms
%%     alpha2=-2*substitute(alpha,XY)**substitute(mult1Withst,XY);
%%     alpha2=substitute(contractMap(alpha2),matrix(K,{toList(numgens X:1)}));
%%     alpha3=substitute(alpha,XY)**substitute(mult1Witht2,XY);
%%     alpha3=substitute(contractMap(alpha3),matrix(K,{toList(numgens X:1)}));
%%     --THE SAME WITH THE S_b, with variables (w,z)
%%     w=symbol w;W=K[w_1..w_b];
%%     mult2Withs2=matrix(apply((l-2)+1,i->(apply((b-1-l)+1,j->w_(i+j+1)))));
%%     mult2Withst=matrix(apply((l-2)+1,i->(apply((b-1-l)+1,j->w_(i+j+2)))));
%%     mult2Witht2=matrix(apply((l-2)+1,i->(apply((b-1-l)+1,j->w_(i+j+3)))));
%%     z=symbol z;Z=K[z_1..z_b];
%%     beta=transpose koszul(l,vars Z);
%%     QW=sum(numgens W,i->w_(i+1)^2);
%%     JW=ideal (symmetricPower(2,vars W)*(syz diff(QW, symmetricPower(2,vars W))));  
%%     Wquot=W/JW;WZ=W**Z;contractMap=map(Wquot,WZ,vars Wquot|vars Wquot);
%%     beta1=substitute(beta,WZ)**substitute(mult2Withs2,WZ);
%%     beta1=substitute(contractMap(beta1),matrix(K,{toList(numgens W:1)}));
%%     beta2=substitute(beta,WZ)**substitute(mult2Withst,WZ);
%%     beta2=substitute(contractMap(beta2),matrix(K,{toList(numgens W:1)}));
%%     beta3=substitute(beta,WZ)**substitute(mult2Witht2,WZ);
%%     beta3=substitute(contractMap(beta3),matrix(K,{toList(numgens W:1)}));
%%     --FINAL EQUATIONS
%%     equat=alpha1**beta1+alpha2**beta2+alpha3**beta3;
%%     <<numgens source equat<<"x"<<numgens target equat<<endl;
%%     rank equat==min(numgens source equat,numgens target equat));>
%%
%%<testGreensConjectureForOddGenera = (g,p) -> (
%%     if p==0 then K=QQ else K=ZZ/p;
%%     test=true;
%%     a:=floor((g-1)/2);
%%     k:=2;while k<=ceiling(a/2) do (
%%          <<k<<","<<a+1-k<<": ";
%%          time test=testGreensConjectureForGivenParams(a,a,k,a+1-k,K); 
%%          if test==false then k=a else k=k+1;);
%%     test);>
%%








\section{Pfaffian Calabi-Yau Threefolds in $\PP^6$}

\ie{Calabi-Yau 3-folds} caught the attention of physicists
because they can serve as the compact factor of the
Kaluza-Klein model of spacetime
in superstring theory.
One of the remarkable things that grows out of the work in physics
is the discovery of mirror symmetry, which associates to a family of
Calabi-Yau 3-folds $(M_\lambda)$, another family $(W_\mu)$ whose Hodge
diamond is the mirror of the Hodge diamond of the original family.

Although there is an enormous amount of evidence at present, the existence 
of a mirror is still a hypothesis for general Calabi-Yau 3-folds. 
The thousands of cases where this was established all are close to toric
geometry, where through the work of Batyrev and others \cite{CO:Ba,CO:CK} 
rigorous mirror
constructions were given and parts of their conjectured properties proved.


\medskip
From a commutative algebra point of view the examples studied so far are 
rather trivial, because nearly all are hypersurfaces or complete 
intersections on toric varieties, or zero loci of sections in 
homogeneous bundles on homogeneous spaces.  

Of course only a few families of Calabi-Yau 3-folds should be of this kind.
Perhaps the easiest examples beyond the toric/homogeneous range are 
Ca\-la\-bi-Yau 3-folds in $\PP^6$. 
Here examples can be obtained by the Pfaffian
construction of Buchsbaum-Eisenbud \cite{CO:BE} with vector bundles; see 
section \ref{Pfaffian complex} below. 
Indeed a recent theorem of Walter \cite{CO:Wa} says 
that any smooth Calabi-Yau in $\PP^6$ can be obtained in this way. 
In this section we report on our construction of such examples. 

As is quite usual in this kind of problem, there is a range where the
construction is still quite easy, 
e.g., for surfaces in $\PP^4$ the work in \cite{CO:DES,CO:Po} 
shows that the construction of nearly all the 50 known families 
of smooth non-general type surfaces is straight forward 
and their Hilbert scheme component unirational. 
Only in very few known examples is the construction more difficult
and the unirationality of the Hilbert scheme component an open problem.

The second author did the first ``non-trivial'' case of a construction of 
Calabi-Yau 3-folds in $\PP^6$. 
Although in the end the families  turned out to be unirational, 
the approach utilized small finite field constructions as a research tool.




\subsection{The Pfaffian Complex}
\index{Pfaffian complex}
Let $\kf$ be a vector bundle of odd rank $\rk \kf = 2r+1$ on a projective
manifold $M$, and let $\kl$ be a line bundle. Let $\varphi 
\in H^0(M,\Lambda^2 \kf \otimes \kl)$ be a section. We can 
think of $\varphi$ as a skew-symmetric twisted homomorphism
$$\kf^* \stackrel{\varphi}{\longrightarrow} \kf \otimes \kl.$$
 
The $r^{th}$ divided power of $\varphi$ is a section
$\varphi^{(r)} = \frac{1}{r!}(\varphi \wedge \dots \wedge \varphi) \in 
H^0(M,$ $\Lambda^{2r}\kf \otimes \kl^r)$.  Wedge product with $\varphi^{(r)}$ 
defines a morphism 
$$ \kf \tensor \kl \stackrel{\psi}{\longrightarrow}  \Lambda^{2r+1} \kf \otimes \kl^{r+1}=\det(\kf)\tensor\kl^{r+1}.$$

The  twisted image 
$\ki = \image(\psi) \tensor \det(\kf^*) \tensor \kl^{-r-1} \subset \ko_M$
is called the {\sl Pfaffian ideal} of $\varphi$, because working locally
with frames, it is given by the ideal generated by the  $2r\times 2r$ principle
Pfaffians of the matrix describing $\varphi$.
Let $\kd$ denote the determinant line bundle $\det(\kf^*)$.

\begin{theorem}[Buchsbaum-Eisenbud \cite{CO:BE}]\label{Pfaffian complex}
With this notation
$$0 \to \kd^2\tensor\kl^{-2r-1}
\stackrel{\psi^t}{\longrightarrow} \kd\tensor\kl^{-r-1}\tensor\kf^* 
\stackrel{\varphi}{\longrightarrow} \kf\tensor\kd\tensor\kl^{-r} 
\stackrel{\psi}{\longrightarrow} \ko_M$$
is a complex. 
$X=V(\ki) \subset M$ has codimension $\le 3$ at every point, and in case equality
holds (everywhere along $X$) then 
this complex is exact and resolves the structure sheaf $\ko_X=\ko_M/\ki$ 
of the  locally Gorenstein subscheme $X$. 
\end{theorem}

We will apply this to construct Calabi-Yau 3-folds in $\PP^6$. 
In that case we want $X$ to be smooth and 
$\det(\kf)^{-2}\tensor\kl^{-2r-1} \cong \omega_\PP\cong \ko(-7)$,
so we may conclude that $\omega_X \cong \ko_X$. 
A result of Walter \cite{CO:Wa} for $\PP^n$ guarantees the existence 
of a Pfaffian presentation in $\PP^6$ for every subcanonical embedded 3-fold. 
Moreover Walter's choice of $\kf\tensor\kd\tensor\kl^{-r}$ for Calabi-Yau 3-folds $X \subset \PP^6$
is the sheafified first syzygy module $H^1_*(\ki_X)$ plus possibly 
a direct sum of line bundles 
(indeed $H^2_*(\ki_X)=0$ because of the Kodaira vanishing theorem). 
Under the maximal rank assumption for
$$H^0(\PP^6,\ko(m)) \to H^0(X,\ko_X(m))$$
the Hartshorne-Rao module is zero for $d=\deg X \in \{12,13,14\}$ and 
an arithmetically Cohen-Macaulay $X$ is readily found. 
For $d \in \{15,16,17,18\}$
the Hartshorne-Rao modules $M$ have Hilbert functions with values
$(0,0,1,0,$ $\ldots)$, $(0,0,2,1,0,\ldots)$, $(0,0,3,5,0,\ldots)$ and 
$(0,0,4,9,0,\ldots)$ respectively.

We do not discuss the cases $d\le 16$ further. 
The construction in those cases is obvious; see \cite{CO:To}.


\subsection{Analysis of the Hartshorne-Rao Module for Degree $17$}

Denote with $\kf_1$ the sheaf $\kf\tensor\kd\tensor\kl^{-r}$.
We try to construct $\kf_1$ as the sheafified first syzygy module of $M$. 
The construction of a module with the desired Hilbert function is straightforward.
The cokernel of $3S(-2) \stackrel{b}{\longleftarrow} 16S(-3)$ for a general 
matrix of linear forms has this property. 
However, for a general $b$ and 
$\kf_1 = \ker( 16\ko(-3) \stackrel{b}{\longrightarrow} 3\ko(-2))$ 
the space of skew-symmetric maps $\Hom_{\rm skew}(\kf_1^*(-7),\kf_1)$ is zero:  
$M$ has syzygies
$$
\begin{tabular}{|cccccccc}
\hline
3 & 16 & 28 & - & - & - & - \cr
- & - & - & 70 & 112 & 84 & 32 & 5 \cr
\end{tabular}
$$
\noindent
Any map $\varphi \colon \kf_1^*(-7) \to \kf_1$ induces a map on the free
resolutions:
$$
\xymatrix{
0 &\kf_1 \ar[l] &28\ko(-4) \ar[l] &70\ko(-6) \ar[l] &112\ko(-7) \ar[l] \\
0 &\kf_1^*(-7) \ar[l] \ar[u]_\varphi &16\ko(-4) \ar[l] \ar[u]_{\varphi_0}
&3\ko(-5) \ar[l] \ar[u]_{\varphi_1} &0 \ar[l]
}
$$
%\begin{diagram}[small] 
%0 &\lTo &\kf &\lTo &28\ko &\lTo &70\ko(-2) &\lTo &112\ko(-3) \cr
%&&\varphi\; \uTo && \varphi_0 \;\uTo && \varphi_1\; \uTo&& \cr
%0&\lTo & \kf^*(1) &\lTo & 16\ko & \lTo &3\ko(-1) &\lTo & 0 \cr 
%\end{diagram}
\noindent
Since $\varphi_1=0$ for degree reasons, $\varphi=0$ as well, and
$\Hom(\kf_1^*(-7),\kf_1)=0$ for a general module $M$.

What we need are special modules $M$ that have extra syzygies 
$$
\begin{tabular}{|cccccccc}
\hline
3 & 16 & 28 & $k$ & - & - & - \cr
- & - & $k$ & 70 & 112 & 84 & 32 & 5 \cr
\end{tabular}
$$
with $k$ at least 3.

In a neighborhood of $o \in \Spec B$, where denotes $B$ the base space of 
a semi-universal deformation of $M$, 
the resolution above would lift to a complex over $B[x_0,\ldots,x_6]$ 
and in the lifted complex there is a $k\times k$ matrix $\Delta$
with entries in the maximal ideal $o \subset B$. 
By the principal ideal theorem we see that Betti numbers stay constant 
in a subvariety of codimension at most $k^2$. 
To check for second linear syzygies on a randomly chosen M 
is a computationally rather easy task.
The following procedure tests the computer speed of this task.
\beginOutput
i19 : testModulesForDeg17CY = (N,k,p) -> (\\
\           x:=symbol x;R:=(ZZ/p)[x_0..x_6];\\
\           numberOfGoodModules:=0;i:=0;\\
\           usedTime:=timing while (i<N) do (\\
\                b:=random(R^3,R^\{16:-1\});\\
\                --we put SyzygyLimit=>60 because we expect \\
\                --k<16 syzygies, so 16+28+k<=60\\
\                fb:=res(coker b, \\
\                     DegreeLimit =>0,SyzygyLimit=>60,LengthLimit =>3);\\
\                if rank fb_3>=k and (dim coker b)==0 then (\\
\                     fb=res(coker b, DegreeLimit =>0,LengthLimit =>4);\\
\                     if rank fb_4==0 \\
\                     then numberOfGoodModules=numberOfGoodModules+1;);\\
\                i=i+1;);\\
\           collectGarbage();\\
\           timeForNModules:=usedTime#0;\\
\           \{timeForNModules,numberOfGoodModules\});\\
\endOutput

Running this procedure we see that it takes not more than 
$0.64$ seconds per example.
Hence we can hope to find examples with $k=3$ within a reasonable time 
for a very small field, say $\FF_3$. 

\medskip
The first surprise is that examples with $k$ extra syzygies 
are found much more often, 
as can be seen by looking at the second value output by the function {\tt testModulesForDeg17CY()}.

This is not only a ``statistical'' remark, in the sense that 
the result is confirmed by computing the semi-universal deformations of these modules.
Indeed define $\MM_k=\{ M \mid \Tor^S_3(M,\FF)_5 \ge k \}$ and 
consider a module $M\in \MM_k$:
``generically'' we obtain $\codim(\MM_k)_M=k$ instead of $k^2$
(and in fact one can diagonalize the matrix $\Delta$ over 
the algebraic closure $\bar{\FF}$).

The procedure is straightforward but a bit long. 
First we pick up an example with $k$-extra syzygies.
\beginOutput
i20 : randomModuleForDeg17CY = (k,R) -> (\\
\           isGoodModule:=false;i:=0;\\
\           while not isGoodModule do (\\
\                b:=random(R^3,R^\{16:-1\});\\
\                --we put SyzygyLimit=>60 because we expect \\
\                --k<16 syzygies, so 16+28+k<=60\\
\                fb:=res(coker b, \\
\                     DegreeLimit =>0,SyzygyLimit=>60,LengthLimit =>3);\\
\                if rank fb_3>=k and (dim coker b)==0 then (\\
\                     fb=res(coker b, DegreeLimit =>0,LengthLimit =>4);\\
\                     if rank fb_4==0 then isGoodModule=true;);\\
\                i=i+1;);\\
\           <<"     -- Trial n. " << i <<", k="<< rank fb_3 <<endl;\\
\           b);\\
\endOutput
Notice that the previous function returns a presentation matrix $b$ of $M$, 
and not $M$.

Next we compute the tangent codimension of $\MM_k$ in the given example 
$M=\Coker b$ by computing the codimension of the space 
of the \ie{infinitesimal deformations} of $M$
that still give an element in $\MM_k$.
Denote with $b_i$ the maps in the linear strand of a minimal free resolution of $M$, 
and with $b_2'$ the quadratic part in the second map of this resolution.
Over $B=\FF[\epsilon]/{\epsilon^2}$ 
let $b_1+\epsilon f_1$ be an infinitesimal deformation of $b_1$. 
Then $f_1$ lifts to a linear map $f_2\colon 28S(-4) \to 16S(-3)$ determined by 
$(b_1+\epsilon f_1)\circ(b_2+\epsilon f_2)=0$, and $f_2$ to a map 
$f_3\oplus\Delta\colon k S(-5) \to 28 S(-4)\oplus k S(-5)$ determined by
$(b_2+b_2')\circ\epsilon (f_3\oplus\Delta)=0$.
Therefore we can determine $\Delta$ as:
\beginOutput
i21 : getDeltaForDeg17CY = (b,f1) -> (\\
\           fb:=res(coker b, LengthLimit =>3);\\
\           k:=numgens target fb.dd_3-28; --# of linear syzygies\\
\           b1:=fb.dd_1;b2:=fb.dd_2_\{0..27\};b2':=fb.dd_2_\{28..28+k-1\};\\
\           b3:=fb.dd_3_\{0..k-1\}^\{0..27\};\\
\           --the equation for f2 is b1*f2+f1*b2=0, \\
\           --so f2 is a lift of (-f1*b2) through b1 \\
\           f2:=-(f1*b2)//b1;\\
\           --the equation for A=(f3||Delta) is -f2*b3 = (b2|b2') * A\\
\           A:=(-f2*b3)//(b2l|b2');\\
\           Delta:=A^\{28..28+k-1\});\\
\endOutput
Now we just parametrize all possible maps $f_1\colon 16S(-3) \to 3S(-2)$,
compute their respective maps $\Delta$,
and find the codimension of the condition that $\Delta$ is the zero map:
\beginOutput
i22 : codimInfDefModuleForDeg17CY = (b) -> (\\
\           --we create a parameter ring for the matrices f1's\\
\           R:=ring b;K:=coefficientRing R;\\
\           u:=symbol u;U:=K[u_0..u_(3*16*7-1)];\\
\           i:=0;while i<3 do (\\
\                <<endl<< " " << i+1 <<":" <<endl;\\
\                j:=0;while j<16 do(\\
\                     << "    " << j+1 <<". "<<endl;\\
\                     k:=0;while k<7 do (\\
\                        l=16*7*i+7*j+k; --index parametrizing the f1's\\
\                        f1:=matrix(R,apply(3,m->apply(16,n->\\
\                             if m==i and n==j then x_k else 0)));\\
\                        Delta:=substitute(getDeltaForDeg17CY(b,f1),U);\\
\                        if l==0 then (equations=u_l*Delta;) else (\\
\                             equations=equations+u_l*Delta;);\\
\                        k=k+1;);\\
\                     collectGarbage(); --frees up memory in the stack\\
\                     j=j+1;);\\
\                i=i+1;);\\
\           codim ideal equations);\\
\endOutput


\medskip
The second surprise is that for $\kf_1 = syz_1(M)$ we find
$$\dim \Hom_{\rm skew}(\kf_1^*(-7),\kf_1) =k= \dim \Tor_3^S(M,\FF)_5.$$
$\Hom_{\rm skew}(\kf_1^*(-7),\kf_1)$ is the vector space of skew-symmetric
linear matrices $\varphi$ such that $b \circ \varphi = 0$.
The following procedure gives a matrix of size
$\binom{16}2\times\dim \Hom_{\rm skew}(\kf_1^*(-7),\kf_1)$
whose $i$-th column gives the entries of a $16\times16$ skew-symmetric matrix 
inducing the $i$-th basis element of the vector space $\Hom_{\rm skew}(\kf_1^*(-7),\kf_1)$.
\beginOutput
i23 : skewSymMorphismsForDeg17CY = (b) -> (\\
\           --we create a parameter ring for the morphisms: \\
\           K:=coefficientRing ring b;\\
\           u:=symbol u;U:=K[u_0..u_(binomial(16,2)-1)];\\
\           --now we compute the equations for the u_i's:\\
\           UU:=U**ring b;\\
\           equationsInUU:=flatten (substitute(b,UU)*\\
\                substitute(genericSkewMatrix(U,u_0,16),UU));\\
\           uu:=substitute(vars U,UU);\\
\           equations:=substitute(\\
\                diff(uu,transpose equationsInUU),ring b);\\
\           syz(equations,DegreeLimit =>0));\\
\endOutput
A morphism parametrized by a column $\tt skewSymMorphism$ is then recovered
by the following code.
\beginOutput
i24 : getMorphismForDeg17CY = (SkewSymMorphism) -> (\\
\           u:=symbol u;U:=K[u_0..u_(binomial(16,2)-1)];\\
\           f=map(ring SkewSymMorphism,U,transpose SkewSymMorphism);\\
\           f genericSkewMatrix(U,u_0,16));\\
\endOutput


\partitle{Rank 1 Linear Syzygies of $M$}
To understand this phenomenon we consider the multiplication tensor of $M$:
$$\mu \colon M_2 \tensor V \to M_3$$
where $V=H^0(\PP^6,\ko(1))$. 

\begin{definition} A decomposable element of $M_2 \tensor V$ in the kernel of 
$\mu$ is called a \ie{rank 1 linear syzygy} of $M$.
The (projective) space of rank 1 syzygies is
$$Y=(\PP^2 \times \PP^6) \cap \PP^{15} \subset \PP^{20}$$
where $\PP^2=\PP(M_2^*),\PP^6=\PP(V^*)$ and $\PP^{15}=\PP(\ker(\mu)^*)$
inside the Segre space $\PP((M_2 \tensor V)^*)\cong \PP^{20}$. 
\end{definition}

Proposition 1.5 of \cite{CO:Gr1} says that, for $\dim M_2 \le j$,
the existence of a $j^{th}$ linear syzygy implies $\dim Y \ge j-1$. 
This is automatically satisfied for $j=3$
in our case: $\dim Y \ge 3$ with equality expected. 

The projection $Y \to \PP^2$ has linear fibers, and the general fiber is a 
$\PP^1$. However, special fibers might have higher dimension. In terms
of the presentation matrix $b$ a special 2-dimensional fiber (defined
over $\FF$) corresponds to a block
$$
b=\begin{pmatrix}
0 & 0 & 0 & * & \ldots \cr
0 & 0 & 0 & * & \ldots \cr
l_1 & l_2 & l_3 & *& \ldots \cr 
\end{pmatrix},
$$
where $l_1,l_2,l_3$ are linear forms, in the $3\times16$ presentation matrix of $M$.
Such a block gives a 
$$
\begin{tabular}{|cccccccc}
\hline
1 & 3 & 3 & 1 & - & - & - &- \cr
- & - & - & - & - & - & - & - \cr
\end{tabular}
$$
subcomplex in the free resolution of $M$ and an element 
$s \in H^0(\PP^6,\Lambda^2 \kf_1 \tensor \ko(7))$
since the syzygy matrix
$$
\begin{pmatrix}
0 & -l_3 & l_2  \cr
l_3 & 0 & -l_1  \cr
-l_2 & l_1 & 0  \cr 
\end{pmatrix}
$$
is skew. 

This answers the questions posed by both surprises: 
we want a module $M$ with at least $k \ge 3$ special
fibers and these satisfy $h^0(\PP^6,\Lambda^2 \kf_1 \tensor \ko(7)) \ge k$, 
if the $k$ sections are linearly independent. 
The condition for $k$ special fibers is of expected codimension $k$ in the
parameter space $\GG(16,3h^0(\PP^6,\ko(1))$ of the presentation matrices.
In a given point $M$ the actual codimension can be readily computed
by a first order deformations and that 
$H^0(\PP^6,\Lambda^2 \kf_1 \tensor \ko(7))$ is $k$-dimensional, and spanned by the
$k$ sections corresponding to the $k$ special fibers can be checked as well. 

First we check that $M$ has $k$ distinct points in $\PP(M_2^*)$ where 
the multiplication map drops rank. 
(Note that this condition is likely to fail over small fields. 
However, the check is computationally easy).
\beginOutput
i25 : checkBasePtsForDeg17CY = b -> (\\
\           --firstly the number of linear syzygies\\
\           fb:=res(coker b, DegreeLimit=>0, LengthLimit =>4);\\
\           k:=#select(degrees source fb.dd_3,i->i==\{3\});\\
\           --then the check\\
\           a=symbol a;A=K[a_0..a_2];\\
\           mult:=(id_(A^7)**vars A)*substitute(\\
\                syz transpose jacobian b,A);\\
\           basePts=ideal mingens minors(5,mult);\\
\           codim basePts==2 and degree basePts==k and distinctPoints(\\
\                basePts));\\
\endOutput
Next we check that $H^0(\PP^6,\Lambda^2 \kf_1 \tensor \ko(7))$ is $k$-dimensional,
by looking at the numbers of columns of {\tt skewSymMorphismsForDeg17CY(b)}.
Finally we do the computationally hard part of the check, which is to
verify that the k special sections corresponding to the k special
fibers of $Y \to \PP^2$ span $H^0(\PP^6,\Lambda^2 \kf_1 \tensor \ko(7))$.
\beginOutput
i26 : checkMorphismsForDeg17CY = (b,skewSymMorphisms) -> (\\
\           --first the number of linear syzygies\\
\           fb:=res(coker b, DegreeLimit=>0, LengthLimit =>4);\\
\           k:=#select(degrees source fb.dd_3,i->i==\{3\});\\
\           if (numgens source skewSymMorphisms)!=k then (\\
\                error "the number of skew-sym morphisms is wrong";);\\
\           --we parametrize the morphisms:\\
\           R:=ring b;K:=coefficientRing R;\\
\           w:=symbol w;W:=K[w_0..w_(k-1)];\\
\           WW:=R**W;ww:=substitute(vars W,WW);\\
\           genericMorphism:=getMorphismForDeg17CY(\\
\                substitute(skewSymMorphisms,WW)*transpose ww);\\
\           --we compute the scheme of the 3x3 morphisms:\\
\           equations:=mingens pfaffians(4,genericMorphism);\\
\           equations=diff(\\
\                substitute(symmetricPower(2,vars R),WW),equations);\\
\           equations=saturate ideal flatten substitute(equations,W);\\
\           CorrectDimensionAndDegree:=(\\
\                dim equations==1 and degree equations==k);\\
\           isNonDegenerate:=#select(\\
\                (flatten degrees source gens equations),i->i==1)==0;\\
\           collectGarbage();\\
\           isOK:=CorrectDimensionAndDegree and isNonDegenerate;\\
\           if isOK then (\\
\                --in this case we also look for a skew-morphism f \\
\                --which is a linear combination of the special \\
\                --morphisms with all coefficients nonzero.\\
\                isGoodMorphism:=false;while isGoodMorphism==false do (\\
\                     evRandomMorphism:=random(K^1,K^k);\\
\                     itsIdeal:=ideal(\\
\                          vars W*substitute(syz evRandomMorphism,W));\\
\                     isGoodMorphism=isGorenstein(\\
\                          intersect(itsIdeal,equations));\\
\                     collectGarbage());\\
\                f=map(R,WW,vars R|substitute(evRandomMorphism,R));\\
\                randomMorphism:=f(genericMorphism);\\
\                \{isOK,randomMorphism\}) else \{isOK\});\\
\endOutput
The code above is structured as follows. 
First we parametrize the skew-symmetric morphisms with new variables.
The ideal of $4\times4$ Pfaffians is generated by forms of bidegree $(2,2)$
over $\PP^6 \times \PP^{k-1}$. We are interested in
points $p \in \PP^{k-1}$ such that the whole fiber 
$\PP^6 \times \{ p \}$ is contained in the zero locus of the Pfaffian
ideal. The next two lines produce the ideal of these points on
$\PP^{k-1}$. Since we already know of $k$ distinct points by the
previous check, it suffices to establish that the set consists of 
collection of k spanning points. Finally, if this is the case, a further
point, i.e., a further skew morphism, is a linear combination with all
coefficients non-zero, if and only if the union with this point is a Gorenstein set
of $k+1$ points in $\PP^{k-1}$.
\beginOutput
i27 : isGorenstein = (I) -> (\\
\           codim I==length res I and rank (res I)_(length res I)==1);\\
\endOutput


It is clear that all 16 relations should take part in the desired
skew homomorphism $\kf_1^*(-7) \stackrel{\varphi}{\longrightarrow} \kf_1$.
Thus we need $k \ge 6$ to have a chance for a Calabi-Yau.
Since $3\cdot5 <16$ it easy to guarantee 5 special fibers by suitable choice
of the presentation matrix. 
So the condition $k \ge 6$ is only of codimension $k-5$ on this subspace, 
and we have a good chance to find a module of the desired type.
\beginOutput
i28 : randomModule2ForDeg17CY = (k,R) -> (\\
\           isGoodModule:=false;i:=0;\\
\           while not isGoodModule do (\\
\                b:=(random(R^1,R^\{3:-1\})++\\
\                     random(R^1,R^\{3:-1\})++\\
\                     random(R^1,R^\{3:-1\})|\\
\                     matrix(R,\{\{1\},\{1\},\{1\}\})**random(R^1,R^\{3:-1\})|\\
\                     random(R^3,R^1)**random(R^1,R^\{3:-1\})|\\
\                     random(R^3,R^\{1:-1\}));\\
\                --we put SyzygyLimit=>60 because we expect \\
\                --k<16 syzygies, so 16+28+k<=60\\
\                fb:=res(coker b, \\
\                     DegreeLimit =>0,SyzygyLimit=>60,LengthLimit =>3);\\
\                if rank fb_3>=k and dim coker b==0 then (\\
\                     fb=res(coker b, DegreeLimit =>0,LengthLimit =>4);\\
\                     if rank fb_4==0 then isGoodModule=true;);\\
\                i=i+1;);\\
\           <<"     -- Trial n. " << i <<", k="<< rank fb_3 <<endl;\\
\           b);\\
\endOutput


\medskip
Some modules $M$ with $k=8,9,11$ lead to smooth examples of 
Calabi-Yau 3-folds $X$ of degree 17. 
To check the smoothness via the Jacobian criterion 
is computationally too heavy for a common computer today.
For a way to speed up this computation considerably 
and to reduce the required amount of memory to a reasonable value (128MB), 
we refer to \cite{CO:To}.

Since $h^0(\PP^6,\Lambda^2 \kf_1 \tensor \ko(7))=k$ 
and $\codim \{ M \mid \Tor^S_3(M,\FF)_5 \ge k \} =k$ all three families
have the same dimension. In particular no family lies in the closure 
of another.

A deformation computation verifies $h^1(X,\kt)=h^1(X,\Omega^2) = 23$. 
Hence a computation of the Hodge numbers $h^q(X,\Omega^p)$ gives the diamond
$$
\begin{matrix}
&&& 1 &&&\cr
&& 0 && 0 &&\cr
& 0 && 1 && 0 &\cr
 1 &&{23} &&{23} && 1\cr
& 0 && 1 && 0 &\cr
&& 0 && 0 &&\cr
&&& 1 &&&\cr 
\end{matrix}
$$

\begin{example}
The following commands give an example of a Calabi-Yau 3-fold in $\PP^6$:
\beginOutput
i29 : K=ZZ/13;\\
\endOutput
\beginOutput
i30 : R=K[x_0..x_6];\\
\endOutput
\beginOutput
i31 : time b=randomModule2ForDeg17CY(8,R);\\
\     -- Trial n. 1757, k=8\\
\     -- used 764.06 seconds\\
\emptyLine
\              3       16\\
o31 : Matrix R  <--- R\\
\endOutput
\beginOutput
i32 : betti res coker b\\
\emptyLine
o32 = total: 3 16 36 78 112 84 32 5\\
\          0: 3 16 28  8   .  .  . .\\
\          1: .  .  8 70 112 84 32 5\\
\endOutput
\beginOutput
i33 : betti (skewSymMorphisms=skewSymMorphismsForDeg17CY b)\\
\emptyLine
o33 = total: 120 8\\
\         -1: 120 8\\
\endOutput
We check whether the base points in $M_0$ are all distinct.
\beginOutput
i34 : checkBasePtsForDeg17CY b\\
\emptyLine
o34 = true\\
\endOutput
Now we check whether the $k$ sections span the morphisms.  
If we get {\tt true} then this is a good module.
\beginOutput
i35 : finalTest=checkMorphismsForDeg17CY(b,skewSymMorphisms);\\
\endOutput
\beginOutput
i36 : finalTest#0\\
\emptyLine
o36 = true\\
\endOutput
We pick up a random morphism involving all $k$ sections.
\beginOutput
i37 : n=finalTest#1;\\
\emptyLine
\              16       16\\
o37 : Matrix R   <--- R\\
\endOutput
 
If all the tests are okay, there should be a high degree syzygy.
\beginOutput
i38 : betti (nn=syz n)\\
\emptyLine
o38 = total: 16 4\\
\          1: 16 3\\
\          2:  . .\\
\          3:  . 1\\
\endOutput
\beginOutput
i39 : n2t=transpose submatrix(nn,\{0..15\},\{3\});\\
\emptyLine
\              1       16\\
o39 : Matrix R  <--- R\\
\endOutput
\beginOutput
i40 : b2:=syz b;\\
\emptyLine
\              16       36\\
o40 : Matrix R   <--- R\\
\endOutput
Finally, compute the ideal of the Calabi-Yau 3-fold in $\PP^6$.
\beginOutput
i41 : j:=ideal mingens ideal flatten(n2t*b2);\\
\emptyLine
o41 : Ideal of R\\
\endOutput
\beginOutput
i42 : degree j\\
\emptyLine
o42 = 17\\
\endOutput
\beginOutput
i43 : codim j\\
\emptyLine
o43 = 3\\
\endOutput
\beginOutput
i44 : betti res j\\
\emptyLine
o44 = total: 1 20 75 113 84 32 5\\
\          0: 1  .  .   .  .  . .\\
\          1: .  .  .   .  .  . .\\
\          2: .  .  .   .  .  . .\\
\          3: . 12  5   .  .  . .\\
\          4: .  8 70 113 84 32 5\\
\endOutput
\end{example}




\subsection{Lift to Characteristic Zero}

\index{lift to characteristic zero}
At this point we have constructed Calabi-Yau 3-folds $X \subset \PP^6$ over
the finite field $\FF_5$ or $\FF_7$. 
However, our main interest is the field of complex numbers $\CC$. 
The existence of a lift to characteristic zero follows by the following argument.

The set
$\MM_k=\{ M \mid \Tor^S_3(M,\FF)_5 \ge k \}$ has codimension at most $k$.
A deformation calculation shows that at our special point
$M^{\rm special} \in \MM(\FF_p)$ the codimension is achieved and that $\MM_k$
is smooth at this point. 
Thus taking a transversal slice defined over $\ZZ$ through this point 
we find a \ie{number field} $K$ and a prime $\gp$ in its ring of integers $O_K$ 
with $O_K/\gp \cong \FF_p$ such that 
$M^{\rm special}$ is the specialization of an $O_{K,\gp}$-valued point of $\MM_k$. 
Over the generic point of $\Spec O_{K,\gp}$ we obtain a $K$-valued point. 
From our computations with {\tt checkBasePtsForDeg17CY()} and 
{\tt checkMorphismsForDeg17CY()},
which explained why $h^0(\PP^6,\Lambda^2 \kf_1^{\rm special} \tensor \ko(7))=k$,
it follows that 
$$H^0(\PP^6_\ZZ\times\Spec O_{K,\gp},\Lambda^2 \kf_1 \tensor \ko(7))$$
is free of rank $k$ over $O_{K,\gp}$.
Hence $\varphi^{\rm special}$ extends to $O_{K,\gp}$ as well,
and by semi-continuity we obtain a smooth Calabi-Yau 3-fold defined over
$K \subset \CC$.

\begin{theorem}[\cite{CO:To}] The Hilbert scheme of smooth Calabi-Yau 3-folds of 
degree $17$ in 
$\PP^6$ has at least 3 components. These three components are reduced  and 
have dimension $23+48$. The corresponding Calabi-Yau 3-folds differ in the
number of quintic generators of their homogeneous ideals,
which are $8$, $9$ and $11$ respectively.   
\end{theorem}

See \cite{CO:To} for more details.
\medskip


Note that we do not give a bound on the degree $[K:\QQ]$ of the number field,
and certainly we are far away from a bound of its discriminant.

This leaves the question open whether these parameter spaces of Calabi-Yau 
3-folds are unirational. Actually they are, as the geometric construction
of modules $M \in \MM_k$ in \cite{CO:To} shows.

A construction of one or several \ie{mirror families} of these Calabi-Yau 3-folds
is an open problem. 

\nocite{CO:vB}




% Local Variables:
% mode: latex
% mode: reftex
% tex-main-file: "chapter-wrapper.tex"
% reftex-keep-temporary-buffers: t
% reftex-use-external-file-finders: t
% reftex-external-file-finders: (("tex" . "make FILE=%f find-tex") ("bib" . "make FILE=%f find-bib"))
% End:
