% $Source: /home/cvs/M2/Macaulay2/ComputationsBook/chapters/preface/chapter.tex,v $
% $Revision: 1.36.2.3 $
% $Date: 2001/05/22 14:47:48 $

\chapter*{Preface}
\markboth{Preface}{Preface}
\addcontentsline{toc}{section}{Preface}

\bgroup

Systems of polynomial equations arise throughout mathematics, science, and
engineering.
Algebraic geometry provides powerful theoretical techniques for
studying the qualitative and quantitative features of their solution sets.
Recently developed algorithms have made theoretical aspects of the subject
accessible to a broad range of mathematicians and scientists.
The algorithmic approach to the subject has two
principal aims: developing new tools for research within mathematics,
and providing new tools for modeling and solving problems
that arise in the sciences and engineering.  A healthy
synergy emerges, as new theorems yield new algorithms and emerging
applications lead to new theoretical questions.

This book presents algorithmic tools for algebraic geometry and experimental 
applications of them.  It also introduces a
software system in which
the tools have been implemented and with which the experiments can
be carried out.  \Mtwo is a computer
algebra system devoted to supporting research in algebraic geometry,
commutative algebra, and their applications. 
The reader of this book will encounter \Mtwo in the context of concrete
applications and
practical computations in algebraic geometry.  

The expositions of the algorithmic tools presented here are designed to serve
as a useful guide for those wishing to bring such tools to bear on their own
problems.
A wide range of mathematical scientists should find these expositions valuable.
This includes both the users of other programs
similar to \Mtwo (for example, Singular and CoCoA) and those who
are not interested in explicit machine computations at all.  

The chapters are ordered roughly by increasing mathematical
difficulty. The first part of the book is meant to be accessible to
graduate students and computer algebra users from across the
mathematical sciences and is primarily concerned
with introducing \Mtwo.  The second part emphasizes the
mathematics: each chapter exposes some domain of mathematics at an
accessible level, presents the relevant algorithms, sometimes with
proofs, and illustrates the use of the program.  In both parts, each chapter comes with
its own abstract and its own bibliography; the index at the back of
the book covers all of them.

One of the first computer algebra packages aimed at algebraic geometry
was {\sl Macaulay,} the predecessor of \Mtwo, written during the years
1983-1993 by Dave Bayer and Mike Stillman. Worst-case estimates
suggested that trying to compute Gr\"obner bases might be a hopeless
approach to solving problems. But from the first prototype, \Macaulay
was successful surprisingly often, perhaps because of the geometrical
origin of the problems attacked.  \Macaulay improved steadily during
its first decade.  It helped transform the theoretical notion of a
projective resolution into an exciting new practical research tool,
and became widely used for research and teaching in commutative
algebra and algebraic geometry.  It was possible to write routines in
the top-level language, and many important algorithms were added by
David Eisenbud and other users, enhancing the system and broadening its
usefulness.

There were certain practical drawbacks for the researcher who wanted to use
\Macaulay effectively.  A minor annoyance was that only finite prime fields
were available as coefficient rings.  The major problem was that the language
made available to users was primitive and barely supported high-level
development of new algorithms; it had few basic data types and didn't support
the addition of new ones. 

\Mtwo is based on experience gained from writing and using its predecessor
\Macaulay, but is otherwise a fresh start.  It was written by Dan Grayson and
Mike Stillman with the generous financial support of the U.S. National
Science Foundation, with the work starting in 1993\footnote{NSF grants DMS
  92-10805, 92-10807, 96-23232, 96-22608, 99-70085, and 99-70348.}.  
It also incorporates
some code from other authors: the package
SINGULAR-FACTORY\footnote{\texttt{SINGULAR-FACTORY},
        a subroutine library for factorization,
        by G.-M. Greuel, R. Stobbe, G. Pfister, H. Schoenemann, and J. Schmidt;
        available at ftp://helios.mathematik.uni-kl.de/pub/Math/Singular/Factory/.}
provides for factorization of polynomials; 
SINGULAR-LIBFAC\footnote{\texttt{SINGULAR-LIBFAC},
        a subroutine library for characteristic sets and irreducible decomposition,
        by M. Messollen;
        available at ftp://helios.mathematik.uni-kl.de/pub/Math/Singular/Libfac/.}
uses FACTORY to enable the computation of characteristic
sets and thus the decomposition of subvarieties into their irreducible
components; and
GNU MP\footnote{\texttt{GMP}, a library for arbitrary precision arithmetic,
        by Torbj\"orn Granlund,
        John Amanatides,
        Paul Zimmermann,
        Ken Weber,
        Bennet Yee,
        Andreas Schwab,
        Robert Harley,
        Linus Nordberg,
        Kent Boortz,
        Kevin Ryde, and
        Guillaume Hanrot;
        available at {ftp://ftp.gnu.org/gnu/gmp/}.}
by Torbj\"orn Granlund and others provides
for multiple precision arithmetic.

\Mtwo aims to support efficient computation associated with a wide
variety of high level mathematical objects, including Galois fields,
number fields, polynomial rings, exterior algebras, Weyl algebras,
quotient rings, ideals, modules, homomorphisms of rings and modules,
graded modules, maps between graded modules, chain complexes, maps
between chain complexes, free resolutions, algebraic varieties, and
coherent sheaves.  To make the system easily accessible, standard 
mathematical notation is followed closely.

As with \Macaulay, it was hoped that users would join in the further
development of new algorithms for \Mtwo, so the developers
tried to make the language available to
the users as powerful as possible, yet easy to use.
Indeed, much of the high-level part of the system is written in the same language
available to the user.  This ensures that the user will find it just as easy
as the developers did to implement a new type of mathematical object or to
modify the high-level aspects of the current algorithms.

The language available to the user is interpreted.  The interpreter itself
is written in a convenient language designed to be mostly type-safe and to
handle memory allocation and initialization automatically.  For maximum
efficiency, the core mathematical algorithms are written in C++ and compiled,
not interpreted.  This includes the arithmetic operations of rings, modules,
and matrices, the Gr\"obner basis algorithm (in several enhanced versions,
tailored for various situations), several algorithms for computing free
resolutions of modules, the algorithm for computing the Hilbert series of a
graded ring or module, the algorithms for computing determinants and
Pfaffians, the basis reduction algorithm, factoring, etc.

In one way \Mtwo is like a standard computer algebra system, such as {\sl
Mathematica} or {\sl Maple}: the user enters
mathematical expressions at the keyboard, and the program computes the value
of the expression and displays the answer.

Here is the first input prompt offered to the user.
\[\text{\tt i1\ :\ }\]
In response to the prompt, the user may enter, for example, a simple arithmetic expression.
\beginOutput
i1 : 3/5 + 7/11\\
\emptyLine
\     68\\
o1 = --\\
\     55\\
\emptyLine
o1 : QQ\\
\endOutput
The answer itself is displayed to the right of the output
label \[\text{\tt o1\ = \ }\] and its type (or class) is displayed to the
right of the following label.
\[\text{\tt o1\ :\ }\]
The symbol {\tt QQ} appearing in this example 
denotes the class of all rational numbers, and is meant to be reminiscent of
the notation $\mathbb Q$.

\Mtwo often finds itself being run in a window with horizontal scroll bars,
so by default it does not wrap output lines, but instead lets them grow
without bound.  This book was produced by an automated mechanism that submits
code provided by the authors to \Mtwo and incorporates the result into the
text.  Output lines that exceed the width of the pages of this book are
indicated with ellipses, as in the following example.
\beginOutput
i2 : 100!\\
\emptyLine
o2 = 93326215443944152681699238856266700490715968264381621468592963895 $\cdot\cdot\cdot$\\
\endOutput

Next we describe an important difference between 
general computer algebra systems (such as {\sl Maple} and {\sl Mathematica})
and \Mtwo.  Before entering an expression involving variables (such as {\tt x+y})
into \Mtwo the user must first create a ring containing those variables.  
Rings are important
objects of study in algebraic geometry; quotient rings of polynomial rings,
for example, encapsulate the essential information about a system of
polynomial equations, including, for example, the field from which the
coefficients are drawn.  Often one has several rings under consideration at
once, along with ring homomorphisms between them, so it is important to treat
them as first-class objects in the computer, capable of being named and
manipulated the same way numbers and characters can be manipulated in simpler
programming languages.

Let's give a hint of the breadth of types of mathematical objects available
in \Mtwo with some examples.
In \Mtwo one defines a quotient ring of a polynomial ring $R$ over the
rational numbers by entering a command such as the one below.
\beginOutput
i3 : R = QQ[x,y,z]/(x^3-y^3-z^3)\\
\emptyLine
o3 = R\\
\emptyLine
o3 : QuotientRing\\
\endOutput
Having done that, we can compute in the ring.
\beginOutput
i4 : (x+y+z)^3\\
\emptyLine
\       2        2     3     2               2        2       2     3\\
o4 = 3x y + 3x*y  + 2y  + 3x z + 6x*y*z + 3y z + 3x*z  + 3y*z  + 2z\\
\emptyLine
o4 : R\\
\endOutput
We can make matrices over the ring.
\beginOutput
i5 : b = vars R\\
\emptyLine
o5 = | x y z |\\
\emptyLine
\             1       3\\
o5 : Matrix R  <--- R\\
\endOutput
\beginOutput
i6 : c = matrix \{\{x^2,y^2,z^2\}\}\\
\emptyLine
o6 = | x2 y2 z2 |\\
\emptyLine
\             1       3\\
o6 : Matrix R  <--- R\\
\endOutput
We can make modules over the ring.
\beginOutput
i7 : M = coker b\\
\emptyLine
o7 = cokernel | x y z |\\
\emptyLine
\                            1\\
o7 : R-module, quotient of R\\
\endOutput
\beginOutput
i8 : N = ker c\\
\emptyLine
o8 = image \{2\} | x  0   -y2 -z2 |\\
\           \{2\} | -y -z2 x2  0   |\\
\           \{2\} | -z y2  0   x2  |\\
\emptyLine
\                             3\\
o8 : R-module, submodule of R\\
\endOutput
We can make projective resolutions of modules.
\beginOutput
i9 : res M\\
\emptyLine
\      1      3      4      4      4\\
o9 = R  <-- R  <-- R  <-- R  <-- R\\
\                                  \\
\     0      1      2      3      4\\
\emptyLine
o9 : ChainComplex\\
\endOutput
We can make projective varieties.
\beginOutput
i10 : X = Proj R\\
\emptyLine
o10 = X\\
\emptyLine
o10 : ProjectiveVariety\\
\endOutput
We can make coherent sheaves and compute their  cohomology.
\beginOutput
i11 : HH^1 cotangentSheaf X \\
\emptyLine
\        1\\
o11 = QQ\\
\emptyLine
o11 : QQ-module, free\\
\endOutput

At this writing, \Mtwo is available for GNU/Linux and other flavors of Unix, 
and also for Microsoft Windows and the Macintosh operating system. Although it can be used
as a free-standing program, it is most convenient to use it in an 
editor's buffer;
Emacs (on Unix or Windows systems) or MPW on Macintosh systems are
currently the editors of choice.
To obtain \Mtwo, download it from the
website\footnote{{\Mtwo}, a software system for research in algebraic geometry,
        by Daniel R. Grayson and Michael E. Stillman;
        available online in source code form and compiled for various architectures 
        at \texttt{http://www.math.uiuc.edu/Macaulay2/}.}
and unpack the file.
Among the resulting files will be a file called {\tt Macaulay2/README.txt},
which you should read.  It will tell you how to run the {\tt setup} script,
and how to install a few lines of code in your {\tt emacs} init file to
enable you to run {\tt M2} in an emacs buffer and to edit \Mtwo code.  A
system administrator of a Unix system may optionally arrange for those lines
of code to be available to every emacs user.

The editors thank the authors of the chapters for their valuable
contributions and hard work, and the National Science Foundation for funding
the development of \Mtwo and for partial funding of the authors who have
contributed to this volume.

\nobreak
\bigskip
\nobreak
\vbox{
    \noindent
    May, 2001       \hfill\itshape   David Eisenbud\break
    \raggedleft                      Daniel R. Grayson\break
                                     Michael E. Stillman\break
                                     Bernd Sturmfels\par
}

\egroup
