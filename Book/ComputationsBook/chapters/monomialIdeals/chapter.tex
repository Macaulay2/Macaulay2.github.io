\title{Monomial Ideals}
\titlerunning{Monomial Ideals}
\toctitle{Monomial Ideals}
\author{Serkan Ho\c{s}ten
        % \inst 1
        \and Gregory G.~Smith
        % \inst 2
        }
\authorrunning{S. Ho\c{s}ten and G. G. Smith}
% \institute{Department of Mathematics, San Francisco State University,
% San Francisco, CA 94132, USA\and Department of Mathematics, University of
% California, Berkeley, CA 94720, USA}
\maketitle

\newtheorem*{sproof}{Proof of Proposition}{\itshape}{\rmfamily}
%% old form caused problems:
%%%% \newtheorem*{sproof}{Proof of Proposition~\ref{pro:complexity}}{\itshape}{\rmfamily}

\newcommand{\supp}{\operatorname{supp}}
\newcommand{\depth}{\operatorname{depth}}
\newcommand{\GL}{\operatorname{GL}}
\newcommand{\initial}{\operatorname{in}}
\newcommand{\Hilb}{\operatorname{Hilb}}
\newcommand{\codim}{\operatorname{codim}}
\newcommand{\sg}{{\sf g}}
\newcommand{\IP}{\operatorname{IP}}

\begin{abstract}
Monomial ideals form an important link between commutative algebra and
combinatorics.  In this chapter, we demonstrate how to implement
algorithms in \Mtwo for studying and using monomial ideals.  We
illustrate these methods with examples from combinatorics, integer
programming, and algebraic geometry.
\end{abstract}


%%--------------------------------------------------------------

An ideal $I$ in $S = \bbbq[x_{1}, \dotsc, x_{n}]$ is called a monomial
ideal\index{monomial ideal}\index{ideal!monomial} if it satisfies any
of the following equivalent conditions:
\begin{enumerate}
\item[$(a)$] $I$ is generated by monomials,
\item[$(b)$] if $f = \sum_{\alpha \in \bbbn^{n}} k_{\alpha}
x^{\alpha}$ belongs to $I$ then $x^{\alpha} \in I$ whenever
$k_{\alpha} \neq 0$,
\item[$(c)$] $I$ is torus-fixed; in other words, if $(c_{1}, \dotsc,
c_{n}) \in (\bbbq^{*})^{n}$, then $I$ is fixed under the action $x_{i}
\mapsto c_{i}x_{i}$ for all $i$.
\end{enumerate}
It follows that a monomial ideal is uniquely determined by the
monomials it contains.  Most operations are far simpler for a
monomial ideal than for an ideal generated by arbitrary polynomials.
In particular, many invariants can be effectively determined for
monomial ideals.  As a result, one can solve a broad collection of
problems by reducing to or encoding data in a monomial ideal.  The aim
of this chapter is to develop the computational aspects of monomial
ideals in \Mtwo and demonstrate a range of applications.

This chapter is divided into five sections.  Each section begins with
a discussion of a computational procedure involving monomial ideals.
Algorithms are presented as \Mtwo functions.  We illustrate these
methods by solving problems from various areas of mathematics.
In particular, we include the \Mtwo code for generating interesting
families of monomial ideals.  The first section introduces the basic
functions on monomial ideals in \Mtwo.  To demonstrate these functions,
we use the Stanley-Reisner ideal associated to a simplicial complex to
compute its $f$-vector.  Next, we present two algorithms for finding a
primary decomposition of a monomial ideal.  In a related example, we
use graph ideals to study the complexity of determining the
codimension of a monomial ideal.  The third section focuses on the
standard pairs of a monomial ideal; two methods are given for finding
the set of standard pairs.  As an application, we use standard pairs
to solve integer linear programming problems.  The fourth section
examines Borel-fixed ideals and generic initial ideals.  Combining
these constructions with distractions, we demonstrate that the Hilbert
scheme $\Hilb^{\, 4t+1}(\bbbp^{4})$ is connected.  Finally, we look at
the chains of associated primes in various families of monomial
ideals.


%%----------------------------------------------------------
\section{The Basics of Monomial Ideals}

Creating monomial ideals in \Mtwo is analogous to creating general
ideals.  The monomial ideal\index{monomial
ideal}\index{ideal!monomial} generated by a sequence or list of
monomials can be constructed with the function {\tt monomialIdeal}\indexcmd{monomialIdeal}.
<<<S = QQ[a, b, c, d]; >>>
<<<I = monomialIdeal(a^2, a*b, b^3, a*c)>>>
<<<J = monomialIdeal{a^2, a*b, b^2}>>>
The type {\tt MonomialIdeal}\indexcmd{MonomialIdeal} is the class of all monomial ideals.  If
an entry in the sequence or list is not a single monomial, then {\tt
monomialIdeal} takes only the leading monomial; recall that every
polynomial ring in \Mtwo is equipped with a monomial ordering.
<<<monomialIdeal(a^2+a*b, a*b+3, b^2+d)>>>

There are also several methods of associating a monomial
ideal to an arbitrary ideal in a polynomial ring.  The most important
of these is the initial ideal\index{initial
ideal}\index{ideal!initial} --- the monomial ideal generated by the
leading monomials of all elements in the given ideal.  When applied to
an {\tt Ideal}, the function {\tt monomialIdeal} returns the initial
ideal.
<<<K = ideal(a^2, b^2, a*b+b*c)>>>
<<<monomialIdeal K>>>
This is equivalent to taking the leading monomials of a Gr\"{o}bner
basis\index{Grobner basis@Gr\"obner basis} for {\tt K}.  In our example, the given
generators for {\tt K} are not a Gr\"{o}bner basis.
<<<monomialIdeal gens K>>>

One can also test if a general ideal is generated by monomials with
the function {\tt isMonomialIdeal}\indexcmd{isMonomialIdeal}.
<<<isMonomialIdeal K>>>
<<<isMonomialIdeal ideal(a^5, b^2*c, d^11)>>>
The usual algebraic operations on monomial ideals are the same as on
general ideals. For example, we have
<<<I+J>>>


%%----------------------------------------------------------
\subsection*{Example: Stanley-Reisner Ideals and $f$-vectors} 

Radical monomial ideals --- ideals generated by squarefree monomials
--- have a beautiful combinatorial interpretation in terms of
simplicial complexes\index{simplicial complex}.  More explicitly, a
simplicial complex $\Delta$ on the vertex set $\{ x_{1}, \dotsc, x_{n}
\}$ corresponds to the ideal $I_{\Delta}$ in $S = \bbbq[x_{1}, \dotsc,
x_{n}]$ generated by all monomials $x_{i_{1}} \dotsb x_{i_{p}}$ such
that $\{x_{i_{1}}, \dotsc, x_{i_{p}} \} \not\in \Delta$. The ideal
$I_{\Delta}$ is called the Stanley-Reisner\index{Stanley-Reisner
ideal}\index{ideal!Stanley-Reisner} ideal of $\Delta$.

To illustrate the connections between Stanley-Reisner ideals and
simplicial complexes, we consider the
$f$-vector\index{f-vector@$f$-vector}\index{simplicial complex!$f$-vector}.
Perhaps the most important invariant of a simplicial complex, the
$f$-vector of a $d$-dimensional simplicial complex $\Delta$ is
$(f_{0}, f_{1}, \dotsc, f_{d}) \in \bbbn^{d+1}$, where $f_{i}$ denotes
the number of $i$-dimensional faces in $\Delta$.  From the monomial
ideal point of view, the $f$-vector is encoded in the Hilbert
series\index{Hilbert series} of the quotient ring $S/I_{\Delta}$ as
follows:

\begin{theorem}
If $\Delta$ is a simplicial complex with $f$-vector $(f_{0}, \dotsc,
f_{d})$, then the Hilbert series of $S / I_{\Delta}$ is
\[
H_{S/I_{\Delta}}(t) = \sum_{i=-1}^{d} \frac{f_{i}t^{i+1}}{(1-t)^{i+1}}
\, ,
\]
where $f_{-1} = 1$.
\end{theorem}

\begin{proof}
Following Stanley~\cite{MR98h:05001}, we work with the fine
grading and then
specialize.  The fine grading of $S$ is the $\bbbz^{n}$-grading
defined by $\deg x_{i} = \mathbf{e}_{i} \in \bbbz^{n}$, where
$\mathbf{e}_{i}$ is the $i$-th standard basis vector.  The
support\index{monomial!support} of a monomial $x^{\alpha}$ is defined
to be the set $\supp(x^{\alpha}) = \{ x_{i} : \alpha_{i} > 0 \}$.
Observe that $x^{\alpha} \neq 0$ in $S / I_{\Delta}$ if and only if
$\supp(x^{\alpha}) \in \Delta$.  Moreover, the nonzero monomials
$x^{\alpha}$ form a $\bbbq$-basis of $S / I_{\Delta}$.  By counting
such monomials according to their support, we obtain the following
expression for the Hilbert series with the fine grading:
\[
H_{S/I_{\Delta}}(\mathbf{t}) = \sum_{F \in \Delta}
\sum_{\substack{\alpha \in \bbbn^{n} \\ \supp(x^{\alpha}) = F}}
\mathbf{t}^{\alpha} = \sum_{F \in \Delta} \prod_{x_{i} \in F}
\frac{t_{i}}{1-t_{i}} \, .
\]
Finally, by replacing each $t_{i}$ with $t$, we complete the
proof.\qed
\end{proof}

Since $H_{S/I_{\Delta}}(t)$ is typically expressed in the form
$\frac{h_{0} + h_{1}t + \dotsb + h_{d}t^{d}}{(1-t)^{d+1}}$, we can
obtain the $f$-vector by using the identity $\sum_{i} h_{i}t^{i} =
\sum_{j=0}^{d} f_{j-1}t^{j}(1-t)^{d-j}$.  In particular, we can
compute $f$-vectors from Stanley-Reisner ideals as follows:
<<<fvector = I -> (
     R := (ring I)/I;
     d := dim R;
     N := poincare R;
     t := first gens ring N;
     while 0 == substitute(N, t => 1) do N = N // (1-t);
     h := apply(reverse toList(0..d), i -> N_(t^i));
     f := j -> sum(0..j+1, i -> binomial(d-i, j+1-i)*h#(d-i));
     apply(toList(0..d-1), j -> f(j)));>>>
For example, we can demonstrate that the $f$-vector of the octahedron
is $(6,12,8)$.
<<<S = QQ[x_1 .. x_6];>>>
<<<octahedron = monomialIdeal(x_1*x_2, x_3*x_4, x_5*x_6)>>>
<<<fvector octahedron>>>

\begin{figure}
\begin{center}
\epsfysize=1.8in \epsfbox{octahedron.eps}
\end{center}
\caption{The octahedron}
\end{figure}

More generally, we can recursively construct simplicial $2$-spheres
with $f_{0} \geq 4$, starting with the tetrahedron, by pulling a point
in the relative interior of a facet.  This procedure leads to the
following family:
<<<simplicial2sphere = v -> ( 
     S := QQ[x_1..x_v]; 
     if v === 4 then monomialIdeal product gens S 
     else ( 
          L := {};
          scan(1..v-4, i -> L = L | apply(v-i-3, 
                    j -> x_i*x_(i+j+4))); 
          scan(2..v-3, i -> L = L | {x_i*x_(i+1)*x_(i+2)}); 
          monomialIdeal L));>>>
<<<apply({4,5,6,7,8}, j -> fvector simplicial2sphere(j))>>>
In fact, it follows from Euler's formula that the $f$-vector of any
simplicial $2$-sphere has the form $(v, 3v-6, 2v-4)$ for $v \geq 4$.
The problem of characterizing the $f$-vectors for triangulations of
$d$-spheres is open for $d \geq 3$.  One of the most important results
in this direction is the upper bound theorem for simplicial spheres
(Corollary~5.4.7 Bruns and Herzog~\cite{MR95h:13020}) which states
that the cyclic polytope has the maximal number of $i$-faces for all
$i$.  We point out that Stanley's proof of this theorem depends
heavily on these methods from commutative algebra.

On the other hand, the $f$-vectors for several major classes of
simplicial complexes have been characterized.  The Kruskal-Katona
theorem (Theorem~8.32 in Ziegler~\cite{MR96a:52011}) gives necessary
and sufficient conditions for a sequence of nonnegative integers to be
an $f$-vector of a simplicial complex.  Stanley~\cite{MR98h:05001}
describes the $f$-vectors of pure shellable complexes and
Cohen-Macaulay complexes.  Given the Betti numbers of a simplicial
complex, Bj\"{o}rner and Kalai~\cite{MR89m:52009} specify the
$f$-vectors.  Finally, the $g$-theorem (Theorem~8.35 in
Ziegler~\cite{MR96a:52011}) characterizes the $f$-vectors for boundary
complexes of a simplicial convex polytope.

For a further study of Stanley-Reisner ideals see Bruns and
Herzog~\cite{MR95h:13020} and Stanley~\cite{MR98h:05001}.  For more
information of $f$-vectors, see Ziegler~\cite{MR96a:52011} and
Bj\"{o}rner~\cite{MR96h:05213}.

%%----------------------------------------------------------
\section{Primary Decomposition}

A primary decomposition\index{primary decomposition} of an ideal $I$
is an expression of $I$ as a finite intersection of primary ideals; an
ideal $J$ is called primary\index{ideal!primary} if $r_{1}r_{2} \in J$
implies either $r_{1} \in J$ or $r_{2}^{\ell} \in J$ for some $\ell >
0$.  Providing an algorithm for computing the primary decomposition of
an arbitrary ideal in a polynomial ring is quite difficult.  However,
for monomial ideals, there are two algorithms which are relatively
simple to describe.

We first present a recursive method for generating an irreducible
primary decomposition.  It is based on the following two observations.
\begin{lemma}
Let $I$ be a monomial ideal in $S = \bbbq[x_1, \dotsc, x_n]$.
\begin{enumerate}
\item[$(1)$] If $I$ is generated by pure powers of a subset of the
variables, then it is a primary ideal.
\item[$(2)$] If $r$ is minimal generator of $I$ such that $r =
r_{1}r_{2}$ where $r_{1}$ and $r_{2}$ are relatively prime, then $I =
\big(I + \langle r_{1} \rangle \big) \cap \big(I + \langle r_{2}
\rangle \big)$.
\end{enumerate}
\end{lemma}

\begin{proof}
$(1)$ This follows immediately from the definition of primary.  $(2)$
Since $I$ is a monomial ideal, it is enough to show that $I$ and
$\big( I + \langle r_{1} \rangle \big) \cap \big( I + \langle r_{1}
\rangle \big)$ contain the same monomials.  A monomial $r'$ belongs to
$\big( I + \langle r_{j} \rangle \big)$ if and only if $r' \in I$ or
$r_{j}$ divides $r'$.  Because $r_{1}$ and $r_{2}$ are relative prime,
we have
\[
r' \in \big( I + \langle r_{1} \rangle \big) \cap \big( I + \langle
r_{1} \rangle \big) \Leftrightarrow \text{ $r' \in I$ or $r_{1}r_{2}$
divides $r'$ } \Leftrightarrow \text{ $r' \in I$} \, . \qquad \qed
\]
\end{proof}

The following is an implementation of the resulting algorithm:
<<<supp = r -> select(gens ring r, e -> r % e == 0);>>>
<<<monomialDecompose = method();>>>
<<<monomialDecompose List := L -> (
     P := select(L, I -> all(first entries gens I, 
               r -> #supp(r) < 2) === false);
     if #P > 0 then (
          I := first P;
          m := first select(first entries gens I, 
               r -> #supp(r) > 1);
          E := first exponents m;
          i := position(E, e -> e =!= 0);
          r1 := product apply(E_{0..i}, (gens ring I)_{0..i}, 
               (j, r) -> r^j);
          r2 := m // r1;
          monomialDecompose(delete(I, L) | {I+monomialIdeal(r1),
                    I+monomialIdeal(r2)}))
     else L);>>>
<<<monomialDecompose MonomialIdeal := I -> monomialDecompose {I};>>>
Here is a small example illustrating this method.
<<<S = QQ[a,b,c,d];>>>
<<<I = monomialIdeal(a^3*b, a^3*c, a*b^3, b^3*c, a*c^3, b*c^3)>>>
<<<P = monomialDecompose I;>>>
<<<scan(P, J -> << endl << J << endl);>>>
<<<I == intersect(P)>>>
As we see from this example, this procedure doesn't necessarily
yield an irredundant decomposition.

The second algorithm for finding a primary decomposition of a monomial
ideal $I$ is based on the Alexander dual\index{Alexander
dual}\index{dual, Alexander} of $I$.  The Alexander dual was first
introduced for squarefree monomial ideals.  In this case, it is the
monomial ideal of the dual of the simplicial complex $\Delta$
corresponding to $I$.  By definition the dual complex of $\Delta$ is
$\Delta^{\vee} = \{ F : F^{c} \not\in \Delta \}$, where $F^{c} =
\{x_{1}, \dotsc, x_{n}\} \setminus F$.  The following general
definition appears in Miller~\cite{M},~\cite{MR1779598}.  If
$I \subseteq \bbbq[x_{1},
\dotsc, x_{n}]$ is a monomial ideal and $x^{\lambda}$ is the least
common multiple of the minimal generators of $I$, then the Alexander
dual of $I$ is
\[
I^{\vee} = \left\langle \prod_{\beta_{i} > 0}
x_{i}^{\lambda_{i}+1-\beta_{i}} : \begin{tabular}{l} \text{$\langle
x_{i}^{\beta_{i}} : \beta_{i} \geq 1\rangle$ is an
irredundant} \\ \text{irreducible component of $I$} \end{tabular}
\right\rangle \, .
\]
In particular, the minimal generators of $I^{\vee}$ correspond to the
irredundant irreducible components of $I$.  The next proposition
provides a useful way of computing $I^{\vee}$ given a set of
generators for $I$.

\begin{proposition}  
If $I$ is a monomial ideal and $x^{\lambda}$ is the least common
multiple of the minimal generators of $I$, then the generators for
$I^{\vee}$ are those generators of the ideal $\left( \langle
x_{1}^{\lambda_{1}+1}, \dotsc, x_{n}^{\lambda_{n}+1} \rangle : I
\right)$ that are not divisible by $x_{i}^{\lambda_{i}+1}$ for $1 \leq
i \leq n$.
\end{proposition}

\begin{proof}
See Theorem~2.1 in Miller~\cite{M}. \qed
\end{proof}

Miller's definition of Alexander dual is even more general than the one
above. The resulting algorithm for computing this general Alexander
dual and primary decomposition are implemented in \Mtwo as follows. 
For the Alexander dual we use, the list {\tt a} that appears
as an input argument for {\tt dual}\indexcmd{dual} should be list of exponents of
the least common multiple of the minimal generators of $I$.
<<<code(dual, MonomialIdeal, List)>>>
<<<code(primaryDecomposition, MonomialIdeal)>>>
This direct algorithm is more efficient than our recursive algorithm.
In particular, it gives an irredundant decomposition.  For example,
when we use it to determine a primary decomposition for the ideal {\tt
I} above, we obtain
<<<L = primaryDecomposition I;>>>
<<<scan(L, J -> << endl << J << endl);>>>
<<<I == intersect L>>>

For a family of larger examples, we consider the tree
ideals\index{ideal!tree}:
\[
\left\langle \big({\textstyle \prod_{i \in F}} x_{i}\big)^{n-|F|+1} :
\text{ $\emptyset \neq F \subseteq \{ x_{1}, \dotsc, x_{n} \}$}
\right\rangle \, .
\]
These ideals are so named because their standard monomials (the
monomials not in the ideal) correspond to trees on $n+1$ labeled
vertices.  We determine the number of irredundant irreducible
components as follows:
<<<treeIdeal = n -> (
     S = QQ[vars(0..n-1)];
     L := delete({}, subsets gens S);
     monomialIdeal apply(L, F -> (product F)^(n - #F +1)));>>>
<<<apply(2..6, i -> #primaryDecomposition treeIdeal i)>>>


%%----------------------------------------------------------
\subsection*{Example: Graph Ideals and Complexity Theory}

Monomial ideals also arise in graph theory.  Given a graph $G$ with
vertices $\{x_{1}, \dotsc, x_{n}\}$, we associate the
ideal\index{ideal!graph} $I_{G}$ in $\bbbq[x_{1}, \dotsc, x_{n}]$
generated by the quadratic monomials $x_{i}x_{j}$ such that $x_{i}$ is
adjacent to $x_{j}$.  The primary decomposition of $I_{G}$ is related to
the graph $G$ as follows.  Recall that a subset $F \subseteq \{x_{1},
\dotsc, x_{n}\}$ is called a {\em vertex cover}\index{vertex
cover}\index{cover} of $G$ if each edge in $G$ is incident to at least
one vertex in $F$.

\begin{lemma} \label{lem:graphideal}
If $G$ is a graph and $\mathcal{C}$ is the set of minimal vertex
covers of $G$ then the irreducible irredundant primary
decomposition\index{primary decomposition} of $I_{G}$ is $\bigcap_{F
\in \mathcal{C}} P_{F^{c}}$, where $P_{F^{c}}$ is the prime ideal
$\langle x_{i} : x_{i} \not\in F^{c} \rangle =
\langle x_{i} : x_{i} \in F \rangle$.
\end{lemma}

\begin{proof}
Since each generator of $I_{G}$ corresponds to an edge in $G$, it
follows from the {\tt monomialDecompose} algorithm that $I_{G}$ has an
irreducible primary decomposition of the form: $I_{G} = \bigcap
P_{F^{c}}$, where $F$ is a vertex cover.  To obtain an
irredundant decomposition, one clearly needs only the minimal vertex
covers. \qed
\end{proof}

As an application of graph ideals, we examine the complexity of
determining the codimension\index{ideal!codimension} of a monomial
ideal.  In fact, following Bayer and Stillman~\cite{MR94f:13018}, we
prove

\begin{proposition} \label{pro:complexity}
The following decision problem is NP-complete: 
\begin{equation} \tag{{\sc Codim}}
\text{\begin{minipage}[c]{200pt}
Given a monomial ideal $I \subseteq \bbbq[x_1, \dotsc, x_n]$
and $m \in \bbbn$, is $\codim I \leq m$?
\end{minipage}}
\end{equation}
\end{proposition}

By definition, a decision problem is NP-complete\index{NP-complete} if
all other problems in the class NP can be reduced to it.  To prove
that a particular problem is NP-complete, it suffices to show: $(1)$
the problem belongs to the class NP; $(2)$ some known NP-complete
problem reduces to the given decision problem (see Lemma~2.3 in Garey
and Johnson~\cite{MR80g:68056}).  One of the ``standard NP-complete''
problems (see section~3.1 in Garey and Johnson~\cite{MR80g:68056}) is
the following:
\begin{equation} \tag{{\sc Vertex Cover}}
\text{\begin{minipage}[c]{200pt} Given a graph $G$ and $m \in \bbbn$,
is there a vertex cover $F$ such that $|F| \leq m$? \end{minipage}}
\end{equation}

\begin{sproof}
$(1)$ Observe that a monomial ideal $I$ has codimension at most $m$ if
and only if $I \subseteq P_{F^{c}}$ for some $F$ with $|F|
\leq m$.  Now, if $I$ has codimension at most $m$, then given an
appropriate choice of $F$, one can verify in polynomial time that $I
\subseteq P_{F^{c}}$ and $|F| \leq m$.  Therefore, the {\sc
Codim} problem belongs to the class NP.

$(2)$ Lemma~\ref{lem:graphideal} implies that $I_{G}$ has codimension
$m$ if and only if $G$ has a vertex cover of size at most $m$.  In
particular, the {\sc Vertex Cover} problem reduces to the {\sc Codim}
problem. \qed
\end{sproof}

Thus, assuming P $\neq$ NP, there is no polynomial time algorithm for
finding the codimension of a monomial ideal.  Nevertheless, we can
effectively compute the codimension for many interesting examples.

To illustrate this point, we consider the following family of
examples.  Let $S = \bbbq[X]$ denote the polynomial ring generated by
the entries of a generic $m \times n$ matrix $X = [x_{i,j}]$.  Let
$I_{k}$\index{ideal!of minors} be the ideal generated by the $k \times
k$ minors of $X$.  Since the Hilbert function of $S / I_{k}$ equals
the Hilbert function of $S / \initial(I_{k})$ (see Theorem~15.26 in
Eisenbud~\cite{MR97a:13001}), we can determine the codimension $I_{k}$
by working with the monomial ideal $\initial(I_{k})$.  Because
Sturmfels~\cite{MR91m:14076} shows that the set of $k \times k$-minors
of $X$ is the reduced Gr\"{o}bner basis of $I_{k}$ with respect to the
lexicographic term order induced from the variable order
\[
x_{1,n} > x_{1,n-1} > \dotsb > x_{1,1} > x_{2,n} > \dotsb > x_{2,1} >
\dotsb > x_{m,n} > \dotsb > x_{m,1} \, ,
\] 
we can easily calculate $\initial(I_k)$.  In particular, in \Mtwo we
have
<<<minorsIdeal = (m,n,k) -> (
     S := QQ[x_1..x_(m*n), MonomialOrder => Lex];
     I := minors(k, matrix table(m, n, (i,j) -> x_(i*n+n-j)));
     forceGB gens I;
     I);>>>
<<<apply(2..8, i -> time codim monomialIdeal minorsIdeal(i,2*i,2))>>>
The properties of $I_{k}$ are further developed in chapter~11 of
Sturmfels~\cite{MR97b:13034} and chapter~7 of Bruns and
Herzog~\cite{MR95h:13020}

For more on the relationships between a graph and its associated
ideal, see Villarreal~\cite{MR91b:13031}, Simis, Vasconcelos and
Villarreal~\cite{MR99c:13004}, and Ohsugi and
Hibi~\cite{MR2000a:13010}.

<<<erase symbol x;>>>

%%----------------------------------------------------------
\section{Standard Pairs}

In this section, we examine a combinatorial object associated to a
monomial ideal.  In particular, we present two algorithms for
computing the standard pairs of a monomial ideal from its minimal
generators.  Before giving the definition of a standard pair, we
consider an example.

\begin{figure} 
\begin{center}
\epsfysize=2.4in \epsfbox{standardpairs-fixed.eps}
\end{center}
\caption{Staircase diagram for $I = \langle xy^3z, xy^2z^2, y^3z^2,
y^2z^3 \rangle$}
\end{figure}

\begin{example} \label{std-ex} 
Let $I = \langle xy^3z, xy^2z^2, y^3z^2, y^2z^3 \rangle$ in
$\bbbq[x,y,z]$.  We identify the monomials in $\bbbq[x,y,z]$
with the lattice points in $\bbbn^3$; see Figure~2.  The
standard monomials of $I$, those monomials which are not in $I$, can
be enumerated as follows: $(i)$ monomials corresponding to lattice
points in the $xy$-plane, $(ii)$ monomials corresponding to lattice
points in the $xz$-plane, $(iii)$ monomials corresponding to lattice
points in the plane parallel to the $xz$-plane containing $(0,1,0)$,
$(iv)$ monomials corresponding to lattice points on the line parallel
to the $y$-axis containing $(0,0,1)$, $(v)$ monomials corresponding to
lattices point on the line parallel to the $x$-axis containing
$(0,2,1)$, and $(vi)$ the monomial $y^{2}z^{2}$.
\end{example}

Following Sturmfels, Trung and Vogel~\cite{MR96i:13029}, we make the
following definitions.  Given a monomial $x^{\alpha}$ and a subset $F
\subseteq \{ x_{1}, \dotsc, x_{n} \}$, we index the set of monomials
of the form $x^{\alpha} \cdot x^{\beta}$ where $\supp(x^{\beta})
\subseteq F$ by the pair $(x^{\alpha}, F)$.  A standard
pair\index{standard pair} of a monomial ideal $I$ is a pair
$(x^{\alpha}, F)$ satisfying the following three conditions:
\begin{enumerate}
\item[$(1)$] $\supp(x^{\alpha}) \cap F = \emptyset$,
\item[$(2)$] all of the monomials represented by this pair are
standard, and
\item[$(3)$] $(x^{\alpha}, F) \not\subseteq (x^{\beta}, G)$ for any
other pair $(x^{\beta}, G)$ satisfying the first two conditions.
\end{enumerate}
Hence, the six standard pairs 
\[ 
(1, \{x,y\}), (1, \{x,z\}), (y,\{x,z\}), (z, \{y\}), (y^{2}z, \{x\}),
(y^2z^2, \emptyset)
\]
in Example~\ref{std-ex} correspond to $(i)$--$(vi)$.

Observe that the set of standard pairs of $I$ gives an irreducible
decomposition\index{primary decomposition} of $I = \bigcap \langle
x_{i}^{\alpha_{i}+1} : \text{ $x_{i} \notin F$} \rangle$, where the
intersection is over all standard pairs $(x^{\alpha}, F)$.  Moreover,
the prime ideal $P_F := \langle x_{i} : \text{ $x_{i} \notin F$}
\rangle$ is an associated prime of $I$ if and only if there exists a
standard pair of the form $(\bullet, F)$; see Sturmfels, Trung and
Vogel~\cite{MR96i:13029} for details.

Our first algorithm for computing the set of standard
pairs\index{standard pair} is taken from Ho\c{s}ten and
Thomas~\cite{MR2000f:13052}.  The ideas behind it are as follows:
given a witness $w_1 = x^{\alpha}$ for the associated prime $P_F :=
\langle x_{i} : \text{ $x_{i} \notin F$} \rangle$, that is $(I :
x^{\alpha}) = P_F$, set $w_2 = \prod_{x_i \in \supp(w_1) \cap F^c}
x_i^{\alpha_{i}}$.  It follows that $(w_2, F)$ is a standard pair of
$I$.  Now, consider the standard pairs of the slightly larger ideal $I
+ \langle w_1 \rangle$.  Clearly $(w_2, F)$ is not a standard pair of
this ideal because $w_1$ ``destroys'' it.  This larger ideal might
have standard pairs which cover standard monomials in $(w_2, F)$ that
are not in the pair $(w_1, F)$.  However, all other standard pairs are
the same as the original ideal $I$.  Thus, the problem of finding all
standard pairs of $I$ reduces to determining if a standard pair of $I
+ \langle w_1 \rangle$ is a standard pair for $I$.  To decide if a
pair $(x^{\beta}, G)$ of $I + \langle w_1 \rangle$ is a standard pair
of $I$, we first check that $P_{F}$ is an associated prime of $I$.  If
this is true, we determine if $(w_{2},F)$ is covered by
$(x^{\beta},G)$.

The \Mtwo version of this algorithm takes the following form:
<<<stdPairs = I -> (
     S := ring I;
     X := gens S;
     std := {};
     J := I;
     while J != S do (
          w1 := 1_S;
          F := X;
          K := J;
          while K != 0 do (
               g1 := (ideal mingens ideal K)_0;
               x := first supp g1;
               w1 = w1 * g1 // x;
               F = delete(x, F);
               K = K : monomialIdeal(g1 // x);
               L := select(first entries gens K, 
                    r -> not member(x, supp r));
               if #L > 0 then K = monomialIdeal L
               else K = monomialIdeal 0_S;);
          w2 := w1;
          scan(X, r -> if not member(r, supp w1) or member(r, F)
               then w2 = substitute(w2, {r => 1}));
          P := monomialIdeal select(X, r -> not member(r, F));
          if (I:(I:P) == P) and (all(std, p -> 
                    (w2 % (first p) != 0) or not
                    isSubset(supp(w2 // first p) | F, last p)))
          then std = std | {{w2, F}};
          J = J + monomialIdeal(w1););
     std);>>>
We can compute the standard pairs of Example~\ref{std-ex} using this
\Mtwo function:
<<<S = QQ[x,y,z];>>>
<<<I = monomialIdeal(x*y^3*z, x*y^2*z^2, y^3*z^2, y^2*z^3);>>>
<<<scan(time stdPairs I, P -> << endl << P << endl);>>>

Our second algorithm is taken from section~3.2 of Saito, Sturmfels and
Takayama~\cite{MR1734566}.  The proposition below provides the main
ingredient for this algorithm.  If $I$ is a monomial ideal and $F
\subseteq \{ x_{1},\dotsc, x_{n} \}$, we write $I_F$ for the monomial
ideal in $\bbbq[x_i : \text{ $x_{i} \notin F$}]$ obtained by replacing
each $x_{i} \in F$ with $1$ in every minimal generator of $I$.

\begin{proposition} 
For $(x^{\alpha}, F)$ to be a standard pair of $I$, it is necessary
and sufficient that $(x^{\alpha}, \emptyset)$ be a standard pair of
$I_F$.
\end{proposition}

\begin{proof} 
Lemma 3.1 in Sturmfels, Trung and Vogel \cite{MR96i:13029}.
\end{proof}

The definition of a standard pair\index{standard pair} implies that
$(x^{\alpha}, \emptyset)$ is a standard pair of $I_F$ if and only if
$x^{\alpha}$ is one of the finitely many monomials contained in $(I_F
: P_F^\infty)$ but not contained in $I_F$, where $P_F = \langle x_i:
x_i \notin F \rangle$.  Since ideal quotients and saturations are
implemented in \Mtwo, this reduces the problem to finding a set $D$
which contains $F$ for every associated prime $P_{F}$ of $I$.  One
approach is to simply compute the associated primes of $I$ from a
primary decomposition.

The method {\tt standardPairs}\indexcmd{standardPairs} uses this algorithm to determine the
set of standard pairs for a monomial ideal.
<<<code(standardPairs, MonomialIdeal, List)>>>
<<<time standardPairs I;>>>

As an example, we will compute the standard pairs of the permutahedron
ideal\index{ideal!permutahedron}. Let $S = \bbbq[x_1, \dotsc, x_n]$
and let $\mathfrak{S}_n$ be the symmetric group of order $n$.  We
write $\rho$ for the vector $(1, 2, \dotsc, n)$ and $\sigma(\rho)$ for
the vector obtained by applying $\sigma \in \mathfrak{S}_n$ to the
coordinates of $\rho$. The $n$-th permutahedron ideal is $\langle
x^{\sigma(\rho)} : \text{ $\sigma \in \mathfrak{S}_n$} \rangle$.  We
compute the number of standard pairs for $2 \leq n \leq 5$.
<<<permutohedronIdeal = n -> (
     S := QQ[X_1..X_n];
     monomialIdeal terms det matrix table(n ,gens S, 
          (i,r) -> r^(i+1)));>>>
<<<L = apply({2,3,4,5}, j -> standardPairs(permutohedronIdeal(j)));>>>
<<<apply(L, i -> #i)>>>
<<<erase symbol x; erase symbol z;>>>


%%----------------------------------------------------------
\subsection*{Example: Integer Programming Problems}

As an application of standard pairs\index{standard pair}, we show how
to solve integer linear programming problems\index{integer programming}.
Let $A$ be a $d \times n$ matrix of nonnegative integers, let $\omega
\in \bbbr^{n}$ and fix $\beta \in \bbbz^{d}$.  We focus on the
following optimization problem
\[
\IP_{A,\omega}(\beta) : \,\,\, \text{ minimize $\omega \cdot \alpha$
subject to $A \alpha = \beta$, $\alpha \in \bbbn^n$.}
\] 
We view this integer linear program as a family depending on the
vector $\beta$.  The algorithm we present for solving
$\IP_{A,\omega}(\beta)$ depends on the proposition below.

The toric ideal\index{toric ideal}\index{ideal!toric} $I_{A} \subseteq
S = \bbbq[x_{1}, \dotsc, x_{n}]$ associated to $A$ is the binomial
ideal generated by $x^{\gamma} - x^{\delta}$ where $\gamma$, $\delta
\in \bbbn^{n}$ and $A \gamma = A \delta$.  We write $\initial_{\,
\omega}(I_{A})$ for the initial ideal\index{initial ideal} of $I_A$
with respect to the following order:
\[
x^{\gamma} \prec_{\omega} x^{\delta} \Longleftrightarrow \begin{cases}
\omega \cdot \gamma < \omega \cdot \delta & \text{ or } \\ 
\omega \cdot \gamma = \omega \cdot \delta & \text{ and } x^{\alpha}
\prec_{{\rm rlex}} x^{\gamma} \, .
\end{cases}
\]
For more information on toric ideals and their initial ideals see
Sturmfels~\cite{MR97b:13034}.

\begin{proposition} 
\begin{enumerate}
\item[$(1)$] A monomial $x^{\alpha}$ is a standard monomial of
$\initial_{\, \omega}(I_{A})$ if and only if $\alpha$ is the optimal
solution to the integer program $\IP_{A,\omega}(A\alpha)$.
\item[$(2)$] If $(\bullet, F)$ is a standard pair of
$\initial_{\, \omega}(I_{A})$, then the columns of $A$ corresponding to
$F$ are linearly independent.
\end{enumerate}
\end{proposition} 

\begin{proof} 
See Proposition~2.1 in Ho\c{s}ten and Thomas~\cite{MR2000b:13037} for
the proof of the first statement. The second statement follows from
Corollary~2.9 of the same article.
\qed
\end{proof}

The first statement implies that the standard pairs of $\initial_{\,
\omega}(I_{A})$ cover all optimal solutions to all integer programs
in $\IP_{A,\omega}$.  If $\alpha$ is the optimal solution to
$\IP_{A,\omega}(\beta)$ covered by the standard pair $(x^{\gamma},
F)$, then the second statement guarantees there exists a unique
$\delta \in \bbbn^{n}$ such that $A \delta = \beta - A \gamma$.
Therefore, $\alpha = \delta + \gamma$.  We point out that the
complexity of this algorithm is dominated by determining the set of
standard pairs of $\initial_{\, \omega}(I_{A})$ which depends only on
$A$ and $\omega$.  As a result, this method is particularly well
suited to solving $\IP_{A,\omega}(\beta)$ as $\beta$ varies.


To implement this algorithm in \Mtwo, we need a function which returns
the toric ideal $I_A$.  Following Algorithm~12.3 in
Sturmfels~\cite{MR97b:13034}, we have
<<<toBinomial = (b, S) -> (
     pos := 1_S;
     neg := 1_S;
     scan(#b, i -> if b_i > 0 then pos = pos*S_i^(b_i)
                   else if b_i < 0 then neg = neg*S_i^(-b_i));
     pos - neg);>>>
<<<toricIdeal = (A, omega) -> (
     n := rank source A;
     S = QQ[x_1..x_n, Weights => omega, MonomialSize => 16];
     B := transpose matrix syz A;
     J := ideal apply(entries B, b -> toBinomial(b, S));
     scan(gens S, r -> J = saturate(J, r));
     J);>>>
Thus, we can solve $\IP_{A,\omega}(\beta)$ using the following
function.
<<<IP = (A, omega, beta) -> (
     std := standardPairs monomialIdeal toricIdeal(A, omega);
     n := rank source A;
     alpha := {};
     Q := first select(1, std, P -> (
          F := apply(last P, r -> index r);
          gamma := transpose matrix exponents first P;
          K := transpose syz (submatrix(A,F) | (A*gamma-beta));
          X := select(entries K, k -> abs last(k) === 1);
          scan(X, k -> if all(k, j -> j>=0) or all(k, j -> j<=0)
               then alpha = apply(n, j -> if member(j, F) 
                    then last(k)*k_(position(F, i -> i === j))
                    else 0));
          #alpha > 0));
     if #Q > 0 then (matrix {alpha})+(matrix exponents first Q)
     else 0);>>>

We illustrate this with some examples.
<<<A = matrix{{1,1,1,1,1},{1,2,4,5,6}}>>>
<<<w1 = {1,1,1,1,1};>>>
<<<w2 = {2,3,5,7,11};>>>
<<<b1 = transpose matrix{{3,9}}>>>
<<<b2 = transpose matrix{{5,16}}>>>
<<<IP(A, w1, b1)>>>
<<<IP(A, w2, b1)>>>
<<<IP(A, w1, b2)>>>
<<<IP(A, w2, b2)>>>

%%----------------------------------------------------------
\section{Generic Initial Ideals}

Gr\"{o}bner basis calculations and initial ideals depend heavily on
the given coordinate system.  By making a generic change of coordinates
before taking the initial ideal, we may eliminate this dependence.
This procedure also endows the resulting monomial ideal with a rich
combinatorial structure.

To describe this structure, we introduce the following definitions and
notation.  Let $S = \bbbq[x_{0}, \dotsc, x_{n}]$.  If $\sg = [g_{i,j}]
\in \GL_{n+1}(\bbbq)$ and $f \in S$, then $\sg \cdot f$ denotes the
standard action of the general linear group on $S$: $x_{i} \mapsto
\sum_{j=0}^{n} g_{i,j} x_{j}$.  For an ideal $I \subseteq S$, we
define $\sg \cdot I = \{ \sg \cdot f | f \in I \}$.  Let $B$ denote
the Borel subgroup of $\GL_{n+1}(\bbbq)$ consisting of upper
triangular matrices.  A monomial ideal $I$ is called
{\em Borel-fixed}\index{Borel-fixed}\index{ideal!Borel-fixed} if it
satisfies any of the following equivalent conditions:
\begin{enumerate}
\item[$(a)$] $\sg \cdot I = I$ for every $\sg \in B$;
\item[$(b)$] if $r$ is a generator of $I$ divisible by $x_{j}$ then
$\frac{rx_{i}}{x_{j}} \in I$ for all $i < j$;
\item[$(c)$] $\initial(\sg \cdot I) = I$ for every $\sg$ is some open
neighborhood of the identity in $B$.
\end{enumerate}
For a proof that these conditions are equivalent, see Propositon~1.25
in Green~\cite{MR99m:13040}.  

In \Mtwo, the function {\tt isBorel}\indexcmd{isBorel} tests whether a monomial ideal is
Borel-fixed.
<<<S = QQ[a,b,c,d];>>>
<<<isBorel monomialIdeal(a^2, a*b, b^2)>>>
<<<isBorel monomialIdeal(a^2, b^2)>>>
The function {\tt borel}\indexcmd{borel} generates the smallest Borel-fixed ideal
containing the given monomial ideal.
<<<borel monomialIdeal(b*c)>>>
<<<borel monomialIdeal(a,c^3)>>>

The next theorem provides the main source of Borel-fixed ideals.
\begin{theorem}[Galligo] \label{galligo}
Fix a term order on $S = \bbbq[x_{0}, \dotsc, x_{n}]$ such that $x_{0}
> \dotsc > x_{n}$.  If $I$ is a homogeneous ideal in $S$, then there
is a Zariski open subset $U \subseteq \GL_{n+1}(\bbbq)$ such that
\begin{enumerate}
\item[$(1)$] there is a monomial ideal $J \subseteq S$ such that $J =
\initial(\sg \cdot I )$ for all $\sg \in U$;
\item[$(2)$] the ideal $J$ is Borel-fixed.
\end{enumerate}
The ideal $J$ is called the generic initial ideal\index{generic
initial ideal}\index{ideal!generic initial} of $I$.
\end{theorem}

\begin{proof}
See Theorem~1.27 in Green~\cite{MR99m:13040}. \qed
\end{proof}

The following method allows one to compute generic initial ideals.
<<<gin = method();>>>
<<<gin Ideal := I -> (
     S := ring I;
     StoS := map(S, S, random(S^{0}, S^{numgens S:-1}));
     monomialIdeal StoS I);>>>
<<<gin MonomialIdeal := I -> gin ideal I;>>>
This routine assumes that the random function generates a matrix in
the Zariski open subset $U$.  Since we are working over a field of
characteristic zero this occurs with probability one.
For example, we can determine the generic initial ideal of two generic
homogeneous polynomials of degree $p$ and $q$ in $\bbbq[a,b,c,d]$.
<<<genericForms = (p,q) -> ideal(random(p,S), random(q,S));>>>
<<<gin genericForms(2,2)>>>
<<<gin genericForms(2,3)>>>
Although the generic initial ideal is Borel-fixed, some non-generic
initial ideals may also be Borel-Fixed.
<<<J = ideal(a^2, a*b+b^2, a*c)>>>
<<<ginJ = gin J>>>
<<<inJ = monomialIdeal J>>>
<<<isBorel inJ and isBorel ginJ>>>
Finally, we show that the generic initial ideal does depend on the
term order by computing lexicographic generic initial ideal for two
generic forms of degree $p$ and $q$ in $\bbbq[a,b,c,d]$
<<<S = QQ[a,b,c,d, MonomialOrder => Lex];>>>
<<<gin genericForms(2,2)>>>
<<<gin genericForms(2,3)>>>
A more comprehensive treatment of generic initial ideals can be found
in Green~\cite{MR99m:13040}.  The properties of Borel-fixed ideals in
characteristic $p>0$ are discussed in Eisenbud~\cite{MR97a:13001}.


%%----------------------------------------------------------
\subsection*{Example: Connectedness of the Hilbert Scheme}

Generic initial ideals are a powerful tool for studying the structure
of the Hilbert scheme\index{Hilbert scheme}.  Intuitively, the Hilbert
scheme $\Hilb^{\, p(t)}(\bbbp^{n})$ parameterizes subschemes $X
\subseteq \bbbp^{n}$ with Hilbert polynomial $p(t)$.  For an
introduction to Hilbert schemes see Harris and
Morrison~\cite{MR99g:14031}.  The construction of the Hilbert scheme $\Hilb^{\,
p(t)}(\bbbp^{n})$ can be found in Grothendieck's original
article~\cite{MR26:3566} or Altman and Kleiman~\cite{MR81f:14025a}.
While much is known about specific Hilbert schemes, the general
structure remain largely a mystery.  In particular, the component
structure --- the number of irreducible components, their dimensions,
how they intersect and what subschemes they parameterize --- is not
well understood.

Reeves~\cite{MR97g:14003} uses generic initial ideals to establish the
most important theorem to date on the component structure.  The
incidence graph\index{Hilbert scheme!incidence graph} of $\Hilb^{\,
p(t)}(\bbbp^{n})$ is defined as follows: to each irreducible component
we assign a vertex and we connect two vertices if the corresponding
components intersect.  Reeves~\cite{MR97g:14003} proves that the
distance (the number of edges in the shortest path) between any two
vertices in the incidence graph of $\Hilb^{\, p(t)}(\bbbp^{n})$ is at
most $2 \deg p(t)+2$.  Her proof can be divided into three major
steps.

{\em Step I: connect an arbitrary ideal to a Borel-fixed ideal.}
Passing to an initial ideal corresponds to taking the limit in a flat
family, in other words a path on the Hilbert scheme; see Theorem 15.17
in Eisenbud~\cite{MR97a:13001}.  Thus, Theorem~\ref{galligo} shows
that generic initial ideals\index{generic initial
ideal}\index{ideal!generic initial} connect arbitrary
ideals to Borel-fixed ideals.

{\em Step II: connect Borel-fixed ideals by projection.}  For a
homogeneous ideal $I \subseteq S = \bbbq[x_{0}, \dotsc, x_{n}]$, let
$\pi(I)$ denote the ideal obtained by setting $x_{n}=1$ and
$x_{n-1}=1$ in $I$.  With this notation, we have

\begin{theorem}
If $J$ is a Borel-fixed ideal, then the set of Borel-fixed ideals $I$,
with Hilbert polynomial $p(t)$ and $\pi(I) = J$, consists of ideals
defining subschemes of $\bbbp^{n}$ which all lie on a single component
of $\Hilb^{\, p(t)}(\bbbp^{n})$.
\end{theorem}

\begin{proof}
See Theorem~6 in Reeves~\cite{MR97g:14003}. \qed
\end{proof}

\noindent This gives an easy method for partitioning Borel-fixed
ideals into classes, each of which must lie in a single component.

{\em Step III: connect Borel-fixed ideals by distraction.}  Given a
Borel-fixed ideal, we produce a new ideal via a two-step process
called distraction\index{distraction}.  First, one polarizes the
Borel-fixed ideal.  The polarization\index{polarization} of a monomial
ideal $I \subset S$ is defined as:
\[
\left\langle \prod_{i=0}^{n}\prod_{j=1}^{\alpha_{i}} z_{i,j} : \text{
where $x_{0}^{\alpha_{0}}\dotsb x_{n}^{\alpha_{n}}$ is a minimal
generator of $I$} \right\rangle \, .
\]
One then pulls the result back to an ideal in the original variables
by taking a linear section of the polarization.  Theorem~4.10 in
Hartshorne~\cite{MR35:4232} shows that the distraction is connected to
the original Borel-fixed ideal.  Now, taking the lexicographic generic
initial ideal of the distraction yields a second Borel-fixed ideal.
Reeves~\cite{MR97g:14003} proves that repeating this process, at most
$\deg p(t) + 1$ times, one arrives at a distinguished component of
$\Hilb^{\, p(t)}(\bbbp^{n})$ called the lexicographic component.  For
more information on the lexicographic component see Reeves and
Stillman~\cite{MR98m:14003}.

We can implement these operations in \Mtwo as follows:
<<<projection = I -> (
     S := ring I;
     n := numgens S;
     X := gens S;
     monomialIdeal mingens substitute(ideal I, 
          {X#(n-2) => 1, X#(n-1) => 1}));>>>
<<<polarization = I -> (
     n := numgens ring I;
     u := apply(numgens I, i -> first exponents I_i);
     I.lcm = max \ transpose u;
     Z := flatten apply(n, i -> apply(I.lcm#i, j -> z_{i,j}));
     R := QQ(monoid[Z]);
     Z = gens R;
     p := apply(n, i -> sum((I.lcm)_{0..i-1}));
     monomialIdeal apply(u, e -> product apply(n, i -> 
               product(toList(0..e#i-1), j -> Z#(p#i+j)))));>>>
<<<distraction = I -> (
     S := ring I;
     n := numgens S;
     X := gens S;
     J := polarization I;
     W := flatten apply(n, i -> flatten apply(I.lcm#i, 
               j -> X#i));
     section := map(S, ring J, apply(W, r -> r - 
               random(500)*X#(n-2) - random(500)*X#(n-1)));     
     section ideal J);>>>
For example, we have
<<<S = QQ[x_0 .. x_4, MonomialOrder => GLex];>>>
<<<I = monomialIdeal(x_0^2, x_0*x_1^2*x_3, x_1^3*x_4)>>>
<<<projection I>>>
<<<polarization I>>>
<<<distraction I>>>

To illustrate Reeves' method, we show that the incidence graph of the
Hilbert scheme $\Hilb^{4t+1}(\bbbp^{4})$ has diameter at most $2$.
Note that the rational quartic curve\index{rational quartic curve} in
$\bbbp^{4}$ has Hilbert polynomial $4t+1$.
<<<m =  matrix table({0,1,2}, {0,1,2}, (i,j) -> (gens S)#(i+j))>>>
<<<rationalQuartic = minors(2, m);>>>
<<<H = hilbertPolynomial(S/rationalQuartic);>>>
<<<hilbertPolynomial(S/rationalQuartic, Projective => false)>>>
There are $12$ Borel-fixed ideals with Hilbert polynomial $4t+1$; see
Example~1 in Reeves~\cite{MR97g:14003}.
<<<L = {monomialIdeal(x_0^2, x_0*x_1, x_0*x_2, x_1^2, x_1*x_2, x_2^2), monomialIdeal(x_0^2, x_0*x_1, x_0*x_2, x_0*x_3, x_1^2, x_1*x_2, x_2^3), monomialIdeal(x_0, x_1^2, x_1*x_2^2, x_1*x_2*x_3, x_2^3), monomialIdeal(x_0, x_1^2, x_1*x_2, x_2^4, x_2^3*x_3), monomialIdeal(x_0, x_1, x_2^5, x_2^4*x_3^3), monomialIdeal(x_0, x_1^2, x_1*x_2, x_1*x_3, x_2^5, x_2^4*x_3^2), monomialIdeal(x_0^2, x_0*x_1, x_0*x_2, x_0*x_3, x_1^2, x_1*x_2, x_1*x_3, x_2^5, x_2^4*x_3), monomialIdeal(x_0, x_1^2, x_1*x_2, x_1*x_3^2, x_2^5, x_2^4*x_3), monomialIdeal(x_0^2, x_0*x_1, x_0*x_2, x_0*x_3, x_1^2, x_1*x_2, x_1*x_3^2, x_2^4), monomialIdeal(x_0, x_1^2, x_1*x_2^2, x_1*x_2*x_3, x_1*x_3^2, x_2^4), monomialIdeal(x_0, x_1^2, x_1*x_2, x_1*x_3^3, x_2^4), monomialIdeal(x_0, x_1, x_2^6, x_2^5*x_3, x_2^4*x_3^2)};>>>
<<<scan(#L, i -> << endl << i+1 << " : " << L#i << endl);>>>
<<<all(L, I -> isBorel I and hilbertPolynomial(S/I) == H)>>>
The projection operation partitions the list {\tt L} into $3$ classes:
<<<class1 = projection L#0>>>
<<<class2 = projection L#1>>>
<<<class3 = projection L#4>>>
<<<all(1..3, i -> projection L#i == class2)>>>
<<<all(4..11, i -> projection L#i == class3)>>>
Finally, we use the distraction to connect the classes.
<<<all(L, I -> I == monomialIdeal distraction I)>>>
<<<all(0..3, i -> projection gin distraction L#i == class3)>>>
Therefore, the components corresponding to {\tt class1} and {\tt
class2} intersect the one corresponding to {\tt class3}.  Note that
{\tt class3} corresponds to the lexicographic component.

%%----------------------------------------------------------
\section{The Chain Property}


Ho\c{s}ten and Thomas~\cite{MR2000b:13037} recently established that
the initial ideals of a toric ideal have an interesting combinatorial
structure called the chain property.  This structure is on the poset
of associated primes where the partial order is given by inclusion.
Since a monomial ideal $I \subset S = \bbbq[x_{1}, \dotsc, x_{n}]$ is
prime if and only if it is generated by a subset of the variables $\{
x_{1}, \dotsc, x_{n}\}$, the poset of associated primes of $I$ is
contained in the power set of the variables.  We say that a monomial
ideal $I$ has the chain property\index{chain property} if the
following condition holds:
\[
\text{\begin{minipage}[t]{300pt}
For any embedded prime $P_{F} = \langle x_{i} : x_{i} \not\in F
\rangle$ of $I$, there exists an associated prime $P_{G} \subset
P_{F}$ such that $|G| = |F|-1$.
\end{minipage}}
\]
In other words, there is a saturated chain from every embedded prime
to some minimal prime.  
Experimental evidence suggests that, in fact, most initial ideals 
of prime ideals satisfy this saturated chain condition.
Because
of ubiquity and simplicity of this condition, we are interested in
understanding which classes of initial ideals (or more generally
monomial ideals) have the chain property.  

More recently, Miller, Sturmfels and Yanagawa~\cite{MR1769661}
provided a large class of monomial ideals with the chain property.  A
monomial ideal $I$ is called generic\index{generic monomial ideal}
when the following condition holds: if two distinct minimal generators
$r_{1}$ and $r_{2}$ of $I$ have the same positive degree in some
variable $x_{i}$, there is a third generator $r_{3}$ which strictly
divides the least common multiple of $r_{1}$ and $r_{2}$.  In
particular, if no two distinct minimal generators have the same
positive degree in any variable, then the monomial ideal is generic.
Theorem~2.2 in Miller, Sturmfels and Yanagawa~\cite{MR1769661} shows
that generic monomial ideals have the chain property.


%%----------------------------------------------------------
\subsection*{Examples and Counterexamples}


In this final section, we illustrate how to use \Mtwo for further
experimentation and investigation of the chain property.  The
following function determines whether a monomial ideal has the chain
property:
<<<hasChainProperty = I -> (
     L := ass I;
     radI := radical I;
     all(L, P -> radI : (radI : P) == P or (
               gensP := first entries gens P;
               all(gensP, r -> (
                         Q := monomialIdeal delete(r, gensP);
                         I : (I : Q) == Q)))));>>>
Using {\tt hasChainProperty}, we examine the initial ideals of four
interesting classes of ideals related to toric ideals.

\subsubsection*{An Initial Ideal of a Toric Ideal}  As mentioned above, Ho\c{s}ten
and Thomas proved that any initial ideal of a toric ideal satisfies
the saturated chain condition.  The following example demonstrates this
phenomenon.  Consider the matrix $A$:
<<<A = matrix{{1,1,1,1,1,1,1}, {2,0,0,0,1,0,0}, {0,2,0,0,0,1,0}, {2,2,0,2,1,1,1}}>>>
<<<IA = toricIdeal(A, {1,1,1,1,1,1,1})>>>
<<<inIA = monomialIdeal IA>>>
<<<hasChainProperty inIA>>>

\subsubsection*{An Initial Ideal of a Prime Ideal}  Since toric ideals are
prime, one naturally asks if the initial ideal of any prime ideal has
the chain property.  By modifying the previous example, we can show
that this is not the case.  In particular, making the linear change of
coordinates by $x_{4} \mapsto x_{3}-x_{4}$, we obtain a new prime
ideal $J$.
<<<StoS = map(S, S, {x_1, x_2, x_3, x_3 - x_4, x_5, x_6, x_7});>>>
<<<J = StoS IA>>>
Taking the initial ideal with respect to the reverse lexicographic
term order (the default order), we have
<<<inJ = monomialIdeal J>>>
<<<hasChainProperty inJ>>>

\subsubsection*{An $A$-graded Monomial Ideal}  Let $A$ be a $d \times n$ matrix
of nonnegative integers and let $\mathbf{a}_{i}$ denote the $i$-th
column of $A$.  Consider the polynomial ring $S = \bbbq[x_{1}, \dotsc,
x_{n}]$ with the $\bbbz^{d}$-grading defined by $\deg x_{i} =
\mathbf{a}_{i}$.  An ideal $I \subset \bbbq[x_{1}, \dotsc, x_{n}]$ is
called $A$-graded\index{ideal!$A$-graded} provided it is homogeneous with
respect to the $A$-grading and
\[
\dim_{\bbbq} \left( \frac{S}{I} \right)_{\mathbf{b}} = \begin{cases} 1
& \text{if $\mathbf{b} \in \bbbn A$} \\ 0 & \text{otherwise}
\end{cases}
\]
for all $\mathbf{b} \in \bbbn^{d}$.  Remark~10.1 in
Sturmfels~\cite{MR97b:13034} shows that the initial ideal of the toric
ideal $I_{A}$ is $A$-graded.  Altmann~\cite{Altmann} shows that when
$A$ has rank $2$ every $A$-graded monomial ideal has the chain
property.  However, Altmann~\cite{Altmann} also provides a
counterexample when $A$ has rank $3$.  We can verify his example in
\Mtwo as follows:
<<<A = matrix{{2,0,0,1,0,0,2,1,1,3,2,2,2,3,3,3},
           {0,2,0,0,1,0,1,2,1,2,3,2,3,2,3,3},
           {0,0,2,0,0,1,1,1,2,2,2,3,3,3,2,3}};>>>
In \Mtwo, the first entry in degree vector of each variable must be
positive.  Hence, we append to A the sum of its rows to get a matrix whose
columns will serve as the degrees of the variables.
<<<D = A^{0}+A^{1}+A^{2} || A>>>
<<<D = entries transpose D;>>>
<<<S = QQ[vars(0..15), Degrees => D, MonomialSize => 16];>>>
<<<I = monomialIdeal(d*j, d*k, d*l, d*m, d*n, d*o, d*p, e*j, e*k,
    e*l, e*m, e*n, e*o, e*p, f*j, f*k, f*l, f*m, f*n, f*o, f*p,
    g*j, g*k, g*l, g*m, g*n, g*o, g*p, h*j, h*k, h*l, h*m, h*n,
    h*o, h*p, i*j, i*k, i*l, i*m, i*n, i*o, i*p, g^2, g*h, g*i,
    h^2, h*i, i^2, j^2, j*k, j*l, j*m, j*n, j*o, j*p, k^2, k*l,
    k*m, k*n, k*o, k*p, l^2, l*m, l*n, l*o, l*p, m^2, m*n, m*o,
    m*p, n^2, n*o, n*p, o^2, o*p, p^2, d^2, e^2, f^2, d*h, e*i,
    f*g, f*d*i, d*e*g, e*f*h, c*d*g, a*e*h, b*f*i, c*e*g, 
    a*f*h, b*d*i, c*d*e, a*e*f, b*f*d, c*b*d, a*c*e, b*a*f, 
    c*b*g, a*c*h, b*a*i);>>>
To help convince you that $I$ is an $A$-graded ideal, we compute the
$\dim_{\bbbq} \left( \frac{S}{I} \right)_{\mathbf{a}_{i}}$ for $1 \leq
i \leq 16$.
<<<apply(D, d -> rank source basis(d, (S^1)/ ideal I))>>>
Finally, we check the chain property.
<<<hasChainProperty I>>>

\subsubsection*{The Vertex Ideal}  Lastly, we consider a different family of
monomials ideals arising from toric ideals.  The vertex
ideal\index{vertex ideal} $V_{A}$ is defined as intersection all the
monomial initial ideals of the toric ideal $I_{A}$.  Although there
are (in general) infinitely many distinct term orders on a polynomial
ring, an ideal has only finitely many initial ideals; see Theorem~1.2
in Sturmfels~\cite{MR97b:13034}.  In particular, the above
intersection is finite.  Vertex ideals were introduced and studied by
Ho{\c{s}}ten and Maclagan~\cite{Hosten-Maclagan}.  However, the
question ``Does the vertex ideal $V_A$ have the chain property?''
remains open.



% Local Variables:
% mode: latex
% mode: reftex
% tex-main-file: "chapter-wrapper.tex"
% reftex-keep-temporary-buffers: t
% reftex-use-external-file-finders: t
% reftex-external-file-finders: (("tex" . "make FILE=%f find-tex") ("bib" . "make FILE=%f find-bib"))
% End:
