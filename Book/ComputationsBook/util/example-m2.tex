\documentclass{minimal}
\def\M2{{\em Macaulay 2\/}}
\input merge.tex
\begin{document}
\noindent
Here is an example of how to incorporate \M2{} code into a TeX file.
\beginOutput
i1 : R = QQ[x,\\
\              y,z]\\
\emptyLine
o1 = R\\
\emptyLine
o1 : PolynomialRing\\
\endOutput
\beginOutput
i2 : C = res coker vars R\\
\emptyLine
\      1      3      3      1\\
o2 = R  <-- R  <-- R  <-- R  <-- 0\\
\                                  \\
\     0      1      2      3      4\\
\emptyLine
o2 : ChainComplex\\
\endOutput
Let's look at the Betti numbers.
\beginOutput
i3 : betti C\\
\emptyLine
o3 = total: 1 3 3 1\\
\         0: 1 3 3 1\\
\endOutput
It will even truncate very long answers to a width specified in the Makefile
as an argument to merge.
It will even truncate very long answers.
It will even truncate very long answers.
It will even truncate very long answers.
It will even truncate very long answers.
It will even truncate very long answers.
\beginOutput
i4 : 0 .. 200\\
\emptyLine
o4 = (0, 1, 2, 3, 4, 5, 6, 7, 8, 9, 10, 11, 12, 13, 14, 15, 16, 17, 18, 19, 20, 21, 22, 23, 24, $\cdot\cdot\cdot$\\
\emptyLine
o4 : Sequence\\
\endOutput
How's that look?
\beginOutput
i5 : "asdf{\char`\\}n{\char`\\}nasdf\{\}{\char`\%}\$#@_{\char`\%}{\char`\\}{\char`\\}"\\
\emptyLine
o5 = asdf\\
\emptyLine
\     asdf\{\}{\char`\%}\$#@_{\char`\%}{\char`\\}\\
\endOutput

\end{document}
