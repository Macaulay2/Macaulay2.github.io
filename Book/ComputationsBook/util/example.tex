\documentclass{minimal}
\def\M2{{\em Macaulay 2\/}}
\input merge.tex
\begin{document}
\noindent
Here is an example of how to incorporate \M2{} code into a TeX file.
<<<R = QQ[x,
         y,z]>>>
<<<C = res coker vars R>>>
Let's look at the Betti numbers.
<<<betti C>>>
It will even truncate very long answers to a width specified in the Makefile
as an argument to merge.
It will even truncate very long answers.
It will even truncate very long answers.
It will even truncate very long answers.
It will even truncate very long answers.
It will even truncate very long answers.
<<<0 .. 200>>>
How's that look?
<<<"asdf\n\nasdf{}%$#@_%\\">>>

\end{document}
